
\section{세상보는 가치관에 대한 지적}
\href{https://www.kockoc.com/Apoc/243417}{2015.08.11}

\vspace{5mm}

대통령이란 사람 입에서 어떻게 젊은이들 다 중동으로 가버리라는 말이 나올수 있는지 대략 상상이 가시죠? ㅎㅎ
그동안 칼럼 읽으면서 그래도 깨어있는 분이신 것 같다는 생각이 들어서 한정된 재화의 분배에 대한 최.소.한의 합의는 도출받을 수 있을 것 같았는데 (아재인 것을 알았기에 정말 딱 거기까지만) 안타깝게도 전혀 아니였네요.
제 댓글만 보면 아폭아재가 무슨 욕심덩어리로 보일 수 있겠지만 누구나 나이가 들면서 빠지는 딜레마입니다. 이 딜레마에서 결국 탐욕이 승리하면 손쉽게 금덩어리를 획득할 수 있었던 시절 금덩이를 모두 독치지해놓고 금덩어리가 모두 바닥난 지금 마치 노력이 부족해 금덩어리를 못얻은 것처럼 (노력한다고 없는 금덩어리가 생기나?) 젊은이들이 게으르고 배부르다고 말하게 됩니다. 아니 노력해서 자기가 금덩어리 가져간게 무슨 문제냐? 아폭아재 말대로 정말 굶주렸던 시절 나라를 위해 힘쓴분들이 금덩어리를 많이 가져갔으면 말을 안하겠는데 말이죠.
개인적으로 이런 아재들의 평균적인 사고는 1인 1금덩어리 하자는 이석기의 수준과 별 차이가 없다고 봅니다... (실제로 아폭아재 생각을 양의 무한대로 보내버리면 북한의 모습이 나오네요)
콕콕러 여러분들이 아폭아재의 댓글들을 통해 재벌들이 왜 탈세를 밥먹듯 하는지, 우리나라는 왜 온갖 불법과 공정하지 못한 방법으로 부를 축적한 사람들이 떵떵거리며 잘 사는지, 어떻게 독재라는 틀린 방법이었음에도 결과가 좋았으니 (과연?) 박정희를 찬양하는 사람이 존재하는지 깨달으실 수 있으시면 좋겠습니다.
\vspace{5mm}

모른 척 하고 지나가도 되는 이야기입니다. 왜냐면 저런 생각을 갖고 있는 사람은 그냥 '호구'가 되기 때문입니다.
\begin{itemize}
    \item  첫째, 기본적인 경제 관념도 없다. 부를 기득권이 독점하고 안 나눠주는 것으로만 착각한다.
    \item 둘째, 기본적인 논리도 없으면서 계속 인신공격을 한다.
    \item 셋째, 느닷없이 박정희 이야기나 이석기 이야기를 하면서 자기 꼴리는대로 인신공격을 한다.
\end{itemize}
\vspace{5mm}

보통은 무시하면서 가지만 어차피 이런 얘기도 해주고 싶었는데 $-$ 정치적인 설득은 피하면서 $-$ 그래도 설명해주는 이유는 세가지입니다.
\vspace{5mm}
\begin{itemize}
    \item 첫째, 제가 저랬기 때문임(그 때는 정말 세상을 선악이분법으로만 보았음)
    \item 둘째, 역시 단 한번도 만나본 적이 없던 익명의 댓글러를 통해서 극적으로 저런 오류에서 벗어났기 때문임.
    \item 셋째, 입시보다도 사실 저 이야기가 중요한 점들이 있기 때문.
\end{itemize}
\vspace{5mm}

우선 저 댓글러는 "대한민국이 시장경제에 따라 굴러가고 있다는" 것 자체를 무시하고 있죠.
한정된 재화의 분배에 그리고 집착하고 있는데, 어째서 조선이 가난하게 살았는지 그런 역사적인 공부도 안 했다는 이야기입니다.
조선시대 선비들 똑똑했죠. 그런데 실학자들의 대안이라는 게 다 이런 거였죠. "한정된 토지의 분배"
즉, 재화를 늘릴 생각 자체를 못 했고 어떻게 하면 땅을 분배해서 "다 같이 적게 먹는 이상적인 사회를 구현"하는지에만 골몰해있었습니다.
그 때 서유럽은 이미 산업혁명 단계까지 들어서면서 $-$ 물론 식민지 약탈, 착취와 함께 $-$ 파이를 키울 준비가 착착 되어있었는데 말입니다.
\vspace{5mm}

현재 젊은 세대들이 기성 세대에 대해서 품은 증오감은 이해가 안 가는 건 아닙니다만
1950년대 이후 대한민국 역사라는 것을 살펴보면 그런 식의 선악이분법적으로 접근할 수 있는 게 아닙니다.
박정희 독재가 잘못된 것은 맞죠. 그런데 그 독재를 비난하는 사람들은 "경제성장이 무엇으로 가능했느냐"에 대해선 말을 못 합니다.
아니, 정확히 말해서 경제성장이 결국 "피와 땀"을 '착취'하는 과정에서 이뤄진다라는 진실에 고개를 돌리지요.
세계사적으로 보았을 때 정당한 경제성장이 이뤄진 경우는 단 한번도 없습니다. 민주주의니 인권타령이니 하는 서유럽 역사는 참 지저분하죠.
이게 웃긴 게 뭐냐면, 개인적으로 한번 만나보기도 했던 $-$ 박정희 비판자였던 정운현씨의 저서만 봐도 나와있습니다.
그 책 결론이 그거죠. 그냥 무덤 속 박정희 건드리지 말자.
애당초부터 까려고 집필을 시도했는데 정작 팩트를 발굴해보니까 깔 게 생각 외로 적고 왜 노인 세대들이 독재체제라고 해도
박정희 시대에 대해서 긍정적인 평가를 할 수 밖에 없는가 하는 점을 저자 본인도 발견해버렸다는 걸 읽다보면 알 수 있거든요.
\vspace{5mm}

이런 복잡한 이야기를 글 한편으로 정리하는 건 힘든 일이지만 아무튼 메시지는 간단합니다.
모든 것을 '동화적인 이분법'으로만 접근하지 말라는 것이죠. 거기에 사로잡히면 20세기 공산주의 시행착오의 틀에서 못 벗어납니다.
즉 폭력혁명으로 헤게모니를 잡고 가진 자들의 재산을 털어 나눠먹기만 하면 된다라는 유치한 관념이 된다는 것이죠.
\vspace{5mm}

현재 젊은 세대들이 힘들다, 그거야 간단하죠. \textbf{"저성장"}이니까요.
더 이상 성장하지 못 하니까 일자리가 없고 $-$ 거기다가 하필 신기술들이 기존 산업과 일자리를 뺏어먹는 것이기 때문에 $-$
일자리가 없으니까 취업할 수 없고 이런 식으로 도미노 현상이 벌어지니까 1960년대 이전 상태로 돌아가는 게 보인다는 것입니다.
저출산, 중국과 동남아의 추격, 인간의 노동이 필요없는 산업의 발달, 인간을 대체하는 기계들.
이렇게 이미 다양한 이유가 나와있는데도 여전히 "기성세대들이 욕심이 많기 때문이다"라고 정리할 수 있다니 놀라울 따름입니다.
\vspace{5mm}

남탓하는 이런 얘기는 조금만 반론하면 깨지죠.
흔한 예로 기업유보금이 있습니다(기업유보금이 뭔지 여기 아는 사람이 있으려나) 기업유보금 풀어서 복지 쓰면 된다라는 유치한 주장이 많죠.
그런게 그 유보금은 요즘 같은 시대에 그 기업이 안 망하기 위해서 갖고있는 생명줄입니다.
흔한 대기업 죽이기에 동조해서 그런 것도 다 복지에 써서 기업이 망하면 그럼 대한민국은 뭘로 먹고 살죠?
또 흔히 이야기하는 사례가 부동산 소유자들이 문제니까 세금 팍팍 거두고 임대료 낮추고 그러자.
이 역시 경제학을 모르고 하는 이야기이죠. 우리나라는 이미 종부세를 나름대로 거두고 있는 나라입니다(위헌적 요소가 많음에도)
세금을 올린다? 그 세금은 당연히 임대료 상승을 견인해서 '임차인'에게 전가되겠지요.
권리금을 인정해준다? 그럼 임차인들은 봉이 김선달식으로 가공의 권리금을 만들어내겠죠. 아니, 그거 무서워서 세를 내줄까요?
\vspace{5mm}

경제라는 건 꽤 복잡다단한 시스템이고, 특히나 선한 의도의 시도가 좋은 결과만을 낳지 않습니다.
흔히 박정희를 비판한다라고 하면서 스테레오타입으로 친일, 독재, 불공정, 재벌 독과점을 좔좔 나열하는 사람들.
이들은 사실 생각없이 다른 누군가 떠든 것을 그냥 받아서 얘기하는 수준인데, 문제는 이런 이야기는 사실 한물 간 사상이란 겁니다.
결과만 좋으면 과정이 나빠도 좋다라고 말하는 건 아니죠. 문제는 '좋은 결과'라는 걸 낳는 것 자체가 매우 어렵단 것입니다.
더 웃긴 건 결과만 좋으면 되냐고 하는 사람들이 정작 칼자루를 잡으면 "의도는 좋았다"란 식으로 자기들이 저지른 나쁜 결과는 외면하죠.
더군다나 경제는 우리의 삶과 직결된 문제인데도 말입니다.
\vspace{5mm}

사실 저 글을 쓰는 사람은 과연 '헝그리하게' 살았을까하는 것은 역시 물어볼 필요는 없을 건데
이 역시 과거의 제 모습을 보는 것이라고 생각하기 때문에, 살다보면 철이 들겠지라고 봅니다.
다만 당사자가 철이 들기 위해서 얼마나 많은 비용을 지불해야할지는 대략 짐작이 가지만 말입니다.
\vspace{5mm}

그런데 이제 저런 좌파니 우파니 사회적 정의니. 이거 신경을 쓰나요.
사실 철든 사람이라면 안 쓰죠. 왜냐고요? 그것들도 결국 누군가 팔아먹는 상품이지, 그런 것이 뭘 보장해준다라는 이야기는 없으니까요.
정의를 앞장세운 경우가 알고보니 더 많은 자본을 챙기려는 시도였더라하는 경우가 100$\%$입니다.
분배를 강조하자 정의를 실천하자 하는 경우는 그 메신저가 정계 진출을 하는 건가, 아니면 엽관주의로 공직을 받으려고 하는 건가,
하다 못해 대형출판을 통해서 인세를 챙기려하는 경우인가 삐딱하게 보면 이 중 하나는 반드시 걸리더군요.
친일 논쟁은 더욱 가관이죠. 모 정당에서 친일파 공격하는 주체가 알고보니 정작 자기 조상이 친일파여서 역전된 케이스도 있습니다만,
무려 40년 정도나 계속된 조선$-$일본의 사례를 프랑스$-$나찌독일의 사례에 비교하는 것도 모순이거니와
정작 독립운동가들도 자기들의 헤게모니를 위해 친일파와 손잡지 않나, 서로 자기들끼리 암투벌이는 경우도 있다는 걸 알면
흑백논리로 접근한다는 게 얼마나 무서운건가 깨닫게 됩니다(교과서로 배우는 근현대사는 정말이지 판타지이죠)
\vspace{5mm}

무엇보다 저도 어린 시절에는 군사정권의 문제를 깨닫고 박정희 개객기 어쩌구 하는 입장이었는데
나이먹으면서 남의 부정 탓하는 자들이 정작 똑같은 부정을 저지르는 것을 보고 멘붕해서리 지금은 그냥 회색주의에 가까운 입장입니다.
재벌도 다 나름대로 장점이 없는 것도 아니고, 만약 한국이 1960년대에 민주주의를 할 수 있었을까 하냐면 고개는 설레설레.
역사는 윤리와는 다르고, 특히 경제는 그것만의 로직이 따로 있다는 것.
이런 복잡다단한 것을 모르고 모든 것을 흑백으로 논한다...
자기는 정작 고기 잘만 먹으면서 육식을 비난하는 사람들이 많죠.
\vspace{5mm}

저는 이 글을 읽는 사람들이 어떤 정치성향이건 역사적 입장이건 그건 각자의 판단에 맡길 문제라고 봅니다.
제가 싫어하는 건, "입으로만 놀리는 자기 정의"를 남에게 강요하는 케이스예요. 제가 그런 데 호되게 당해보았기 때문에 더욱 강조하는 겁니다.
현재는 그래도 과거보단 나아졌을 것입니다만 대학에 들어가신 분들은 '선배'라는 종자들에게 일종의 사상을 강요받을 것입니다.
온건한 형태건 강경한 형태건, 알콜이 들어가든 안 들어가든, 혹은 남선배이건 여선배이건 말입니다.
제가 보는 최악의 폭력이 사실 그런 것이죠.
스스로 돈도 벌지 않고 그저 문자로만 세상을 접한 사람들이 뭔 뽕이라도 맞았나 자아도취에 빠져서
후배들에게 너는 반드시 XX주의에 빠져야 한다, XX는 강아지들이니 무찌르자... 뭐 이런 망발을 일삼는 경우 지금도 없기만 바라겠습니다만.
\vspace{5mm}

기억나네요. 저랬던 선배가 정작 자기는 군대갔다오더니 자기가 그토록 욕하던 미국으로 냉큰 날라버린 것(...)
\vspace{5mm}

더 유식한(?) 말로 하면 제발이지 일상의 \textbf{파시즘.} 이런 거나 좀 자제합시다.
뭔 사회정의 외치고 독재 비판한다는 사람들이 정작 자기들이 파시스트라는 건 아시는지.
\vspace{5mm}

아울러 현 대통령의 중동발언.
대통령은 마음에 안 드는 게 많습니다. 전 철저히 결과론자입니다. 의도가 어떻건 정치는 삶이고 결과물이 가장 중요해요.
그런데 저 발언 자체는 헛소리가 아니죠. 이제 동아시아에서는 더 이상 해먹을 수 있는 시장이 거의 없으니까요.
동남아까지 산업화 단계 성숙되고 나면 그 다음은 이제 이란, 이라크 등입니다. 여기가 이제 중국처럼 바뀌겠고 이란은 이미 확정적이지요.
이런 것도 모르면서 '중동 가라'는 발언을 무작정 까대는 건, 흔한 경제신문도 안 읽었다는 반증입니다.
(물론 도대체 정부가 어떻게 해서 중동을 통해 경제발전을 견인할 건지는 저도 의문입니다만)
\vspace{5mm}






\section{심심해서 쓰는 질문들}
\href{https://www.kockoc.com/Apoc/243692}{2015.08.11}

\vspace{5mm}

수험사이트니까 수험 형식으로 냅니다.
세상사에 통달해서 이것저것 비판할 줄 아는 사람이면 당연히 답할 수 있다고 보겠어요
\vspace{5mm}

001. 지난 3년간 연평균 매출 및 영업이익 증가율이 20$\sim$30$\%$에 달하는 화장품 관련 기업 2곳을 적으시오
\vspace{5mm}

002. 3D 프린팅 관련업체인 (     )는 중국 기업들과 총 200만 달러 규모의 카메라 모듈관련 공급계약을 체결했다.   중국 스마트폰 제조사들이 글로벌 점유율을 높이고자 품질기준을 강화하면서 카메라모듈 업체들 사이에서 자동화검사장치를 도입하려는 수요가 늘고 있다.
\vspace{5mm}

003 정보는 독거노인이나 희귀난치성환자가 응급상황에 처했을 때 신속하고 적절한 도움을 받을 수 있게 라이프태그를 도입하기로 했다.   라이프태그는 착용자의 특이 건강상태를 담고있는 팔찌모양의 기기다. 스마트폰을 갖다대면 병명, 보호자 연락처가 나온다. 이 라이프태그로 기대를 모으는 곳은 (    )다.
\vspace{5mm}

004 스마트 전구 관련주를 아는 대로 적으시오.
\vspace{5mm}

005 한글과컴퓨터, MDS테크놀로지, 다윈텍 등의 지분가치 1000억 수준을 보유하고 있는 (       ) 의 경우
금융 전자상거래 공공분야 전 부문에 걸쳐 암호화 인프라를 보유한 것이 강점이다.
\vspace{5mm}

006 핀테크는 (        )과 (        )이 결합된 서비스 그런 서비를 하는 회사이다.
한국은 아직 핀테크 서비스가 나타나기 힘든 조건을 갖추고 있다.
(  어)가 통용되지 않으며 인구가 5000만으로 고립되었기 때문이다.
국내금융, 보안 산업 발전을 가로막던 요인으로 꼽힌 (    브   )가 올해에야 비로소 사라진다.
\vspace{5mm}

007 무인자동차가 도입되면 택시회사는 살아남겠는가?
\vspace{5mm}

위와 같은 수험질문 형태로 만들면 재밌는 소스들이 많이 있습니다.
심심하시면 한번 채워보시기들.
대졸자라도 관심이 없으면 사실 하나라도 답하기 어렵습니다.
\vspace{5mm}

저런 문항을 만들어서 답하는 것이야말로 앞으로 필요한 공부가 아닌가 싶은데.
사실 꽤 유치한 문항들입니다만, 저것도 답 못 하는 사람들도 많을 것입니다.
저 문항들은 실제 돈벌이, 투자와 관계된 문항들입니다.
\vspace{5mm}






\section{예쁜 자식은 매 하나.}
\href{https://www.kockoc.com/Apoc/254570}{2015.08.15}

\vspace{5mm}

미운 자식이나 피섞이지 않은 자식은 오토바이를 사준다는 현대 속담이 있지요(ㄷㄷㄷ)
\vspace{5mm}

오해살 것 같아서 그러는데
제 경우 '가치있다고 생각하는 교재'만 언급합니다.
\textbf{따라서 교재를 비판하는 경우 전제는, 그 교재는 볼 가치가 있다, 다만 xx한 점을 보완하건 조심해야 한다 이런 의미입니다.}
\vspace{5mm}

아마도 앞으로도 꽤 많은 질문이 잇따를 것이고 그 중 90$\%$는 교재질문이 아닐까 하는데
당연히 대답할 수 있는 건 많습니다. 왜냐면 서점에 가서 참고서는 하나씩 다 일별하고 상품가치를 평가해보기 때문이고
출판사, 저자의 동향 같은 것도 찾아보거나 추정해보기 때문입니다.
예컨대 EBS 교재만 보아도 거기 실린 저자들의 이력이나 흐름을 보면 앞으로 동향을 더 확인할 수 있죠.
\vspace{5mm}

현재 방침은 간단합니다. 쓰레기이거나 문제가 많은 교재는 "언급"을 안 하거나 그게 언급된 경우 "취급 안 한다고" 가면 됩니다.
웃긴 세상이라서 교재 저자들이 하라는 공부는 안 하고 돈독이나 올라서 자기가 어떻게 인터넷에서 광고되고 평가되나 그런 것이나 신경쓰며
심지어 고소한 맛을 전파해서 입을 막으려 하는 경우가 있습니다.
우리나라 법이 참 맹점이 많아서 그러는데 아무튼 그런 건 절대 여지를 주지 않기 위해서입니다.
설마 언급 안 한다고 이것 가지고 뭐라고 하는 케이스는 없겠죠.
\vspace{5mm}

일격필살은 지금 2$\sim$3회 둘러보고 특히 B형 1회 30번 같은 것도 다시 봅니다만
머리칼 걸고 권하겠느냐 한다면 권하겠습니다.
단점이야 교재 레이드에서 신랄하게 까서 그런데 장점도 사실 여러가지 많죠.
\vspace{5mm}

첫째, \textbf{영계} 필진의 충원과 다원화 $-$ 저자가 한명인 경우의 위험을 피했습니다.
사실 A형이 좋다는 평가가 있는데 3회까지 보면 B형이 낫습니다. 사이트 관리자 분은 범을 키우셨네요(저러다 독립하려면 어쩌려고)
개인 이름만 내걸고 가는 경우는 풀어보면 느끼지만 편차가 심한 편이나 공동필진으로 가는 경우는 이 점이 덜합니다.
\vspace{5mm}

둘째, 가성비 : 12회에 21600원이면 한 문제당 60원 꼴이니 가격이 괜찮은 것이죠.
더 착한 걸로 EBS가 있긴 하지만 EBS를 제외하고 가성비 좋은 게 이것 말고는 있나. 혹자 알고 게시면 댓글 달아주시면 되겠죠.
사실 일격 말고 다른 것도 검토해두었는데 $-$ 본고사 찌라시라거나 $-$ 지금 보니까 일격 정도면 무난합니다.
\vspace{5mm}

그리고 셋째는 풀면서 발견한 미덕인데
해설은 좀 더 토의하고 파워업할 게 있어도 문제 자체는 꽤 많은 논점을 갖고 있어서
복습을 잘 하면 이 12회 분을 적당히 요약해서 마무리용으로 쓸 수 있다 느껴지는 것들이 있습니다.
이 역시 더 검토를 해봐야하지 않나 싶긴 합니다만.
\vspace{5mm}

무엇보다 그 저자진들이 그래도 열린 사람들이라 고객의 소리를 들을 준비는 되어있는 거죠(그게 제가 콕콕에 온 이유이기도 하고)
단순히 팔아서 돈만 벌려고 하는 사람과 그렇지 않은 차이는 매우 큽니다.
주갤 같은 데에서 흔히 오가는 이야기가 주가 알고 싶으면 그냥 그 사장의 \textbf{관상(...)}을 보면 된다고 하던데
농담 같지만 사실 가장 중요합니다. 결국 뭐로 가든 그 사람이 성실하고 신뢰성이 있으며 돈독에만 빠지지 않았냐가 중요합니다.
(그럼 허사장 관상은 어떠냐.... 비밀입니다만 정말 나쁜 사람이라면 제가 여기 와서 뻘글을 쓰고 있을리는 없겠죠)
\vspace{5mm}






\section{어머니와 아들의 싸움}
\href{https://www.kockoc.com/Apoc/464553}{2015.11.05}

\vspace{5mm}

충효가 강조되는 이유는 뭘까, 현실은 충효와 거리가 멀기 때문이다.
가정화합을 강조하는 이유도 그렇다, 실제로 화목한 가정이 그리 많지 않기 때문이다.
특히 수험생 자녀가 있는 집안은 성적이 '기압'의 역할을 한다.
\vspace{5mm}

보통 아버지들이 나서서 책임지는 경우는 그리 많진 않다. 보통 어머니들이 나선다.
이 경우 가장 많이 빈발하는 게 어머니와 아들의 싸움이다. 가장 애정이 많다보니 성적이 안 좋으면 싸우기까지 한다.
어머니가 아들에 대해서 약자인 경우가 많다보니 관계가 대등해지는 것이고
그래서 아들이 갑, 어머니가 을이 된다.
\vspace{5mm}

그리고 비극은 여기서 발생하는데
성적이 좋지 않은 아들이라고 할지라도 세상돌아가는 건 알기 때문에 어떻게 하면 좋게 갈까 전략적으로 사고하려 한다.
그런데 어머니는 아들이 좋은 데 가길 바라면서도 결과가 안 좋거나 아들이 공부 안 하는 걸로 보이면 매우 답답해 할 수 밖에 없다.
그래서 아들을 위한다고 너 xx대라도 가는 게 어떠니, 너 왜 공부 안 하니... 라고 잔소리를 한다.
당연히 아들 입장에서는 저건 "너 왜 못 하니"로 보이는 공격으로 비친다. 그 뒤에 어떤 과정이 전개될지는 뻔하다.
이 글을 읽는 상당수가 겪어보았을 그런 문제다.
\vspace{5mm}

이 경우 해결책은 그렇다. 70$\%$는 대체로 어머니에게 있다. 어머니가 아들에게 거리를 둬야하고 개입을 적절히 제한해야한다.
어머니들이 나름 자식을 위한다라고 하지만, 실제로는 자기가 느끼는 불안함이나 스트레스를 아들에게 푸는 경우로 가기 십상이다.
나는 열심히 밥 해주고 학교 보내주고 했는데 너는 왜 못 하니.... 라는 것이야 여성으로서 당연하 사고 프로세스일 수 있다.
이건 여자들끼리라면 감성토크가 가능하기 때문에 문제가 없다. 그러나 남자들은 이런 걸 공감으로 받아들이지 않는다.
남자들이 원하는 건 확실한 OX 결론에다가 분명한 대안이다.
어머니가 그런 이야기 없이 자꾸만 부정적인 감정을 드러내면서 비교질을 하면 아들은 폭발해버리는 것이다.
\vspace{5mm}

그리고 아들은 어떤 태도를 취해야 할까.
간단하다, 올해는 힘들 수도 있다, 그러나 난 공부를 계속 하고 싶다. 결과가 안 좋게 나온 건 죄송스럽다, 하지만 열심히 했다.
\textbf{어머니께서 날 위해 고생한 것은 인정한다, 이건 어머니 탓이 아니다.}
하지만 요즘 수험이 보통 수험이 아니다, 그리고 난 노력해서 조금 늦더라도 내 꿈을 이룰테니 협조해달라.
이렇게 쿨하게 이야기해야 한다. 물론 이렇게 이야기해놓고 롤을 하러간다면 답이 없지만.
\vspace{5mm}

거꾸로 아버지와 딸은 어떨까. 이 경우는 정반대인 경우다.
일부 예외가 없는 건 아니지만 아버지들은 두가지다. 나 골프치러가야하니까 몰라 알아서 해, 혹은 개입 한계를 두고 뒤에서 보조해주기.
그런데 이 경우 딸들은 또래 남학생들보단 정신적으로 성숙한 경우가 많다.
아버지가 우리 딸 예뻐, 머리 좋아 라는 식으로 인정해주고 지원만 잘 해주면 대체로 갈등할 이유가 없다
아버지가 무관심해서... 라는 건 모르겠지만 적어도 아버지가 자기가 초조해서 딸에게 스트레스 푸는 경우는 일반적이진 않단 이야기.
이게 수험에 있어서 여학생들이 남학생들보다 유리한 한 가지 이유는 되지 않을까 싶다.
사실 아버지 입장에서야 딸에게 돈봉투와 함께따뜻한 말을 건네주고 인자하게 쳐다봐주면 되기 때문에.
\vspace{5mm}

그럼 아버지와 아들, 어머니와 딸.
이 경우도 갈등은 심해보이지만 실제로 그렇게까지 심하진 않다.
가장 으르렁거리는 건 아버지와 아들이지만 사실 갈등이 있어도 이건 '적분'이 불가능하다. 왜냐면 불연속적이기 때문에.
뭐 그것도 그렇지만 아버지는 아들에게 개입을 할 수도 없고, 보통은 어머니가 그런 역할을 맡는다는 것.
어머니와 딸의 경우는 물론 갈등이 있을 수도 있는데, 이건 보통 어머니가 자기가 못 이룬 걸 딸이 이루는 것으로 대리만족시키는 경우?
그래도 이게 어머니와 아들보단 나은 게 감성토크가 가능할 뿐더러, 이 경우는 상하관계가 분명하기 때문에 낭비 같은 건 그다지 없다.
\vspace{5mm}

이와 더불어 논할 건 이건데 $-$ 만약 학부모가 이 글을 보신다면 도움이 될지 모르겠지만 $-$
요즘 입시는 현 4, 50대분들 때보다도 더 고난이도다. 문제수준이 쉽다 어렵다 그게 문제가 아니다,
과거에야 삼수사수하는 경우가 드물었지만 지금은 삼수사수가 보편적인 케이스로 가고 있고,
심지어 대학을 졸업한 후에도 취업 안 된다고 다시 수능치는 경우까지 생기고 있다.
게다가 인터넷 덕분인가 10대들도 상당히 똑똑해지고 어른들보다도 수험 정보가 빠삭한 경우가 많다(수험 정보'만' 빠삭한 경우도 있지만)
게다가 정시폭도 좁아져서 패자부활전도 어렵다, 수시야 온 가정이 총동원된 총력전이 아닌가.
\vspace{5mm}

+ 핵가족화 영향도 강한데. 보통 형제가 있는 경우가 없는 경우 차이가 크다.
이건 다른 얘기로 남학생들의 사회화 문제와도 관련이 있다. 남학생들은 남자 어른, 형제, 친구들과 적당히 어울릴 필요가 있다.
그런데 그렇지 못 한 케이스는 어떤가. '마마보이'는 기본인 건 그렇다 치고, 사춘기가 오면 이 때가 문제다.
어렸을 때부터 품 안의 자식으로 키운 어머니들은 아들이 영원히 귀여울 줄 알고 엄마 말만 듣고 잘 나갈 거라고 착각을 하시는데
당연히 그렇게 자란 남자애들이 잘 될 리가 있나.
\vspace{5mm}

그래서 마마보이 코스가 나중에 스트레스, 화를 주체 못 하는 케이스가 많다. 그리고 어머니들은 이걸 쉬쉬하기 바쁘고.
남학생들은 '안전한 범위' 내에서 어른에게 빠따질도 맞고 형들에게 삥도 뜯기고 친구들과 싸움도 하고...
표현이 그렇긴 하지만 과장하면 저렇단 것이고.
저런 식으로 남성들만의 수직서열 교육받고 가야 무난해지는데 요즘은 이런 게 힘들어진다.
어처구니없지만 이걸 해결해주는 게 바로 '롤'이다(...)
뭔가 왜곡된 현대사회의 구조 때문에 결핍된 것을 우회적으로 충족시켜주는 면도 있다는 것을 어른들이 모른다는 것.
어머니들은 아들을 이해한다고 하지만 그 아들의 남성성에는 무지하다.
그래서 아들이 엄마 말만 듣고 공부만 하면 된다라고만 생각해버리니 문제가 터지는 것이다.
\vspace{5mm}

남자로 자랐어야 하는데 치마 속에서만 자란 남자애들이 설령 좋은 대학에 간다고 하더라도 앞길이 순탄할지 장담하기는 힘들다.
한국사의 조선왕가도 그렇지만 곳곳을 보면 '창업군주'나 같이 활약한 '2세 군주'는 능력이 좋지만
궁궐에서만 키워진 경우 능력이 별로인 경우도 이런 것과 무관하진 않을 것이다.
\vspace{5mm}







\section{OX}
\href{https://www.kockoc.com/Apoc/488094}{2015.11.14}

\vspace{5mm}

고교시절에 배운 국어영역으로 사람들을 '논리적으로 설득할 수 있다'라고 대부분 착각합니다만.
\vspace{5mm}

그건 \textbf{씨알도 안 먹히는 이야기}입니다.
\vspace{5mm}

살짝 정치 이야기를 하자면
고 박정희 대통령이 고 김재규와 갈등이 벌어지게 된 이유 중 하나가
뭔가 장황하게 설명만 하지 확실히 결론을 내지 못 하는 김재규의 프리젠테이션 실력이라고 하더군요.
그에 비해 고 차지철은 뭔가 무모하긴 해도 확실히 결단을 짓기 때문에 신용을 얻었다나(라고 하지만 우리는 그 결말을 알고있죠)
\vspace{5mm}

인생은 OX가 아니라고들 합니다. 저도 그렇게 생각한 적이 있습니다만,
\textbf{실제로는 OX로 나눠 떨어지는 경우가 거의 대부분입니다.}
\vspace{5mm}

삶의 반대는 죽음이고 죽음의 반대는 삶이죠. 사실 그 외는 문학적 상상력이나 말장난.
합격의 반대는 불합격이고, 불합격의 반대는 합격이죠.
사실 행복과 불행의 중간지대가 있으면 뭐하러 행복을 추구합니까?
합격과 불합격 사이에 다른 길이 있으면 시험공부를 할 이유가 있을까요?
\vspace{5mm}

오히려 회색지대를 설정하는 것이야말로 실제로는 '무책임'을 위한 변명인 경우가 많다는 걸 많이 경험한 바 있습니다.
물론 그냥 심신의 위안을 주는 대화라면 회색지대를 설정할 수는 있습니다.
그러나 시험에 떨어진 사람이 극복을 하려면 재도전해서 합격하거나,
아니면 그에 버금가는 다른 성공을 거두는 수 밖에 없지
누가 와서 하는 위로든, 위로 그 자체로든 사실 아무 소용이 없습니다.
이런 위로를 좋아하는 쪽은 사람을 속여먹는 사이비 종교인들이죠.
\vspace{5mm}

과정이 중요하다... 라고 하는 건 거짓말입니다. 과정도 결과를 위한 것이지요.
다만 인생살이에서 A를 의도한 B라는 과정이 실패하더라도 생각치 못 한 C라는 성과를 거둘 수도 있고,
반면 D를 의도한 E라는 과정이 성공하더라도 F라는 참사를 가져올 수 있으므로 과정을 강조하는 것일 수 있는데
이 역시 결국 "과정이 복합적인 결과를 가져오기 때문"이라는 점에서 그런 것입니다. \textbf{결과가 더 중요하다라는 건 바뀌지 않습니다.}
\vspace{5mm}

아무리 의도가 좋았다고 하더라도 결과가 나쁘면 그건 꽝인 것입니다.
반대로 의도가 나빠도 결과가 좋으면 어떻냐. 인정은 해줘야죠. 다만 그게 다른 나쁜 결과를 가져올 수 있기 때문에 문제가 되겠죠.
결국 \textbf{'결과가 중요하냐', '결과들이 중요하냐'라는} 단수와 복수의 차이지, 결과가 중요하다는 건 바뀐 게 없습니다
\vspace{5mm}

그렇다면 나쁜 결과가 나오면 왜 위로해주냐
열심히 노력한 걸 인정하는 건 좋습니다. 그러나 그 인정이 "그러니까 지금의 상태에서 안주해도 좋다"로 끝나면 최악이 되겠죠.
나쁜 결과를 덮는 건 좋은 결과 뿐입니다.
\textbf{나쁜 결과가 나왔다면 울고 있을 게 아니라 다시 좋은 결과를 얻기 위해 일어서는 수 밖에 없습니다.}
\vspace{5mm}

흑백론적인 태도는 그럼 문제가 없는 게 아니냐.
예, 꼴통들의 사고법이죠.
그런데 살다보니까 시비가 분명하고 책임지는 꼴통들이 낫더군요.
그게 아닌 교묘한 회색주의는 결국 일은 일대로 벌이고 책임은 책임대로 지지 않고 도망가버리는 부류가 대부분이더이다.
\vspace{5mm}

거시적인 역사도 그렇습니다.
서양사에서 수학이 기여한 건 참과 거짓을 분명히 따지는 태도이죠.
검증되지 않은 건 과학에 들어가지 않습니다. 그러니까 서유럽이 세계를 지배한 것이죠.
그에 비해 동양은요? 참과 거짓을 엄중히 따지진 않았죠. 유교적 도그마, 불교적 윤회론, 도가적 순환론으로 검증을 회피했죠.
\vspace{5mm}

인간의 행위에 대한 개인적 관찰은 그렇습니다.
안 그런 사례가 많다고 하지만 사람의 행동은 결국 그 결과가 극단적으로 갑니다.
괜히 빈익빈 부익부란 말이 나오는 게 아니죠.
사람들은 이걸 가지고 그릇이 있다라거나 팔자가 있다라고만 하지만 제 생각은 다릅니다.
\textbf{실은 사소한 차이가 엄청난 차이를 유발하는 것이지요.}
그 사소한 차이를 보지 못 하면 모든 게 다 운명적인 것이라고만 생각하기 쉽지요.
운명론의 장점은 단 하나입니다. '복잡한 걸 생략하고 모든 걸 단순하게 왜곡시켜 결론내리기 좋다'
\vspace{5mm}

한자 공부 안 했으니까 국어 성적이 안 나올 것이다... 라는 진술은 일면 비논리적이고 극단적으로 보입니다.
그런데 유감스럽게도 이런 진술은 현실에서는 맞아떨어집니다.
저런 진술이 나오는 상황은 결국 한자 공부를 하느냐 안 하느냐가 경쟁을 좌우하는 게 된 케이스인 경우입니다.
즉, 저런 진술이 나오게 된 것 자체가 이미 저 진술에서 말하는 바가 꽤 결정적인 요소가 되었음을 웅변하는 것이지요.
\vspace{5mm}

개인적으로는 그래서 충고를 할 때 매우 극단적으로 단정하는 편입니다.
왜냐? 고 김재규식으로 이럴 수도 있고 저럴 수도 있다고 하는 건 '전달'이 되지 않아서입니다.
확실한 의사전달을 하기 위해서는 유감스러우나, \textbf{상대방의 마음에 상처를 주는 정도까지} 가야합니다.
수술할 때 칼로 째지 않을 수 없듯이 말이지요.
당연히 그건 합당한 근거를 갖춰야합니다.
근거가 없는 극단적인 이야기는 씨알도 먹히지 않을 것입니다.
\vspace{5mm}

+ 참고로 화작문 말고... 인간의 심리를 갖고노는 화술 등에 관한 책은 꽤 많이 구비해서 쭉 읽어왔죠.
각자 유파는 다릅니다만. 제가 선택한 건 남자답게 직선적으로 바로 핵심을 찌르지만 상대의 문제점을 해결해주는 직설법입니다만.
인간의 마음이라는 건 참 약합니다, 사소한 잡술로도 흔들리기 쉽습니다.
영화 양들의 침묵에 보면 자기가 좋아하는 스털링을 성희롱한 옆방 죄수를 한니발이 말로써 자살시켜버립니다.
그런데 이게 불가능하느냐. 그런 건 아닌 것 같더군요.
\vspace{5mm}

\href{https://namu.wiki/w/%ED%82%A4%ED%83%80%ED%81%90%EC%8A%88%20%EA%B0%90%EA%B8%88%20%EC%82%B4%EC%9D%B8%EC%82%AC%EA%B1%B4}{링크}
\vspace{5mm}

사채꾼 우시지마의 '세뇌하는 남자'란 에피소드 원안인 실제 사건입니다.
나중에 시간이 되시면 이런 쪽에 대해서 공부해보셔도 되고
현대문명이 정말 '세뇌하는 벡터들의 합'이라는 걸 깨달으실 겁니다.
사실 상담을 할 때에 제가 주목하는 그 수험생이 뭘로 세뇌당했나 가리는 것입니다.
\vspace{5mm}

+ 수학을 공부할 때 탈패턴이 무엇이냐라고 물어보십니다.
간략히 예를 들면 기하와 벡터 문제를 풀 때 "집합" 단원만 가지고 문제를 풀 수 있나, "명제" 단원만 가지고 문제를 풀 수 있나.
심지어 정규분포를 가지고 문제를 풀 수 있나 $-$ 이렇게 해보는 게 탈패턴의 시작입니다.
그럼 대부분 이렇게 얘기하겠죠. '불가능'하지 않느냐.
\vspace{5mm}

그 가능함과 불가능함을 직접들 검증해보아야합니다.
A라는 문제에는 반드시 A', A" 라는 해결방법 밖에 없다고 믿으면 그게 패턴화된 상태고
반면 전혀 상관없어보이지만 논리적 필연성이 있는 Y나 Z로 접근가능하다라고 믿고 시도해보는 게 탈패턴화된 상태의 한 예입니다.
눈 앞에 절벽이 펼쳐져있지만 사실은 투명한 강화유리 다리가 놓여져 있어 뛰어가도 된다라고 믿고 시도해보는 것이지요
(수학문제는 시도 하나가 실패한다고 바로 죽지는 않으니까요. 시간은 좀 걸리려나)
\vspace{5mm}

+ 그럼 탈패턴화가 뭔 상관이냐가 할 건데. 이 탈패턴화를 하기 위한 기초가 바로 저 OX를 믿는 것입니다.
즉, 수학은 정확히 참과 거짓을 구분해준다 $-$ 즉 고교수학 내에서만큼은 시비가 분명하다라는 걸 알고
전혀 연관이 없어보이는 요소들을 결합해보는 것이 탈패턴화입니다.
님들은 문제를 풀면서 A란 문제에 X, Y, Z를 써봐도 되는지 직접 시도해본 적은 없을 것입니다.
수학의 시비 검증성을 믿고 그걸 시도해보아야만 탈패턴화가 되는 것이지요.
\vspace{5mm}



\section{위대한 세기 $-$ 쾨셈 술탄 1회}
\href{https://www.kockoc.com/Apoc/495243}{2015.11.17}

\vspace{5mm}

$\#$ 영상
\href{http://www.alaturcaseries.com/magnificent-century-kosem-episode-1-with-english-subtitles/}{링크}
\vspace{5mm}

$\#$ 나무위키 설명
\href{https://namu.wiki/w/%EB%AC%B4%ED%9D%90%ED%85%8C%EC%86%80%20%EC%9C%A0%EC%A6%88%EC%9D%B4%EC%9D%84:%20%EC%BE%A8%EC%85%88?from=%EC%9C%84%EB%8C%80%ED%95%9C%20%EC%84%B8%EA%B8%B0%3A%20%EC%BE%A8%EC%85%88}{링크}
\vspace{5mm}

전세계인의 성화에 영어자막을 단 버전이 이제 올라옴.
\vspace{5mm}

이슬람권이라고 해도 터키는 좀 달리봐야할 듯. 말이 이슬람 국가지 미국 영향을 많이 받아 우리나라와 비슷하죠.
터키는 성속분리가 이뤄진 국가라서. 이슬람을 믿지만 실제 문화는
1화에서는 엉겁결에 평범한 삶을 살지 못 하게 되는 남녀의 이야기에다가
역시나 전세계 아줌마들을 사로잡는 고부갈등까지 등장하는데
한국 드라마 그냥 능가. 걍 비교가 되지 않는다.
\vspace{5mm}
\begin{itemize}

    \item $-$ 아메드 술탄이 즉위할 때 나이가 거의 수험생과 비슷햇죠. 그래서 긴장하고 두려워하는 장면 구현은 잘한 듯.
    즉위식 전에 떨려서 문 닫고 헉헉댈 때 옆 호위대장 데르비슈가 격려하는 건 따로 메모하고 적어둘 대사인 듯.
    터키 배우들 이름이 어려워서 언급은 안 합니다만 아메드 술탄 배우 잘 생기기도 했지만 연기 잘 하네요.
    \vspace{5mm}

    \item $-$ 납치당하는 우리 주인공이 저 간지나는 흑누나와 교감쌓는 건 뭐 스톡홀름 증후군
    흑누나 눈빛 정말 지리는 듯.
    예니체리 꽃미남(보나마나 주인공 둘러싸고 삼각관계이겠지) 대사를 보니
    저 시대에도 먹고 살기 힘든 취업난 때문에 목숨걸고 싸우는 예니체리 직장이 꿀직장이었구나란 생각이.
    (그런데 꿀직장이면 뭐하나. 예니체리는 원래 독신인데)
    \vspace{5mm}

    \item $-$ 시할머니(?) 사피예 술탄이 포스가 장난이 아님.
    여자는 여러가지 이름을 갖게된다 이름을 가진다라고 하는 것은 꽤 명대사.
    실제로 우리 주인공 쾨셈이 나중에 저 시어머니보다 더 독한 여자가 되죠.
    역사적 사실로 보면 사피예 술탄 $-$ 한덴 술탄 $-$ 쾨셈 술탄 : 고부갈등 최강갑이 되어서 전세계 아줌마들 걍 잡을 듯
    \vspace{5mm}

    \item $-$  하렘은 여성 \textbf{내무반}.
    저 시대 신병 받아라 하는 식의 농담이라는 건 집단생활하는 어디든 있지 않나 싶음.
    역사적으로 술탄 눈에 들기 위해 별 춤을 다 추고 거기서 발달한 게 벨리댄스라고 들은 바 있는데
    과거나 지금이나 경쟁은 똑같은 듯. 자기 아이를 술탄으로 만들어야 Valide Sultan이 되어서 권력을 휘두를 수 있으니.
    \vspace{5mm}

    \item $-$ 로맨스 코드는 적절히 집어넣은 것 같은데
    우리 주인공 아니스타시아보다도 아메드 술탄이 더 예뻐(?) 보인다는 게 참.
    아나스타시아는 그리스인 여배우로 아는데 적국(?) 터키의 드라마에 나온 것을 보니 많이 바뀐 듯
    (그리스와 터키의 감정은 한일감정 저리가라할 정도입니다)
    \vspace{5mm}

    \item $-$ 1화 제목이 사자, 늑대, 양인데 이게 참 절묘한 제목이네요.
    사자 $-$ 아메드 술탄, 늑대 $-$ 주인공 아나스타시아, 양 $-$ 예니체리
    하지만 정작 극중에서 아메드와 아나스타시아는 선량한 인물들이고(적어도 아나스타시아는 지금까지는)
    예니체리는 매우 잔학무도한 광신자들인데 이들이 양(sheep)으로 묘사되다니
    그런데 대사들이 참 문학적인지라 연기도 연기지만 영어자막 대사 읽는 맛에 시간 가는 줄 몰랐습니다.
    \vspace{5mm}

    \item $-$ 단, 터키드라마 단점이 1호가 무려 2시간 30분.
    진짜 보려면 날잡고 보든가 적당히 스킵해서 주요 장면만 봐야하지 않을까 싶음.
    아무튼 배우들 포스가 쥐리고 남녀 주인공 외모가 시원한지라 눈호강은 제대로 한 것 같습니다.
    \vspace{5mm}

\end{itemize}
사극은 일단 우리나란 건 포기했고(무인시대 이후로는 맛갔음, 정도전도 개인적으로는 별로였습니다)
일본 것도 지겨운 센고쿠 지다이.
그래서 결국 알음알이로 알게 된 게 터키 사극인데 영어자막이 지원되니까 볼만하네요.
\vspace{5mm}















\section{서적 : 동경대 강의록,}
\href{https://www.kockoc.com/Apoc/497599}{2015.11.18}

\vspace{5mm}

별 $\bigstar$$\bigstar$$\bigstar$$\bigstar$$\bigstar$ + $\bigstar$$\bigstar$$\bigstar$
\vspace{5mm}

그간 읽었던 기존의 사회과학책들을 모두 리셋시켜버린 책이다.
저자는 사카이야 다이치 $-$ 일본 관료계의 이단아이자 학계의 괴물 중 하나 :
강연을 들은 학생들은 도쿄대 학생들
\vspace{5mm}

동경대라니 뭐 새로운 게 있겠냐라고 들춰보았다가 일주일동안 열병에 앓았던 기억이 새록.
사회과학적으로 사고하는 방법을 익히는 게 굳.
\vspace{5mm}

그런데 이런 책들은 특징이 있더군.
\begin{itemize}
    \item 첫째, 그 쓸만한 책들은 거의 다 일본인들이 쓴 것이다.
    \item 둘째, 우리나라 책들은 쓰레기이다. 좌우를 떠나서 사고할 줄 모르는 애들이 많은 게 그 스랙 책들 때문이다.
\end{itemize}
\vspace{5mm}

이번에 삼사조합으로 가는 분들도 있고 혹은 비문학적인 사고가 뭔지 궁금한 사람들도 많을 건데
이를 위한 사고의 바탕으로서는 이 책을 읽으시면 된다.
이런 책은 비공개로 가도 좋다고 보지만 어차피 추천한다고 해도 안 읽을 놈들은 안 읽기도 하겠고
사실 책을 추천 안 하는 이유가 그게 악용되는 경우 때문인데 이 책만큼은 악용될 이유는 없다고 보기 때문이다.
\vspace{5mm}

그리고 윗 예측은 거의 다 맞았다. 한국사회도 사카야이 다이치가 말한 지가사회로 탈바꿈한지 오래다.
굴지의 재벌에서 그의 견해를 적극 참조해서 반영했다는 뒷 이야기도 있지만 이건 사족일까.
\vspace{5mm}
\begin{enumerate}
    
    \item 일단 역사를 설명하는 부분은 지루해보이지만 그건 아니다. 역사의 해석이 곧 미래의 전망임을 알 수 있다.
    사카이야 다이치의 세계사 설명(아울러 일본사)을 보면 상당히 프레임이 정확하고 미래 예측까지 담보한 것이다.
    그동안 몰랐던 서유럽 문명사에 대한 개괄적 프레임이 이 책 한권으로 깔끔히 정리되었다는 것.
    \vspace{5mm}
    
    \item 일본사회에 대한 실질적 비판과 전망이 두드러진다.
    이 책을 읽고 감전되면서 일본 애니나 만화를 넘어 서적까지도 적극 읽으면서 비교한 것.
    우리나라 지식인들이나 교수들은 일본과 비교가 되기 힘들다는 것.
    \vspace{5mm}
    
    \item 우리나라에서 책을 고를 때는 일본인 저자가 쓴 것부터 참조하도록 하자.
    그게 좋은 책을 고르는 비결 중 비결이다.
\end{enumerate}
\vspace{5mm}







\section{[게임] 46억년전 이야기.}
\href{https://www.kockoc.com/Apoc/500223}{2015.11.19}

\vspace{5mm}

\href{https://namu.wiki/w/46%EC%96%B5%EB%85%84%20%EC%9D%B4%EC%95%BC%EA%B8%B0}{링크}
\vspace{5mm}

세기의 걸작 <46억년전 이야기>
그동안 과학에 많은 변화가 있어서 명왕성도 퇴출당하고
고생물학에서 수정된 것도 있지만 그렇다고 이 게임의 주제는 바뀐 것 같지는 않다.
\vspace{5mm}

초기에 피라미에서 출발해 양서류, 파충류, 포유류, 그리고 인간까지 도달해나가는데
최종보스를 무찌르려면 굳이 인간으로 진화할 필요는 없다(그보다는 드래곤, 박쥐, 아니면 익룡이 더 낫기도하고)
지금보다 파격적인 건 주인공이 진화하기 위해 약자들을 먹어치워 진화포인트 채우고
거의 식인이나 다를 바 없는 짓도 서슴없이 벌인다는 것.
\vspace{5mm}

저 게임이 우리나라에 소개될 때에는 이기적인 유전자가 인기 얻기 훨씬도 전인 90년대였으니까.
사실 일어 못 하면서 걍 액션플레이하는 맛에 하던 건데 지금 보면 정말 무시무시한 주제를 지닌 작품.
그 시대야 IMF 오기 전이라 다 국뽕맞아서 흥청망청 걍 대학가면 무조건 대기업 취업... 그러는 시대였는데
어른들이 우습게 보던 일본 패키지 게임에서 진실을 설파하고 있었으니까.
\vspace{5mm}

죽을 때마다 진화포인트를 절반씩 빼앗아나가는 가이아는 2d 여신의 원조라 할 듯.
요즘 게임이라면 엔딩이야 인간이 되어서 가이아랑 (생략)하고 잘 먹고 잘 살았다이겠지만
저 엔딩은 뭐랄까 '인류의 여명'이랄까. 아무튼 인간이 숲에서 지구의 주인이 되어나가는 동틀녘 장면으로 마무리된다.
\vspace{5mm}

+
\vspace{5mm}

윗 영상은 꽤 슬픈 영상이다. 처음보면 이게 뭐야 ㅋㅋ 그러겠는데 6분대에 가면.
\vspace{5mm}




\section{서적 : 카지노(김진명)}
\href{https://www.kockoc.com/Apoc/501624}{2015.11.19}

\vspace{5mm}

김진명이 극우보수주의자건 쇼비니스트건 알 건 없고
유치뽕짝 어쩌구 전에 소설이 재밌긴 재밌습니다. 뭔가 유치한 것 같으면서도 한번 읽고 끝내지 않고 여러번 읽게되죠
이런 소설에서 문학성을 기대한다... 는 건 현학적인 것 같고 터무니없는 걸 알면서도 그냥 재밌기 때문에 읽는 겁니다.
\vspace{5mm}

그런데 그 많은 김진명 작품 중에서 '카지노'는 좀 색다르죠. 편의점 과자 코너에 놓인 허니버터칩? (안 팔리고 비치돼있는게 수상)
물론 내용이 풍부하다고 보기도 어렵고 등장인물들이 뭔가 비현실적이고 밋밋하긴 한데
도박에 임하는 정신자세라거나 승부기법, 그리고 멘탈관리에 있어서는 핵심적인 것이 생각보다 잘 나와있습니다.
(바카라의 세세한 규칙이 다 나와있는 건 아니지요)
\vspace{5mm}

읽어보시는 분은 왜 제가 이 책이 공부계획에 약간 도움이 될 수 있는가 아실 수 있을 겁니다.
굳이 구입할 건 없습니다. 도서관 어디든 다 비치되어있을테니 대출해보세요.
개인적으로는 생각 외로 재밌단 기억 때문에 헌책방에서 구입할까 고려 중이긴 합니다.
어떤 승부에 임하거나 공부할 때 읽어둘 구절들이 있어서 말이죠.
\vspace{5mm}

+
\vspace{5mm}

허혁재님에게 드렸던 충고 중 하나가 저 책 말미에 나온 겁니다.
\vspace{5mm}






\section{[게임] 화이트데이 신판}
\href{https://www.kockoc.com/Apoc/501645}{2015.11.19}

\vspace{5mm}

$\#$ 아래 영상은 공포일 수도 있으니 함부로 플레이하지 마시길
\vspace{5mm}

오늘따라 학교가 좀 이상한 것 같아. 그럼 언제 학교가 정상이었니?
\vspace{5mm}

나무위키 소개
\vspace{5mm}

\href{https://namu.wiki/w/%ED%99%94%EC%9D%B4%ED%8A%B8%EB%8D%B0%EC%9D%B4%3A%20%ED%95%99%EA%B5%90%EB%9D%BC%EB%8A%94%20%EC%9D%B4%EB%A6%84%EC%9D%98%20%EB%AF%B8%EA%B6%81(2015)$\#$rfn$-$18}{링크}
\vspace{5mm}

원작은 복돌이 때문에 망한(...) 비운의 걸작 귀신보다 무서운 불법복제, 허혁재군이 빡칩니다
염가판 주얼판으로 우연히 구매했다가 감동 처먹고
스트레스 받을 때마다 방 안 조명 다 끄고 서라운드 스피커로 플레이..... 하다가 안경을 끼게 된 무시무시한 게임.
\vspace{5mm}

우리나라에서만 나올 수 있었던 걸작인 게 아니라 다시 나와서 기쁘네요.
\vspace{5mm}

매우 독창적인 설정에다가 스토리 철학이 꽤 마음에 들었음.
배경음악인 황병기 미궁가지고 이토록 사운드 잘 뽑아낸 게(... 황병기 미궁 이 게임 때문에 유명해졌을 듯 ...)
\vspace{5mm}

일단 주인공이 공격을 할 수가 없어요. 무조건 도망가야함. 게다가 귀신보다 사람이 무섭다고 정신나간 수위가 더 무서움.
이 놈의 대머리 수위 아저씨 봉구형님이 주무시지도 않고 밤새 일하심
학생들은 하라는 공부는 안 하고 학교에서 잉여짓 중인데 장년인 봉구형님과 달수형님은 휴식도 안 하고 근무 중.
조금만 소음 내도 내일 귀신들 저승사자 입시인데 방해하냐고 낄낄낄 웃으며 배트 들고 쫓아오심.
신나게 따돌리고 여자 화장실에 처박혀서 흐응하고 있으면 스윽하는 마찰음와 함께 머리귀신님 출현(...)
네가 거기 있으면 내가 모를 줄 알았니?
\vspace{5mm}

거기다가 등장하는 주인공들 사연이. 뭐 이미 스포 뜰 때로 다 떴지만 스토리 변경은 있지 않았을까 싶은데.
국산 게임 치고 정말 설정 잘 해서 누가 귀신인지 알 수가 없었죠.귀신 사연도 참 불쌍.
학교괴담에서만 등장했던 귀신 거의 다 구현시킨 것 같네요.
(칠판 귀신 $-$ 수학문제가 안 풀린 걸로 한이 맺혀 죽은 여고생 귀신도 나온 것 같고)
\vspace{5mm}

...
\vspace{5mm}

지금 생각해보면 좀 어이없음.
걍 얼굴보고 예쁘다 생각한 여고생에게 화이트데이 선물주겠다고 밤늦게 들어간 우리 이희민군, 이거 \textbf{스토커} 아녀?
도대체 연두고등학교는 뭘 숨길 게 많기에 수위 아저씨가 무슨 특전사, 해병대 수준
이런 인재들은 대기업 총수 사저나 청와대 경비로 갔어야하는 것 아녀?
도대체 자정녘에 청순한 여고생 3명이 수학교재 한권도 없이 학교에 기어들어와있는지도 이해가 안 가고
아니, 그리고 저 수위들은 보안 철저히 한다면서 미친 아줌마는 냅두고 있음?
당직교사는 없어?
\vspace{5mm}

그런데 더 경악스러운 사실은
\vspace{5mm}

\textbf{교실 돌아다니다보면 참고서 한권도 안 보임(...)}
학교 시설만 좋지 졸라 꼴통 학교였던 듯. 구관이든 신관이든 공부한 흔적 일체 안 보임.
애새끼들 대학진학은 걍 포기하는 참교육 실현의 현장이었구나.
\vspace{5mm}

이러니까 학교 강당이 작살나도 \textbf{소방차는 커녕 아무도 오지 않지}.
누군가 학교를 개박살내서 특별재난지역 선포, 재학생들 대입특례전형을 받는 걸 기대했던 것 같음(...).
\vspace{5mm}

신판에서 지현양 비중이 좀 높아진 것 같고(구판에서는 정말 꿔다놓은 보릿자루도 아니고)
주인공 외모가 좀 기생오래비급으로 바뀐 것은 그런데 우리 수위 아저씨들 경기가 어려워서 그러나 핼쑥해지셨고
메인 히로인 \textbf{1} 소연이는 수학 \textbf{1}000문제는 풀어댔는지 졸라 초췌해졌고
메인 히로인 2 성아는 수시합격 개꿀이라도 빨았는지 왜 이리 피부가 좋아졌어?
\vspace{5mm}

...  어 그런데 양정화 성우는 10여년 넘게 흘렀는데 목소리가 그대로다 ... 이것도 꺄악.
\vspace{5mm}

그나저나
\vspace{5mm}

이거 입수 못한 건 아깝다.
초레어 극강아이템 머리귀신 우산
\vspace{5mm}

저거 비오는날 밤 아래와 같이 이렇게 쓰고다니면
\vspace{5mm}

\href{http://gall.dcinside.com/board/view/?id=fantasy$\_$new&no=3200399&page=1&search$\_$pos=&s$\_$type=search$\_$all&s$\_$keyword=%ED%99%94%EC%9D%B4%ED%8A%B8}{링크}
\vspace{5mm}

경찰서에 끌려가겠죠 뭐
\vspace{5mm}










\section{머리가 좋은 경우}
\href{https://www.kockoc.com/Apoc/504246}{2015.11.21}

\vspace{5mm}

간단히 말해서
공부할 환경이 아닌데 $-$ 정상대로라면 농사나 짓거나 노가다나 뛰고 있어야하는데
환경의 강요와 정반대로 \textbf{'공부해서' 올라가려고 하는 케이스}.
\vspace{5mm}

이 경우야말로 머리가 좋다라고 보는 케이스다.
두뇌회전이 빠른데 환경이 좋다면, 이건 환경 덕분이다.
이런 케이스는 좌절 몇번 경험하거나 환경이 나빠지면 다시 멍청해진다.
\vspace{5mm}

하지만 환경이 매우 안 좋은데도 진짜 아득바득 공부해서 올라가는, 포기하지 않으려하는 케이스가 있다.
실제로는 이게 다이아몬드 원석이다.
같은 환경에 처한 동기들이나 친구들은 그 환경의 노예가 되어 살아간다.
그런데 본인들은 계속 공부하려고 하면서 결국 '탈출'에 성공한다.
\vspace{5mm}

\textbf{머리가 좋다고 하려면}
\textbf{환경을 극복하느냐로 봐야하는 것이다.}

흥미로운 건 이런 케이스들은 자기들이 머리가 좋다는 걸 모른다.
이 케이스들은 방법론만 잘 가르쳐주고 정신적 서포틀 잘 해주면, 요즘 말로는 캐리 잘 해주면 정말 자알 올라간다.
바꿔 말해 자기가 환경이 엿같고 머리가 나쁘다라고 해서 안타깝게도 몰락하는 경우도 많다.
머리가 좋더라도 자기 확신이 지나치면 스스로를 몰락시킬 수도 있는 것이다.
부모가 밀어준다고 해서 그냥 좋은 대학 간다.... 이런 케이스는 잘 되는 경우는 드문 것 같다.    그러나 환경을 넘어선 사람들은 사실 뭘 해도 먹고 살거니와 결국 이겨내는 걸 보게 된다.








\section{위대한 세기 $-$ 쾨셈 술탄 2화}
\href{https://www.kockoc.com/Apoc/508026}{2015.11.24}

\vspace{5mm}

http://www.alaturcaseries.com/kosem-episode-2-english-subtitltes-please-use-the-donate-button-below-to-support-the-translations/
\vspace{5mm}

2화 총평 : 전세계인 공략을 감안한 웰메이드
\vspace{5mm}

이번 화에서는 예니체리가 상세히 그려져 있어서 좋았고
하렘에서 여자들끼리 벌이는 온갖 종류의 싸움(신경전, 예의전, 그리고 육박전)까지 벌어져서 만족.
로맨스, 음모, 가족애, 액션 등이 정말 잘 어우러져있음.
\vspace{5mm}

거기다가 주변부 인물들의 개성이 두드러지는데
칭기스칸의 자손으로 오스만 황제 자리를 노리는 샤힌 기라이는 정신적으로나 육체적으로는 만렙 캐릭터이고
왠지 쩌리 기운이 강한 메메드 기라이는 알고보니 아흐메드의 고모와 러브레터 주고받는 사이(이 녀석이 어째 샤힌 발목잡을 삘이다)
\vspace{5mm}

우리의 주인공 아나스타시아는 아직까지는 순수한 소녀
남주 아흐메트가 동생 무스타파를 죽이려다가 살리는 치유계 인물 역할 잘 해냈음((그런데 역사상 최악의 악녀 중 하나로 흑화할 건데 어찌될까)
마흐피뤼제가 아흐메트와 첫날밤 보낼 때 질투하는 표정 연기가 두드러지는데..
아, 이 배우 아나스타시아 치칠리우(주인공과 배우 이름이 같고 국적도 같다)는 시즌 1에서만 나오고 그 뒤에 하차할 테고
흑화된 쾨셈(=어린 시절 이름 : 아나스타시아)은 Beren Satt라는 나름 유명한 터키 여배우가 맡는다고 하는데 위화감 잘 극복할 수 있을까.
\vspace{5mm}

지금 봐도 귀여워 소리나와 쇼타증후군을 부추기는 무스타파 꼬맹이는 연기할 때 힘들었을 것 같다.
거의 1시간 분량동안 울고 쫓기고 감금당하고 심지어 교살당하기 직전까지 최악의 경험까지 연기 다한 건데.
실제 역사에서 무스타파는 목숨을 부지하는 대신 감금당해 살았고 술탄으로 즉위했을 때는 금치산자 수준이었음.
그 미치광이 술탄이 꼬맹이 시절에는 저렇게 귀여웠다니... 에서 저 쪽 시청자들의 심금을 수백번은 울렸을 듯.
\vspace{5mm}

샤힌 기라이는 카리스마 장난 아니네
\vspace{5mm}

22:30 $-$ 아흐메트 살해용으로 쓰려다 실패한 사자 시체를 해부하는 장면(사자=황제라는 점에서 은유적)
54:30 $-$ 콘스탄티노플을 탈출하려는 무스타파 일행을 공포스럽게 막는 장면
56:51 $-$ 무스타파 일행의 공범에게 죽을지도 모르는데 태연히 양들의 침묵에 나오는 한니발 렉터급으로 대처하는 장면.
\vspace{5mm}

배우가 보통이 아니라서(캬. 눈빛 보소, 형님 저를 가지세요... ) 검색해보니
\vspace{5mm}

\href{https://www.google.co.kr/search?q=Erkan+Kol%C3%A7ak+K%C3%B6stendil&biw=1366&bih=614&source=lnms&tbm=isch&sa=X&ved=0ahUKEwiMwJ6MvajJAhVFE6YKHYKJChcQ$\_$AUIBigB}{링크}
\vspace{5mm}

교육적 차원에서 예니체리 훈련장면도 흥미
일단 주인공 이스켄데르(=알렉산더)는 '안드로'가 본이름이 아님.
타인사칭을 보니 이거 분명 첩자 같다. 능력이 범상치 않고 예니체리 내부를 유심히 탐문하는 걸 보니
\vspace{5mm}

34:00 $-$ 내무반 우두머리를 뽑을 때 달리기 시합한다는 건데 실제로는 brain test. 이스켄데르는 꼴찌하면서도 1등하는데
01:40:30 $-$ 군대 내무반의 아침, 오스만이 당시 잘 나갔던 건 일찌기 저런 식으로 근대적 조직체제의 군대를 훈련시켰다는 장면
01:45:00 $-$ 예니체리가 실은 거대한 요리사 집단이기도 한 것을 보여줌 ; 참고로 예니체리의 거의 모든 용어는 요리 용어였음.
\vspace{5mm}

극 중 마지막에 손자에게 복수하려는 사피예 술탄의 대사가 "솥을 뒤얹고 수프가 끓어오른다"라는 건데
예니체리가 솥을 뒤얹는다는 건 말 그대로 '역모'를 의미하는 것임.
이 드라마에서 예니체리가 반역하는 건 최소 2번은 일어나기 때문에 솥을 뒤얹는 장면은 여러번 나온다.
\vspace{5mm}

가슴아픈 캐릭은 마흐피뤼제 술탄.
자기 부모도 모르고 그냥 하렘에 헌납해서 황제에게 몸바치고 아들 낳을 것만 생각하는 걸로 훈련된 소녀인데
지금으로 치면 정시로 대학가려는 친구인데 낙하산으로 아나스타시아가 훽 오더니 술탄 마음을 빼앗아 갔으니
실제 역사도 불행함
아들인 오스만 2세가 즉위하긴 하는데 젊은 개혁군주 오스만 2세의 운명도 참 박복했거니와
마흐피뤼제 술탄도 30세의 나이로 사망했는데 이유가 불명. 아마도 쾨셈에게 암살당했을 것으로 여겨짐(드라마에선 그렇게 그려지겟지)
배우는
\vspace{5mm}

\href{https://www.google.co.kr/search?q=Erkan+Kol%C3%A7ak+K%C3%B6stendil&biw=1366&bih=614&source=lnms&tbm=isch&sa=X&ved=0ahUKEwiMwJ6MvajJAhVFE6YKHYKJChcQ$\_$AUIBigB#tbm=isch&q=+Ceyda+Olguner}{링크}
\vspace{5mm}

사실 외모는 좀 투박한데 몸매는 육감적인 $-$ 애 잘 낳게 생긴 걸로 묘사되었으니가 캐스팅은 나쁘지 않은 듯.
\vspace{5mm}

반면 아나스타시아는
\vspace{5mm}

\href{https://www.google.co.kr/search?q=Erkan+Kol%C3%A7ak+K%C3%B6stendil&biw=1366&bih=614&source=lnms&tbm=isch&sa=X&ved=0ahUKEwiMwJ6MvajJAhVFE6YKHYKJChcQ$\_$AUIBigB#tbm=isch&q=Anastasia+Tsilimpiou}{링크}
\vspace{5mm}

이 정도면 고생한 엘프는 되지 않을까 싶고
\vspace{5mm}

기대된다는 배우 Beren Saat(성년의 쾨셈 연기)는
\vspace{5mm}

\href{https://www.google.co.kr/search?q=Erkan+Kol%C3%A7ak+K%C3%B6stendil&biw=1366&bih=614&source=lnms&tbm=isch&sa=X&ved=0ahUKEwiMwJ6MvajJAhVFE6YKHYKJChcQ$\_$AUIBigB#tbm=isch&q=beren+saat}{링크}
\vspace{5mm}

사랑과 전쟁을 잘 연기할 배우로 보임(터키에서는 매우 유명하다던가)
나머지 꽃청년(주인공인데 나머지라고 하는 건 좀 이상하지만)
\vspace{5mm}

예니체리(첩자) 이스켄데르 역
\vspace{5mm}

\href{https://www.google.co.kr/search?q=Erkan+Kol%C3%A7ak+K%C3%B6stendil&biw=1366&bih=614&source=lnms&tbm=isch&sa=X&ved=0ahUKEwiMwJ6MvajJAhVFE6YKHYKJChcQ$\_$AUIBigB#tbm=isch&q=Berk+Cankat}{링크}
\vspace{5mm}

술탄 아흐메트 역
\vspace{5mm}

\href{https://www.google.co.kr/search?q=Erkan+Kol%C3%A7ak+K%C3%B6stendil&biw=1366&bih=614&source=lnms&tbm=isch&sa=X&ved=0ahUKEwiMwJ6MvajJAhVFE6YKHYKJChcQ$\_$AUIBigB#tbm=isch&q=Ekin+Ko%C3%A7}{링크}
\vspace{5mm}

저거 블라인드 테스트 하면 무슬림이라고 찍을 사람 아무도 없겠지.
학교에서도 터키 형님들이 오셔서 같이 공부한 적이 있는데 돼지고기 못 먹는 백인 정도.
이슬람 관습만 아니면 사실 터키, 이란 남녀가 전세계를 쏵 발라버릴 거라고 생각함.
\vspace{5mm}

중간에 무슨 배우사진 올리고 그랬는데 아무튼 2화는 참 잘 짜여졌다고 생각.
우리나라 제작진들도 저걸 열심히 표절 벤치마킹하겠지.
\vspace{5mm}

+
\vspace{5mm}

한편 흥미로운 건 그리스$-$터키 관계는 한일 관계 저리가라할 정도로 앙숙.
사실 민족주의도 그리스 독립에서 시작된 것이기도 하고 그리스도 비잔티움 제국의 역사가 후덜덜한지라.
그런데 터키 사극에서 그리스 출신인 쾨셈 역을 그리스 배우인 아나스타시아가 맡은 것을 보면 문화적으로는 역시.
\vspace{5mm}

위대한 세기 이전 버전에서야 하기야 그리스에서도 이 터키사극이 인기가 매우 좋아서 일부 극우주의자들이 문화적 침략이라고 시위까지 했다던데
그와 별개로 저 드라마 제작진들은 그리스에서 추파를 던지기로 한 듯. 하기야 쾨셈이 그리스 여자가 아닌가.
거기다가 베네치아를 강조하고 이탈리아어까지 극중에 집어넣은 것을 보면 유럽 시장까지 다시 정복하려고 한 듯.
\vspace{5mm}

그런데 더 무서운 건 작가를 포함한 드라마 진들이 국뽕은 그리 빠지지 않았다는 것.
자기 조상들이 잔인하고 야만적인 것도 걍 사실 그대로 집어넣음. 그리스인들이 왜 독립할 수 밖에 없었나 하는 것도 집어넣는데.
우리나라에서 그런 걸 집어넣는 건 힘들겠지?
\vspace{5mm}

다른 이야기지만 중국에서는 대만을 역사적으로도 다시 흡수하기 위해서
장제스조차도 좋은 의도로 그랬다라는 식으로 띄우기 시작하던데.
\vspace{5mm}




\section{환경 얘기}
\href{https://www.kockoc.com/Apoc/514971}{2015.11.28}

\vspace{5mm}

\textbf{상식적인 '환경'에서 자란 사람은}
\textbf{10명 중 2명 꼴도 못 됩니다.}
\vspace{5mm}

그러나 \textbf{저 환경에서 자란다고 다 공부를 잘 하거나 비뚤어지지 않느냐}
환경은 가정환경만 있는 게 아니거든요.
\vspace{5mm}

반대로 \textbf{안 좋은 환경에서도 공부하고 올라가는 친구들이 있습니다.}
전 이 경우만 머리가 좋다고 평가하는데 그렇다고 이 친구들이 잘 나가느냐, 그것도 아닙니다.
\vspace{5mm}

대한민국 국민들은 자유롭고 평등하다라는 당위가 현실을 보장해주는 건 아닙니다.
금수저 은수저 동수저 담론이 놓치는 것은 정말 "상위 1$\%$"라는 건
\textbf{'상식적이고 정상적인 가정에서 인격자 부모 밑에서 경제적 어려움 없이 자랐다'}
는 것으로 정의해야 정확하다는 겁니다.
\vspace{5mm}

유전자 얘기하는 사람들이야말로 전 유사과학자로 봅니다. 그래서 어느 염색체냐고 농거는 순간 말을 못 하죠.
왜냐면 자기들이 검증한 것도 아니거든. 그냥 들은 얘기 썰 푸는 수준이죠, 그게 그 사람들의 수준이죠.
그러나 환경은 나이먹다보면 직접 체험하게 됩니다. 공부 뿐만 아니라 전 분야에서 어떤 환경에서 자랐느냐의 차이가 극명히 드러납니다.
무엇보다도 도대체 환경 차이도 논하지 않는 사람들이 뭔 유전 타령하는지 도무지 이해가 안 가지 말입니다.
눈으로 직접 보고 확인하지 않고 그저 들은 썰대로 읊는 사람들이 어떨지 뻔하기 때문에 성인들이 이러면 차단리스트 감이죠.
\vspace{5mm}

우리의 도전이라는 건 그냥 좋은 환경에 들어가기 위한 시도입니다.
입시도 "최고의 교육환경, 취업도 "최고의 근무환경", 결혼도 "최고의 생활환경"에 들어가려는 것이죠.
그런데 나쁜 환경에서 좋은 환경으로 올라설 때의 차이를 무엇으로 설명하겠는갸.
그 차이에서 '노력', '사교육', '교재' 등 분명히 지적할 수 있는 것을 다 빼고 남은 것이 비로소 '머리'라고 정리하면 되는 것입니다
부유한 집에서 태어나 온갖 교육 다 받고 노력을 습관화시켜 서울대에 갔다고 머리가 좋다고 볼 수는 없지만,
개막장 집안에서 태어나 어디 북한 저리가라하는 동네에서도 교과서만 가지고도 인서울 들어가면 이건 머리가 매우 좋은 것입니다.
\vspace{5mm}

노력 해보았자 소용없다라는 사람들, 한명한명 보면 그 사람들이 흔한 택배 상하차 알바부터
정말 밑바닥까지 고생한 적이 있는 경우는 단 한번도 없었기 때문에 저는 그냥 이 경우는 무시합니다.
정말 고생해 본 사람이거나 진짜 올라간 사람이면 노력은 기본으로 깔고 가는 거예요.
특히 경쟁의 핵심은 자기 경쟁자들을 노력하지 않도록 선동, 도발하는 겁니다. 제가 경쟁한다고 해도 그럴 것입니다. \textbf{실제로 그랬고요.}
\vspace{5mm}

하지만 노력보다 더 중요한 건 \textbf{"환경"의 준비라고} 말씀드리겠습니다.
전생에 나라를 구해서 좋은 집안에 태어나는 것도 환경이지만   본인이 노력하거나 머리를 굴려서 자기가 만드는 것들도 역시 환경입니다.
즉 노력해서 바꿀 수 있는 환경과 없는 환경이 있습니다.   당연 우리의 초점은 노력해서 바꿀 수 있는 환경이 되겠죠.
\vspace{5mm}

어떤 강의냐 교재냐 그런 게 중요한 게 아닙니다.
수능이란 시험 $-$ 즉 제시간에 문제 정확히 풀어내기 위해서 일단 자기가 어떤 환경에서 공부해야하느냐부터 고민하세요.
수능 당일날 집중하려면 체력이 필요하니 꾸준히 유산소운동한다거나  기상시각 조절하는 것.
자기의 경우는 아침에 저혈압 증세가 있으니 무조건 걸어서 도서관에 가서 3시간만 공부하고 집에 가서 낮잠 30분 잔다거나
부모님이 시도 때도 없이 부부싸움하니 어쩔 수 없이 혼자 나가서 살거나 아니면 하루 웬종일 도서관이나 독서실에 간다거나
시각적 사고가 강해 교재 읽는 게 더 잘 되면 교재 읽기로 커버치기,
반면 청각적 사고가 더 강해서 강의 듣는 게 편하다 싶으면 학원을 간다거나 도서관에서 인강을 줄창 듣는다거나.
... 이런 식으로 하나하나 개선해보고 그렇게 가세요. 무조건 생각없이 업자들 손에 놀아나지 마시고요.
도대체 자기 인생 바꾼다라는 사람들이, 절박해 미치겠다는 사람들이, 자기 자신이 어떤지도 모르고 환경 개선에 의지도 없이
안이하게 나중에 수험상품 쇼핑하는 것보면 저 색기는 걍 대학 떨어져도 싸다란 생각이 들 때가 한두번이 아닙니다.
실제로 그렇게 대학에 간다고 한들 요즘 대학도 살벌하기 때문에 못 따라갑니다. 그 이후 취업전쟁은 애당초 불공정을 깔고 들어가거요.
\vspace{5mm}

일지를 쓰던 일기를 쓰건 자기가 '매니저'라면 나라는 답없는 녀석을 어떻게 매니지먼트할 건가 하는 마인드로 가세요.
혼자 관리가 안 되면 경쟁하는 친구와 서로 딜해도 좋고, 그게 안 되면 현강 들으러 가세요.
학원 가는 건 강의력 때문이 아닙니다. 사람은 간사해서 남들의 시선에 놓이면 방만해지거나 딴짓하는 게 현저히 줄어들기 때문입니다.
뭐 머리가 나빠서 공부를 못 한다? ㅋ
CCTV 놓고 찍으면 그거 거의 다 본인이 딴짓하고 게으름피우고 태만한 걸로 귀결됩니다.
자기야 안 그렇다고 하고 싶죠? 그럼 님들이 한번 도서관 가서 다른 학생들 어떻게 공부하나 뒷짐지고 평해보세요.
자기에게는 한없이 관대한 자가 남들 공부는 참 신랄하게 평가합니다(저 역시 마찬가지고요)
혼자서 공부 안 되면 그 환경 떠서 남들과 같이 싱크로하는 게 현재로서는 최적의 답입니다.
합격이고 뭐고 그거 걱정하지 말고 도서관 가서 가장 먼지 앉고 가장 늦게 일어나는, 누가 오래 자리 지키나 하는 것도 좋습니다.
공부가 안 되고 올해도 시험 망했다 하는 사람들,
슬그머니 말 걸어보면 공부 외의 온갖 연예, 문화, 시사 이런 건 '차암' 빠삭합니다.
공부는 하기 싫은데 그런 걸 할 시간은 널렸나보지요.
\vspace{5mm}






\section{A, B, C의 비교}
\href{https://www.kockoc.com/Apoc/515764}{2015.11.28}

\vspace{5mm}

특정인 지적 댓글은 하지 않습니다. 본인들 느끼고 정신차리라고 쓰는 글입니다. 칭찬도 있고 훈계도 있습니다.
\vspace{5mm}

\begin{itemize}
    \item A $-$ 환경을 극복하고 올라서려 한다, 과거에 사로잡혀있다.
    \vspace{5mm}

    공부는 엉덩이로 한다는 걸 보여주고 있죠. 그리고 커리를 제가 잘 압니다. 양치기 기본에다가 효율적 인강으로 단기적 목표 성공
    그런데 가끔 얘기하면 과거에 많이 집착합니다. 당연히 그건 아무 도움이 안 되지요.
    안 좋은 환경에서 스스로 공부환경을 쟁취하고 이겨낼 테니까 슬럼프 몇번 겪겠지만 극복하겠죠
    \vspace{5mm}
    
    \item B $-$ A와 비슷한데 자신감이 떨어진 케이스
    \vspace{5mm}
    
    잘못된 정보에 사로잡혔다가 어느 정도 올해 극복했습니다. 흐름을 탈 줄 아니 오버하지 않으면 성장하겠죠.
    A와 차이가 있다면 이 경우는 그냥 자신감 문제입니다. 자신감은 스스로 성과를 거두고 대화하면서 쟁취할 문제죠.
    다만 대화할 상대들의 수준을 높이는 게 필요할 겁니다.
    \vspace{5mm}
    
    \item C $-$ 머리는 좋은데 잘못된 습관
    \vspace{5mm}
    
    머리는 기가 막히게 좋습니다. 그러나 두뇌가 습관을 못 이긴다라는 걸 보여주고 있죠.
    당장 해야할 일을 늦게야 하고 후회하는 경향이 있습니다만, A, B보다 절박감이 매우 덜 합니다.
    지금도 안이해졌는데 자칫하면 내년에도 올해 전철 또 밟는 거죠.
    \vspace{5mm}
    
\end{itemize}
대충 본인들은 이걸 알고 쪽지로 테러할 건데 뭐 겪는 일이니까.
차이가 있다면 A는 제가 하는 말은 그대로 잘 들었고, B는 남에게 한 말까지 캡처해서 다 들었고(이건 소름끼칠 정도였음)
C는 제가 가장 많이 말해주었는데 가장 많이 안 들었다는 겁니다.
제 말이 무조건 옳다는 건 아니지만 하라는대로 했는데 제대로 안 하고 고집피우다가 나중에 제 탓 하면 심히 골룸하다고 지적드리겠습니다.
\vspace{5mm}

+
\vspace{5mm}

그리고 또 첩자들과 알바들 들어온 티 나던데
그 자들이 아니면 너무 관심이 많은 댓글을 쓰시더군요.
\vspace{5mm}

뭐가 무서워서 여기 와서 첩자질하는 건 모르겠는데 그러면 그 사람들 다 저주받아 죽으라고 덕담 기도해주고(이건 관계없잖아)
그리고 일지 꾸준히 써서 인증한 사람 빼고는 역시 대접할 필요 없다는 쪽으로 가닥잡으면 됩니다.
반면 눈치코치 모르고 활동하는 사람은 관심 끊습니다.
\vspace{5mm}

이중아이디로 활동하다가 걸린 사람들이야 많죠. 관상에도 티가 나죠.
개버릇은 못 준다는 이야기 사실인 듯.
\vspace{5mm}




\section{빚개념에 대해서}
\href{https://www.kockoc.com/Apoc/518747}{2015.11.30}

\vspace{5mm}

5수생이 있다칩시다.
\vspace{5mm}

보통 이런 경우 어떤 관념이 문제나면
자기가 날려먹은(?) 4년만큼의 본전을 챙겨야한다는 \textbf{보상심리} 라는 게 있습니다.
그래서 목표치를 더 높게 잡으면서 자기가 수험고수이니 더 많이 하겠다 그래서 꼭 성공해야한다는 강박관념이 있죠.
\vspace{5mm}

사실 생각해보면 별 의미없는 자기학대에 불과합니다.
목표치를 높게 잡는다고 해보았자 그 4년이 빛나는 것도 아니죠.
4년동안 공부했다면 당연히 구력은 있습니다, 그러나 '실패'도 학습되었을 뿐더러 '해결되지 않은 원인'이 있단 거죠.
\vspace{5mm}

그럼 어떻게 해야하느냐.
\vspace{5mm}

일단 4년은 잊어버려야합니다. 그냥 4년동안 병원생활, 식물인간, 징역살이, 외계인에게 납치... 당했다고 생각하는 편이 나아요.
그 4년은 경제학적으로는 매몰비용입니다. 뭘 하더라도 사실 복구는 못 해요. 심지어 성공한다 하더라도 4년이 의미있느냐 그건 아닙니다.
다들 이런 매몰비용을 복구하겠다고 목표를 무리하게 잡는 걸 넘어 학습방법도 터무니없는 걸 선택하기 때문에 실패하는 겁니다.
\vspace{5mm}

저 4년은 안 돌아옵니다.
내년에 시험치는 사람이면 겸손하게 자기가 고3과 똑같다고 여기세요.
\vspace{5mm}

만약 개인의 성찰과 반성, 그리고 기본 지식을 쌓는 과정에서라면 유의미하다고 반문할 수 있긴 하겠죠.
그러나 이 경우 손해는 더 큽니다. 4년동안 해서도 되지 않는 \textbf{실패도 따라오기 때문}입니다.
시험을 여러번 쳐도 안 되는 이유는 공부가 부족한 것도 있지만, \textbf{실패하는 패턴을 반복하는 게 가장 큽니다}.
학원에서는 공부하는 방법이나 지식을 전수해주겠지만, \textbf{학생 개개인의 실패 패턴을 지적해주거나 잡아주진 못 합니다.}
본인의 과제죠.
\vspace{5mm}

하지만 이걸 하는 건 자존심을 포기하는 것에 근사하기 때문에 혼자 하기 힘들 수도 있습니다.
달리기를 잘 하는 친구에게 너는 달릴수록 불행해지니까 달리지 마라고 하거나
아주 얼굴이 예쁜 여학생에게 자네는 얼굴이 불행의 근원이니 차도르를 쓰고 알라후 아크바르를 외치도록 하는 것과 동급입니다.
하지만 그런 자존심을 포기하고 여태껏 살아온 방식을 과감히 바꾸지 않으면 실패는 또 반복되죠.
\vspace{5mm}

빚을 못 갚으면 파산신청하고 갱생하는 게 낫습니다.
내년에 다시 시험 응시할 분은 과거는 싹 잊으세요. 과거에서 챙길 건 오직 교훈, 그리고 자기의 실패하는 패턴에 대한 반성입니다.
그래서 다시 시작해야 하는 겁니다.
\vspace{5mm}

과감하게 구식무기를 버리고 신식으로 갈아타면서 자기를 잊는 사람은 살아남겠지만
계속 한탄만 하면서 자기를 너무 사랑하는 사람은 또 실패합니다요.
\vspace{5mm}








\section{산업공학에 관하여(초안)}
\href{https://www.kockoc.com/Apoc/519751}{2015.11.30}

\vspace{5mm}

일명 Industrial Engineering :
\vspace{5mm}

그 시조는 Fredreick Winslow Taylor
https://en.wikipedia.org/wiki/Frederick$\_$Winslow$\_$Taylor
경영학의 아버지.
\vspace{5mm}

경영학과 같은 계보. 산업공학이라기보다는 공업경영이란 말이 맞음.
공업에 더 특화된 경영학이라고 볼 수도 있지만, 실제로는 "시스템공학"이라는 말이 적절.
\vspace{5mm}

그러나 실체는 어디든 마찬가자이지만 희대의 '박쥐학과'
일단 좋다는 것에는 다 발 걸치고 있음. 그래서 학부과정에서 돈과 무관한 건 안 배운다.
우선 수학도 공수는 빼고 선형대수학과 통계학으로 쇼부.
나머지 과목들은 경영학과와 공학에서 쓸만한 것만 가져와 '뿌리없이' 배움
대신 수행할 과제는 컨설팅업체의 결과물 비스무리하게 내놓아야함.
\vspace{5mm}

한마디로 이것저것 다 건드려보는데 깊이 다루지 못 함.
그러나 이것이 장점이 될 수 있는 것은 학부과정에서는 어차피 전문성은 크게 기할 수 없음.
학부 과정에서는 시야가 넓어야 함. 그 점에서는 이 박쥐학과적 안목이 도움이 된다는 것.
그래서 막말로 진출분야는 거지와 국회의원 빼고는 다 가능하다는 말이 있음.
\vspace{5mm}

뿌리는 없지만 꿀이라고 하는 이유는 취업부터 대학원인데 쓸만하다고 하는 알짜 분야의 대학원으로 빠져나갈 수 있음.
수학, 컴퓨터, 경영과학, 생산관리, 품질관리 금융공학, 제조공학, 통계분석, 시뮬레이션 기타 등.
사실 대학교에 들어가서 배우는 공학은 "관리자"의 안목을 요구한다는 점에서 이건 정말 알짜들만 제대로 모아놓은 것임.
\vspace{5mm}

교수에게 뭘 배워야하느냐 물어보았을 때 나온 3가지 답변은
산업공학의 3대 학문 : \textbf{수학, 컴퓨터, 통계학}
\vspace{5mm}




\section{20대를 넘기면 부모가 아니다.}
\href{https://www.kockoc.com/Apoc/521648}{2015.12.01}

\vspace{5mm}

어디서든 상담을 하다보면 가정문제가 많다.
가정이 화목한 사람은 돈이 없더라도 행복한 줄 아시길.
다시 말해 하하호호 행복한 케이스는 정말 찾기가 어렵다.
\vspace{5mm}

대가족 체제에서는 권위있는 할아버지, 할머니들이 있으니 부모들도 윗사람 눈치보는 게 있었다.
그리고 가부장 시절에는 남편은 돈을, 마누라는 밥을 주기 때문에 역시 견제기능이 발휘되었다.
그러나 지금처럼 개개인이 돈도 벌고 밥도 다 하는 시대에는 눈치볼 것조차 없다.
사소한 갈등으로도 공격성을 내비친다.
\vspace{5mm}

3, 40대가 과연 성숙하다고 할 수 있는가.... 그건 나조차도 회의적이다.
나이와 인격은 절대 비례하지 않는다. 얼마만큼 배우고 경험했느냐에 달렸을 뿐.
그래서 이야기들 들어보면 \textbf{부모가 부모 역할을 제대로 못 하는 케이스도 정말 많다}.
부모에게 기눌린 케이스, 부모에게 기대하지만 좌절하는 케이스.
\vspace{5mm}

내가 제안하는 건 거리두기이다. 그리고 본인이 성인이 되었으면 부모도 과거의 부모로 보지 말길.
키워주고 교육시킨 것에 대해선 빚이라는 건 있다. 그러나 그 이상의 것을 부모에게 기대하진 말라는 이야기.
부모가 하라는 대로 할 필요도 없다, 어차피 그 세대 어른들이 세상 돌아가는 걸 잘 아는 것도 아니고 책임질 수도 없기 때문.
그러나 부모에게 많은 걸 기대도 하지 말란 이야기다. 자기 인생은 자기가 살아야하는 것이다.
부모님이 폭언을 퍼붓거나 해도 그냥 이건 '개무시'하는 게 답이라고 본다.
\vspace{5mm}

가능하면 부모님에게 도움을 받더라도 부모님과 떨어져있는 것, 대면시간을 줄이는 것이 현명한 답이라고 생각된다.
상담해보면서 느끼는 대부분의 문제가 '대면시간이 기니까 의견이 충돌하는 것'으로 정리되는데
인간관계 잘 유지하는 비법은 '거리를 적당히 유지'하는 것이다. 아주 가까워지지도 않고 아주 멀어지지도 않고.
부모자식이니까 가까워야한다고 보기 좋지만 글쎄, 원래 자식이 성년이 되면 부모와 충돌할 수 밖에 없는 게 정답이 아닌가.
\vspace{5mm}

이거 생각보다 많은 사람들이 겪는 문제다, 다 자기들만 겪는다라고 착각하겠지.
\vspace{5mm}






\section{환경 넘어서기}
\href{https://www.kockoc.com/Apoc/521737}{2015.12.01}

\vspace{5mm}

자기 처지를 불행히 여기는 사람들도 많음.
꽤 많이 공감해주었지만 한편으로는 나는 꾸짖을 때는 꾸짖는다.
\vspace{5mm}

불행한 환경에 처한 사람들이 \textbf{그 불행함 자체를 핑계로 대는 경우도 많기 때문이다.}
자기가 판자촌에 살았다고 공부 안 하고 유흥 즐기는 게 정당화될 수는 없다.
그건 인육까지 먹으며 엽기살인 일삼은 지존파나 유영철이 동정받아야한다는 논리와 똑같기 때문이다.
그런데 불행한 환경에서 자라난 사람들의 문제는 그걸 극복해야하기보다는 핑계로 쓴다는 것이다.
\vspace{5mm}

한번 안 되는 사람이 계속 안 되는 이유다.
좋은 집안에서 교육 잘 받은 친구들이 잘 나가는 이유 중 하나는 핑계를 댈 수가 없기 때문이다.
그 정도까지 부모가 잘 해주었는데 네가 못 하면 누구 책임인 줄 알지... 에 답변을 못 하니까 열심히 한다.
그러나 안 좋은 환경에서 자라는 경우 실패를 전가할 대상은 많다. 그러나 전가해보았자 피해는 누가 입을까.
\vspace{5mm}

속칭 이걸 \textbf{피해의식}이라고 한다.
그리고 사실 사회적으로 피해의식이 있는 사람과 계약질을 하는 건 피하는 게 좋다.
이들은 그 피해의식만으로 자기들이 계약불이행이나 불법행위로 가기 딱 좋은 사람들이어서이다.
그렇기 때문에 이런 사람들일수록 더욱 엄하게 대해야 한다.
\vspace{5mm}

환경 탓만 하다간 아무 것도 못 바꾼다.
바람피우고 술마시는 아버지 밑에서 자란 아들이 부전자전이 되는 이유?
그 아버지 탓만 하는 가운데에서 그 아버지를 배워버리기 때문이다.
진정 극복하려면 그런 아버지를 혐오하고 우습게 보면서 나는 저러지 말아야지라는 경멸을 했어야 한다.
하지만 대부분은 자기 아버지가 어쩌구... 하면서 그걸로 환경 탓만 하면서 벗어날 노력을 안 한다. 그리고 자기도 똑같은 길을 겪는다.
\vspace{5mm}

아래 부모글과 연관짓는다면 바람직하지 않은 부모를 넘어서지 못 하고 계속 억압만 당하면
역설적으로 그 부모가 선망 대상의 위치에 놓이고 자기도 모르는 사이에 그 부모를 배워버리고 만다.
넘어선다는 건 2차원에서 놀다가 3차원으로 시야가 넓어지는 걸 말한다.
자기를 학대하거나 못 되게 구는 사람을 마치 체스의 말처럼 바라보면서 왜 그렇게 행동하고 어떻게 하나 읽어야 한다.
그렇지 않고 두려워하다간 똑같아져버린다.
\vspace{5mm}

넘어선다는 건 여러모로 쉬운 일은 아니다. 하지만 이것 빼고 길이 있을까?
실패한 것을 만회하려면 그 실패를 반드시 객관화하고 사소한 걸로 여길 수 있어야 한다.
바이러스에 감염당한 좀비를 불태워버리듯 과거의 자기를 서슴없이 버려야 한다.
\vspace{5mm}

적어도 내가 관찰하고 느낀 바 $-$ 그리고 수험상담을 해주는 것도 나 나름대로의 수행이지만 $-$ 는 그렇다.
부모가 학대한다면 가장 편한 물리적 방법은 대면시간을 줄이고 공간도 따로 하는 것이다.
그러나 가장 중요한 건 \textbf{부모에게 감염된 나를 어떻게 처리해버리느냐}이다.
폭력이 무서운 건 고통과 상해 때문만 아니다. 로봇을 때린다고 해도 고치면 그만이다.
하지만 인간은 얻어맞는 걸로 그치는 게 아니라 정신까지 안 좋은 방향으로 감염되어버린다.
그래서 같은 행동을 반복하는 경향을 보여준다. 마법사들은 이걸 읽고 그 사람들을 세뇌시킬 수 있다.
\vspace{5mm}

상처에 공감해준다는 것만이 절대 능사는 아니다.
상처입었다는 사람들도 곰곰히 보면 그 상처를 이용하는 비겁한 모습이 없지는 않기 때문이다.
\vspace{5mm}





\section{3s}
\href{https://www.kockoc.com/Apoc/528601}{2015.12.05}

\vspace{5mm}

사고의 순서를 지키지 않아서 머리가 나쁘다고 착각하는 경우가 그냥 99$\%$라고 생각하면 된다.
\vspace{5mm}

머리가 나쁘다고 하는 사람들조차도 사실 머리가 나쁘다라고 하는 '명제'를 생각없이 자신과 타인에게 강요하는 경우고
보통 그런 사람들과 대화해보면서 분석해보면 알고리즘이 엉켜있는 경우가 태반이다.
스피드를 포기하고 순서대로만 사고, 행동하면 해결될 것을 엉뚱한 데 답 찾는 경우가 많다.
\vspace{5mm}

1년간 분석해보면서 느낀 결론이란 게 참 허망.
\vspace{5mm}

\begin{itemize}
    \item[$-$] system
    \item[$-$] sequence
    \item[$-$] speed
\end{itemize}

\vspace{5mm}

이 3s로 그냥 결정나는 듯.
사람과 대화하다보면 어느 사이에 그 사람을 추상명제화시켜서 분석가능하게 되고
그 명제들끼리 서로 논증되면서 추론되는 결말이라는 것이 바로 운명이라는 게 아닌가하는 중2병급 망상도 들긴 하지만
공부 안 된다고 하는 경우는 잘못된 시스템에 사로잡혀서 그렇고
문제풀이가 안 된다는 경우는 시퀀스 통제력이. 그냥 순서대로 사고하면 되는 데 그걸 못 해서
무조건 추종할 수도 버릴 수도 없는 게 스피드. 시스템과 시퀀스가 잡힌 다음에 스피드를 높여야.
스피드를 추종하면 시스템과 시퀀스가 망가진다.
\vspace{5mm}









\section{메모 : 칭찬}
\href{https://www.kockoc.com/Apoc/528801}{2015.12.05}

\vspace{5mm}
\begin{itemize}

    \item 가장 아름다운 미인 : 칭찬
    \vspace{5mm}

    남녀외모야 30대 넘기면 어차피 다 아재 아짐 되니까 그건 알 바 아니고.
    사람들이 정말 고파하는 건 \textbf{'칭찬'}이라는 사실
    엘리자베스 테일러 리즈 시절의 외모라도 \textbf{진정성이 담보된 칭찬보다 예쁘지 않음.}
    \vspace{5mm}

    만약 모든 사람이 서로를 칭찬해주고 다닌다면 칭찬이 먹히진 않을 것이다.
    하지만 칭찬이 잘 먹힌다는 건, 사람들이 서로 차갑고 혹독하단 이야기이겠지. 늘 부정적인 평가에 스트레스 받으며 살아가는 것이려니.
    나폴레옹조차도 "우리 황제폐하는 칭찬하는 아첨배를 좋아하지 않습니다"란 칭찬에 흡족해하지 않았다고 하던가.
    \vspace{5mm}

    \item 인간관계, 저주풀기
    \vspace{5mm}

    바른 소리를 하면 인정받겠지라고만 생각하지만 실제로 부딪혀보면 다를 것이다.
    아첨하지 않고 정말로 칭찬할 수 있는 것만 골라서 칭찬하는 것도 하나의 실력이라는 걸 느끼게 되는데
    자기가 하는 일이 실속이 없더라도 은연 중에 상대방을 '무의식적'으로 칭찬할 수 있는 거라면 그 관계는 꽤 오래간다.
    \vspace{5mm}

    칭찬이 지나치면 아첨이 되는 건 어쩔 수 없긴 하지만
    그게 상대방을 옭아매고 있는 저주를 푸는 것이라면(판타지스럽긴 하지만 이건 실제로 그렇다) 권장할만하다.
    예컨대 본인이 아름다운데도 못 생겼다고 착각하는 케이스들이 생각보다 많다. 왜? 주변 사람들이 그렇게 험담하고 깎아내려서.
    이 경우 예쁘다라는 말을 10번은 해줘야 하지만 다른 사람들의 동의도 필요하다. 그래야만 그 저주가 풀린다는 황당한 일이.
    \vspace{5mm}

    \item 트라우마
    \vspace{5mm}

    어린 시절에 윗사람에게 받은 상처가 자아의 핵심인 경우가 많다. 생각없이 던진 걸로 보인 '폭언'을 수십년 품고 산다든가.
    특정인의 언행을 분석해보고 상호피드백을 하면 남들과 다른 '아픈 개성'이라는 게 있는데
    그 아픈 개성은 어린 시절로 추적해 가다보면 보통 상처와 연관되어있다.
    \vspace{5mm}

    미국에서는 성폭행을 당한 여자들이 비만일 가능성이 높다는 연구결과가 있는데
    해석하기로는 그렇게 해서 성적매력을 줄여서 도피하고자하는 것이었다든가.
    꼭 이 경우로 해석할 필요는 없겟지만, 아무튼 이해가 안 갈 수도 있는 "자기파괴적인 행위"는 도피를 의도한 경우가 많다.
    도피를 해야만 괴로웠던 과거나 현재에서 유리될 수 있기 때문이 아닐까 싶다.
    \vspace{5mm}

    \item 미래는 불확정적 명제.
    \vspace{5mm}

    과거든 현재든 미래든 사실 명제일 뿐.
    그나마 정확하다고 말할 수 있지만 사실 매우 부정확.
    운명이 있다라고 얘기하는 근거 : 개인은 자기 미래를 명제화시켜 거기에 맞춰 행동하기 때문.
    사실 인과관계가 딱히 있다고 볼 수는 없는데 인과관계가 있다고 믿어버림. 일단 믿기 시작한 순간 절대불변의 명제가 되어버림.
    \vspace{5mm}

    시간이 뭐냐하면 보통 시계를 가리키겠지만 그것도 인위적으로 정한 하나의 계량수단.
    그리고 심리적 시간이라고 하면 어떤 사람은 1초가 1시간처럼, 1시간이 1초처럼 실제로 와닿기 때문에 이것도 딱히.
    물리적 시간은 객관적이지 않냐하지만 우리가 사는 세상은 물리적 세상이 전부가 아니기 때문.
    \vspace{5mm}

    \item 행복은?
    \vspace{5mm}

    생각이 정지될 때.
    논리는 거짓말아니고 '현자타임'을 유발, 자기가 몰입하는 상황을 칼로 자르고 해부하는 것이 "의문"이고
    그 의문문을 던진다는 건, 주어진 명제나 상황을 '부정'해본다는 이야기. 그 때는 절대 행복을 느낄 수 없다.
    \vspace{5mm}

    칭찬의 기능은 저런 '의문문'을 차단해준다는 것임. 그럼 칭찬이 사람을 바보로 만드는 게 아닙니까.
    개인이 품는 의심 중에선 별 소득도 없는 쓸데없는 의심들도 많기 때문.
    나 못 생겼죠, 머리 나쁘죠, 이렇게 살다간 이번 생 답 없죠.... 등은 하나마나한 의심. 이런 건 보통 저주받은 결과임
    그 저주를 풀어주는 칭찬은 적절한 것임.
    \vspace{5mm}
\end{itemize}









\section{글을 읽을 때 3가지 문장과 3가지 틀}
\href{https://www.kockoc.com/Apoc/533693}{2015.12.10}

\vspace{5mm}

3가지 문장
\vspace{5mm}

\begin{itemize}
    \item[] ⓐ 기술
    \item[] ⓑ 설명
    \item[] ⓒ 주장
\end{itemize}
\vspace{5mm}

기술은 저자를 배제해도 상관없는 있는 그대로의 진술.
\vspace{5mm}

설명은 특정한 대상이나 현상을 이치에 맞춰 논리적으로 풀어쓴 것.
어떻게 보면 기술이 아니냐고 할 수도 있지만 그게 아님.
설명은 기술이 글쓴이를 통해 해석되어 \textbf{재기술}된 것임.
그래서 '의도'와 '관점'이 있음.
\vspace{5mm}

주장은 말 그대로 글쓴이가 하고싶은 이야기를 위 기술과 설명에다가 연역추론 혹은 귀납추론으로 논증하는 것.
물론 현실에서는 논증 없는 주장도 많음. 그러나 수능 지문에 등장하는 주장은 거의 다 논증임.
의도, 관점, 강조, 한계는 당연하지만 '토론', '비판'도 덩달아 따라나옴.
\vspace{5mm}

글을 읽을 때 \textbf{기술, 설명, 주장}을 나눠서 구분하는 것이 체화되어 있어야함.
기술은 상대가 아무리 개새끼더라도 받아들여도 되는 것
설명은 일단 수용은 하되 내가 다시 한번 검증해보아야하는 것
주장은 상대가 여친 혹은 엄마더라도 일단 '아니오'라고 부정해봐야하는 것.
\vspace{5mm}

그 다음으로 글을 읽을 때 중요한 3가지 틀
\vspace{5mm}

소재 $-$ 화제 $-$ 주제
\vspace{5mm}
\begin{itemize}
    \item[] ⓓ 소재 : 글의 대상.
    \item[] ⓔ 화제 : 글쓴이가 소재를 가공한 것 $-$ 설명문은 여기서 그치는 경우가 많음.
    \item[] ⓕ 주제 : 글쓴이가 화제로서 말하고자 하는 것 $-$ 설명문이 아닌 글은 이게 부각됨.
\end{itemize}
\vspace{5mm}

글을 읽을 때는 반드시 저런 소재$-$화제$-$주제를 묻고 답해야 함.
\vspace{5mm}

...............
\vspace{5mm}

국어 독해의 시작과 끝은 저걸로.
상세한 내용은 공개된 것으로 적기도 그렇지만 어차피 공부할 인간들은 저걸 체화시켜 알아서 끝낼 것임.
저 틀도 못 잡고 문제푼다면 점수가 올라갈 리도 없겠지만 앞으로 인생살이도 힘들 것임.
\vspace{5mm}

과거보다 국어교재는 확실히 퇴화
90년대에서 2000년대 초반까지 국어교재가 현재보다 10배는 더 나았다는 생각.
수학은 야매방향으로 강조되는 기이한 풍토기도 하지만(그렇게 공부한 사람들이 자연대, 공대 가서 잘 할지도 의문)
사실 심각해진 것은 국어. 인터넷 이전의 미개한(?) 시대에도 기본적으로 교육되던 것들조차 찾기 어려워짐.
거기다가 독서하는 학생들을 보기 힘듬.
\vspace{5mm}

글을 읽을 줄 아는 사람이 생각보다 적음.
윗 방법론도 그렇지만 기본교양조차 부실한 학생들이 많음.
어디선가 읽은 게 16 수능 A형은 특정 지식에 치중해있어서 불공정한 출제다...
옛 사람들이 보면 웃을 일임. 그 정도는 우등생이라면 기본적으로 익힌 교양수준이었다고 얘기했을 것임.
\vspace{5mm}

딴 이야기하면 과목 중요도는 영어 > 국어 >>>>>>>>>>>>>>> 수학
\vspace{5mm}

업자들이야 화날 이야기겠지만 사실 지금 입시수학이 정말 제대로 된 것인가...
본인이 수학 관련 학과에 진학해서 그걸로 호구들 낚아 장사하고 싶다면 입시수학의 중요성을 강조할지도 모르지만
실제로는 공대조차도 수학은 별로 필요없음. 공업수학조차도 그냥 주어진 텍스트 암기도 벅찬 판인데 뭘.
고교수학에서 배워야하는 건 근대적인 사상, 그리고 '문제풀이의 자세' 그 이상 그 이하도 아님.
\vspace{5mm}

가장 중요한 건 영어. 왜냐면 영어텍스트는 항상 신선한 충격을 주기 때문.
영어(와 일본어)로 쓰여진 책만 읽고 살아도 미개해지는 건 막을 수 있고 세상 돌아가는 걸 남보다 앞서서 판단할 수 있음.
어차피 대학가면 다 영어 원서 본다는 거야 알고 있음. 공학 같은 경우도 교수 강의는 버려도 되지만 원서는 버리면 안 됨.
\vspace{5mm}

영어를 잘 하다보면 국어도 덩달아 잘함. 어차피 국어의 읽기 기술이라는 것도 영어에서 가져온 것 아닌가?
그보다도 국어를 열심히 하면 '사기' 당할 확률이 줄어듬.
위의 6가지 틀을 응용하면 상대가 말하는 걸 기술, 설명, 주장으로 분류해서 기술만 수용하고 설명은 반수용하고 주장은 씹으면 끝남.
그리고 상대가 말하고자 하는 주제를 역추적해서 화제, 소재로 분해하면 끝나는 일임. 의도파악이라는 건 이걸로 정리됨.
\vspace{5mm}






\section{자아 집착}
\href{https://www.kockoc.com/Apoc/536401}{2015.12.09}

\vspace{5mm}

무사가문 출신의 료마는 20대 초반까지 검술훈련을 받았다.
어느날 료마가 동향 무사인 하가키 세이지라는 엣 동료를 만났는데
그는 무사들이 보통 많이 갖고 다니는 장검을 차고 있었다.
료마는 그에게 실전에서는 단검이 다루기가 좋다고 하였다.
그렇겠다고 생각한 친구는 단검을 갖고 다녔다.
단검을 찬 그를 보자 이번에는 료마가 품 속에서 총을 꺼내며 총 앞에서 칼은 아무 소용이 없다고 말하였다.
그 총이 미국회사 스미스$\&$웨슨(Smith$\&$Wessen)이 만든 것이었다.
이 말을 듣고 친구도 바로 총을 샀다.
세번째 만나 친구의 총을 본 료마는 이제부터는 세계를 알아야한다며 서양의 만국공법(국제법)책을 보여주었다고 한다.
\vspace{5mm}

실제로 구글에서 검색하면
기쿠수이(菊水)에서 발매한 료마 소주가 있다.
칼 레벨 $-$ Smith $\&$ Wseen 레벨 $-$ 만국공법 레벨이라는 참 해괴하지만 의미심장한 라벨링이 되어있다.
\vspace{5mm}

일본이 운만 좋아서 조선을 집어삼킨 게 아니다.
료마가 사망한 연도가 1867년.
19세기 중반에 조선의 집권자들은 도저히 바꿀 수 없는 유교 체제에서 허우적대고 있엇지만
일본의 엘리트들은 이기는 방법을 찾기 위해 과감히 과거의 것을 쳐냈다.
\vspace{5mm}

승부에서 이기려면 과감히 바꿔야 한다.
시험에 장애가 되는 건 운도 출제 경향도 제도의 불합리도 아니다.
가만히 생각해보면 그 승부에서 이기기 위해서 정말 나를 버리면서 과감히 베팅했나... 하며 그것도 아니란 이야기다.
칼을 좋아하며 냉병기만이 진정한 무기라고 생각하는 사람이 총탄이 오가는 전쟁에 나가면 어찌되겠나.
사서삼경만 외운 선비가 오랑캐 학문을 배울 수 없으니 수능에서 수학, 영어를 빼달라고 하면 용납되겠는가.
\vspace{5mm}

"나 자신"을 사랑하는 건 어쩔 수 없다. 그러나 자아는 바로 '임종 직전에야 완성될지도 모르는' 미완성 그 자체다.
하지만 대화나 상담을 해보면 \textbf{각자가 지금의 모습을 너무나도 사랑하는 것을} 느끼곤 한다.
다들 유치원 다닐 시절의 꿈을 지금에 품고 있나? 사달라고 했던 장난감이나 인형을 지금도 고집하나?
자기가 \textbf{바뀐다}는 것을 다들 인정하려는 경향이 없다.
그러나 모든 것은 분명 변화한다. 자기가 \textbf{바뀌느냐}, 아니면 스스로 \textbf{바꾸느냐} 그 차이일 뿐이다.
승리를 생각하면 과거고 나발이고 과감히 포기하고 버려야 한다.
\vspace{5mm}

가장 방해가 되는 것은 그래도 과거에 많이 노력한 것들이 언젠가는 도움이 될 것이다... 라는 망상인데
패러다임은 연속적인 게 아니라 불연속적인 것이다.
구시대적 체제가 발전해서 진화하는 게 아니라 그것들이 모순이 누적되고 폭발하고 붕괴해야 새로운 패러다임이 등장하는 것이다.
만약 경쟁자가 없다면 아무렴 상관없을지도 모른다. 그러나 경쟁자가 있으면 \textbf{먼저 바꾼 사람이 이긴다}.
\vspace{5mm}

10년 뒤에도 현재의 마음을 유지하고 싶은가.
다들 그렇다고 생각하겠지만 10년 뒤에 우리는 절대 지금의 우리가 아니다.
미래에는 $\sim$ 할 거야라가는 포부에 대해서 난 99$\%$는 그냥 무시한다.
그 마음이 과연 10년, 아니 5년 뒤에도 유지될까?
그나마 스스로 바꾸면 성장할 수 있다.
하지만 스스로 바꾸지 않으면 안 좋은 방향, 퇴화하는 방향으로 \textbf{바뀔} 가능성이 상당히 높다.
안 좋게 바뀐 다음 얻어터진 다음에야 나도 한 때 잘 나갔는데 왜 이렇게 살까 하다가 세상 탓을 또 하겠지.
\vspace{5mm}

구체적으로 말하며 20대의 모습이란 거의 다 허상이다, 그건 30대도 40대도 마찬가지겠지.
죽기 직전이야말로 참자아. 그런데 이게 좋다는 보장은 없다.
\vspace{5mm}

각설하고 저 에피소드는 우리 한국인에게만 비극으로 다가왔다는 후일담.
실제로 일본의 엘리트들은 만국공법(=국제법)을 잘 활용해 조선을 참 평화롭게 먹어치웠기 때문이다.
강화도 조약부터 시작해 청, 러시아와 전쟁 벌일 때 맺은 조약들로 무난히 먹어치우기 위한 준비를 다 해버렸다는 것.
기득권 지키기에 여념없던 유생들이야 뒤늦게 국제법을 알아보면서
혹시 미국이 도와주지 않을까 부질없는 희망을 품었지만 아무 소용이 없었다.
\vspace{5mm}

보통 을사조약을 을사오적 매국노 이토 히로부미 개객기로만 접근하지만
그냥 먹어치워도 되는 것을 왜 굳이 '외교권을 박탈'하는 속국화로 나갔을까 생각해보는 것이 더 중요하다.
일본 애들은 시모노세키 조약에서 조선이 자주국이라고 하여 수천년간 중국의 속국이었던 조선을 독립시킨다(...)
당시 일본의 속내를 모르는 조선의 엘리트야 일본이 진정한 은인 T$\_$T 그랬지만 그럴 리가 있나.
일본은 당시 미국, 러시아, 중국, 영국, 프랑스, 독일 눈치를 보면서 '합법적'으로 조선을 합병하기 위한 과정을 참 천천히 \textbf{밟아왔던 것}이다.
아무리 국력이 개차반이어도 어째서 허무하게 먹혔을까라는 걸 우리가 되물어보아야 하는 과정을 보면
참 일본은 치밀하게 수십년간 준비해서 그 보답을 받은 것이다.
그리고 역설적이지만 그런 치밀함이 군부의 폭주로 날라가면서 나중에 다 토해내고 일시적으로나마 망했던 것.
\vspace{5mm}

여기 오는 수험생들은 두가지다.
부모가 참 공부를 잘 해서 자녀에게 그 교육모델을 세습시킬 수 있는 경우와 그렇지 못 한 경우.
교육모델이 세습된 녀석들은 순둥이에다가 뭔 이런 어리버리가 있냐 해도 성적은 잘 나온다.
자기들만의 검증된 모형으로 성과를 거둘 수 있고 그건 이미 나이먹은 부모가 검증한 경우여서이다.
\vspace{5mm}

하지만 그렇지 못 한 경우라면 정말 과감히 '나'를 버려야 한다.
자기 삶의 방식이 부정적인 결과이면 편집증적으로 개선하거나 아니면 정반대 방향으로 가야한다.
\vspace{5mm}






\section{평가라는 건 주체에 따라 달라진다}
\href{https://www.kockoc.com/Apoc/538088}{2015.12.10}

\vspace{5mm}

A가 잘 나갈 때 A를 칭찬하는 병신보단
A가 망하더라도 A의 가능성을 읽고 격려해주는 사람이 더 현명한 것이 아닌가.
\vspace{5mm}

잘 나갈 때에 칭찬해주는 건 사실 개나소나 할 수 있다.
왜냐면 눈에 보이는 현상만 가지고 그냥 적당히 아부만 하면 되기 때문이다.
그리고 그런 아부에 중독된 A가 어떻게 될지야 우리는 너무나도 잘 알고 있다.
반면 A가 해변에서 파도만 보면서 나도 한 때에는 잘 나갔는데$\sim$ 한탄하고 있을 때
앞으로 크게 되겠군... 지금이 기회다라고 격려해주는 건 그냥 빈말할 게 아니라 정말 '근거'가 필요하다.
눈에 보이는 현상이 아니라 그 현상의 이면을 보고 미래의 가능성을 읽어내는 작업이야말로 논리적이어야 한다.
\vspace{5mm}

타임머신을 타고 과거로 돌아가면 똥밭 시절의 강남을 빚내서라도 모두 매입했을 것이라거나
1990년대로 돌아가면 SK나 삼전 주식을 샀을 것이다라는 탄식은 흔하게 듣는다.
그러나 이건 아무 의미 없는 이야기가 아닌가. 그냥 지금 하면 되는 거지.
지금도 저평가된 것과 고평가된 것들이 널려있다. 그걸 정확히 꿰뚫어보는 사람이 자본주의 사회에서 승자가 된다.
하지만 실제로 이걸 할 수 있는 사람은 별로 없다.
고작 머리 쓴다는 게 "그럼 자금 흐름이 있는 곳으로 가면 되겠군. 부자들을 따라가면 되니까"라는 것이겠지만
이걸 역이용한 게 바로 작전세력들이 아닌가.
\vspace{5mm}

이건 수험도 마찬가지이다. 입시에 한번 실패하고 나면 자기를 '쓰레기'로 취급한다.
쓰레기로 취급당하는 사람은 그 다음부터 막 나가게 된다.
반면 운이 좋아서 찍은 게 다 맞아 합격한 사람은 정말 자기가 선택받은 사람이라고 착각한다.
이 역시 막 나가게 된다는 건 마찬가지이다.
\vspace{5mm}

올해 수능에서 실패했다고 치자. 그런데 우리가 1년살이 운명인가?
10년의 스케일을 감안해보자
B는 올해 수능에 승리해서 대학에 간다. 그런데 적성에 안 맞아서 학점이 안 나온다. 3년 뒤 또 수능쳤는데 이번에는 안 된다.
C는 올해 수능에 실패했지만 머리털이 다 빠질 정도로 공부해 내년에 자기가 원하는 곳에 합격한다. 3년 뒤 대박나서 수억대 부자가 된다.
터무니없는 시나리오가 아니라 실제로 비일비재하게 일어나는 일들이다.
\vspace{5mm}

올해 수능에서 실패했거나 아쉬운 전적을 거두었다면 이걸 철저히 분석해서
그 다음 승부는 어떻게 임할 것인가 싸우고 있어야지, 과거에 집착하는 건 정말 미련한 짓이다.
게다가 주위의 평가라는 건 무시해도 좋다. 평가는 그 내용보다도 \textbf{평가자의 스펙을 더 보아야하는} 문제다.
한국사회에서는 객관적인 평가라는 걸 기대하는 건 미련한 짓이 아닌가.
평가할 능력을 갖춘 사람이라면 이미 부자가 되었을 것이다. 그런 사람들은 저평가된 것을 싸게 매입해 비싸게 팔고 있었을 테니까.
\vspace{5mm}

자기 교재만 보면 잘 된다고 하는 업자들이 현재 수능결과에 책임지고 있나?
반면 저 애는 도저히 수능 안 될 거예요하는 사람들은 그 당사자가 좋은 결과가 나오면 급선회해서 찬양하는 건 콕콕에서도 있었다(...)
현명한 사람이면 올해 수능은 싹 지우고 내년 수능을 생각하며 지금 달리고 있어야 맞다.
\vspace{5mm}









\section{머리좋은 사람들의 꼼수}
\href{https://www.kockoc.com/Apoc/551136}{2015.12.18}

\vspace{5mm}

전문직 A의 경우는 원래부터 1000명씩 뽑아대던 탓에 사실 위기가 오고 있었다.
(다른 전문직 B와는 다르다. 전문직 B야 분과별로 나눠지거니와 워낙 로딩이 길기 때문에 선후배 시장경쟁이 심하지 않다)
\vspace{5mm}

그 와중에 대학교들은 시멘트 비용 선투자를 해서 전문교육기관(?)을 신설한다.
그리고 정부관료와 의원들은 A를 전문교육기관으로 뽑는 걸로 사실상 밀어붙여버렸다.
\vspace{5mm}

사실 이건 매우 큰 사건이다.
저 교육기관 통과 이후로 "돈있고 집안좋은 애가 다 해먹는 건 당연하다"라는 식의 사회적 인식이 퍼지기 시작.
적어도 그 전까지는 가난하든 안 하든 '공부만 잘 하면' 존중해주는 사회적 분위기라는 게 있었는데
저런 법안이 통과되어버리고나자 공부 잘 하면 뭐하나 \textbf{집안이 좋고 돈이 많아야지}라는 대중적 인식이 확산되어버렸다.
\vspace{5mm}

그리고 8년 정도가 흘렀다. 원래 그전부터 우려가 많았지만 교육의 질이나 평가에 대해서도 참 듣는 이야기가 많다.
무엇보다 그 전문직 A는 매우 위기감에 시달리게 된다.
그리고 예정된 코스로 진행되기 시작한다.
\vspace{5mm}

\begin{itemize}
    \item[$-$] 이 사회의 거룩한 분들이 \textbf{자녀들의 직업세습}에 성공하셨다. 7년은 매우 충분한 기간이었다.
    \item[$-$] 인원수 과다를 해결하는 방법은 간단하다. 다시 성골과 진골을 나누면 된다
    \item[$-$] 성골과 진골을 나누면 평범한 A들은 성공할 수 없다. 어려운 시험에 통과하거나 SKY를 나오거나 부모 쉴드가 강해야 한다.
\end{itemize}
\vspace{5mm}

물론 위와 같은 걸로만으로는 다 설명할 수 없다.
그러나 머리좋고 높으신 분들의 잇속 챙기기라는 건 정말 '법망의 틈새' 내지 '반사적 이익'이라는 형태로 실현된다라는 씁쓸한 것만 재확인.
내가 10대 시절에도 돈많은 사람들을 위한 '기부금 입학제' 같은 게 나오다가도 철퇴맞기도 했다.
그게 불완전하게나마 관철된 형태가 저것이다.
\vspace{5mm}

자세히 적긴 그렇지만 지금 10, 20대들은 그걸 알까. 정작 자기들의 기회를 앗아간 게 바로 자기들보고 사회를 바꾸라고 하던 그 사람이라던 걸.
언론에 간혹 B나 C같은 사람들이 젊은이들보고 사회를 바꾸라고 한다.
그런데 그 B나 C가 직접 나서서 사회를 바꾸긴 하셨나, 아니 자제분들은 어떱니까라고 \textbf{물어보면 간단히 끝나는 문제다}.
요즘 유행하는 말이 메시지를 반박할 수 없으면 메신저를 공격한다라는 건데 웃기는 이야기다.
그 메신저가 실천하지도 못 하는 메시지를 들을 필요가 있겠나.
\vspace{5mm}

왜 그 거룩하신 분들께서
한국은 공부에 찌들었다, 입시의 다양화가 필요하다, 능력이 아닌 인성을 보아야한다라고  외치셨는지는
그런 분들의 '자제' 분들이 어떤 식으로 좋은 학교에 가서 좋은 직업을 갖는지 보면 알 수 있다.
\textbf{"자기들이 불리할 때만 서민과 대중"}이라고 하는데 맞는 말이다.
\vspace{5mm}

요즘 들어 더 재밌는 건 금수저들이 흙수저 흉내를 내기 시작한다는 것이다.
저 A와 관련된 교육기관 학생들이 일부러 흙수저 흉내내면서 금수저 아니예요라고 하는 것도 좋은 예가 되겠지만
그나마 아직까지는 '부모 잘 두었네'라는 소리는 비아냥으로 들릴 수 있는 세상이라 그런 점도 있지만
어찌되었든 법이라는 건 국회 소관이고, 의원들은 여론에 신경쓰기 때문에 그 여론을 속이려면 \textbf{'서민'인 척 해야}한다.
더 웃긴 건 그 서민인 척 하는 사람들의 주장을 곧이 곧대로 듣고 자기가 기득권과 싸운다라고 착각하는 부류가 아닐까 싶지만.
\vspace{5mm}










\section{미래예측}
\href{https://www.kockoc.com/Apoc/564639}{2015.12.26}

\vspace{5mm}

\textbf{논리 ≠ 현상}
\vspace{5mm}

why?
\vspace{5mm}

우리가 아는 질서 ∈ 환상
세상 ∈ 환상
진짜 세상 = 우리의 오감이 불완전한 이상 영원히 도달할 수 없다.
수학자들의 기막힌 수리모형과 컴퓨터만 구비되면 영원히 부를 창출할 수 있다... 라는 환상이 무너진 것이
서브프라임 모기지 사태였습니다.
\vspace{5mm}

\href{http://www.aladin.co.kr/shop/wproduct.aspx?ItemId=12470584}{링크}
\vspace{5mm}

시간이 나시면 윗 책을 읽어보시는 것도 괜찮습니다.
우리나라 학생들이 너무 우상화한 '이공계 천재'들이 실제로 어떤 사건을 쳤는가 잘 나온 책이죠.
\vspace{5mm}

\begin{enumerate}

    \item 미래예측이라는 건 믿을만하나.
    \vspace{5mm}

    역사는 타임머신이 개발되지 않는 이상 바꿀 수 없습니다.
    그래서 1명이 알건 10,000명이 알건 상관없습니다.
    그러나 미래는? \textbf{바꿀 수 있습니다}.
    A라는 직업이 잘 나갈 거라는 고급정보가 생겼다고 칩시다. 그리고 그 고급정보가 99$\%$로 맞다고 합시다.
    무엇보다 그 A라는 정보를 다 해독할 수 있다 치면 그 이 예측은 의미가 없어집니다.
    왜냐면 다수가 A라는 직업으로 몰려서 경쟁이 치열해지겠고, A라는 직업분야의 과잉공급이 이뤄질 것입니다.
    그 이후는 설명할 필요가 없겠죠.
    \vspace{5mm}

    우리가 접하는 예측이라는 것은 거의 다 대중적으로 공개된 것들입니다.
    그래서 이 예측만 믿고 움직이는 건 위 메커니즘 때문에 이득을 주지 못 합니다.
    대중들이 읽지 못 하는 파격적인 가설을 읽어낸다면 모르겠지만 그건 당사자의 독해력에 달린 것이라.
    \vspace{5mm}

    \item 프로그래밍을 모르는 사람도 훨씬 좋은 컴퓨터를 이용한다.
    \vspace{5mm}

    앞으로 세상은 더 진보될 것이고 그래서 이공계 지식이 필요하다... 라고 하지만 사실 이건 좀 갸우뚱합니다.
    예컨대 C가 뭔지 모르는 사람들도 16비트 시절 컴퓨터보다 기하급수적으로 성능이 좋은 맛폰을 잘만 쓰고 있죠.
    만약 16비트 시절 흑백모니터 당시 컴퓨터 프로그래밍을 가르치던 사람이 단순히 기계의 진화만 보자면
    2010년대에는 프로그래밍을 하지 못 하면 도태될 것이다.... 라고 엉터리 예측하겠지요.
    그런데 현재 미래 예측에 대한 신문기사가 거의 이 수준입니다.
    만약 맛폰으로 이성을 유혹한다거나 하는 등 SNS질을 잘하기 위해 필요한 것은 오히려 국어실력이죠.
    \vspace{5mm}

    그 점에서 보자면 미래예측은 참 쉽습니다.
    "어라, 컴퓨터가 발전하네. 컴퓨터가 다 대체하겠네?" 이거 하나로 결론을 정해놓고 부합하는 근거만 모아놓으면 되니까요.
    어차피 미래예측은 맞지 않아도 누가 와서 돈 내놔 그렇지 않기 때문에 대충 써도 되는 것입니다만
    이와 별개로 그런 예측글을 쓰는 사람들의 교양이라는 게 짐작이 가죠.
    \vspace{5mm}

    실제로 프로그래밍을 배워야하나.... 차라리 그 시간에 서양철학을 공부하고 체화시켜서 올바르게 사고하는 것을 익히는 걸 권하고 싶습니다.
    기술로서의 프로그래밍은 사실 별 도움도 되지도 않고, 오히려 기계론적인 것만을 심화시켜 편협한 관점을 불러일으킬 수 있습니다
    \vspace{5mm}

    \item 미래를 읽고 싶으면 역사 공부를 해야 한다.
    \vspace{5mm}

    한 15년 넘게 보면서 '검증'이 되었다고 개인적으로 확인해보는 저자들이 있습니다.
    그 저자들이 누군지 가르쳐줄 필요는 없다 봅니다만 비결은 적을 수 있을 것 같은데 "역사"와 "심리"에 강하다는 것.
    저기 어디도 수학 등 이공계 과목은 없습니다. 우리가 사는 세상은 인간이 사는 문명세계이고
    우리가 배우는 역사는 이 문명세계의 패턴입니다. 그러나 현실은 \textbf{역사 = 점수 따기 엿같은 암기과목}이라는 잘못된 인식이죠.
    동서양 고중세의 '상품, 교역' 역사 같은 것을 읽다보면 이게 단순하지 않다는 것을 느낄 텐데 말입니다.
    \vspace{5mm}

    정말 공부가 안 된다 그런데 시간은 헛되이 보내기 싫다라는 분께 권하는 건
    첫째는 외국어를 공부하라는 것.
    그런데 첫째도 싫다면 도서관에 가서 그냥 재밌게 쓰여진 역사책들을 수십권씩 읽어보라는 겁니다.
    다들 꿈이 커서 내 재능을 발휘... 는 헛소리고 걍 사람들에게 명령하고 다니면서 간지나게 살아보고싶다 그러는 건데
    실천을 제외하고 비결만 보자면 그건 역사책에 나와있지 수학이나 과학에 나온 것입 아닙니다.
    \vspace{5mm}

    \item 미래예측의 정확성 담보 : 추상성
    \vspace{5mm}

    구체적 현상을 반드시 부정할 수 없는 '추상화된 개념'으로 환원해보면서 그것이 왜 다른 구체적인 현상으로 나타날 수 밖에 없는가
    이게 논증이 되지 않으면 미래 예측이 아니라 공상과학 수준도 못 미치는 픽션이 된다고 하겠죠.
    예컨대 유력한 예측 중(뭐 이건 출처는 묻지 마시길. 그냥 저도 검증해보고싶어하는 것이라서)
    하나는 다시 종교의 시대가 온다는 것인데 이런 논증이더군요.
    \vspace{5mm}

    \begin{itemize}
        \item 첫째, 미국, 유럽, 동아시아는 공동체가 모래알처럼 쪼개지는 개인주의로 갔거나 가고 있다.
        이런 모더니즘이 극에 달하면 개인의 자아까지 해체위기를 겪으며 이건 상품이나 이미지로 해결되지 않는다.
        그 빈 공극을 채워주는 건 바로 신앙의 문제고 또한 다시금 공동체를 복구하고자 하는 움직임으로 나타난다.
        \vspace{5mm}
        
        \item 둘째, 출산율이 높은 집단은 종교를 믿는 사람들이다.
        종교를 믿지 않는 개인들은 결혼과 출산을 기피한다. 시간이 지날수록 공동체에서 비율이 줄어든다.
        반면 근본주의적인 종교를 믿는 공동체들의 출산율은 높다. 그들이 사회의 주류가 되어간다
        \vspace{5mm}
    \end{itemize}

    그렇다면 종교 이전에 카운셀러가 $-$ 이게 수익을 창출할지는 몰라도 $-$ 주류가 되어간다는 판단인데
    사실 우리나라만 하더라도 이 카운셀링을 10대들은 사교육, 그리고 그 이상은 역술가, 무당, 목사, 승려들이 담당하고 있죠.
    \vspace{5mm}

    이미 검증되었다고 하는 예측은 이른바 "호연사회" $-$ 즉 개인들은 자기들의 취향에 따라 뭉치고 그걸 중심으로 살아간다인데
    콕콕사이트도 그렇겠지만 요즘은 아프리카 잘 나가는 BJ 들이면 (뭐 잘 나가야겠습니다만) 남 부럽지 않다고도 하죠.
    과거 법인사회에서 이제 초개인사회로 가고 있다.... 가 되겠고 그 때문에 각자의 '자기주도학습'이라는 게 매우 중요해진 것 같습니다.
    개개인이 각자 방송하는 시대라면 그만큼 더 많이 배우고 훈련해야하는 것.
\vspace{5mm}
\end{enumerate}







\section{공포게임에 등장한 참고서들}
\href{https://www.kockoc.com/Apoc/567960}{2015.12.28}

\vspace{5mm}

수학의 정석 색깔보니 개정 과정인데.
깨알 같은 현장재현(...)
거기다가 피묻은 정석도 등장.
\vspace{5mm}

챗과 결합된 인터넷 방송이야말로 수십명 앉혀놓고 히틀러식 연설하는 강의보다
훨씬 더 집중하기 좋고 상호 피드백이 가능한 진화된 체제라는 것이죠.
기존 인강이 영화라면 아프리카 방송은 '연극'이 아닐까 싶은데.
\vspace{5mm}

저러면 기존 과외 + 학원 강의 장점만 살릴 수 있다는 것.
\vspace{5mm}

갑자기 과거가 생각납니다만.
고등학교 시절에 언어영역 문제집을 풀 시절. 참고서 저자들이 바로 학교 선생님들이라 무려 저자직강이기도 했는데
당시 등장했던 지문이 "흥을 돋구면서 관객과 상호작용하는 마당놀이에서 영화의 한계를 극복할 수 있다" 뭐 그런 내용인데
국어선생님이 그랬죠. 뭐 썰이야 그럴싸한데 어떻게 스크린으로 관객과 영화가 상호작용할 수 있냐.
\vspace{5mm}

라는 말씀하신 게 기억이 납니다만 그 피드백이 현실적으로 가능한 세상이 왔죠.
그 당시 와우를 어찌 알았을 것이며 \textbf{1}인 방송이 쉬워졌다는 것을 감히 예측?
물론 그렇다고 하더라도 근본적인 건 안 바뀌었습니다만.

$-$ ex) 저런 방송을 잘 하려면 어찌되었든 꾸준히 공부하고 노력해야하는 건 변함이 없으니까 말입니다 $-$
\vspace{5mm}






\section{부모 자격}
\href{https://www.kockoc.com/Apoc/568250}{2015.12.29}

\vspace{5mm}

\textbf{가정이 '사랑'으로 유지된다}라는 말만큼 허황된 이야기는 없을 것입니다.
더 극단적으로 나가면 남녀관계도 마찬가지라고 보는데
사랑으로 유지되는 것이라면 그렇게 많은 커플이나 부부가  '속물적' 이유로 헤어지거나 추태를 보이진 않겠죠.
하지만 그럼에도 불구하고 이야 진짜 인격자들이라고 느껴지는 케이스가 없지는 않은데
\textbf{"윤리적, 종교적 의존 관계"}인 경우가 많습니다.
\vspace{5mm}

남자든 여자든 능력이 좋으면 한 배우자에게만 성실함을 다하긴 힘들 것입니다.
물적 조건이라는 건 바뀔 수 밖에 없고 상대의 성적 매력도 곧 질리거나 노화되어버리기 때문입니다.
여자들이 찾는 키크고 잘생긴 부자 남자들은 '바람을 피우기' 딱 좋고,
남자들이 찾는 예쁘고 어린 베이글녀들 역시 유혹하는 남자들이 많으므로 마찬가지입니다.
물론 판이 대문호 양성소라고 하지만 전부다 주작은 아닐 것이고 저 역시 보고듣고 경험하는 것 보면 막장.
가령 건물주들이 웃는다는 번화가의 유흥 숙박은 '남녀의 바람'이 아니면 영위되기 어렵단 말조차 있죠.
\vspace{5mm}

그렇다면 \textbf{사랑만으로(?) 맺어진} 남녀가 자식을 낳더라도 잘 키울지는 의문이지요.
그 사랑은 허구에 가깝죠. 그리고 자녀를 낳고 키우는 건 매우 힘든 일입니다.
거의 10억에 가깝게 들어가는데 보상받을 수 있을지도 의문이죠.
그래서 자녀에게 폭력을 가하는 일이 비일비재하게 일어납니다(그 반대도 있습니다만)
이것이 바로 실체가 없기 때문에 O, X 를 가릴 수 없는 \textbf{위대한 사랑의 힘}입니다.
\vspace{5mm}

온오프라인으로 이런 문제로 적지않게 상담하면서 느낀 건
"사랑만능론"을 주장하는 인간부터 잡아서 조리돌림을 해야하지 않느냐는 것입니다.
피해자들은 가족이 자기를 사랑하지 않아서, 혹은 자기가 가족을 사랑하지 않아서라는 잘못된 풀이법을 전개하더군요.
해법은 간단할텐데 말입니다. 그건 그냥 그 사람이 '악해서' 벌어진 것입니다.
\vspace{5mm}

그리고 가정은 최소한의 도덕도 강제하기 힘든 곳입니다.
\textbf{가족구성원에게는 온갖 육체적/정신적 폭행을 서슴지 않는 사람이}
\textbf{집 밖에만 나가면 인격자가 되는 경우} 그리 낯설지 않을 것입니다.
그거야 간단하죠. 집 밖에 나가면 법의 눈치를 보아야 합니다. 법은 최소한의 도덕이니 그건 지켜진다는 것입니다.
그러나 집 안에서는 그런 일이 지켜질 수가 없지요.
\vspace{5mm}

게다가 더 심각한 차이는
그나마 아이들은 학교에서 교육을 받으면서 무엇이 옳고 그른지 배우는 반면에
어른들은 교육이 중단된 상태이기 때문에 그런 윤리적 교육이 장기간 중단된 상태라는 것입니다.
아이들은 어른들이 자기보다 많이 배우고 경험도 많으니 이 분들의 언행이 옳겠구나라고 착각을 하면서
자기가 부당한 대우를 받더라도 \textbf{'내가 뭔가 잘못했구나'라고 착각}합니다.
그래서 모든 것을 다 자기 탓이다라고 생각하면서 인격이 망가지다가 나중에는 폭발하는 것이지요.
\vspace{5mm}

부모에게 의존하지 말라는 건 금전적인 것만 의존하지 말라는 이야기가 아닙니다.
형식은 부모인데 실제 인격으로 치면 부모 자격이 의심스러운 사람도 꽤 있습니다.
본인들이 냉정히 판단해서 "낳아준 분들이지만 인격적으로는 그렇다"라고 생각하면
지체없이 그냥 \textbf{"낳아주고 키워준 건 고맙고 그 빚은 갚겠습니다만 제 인생은 제가}" 이렇게 나서야합니다.
\vspace{5mm}

물론 정반대의 경우도 있습니다.
부모에게 폭행당하거나 상처입은 건 인정하지만 그걸 본인들의 '게으름'을 정당화시키려는 경우도 간혹 없지는 않더군요.
부모가 뭘 어떻게 하든 자녀 양육을 게을리 하지 않았다면 그건 본인들의 태만한 학업 핑계가 될 수 없는데 그러는 친구도 가끔.
\vspace{5mm}

아무튼 요새 제가 느끼는 건 그건데
그 놈의 사랑만능주의가 생각보다 문제가 많다는 것.
시간이 갈수록 부모자격이 없는, 그리고 그걸 상실해나가는 어른들도 많아지고 있단 것입니다.
\vspace{5mm}





\section{저녁이 있는 삶이 불가능한 이유}
\href{https://www.kockoc.com/Apoc/580015}{2016.01.08}

\vspace{5mm}

'이게 다 기득권층 때문이다'
는 그냥 일본을 공격한다 식의 도피.
\vspace{5mm}

실제로는 저녁없는 삶을 살 수 없는 이유는 "모두에게 살기 좋아졌기 때문"이라는 역설.
\vspace{5mm}

\begin{enumerate}
    \item [$-$] 교육수준이 높아졌다, 모두가 현재에 만족하지 않는다.
    \item [$-$] 실제로는 계층간 이동이 활발하다
    \item [$-$] 전지구적 경쟁이 벌어지면서 경쟁 싸이클이 0918에서 2400으로 바뀌어버렸다(해가 지지 않는 경쟁)
\end{enumerate}
\vspace{5mm}

우리가 자고 있을 동안에도 부지런히 사업계획을 짜면서 어떤 상품이나 서비스를 어떤 고객에게 팔까,
그리고 경쟁자 누구를 몰락시켜버릴까 고민하는 사람들이 전세계에 널려있다.
전세계를 향해 뛰는 대기업일수록 그래서 저녁 따위는 있을 수 없다. 저녁을 즐기면 그대로 추격당하기 때문이다.
그럼 공무원은? 기본적으로 관(官)은 경쟁을 하지 않는다.... 라고 하지만 고위직일수록 인간다운 삶은 없다.
\vspace{5mm}

모두가 인간답게 살 수 있다라는 건 결국 모두가 경쟁에 뛰어들 수 있단 이야기이다.
즉, 살기 팍팍해진 건 역설적으로 모두가 살기 좋은 세상이 되어버렸기 때문이다.
나 혼자 공부하고 노력하는데 다른 사람은 공부 안 하고 노력 안 하면 '저녁있는 삶'은 가능하다.
그러나 인터넷이 \textbf{촌놈들을 멸종}시켜버렸다.
\vspace{5mm}

흙수저 금수저라고 이야기하지만 이거야말로 인터넷에 떠도는 밈을 무분별하게 복제한 것이 아닌가.
역사상 지금처럼 가진 사람들이 없는 사람들 눈치를 보는 세상은 없다. 안 그럴 것 같지만 냉정히 비교해보아도 그렇다.
소위 금수저에 해당하는 사람들조차도 "전 금수저 아니예요", "민중 만세, 모두가 행복해질 수 있는 사회"라고 빈민 가면을 쓴다.
가진 사람들일수록 인기를 모으기 위해 일부러 없는 첫, 못 버는 척, 힘든 척 한다.
왜냐면 많은 사람들을 적으로 돌려버리는 순간, 그리고 있는 척 해대면 오래 못 갈 것임을 잘 알고 있기 때문이다.
\vspace{5mm}

가진 자들이 신경쓰는 건 교육이다. 그건 후계자 교육도 그렇지만 이제는 본인 교육도 무시할 수가 없다.
교육받지 못 하면 야생에서 생존할 수라도 있었던 석기시대 사람만도 못 하기 때문이다.
20세기라면 10년 교육을 받아 30년을 버틸 수 있었다. 그러나 현재는 10년 교육을 받아도 10년을 따라잡을 수 없다.
모두가 자유롭다, 모두가 경쟁상대다, 그래서 세상은 계속 진보한다 $-$ 그러니 자기교육을 하지 못 하면 도태당해버린다.
다들 먹고살기힘드니 뭐니해도 계속 승부에서 이기거나 대박을 터뜨리는 사람들은 있다.
그 사람들은 학벌이 보잘 것 없을지 몰라도 변화하는 세상에 대처할 수 있는 자기 교육 및 훈련에 성공한 사람들이다.
\vspace{5mm}

이런 데 저녁이라는 게 존재할 수 있을까.
\vspace{5mm}









\section{외모지상주의.}
\href{https://www.kockoc.com/Apoc/582733}{2016.01.10}

\vspace{5mm}

아주 보기 흉하다면야 어쩔 수 없다쳐도 그냥 평범한데도
잘 생겨지고 싶다 예뻐지고 싶다고 하는 병자들이 정말로 많은데요.
그냥 컴플렉스 덩어리들입니다.
생긴 대로 걱정없이 근심없이 씨뿌리는대로 먹고살면 되지
저 존못이니라는 개드립치면서 괜히 "나 잘 땡겼지? 나 에프지?"라고 하는 사람들은 걍 극혐.
얼마나 결핍감을 느끼면 지 얼굴 가지고 저러고 있나 그러는 것이죠.
\vspace{5mm}

물론 잘 생긴 남자나 예쁜 여자는 보기는 좋습니다만 그건 어디까지나 피상적인 관계일 때만입니다.
참매력이라는 건 얼굴 이전에 그 사람의 "뇌"에서 비롯되는 것이죠.
그럼 그 뇌는? 많은 경험과 독서와 공부를 해야죠.
어느 자리건 가서 화제를 유도하는 이슈메이커 아니면 분위기 잘 살리고 노는 회식부장 스타일이 낫죠.
\vspace{5mm}

일단 이건 진짜 나이 처먹으면서 느끼는 거라서 적겠음.
남자의 경우는 기생오래비과는 잘 해보았자 소용없습니다. 눈빛이 맛탱이 간 경우가 많거니와 어차피 노화 못 이김.
남자는 그냥 눈빛 하나로 90$\%$인데 (뭔 나루토 찍냐하면 할 말 없지만) 이건 진짜 그 사람의 경험치, 공부치에 비례하고 있음.
남창할 것도 아닌데 꿀피부 자랑하고 무슨 지가 옷가게 마네킹도 아니면서 고급 옷 걸쳐보았자 눈빛이 마약한 듯 맛탱이가면 답 없음.
여자들은 아무리 외모 그래도 슬프지만 연령빨이 가장 강해요. 마찬가지로 노화 못 이깁니다요.
얼마나 곱게 늙느냐 $-$ 평소에 많이 웃고 다니고 좋은 생각하고 다니면서 청정하게 살았냐 그게 가장 중요합니다.
길거리에서 보는 흔한 아지매들이 10년 전에 소라넷급(...) 모델이었다 생각하시면 됩니당.
\vspace{5mm}

외모만으로 서로 평가하는 그런 사람이면 그냥 관계 끊는 게 좋습니다.
얼굴 잘 생기고 옷 잘 입는다... 하는 인간들이 나이먹으면 가는 게 걍 콜라텍, 등산 그런 것이고(뭘 말하는지 알 것임)
사람을 평가하려면 그 사람이 어떤 책을 읽는지, 어떤 예술을 감상하는지, 그리고 무엇을 계속 공부하고 있는지 그걸 보아야합니다.
저 중 하나라도 안 되어있으면 그냥 멀리하는 게 좋습니다.
예를 들어서 외모에 치중하는데 도덕관윤리관도 없고 머리에 든 게 없으며 학습할 의욕이 없으면 그게 사람입니까, 짐승이지.
먹고자는 것이나 좋아하고 나중에는 남녀관계도 로마제정이나 고려시대 말기를 연출하죠.
그런 사람들이 결혼을 해도 과연 배우자에 충실할까, 그런 경우는 거의 못 보았습니다. '난 안 그러는데요' 해도 그래서 전 안 믿음.
트리밍 잘 되어있어도 향수를 뿌리고 화장을 해도 돼지는 그냥 돼지인 겁니다.
\vspace{5mm}



\section{머리가 좋다는 것의 정리}
\href{https://www.kockoc.com/Apoc/608996}{2016.01.25}

\vspace{5mm}

대체적으로 여러가지 특징이 있겠지만
단언코 말하면 이건 훈련으로 어느 정도 보정이 가능한 것들.
\vspace{5mm}
\begin{enumerate}
    \item 인내심이 강하다.
    \vspace{5mm}

    극단적으로 말하면 자기 부모를 죽인 원수 앞에 무릎을 꿇을 수 있다,
    즉 감정에 사로잡히지 않는다, 마법이 먹히지 않는 골렘과 똑같은 내성이라 보면 되겠는데
    혹자는 차분하다라고 하지만 사실 이건 인내심이 매우 강한 것이다.
    그만큼 스트레스를 덜 받으며, 냉정한 판단과 확실한 실천을 보장받을 수 있다.
    이런 친구들은 사소한 데 안 흔들리는 걸 넘어, 중요한 것에도 안 흔들린다.
    그래서 "독하다"라고 할 수 있다.
    \vspace{5mm}

    \item 끈기가 있다.
    \vspace{5mm}

    인내심이 자알 버티는 것이면 끈기는 \textbf{오랫동안} 버티는 것이다.
    이것 역시 중요하다. 끈기가 있어 오래 버틸 수 있다면 공부를 오래할 수 있고 그래서 가시적 성과도 맛본다.
    12시간 공부해야 진전이 있는데 A는 끈기가 없어 8시간만 하고, B는 끈기가 많아 16시간을 한다면?
    A는 아무 것도 얻지 못 하지만, B는 뭘 하든 성과를 보기 때문에 슬럼프가 덜하다.
    \vspace{5mm}

    \item 약속을 잘 지킨다
    \vspace{5mm}

    계획은 누구라도 거창하게 잘 세운다. 그러나 계획을 지키는 사람은 정말 1000명 중 1명이다.
    계획을 안 지키는 사람이 공부했다라고 자부할 자격이 없다, 물론 현실은 노오력했다는 사람들 보면 계획을 지킨 경우 별로 없음.
    가장 지키기 어려운 약속은 "자신과의 약속"이다. 왜냐면 얼마든지 파기해버릴 수 있기 때문이다.
    물론 어긴 대가를 심하게 치러야 할 것 역시 자기와의 약속이다.
    계획이란 말을 쓰기보다는 자기와의 약속이란 말을 쓰는 게 더 적절
    \vspace{5mm}

    \item 언어능력이 풍부하다.
    \vspace{5mm}

    수학조차도 언어능력이 풍부한 사람이 잘 한다.
    첫째는 이해능력, 둘째는 표현능력이겠지만 본질적으로는 구체적인 것을 추상화할 수 있는 능력.
    언어능력을 좌우하는 건 바로 '대화 경험'과 '독서'이다. 언어능력이 없으면 국어와 영어에서 바로 타격을 입는다.
    아래에 논하는 이미지 능력과 차이는 바로 가치 판단이다. 언어능력이 좋아야 가치판단을 정확히 내릴 수 있다.
    조기교육 하에 이미지 능력이 좋은 친구들은 많다(인터넷 덕분이기도 하지만), 하지만 실제로 언어능력이 좋은 경우는 드물다.
    \vspace{5mm}

    \item 이미지 능력이 좋다.
    \vspace{5mm}

    추상적인 이야기를 던지면 그걸 구체화된 이미지로 상상하고 표현할 줄 아는 능력이다.
    두뇌회전이 빠르다는 게 별 게 아니라, 감각 이미지와 행동 이미지가 잘 형성된 경우다.
    특정한 상황에 필요한 이미지를 즉시, 선명하게 떠올릴 수록 신속, 정확하게 대처해나갈 수 있다.
    여기서부터 머리빨을 타고난다고 하지만 실제로는 심각한 장애가 아닌 한 훈련을 통해 키울 수 있다.
    \vspace{5mm}
\end{enumerate}
중요도는 1>2>3>4>5
\vspace{5mm}





\section{인간의 탐욕}
\href{https://www.kockoc.com/Apoc/610237}{2016.01.26}

\vspace{5mm}

여전히 철지난 진보 vs 보수 이야기가 있던데
뭐 이건 정치 글을 쓰기보단 그냥 시행착오하지 말라는 이야기입니다만.
대략 과거대선을 보면
1997년과 2002년은 극적이긴 해도 모두 현재 야당 쪽이 뽑혔는데
그 때도 사실 그럴만한 이유는 있었어요.
\vspace{5mm}

첫째, 당시 여당의 장기집권에 질렸다
둘째, 사회정의를 바로 잡고 싶은 열망이 강했다.
\vspace{5mm}

그런데 재미난 게 그 지지계층들이 2007년에는 정반대의 선택을 합니다.
뭐 여러가지 이유가 있겠지만 극적인 게 그거죠.
\vspace{5mm}

"부동산 폭등"
\vspace{5mm}

당시 대통령님께서는 집을 사지마라, 떨어질 것이다라고 얘기했고 정말로 사람들은 그걸 믿었죠.
그런데 그러면서 행정계획도시라고 해서 세종시를 개발하자 시중에 물경 200조나 되는 돈이 토지보상금으로 풀립니다.
그리고 그 돈은 다시 서울의 강남부동산으로 몰려버리면서 '투기열풍'이 조장되죠.
투기열풍이 조장되면서 아파트 공급이 늘어나고 그러면서 건설사들에 자금이 돌면서 일시적으로 경기부양 효과가 나옵니다.
그럼 이게 우연인가 아니면 '미필적 고의'인가 하는 건 해석에 맡겨둘 문제입니다만(사실 고의 같은데)
이렇게 레드x나 핫식x를 과잉복용한 대가는 언젠가 치르게 될 거라는 예측이 있었죠. 그게 지금 젊은 세대들이 치르고 있는 것입니다만.
\vspace{5mm}

아무튼 당시에 집 판 사람들은 수억을 손해보고
반면 대통령 말을 믿지 않고 강남의 돼지엄마들처럼 빚내서 집산 사람들은 수억을 벌었죠.
그리고 집집마다 가정불화가 많아집니다. 수백만원 가지고도 싸우는데 수억이면 칼부림 안 나는 게 다행이죠.
종부세 그런 것 하나도 안 먹혔죠. 거대한 자금이 풀렸는데 그게 뭔 소용이겠음
그런 걸 눈치까는 사람이 집 한채만 구입할 리가 없죠. 그렇게 당시에 빈부격차가 커집니다.
거기다가 펀드열풍도 장난이 아니었으니.
\vspace{5mm}

그런 빈부격차의 카오스를 겪고난 사람들이니 정의고 뭐고 다 모르겟다, 아니 정의에 대한 회의론이 감도는 것이죠.
그 다음 선거 결과는 아실 것입니다. 아마 이건 앞으로도 10년 넘게 쭈욱 계속되지 않을까 하는데.
이 부동산 게임 앞에서는 기존의 역사라느니 철학이라느니 정의라느니 다 무색해졌습니다.
긍정적으로만 보자면 한국이 본격적으로 자본주의 국가로 발돋움했구나 할 수도 있겠지만 당하는 사람 입장에서는 별로인데
역사적으로는 지혜가 쌓인다고 할 수 있는 거죠.
사상의 논리가 중요한 게 아니라, 그게 실제로 정치에서 어떤 식으로 작용해서 이해관계에 영향을 주느냐하는 게 중요하다는 것을.
\vspace{5mm}

그럼 부동산을 바로 잡으면 되지 않느냐.... 쉽지는 않습니다.
이미 적지않은 중산층과 서민의 자산이 그 부동산에 연동되어버렸기 때문이고
만약 세금을 많이 거둔다고 하더라도 그건 바로 조세전가가 이뤄질 것입니다.
세금을 거둔다고 하면 어차피 임대료를 높여버리겠죠.
이런 문제의 해법은 늘 그렇지만 '공급' 확대 $-$ 신도시를 개발하거나 중요한 관공서를 지방으로 보내거나 하는 것인데
상당히 시간이 많이 걸리는 문제고, 게다가 그런 시도가 있으면 항상 투기가 엮이게 되어버립니다.
지방혁신도시의 사례처럼 실패한 경우도 적지 않고.
\vspace{5mm}

그런데 정말 중요한 것은
당시의 대학교육 $-$ 고급교육이 저런 흐름을 읽을 수 있는 게 되어있느냐엿는데 그게 아니었죠.
욕을 먹는 저자 분이 바로 공병호, 그리고 역시 비판을 많이 받은 책이 부자 아빠 가난한 아빠 같은 책인데
지금 돌이켜보면 그나마 적중을 시킨 게 저런 '탐욕스러워보이는' 저자들의 책입니다.
\vspace{5mm}

아무튼 그래서 대학수험에 몰두하는 학생들은 다 돈이 안 되니까 의치한으로 몰린다 그건데
\textbf{만약 의치한조차도 겨우 본전을 챙기는 수준일 것이다.}.. 로 되어버리면 그럼 그 때는 어떻게 되는 건가 라는 의문을 제기할 수 있죠.
그런 일이 가능할 리가 없어... 라는 말이 나오는 것일수록 그 불가능한 것이 현실이다라고 가정해봐야 합니다.
\vspace{5mm}

이 점에서 사실 한국인들은 트랩에 종속된 것인데
하나는 부동산에 지나치게 묶여있어서 열심히 부동산에 착취당한다는 것이고
다른 하나는 교육이 대학에만 국한되어 있지, 어떻게 먹고살고투자하는 가에 대해선 거의 까막눈이나 다를 바 없단 것입니다.










\section{화낼 줄 아는 법}
\href{https://www.kockoc.com/Apoc/612402}{2016.01.28}

\vspace{5mm}

우리나라 교육 중 잘못된 것 : \textbf{무조건 참으라고 하는 것}이다.
\vspace{5mm}

참는다는 건 본인의 의사, 의지에 따라서 주어진 고통스러운 상황을 견디는 것이다.
이것은 부조리한 것에 무조건 침묵하는 것과 다른 것인데.
\vspace{5mm}

어른들은 어린 사람들에게 참으라는 건 후자의 의미에 가깝다는게 불편한 진실.
\vspace{5mm}

사람은 웃을 줄도 알아야하지만 더 중요한 건 화를 잘 낼 줄도 알아야한다는 것이다.
화를 낼 때 내지 않으면 나중에는 통제를 할 수 없게 된다.
화를 내지 않으면 폭력이나 자해, 그보다 더 끔찍한 극단적 행동에 호소해버리고야 만다.
\vspace{5mm}

그런데 우리나라 꼰대들은(... 그러고보니 나도 꼰대인가 ...)
어이어이 좋게 해결해, 다 참으면 그만이야(물론 본인이 그런 경우는 참지 못 한다)
화내지마, 웃어야지 .... 뭐 이런 식으로 강요한다.
\vspace{5mm}

우리나라만큼 인내를 강요하는 나라도 없을 것이다, 하지만 그로써 모두가 인격자가 되는지는 의문이다.
그보다 더 필요한 건 \textbf{"싫은 것은 싫다"라고 분명히 얘기하고 의사소통하는 것}인데 이런 건 왜 안 가르칠까.
무조건 "네", "Yes"라고 하지 않으면 나쁜 사람이라고 하는 인식 때문에 제 권리를 못 찾는 사람들이 많지 않나?
3$\sim$4월에 대학가에서는 신입생들에게 이상한 걸 강매하거나 괴상한 동아리로 유혹하는 사람들이 있다.
어린 애들일수록 "아니오"라는 걸 못 하는 심리적인 약점을 이용한 결과다.
\vspace{5mm}

화를 내는 것은 고함지르는 것과 다르다.
무작정 고함을 지르고 샤우팅을 하는 것은 그냥 히스테리를 부리고 자기 통제력을 상실해버리는 것이다.
고수들은 상대가 히스테리를 부리도록 냅둔다. 히스테리를 부린 사람은 그게 부끄러워서 저자세가 되어 불리한 조건도 감수하기 때문이다.
\vspace{5mm}

\textbf{제대로 화를 내는 사람은 폭발하지 않는다.}
\vspace{5mm}

첫째, 눈을 똑바로 뜨고 상대방을 쳐다본다.
둘째, 하고싶은 말, 해야하는 말을 아주 정확하고 또렷하게, 그리고 분명히 전달한다.
셋째, 자기가 화를 낼 수 밖에 없는 윤리적이거나 상식적인 이유를 말하며 상대방의 부조리함을 고발한다.
\vspace{5mm}

자신이 진정 분노했다는 것, 그리고 이를 바로 잡을 수 밖에 없다는 의사, 즉 뜨거운 마음을 상대에게 전도시키는 작업이다.
\vspace{5mm}

타겟에 정확히 미사일을 쏘는 것이다, 그런데 대다수 사람들은 미사일이 발사대에서 폭발해버리는 걸 화를 내는 것이라 착각한다.
제대로 화를 내는 건다고 내 요구사항이 관철되지는 않는다.
그러나 상대방이 '부끄러워'는 한다. 즉 '수치심'을 느끼는 것이다.
\vspace{5mm}

만약 수치심을 느끼지 못 한다면 그 때에는 화를 내야하는 문제가 아니다. 그 상대는 이미 정상의 영역이 아니기 때문이다.
그럴 때에는 정말 싸우거나 아니면 멀리해버리는 단계로 가야한다.
\vspace{5mm}









\section{공부보다 인간성이 중요하다?}
\href{https://www.kockoc.com/Apoc/626862}{2016.02.07}

\vspace{5mm}

이건 논란이 있다.
일단 공부보다 인간성이 중요한 건 맞다. 공부는 노력해도 되지만 인간성은 노력에 비례하는 건 아니다.
그런데 문제는 그럼 \textbf{인간성이 좋은지 안 좋은지 어떻게 아냐}는 것이다.
\vspace{5mm}

실제로 안 좋은지를 아려면 겪어보아야하는데, 겪어본다는 것은 상대가 악인이면 우리가 이미 손해를 본다는 얘기다.
게다가 \textbf{악인일수록 미소짓고 착한 척 하는 경우가 많다}.
그나마 영업용 미소라거나 장롱 윤리라면 애교지, 실제로 희생정신을 발휘하는 것처럼 보이는 놈이 더 무서운 녀석인 경우가 많다.
\vspace{5mm}

이 이야기를 쓰는 이유는 갑자기 옛날이 기억나서.
어린 시절에 한 어르신이 그랬다. "공부가 뭐가 중요해? 인간이 되어야지"
그런데 세월이 지나고보니까 그 어른은 인간보다는 개과에 가까우신 분이었고
그나마 \textbf{배신하지 않고 남은 게 공부였다}.
\vspace{5mm}

그렇다고 해도 인간성이 안 중요하나. 그건 아니다. 인간성이 안 좋은 사람과 얽히면 본인 인생도 망한다.
그런데 그 인간성을 알기 위해서는 역시 사람 공부를 해야한다고 정리되는 것 보면 결론은 정말 간단한 것이다.
\vspace{5mm}

인간성도 좋고 공부도 열심히 한다면 더할 나위 없이 좋다.
당연히 최악은 인간성이 나쁘고 공부도 안 하는 경우겠지만
굳이 우선순위를 따진다면 "공부"하는 걸 믿는 게 나을 것이다.
그나마 공부한다는 건 그 사람도 '노력을 중시하는' 룰을 준수한다는 이야기고
사실 노력을 존중하는 것이 좋은 인간성의 전제이기 때문이다.
\vspace{5mm}






\section{하류교육}
\href{https://www.kockoc.com/Apoc/627628}{2016.02.08}

\vspace{5mm}

애시당초 의전원이나 로스쿨이 독과점 구조 깨고 자유로운 교육 하겠다 다양성을 충족시키겠다... 는 취지로 알고 있는데
결과야 보시다시피. 제도 하나 바꾼다고 공부의 본질이 달라지겠냐만, 그리고 공부의 본질은 \textbf{"주입식"}이 아닌가.
특히나 카리스마 강사들이 잘 먹히는 것도 그런 이유,  사고는 정지시키질 몰라도 주입은 정말 잘 시켜주기 때문이다.
\vspace{5mm}

내 경우도 기성세대에 대한 반발이 없는 게 아니라서 '자유로운 토론'이 중요하다고 생각은 하지만 교육은 아닌 것 같다.
사실 생각해보면 당연한 게 부모들도 하늘에서 돈이 떨어진 게 아니다, 나름 성과를 보고싶어한다.
거기다가 학생들조차도 참교육 그런 건 사실 별 관심이 없다, 오로지 \textbf{성적이 오르거나 수능에 대비할 수 있기}를 바란다.
그렇다고 이게 잘못된 것이라고 할 수는 없다, 사실 이거야말로 절박한 것이다.
\vspace{5mm}

그리고 (나를 포함한) 사람들도 간사하긴 마찬가지다.
겉으로는 과정이 중요하다거나 취지가 좋다고 이야기하지만
실제로는 어떤 것이든 \textbf{물질적 보상이 따라오길 바라며 결과가 좋길 바란다}는 점에서는 속물과 큰 차이는 없다.
자기 전공이나 소신을 보통은 20대 초까지는 강조한다.
물론 20 후반이 되면 왜 진작 주입식 공부해서 의대에 안 갔을까, 혹은 공무원 시험을 치지 않았을까 후회하는 경우도 많다.
그런데 신기한 건 그렇다고 부모들이나 어른들이 그런 조언을 안 해주었냐면 \textbf{사실 그것도 아니라는 것}이다.
\vspace{5mm}

취업난이다 백수 천지다라고 해도 할 놈은 한다. 다만 그 준비는 10여년에 걸친 장구하고 무서운 과정이다.
자기 주관을 강조하면서 자유분방하게 간 사람들이 사실 목적을 이루는 경우는 별로 없고(역시 개인적으로 본 적은 없다)
주관이라는 것은 없지만 극성 부모가 시키는대로 하면 나중 일은 모르겠지만 어느 정도 평타는 치는 결과까지 나오는 경우는 많다.
부모가 강하게 잡아준 것에 고맙다라고 말하면 훈훈한 해피엔딩은 되지 않을까 싶지만.
\vspace{5mm}

다들 부의 세습만 경계하지만, 진정한 의미에서 격차를 낳는 건 바로 지(知)의 세습니다.
부의 세습은 계량화할 수 있고 세금으로 조정할 수도 있다, 실제로 우리나라의 상속세는 상당히 가혹한 편이다(물론 탈세도 많이 하지만)
그러나 지의 세습을 막을 수 있는 방법은 없고, 사실 이걸 막아야하는지도 의문이다.
오늘자 금수저 기사에서 영어유치원 월 200만원이 지적되던데 사실 월 200만원이 문제가 아니라
저 사람들이 어린 시절부터 자녀들을 파시즘적으로 교육시키려고 하는 것 자체가 더 중요한 것이다.
부의 세습만 가지고 세금을 더 많이 걷어 복지를 하면서 밥먹이자고만 하면 그건 바보들이다.
저 사람들은 물론 많은 걸 사회에 환원하겠지만, 이 경우 환원되는 건 어차피 '뻥튀기된 재산'이니만큼 그리 타격이 크지 않다.
\textbf{그보다도 그들은 자녀들에게 으뜸가는 교육을 시켜서 실력과 지위를 보장해주려고 하지 않는가.}
\vspace{5mm}

반면 하류층의 교육은 민주주의를 따라간 결과 $-$ 무슨 선행이냐 책이냐, 마음껏 뛰어놀아라 $-$
일본에서 비판받는 여유교육을 완전히 따라갔다고 하지는 않더라도 하향평준화되어버린 것만큼은 사실이다.
다소 논외적인 사실이나 서울의 강남과 강북의 교육격차 뿐만 아니라 같은 민족이라면서도 체형, 얼굴에서도 인종분화(...)가 이뤄지고 있다.
영어 절대평가를 한 이유야 여러가지겠지만 개인적으로 추정하는 건 그렇다.
현재 수험생들의 영어 격차가 지나치게 크기 때문에 이제는 정상적으로 출제하기가 참 어렵다.
그나마 EBS 연계로 어떻게든 강제로 평준화시키긴 했지만 이것도 극에 달해버린 것이다.
정말 궁금해지는 데 10년 전에 영어광신이라고 하면서 비난하던 지식인들은 자기 자녀들은 어떻게 시켰을까,
그렇게 비판받던 그 영어 광신도(?)들이 결국 그 부모들 뜻대로 잘 나가는데 말이다.
\vspace{5mm}

금수저 흙수저 논쟁도 사실 대중들의 자업자득(?)이 아닌가.
이게 갑자기 이뤄진 결과가 아니다. 적어도 2000년대 초에는 이 정도까지 격차가 지적되지는 않았다.
다만 2000년대 중반부터 교육의 하향평준화라는 게 얘기되었고
하필 이걸 예리하게 지적한 데가 바로 조선일보였다(...)
조선일보에서 격찬한 책이 바로 미우라 아츠시란 사람의 '하류사회'다.
\vspace{5mm}

thttp://book.naver.com/bookdb/book$\_$detail.nhn?bid=2418162
\vspace{5mm}

지금 보면 뭘 이런 당연한(...) 내용이냐고 할 건데 당시에는 설마 저럴리가 하면서 판타지로 치부하는 분위기도 있었는데
결론적으로는 저 책과 비슷하게 되어버린 것이다(목차만 훑어보아도 다들 반론 못 할 것을?)
(그렇기 때문에 정치적 성향과 관계없이 조선일보는 일단 집중해서 볼 수 밖에 없다. 중요한 떡밥을 정말 잘 짚어내기 때문이다)
\vspace{5mm}

그럼 2006년?
직장인 $-$ 재테크 $-$ 부동산 주식 펀드 열풍
학생들 $-$ 열심히 해서 학벌 세탁하기, 의치한 갈아타기.
당시에 뭐하러 토익을 보냐, 나라를 바꿔야한다(...)라고 했지만 결과야 보시다시피.
천박하다 소리 들어도 열심히 재테크하거나 학벌세탁이나 의치한 갈아타기에 성공한 사람은 그나마 마지막 열차를 탄 것이다.
\vspace{5mm}

사실 헬조선 어쩌구하는 것도 거부감이 드는 게
그걸 한탄만 할 게 아니라 왜 그러한 상태로 바뀌게 되었는지 인과관계를 짚어야하는데 역시 그런 건 없다.
\vspace{5mm}

+
\vspace{5mm}

더불어
\vspace{5mm}

http://book.naver.com/bookdb/book$\_$detail.nhn?bid=2128396
\vspace{5mm}

이 책도 당시 참 관심있게 읽고 정말 그렇게 되나 보았는데 다는 맞지 않지만 본질적으로 그렇게 된 것 같다.
지금이야 수저론으로 상식이 되었지만, 역시 2000년대 초중반에는 부모 쉴드라는 게 먹히리라고 보지 않았기도 해서리.
\vspace{5mm}

교육측면에서는
\vspace{5mm}

http://book.naver.com/bookdb/book$\_$detail.nhn?bid=7276285
\vspace{5mm}

이건 정말 경악스러울 정도로 맞다. 더군다나 상담하면서도 느낀다.
특히 지금 수험생들은 당시로 치면 거의 초딩들이었다고 한다면 한국에서도 인과관계는 뚜렷해지지 않을까 싶다.
\vspace{5mm}

++
\vspace{5mm}

그런데 당시 우리나라 사람이 저런 걸 지적하는 책을 쓴 것은 본 적이 없다(... 당시도 예송논쟁이었지 아마 ...)
일본인들이 쓰는 미래예측이라거나 적어도 거기에 버금가는 건 일단 믿고가는 게 좋다.
\vspace{5mm}

+++
\vspace{5mm}

뭔 얼치기 사회과학도 아니고
\vspace{5mm}

http://www.yonhapnews.co.kr/economy/2013/03/05/0301000000AKR20130305217500002.HTML
\vspace{5mm}

통계자료 댈 필요조차 없는 문제일텐데.
\vspace{5mm}

첫째, 널려있으니까. 양극화는 지금 다들 동감, 동의하는 사안이군요.
둘째, 그건 그냥 2000년대 당시 바로 피부에 와닿았고 지금도 지속되는 문제인데 참.
\vspace{5mm}

그 시대 양극화 진행을 겪고 목격한 사람에게 \textbf{'뇌피셜'}이라고 이야기하는 게 바로 사회과학이나보군요.
이 경우라면 정상적인 태도는 "교육비 지출이 부의 양극화를 낳지 않는다"라는 통계자료를 먼저 본인이 제시하면 되는데
그게 있으려나요?
\vspace{5mm}

+++
\vspace{5mm}

아니 무엇보다 이것도 참 어이없는 정책이었죠.
\vspace{5mm}

참여정부 수능/내신 등급제
\href{https://namu.wiki/w/%EB%82%B4%EC%8B%A0/%EC%88%98%EB%8A%A5%209%EB%93%B1%EA%B8%89%EC%A0%9C}{링크}
\vspace{5mm}

이해찬 세대
\href{https://namu.wiki/w/%EC%9D%B4%ED%95%B4%EC%B0%AC%20%EC%84%B8%EB%8C%80}{링크}
\vspace{5mm}

다수 대중들이 저기 휩쓸리는 동안 상류층들은
\vspace{5mm}

\href{http://biz.khan.co.kr/khan_art_view.html?artid=200909131803125&code=920100&med=khan}{링크}
교육비 격차 자료 보시면 끝나네요.
\vspace{5mm}

당사자가 겪지 않았다면 스스로 검색해보고 통계자료만 보아도 뇌피셜 이야기 못 할 것이고
이거 검색하는데 5분도 안 걸립니다. 이 정도면 그냥 '뇌피셜'이라고 말하고 싶었던 것이 진의가 아닌가 의심됩니다.
\vspace{5mm}

아니, 무엇보다 사회과학 강조하는 사람은 글을 읽고 "뇌피셜"이라고 함부로 이야기할 리는 없을텐데 말입니다.







\section{대중의 선택이 현명한가}
\href{https://www.kockoc.com/Apoc/628803}{2016.02.08}

\vspace{5mm}

사실 이건 너무 간단히 논증되는데
$-$ 간략히만 보아도 주식투자만 해보아도 체감할 수 있지만 $-$
\vspace{5mm}

\textbf{대중들의 선택이 어리석은 게 아니면 똑똑한 소수가 절대 군림할 수가 없다}.
\vspace{5mm}

대중의 선택이 우매하다라거나 자업자득(?)이라는 데 발끈했다는데 그건 참 이해가 안 가는 대목이다.
사실을 사실대로 지적하는 건데 뭘 어쩌란 건가.
\vspace{5mm}

물론 다수의 선택이 옳은 경우도 없지 않겟지만 이 경우는 \textbf{"다수의 선택으로 위험이 분산"되는 경우}에 한해서이다.
식료품이나 전자제품, 그리고 살아가는 몇가지 상식에 있어선 소위 다수의 말이 맞는 경우가 많다.
하지만 그것은 어디까지나 검증된 것에 한한다.
실제로 다수가 한다고 해도 다 옳은 것도 아니고, 그 중 검증되지 않은 건 나중에야 해악이 드러나 경악하는 경우가 많다.
(ex. 폐섬유화를 촉진하는 걸로 알려진 가습기. 로마시대의 납 등)
\vspace{5mm}

실제로 대중들은 어리석은 선택을 더욱 많이 하며 그로써 생기는 차익을 똑똑한 소수가 누린다.
대중들에게 물건과 서비스를 파는 기업이건, 돈놀이를 하는 금융사나 보험사건 어떻게 대중들을 갖고놀까 그걸로 매일 머리를 쓰고 있고
국회에 통과되는 법률과 명령, 규칙 등은 그것의 진정한 의도나 side effect가 알려져 있지 않다. 사실 대중들은 그런 것을 알지도 못 한다.
\vspace{5mm}

혹자는 이것조차 대중 탓이 아니라 지배계급 덕분이라고 이야기하겠지만
이것도 무의미한 이야기다. 대중이 똑똑하고 현명하면 지배계급에 착취당하는 일이 벌어질까?
가령 노동문제만 하더라도 원샷보다 사실 더 해악인 것은 바로 불체자들을 마구 수용하고 지문날인을 받지 않는 것이었지만
그걸 막을 수 있었던 당시에 그걸 반대하는 목소리가 있었는지는 심히 의문이다.
나중에야 이게 다 국가탓이야라고 할 때에는 되돌이킬 수 없는 경우가 많다.
\vspace{5mm}

가령 지금 무상보육(누리과정)이나 급식 같은 게 정말 중요한 문제일까.
복지를 한다고 하지만 실제로는 그것도 다들 본인 호주머니에서 나가는 것이다.
그것도 국가를 거치기 때문에 100원을 내면 80원만 서비스받는 격이다.
그에 비하면 사실 지금 가장 심각한 문제인 '지'(知)의 격차에 대해선 아무도 신경쓰지도 관심쓰지도 않는다.
사실 교육격차는 국내문제만 아니라 한중일 전체로 보면 심각한 것인데, 결국 노동경쟁력이 교육수준에 좌우되는 것.
그리고 저 인구수 많은 중국에서 어떻게 밀고 들어오나 보면 사실 지금보다도 10년 뒤가 가망이 없는 것이다.
예컨대 10년 뒤에 그 때에는 중국제가 더 이상 대륙의 기상이 아니고 중국인들의 교육수준을 우리가 못 따라잡으면 끝나는 것이다.
말하지만 이런 건 우려될 때에는 다들 신경을 쓰지 않는다. 항상 현상으로 나타나고 난 뒤에야 다 알고 있다는 듯이 나온다.
\vspace{5mm}

수저론에서도 불편한 것은 대안이다. 흙수저라고 무작정 우대할 건 아니다. 중요한 건 능력있는 사람을 대우해주느냐이다.
현재의 논란은 빈부격차를 넘어, 금수저면 별 노력을 안 하고도 프리패스하고, 흙수저는 노오력을 해도 기회를 안 주는 경우다.
그런데 이 역시 입시제도를 '다양성' 확보라고 수시나 입학사정관제로 바꾸거나 면접을 확대하는 것 자체가 더 중요한 문제가 아니었나.
혹자는 수시도 열심히 한다, 학점이 좋다고 얘기할 것이다. 그러나 다들 알지 않나,
수시만큼 가정환경 빨을 많이 받는 경우가 없다는 것.
\vspace{5mm}

다시 말해서 대중들은 언론에서 고발하는 '결과'로서의 현상에만 예민하지
진짜 미래에 나타날 수 있는 바람직하지 않은 현상의 원인에 대해서는 둔감하거나 사실 알지 못 한다.
혹자는 이런 걸 거리에 나가 시위하면 된다고 하지만 그게 가능한 일인가? 이제는 모든 것이 '시장 논리'에 연계되는 식으로 나오는데?
가령 이번에 사드 배치와 같은 미국 무기를 쓰는 문제를 자주 국방으로만 접근하는 사람은 별로 없을 것이다.
한국이 미국의 비싼 무기를 사주는 대신 미국에 수출시장을 보장받을 수 있다는 일종의 리베이트성 관계를 안다면 말이다.
다시 말해 사드를 배치한다 안 배치한다하는 문제도 결국은 중국에 대한 수출시장을 포기할 것이냐 말 것이냐 하는 문제인데
이건 정권조차도 어떻게 할 수가 없는 문제다.
\vspace{5mm}

물론 이런 배후사정에 표피적으로만 접근하면 음모론의 함정에 빠지기 좋다.
즉, 그 모든 것이 특정주체가 설계한대로만 된다는 것인데 사실 이건 말도 안 되는 이야기라는 건 다들 아실 것이다.
그러나 미래의 시장을 확보하거나 룰을 더 유리하게 고치기 위한 경쟁은 지금도 벌어지고 있다.
\vspace{5mm}

그리고 무엇보다 교육적인 차원에서라면 우리가 속한 대중이 멍청하다라는 걸 인지하는 게
'그래서 끊임없이 공부하고 배워야한다'라는 걸 일깨워주는 측면에서도 좋다.
반면 대중이 현명하고 역사를 바꿔나갈 것이라고 착각한다면(슬프지만 역사에서 그런 대목은 없다. 프랑스 혁명조차도 배후는 부르주아)
공부하지 않고 누군가의 마리오네트인형이 될 뿐인 것이다.
\vspace{5mm}

한국근대사로 따지면 군부독재시절은 어차피 독재자들의 뜻대로였으니 대중들이 면피를 할 수가 있었지만,
정말로 직선제로 정치적 의사를 행사할 때부터는 다르다. 그 때부터는 스스로의 책임이 된다.
그런데 현실은 현저히 낮은 투표율 $-$ 선거날에는 놀러갑니다라고 정리되기도 하지만
역시 역설적으로 민주화가 된 이후부터는 오히려 빈부격차는 커지고 가계부채는 날로 늘어간다라는 건
"그래도 대중들이 멍청한 건 아니예요'라고 면피하기만은 힘들다.
\vspace{5mm}

+
\vspace{5mm}

조선왕조가 일본에 먹힌 것은 백성들 탓이라고 할 수는 없다. 당시 조선은 군주제였으니까.
그러나 4.19 혁명 뒤 다시 독재자가 등장한 건 국민들 본인의 책임도 있다.
독재자들은 '혼란'을 틈타 집권한다. 그런데 그건 국민들 스스로가 그 혼란을 수습하지 못 하거나 부추긴다.
그리고 그게 커져서 스스로 감당 못 하면, \textbf{누군가 대신 해결해주길} 바라고, 그게 독재의 탄생을 낳기도 한다.
\vspace{5mm}

미대입시에 떨어진 히틀러가 독일을 장악한 것도 저런 메커니늠 덕분이다.
1차 세계대전 이후 베를린은 혼란의 극치였고 당시 독일 사람들은 그걸 자기들이 수습하지 않고 누군가 정리해주길 바랬다.
만약 독일국민들이 스스로들 그걸 해결했다면 나찌독일은 없었을지도 모른다.
다들 히틀러를 욕하지만 실제로 욕먹어야할 사람은 당시 독일 국민들이었던 것이다.
\vspace{5mm}

작금의 풍토도 사실 크게 다를 바는 없다.
먹고살기 힘든 걸 스스로 해결하려하지 않고 국가가 알아서 해달라, 우리에게 인간답게 살 권리를 줘라 하고 요구만 한다면
그건 또 다른 의미에서의 독재를 출현시키는 배경이 될 수도 있다라는 걸 직시해야 한다.
재밌는 건 가장 극우적인 독재는 초기에는 사회주의적인 색채를 띠고 나타난다는 것.
\vspace{5mm}







\section{교육격차와 지리격차}
\href{https://www.kockoc.com/Apoc/628985}{2016.02.09}

\vspace{5mm}

\textbf{교육격차 $\rightarrow$ 지역격차 $\rightarrow$ 부동산 계급 $\rightarrow$ 끼리끼리 $\rightarrow$ 조선시대}
\vspace{5mm}

따분하지만 인상적인 책이다(출판사부터가 김영사다)
도서관에서 빌려읽은 계기는 아마존저팬의 한 깐깐한 일본인 리뷰어가 보기드문 별 다섯개를 주었기 때문이다.
사실 그렇게 재밌지는 않았다. 미국의 사레를 분석하여 어찌보면 너무 당연한 이야기를 했기 때문이다.
그런데 책을 읽은 후 분석해보는 사회 현상이 정확히 위 책에서 얘기한 룰을 따라간다.
\vspace{5mm}

진정한 격차는 사실 교육격차이다.
부의 격차는 정부정책으로 일정 부분 해소될 수 있다. 유권자들이 마음먹고 투표해서 영향줄 수 있다.
그러나 교육격차를 되돌리기는 매우 어렵다.
\vspace{5mm}

그런데 교육격차는 결국 지역격차를 낳고,
한국사회 특성상 지역격차는 부동산 자산의 가치에 영향을 주고, 그들만의 리그를 조성하게 된다.
그렇게 해서 한국사회는 다시 조선시대로 돌아간다... 라는 이야기를 거칠게 할 수 있을 것이다.
\vspace{5mm}

여기서 희망은 그런 교육이 차라리 무쓸모하다는 것이기만 하면 되는 건데(?)
실제로 직업의 지리학이란 책에 나온 사례를 보면 \textbf{교육 강화  $\rightarrow$ 첨단산업 분야의 직업 $\rightarrow$ 가치 창출 및 혁신$\rightarrow$ 해당 지역이 부유해짐}
이런 루트를 밟는다.
\vspace{5mm}

우리나라만 하더라도 이미 몇몇 교육특구는 집값이 비싼 편이다, 그리고 이러한 헤게모니가 무너질 일은 당분간 없어보인다.
그럼 거꾸로 교육이 지역격차를 낳는다면, 거꾸로 지역도 교육에 영향을 주느냐 할 수 있다.
이건 \textbf{'클러스터링'}으로 설명된다.
극성 학부모나 경쟁이 심한 학원이 있는 곳은 정보가 빠르고 분위기도 분위기에다가 효율도 높기 때문에 다른 지역보다 질좋은 조기교육이 가능하다.
\vspace{5mm}

서울 강남, 서초, 송파의 집값이 비싼 이유는 여러가지가 있지만
가장 중요한 건, '사는 사람들의 직업, 소득, 교육'이 남다르기 때문이다.
결국 직업과 소득도 교육이 좌우한다는 측면에서 보자면
사회에 관심있는 사람들은 별 의미없는 담론에 빠질 바에 과거 교육정책을 복기해보는 것이 훨씬 나을지도 모른다.
한국 최강의 첩보종족 강남 아줌마들은 진작에 이런 것을 알고 있었던지도 모른다.
그들이 진작에 은밀히 교육정책이나 교육시장에 영향을 주었고 아파트 재건축 등에 힘을 행사하며 부를 유지해온 것은 그야말로 연구대상감이다.
\vspace{5mm}

그렇다면 여기서 한가지 명쾌한 결론은 나온다.
미래에 어떤 직업이 좋을지는 불확실하지만, 분명한 건 '고급두뇌'가 되어야한다는 것만큼은 부정할 수 없다는 것이다.
그리고 돈을 버는 건 결국 저런 고급두뇌들을 상대하는 직업으로 보면 간단해진다(의사부터가 그렇다)
\vspace{5mm}

비싼 돈을 들여서 외모까지도 어느 정도는 바꿀 수 있다. 다만 바꿀 수 없는 건 '두뇌'와 '키'와 머리카락이다.
그렇다면 이제 초점은 한 인간의 두뇌를 어느 수준까지 달성시킬 수 있느냐하는 교육 시스템의 문제로 간다.
교육이 결국 장래의 직업, 소득, 그리고 지리까지 결정해버리는 키로 작용하기 때문이다.
특히나 앞으로 얻을 수 있는 가치는 결국 고급교육+미개척분야라고 한다면 이 분야로 미리 눈돌리는 게 매우 중요할 것이다.
\vspace{5mm}






\section{복부인들}
\href{https://www.kockoc.com/Apoc/629218}{2016.02.09}

\vspace{5mm}
\begin{itemize}
    \item[] "의사 좋아요?"
    \item[] "공무원 좋아요?"
    \item[] "앞으로 뭔 직업이 좋아요?"
\end{itemize}
\vspace{5mm}

좋은 직업을 택한다고 하더라도 어차피 본인이 공부를 안 하는데 소용이 있을까.
앞으로는 직업도 직업이지만, 이제 사람을 더욱 볼 건데?
의사나 변호사라도 실력있는 사람 찾는 것 아냐?
\vspace{5mm}

실력없는 의사라면 사람 죽이고 작살나겠고
지적능력없는 공무원이라면 일 엉터리로 하거나 민간에 휘둘리다가 망하겠지.
수험사이트에서 대학서열화나 의대 강조하는 이유가 정말 좋아서 그럴 것 같냐.
그런 걸로 '장사'해야하니까 그런 거지.
\vspace{5mm}

과거 50년간 대통령 빼고 진짜 승리자가 누굴까?
공대 박사? 의사? 변호사?
\vspace{5mm}

아니다, "강남 아줌마"들이다.
농담 하는 게 아니라 실제로 그렇다.
생각보다 저런 직업들이 돈을 그리 많이 벌지는 못 한다.
의사만 하더라도 사실 학업기간이나 로딩을 보면 그것에 비해 많이 받지는 못 한다.
그 사람들은 다 그만한 노력을 기울이고 그만큼 받는 것인데, 우리나라는 생각보다 세금을 많이 거두며 수입도 제한한다.
그러니 실제 재산 증식는 \textbf{'부동산 투기'}였고, 이건 아줌마들이 정말 맹활약한 것이다.
이 분들이 무서운 게 '폭락'할 거라는 경제전문가의 예측도 가볍게 무시하게 만들고 집값 유지를 넘어 상승을 유지한다는 것이다.
\vspace{5mm}

이게 도덕적으로 바람직한가 아닌가를 떠나서 눈여겨볼 건, 저 분들도 과연 공부를 했을까 안 했을까.
아마 가방끈은 생각보다는 짧을지도 모른다(라고 하지만 계층이 계층이니 그렇다고 보기는 어렵겠지만)
그러나 그 아줌마들끼리 정보교환하고 탐색첩보전하고 하는 것을 보면 그 정도 정성이면 차라리 본인들이 입시치는 게 나을 정도다.
뭉칠 때는 똘똘 뭉쳐서 자기 지역의 이권을 수호하고,
사회적으로 힘이 있는 남편들을 침대에서 사주해서 사실상 밤의 대통령(...)으로 활약하시고
거기다가 자녀들까지 고급교육을 시켜 계속 헤게모니를 유지해나간다.
건물주 금수저의 배경은 사실 저런 치열한 노력에 있다.
\vspace{5mm}

진부한 이야기지만 결국 공부하는 게 살아남는 것이다.
\vspace{5mm}

김광수나 선대인 같은 전문가들의 부동산 폭락론은 설날특선드라마도 아니고 10년넘게 지속되어 왔지만 실제로 실현되지 않았다.
왜 그런 걸까, 정량적인 분석은 그 방법론이나 데이터로는 틀린 건 없다.
그렇다면 정성적인 분석에 문제가 있다는 이야기이다.
이 분들의 문제는 부동산이 단순히 돈의 문제가 아니라
'인간'의 문제인 걸 까먹었다는 것,
그리고 인간은 대단히 불확실하며 예측을 벗어난 행동을 한다는 걸 모른다는 것도 있다.
(사실 이 분들의 예측을 보면 부동산이 뭔지 잘 모른다라는 지적이 많다. 부동산 전문가들의 이야기를 듣고 배운 게 많다)
\vspace{5mm}

하지만 결정적으로 이들이 놓친 건 아랫 글에도 적었지만 \textbf{교육격차}다.
국토의 진정한 의미 : 농경시대는 농지, 공업시대는 공장과 식민지, 정보시대에는 인터넷을 할 수 있는 누리꾼 이라는 말도 있지만
국가 차원에서 부는 '첨단산업'을 주도해나가는 \textbf{과학기술과 그 고급인재로부터} 나온다.
미국이 자국의 비밀을 유출해나갈 수 있는 외국인들에게 교육문호를 개방하고 장학금 지원해주는 것?
잘 해주면 어차피 알아서 미국으로 귀화할 것이고 그게 미국의 국력이 된다, 실제로 우리나라 고급두뇌들도 미국으로만 가버린다.
폭락론을 주장하는 사람들은 아파트, 토지, 이자율만 보지, \textbf{거기 사는 사람들의 '교육'을} 보지 않는다.
실제로 대한민국의 국부는 그런 교육동네가 좌우하고 있다, 그런데 여기 집값이 쉽게 떨어지겠는가.
(혹자는 교육과정 개편이 이를 작살내리라고 하지만 그다지. 어차피 저런 동네의 교육은 알아서 잘 돌아간다)
\vspace{5mm}

만약 진로 선택이 그 전공으로 진학한 다음 철밥통을 누리는 거라면 걍 망상을 버리는 게 좋다.
이제는 어느 직업이건 공부해야 한다.
역으로 말하면 남들보다 공부를 많이 하면 어떤 직업이건 운신의 폭이 넓어진다는 얘기다.
의치한도 생각만큼 대단하지 않을 수 있다라는 건, '공부하지 않은 직업인'에 한해서이다. 거기서 탑인 사람은 잘 나갈 것이다.
당연히 공대든 자연대든 인문사회대든 거기서 공부하는 사람은 살아남는다.
\textbf{'공부를 안 해서 실력이 없는 사람'들이 도태되는 것을 가지고 그 분야가 망한다라고 오버하는 게 현실 아닌가?}
\vspace{5mm}

가능하면 콕콕에는 \textbf{부자}들만 왔으면 좋겠다.
10대와 20대의 부자란 수억 소득이라거나 외제차라거나 아파트가 아니다.
부의 기준은 바로 \textbf{두뇌}다.
매일매일 푸는 국영수 문제나 읽는 책이 자산인 것이다.
실제 이건 문학적인 비유가 아니라 정말 냉정한 현실적인 지적이다.
공부를 하는 건 좋은 대학에 가기 위해서라기보다는, 미래를 위한 '자본'을 축적해나가는 것이다.
내가 1000문제를 풀었다면 10문제 푼 녀석보다 100배나 부유한 것이다.
\vspace{5mm}

나는 머리가 안 좋아, 공부해도 소용없어, 좋은 대학 가는 놈은 한정되어있어, 아 이 빌어먹을 운명... ?
이거야말로 삼류 업자들이나 공부 모략가에게 '세뇌', '선동'당한 결과가 아닌가?
\vspace{5mm}

노오력해도 소용없어, 이게 다 기득권 때문이야, 민중은 진보한다... ?
그냥 그런 사람들은 그렇게 놀도록 냅둬라. 망하려면 자기들이나 망하라고 하든가. 사실 실천도 안 하고 공부도 안 하는 잉여들이다.
정말 노오력하고 노오력 욕하는 사람들은 단 한번도 못 보았다. 정작 기득권층들이라고 하는 사람들도 노오력은 잘만 한다.
그리고 민중이 진보한다는 데 왜 그 사람들은 공부는 안 하고 유흥산업의 호구가 되어주시나.
\vspace{5mm}






\section{고소득층 교육비가 7.8배}
\href{https://www.kockoc.com/Apoc/629310}{2016.02.09}

\vspace{5mm}

\href{http://news.nate.com/view/20160209n04928?modit=1454984815}{링크}
\vspace{5mm}

저런 현상은 바람직하냐 바람직하지 않냐를 넘어서 왜 저런 일이 벌어지나 봐야한다.
이건 앞서 얘기한 '지의 격차'로도 설명되는데
\vspace{5mm}
\begin{itemize}
    \item \textbf{하류 : 돈만 본다}
    \item \textbf{상류 : 돈을 컨트롤할 수 있는 능력을 본다}
\end{itemize}
\vspace{5mm}

국가에서 돈은 뺏아갈 수 있다, 그러나 \textbf{능력을 뺏아갈 수는 없다}.
그리고 미래에는 \textbf{고급인재는 어느 나라건 환영한다}.
그 미래에도 '대한민국 만세'거리고 있을 리야 없지않나.
\vspace{5mm}

지금도 하류들은 수저론 T$\_$T 거리면서 대기업 법인세 빼액$\sim$ 복지 확대 꺄악$\sim$ 만 거린다.
그런데 이게 사실 도움이 되느냐 하면 안 그렇다는 게 문제지. 실제 '부'는 교육으로 이전되고 있구만.
\vspace{5mm}

다만 이런 질문을 던질 수 있을 것이다. 만약 저소득층이 구매력이 없어서라고 해도 7.8배는 너무한 것이 아니냐고.
그건 바꿔 말해야지. 현명한 선택을 상류와 하류 중 어디서 하겠나?
하류들은 그저 국내에만 시선이 머물러있어서 복지확대나 부의 재분배만 주장할 것이다.
하지만 상류들은 이미 세계 전체를 바라보고 있고 국경을 넘는 능력의 확충을 목표로 한다면 교육비 지출이 늘어날 수 밖에 없다.
게다가 저건 수험사이트에서 말하는 쉬운 수능 비판과는 거리가 있어보이기도 한다.
\vspace{5mm}

더군다나. 저출산까지 감안하면 사실 한 사람에게 들어가는 교육비 격차는 훨씬 더 커진 것이다.
\vspace{5mm}

참조할 수 있는 모형이 중국 화교들일 것이다.
화교들은 원래 국가를 안 믿는다. 그들의 기원부터가 정부의 탄압을 피해 본토에서 탈출한 것.
특정 국가에서 가서 유대인과 같은 짓을 한다. 동향들끼리 밀어주고 끌어주고 그 지역의 경제를 손에 쥔다, 단 나머지는 그리 욕심내지 않는다.
쓸데없는 낭비는 안 한다. 그러나 자녀 교육 투자만큼은 정말 천문학적 수준으로 한다.
\vspace{5mm}

지금이야 대중들은 낙수효과 어디있냐 대기업이 부를 분배해야 한다 복지를 늘려야한다고만 소리치겠지만
저렇게 해서 능력치가 넘사벽이 되어버리면 그 때에는 어떻게 되는 걸까.
사실 이거야말로 답이 안 나오는 문제다.
\vspace{5mm}

저거의 해결책은 공교육 강화 $-$ 즉 학교도 결국 학원화시키는 것, 신자유주의 개혁을 하는 것이다.
그러나 이걸 하는 순간 교육이 무슨 산업이냐 입시냐하면서 여론의 질타를 받게 된다. 물론 그걸 주창하는 사람들은 자기 자녀는 좋은 교육을 시키지.
사실상 계층이동은 이제 맥이 끊어지는 게 아닐까.
\vspace{5mm}

'행복은 성적순이 아니잔하요'라는 영화가 1989년에 히트친 적이 있었다.
성적을 비관한 여학생이 자살한다는 충격적 엔딩이었다고 하지만... 요즘 이런 게 개봉되면 뭔 한물간 소리하냐는 평을 들을 것이다.
실제로 행복은 성적순이 되어버렸다는 게 불편한 진실이기 때문이다.
지금은 성적이라도 매겨주세요... 라고 애원하는 현실이다.
사람들이 응사 같은 드라마를 보는 건 재미있어도 그러겠지만, 그 시대가 지금보다 살기 좋았다라는 미화된 추억도 한몫한다.
\vspace{5mm}

그러나 그런 고성장 시기가 다시 돌아올 일은 없어보인다.
적어도 저 시기에는 소위 재벌이나 졸부(강남 개발로 앉은자리에서 떼돈 번 사람들)들은 자녀들이 개차반이다 망나니다 하는 얘기가 있었고
특히 그런 졸부 자식들이 '오렌지족'이라고 해서 문제가 되었다.
사람들은 그걸로 혀를 쯧쯧 차면서 말세라고 소리쳤는데...
\vspace{5mm}

생각해보면 저건 희망적(?)인 것이다. 부유층 자제들이 공부를 안 하니 부를 대물림하기 힘들어지고 거기다가 소비까지 증진시키지 않나.
그런데 지금은 있는 놈들이 더 한다는 의미가, 가진 자들도 자녀들을 철저히 \textbf{교육시킨다}.
\vspace{5mm}

사실 이 글을 읽는 사람들이 사회정의나 형평성을 고민할 위치는 아닐 것이다.
결국 교육해서 올라가는 저 라인에 끼느냐 못 끼느냐 그게 중요한 것일 뿐.
\vspace{5mm}






\section{특혜}
\href{https://www.kockoc.com/Apoc/630168}{2016.02.10}

\vspace{5mm}

대한민국은 감성팔이가 통하는 곳이라고 생각하면 되겠습니다.
긍정적인 건은 있죠. 요구하라 얻어낼 것이다.
부정적인 것도 있죠, 하지만 브레이크는 없다.
\vspace{5mm}

역사의 진실을 밝힌다거나 정의를 추구한다면 '부당한 특혜'를 누리지 않는 걸로 실천해야하는데
우리나라에서는 신기하게도 메시지는 메시지고 실천은 상관없다라는 논법을 써먹는 사람들이 많죠.
실천이 담보되지 않는 구호는 선동이 되고, 선동은 선량한 사람들에게 결국 피해를 줍니다.
\vspace{5mm}

특히 정치가 그런데, 이건 삶의 문제로 직접 느껴보아야하는데, 그 이전에 메시지로만 접한 사람들이 많아서리.
그래서 명문대 진학해서 증세와 복지 정책 지지하다가 자기가 대기업 입사한 뒤 세금 뜯기는 것에 빡쳐서 변절해버리죠.
북유럽식 복지 해야하니 부자증세해야한다고 하는데 자기가 그 대상에 해당한다는 걸 아는 순간 소신이고 나발이고 없는 거예요.
\vspace{5mm}

결과적으로는 선량한 시민들만 손해를 봅니다.
과연 그런 특혜까지 가는 걸 알았어도 그들을 응원해주었을 것인가... 라는 건 다소 회의적입니다.
동원되는 논거가 '어차피 대학경쟁은 불공정하다'라거나 (아니 그럼 대학을 없애버리든가 그럼)
국가가 책임을 져야한다(마찬가지로 국가는 과한 보상이나 부조리한 것도 책임져야하는데?)는 논거인데
엄밀한 검토 없는 정치적 의사란 늘 이렇게 부조리함을 낳습니다.
\vspace{5mm}

이런 걸 일찍 깨닫는 게 좋을 건데 말입니다.
약자를 자칭하는 사람들이 실제로는 더할나위없이 강자들이고,
반면 강자라는 사람들이 내놓는 정책이 실제로는 약자를 배려하는 면도 없지 않다는 것.
\vspace{5mm}

결과적으로는 광우병 집회 이후 퍼졌던 냉소주의와 같은 결과를 낳아버리겠네요.
지인들은 그 당시 우리가 승리한다(...)라고 했지만 저는 '기껏 뜻있는 사람이 일궈놓은 진보운동이 이제 다 망했다'라고 생각했고
실제로도 그렇게 되었습니다. 그렇게 판단한 이유는 간단, "근거"도 없거니와 "일관성"도 확보하지 못 했기 때문이죠.
취지는 좋지 않았냐... 이건 아무 소용이 없습니다. 사실 광우병은 한우도 자유롭지 않은 문제였고(한우가 오히려 미덥지 않은 게 많았죠)
처음에는 국민 건강을 강조하더니 급기야는 청와대로 진격하자 해버렸으니(...)
\vspace{5mm}

더군다나 이해찬 세대와 더불어 그 때의 10대들이 지금 사상 최악의 취업난을 겪고있죠.
아마 성향이 안 바뀌었다면 작년에 광화문에 나갔겠지만 그런 건 없습니다. 다 자기 앞가림도 바쁘거든요
자기들보고 사회를 바꾸라고 하던 어른들이 사적으로 잘 먹고 잘 사는 걸 보고 배신감을 느꼈을지도 모르겠고
특히 대입보다 더 부조리하고 '잘 태어났냐'까지 확인하는 취업시장에서 물먹고 후회하고 있을 겁니다.
\vspace{5mm}

결국 꼰대충고가 이 점에서는 맞습니다. \textbf{"다른 생각하지 말고 공부나 하세요"}
그 다른 생각이 본인에게 도움이 되는 경우가 별로 없기 때문입니다.
물론 제가 싫은 사람이나 그 자녀에게는 이렇게 말하겠죠. \textbf{"지금 공부할 때입니까? 나가서 사회를 바로 잡아야지"}
\vspace{5mm}






\section{혁신}
\href{https://www.kockoc.com/Apoc/630187}{2016.02.10}

\vspace{5mm}

\textbf{기술은 기하급수적으로 진보하는데}
\textbf{정부는  1차함수적으로만 움직이고 있으니} 따라잡을 수가 없죠.
\vspace{5mm}

의치한교대가 몰리는 이유 중 하나만 불편한 걸 제시하면 저기는 상당히 보수적이기 때문입니다.
물론 그것만이 전부라고 할 수는 없는데, 과연 현재의 기득권이 10년 뒤에도 유지될지는 회의적입니다.
\vspace{5mm}

원격진료는 계속 반대에 부딪치고 있으며
교육기관들은 사실 가장 보수적입니다. 아이들과 연관되어있어서 건드리기 까다롭습니다.
\vspace{5mm}

그 이야기는 바꿔 말해 다른 곳의 취업이 힘든 건 경기도 그렇지만, 역설적으로 다른 분야는 '진보'했기 때문입니다.
담배 피고 노가리까기나 하는 화이트칼라들이 컴퓨터로 대체됩니다.
정말로 이윤창출을 하는 영업이라거나, 상품과 서비스를 기획하고 창조하는 기획연구개발이라거나 이게 아니면 쓸모없어진 건데
이건 거꾸로 말해서 취업이 잘 안 된다 $\rightarrow$ 시스템 진보에 성공했다 $\rightarrow$ 앞으로도 살아남을 것이며 리스크가 적을 것이다
라고 얘기할 수 있을 것입니다.
\vspace{5mm}

그럼 여기서 질문던져봅시다. 그럼 시대 진보를 따라가지 못 하는 분야에서 언제까지 자기 밥그릇을 수호만 할 수 있을까.
그들이 이용할 수 있는 건 어디까지나 제도적인 것입니다. 그런데 그 제도적인 것으로도 커버 안 되는 \textbf{혁신적인 게 등장하면 애매해집니다.}
예를 들자면 p2p, 토렌트, 웹하드로 저작권도 무용지물화되었지만 방송국은 헤게모니를 잃었습니다(거기다가 종편 등장도 한몫하겠지만)
저작권을 수호하려고 해도 막을 수가 없습니다. 거기다가 아프리카 1인 방송 체제가 생각보다 쏠쏠합니다
인터넷이 들어오기 전까지는 이런 것을 생각하는 사람이 없었겠고, 지상파 방송사만 가면 된다고 생각하던 사람도 많았을 겁니다.
그리고 그렇게 노력해서 들어왔는데 그 헤게모니가 붕괴되는 것을 보고 망연자실한 사람도 있겠죠.
하지마 이런 건 미래예측서에서 얘기된 것입니다. 당시에는 터무니없이 느껴졌지만요
여담이지만 연예기획사들은 그런 사소한(?) 저작권에 신경쓰기보다는
세계적으로 규모를 넓혀서 공연료, 광고비 등을 받는 전략으로 바꿔나갔고 그래서 살아남은 것이지요.
\vspace{5mm}

이미 정착된 키워드가 \textbf{1인 기업}입니다. 시대가 이미 그렇게 바뀌어나가고 있습니다.
아주 거대해지거나, 아니면 아주 미소해지거나. 즉, 전세계를 아우르는 다국적 법인화되거나, 혹은 뛰어난 초개인으로 영업하거나.
그런데 이걸 대학에서까지도 가르치는지는 모르겠습니다. 그리고 정부도 이에 대처 못 하고 있죠.
거기다가 아프리카 1인 방송, 1인 기업에 더불어서 쏠쏠이 이문을 남긴 게 \textbf{'직구'}라고 알고 있습니다.
이 역시 처음에 정부에서는 어떻게 할 방도가 없고, 국내 호객들을 갖고놀던 대기업은 할 말이 없으니 말돌리기 시전했죠.
\vspace{5mm}

특정업계가 잘 나가는 것을 '과거'와 '현재'만 바라보고 그걸 자기가 누리겠다라고 할 때 주의하시길.
결국 혁신은 하게 됩니다. 그 업계가 자발적으로 혁신해서 사람을 잘라버리거나, 아니면 \textbf{혁신당해서} 사람을 잘라버리거나
물론 이걸 특정업종이 잘 나간다 어쩐다 이런 걸로 단순화시키면 안 됩니다. 공부하는 사람은 살아남겠죠, 다만 철밥통은 없습니다.
예컨대 문과를 예로 들자면 대학교 문과들이 그렇게 된 건 자기들이 혁신을 안 했기 때문입니다.
조기에 구조조정하고 과 통폐합으로 인원수 조절하고 고급과정은 대학원 과정으로 바꾸고 경쟁을 강조했으면 이 지경까지 안 왔습니다.
그래서 지금 혁신당하고 있는 것이죠. 하지만 문과에서도 초엘리트들은 잘 나갑니다.
\vspace{5mm}

+
회사가 사람을 자르는 건 비극으로 인식됩니다만, 주식시장에서는 정반대로 나타납니다.
구조조정을 원활히 하는 곳이 주가가 유지 혹은 상승합니다. 그래야 회사의 미래가 밝으니까요.
정반대로 인력을 뽑아놓고 방만하게 운영하는 곳은 썩게 되어있습니다.
\vspace{5mm}






\section{교육비 격차 : 가축에게도 밥은 먹인다.}
\href{https://www.kockoc.com/Apoc/630883}{2016.02.11}

\vspace{5mm}

대략 2000년대 초반에는 약 3배 정도였던 게
\vspace{5mm}

\href{http://www.eduict.org/$\_$new3/?c=3/51&p=2&uid=22038}{링크}
\vspace{5mm}

극빈층의 교육비 비중은 3.97$\%$에 불과해 최고 부유층과의 격차가 \textbf{2.65}배에 달했다. 이같은 격차는 2·4분기 기준으로 1998년 외환위기 이후 가장 크게 벌어진 것이다.
\vspace{5mm}

현재는
\vspace{5mm}

\href{http://news.naver.com/main/read.nhn?mode=LSD&mid=sec&sid1=102&oid=020&aid=0002943121}{링크}
\vspace{5mm}

계층 간 교육비 격차는 점차 확대되는 추세다. 연간 기준으로 2010년 소득 5분위는 월평균 교육비로 1분위의 6.3배, 2011년에는 6.1배를 쓰는 것으로 집계됐다. 하지만 2012년(6.5배)부터 2013년(6.6배), 2014년(\textbf{7.9배})으로 갈수록 격차가 벌어지고 있다.
\vspace{5mm}

저 격차가 좁혀질 일은 이제는 요원해보인다.
10년 전에 약 3배, 지금이 8배라고 한다면, 그 다음 10년 뒤에는 어림잡아 15배가지도 치달을 수도 있다.
그렇다면 이건 이미 교육포기를 얘기하는 게 아닐까.
\vspace{5mm}

교육격차야말로 가장 중요한 것이다.
선진국과 개도국의 넘사벽 차이도 이로써 설명된다.
\vspace{5mm}

\href{http://news.naver.com/main/read.nhn?mode=LSD&mid=sec&sid1=104&oid=005&aid=0000766838}{링크}
\vspace{5mm}

. 선진국과 개발도상국의 교육 격차는 무려 100년에 달했다.브루킹스 산하 ‘보편교육센터’를 책임지고 있는 레베카 윈스롭 선임연구원이 29일(현지시간) 영국 BBC방송 인터넷판에 기고한 보고서에 따르면 미국과 유럽 등 선진국의 경우 누구나 일정 수준의 보편교육을 받아야 한다는 인식이 널리 확산된 게 산업혁명 이후 번영기인 19세기 중반이다. 반면 개발도상국의 경우 1948년 유엔인권헌장이 발표된 뒤에야 그런 인식이 퍼지기 시작했고 현재와 같은 학교교육이 이뤄지게 됐다.단지 학교교육이 본격화된 시점만 100년의 차이가 있는 게 아니었다. 학교에 들어가서 이수받는 교육 기간을 따지면 문제는 더 심각했다. 선진국으로 분류된 나라의 성인들을 조사해봤더니, 학교교육 이수 기간이 평균 12년이었다. 하지만 개발도상국은 6.5년으로 절반 정도에 그쳤다.그럼 이 격차가 줄어들려면 얼마나 시간이 걸리는 것일까. 레베카 선임연구원은 개발도상국의 6.5년의 교육 이수 기간이 12년으로 늘어나는 데에만 65년이 걸릴 것으로 예측됐다고 밝혔다. 또 교육 이수 기간이 4.5년인 후진국의 경우 85년이나 필요하다고 덧붙였다.그렇다고 그 사이에 선진국이 가만히 있을 리가 없다. 2100년에는 선진국의 평균 학교교육 이수 기간이 14년을 넘어서는 반면 개발도상국은 12년, 후진국은 11년을 넘는데 그쳐 여전히 격차가 존재할 것이라고 했다. 보고서는 교육 기간과 함께 교육의 질도 문제라면서 “현재 선진국 학교에서 가르치는 교육내용과 수준을 개발도상국이 따라잡으려면 역시 100년 이상의 시간이 걸릴 수밖에 없다”고 분석했다.
\vspace{5mm}

중동이나 아프리카에 벌어지는 내전은 물론 방치되는 것도 있지만 간접적으로는 조장되는 것도 있지 않을까.
그런 내전이 만약 '교육'으로 이어진다면 더 이상 유럽, 미국, 동아시아는 저 지역의 자원을 착취할 수 없다.
\vspace{5mm}

다시 국내로 돌아와도 그렇다.
당신이 상류층이면 하류에게 밥을 먹이겠나, 책을 읽히겠나.
당연히 밥을 먹인다. 온갖 휴머니즘적인 배려로 $-$ 어차피 \textbf{가축도 밥은 먹여야하니까.}
한 때 무상급식이 정의인양 포장되었다,
물론 그걸 추진하신 분들의 자녀교육은 절대 평범한 게 아닌 걸로 안다.
남의 자식들에게는 밖에 나가 뛰어놀라고 하고 경쟁을 줄여야 한다고 얘기하지만
자기 자녀들은 특목고에다가 좋은 대학을 나와야하는 이중성.
\textbf{물론 밥먹이는 것은 중요한 문제다, 그러나 가장 심각한 교육격차는 다들 눈을 감았다, 이런 게 세상인 것이다.}
\vspace{5mm}

하류들은 지금 복지가 확충되어야한다고 주장한다, 물론 그건 실현될 것이다. 하류들 자신의 빚으로
어차피 상류들은 겉으로는 싫은 표정을 지으면서 자기들이 돈 낸다 생색내지만, 그건 다시 \textbf{벌어들일 수 있다.}
포퓰리스트들은 증세 없이 그냥 고소득자와 대기업이 다 책임지면 된다고 '말'로만 이야기한다.
물론 언젠가는 그렇게 될지도 모른다, 지금과 같은 교육격차면
고소득자와 대기업은 그렇게 복지를 책임지는 대신 \textbf{더 많이 벌어들일 테니까.}
\vspace{5mm}

사실 그 포퓰리스트들의 진정성도 의심이 간다.
그들이 서민들의 교육이나 실력 양성에 대해 제대로 신경쓴 적은 없다.
특히 교육에 있어서는 과정을 쉽게 내야한다면서 주장하는 게 하향평준화이다.
그들이 활약하면서 살기가 좋아졌다면 모르겠다.
혹자 이 글을 읽는 사람들은 나에게 발끈할 지도 모른다.
하지만 내가 경험하고 관찰한 바, 그런 이들이 활약할수록 격차는 계속 커지기만 했다.
그리고 그게 10년 넘게 지속된 현상이다.
그리고 그들 중 명망가는 정계에 투신한다.
\vspace{5mm}

심지어 몇몇 매체에서는 토익과 공부가 아니라 거리로 나가라는 무책임한 소리를.
자기들 기자 뽑을 때 그래서 학벌, 스펙, 영어점수를 안 보시던가?
기사 쓸 때에는 서민들을 위한다 하지만 정작 인재채용에서부터 경영은 정반대로 하면서 뭔 이야기하시던지.
\vspace{5mm}

구원받을 수 있는 건 \textbf{스스로 하는 공부 밖에} 없다.
공부를 하려면 대학도 대학이지만 저런 상류들에게 말빨이나 머리빨리 지지 않는 수준까지는 달성해야 한다.
도서관에 가서 공부하라는 건 단지 공부장소만의 문제가 아니다, 거기서 책을 읽으라는 이야기이다.
살아남는 비결이 거기 도서관에 있다, 단지 우리가 그것들을 안 읽을 뿐이다.
그저 게시판에 올라오는 말초적인 화제에다가 뉴스 댓글만 보고 추종, 따라하면서 남의 호구나 되는 삶을 살 것인가.
\vspace{5mm}






\section{수학 과학 과잉}
\href{https://www.kockoc.com/Apoc/631442}{2016.02.11}

\vspace{5mm}

이게 정말 쓸모가 있는지는 의문입니다.
전문적 지식이 아닌 '사고하는' 방식이라면 다양한 철학을 익히고 문제해결적 태도를 익히는 건데
지금의 수능 수학이나 과학은 조선시대 과거제도처럼 뭔가 이상하게 바뀌어나가고 있다는 것입니다.
\vspace{5mm}

솔직히 수학은 정말 현실의 문제해결을 수리적으로 해결해보는 경험을 해보지 않으면 별 쓸모 없습니다.
거꾸로 말해서 그런 경험을 하나 제대로 해보는 게 비싼 인강보다 낫다는 것이고,
과학도 실험을 제대로 하고 보고서를 쓰지 않은 것은 그냥 죽은 사람들의 똥만 먹는 행태일 터인데
\vspace{5mm}

개인적인 인상입니다만 (물론 근거는 꽤 많이 찾아질 겁니다. 이게 참 뚜렷해지고들 있어서)
현재 토끼띠 이하부터는 수학과 과학이 문제가 아니라 '국어'와 '인문사회' 교양 결핍이 아마 대두될 겁니다.
해리포터나 나니아 연대기를 원서로 읽히지만 고전 소설이나 세계 명작, 그리고 삼국지를 읽는 애들이 생각보다 없습니다.
똑똑하다고 자부하는 친구들도 탐문해보면 이런저런 교양은 없구나... 라는 판단인데
\vspace{5mm}

이게 실제 스펙으로도 적지않은 영향을 미칠 것입니다.
그 이야기는 지금 10대들이 나중에 차별화되고 싶으면 독서를 많이 하는 게 좋다는 뻔한 이야기입니다만.
그렇게 하기만 하더라도 엄청나게 많은 차별화가 가능할 겁니다.
남들이 영어에 빠져있다면 몰래 한자와 한문을 익히고,
남들이 수학과학에만 빠져있다면 온갖 역사와 고문(古文)에 통달하고
사실 이렇게 해서 '독특한' 인재를 완성해나가는 것이 중요하다고 생각, 실제로 이런 친구들이 잘 나갑니다.
\vspace{5mm}

스마트폰과 가상현실의 증가로 기존 공동체가 해체되고 개인들이 외로워질거라고 했습니다만.
역설적으로 지금은 개인이 쏟아내야 할 텍스트의 양이 많아졌습니다.
외모와 체형이 중요하다고 하지만 실제로는 '문자메시지'로 소통하는 경우가 많아졌습니다.
이런 경우라면 다양한 언어와 문자, 그리고 육감적인 표현력에다가 재밌는 이야기들을 하는 사람의 매력이 높아지는 것이죠.
종이썩는 냄새가 풀풀 나는 헌책방 애용자들이 유리해지는데 아무도 이런 생각을 안 하더군요.
\vspace{5mm}






\section{강남강북격차}
\href{https://www.kockoc.com/Apoc/631601}{2016.02.11}

\vspace{5mm}

며칠 전 의새 친구가 해준 이야기.
\vspace{5mm}
\begin{itemize}
    \item[] xx대 병원에서 일할 때는 : 주로 오는 사람들이 강남 노인네들인데 말만 노인네지 의욕적이어서 이것저것 적극적으로 치료받으려 한다.
    \item[] 현재 xxx 병원에서는        : 사람들이 순박하고 착하고 어리석다, 암수술 보험되는 것도 있는데 그것도 모르고 암이라고 하면 죽으려고 한다(...)
\end{itemize}
\vspace{5mm}

사실 이야기하다보면 기승전'부동산'되는 게 기성세대(...)의 성숙한(...) 대화가 아닌가 하지만
어떤 화제건 '격차'는 참 공통주제다.
물론 이건 일종의 쾌락을 준다, 사파리 동물원의 즐거움과 같다, 밖은 위험하지만 나는 안전한 데에 있다는 느낌?
적어도 상류에 속하는 사람들로서는 격차 이야기는 즐거울 수도 있는 것이다.
\vspace{5mm}

강남강북 격차가 뚜렷이 나타난 게 아마 2000년대 중반이었을 것이다.
그 이야기는 다시 말해 그 전까지는 단지 '돈많은 졸부들, 천박한 것들'이라는 인식만 있었지 실제 그런 격차는 적었단 이야기다.
아이러니컬하지만 그 강남 사람들이야말로 $-$ 비록 몇가지 범죄나 불성실한 걸로 연루되더라도 $-$ 평균적으로는 근면 성실 의욕적인 건 사실이다.
\textbf{부자인데 잘 생기고 키도 크고 예의바르고 인격자이기조차 해서 열받는다....} 라는 말이 헛말이 아닌 것이다.
일부 정의론자들은 그럴 리 없다고 하지만 양쪽을 경험해보면 확실히 느낀다(물론 부자들에게는 그들 특유의 위선이라는 게 분명히 있다)
\vspace{5mm}

2000년대 중반부터 10년동안 격차도 벌어졌지만 동시에 '경쟁'의 기회도 줄어들었다.
사람들이 공무원 시험에 몰리는 게 철밥통 때문이기만은 아니다.
사실 수능과 공무원 시험 빼고는 이제는 \textbf{경쟁의 기회가 사실 없다}.
언론에서는 공무원 시험 열풍이 망국적이다라고 참 속편하게 개탄하지만, 그럼 다른 경쟁의 기회나 패자부활전은 존재하는가?
빈부격차를 걱정하는 기자나 아나운서조차도 실제로는 '상류'들이다.
방송에서 가난한 아이나 애들 보고 눈물짓는(?) 여자 연예인들도 연상의 사업가 만나 귀족처럼 살며 육아예능을 찍으려 한다.
\vspace{5mm}

물론 심증 뿐이긴 하고 물증은 없다. 그런데 이 사회에서 능력을 테스트한다는 게 참 상류들에게 유리하게 바뀌고 있다.
온리 필기시험이 문제가 많을 수는 있다, 그러나 필기시험의 대안들이라는 것들이 '부모 잘 만난 애들에게 유리하다'라는 진실을 왜 말을 못 하나.
그 이야기는 과거처럼 필기시험의 비중이 컸다면 패자부활이 가능했던 사람들이 지금은 제도에 막혀 좌절한다는 이야기이기도 하다.
\vspace{5mm}

그런데 지식인들은 단지 '혐오'의 폭력성만을 문제 삼는다.
그 지식인이 어디서 살며 무엇을 먹으며 자식교육은 어떻게 시키냐가 더 중요한 문제가 아닐까 하지만.
저런 격차에서 발생하는 혐오가 어째서 '자기혐오성'에 가깝냐는 것에 대한 분석은 없다.
보통 혐오하는 사람들은 지금 2$\sim$30대들인데 이들은 2000년대부터 정말 여기저기 휘둘리면서 사실 얻어맞은 세대들이다.
격차사회가 되어간다는 것을 알았으면 부지런히 공부했을 사람들이 부조리한 제도나 여러가지 선동 등에 참 많은 걸 낭비했다.
그러니 이는 자기 환멸로 이어질 수 밖에 없는 거다. \textbf{"내가 괜히 이상한 꾐에 빠져서... 걍 실속있는 공부나 하거나 재태크나 할 것을 T$\_$T"}
그런데 진짜 기성세대는 저렇게 젊은 세대들을 속이거나 이용해먹고 또 다시 거리로 나가라고 하니까 이젠 먹히지 않는 것이다.
\vspace{5mm}






\section{결혼 격차}
\href{https://www.kockoc.com/Apoc/633064}{2016.02.12}

\vspace{5mm}

교육격차 말고도 간과하고 있는 것은 결혼격차.
단순히 남녀가 만나는 수준이 아니다, 자녀등급까지도 직결된다.
고스펙과 저스펙이 만나는 일은 드물다, \textbf{고스펙은 고스펙끼리 어울리게 되어있다}.
남자의 고스펙과 여자의 고스펙이 어우러지면서 빈부격차의 효과는 더욱 더 벌어져버린다.
\vspace{5mm}

결혼을 꺼리는 이유가 국가의 복지가 부족해서는 아니다. 사실 육아나 결혼을 복지 탓을 하는 건 '베이비붐 세대'를 설명할 수 없다.
그보다는 결혼은 '신분과 계급'을 결정하는 \textbf{일생일대의 도박}이라는 게 중요하다.
결혼 전에도 할 것은 다 하는 그런 시대라면 결혼이 지닌 의미는 결국 '신분'을 결정하는 것이다.
이건 여성 뿐만 아니라 남성도 신경을 쓴다, 자신과 자녀의 계급을 결정하기 위해서는 \textbf{고스펙 배우자를 만나}야한다.
한데 그런 고스펙 배우자는 드물다. 저스펙 배우자를 만나기 보다는 그냥 혼자 사는 게 낫다고 판단한다.
\vspace{5mm}

교육비 격차도 여기서 발생한다.
부부 양쪽 전문직이라고 하자, 양쪽이 월 \textbf{1}000을 번다면 월 2000만원 소득이므로 교육비의 여유가 생긴다.
하지만 남편이 일용직이고 아내는 살림한다면 많이 잡아도 월 300만원이다, 교육 투자는 꿈도 굴 수가 없다.
아울러 이런 빈부격차로 인해 지역간 격차도 커져버리고 이것이 현재의 미친 부동산으로 나타난다.
이것이야말로 피부로 느끼는 삶의 문제지만 언론이나 지식인들이 이런 실태를 고발하든가?
사실 이건 거의 언급하지 않는다. 불편한 진실이기 때문이다.
그 이전에 방송인과 기자들도 결국 '상류'다(방송인이나 기자가 '서민'이라고 보는가)
이런 넘사벽 격차 앞에서 '죽창', '헬조선'이란 말이 나오는데 그래서 어쩌란 말인가.
근원은 여러가지이겠지만 확실한 건 "교육의 차이"가 가장 결정적이라는 것이다.
\vspace{5mm}

화제를 바꿔서 그럼 베이비붐 세대는 어떻게 설명될 수 있을까?
복지는 커녕 먹고 살기 힘들었는데도 출산율이 폭증했다. 단지 피임율이 낮거나 아니면 별 다른 방도가 없어서일까?
그 때는 한국전쟁 이후였다. 상류층이 없는 건 아니었지만 극소수였고 사실 대부분이 가난했다.
그렇기 때문에 결혼으로 신분과 계급이 결정된다고 할 수가 없었다.
어떤 배우자와 결혼하느냐가 결정적이지는 않았단 이야기다. 더군다나 여성들의 사회진출이 적었다.
\vspace{5mm}

그렇다면 해결책은 격차를 줄이는 수 밖에 없다라고 설명될 수 있다.
그 격차를 줄이는 건 절대 하향평준화가 아니다, '상향평준화'였다.
반대를 무릎쓰더라도 수학과 영어 교육을 강화하고 아울러 하류층들의 입시 실력을 높이는 게 방법이다.
아울러 대학들도 진작 구조조정을 하고 보다 과통폐합과 더불어 현실적인 교육을 실시해야 하였다.
그러나 이런 것들이 소위 '진보'라는 미명 하에 지체되거나 정지당했고, 그 대가를 지금 20대들이 치르고 있다(...)






\section{쉬운 수능이 격차를 더 벌였다.}
\href{https://www.kockoc.com/Apoc/633865}{2016.02.13}

\vspace{5mm}

\href{http://news.naver.com/main/read.nhn?mode=LSD&mid=sec&sid1=102&oid=023&aid=0003139924}{링크}
\vspace{5mm}

본지가 최근 10년간(2005$\sim$2015학년도) 수능 성적 원자료를 분석한 결과, 상대적으로 쉬운 수능을 쳤든 어려운 수능을 쳤든 고득점이 많이 나온 고교 순위는 큰 차이가 없는 것으로 드러났다. 문·이과 수능 상위 각 5000등 이내(서울 최상위권대 합격 가능 수준)에 드는 학생을 가장 많이 낸 상위 학교를 쉬웠던 수능과 어려웠던 수능으로 나눠 분석했다.그 결과, 비교적 쉬운 수능이었던 2007학년도 수능(국·영·수 평균만점자 비율0.85$\%$)과 어려웠던 2011학년도 수능(만점자 0.21$\%$)에서 상위 5000등 이내 수험생을 많이 배출한 고교는 2007학년도 수능에선 상위 10개교 모두 외고 또는 자사고였고, 2011학년도에는 10개교 중 9곳이 외고·자사고, 1개가 일반고였다. 상위 20개교로 보더라도 2007학년도 수능에선 일반고가 6개, 2011학년도엔 일반고 5개로 큰 차이가 없었다.
\vspace{5mm}

\href{http://news.naver.com/main/read.nhn?mode=LSD&mid=sec&sid1=110&oid=081&aid=0002679818}{링크}
\vspace{5mm}

불수능 반대입장
\vspace{5mm}

흔히 문제가 쉬우면 작은 실수 하나에 등급이 갈라진다는 이유를 드는데, 문항의 난이도와 실수 여부가 서로 상관관계에 있다는 근거는 없다. 또 시험의 변별도가 낮으면 대학의 학생 선발 과정에 어려움이 있다는 말도 한다. 하지만 수능이 대학 입학 사정의 유일하고도 절대적인 기준은 아니다. 대학은 수능뿐 아니라 내신 등급, 비교과 활동, 면접, 논술시험 등 다양한 기준을 활용하고 있다. 선발 기준을 다양화하면 수험생의 창의적 소양을 도출하는 데도 훨씬 유익하다. ... 비교육적 평가다. 만점자 비율이나 1등급 컷 등 최상위권에 초점을 맞추어 시험의 난이도를 판정하려는 태도는 교육적으로 적절하지 못하다. 특히 만점자는 거의 예외적인 사례에 속하는데, 이 기준으로 시험의 난이도를 해석하는 것은 전체 시험의 난이도를 정확하게 평가하는 잣대가 될 수 없다. .... 수능 출제에서 가장 핵심적이고 이상적인 덕목은 일관성과 안정성 유지다. 이 원칙이 지켜지는 한 수험생은 예측 가능하고 합리적인 대비책을 마련할 수 있다. 교육부와 한국교육과정평가원이 매년 초 해당 학년도 수능 출제의 기본 방향을 공지하면서, 큰 틀에서 전년도의 기조를 유지하는 이유도 바로 이 점을 중시하기 때문이다. 변별력 강화 혹은 대학 선발의 편의를 위해 지난 20여년간 지켜온 기조가 흔들려서는 안 된다. 그간 축적된 이 중요한 노하우를 가벼이 방기해 버릴 이유는 없다. .... 사교육비 조장 문제다. 수능의 난이도가 올라가면 변별력 논란은 어느 정도 해소될 수도 있겠지만, 그에 따른 학생들의 학습 부담과 사교육비는 과도하게 증가할 게 뻔하다. 한 번 시험의 고난도를 체감한 수험생이나 학부모라면 그 불안감에 비례해 사교육에 의존하려는 심리는 가일층 팽배해질 것이고, 공교육의 정상화는 더욱 요원해질 뿐이다.
\vspace{5mm}

\textbf{불수능 찬성입장}
시험이 어렵기만 하다고 변별력이 높아지는 것은 아니다. 쉬운 문제와 어려운 문제를 적절히 배합해야 변별력을 높일 수 있다. 이를 위해 필요한 것이 바로 평가 전문가다. 그러나 우리나라에서는 수능의 출제 경향과 난이도 등이 평가 전문가들의 의견보다는 교육부의 판단에 의해 좌지우지되는 것 같다. 이렇다 보니 거의 해마다 수능에 대한 논란과 항의 사태가 끊이지 않는다.여기서 문제가 되는 것은 수능에 대한 교육부의 인식이다. 언제부터인지 우리나라의 교육부는 “어려운 문제를 출제하면 사교육이 기승을 부릴 수 있다”는 강박관념에 사로잡혀 있는 듯하다. 이런 상황에서 교육부의 수장이 ‘쉬운 수능’을 고수하는 것은 지극히 당연해 보인다. 하기야 선행학습금지법이라는 전대미문의 해괴한 법이 제정되는 정치권의 수준을 고려할 때 교육부의 강박관념을 이해하지 못할 것은 아니다.
\vspace{5mm}

개인적으로는 불수능 쪽이 더 논리적이라고 보는지라 사실 저기서 불수능을 반대하는 분의 논거는 몇가지가 뒤집혔죠.
\vspace{5mm}
\begin{itemize}
    \item 첫째, 수능 이외의 평가항목들이야말로 부모님 빨을 많이 받습니다. 인성, 면접, 스펙... 이런 것만큼 부모님 능력 테스트죠.
    \item 둘째, 만점자 비율에 대한 답변은 그냥 회피로 보입니다. 여기에 대해서는 교육부가 걍 무책임으로 나선 거죠
    \item 셋째, 일관성과 안정성 유지는 매년 난이도에 대해 거짓말한 것과 다름없는 교육부에서 할 수 있는 주장이 아니죠.
\end{itemize}
\vspace{5mm}

사실 사교육비 이야기도 이제는 한물간 이야기입니다.
그럼 EBS에서 거금으로 잘 나가는 강사 스카웃해서 고급과정 신설해도 되는 겁니다(이게 그리 어려운 건지)
사실은 지금 사교육 과잉이 문제가 아닙니다, \textbf{사교육의 질이 떨어진다}는 게 더 심각한 문제죠
잘 가르친다라는 것이 과연 입담이 좋고 농담을 잘 하는 건지, 아니면 정말 근본적으로 생각하는 법을 가르쳐주는 건지가 관건인데
잘 나간다는 강의들이 사실 후자보다는 전자에 가깝습니다. 그래서 온갖 알바공작질이 성행하는 것이지요.
\vspace{5mm}

난이도 하향은 지금처럼 공부할 게 늘어나는 시대에 적응하지 못 한, 이른바 관료주의적 편의라고 볼 수 밖에 없지요.
\vspace{5mm}

게다가 장기적으로는 오히려 격차를 벌였습니다.
수능 쉽게 낸다고 상위권들이 공부를 안 할리는 없지요.
반면 중하위권들은 그런 하향평준화 기준에 맞춘 공부를 하니 대학 과정까지 감안하면 그냥 내리막길입니다.
\vspace{5mm}

결국 이건 대안이 있느냐 없느냐가 관건인데 EBS라는 훌륭한 대안이 있음에도 이를 방기하고 있으니.
\vspace{5mm}

이 사태의 본질은 다음에서 노골적으로 드러나지 않습니까.
\vspace{5mm}

\href{http://news.naver.com/main/read.nhn?mode=LSD&mid=sec&sid1=102&oid=028&aid=0002306493}{링크}
\vspace{5mm}

인권위가 서울대·연세대·고려대 로스쿨에 30대 지원자와 합격자 수 등 원자료 제출을 요구하는 공문을 수차례 보냈지만 이 가운데 고려대만 자료를 제출했다. 이는 지난해 서울지방변호사회가 “일부 로스쿨이 신입생 선발에 나이를 중요한 기준으로 삼고 있다. 응시자 제출 서류에서 지원자의 나이를 알 수 있게 하는 항목을 삭제해 달라”며 이 대학들을 상대로 인권위에 진정을 낸 데 따른 것이다.  서울대와 연세대는 ‘학생들의 개인정보 보호’를 이유로 거절했다. 이들 대학은 <한겨레>에 “인권위가 요구한 자료는 개인정보에 해당한다. 아무리 인권위라도 해도 원자료를 마음대로 요구할 권리는 없다”고 밝혔다.  인권위법은 업무 수행에 필요한 경우 인권위가 관계 기관 등에 필요한 자료 제출을 요구할 수 있고 해당 기관은 지체 없이 협조하도록 돼있다. 개인정보보호법에도 인권위법처럼 다른 법에서 자료 제출을 요구하는 규정이 있다면 개인정보를 제3자에게 제공할 수 있도록 예외 규정을 두고 있다. 인권위의 자료 제출 요구에 응하지 않으면 1000만원 이하의 과태료 처분을 받지만, 로스쿨들은 ‘과태료를 내고 말겠다’는 태도다. 연세대 로스쿨 관계자는 “전국에 로스쿨이 25개나 있는데 3개 대학에 집중해서 나이 차별 논란을 제기하는 건 맞지 않다”고 말했다. 서울대 로스쿨 관계자도 “(인권위에 자료를 제출하면) 오해의 소지가 많을 것 같다. 서울대의 경우 법학적성시험(리트) 성적이 전국 꼴찌인 30대 이상 ‘허수’ 지원자들이 10여명이나 지원한다”라고 말했다.
\vspace{5mm}

이런 게 현실입죠.
\vspace{5mm}

실제로는 사교육비를 줄인다 과도한 경쟁을 막는다하지만
뒤로는 결국 '학력'까지 상속하는 걸 방기하고 있던 것입니다.
\vspace{5mm}






\section{어째서 하류가 더 막장이 되었나}
\href{https://www.kockoc.com/Apoc/639454}{2016.02.17}

\vspace{5mm}

이건 가설입니다만 $-$ 물론 개인적으로는 당연히 맞다고 생각하는 명제죠 $-$
대중의 자업자득(?)인 하류들의 교육빈곤화에 맞먹는 이야기이니 불쾌하실 분도 있는데 그런 분은 패스하시길.
\vspace{5mm}

IMF 이전에는 조금이라도 빚을 지는 걸 수치스럽게 생각했으며 돈을 벌면 무조건 저축하는 게 일반적이었던 게 있습니다.
그 전까지는 검소하게 사는 것이 장려되었으며 특히 신용카드는 딱히.
(그래서 저는 지금도 신용카드는 안 씁니다. 딱 한번 발급받은 게 있는데 지인 분이 실적 올려야한다고 등록해준 뒤 한번도 안 쓰고 잘라버렸죠)
\vspace{5mm}

그런데 IMF 이후에는 \textbf{빚을 지는 게 장려되었고 신용카드가 남발됩니다}.
개인들이 빚을 지면서 소비를 늘리니 기업들은 살아납니다. \textbf{"기업부채가 가계부채로 전용되었다"}라는 말이 이것이죠.
IMF 극복이니 뭐니 사실 그건 체감 못 하는데 해외여행 갈 사람은 다 가고 쓸 사람은 다 씁니다.
그리고 2000년대 중반 들어서부터 빚을 안 지는 사람을 찾아보기 어려워지더라는 것이죠.
\vspace{5mm}

문제는 \textbf{커진 씀씀이는 줄이기 힘들다}는 것입니다.
지금 문제가 많은 3, 40대가 앞으로도 막장일 수 밖에 없는 게,
실질강건의 풍토가 사라져버리고 빚내서 흥청망청하기 시작한 때 20대를 보낸 세대라는 것이죠.
지금 60대 이상처럼 보릿고개를 겪거나 고생하지도 않았어요. 평균적으로 10대는 평탄하게 보냈단 것이죠.
어떻게 하면 검소하게 살면서 윤리적인 걸 실천할까 하기보다는,
남들이 구입한 명품은 왜 나도 구입하지 못 할까, 친구는 유럽갔다는데 나도 가야겠네... 이런 마인드가 대부분이란 겁니다.
\vspace{5mm}

그래서 살기 힘들다라는 말은 가려들어야한다는 것입니다.
롤스로이스를 못 모니까 불우이웃, 결혼하는데 남자가 3억 집 못 해오면 파혼... 이런 게 현재 상식이 되어버렸습니다.
저럼 삶을 살려면 월 1000은 기본이지... 라고 생각하는데 어라 그런 직업이 의료계 쪽이네... 하면서 다 서로서로 의치한 거리는 것이죠.
그저 많이 번다라고만 생각하고 그에 따르는 사명감이나 책임감 그딴 건 없습니다.
이제 개인이나 사회나 완전히 병들어버리기 시작한 것이죠.
10대들은 어떨지 모르지만 최소한 20대$\sim$40대들은 혹독한 도태를 경험하게 될 것입니다.
\vspace{5mm}

아무튼 기업부채를 가계부채로 옮기는 전략 $-$ 즉 신용카드를 남발해서 서민들의 소비를 늘리는 후유증이 지금도 지속되는 것이죠.
그래서 돈을 쉽게 버는 방법을 모색하면서 법과 윤리도 아작내기 시작하죠.
자기가 돈많이 번다고 자랑하는 속물도 있지만, 더 한심한 건 그런 속물을 보고 나도 그래야지하면서 배금주의에 눈먼 사람들이 생겨난다는 겁니다.
어떻게 하면 아껴쓰고 더 많이 배우고 미래를 향해 투자할까... 그러기보다는
배금주의나 한탕주의적인 사고에 빠져서 많이 벌고 많이 쓰는 게 정상이라고 착각해버린다는 것이죠.
\vspace{5mm}

그러니까 이제는 아이까지도 학대해서 몰래 살해하여 암매장하는 것도 적발되어버리는 겁니다.
스스로의 도덕 그딴 건 없습니다. 오직 '타인'에게 잘 보이는 게 중요합니다.
남들에게 설교하는 번듯한 종교인이어서 xx님이라고 인정만 받으면 딸아이 죽여도 관계없는 것이죠.
이제 윤리관 그딴 건 없거든요? 많이 벌어서 타인들에게 '상류층'이라고 인정받으면 되는 것입니다.
실제 오프라인에서 인간답게 사느냐 그딴 건 모르는 겁니다. SNS에 어떤 뽀샤시한 사진이나 올릴까 이 고민이나 하는 거죠.
\vspace{5mm}

그리고 이건 교육도 마찬가지입니다.
\vspace{5mm}

콕콕에서도 저에게 커리 물어보는 사람은 무조건 까고 볼 건데 간단해요.
커리를 제대로 물어보는 사람은 기본적인 걸 공부하면서도 꽤 크리티컬한 질문을 던지는 경우입니다.
이런 건 일방적인 Q$\&$A가 아니라 토론이 될 수 있기 때문에 기분좋게 대화할 수 있습니다
그러나 그렇지 않은 경우는, 문제집 한권도 제대로 풀지도 않으면서 \textbf{"$\sim$ 좋냐?"}라고 확인만 받고 하라는 공부는 절대 안 합니다.
다시 말해 공부도 안 하는 병신들이 참고서니 강의 평가니 하고 있다는 게 문제죠
\vspace{5mm}

하류들이 신용카드 빚내서 명품이라고 알려진 것 충동구매해놓고 나서 나라탓하는 짓을 자기들이 저지르고 있다는 것이죠.
디씨든 ㅇㅂ든 어디든 간에 정말 '합격'은 못 했으면서 수년째 똑같은 강의, 교재평만 하고 있는 고정닉들이 있습니다.
그렇다고 알바를 뛰면서 돈을 버는 것도 아니고 사실상 쓰레기짓하면서 똑같은 썰을 풀고 있죠.
그 양반들이 그럼 머리가 없어서 그럴 것 같습니까? 이 인간들은 잘못 교육받아서 그래요.
실천이 우선이니 정신없이 공부하고 있어야한다,
코 앞의 일도 처리 못 하면서 먼 미래를 바라보는 미친 짓을 하지 말아야하는데
이 인간들 보면 수년째 참고서 한권도 제대로 안 보고 뭐가 좋냐하다가 2$\sim$3년 날리고 재종이나 고시원 가야하냐 이러고 앉아있죠.
\vspace{5mm}

자, 이것들의 공통점이 무엇일까요?
\vspace{5mm}
\begin{itemize}
    \item 첫째, 자기통제를 못 한다.
    \item 둘째, 배수의 진이 없다
    \item 셋째, 무한한 자유
\end{itemize}
\vspace{5mm}

만약 신용카드 남발이 없었다면 하류들이 빚지는 일은 적었을 겁니다. 쉽게 빚을 낼 수 있으니까 마음껏 소비해서 기업 좋은 일만 한 것이죠.
그렇게 씀씀이가 커진 건 잘못 교육받은 것과 똑같습니다. 이건 '잘못된 것이다'라고 따끔하게 혼나고 배워야하는데 그렇지 못 한 것이죠.
자본주의 사회에서는 소비를 많이 해야 경제가 좋아진다라는 그럴 듯한 궤변으로 사치를 합리화합니다.
그러면서 그 모든 것이 자기 탓이 아니라 나라탓, 정부탓, 사회탓이라고 미쳐가기 시작하죠.
\vspace{5mm}

수년째 공부 안 하고 그러는 잉여들도 마찬가지예요.
그냥 부모들이 바로 내쫓아버리거나 일을 시키거나 했어야하는데 그 부모들이 자녀를 방임하거나 무서워해서 못 건드리는 것입니다.
공부를 안 하면서 그저 썰뿐인 수험을 인터넷에 풀면 몸에 전기라도 통하거나 밥이 안 나오거나 하는 등 제재가 있어야하는데 그런 게 없죠.
사실상 자유가 무한하게 허용되고 브레이크가 없으니까 잘못된 패턴으로 살아가는 것입니다.
\vspace{5mm}

다양성을 존중한다? 돈자랑하는 게 문제가 없다? 헛소리입니다.
결국 무엇이 옳은지 그른지 안 따진다면, 그리고 잘못된 것을 까다롭게 증오하지 않으면 본인도 모르는 사이에 \textbf{막장이 되는 것}이지요.
물론 돈 많이 벌고 명품 사는 건 안 말리는데, 그런 걸 자랑하고 다니는 것이 부끄러운 것임을 모른다는 게 문제죠.
명품 자랑하고 다니면 본인은 결국 '윤리적인 일이나 봉사'도 안 하고 그저 돈쓰는 걸 자랑하고 다니는 돼지라는 걸 모르는 것이죠.
우왕, 저 친구 성형 괜찮은 데에서 했나봐, 차 외제차네, 옷은 이태리제 명품이네... 하면서 부러워하는 인간들도 돼지들이죠.
\vspace{5mm}

자기들도 황금만능주의나 방종에 빠진 걸 모르면서 부자들 증오해보았자 소용이 없는 것이 아닐까 싶은데 말입니다.
부자들을 정말 이기려면 본인이 부자가 되거나, 아니면 검소하게 살면서도 부자들보다 가치있게 산다는 걸 실천하시면 됩니다.
그런데 이런 것이 어느 순간에서부터인가 사라집니다.
\vspace{5mm}

이 나라에 희망이 없다면 그건 재벌이 독점해서도 혹은 정부가 무능해서도 아니면 안보가 위험해서도 아니죠.
어떤 위기가 오더라도 사람들이 근면, 건실하고 물질이 부족하더라도 지혜를 짜낼 수 있다면 다시금 일어섭니다.
그러나 그런 게 지금은 사라졌습니다.
\vspace{5mm}











\section{전문가 드립}
\href{https://www.kockoc.com/Apoc/640751}{2016.02.18}

\vspace{5mm}

잘 알지도 못 하는 자가 '책임있는 발언'을 회피하기 위해 쓰는 흔한 말장난이죠.
\vspace{5mm}

A : "$\#@\$@\#$*(*(ㄴㅇㄹㄴㅇ라고. 그럼 네 주장을 해보렴"
B : (할 말이 없다) "너는 전문가 아니잖니. 전문가도 아니면서 왜 그래?" \textbf{$\rightarrow$ 부적합한 권위에 호소하는 오류}
\vspace{5mm}

이 경우 A는 이렇게 이야기할 수 있죠
\vspace{5mm}

A : "그럼 그렇게 말하는 너부터 전문가도 아니잖아"
B : "그래, 너도 나도 전문가도 아니니까 이건 끝도 없어" $\rightarrow$ \textbf{양비론}
\vspace{5mm}

시도 때도 없이 전문가 드립 쓰려면 그냥 자가기 전문가 데려와서 전문가 보고 대신 발언하라 하는 게 낫죠.
하고 싶은 말은 많다, 하지만 자기는 '모른다', 하지만 네 말을 공격하고 싶어라고 할 때 흔히 하는 말이 전문가 드립입니다.
말하지만 그럴려면 직접 전문가들을 데려오시거나, 그 문제에 해당하는 전문가의 전문적 답을 제대로 인용해오면 됩니다.
그러나 늘상 그렇듯 B와 같은 인간들은 "내가 왜 그래야하는데"라고 또 회피해버리죠.
\vspace{5mm}

여기서 인간성이나 능력 테스트는 사실 걸러지죠.
\vspace{5mm}

그럼 믿거나말거나인데 진짜 전문가들은 어떤가 제 뇌피셜 기억에 살려서 이야기 적죠
\vspace{5mm}

\textbf{$-$ 의새 $-$}
\vspace{5mm}
\begin{itemize}
    \item[] A : "야, xx 병역 사건 어떻게 생각해. 그거 확실히 그 사진 이상하지 않아?"
    \item[] C : "으으음, 그건 더 신중히 검토해야 하는데. 사실 $\sim$ 라고 말은 하는데 나도 정확히 장담할 수 없어"
\end{itemize}
\vspace{5mm}

\textbf{$-$ 변새 $-$}
\vspace{5mm}
\begin{itemize}
    \item[] A : "야, 그 사건은 $\sim$ 잘못한 것 아냐. "
    \item[] D : "사실 나도 잘 몰라. 그건 그 분야 다른 사람에게 물어보아야지. 기록을 더 봐야 알 것 같은데"
\end{itemize}
\vspace{5mm}

이상은 지인에서 기억나는 사례인데
\vspace{5mm}
\begin{itemize}
    \item 첫째, 신중함을 강조한다.
    \item 둘째, 자기도 모른다라고 무조건 말한다.
\end{itemize}
\vspace{5mm}

그럼 과연 '전문가' 드립을 치시는 분들은 정말 전문가를 만나보셨는지는 가히 의문.
\vspace{5mm}

왜냐면 사회에서 xx 전문이라고 자부하는 사람들은 실제로 전문가라기보다는 사기꾼인 경우가 많죠(....)
그리고 정말 전문가라고 할 수 있는 사람들은 오히려 바쁘고 피곤해하고 꼴통스럽습니다. 귀찮은데 왜 불러 그런 케이스
\vspace{5mm}

\textbf{$-$ 기계 전자 $-$}
\vspace{5mm}

\begin{itemize}
    \item[] A : "아이고 이 노트북 이상해요. 자꾸만 꺼지고 아무래도 CPU나 팬이 잘못된 것 같아요. AS 되나요 ? T$\_$T"
    \item[] D : "... 이거 걍 포맷하고 윈도우 다시 까세요 ..."
    \vspace{5mm}
    
    \item[] A : "이 자전거 브레이크 고칠 수 있어요?"
    \item[] E : "(뜸 들이다) 이거, 걍 새거 사는 게 빠르겠는뎅. 아, 돈이 좀 깨지겠는데 그래도 수리하시겠나?"
\end{itemize}
\vspace{5mm}

가장 흔히 접하는 사례입니다.
그리고 우리는 다시 상기할 수 있죠.
\vspace{5mm}

전문가들일수록 '돈 안 되고 귀찮은 것'은 매우 싫어한다.
하지만 '돈'을 주면 그래도 구색을 갖추려고는 한다. 물론 소위 그 전문성이란 '현금'에 따라 달라진다.
\vspace{5mm}

전문가 드립치려면 적어도 이런 것들은 상기했으면 좋겠습니다. 그리고 \textbf{본인들부터 전문가가 되시든가.}
정말 공부하고자 하는 사람은 전문가 드립은 안 칩니다. 자기가 공부해서 그걸 줄줄 이야기하고 토론하고 배우려고 하지.
\vspace{5mm}

공부 안 하고 전문가 따져보았자 전문가 사칭하는 사기꾼들 $-$ 즉 거짓말의 전문가들에게나 넘어가죠.
입시판에서도 아이고 선생님$\sim$ 거리는 아줌마들이 많죠.
선생이 어디 학교 출신이라더라 진도는 어디까지 빠진다더라 하면서 간 빼줄 것처럼 다 하지만
자기가 그 과목도 모르는 채 '어디 나온 선생이라더라' 그런 거나 따지는, 즉 자기는 공부 안 하면서 자녀가 공부 잘 하길 좋아하는 부류.
이런 케이스들이 자녀를 정말 잘 망칩니다. 하도 흔해서 그냥 얘기하다가 스포 때리면 점장이냐는 반응도 지겹게 나오죠
자기들은 모르죠, '흔한 케이스'라는 걸요.
\vspace{5mm}

+
전문가들도 결과를 100$\%$ 보장은 못 하죠. 그리고 의견들도 갈리고
다만 자기들이 책임회피를 하는 방법만큼은 정말 잘 알고 있습니다.
\vspace{5mm}

++
진짜 전문가라고 하면 후쿠시마 원자력 발전소 사건 때 "죄다 공구리쳐야한다"라는 서균렬 교수였죠.
그런데 그 때 전문가가 서균렬 교수만 있는 것도 아니지만. 아무튼 특정한 사건에 "냉소적"으로 "잔인한 해답" 제시하는 게 가장 전문가스러운 듯.
\vspace{5mm}




