


\section{관계 = 돈}
\href{https://www.kockoc.com/Apoc/643932}{2016.02.21}

\vspace{5mm}

모 소설가의 소설을 읽다보니 재밌는 이야기가 하나 나옴.
\vspace{5mm}

자식이 공부를 못 해서 고민하던 아버지가 재산 1억 5천만원을 털어
공부 잘 하는 학생 셋에게 급부로 제시하면서 계약을 함.
자기 아이가 시험에 붙을 때까지 친구로 지내달라고. 그리고 그 세 친구 덕분에 무사히 시험에 합격.
\vspace{5mm}

픽션이면 픽션이라지만 사실 저게 정말 '환경'의 핵심을 찌른게 아닌가 싶다는 생각이 들었음.
부모들이 극성을 떨면서 학군 좋은 데로 이사가거나 자녀들을 좋은 학교에 보내려고 하는 이유는
만나는 사람들의 클라스에 따라서 자녀가 영향을 받는 것이 성적에 큰 영향을 끼칠 수 있기 때문.
\vspace{5mm}

물과 공기를 사먹어야하는 세상이 설마 오겠어 하는데 오고야 말았음.
마찬가지로 인간관계도 대가를 지불하고 구입하는 것이다.... 사실 성매매도 그런 케이스가 아닌가 싶지만
친구 관계도 이미 간접적으로는 '부동산 가격'에 반영된 것이 아닌가 싶음.
\vspace{5mm}

이렇게 보자면 "교재", "강의" 이전에 훨씬 중요한 게
바로 '무형'에 가까운 삶의 양식 $-$ 즉 행동과 습관이야말로 가장 중요한 게 아닌가 싶음.
1등하는 친구가 있다면 그 1등하는 친구에게서 배울 것을 배우고 그 친구와 경쟁하면서 덩달아 올라가는 것임.
반대로 실패하는 친구가 있다면 본인이 그걸 비판하고 개선하지 못 한다면 덩달아 실패할 수 있음.
\vspace{5mm}

물론 절대적인 성공, 실패 공식은 없음. 왜냐면 게임의 룰이 달라지기 때문에
수험도 매년마다 룰이 조금씩 바뀌고 있음. 예컨대 수학도 꿀강의 꿀교재본다에서 기초교재본다는 걸로 바뀌는 것도 그런 것임.
만약 당사자가 성공하는 양식을 복제하거나 거기에 전염된다면 성공하는 것이고 그 반대면 실패하는 것임.
\vspace{5mm}

그래서 이게 오싹한 이야기임.
어떻게 보면 인간관계란 자기를 흥하게 할 수 있는 사람들과
망하게 할 수 있는 사람들 속에서 핀볼처럼 충돌해나가는 과정임.
본인이 분별력이 있어서 잘 취사선택한다면 모르겠지만,
\textbf{우유부단하고 잘 휩쓸리는 성격이라서} 좋은 게 좋은 거야 하다간 막장이 되어가는 것임.
\vspace{5mm}

그렇기 때문제 자기 인생이 걸린 문제는 마녀사냥하는 종교재판관의 포지션을 취하는 것도 나쁘진 않음.
조금이라도 해롭다라고 생각하면 과감히 컷하고 검증된 방식만 채용해 그걸로 노력해가야한다는 것인데
문제는 10대들이 이걸 알아서 할 능력을 기대할 수는 없고, 그렇기 때문에 부모나 교사나 멘토의 역할이 중요해지는 것임.
\vspace{5mm}

무엇보다 관찰하면서 느낀 것은
\vspace{5mm}

성공과 달리 \textbf{실패는 반복되는 경향이 있다}는 것.
\vspace{5mm}

사실 그럴 수 밖에 없음. 성공패턴이 쉽게 학습된다면 누구라도 성공할 것임.
반면 실패가 많다는 건 실패하는 패턴은 학습하기 쉽거나, 따로 학습이 필요하지 않아도 몸과 마음이 그걸 따라간다는 이야기임.
자영업하다가 말아먹는 경우도 그렇고 n수하다가 깨지는 경우도 관찰 분석하면서 느끼는 건
당사자가 종교처럼 집착하면서 버리지 못 하는 뭔가 있음. 사실 그게 실패의 요인인데 그걸 '버리려'하지 않음.
반면 성공한 사람들의 공통점은 상투를 자를 때는 자르고 창씨개명도 할 때는 해버림.
아예 자기를 버림. 자기를 버리니까 실패하는 패턴도 덩달아서 날라감, 그러니까 청소는 확실한 것임.
\vspace{5mm}

결국 자녀교육을 시킬 때는 비싼 돈을 들여서라도 성공하는 사람들 곁에 가깝게 둬서 거기에 전염되도록 한다...
이거야말로 잔인하지만 가장 간편한 방법임.
\vspace{5mm}






\section{대안없는 비판}
\href{https://www.kockoc.com/Apoc/645370}{2016.02.22}

\vspace{5mm}

보통은

ⓐ 인신공격, ⓑ 비난,  ⓒ 대안없는 비판, ⓓ 대안있는 비판, ⓔ 대안
어떤 문제에 대한 반응은 저 다섯가지로 나뉘어지는데
\vspace{5mm}

ⓐ, ⓑ는 그냥 노골적으로 자기 감정을 표출한 것이니 적대관계를 도출하는 반면
ⓓ, ⓔ는 그래도 뭔가 준비하면서 책임을 지는 것이라 우호관계를 조성하는데
\vspace{5mm}

가장 애매한 것이 ⓒ이다.
\vspace{5mm}

사실 이건 그 당사자는 자기가 '비난'했다라는 걸 면피하고 싶으면서도
자기 이야기가 논리적이고 타당성 있다라고 하면서 비판의 형식을 취하고 싶은 것인데
이 때 "그럼 대안은 뭐니"라고 하면 당연히 말을 못 하면서 "그럼 대안이 있어야 비판을 하느냐"라는 말을 한다.
\vspace{5mm}

물론 '대안'이라는 게 항상 존재할 수도 없기 때문에 대안있는 비판만을 주문한다는 건 가혹해보일 수도 있다.
엄격한 논증과 고찰 하에서 비판을 하면 구체적 대안까지는 아니더라도 \textbf{추상적 대안}까지는 나온다.
\vspace{5mm}

아무튼 20대 초에는 저런 사람들이 있어서 정말 대안없는 비판도 괜찮은 건가 아닌 건가 그랬는데.
\vspace{5mm}

지금 내린 결론은 "대안 없는 비판"은 참고할 수 있지만 그걸 하는 사람, 즉 대안없이 비판만 하는 사람은 무시해도 된다는 것이다.
왜냐면 그 사람들은 그냥 문제해결의 의욕이 없기 때문이다.
어디든 끼어들어서 시어머니처럼 잔소리만 하고 그런 걸로 한자리 해먹으려는 경우가 많지
실제로 정말 중대한 권한이 주어지거나 하면 일처리는 제대로 못 한다.
사실 비판만 할 줄 아는 사람이 뭘 해먹겠나.
\vspace{5mm}

정말 문제해결에 의욕이 있는 사람은 먼저 '해결'해놓거나 그런 노력을 기울이고 나서  '비판'한다.
다시 말해 불이 났다면 누가 불냈어라고 짜증만 내지 않고, 먼저 불부터 다 끈 다음에 범인을 심판하기 시작한다는 것이다.
물론 그 실천이 '비판'을 금지하는 건 아니라는 것이 중요,
다만 비판은 실천이 담보되었을 때에만 진정성이 있다는 것이다.
어차피 문제해결이 중요한 것인데 그것과 관계없는 미주알고주알은 그냥 소음 공해 아닌가?
\vspace{5mm}

어떤 일을 추진할 때 가장 먼저 걸러내야할 사람은 다시 말해서 \textbf{"실천 여부"}만 보면 된다.
뭔가 비판을 하려고 하는데 그럼 \textbf{네가 해결해 봐...} 라고 하면 온갖 핑계 대서 빠져나갈 사람부터 먼저 걸러내야 한다.
사실 살아가면서 그런 사람들만 멀리 하더라도 삶은 매우 평화로워진다.
갈등 대부분이라는 게 사실 '언행불일치'에 시작되는 것이다. 이게 상당수 차단되기 때문이다.
말로만 $\sim$ 하는 사람과 거리를 두면 피곤한 일들이 사라진다.
\vspace{5mm}

+
\vspace{5mm}

개인적으로는 실천도 높은 사람은 뭔 이야기를 해도 일단은 듣는다. 적어도 이런 사람은 뭔가 진정성이 있으니까.
하지만 그렇지도 않은 사람이 그렇다면 그다지. 일단 이런 사람들은 아직 철이 없나, 그 생각 밖에 없다.
자기들이 뭔가 구체적으로 준비한 것도 아니고 그렇다고 새로운 정보도 준 게 아닌 데 자기가 뭐라고 나한테 나서는 거지?
이 사람들은 기본적으로 관계는 '\textbf{거래}'에서 출발하는 것이고, \textbf{거래}는 상대방에게 급부를 제공해야한다는 것임을 모르는 것 같다.
\vspace{5mm}

아주 어린 애라거나 10대 학생의 경우는 처지상 거래로까지 따질 수는 없을 것이다.
그러나 당사자가 성인이라면 달라지는 문제지.
타인의 글에 대해서 뭐라고 하고 싶으면, 자기부터 구체적 근거에 기반해 설명을 상세히 하시면 된다,
적어도 그 정도는 해야 거래가 되는 거지.  이건 인격 이전에 \textbf{'거래관념조차 없는'} 뭔지 모르는 무전취식범에 다름이 아니지.
\vspace{5mm}








\section{"때리면서 말렸어야죠"}
\href{https://www.kockoc.com/Apoc/645641}{2016.02.22}

\vspace{5mm}

\textbf{"때리면서 말렸어야죠"}
\vspace{5mm}

진짜 이건 내가 화자이기도 하고 청자이기도 하다능
인간은 애초에 간사
\vspace{5mm}

\vspace{5mm}

\begin{itemize}
    \item[] ex 1 $-$
    
    \begin{itemize}
        \item[] A : "야, xx고는 졸라 빡세. 내신 따기 엿같다니 다른 꼴통고 가자"
        \item[] B : "아, 열심히 해서 이겨낼 수 있어요. 괜찮아욧"
    \vspace{5mm}
    
    일년 뒤
    \vspace{5mm}
    
        \item[] B : "아니, 왜 안 말렸어요 T$-$T"
        \item[] A : "네가 간다면서"
        \item[] B : "때리면서도 말렸어야지 으흑흑"
    \vspace{5mm}
    \end{itemize}
    \item[] ex 2 $-$

    \vspace{5mm}
    
    \begin{itemize}
        \item[] A : "공무원 시험 쳐"
        \item[] B : "에이, 별 거 없잖아요. 나 해외 나가서 원피스를 찾을거야"
    \vspace{5mm}
    
    오년 뒤
    \vspace{5mm}
    
    \item[] B : "나 왜 그 때 붙잡아두고 공부 안 시켰어요 T$\_$T"
    \end{itemize}
\end{itemize}
\vspace{5mm}

더 쓰다보니 패턴이 똑같아서 .... 라는데 정말 많다. 결혼 버전이야 음성지원되는 수준이다보니 생략
\vspace{5mm}

그게 사람은 초기에는 꿈과 이상을 보지말, 말기에는 \textbf{결과와 금전을 본다}는 씁쓸한 진실.
무조건 어른들 말을 들을 필요는 없지만, 어른들이 뭔가 말리면 왜 그런지 그 이유는 알아야한다는 것.
\vspace{5mm}

단순화시킬 수 없지만
독재자 부모가 시키는대로 해서 성공하는 경우와
본인이 정말 민주주의적으로 판단해서 성공하는 경우 중 어느 게 많을까.
\vspace{5mm}

경쟁은 그 의사결정의 민주성을 따지진 않는다.
\vspace{5mm}

사실 본인은 특정 시험에 합격 못 하거나 혹은 특정 지역에 못 산다고 하더라도 정신승리할 수 있다.
그런데 \textbf{'친구'가 특정 시험에 합격하거나 특정 지역 고급 아파트에 입중했다는 걸 듣는 순간 피가 거꾸로 솟는다.}
그리고 여기서 더 심하면 우울증이나 정신병 차원으로 가기도 한다.
\vspace{5mm}

엄마들은 자기 친구 아들딸을 언급하면 안 된다. 오히려 자기 아들딸의 친구들을 언급해야지
친한 친구가 자기보다 잘 나간다고 느끼면 빡쳐서 공부하는 경우가 많은데 왜 이걸 모를까
\vspace{5mm}

아무튼
처음에는 민주주의적이고 자상해서 좋다는 사람들도 나중에 결과가 시망이면 욕하게 되어있다.
반면 독재자적이고 성격 안 좋다고 하는 사람들이 좋은 결과를 보장해주면 그 때에는 굽신거리게 되어있다.
과정이 결과보다 중요하죠... 라는 건 어디까지나 좋은 결과가 보장될 때에나 할 수 있는 이야기다.
\vspace{5mm}

만약 어른들이 반대하는데도 특정한 선택을 한다면 나중에 왜 안 말렸냐느냐 그런 걸 안 따지는 각서는 쓰시라는 이야기
\vspace{5mm}






\section{직업소득 비교가 한심한 이유}
\href{https://www.kockoc.com/Apoc/649588}{2016.02.24}

\vspace{5mm}

보통 대학가는 사람이 졸업하면 10년 뒤일건데 그 때 어떻게 될지는 아무도 모릅니다.
그렇다고 역대 직업예측이 맞아떨어졌냐. 그건 아니지요.
오히려 10년 뒤에 잘 나갈 직업들은 저평가되었거나 혹은 태어나지 않은 경우가 많아요.
\vspace{5mm}

중요한 건 해당 직업이 좋냐 나쁘냐가 아니라 어떻게 돈을 버느냐 그걸 봐야한다는 것이죠.
\begin{itemize}
    \item \textbf{첫째, 누구 돈을 먹느냐.}
    \item \textbf{둘째, 기술 진보에 따른 영향을 어떻게 받느냐}
    \item \textbf{셋째, 공급과 수요는 어떠냐는 것입니다.}
\end{itemize}
이렇게 구체적으로 들어가야죠. 이렇게만 분석해보면 수험사이트에 올라오는 썰은 구라가 될 가능성이 꽤 높아집니다.
\vspace{5mm}

그럼 의사들은 그럼 서로 치료해줘서 경제활동하나요?
떡팔러가는 친구 A와 술팔러가는 친구 B가 서로 물물교환해서 떡과 술 같이 먹고 한 것도 수입으로 쳐줘야하나요?
\vspace{5mm}

의치한 말고 다른 직업이 망했다라는 논리가 말이 안 되는 게, 전문직이 버는 돈이 \textbf{그 다른 직업인들로부터 나오는 겁니다.}
(의치한 백날타령하는 인간들이 기본적인 사고조차 안 된다는 걸 여기서 확인할 수 있죠)
아니 돈을 써줄 사람들이 망하는데 그럼 의료인들이 버는 돈은 어디서 나오는 건가요?
그리고 다른 직업이 망하면 그들의 상품과 서비스도 개판일 건데 돈을 벌어서 뭐합니까. 쓸 수가 없을 건데
\vspace{5mm}

기본적으로 우리나라의 부는 \textbf{"수출"에서 나오는 겁니다.}
다들 주제파악 못 하고 왜 기업 우대하느냐 어쩌냐하는데 자원도 없고 이제 노동력도 중국만도 못 하게 된 나라에서 뭘로 벌어먹죠?
좋든싫든 대기업 밀어줄 수 밖에 없는 이유가 여기서 나와요. 경쟁이 되어야 달러 벌어와서 수입해오는 거지.
입으로만 단군 한민족거리지만 사실 우리의 삶을 이루는 태반은 거의 다 \textbf{'수입'한 겁니다}.
그리고 말하지만 꼬우면 각자가 한번 대기업 역할을 해보시면 됩니다. 개인이 수출해보고 달러벌어오면 되는 거지 뭘 그러나.
입으로는 딱 대기업 욕하면서 정작 취업할 때는 x성 어떻게 하면 들어가요하는 케이스 참 지겹게 보지요.
\vspace{5mm}

지금이야 그나마 의료인들이 상대하는 고객들이 돈이 있으니까 의사도 고소득 거둔다하는 거지
\textbf{그 의사 이외의 직업이 다 망해보셈. 그럼 어떤 일이 벌어지겠나.}
이런 근본적인 것도 따지지 않고 소득이 어쩌고 비교하는 것보면 참 개한심들 하죠.
\vspace{5mm}

저성장시대인데 그럼 거기서 공무원과 의사는 잘 먹고 나머지는 다 백수되고 굶어뒤진다?
조금만 생각해보면 알지만 말이 안 되죠.
그럼 지금 공무원과 의사가 나아보이는 건 왜 그러냐
다소 과장되어있어서 그렇지 \textbf{아직까지는 다른 직업들이 '망한' 건 아니라는 겁니다.}
1980$\sim$90년대의 경이적인 고성장 시대에 비해서 실업률이 높아지고 변동성이 커져서 그렇지
객관적으로 따지면 아직 '망한' 건 아니기 때문이죠. 어떤 분야건 잘 나가는 사람은 잘 나가고 있습니다.
다만 100명 중 10명이 실업자가 된 것으로 이게 심각해보여서 그렇지 아직까지는 한국 제조업은 잘 나가는 편입니다.
미래가 암담해보여서 그렇지
\vspace{5mm}

오히려 특정직업을 강조하는 것이야말로 "시장 논리"가 작용하는 게 아닌가 생각해보는 게 더 낫겟죠.
제가 업자라고 하면 안 그래도 저출산 때문에 \textbf{고객(=수험생)도 줄어가니 신규 고객 창출하기 위해}
의대 빼고 다른 직업 다 망한다 그럴 듯.
\vspace{5mm}

+
그리고 의료 바이오로 먹고 산다... 그럼 미, 중, 일은 가만히 있을까요.
우리나라 사람들의 문제가 수입품은 다 잘 들 쓰면서 시야가 한반도에 국한되어 있다는 겁니다.
미국, 중국, 일본 돌아가는 꼴 보면 대충 보이죠 뭘
\vspace{5mm}

아주 어린 시절에 중공이 있었던 시절 $-$ 삼국지 읽으면서 중국 관광갈 수 있을까, 공산권인데.
그리고 메이드 인 차이나... 그런 건 없었죠. 그러나 지금 현실은?
\vspace{5mm}

++
똑똑한 척 하는 지금 담론에서 걸러들어야할 것. "강남 2세 마인드"
부모들은 강남에 정착해서 직접 개척한 사람들인데 자녀들은 그냥 강남 수저들인 케이스인데
이 친구들은 똑똑한 척 하지만 실제로는 실속이 없어요. 파더 쉴드 장난 아닌데다가 시야가 강남을 못 벗어나죠.
\vspace{5mm}

대체로 금수저로 지목되는 게 이들인데 별로 열등감 안 느껴도 됩니다. 유능한 사람들은 생각보다 없기 때문에.
다만 이들이 하는 이야기를 하류들은 곧이 곧대로 믿죠.
\vspace{5mm}





\section{가격 가치}
\href{https://www.kockoc.com/Apoc/650743}{2016.02.25}

\vspace{5mm}

소득은 가격에 속하는 것이지만 그건 '가치'의 일부일 뿐.
\vspace{5mm}

가치 ≠ 가격
\vspace{5mm}

공기의 가격이 0원이라고 해서 공기의 가치가 없다고 할 리는 없다.
반면 수백만원한다는 금반지가 배고픈 상황에서 소용이 있을까.
\vspace{5mm}

뭐가 좋나요... 하는 흔한 질문은 결국 "가치"를 물어보는 문제다.
가령 인강의 경우 가격으로 치면 사설 > EBS로 보인다.
그러나 때에 따라서는 사설 인강보다 EBS 무료 인강이 더 가치가 높을 수도 있다.
\vspace{5mm}

가격은 시장이 형성되어서 거래자들간의 경합이 붙었을 때 붙는 것.
그 이야기는 시장의 범주에 들지 않거나 혹은 든다 하더라도 거래자들의 경합이 없다면 가격<가치 일 수 있다는 것이다.
\vspace{5mm}

그래서 흔히 나오는 말이 결국 \textbf{'가치투자'가 진리}라는 것인데
그럴 듯한 논리이지만 사실 실천성에서는 많은 면이 공백이다. 그럼 \textbf{'가치'를 어떻게 파악할 것인가?}
\vspace{5mm}

특정직업이 돈을 많이 번다.. 그걸 못 한 게 후회가 된다..
정작 그런 직업인들과 얘기해보면 자기들도 불평불만이 많다.
무엇보다 돈을 많이 주는 직업이라면 빡센 것은 기본이다.
\vspace{5mm}

세상에 공짜가 어디 있나.
\vspace{5mm}

그런데 신기하게도 사람들은 그 모든 걸 '가격'으로 평가하려 한다.
가치는 측정하기 어렵지만 그것이 일단 잡히고 나면 변동이 덜 한 반면에
가격은 \textbf{시장 논리에 따라서 폭등, 폭락할 수 있다}는 걸 간과하고 있다.
\vspace{5mm}

그런 가격만능주의자는 싸구려 상품도 비싼 가격이 매겨지면 명품인 줄 아는 무식한 고객과 똑같다.
직업을 이야기하려면 어떤 일을 하는 것이며 그로써 당사자가 어떤 삶을 사는 것인가 생각해보아야하는데
무작정 돈을 많이 버니까 좋은 직업이라고 이야기 한다.
이 경우는 논쟁할 필요가 없다. 그냥 그런 사람 자체를 멀리하면 된다.
\vspace{5mm}

가격으로만 따지면 소위 '오피', '조폭', '사기꾼'도 매우 좋은 직업이 될 것이다.
어찌되었든 돈만 많이 벌면 되는 것이니까 말이다.
\vspace{5mm}

수험 외적으로도 공부해야하는 이유
\textbf{첫째, '가치를 보는 눈을 키우기 위해서이다.}
\textbf{둘째, 스스로 가치를 창출할 수 있어야 하기 때문.}
\vspace{5mm}

그런데 그런 가치와는 별개로 그냥 가격에 환장한 돼지들이 있다.
그 사람들은 짝퉁명품도 고가에 내세우면 아무 생각없이 빚내서 사는 사람들과 똑같다.






\section{심리회계 $-$ 원금을 어느 선으로 둘 것인가.}
\href{https://www.kockoc.com/Apoc/658036}{2016.03.01}

\vspace{5mm}

공황이 오는 기준은 "원금"을 잃었느냐
\vspace{5mm}

예컨대 100만원을 걸었다고 치자.
A는 101만원, B는 99만원이다. 별 차이는 없어보일 것 같지만 심리적인 동요는 다르다
A는 원금을 안 잃고 1만원을 번 것이라서 차분하게 대책을 수립할 수 있다.
B는 원금을 1만원이라도 까먹은 것이기 때문에 어떻게 회수할까에 집착하면서 판단을 못 하고 만다.
\vspace{5mm}

그럼 우리가 B에게 해줄 수 있는 충고는 ?
원금 기준을 95만원으로 낮추라고 하면 될 것이다
아니 애당초부터 100만원으로 투자할 때 원금을 80만원이라고 생각한다면
훨씬 더 마음 편하게 움직일 수 있을지 모른다
\vspace{5mm}

이건 수험도 마찬가지이다.
\vspace{5mm}

본인이 어떤 눈높이를 갖고있느냐에 따라서 성패가 달라지는데
\vspace{5mm}

흥미로운 사실은 특목고나 자사고 출신들이 생각보다 성과가 안 좋으며
\textbf{한번 실패를 해버리면 미친 듯이 내리막길을 걷는 경우가 많은데} 이것도 저걸로 설명될 수 있다고 본다.
\vspace{5mm}

우선 그런 특목이나 자사고는 학교가 교육을 하기보다는 평가를 하는 쪽인데, 보통 내신기출만 봐도 뭐 이런 SM쇼가 다 있나 할 정도.
문제는 그게 수능과 사실 거리가 먼 경우도 많고, 만약 학생 본인이 소화 못 시키는 경우 주화입마되는 케이스도 있다 짐작되지만
이건 더 상세하 검토할 사안이라서 차후에 얘기하기로 하고 눈높이를 말하면
\vspace{5mm}

특목자사고에 간 친구들의 눈높이는 보통 사람보다 높다.
허영심에 가득찬 부모님들이야 거액을 내면서 자기 자녀가 남과 유별나게 다르다라는 걸 강조하고 싶겠지만
이건 다시 말해서 저 친구들의 \textbf{"기준 원금"이 높다}는 것이다.
그래서 이 친구들은 조그만 성과에 감사하지 않으면서 대박을 거두지 않으면 실패라고 생각.
위에서 바로 B에 해당하는 케이스가 되기 좋다.
만약 한번 시험을 쳐서 실패해버리면 자존심도 크리티컬하게 상처입고
이걸 어떻게 '대박'으로 만회할까.... 하는 생각에  계속 무리수를 두고만다.
차라리 처음부터 본인이 운이 좋아서 좋은 학교에 왔으며 자기는 머리가 좋지 않고 언제든지 망할 수 있다라고 생각하면서
겸손하게 기초부터 다지고 가면 되었던 것이 무리수를 두면서 물경 5년 이상을 낭비해버리는 결과를 낳는다
\vspace{5mm}

시험을 치른다고 하면 오히려 꼴찌에서 시작한다고 마음먹는 편이 그래서 바람직하다.
가령 자기가 5등급에서 시작한다고 본다면 2, 3등급이 뜬다고 해도 차분하게 왜 더 올라갈 수 없는지 여유있게 검토할 수 있다.
그러나 본인이 무조건 1등급이라고 한다면 저건 성적이 떨어진 것이라 감정적인 대응을 할 수 밖에 없지.
\vspace{5mm}




\section{이미지 차크라}
\href{https://www.kockoc.com/Apoc/659029}{2016.03.02}

\vspace{5mm}

좌뇌와 우뇌로 설명하자면
\vspace{5mm}
\begin{itemize}
    \item[] ⓐ 논리적인 사고를 꼼꼼히 할 수 있느냐,
    \item[] ⓑ 아니면 이미지를 잘 구사할 수 있느냐.
\end{itemize}
\vspace{5mm}

이 두가지로 나뉨.
\vspace{5mm}

천부적으로 머리가 정말 좋아서 이해가 빠르고 문제를 신속히 \textbf{푸는 것으로 보이는} 괴수들이 있음.
물론 공부를 얼치기로 하거나 적당히 야매교재 만드는 파들이야 이걸 머리가 좋다고 퉁치겠지만
적어도 내가 분석하고 탐문한 결과는 "이미지, 즉 심상구사능력"임.
문제만 보아도 관련 정의와 개념, 그리고 식과 그래프와 도형이 바로바로 연상되는 것임.
엄밀한 테스트는 아니지만 이런 친구들에게 "이미지"와 무관한 문제를 내면 평범한 학생 수준이거나 그만도 못 함.
신속히 풀어댄다는 것은 다년간 쌓인 차크라... 아니 \textbf{심상 능력}이라는 가설을 반박하기는 힘들어보임
\vspace{5mm}

그런데 이 경우 심상만 강조하다보면 술법에 환장하다가 나중에 개차반이 되어버린 오로치마루로 전락할 수 있음.
이미지로 다 풀리다보니까 '논리적'인 것이나 '기본 개념을 근본적인 것에서부터 복기'하는 것을 등한시하게 되는 것임.
게다가 공부를 잘 한다고 주변에서 떠받들어주고 본인도 거기에 만족한 나머지 이런 단점을 눈치채지 못 하다가 결정적 크리 한방 먹음.
적어도 수학은 이런 이미지파에게 다소 불리하게 출제되어가는 방향으로 가고 있음, 즉 논리를 강조하기 때문.
다만, 탐구에 있어서만큼은 아직 이런 이미지파들에게 불리한 출제는 되고 있지도 않고 되기도 힘들 것이라고 생각함.
\vspace{5mm}

즉, 결국 이미지 차크라를 갖추는 것이 손해볼 것은 아니라는 이야기임.
하다 못해 본인이 록리여서 그게 힘들다 하면 '풀이과정'을 신속정확하게 적는 훈련을 해서 보조를 해야함(전과목 모두)
어정쩡한 이미지 차크라는 부정확하거나 잘못된 편견을 조장할 수도 있음.
사실 수학을 망하는 흔한 코스인 \textbf{"중딩 때 잘 했는데 고딩 올라와서 망했다"}가 이 케이스임.
중3 때까지야 어설픈 차크라를 발현하더라도 스피디하게 100점 맞는 게 가능하지만 고딩수학은 이런 게 잘 안 먹히기 때문에.
다만 본인이 어렸을 때부터 '논리력'을 훈련한 경우에 논리적 사고를 정확히 하는 경우는 확인했음.
물론 이 경우는 부모빨이 큼 $-$ 유전자보다는 바로 가정, 교육환경.
논리력을 갖추었다고 확인하는 이유는 "왜 그럴까요", "근거가 있어야하지 않습니까"라고 말하기 때문.
\vspace{5mm}

아무튼 이런 괴수들은 패턴화도 필요없음, 문제를 한번만 풀어도 그 패턴을 그대로 이미지 차크라로 흡수함(...)
다시 말해 어떤 새로운 과제나 문제를 풀 때에 거기 나온 패턴들을 효율적으로 학습하는 습관도 들어있거니와
어린 시절부터 축적된 이미지의 중력으로 그런 것들을 빨아들인다고 보면 되는 것임.
\vspace{5mm}

대략 초딩 때부터 그렇게 훈련받았다고 가정하면 반올림해서 \textbf{10년 정도 그렇게 훈련된 것이니} 능력 차이가 크다고 할 수 밖에(...)
그런데 여기서 유의할 사실은 최근에 등장하는 이런 괴수들이 과거의 괴수들보다 더 하드코어하다는 것임.
소위 영재라고 불리는 사람들이 나이 먹고 활약하는 경우는 사실 별로 없음. 가짜 영재였거나 아니면 재능이 잘못 발현된 경우이기도 하지만
무엇보다 과거의 평가 시스템도 좀 병맛이었다는 것인데
최근에 등장하는 무서운 괴수들은 인터넷 시대가 펼쳐진 이후에 부모들의 체계적인 관리로 양성된 애들임.
아마 다들 관심없겠지만 유치원, 초등학교 교재들을 보면 두뇌개발을 섬세하게 신경쓴 내용과 편집이 돋보임.
\vspace{5mm}

이 추세라면 지금 20대도 현재 10대들에게 밀리겠고, 현 10대들도 아마 ...
물론 "아니 댁말과 달리 요즘 애들 바보이던데요"라는 얘기도 맞는 말임. 중하위권은 더욱 더 뒤쳐지고 있으니까.
\vspace{5mm}







\section{제가 느낀 10대 후반$\sim$20대 중반의 문제}
\href{https://www.kockoc.com/Apoc/666852}{2016.03.07}

\vspace{5mm}

\textbf{"경험"해보지 않고 남의 판단에 의존하려 한다.}
\vspace{5mm}

재밌는 건 이른바 환락(술담배19금) 같은 건 하지 말라고 해도 잘만 하면서
수험에 있어서 참고서나 강의는 그냥 서슴없이 본인이 들어보고 평가해보는 등 경험해보는 게 중요한데
\vspace{5mm}

경험해보지 않고 \textbf{'너는 어떻게 생각하냐'}라고 물어보는 케이스가 참 많은 것 같습니다.
그래서 입시 분야에서 업자들이 돈을 많이 벌지 않나.... 그런 생각이 듭니다. 적어도 소비자들이 제정신이 아니니 돈을 팍팍 쓸 테니까요.
\vspace{5mm}

자기들이 부딪쳐보지 않는 사람들은 절대 뭔가 창출해 볼 수가 없을 텐데 말입니다.
대학에 들어가서 팀 단위로 과제하거나 프로젝트를 할 때에도 대부분 맨땅에서 시작해야 할 것인데
부딪쳐보지 않고 계속 계산하기만 해서 뭐가 좋을까... 우유부단하게 고민하는 사람들은 나중에 프리라이딩만 하기 딱 좋죠.
사실 이런 사람들은 사회에 나와서 경제활동을 잘 할 수 있을지도 의문이고, 결국 남이 시키거나 가르쳐주는 일만 하겠다는 이야기이겠죠.
\vspace{5mm}

웃지 못 할 현상이지만 인터넷 쇼핑이 그런다죠.
특정한 상품의 이미지와 평가를 보고 하앍... 하면서 구매해서 배송오길 기다림
상품 포장을 뜯을 때까지 행복함, 그런데 상품을 본인이 만져보고는 현자타임, 그리고 그 상품은 훽 던져둠.
소위 이미지와 평가에 중독된 '똑똑한 소비자'들의 자화상입죠.
\vspace{5mm}

50대 중반 이상 꼰대들이 뭐 이런저런 걸로 욕먹어도 절대 현재 젊은 세대들이 뭐라할 수 없는 게
저 세대 분들은 가방끈도 짧고 뭐해도 맨땅에 헤딩해서도 만들어낼 건 다 만들어냈다는 것이죠.
뭐가 좋냐... 그러기보다 직접 부딪쳐서 경험해보고 스스로 판단하고 배워나갔기 때문에 황무지에서 나라를 일궈낸 것이죠.
그러나 뭐 이건 저도 반성해야하지 않나 싶긴 합니다만 앞으로 젊은 세대들이 그럴 수 있을지는 의문입니다.
주어져있을 건 다 주어져있으면 그 꼰대들보다 생산성이 높아야 하는데 정작 경험은 안 하고 평가가 어떻냐... 그것만 보다가 아무 것도 안 함.
\vspace{5mm}

청년들에게 희망이 없는 시대라고 합니다. 확실히 경쟁이 빡세지고 힘들어진 건 맞다고 보는데
한편으로는 과연 시대탓이기만 할까.... 하는 생각도 드는 게 있습니다.
그리고 인생 고난이도는 6.25 이후 폐허에서 시작한 노인 세대들이 최고지 지금은 그에 비하면 훨씬 낫다보이느데 말입니다.
\vspace{5mm}

그리고 일반화시키는 것 같아서 적자면 전반적으로 저렇단 것이지, 지금 10대 후반$\sim$20대 중반도 상위권은 예외입니다.
아래 언급한 미래준비자들의 특징이 강해서 학습 성취라거나 경험 스펙 쌓는 건 정말 무섭습니다.
아재 이거 환타지 쓰는 거지 뇌피셜이지 하면 '이 색기 정말 세상물정 하나도 모르는구나'라고 한숨이나 팍 내쉬어야죠.
학력보다 무서운 게 '계량할' 수도 없는 경험 스펙입니다.
부모들이 이것저것 다양하게 경험시켜주면서  자식들을 \textbf{예비 정치가, 예비 CEO}로 키우는 케이스는 거의 티가 안 나죠.
\vspace{5mm}

다른 이야기지만 대학교에서 수시나 지균러들이 학점이 잘 나온다.... 라고 하는 것도 눈여겨보아야하는데
그럼 대학공부가 잘못된 것이냐 하면 그건 아니죠. 대학공부야말로 사실 본인들이 알아서 해야하거든요.
프리라이딩에다가 족보빨도 있다고 하지만 사실 사회에서의 경쟁상황에서는 저건 사실상 허용되기 때문에 뭐라할 수도 없죠.
지금도 많은 수험생들이 책만 졸라 파서 좋은 대학만 가면 인생 필거야... 라고 하지만 이건 시작일 뿐이고,
대입 후에는 정말 공부 빼고도 온갖 음주가무부터 시작해서 잘 생기고 키도 훤칠하고 성격도 좋고
거기다가 정치에다가 사업까지 잘 할 것으로 보이는 애들이 넘친다는 것,
내가 수험만 죽어라했는데 저 녀석들은 수험 뿐만 아니라 저런 것조차 부모나 친척을 통해 견습받았다는 데에 아주 경악할 겁니다.
\vspace{5mm}

저런 경험 스펙을 이기기 위해선 본인들이 용기있게 이것저것 하지 말라는 것만 빼고 부딪치고 느껴보아야하는데
정작 현실은 하지말라는 건 하지 말라고 해도 경험하고, 해보아야하는 건 이런저런 핑계를 대면서 멀리들하죠
\vspace{5mm}



\section{정치적 낭만주의의 종말}
\href{https://www.kockoc.com/Apoc/668509}{2016.03.08}

\vspace{5mm}

앞으로는 정치적 냉소와 무관심은 더욱 심해질 것으로 보인다.
이건 정치에 관심이 없어서가 아니다, 오히려 정치에 대한 관심이 더욱 깊어졌기 때문에 그렇다.
단지 \textbf{'낭만주의'}가 사라졌을 뿐이다.
\vspace{5mm}

그럼 낭만주의가 사라진 배경 : \textbf{가난}
\vspace{5mm}

이런 가정을 해보았던 사람도 있었을 것이다.
히틀러 같은 독재자가 전쟁을 하지 않거나 최소한으로만 하고 내치에만 신경썼으면 낫지 않았을까.
독재자들은 뭐하러 그렇게 욕심을 부릴까, 현명한 의사결정으로 국민들도 부유하게 키워주면서 떵떵거려도 되지 않을까.
\vspace{5mm}

물론 독재자들도 바보는 아니다.
독재의 비결은 별 게 아니다. 피지배자들을 궁핍하고 무지하게 냅두는 것이다.
북한 독재가 오래가는 것도 막장독재를 했기 때문이다.
북한에 경제적 지원을 해주면 된다... 라고 하는 게 순진한 발상.
쟤들은 절대 인민들이 풍족해지면 안 된다는 것 정도는 알고 있다.
\vspace{5mm}

\textbf{의식주가 충족돼야  '정치적인 욕망'이 생긴다.}
\vspace{5mm}

대학 1학년 때 심리학개론에서 배우는 매슬로우의 욕구단계설 그대로이다.
\vspace{5mm}

북한과 반대되는 경우가 바로 우리나라이다.
박정희 정권 말년이나 전두환 정권 때 민주화 운동이 나름 활발했던 건
역설적으로 저런 독재 체제에서 경제 발전이 성공적이었기 때문이다.
바꿔 말해서 지금은 민주주의도 별 관심도 없고 정치에 대해서도 무관심에 수렴해가는 것은 사는 게 정말로 팍팍해졌기 때문이다.
죽어라 공부해서 학점도 4점대 만들고 인턴질 해도 취업하기 어려운 2010년대
반면 적당히 시위에 참여만 하고 술먹고 연애질해도 과사무실에 제발 취업해달라 대기업에서 홍보하던 1990년대
\vspace{5mm}

독재자들이 일반적으로 막장 통치를 하는 건 저러한 사실을 너무나 똑똑히 알고 있어서이다.
아랫 것들을 궁핍하게 못 살게 하면 '앞가림'하는 데 바뻐서 기어오르지 못 할 것이다라는 것을 안다.
독재자들이라고 하면 총칼만 떠오르기 쉽지만 실제로 그들은 '엘리트 문돌이'들이자 '낭만주의자'들.
어떤 언어를 구사해서 사람들을 휘어잡을지, 그리고 어떤 식의 책략을 써야할지 잘 알고 있는 것이다.
그래서 독재자가 오래 해먹고 싶다면 좋은 정치를 하면 안 된다.
$-$ 김일성도 말글에는 한가닥하는 양반이고 폴포트도 그렇다, 정작 이 분야는 이공계 출신은 그리 많지는 않아보인다 $-$
\vspace{5mm}

그럼 한국은 어떻게 될까.
겉으로는 정치인들을 꾸짖고 이 나라가 부동산에 빠져있다고 비판하는 지식인들조차도
자기 자녀들은 유학보내고 있고 그 자신도 부동산 게임에 골몰하는 데다가 대학에서 갑질하는 경우가 많다.
(여담이지만 그런 지식인들이 대학등록금에 대해서 어떤 태도를 취하는지 찾아보시길. 그들의 진면목이다)
이제는 저성장 시대인 건 부인할 수 없고 어떻게든 판에서 살아남아야한다... 라고 해서 다들 착취에 몰두한다.
물론 착취당하는 젊은이들도 그 판에서 살아남아서 그 다음 젊은 세대들을 착취해서 그 빚을 갚아야지라고 계획 중일 것이다.
그럼 이 판이 뒤엎어질 수 있을까. 사실 불가능에 가깝다.
\vspace{5mm}

이와 같은 추상적인 원리가 현실에서는 사교육비 증가로 나타난다. '학생부 스펙'을 위한 사교육까지 창궐한다.
빚을 내더라도 순위 안에만 들면 나중에 갚을 수 있을 거이다라고 다들 기대를 한다.
단지 무서운 건 어느 순간 그 모든 것이 \textbf{'폭락'}하지 않을까라는 불안감 하나일 뿐이다.
\vspace{5mm}

그런데도 아직까지 다수가 정치에 관심이 없다고 하는 참 무지한 사람들이 많다.
그 사람들은 '가난해져버린 사람'들이 이제 정치적 낭만주의를 버렸다는 것을 인정하지 않으려고 한다.
세상이 바뀌어서 다들 아프리카 방송이나 게임 방송을 보고 해외 드라마를 시청하며 주말에 가끔 대형영화관 가는 세상인데
비싼 돈을 내고 자신들의 열정이 담긴 연극을 보라고 강요하는 꼴인데 이걸 누가 믿겠는가.
거기다가 그 정치적 메시지에는 실제로 중요한 사회의 "룰"을 제대로 고치는 건 없다.
\vspace{5mm}

그럼 룰의 개혁은 약자에게 유리한가.
이게 중요한 대목이다.
\vspace{5mm}






\section{딜레마 : 진보는 강자들에게 유리하다.}
\href{https://www.kockoc.com/Apoc/668589}{2016.03.08}

\vspace{5mm}

콕콕에서도 가정환경이 힘든 친구들과 적지않게 상담하는 편이다.
물론 내 경우는 잔인한 해답으로 충고를 해주지만 사실 그 뿐이다. 물질적인 지원까지는 할 수가 없기 때문이다.
다만 "그래도 나는 애들 위하는 척 하면서 장사는 안 하잖아"라고 얘기하겠고 사실 그 뿐이다.
\vspace{5mm}

그런데 이들조차도 어떻게 보면 2000년대에 벌어졌던 진보 실험의 피해자일 수도 있다는 생각이 든다.
다시 경고하지만 이걸 보고 '정치적 성향에 거부감 느낀다'라고 생각되면 그냥 뒤로 가기를 눌러주길 바랄 뿐이다.
\vspace{5mm}

일반적으로 \textbf{진보가 서민에게 유리하다} 라고 믿고 있는 것 자체가 엉터리이다.
사실 이런 간단한 거짓말을 다들 간파하지 못 한다는 것 자체가 놀라운 일이기도 하지만,
그만큼 특정한 메시지의 주입은 사람들의 사고를 마비시킨다는 중요한 얘기가 되지 않을까 싶다.
\vspace{5mm}

많은 청년들이 나중에 당황한다. 서민들이 진보를 지지하고 부자들은 보수를 지지해야 한다고 믿는다.
그러나 현실은 서민들이 보수를 지지하고, 오히려 강남부자(특히 젊은 자제들)들이 진보에 호의적인 딜레마가 발생한다.
거기에 대해서 사람들은 국개론을 펴거나 서민들이 우매해서라는 정신승리를 하기 시작한다, 참 한심한 이야기다.
\vspace{5mm}

그런데 수험생들은 왜 그럴 수 밖에 없는지 알지 않는가.
여태까지 모든 입시제도는 "교육의 진보"를 위해 변경되어왔다. 수시든 입학사정관제든 어떤 것이든 표면적 명분은 좋았다.
아니 실제로 그것들이 정말 입안자들의 진심이 담겨있었을지도 모르는 것이다.
\vspace{5mm}

그러나 이런 것들은 모두 무의미해진다.
룰이 바뀌었을 때 빨리 적응하고 이용할 수 있는 사람들이 서민들이겠나? 아니면 \textbf{똑똑한 사람들이나 부자들이겠나}
사회적 약자를 배려한다는 온갖 제도들조차도 똑똑하고 부유한 학부모들이 이용하는 일이 벌어진다.
물론 사람들은 이렇게 얘기할 것이다. 그건 소수에 지나지 않는다고.
그런데 그 소수의 예라는 게 그나마 '입시경쟁'의 일정 부분 순기능이 있던 부와 지위의 세습 차단 효과를 무색하게 만든다는 건 다들 알고있다.
\vspace{5mm}

입시만 해도 이 정도인데 그럼 경제, 노동은 어쩌겠나.
이건 내 성향이지만 내가 스스로 진보에 표를 던지는 일은 없을 것이다.
그 이유는 그런 진보적인 것 자체가 빈부격차를 벌이고 부자들을 더욱 부유하게, 서민들을 더욱 노답으로 만든다는 걸 경험했기 때문이다.
심지어 파업 같은 것도 단지 대기업$-$노동자의 틀만 강요하는데 실제로는 "정규직"들이 자신들의 지위를 더욱 공고히 하고
나머지 하청업체의 비정규직들은 나가 죽으라고 하는 진심이라는 건 그리 알려져 있지 않다.
2000년대 당시야 그럴 리야 있겠느냐 하겠지만 지금은 어떤가, 이런 게 다 드러나지 않았나.
\vspace{5mm}

현재 보수에 표를 던지는 서민들은 그런 데 정말 제대로 치여버린 사람들이다. 그래서 룰이 가능하면 변경되지 않길 바란다.
다시 말하지만 룰이 변경되어버리면 서민들이 해왔던 그간의 노력이 수포로 돌아가는 일이 벌어진다.
반면 정보력이 빠르며 제도나 법률을 악용하는 강자들일수록 더욱 해먹을 기회가 많아진다.
그래서 겉으로는 서민들을 위한다고 하면서 진보를 추종하며 착한 척 하는 것이다.
이거야말로 뇌피셜이 아니냐고 하겠지만 장기적으로 볼 때 그들은 절대 자기 것을 양보하지 않는다는 것을 알 수 있다.
\vspace{5mm}

다시 교육제도로 돌아가보자. 돈없고 빽없는 학생들에게는 정시 확대야말로 최고다.
물론 돈있는 애들이 앞서나가는 경향이 없지는 않지만, 적어도 정시는 노력으로 커버칠 수 있는 영역이란 게 있다.
그러나 절대 힘있는 자들은 이걸 바꾸지 않을 것이다. 특히 '공교육 활성화'라는 명목 하에서 절대 정시확대를 하려하지 않을 것이다.
물론 그것이 빈익빈 부익부를 더 늘리는 건 자기들 알 바는 아니며
오히려 대학에서는 이런 걸 더 반길 수가 있다. 대학도 '힘있고 부유한 학생'들이 입학하길 원하지 더 이상 개천 용은 바라지 않기 때문이다.
\vspace{5mm}

현재 교육제도든 하다 못해 대중적인 교양서든 "룰을 바꿀 때 차익을 챙기는 방법"에 대해선 잘 소개되지 않았다.
그래서 사람들은 세번 사기당한다.
\vspace{5mm}
\begin{itemize}
    \item 첫째는 공교육에 사기당한다.
    \item 둘째는 공교육이 기득권의 프레임이라고 하는 정치적 낭만주의자들에게 사기당한다
    \item 셋째는 투자만 잘 하면 부유해질 수 있다라고 하는 자칭 투자의 달인들에게 사기당한다.
\end{itemize}
\vspace{5mm}

적어도 '속지만' 않는다면 공부의 성과는 톡톡히 누리는 것이라고 이야기할 수 있다.
\vspace{5mm}






\section{인공지능}
\href{https://www.kockoc.com/Apoc/669606}{2016.03.09}

\vspace{5mm}

인공지능의 발전으로 로봇은 이제 변호사, 회계사, 금융 애널리스트, 의사 등 전문직 영역까지 넘보는 수준으로 진화했다.특히 방대한 양의 문서들이 오가는 법률 서비스 영역에서 인공지능의 역할은 더욱 기대되고 있다. 변호사의 경우 재판에 들어가기 전 수백, 수천장의 문서를 일일이 읽고, 새로운 사실관계를 찾아내 그에 맞게 서류를 다시 작성한다. 대량의 정보를 분석하고 처리하는 데 강점이 있는 인공지능이 이를 대신하게 되면 시간과 비용 면에서 보다 효율적일 뿐 아니라 정확성 면에서도 사람을 능가할 것으로 업계는 보고 있다.실제 미국의 신생 벤처기업 ‘주디카타(Judicata)’는 법리와 판례 등이 담긴 문서를 구조화된 정보로 바꿔주는 기술을 개발해 변호사들의 업무를 일부 대체하고 있다. ‘부당 해고를 당한 히스패닉계 동성애자 남성’에 관한 기존 판례를 모두 찾아줘 변호사가 도서관을 직접 가거나 검색하는 데 드는 막대한 시간을 절약해주는 식이다.
\vspace{5mm}

이제 현실로 다가왔습니다. 아마 이 잡글을 읽는 분들이 부딪칠 문제이니 각오하셔야하지 않을까
\vspace{5mm}

\href{http://news.naver.com/main/read.nhn?mode=LSD&mid=shm&sid1=105&oid=015&aid=0003544035}{링크}
\vspace{5mm}

4차 산업혁명은 제조업과 ICT의 융합이 골자다. ‘인더스트리(Industry) 4.0’으로도 표현되며 우리나라가 추진하는 ‘제조업혁신 3.0 전략’과 개념이 비슷하다. 빅데이터, 인공지능(로봇), 사물인터넷은 4차 산업혁명의 엔진이다. ‘스마트(smart)’는 4차 산업혁명의 핵심 키워드다. 기계들은 갈수록 똘똘해진다. 스스로 알아서 일을 처리하는 기계가 늘어난다. 자동화보다 기계의 기능이 업그레이드되면서 생산성은 더 높아진다. 사람의 손을 대체하는 기계도 빠르게 늘어난다. 4차 산업혁명은 곳곳에서 진행 중이다. 정보기술(IT)은 물론 자동차, 바이오, 의료, 사물인터넷, 인공지능, 가상현실, 증강현실 등으로 빠르게 확산되고 있다. 디지털은 4차 산업혁명을 이끄는 중추다.디지털에 기반한 ‘자율주행 자동차’(무인 자동차)의 상용화시대는 성큼성큼 다가오고 있다. 이 분야 선두는 구글이다. 자동차 메이커가 아니라 인터넷 검색 업체가 무인 자동차 개발을 선도하고 있다는 사실은 4차 산업혁명의 엔진이 ‘소프트웨어’라는 것을 말해준다. 제너럴모터스(GM), 도요타, 현대자동차 등 기존의 글로벌 자동차업체들도 무인 자동차 개발에 경쟁적으로 나서고 있다. 4차 산업혁명 시대에는 ‘소프트웨어 강자’가 혁신을 주도한다.기계들이 서로 연결되다사물인터넷(IoT:internet of things)은 4차 산업혁명이 몰고올 변혁적 풍경이다. 가전제품·전자기기뿐 아니라 헬스케어, 원격진료, 스마트홈, 스마트카 등 다양한 분야에서 사물(기기)들이 서로 연결돼 정보를 공유하고 상황에 맞춰 일을 처리한다. 냉장고가 주인 마음을 읽고 필요한 물건을 주문하는 식이다. 영화 같은 얘기지만 인간의 현실로 하나둘씩 파고드는 기술이다. 사물인터넷은 초연결사회다. 인간이 인터넷을 통해 기계와 연결되고, 기계와 기계가 서로 연결되는 사회다. 스마트폰의 진화는 IoT가 어떤 모습으로 세상을 바꿀지 어느 정도 짐작할 수 있게 한다.4차 산업혁명은 ‘빅데이터 시대’의 도래를 의미하기도 한다. 빅데이터의 특징은 방대한 정보량, 엄청나게 다양한 정보형태, 초고속 전파속도, 새로운 가치 창출이다. 데이터의 중요성은 갈수록 커지고 있다. 데이터가 바로 자본인 시대다. 개인이든, 기업이든 데이터를 잘 활용해야 앞서간다. 빅데이터는 인간의 사고방식에도 영향을 미친다. 빅데이터의 대가 쇤베르크는 \textbf{“빅데이터 시대에는 인과관계에 집착하는 사고의 습관을 버리라”고 강조한다. 빅데이터가 어떤 지표나 성향을 보여주면 ‘왜’라는 데 지나치게 집착해 시간을 낭비하기보다 그 연관성을 빨리 받아들이고, 거기에 대처하는 게 바람직하다는 것이다.}
\vspace{5mm}

어린 시절에야 인터넷이라는 게 정말 상용화될지 상상도 못 했습니다.
그런데 정말 1$\sim$2년 사이에 다 바뀌더군요.
사람들이 손바닥만한 것으로 영화를 보고 다니면서 화상통화한다... 뭐 이미 과거가 되어가고 있고
\vspace{5mm}

기사에서 드러난 인공지능은 이제 '상용화'를 앞둔 것이라고 하겠는데(대표적으로 무인자동차)
거두절미하고 이건 사람 vs 인공지능이 아니라, \textbf{인공지능을 지배하는 사람 vs 인공지능에 지배당하는 사람}이라고 보는 게 타당할 겁니다.
저기서 굵은 글씨 처리한 쇤베르크의 이야기가 상당히 의미심장한데
저렇다면 오히려 이공계 사람들이 더 시대착오적인 가치관으로 전락할 수도 있단 이야기입니다.
이공계의 장점이 핵심만 찾아 인과관계를 추론하는 건데 빅데이터 시대에는 그게 단점이 될 수 있기 때문이죠
\vspace{5mm}






\section{실패를 미리 경험해보라고 하는 어른들}
\href{https://www.kockoc.com/Apoc/670071}{2016.03.09}

\vspace{5mm}

젊었을 적에 실패를 경험해보아야한다... 라고 말은 하는데 정작 그 메신저들은 잘 알고 있는가 하는 의심이 든다.
생각해보자면 소위 루저들은 이미 실패를 밥먹듯이 경험하고 있지 않나.  그래서 루저라는 게 바뀌기는 한 것인가.
더군다나 금수저라고 일별되는 사람들이 정작 실패 경험은 적은 것은 무얼 이야기하는 걸까.
마리텔에서 뜬 다음 편의점에 온갖 사진이 도배되는 한 요식업체 경영자님도
왕년에 사업에 크게 실패했다고 하지만 집안 배경이 막강한 점을 본다면 이게 진짜 실패일까... 라는 점에서 의구심이 안 들 수가 없는 것이다.
\vspace{5mm}

오히려 한번 실패한 다음에 그걸 수습하지 못 하면 그게 트라우마가 되어 당사자를 계속 실패의 구렁텅이에 빠뜨리는 패턴이 흔하지 않나?
잘 나가던 사람이 한번 삐끗한 다음에는 재기불능 상태에 빠졌다라는 것도 어디든 가면 흔히 들을 수 있는 이야기이다.
그리고 개인적으로는 상담이라는 걸 해주다보면 분명 상대방이 실패하는 패턴이라는 게 있는데
상대방은 그게 망하는 길인줄도 모르고 있고, 무엇보다 그걸 고치라고 조언하면 화부터 내는 케이스도 없지 않아서
\vspace{5mm}

제대로 말해야지, "일찍 실패를 겪어보되 그걸 \textbf{제대로 수습하고 교훈을 얻어야만} 자산이 된다"
그리고 여기서 불편해진다. 10대들이 실패하는 경우 그럼 누가 수습하고 가르쳐줘야하나, 그건 부모님들이 아닌가?
불편한 진실이지만 자녀에게 공부하라 이것저것 경험해보라 말로만 그러지, 실제로 사건 터졌을 때 수습해준다거나
실질적으로 지원해주는 사람이 몇이나 되겠나.
\vspace{5mm}

실패를 위로하는 것 : 사실 위로야말로 가장 싸가지가 없는 짓이 아닌가. 왜냐면 위로는 \textbf{돈이 안 들거든.}
돈이 들지도 않고 뻔한 소리 하는 것만으로도 상대의 아픔을 치유해줄 수 있는 \textbf{'착한' 사람이 될 수 있는} 게 위로의 순기능이다.
위선자가 되고 싶지 않으면 정말 물질적으로 돕거나 그 일을 해결하는 데 기여해주거나, 아니면 걍 아무 말도 안 하는 게 깔끔하다.
더군다나 실패한 애들이 '위로받는 쾌감'에 빠져버리면 그건 불행중독과 비슷한 루트를 밟아버리고야 만다.
젊었을 적 고생은 돈주고 산다... 라고 말이 들리면 일단 그 말을 어떤 어르신께서 하나봐야한다.
물론 메시지는 메신저에 구애받지 않는다, 그러나 메시지의 진정성은 그 메신저로 결정된다.
\vspace{5mm}

남의 아이에게는 공부하지 말고 거리로 나가서 굴러봐라, 실패를 많이 해보라고 하는 상류층 인사가
자기 자식들도 과연 그렇게 비효율적으로 굴릴까.
하다 못해 고생시키는 경우라도 최악의 상태로 가지 않게,
그리고 다시 일어설 수 있게 지원해준다.
\vspace{5mm}

난 노오력주의자이긴 한데 무조건 노오력한다고 결과는 나오지 않는다고 생각한다.
아래에서 말했지만 진보(?)를 혐오하는 건, 진보를 핑계로 룰을 바꿔 타인들의 노력을 수포로 돌아가게 하는 사례가 빈번했기 때문이지만.
불행중독자들나 실패중독자들은 노오력만으로는 부족하다, 이 친구들은 잘못된 패턴 자체를 다 갈아엎어야 한다.
\vspace{5mm}

그런데 그 잘못된 패턴이 어디서 비롯되었을 것 같나. 혹자는 이렇게 말하겠지, 성공에서 우러나온 자만심.
물론 그것 때문일 수도 있다. 하지만 이건 소설, 만화, 애니에 자주 나오는 것이라서 과장된 것 뿐이다. 일단 성공할 확률부터가 낮지 않나.
그보다는 아예 조명되지조차 않는 실패의 트라우마부터 언급하는 게 더 타당함에도 이건 얘기되지도 않는다.
구체적으로 예를 들면 한번 실패한 사람은 과거에 집착하는 성향도 강해지고, 냉정한 판단보다는 기도에 의존하게 되며
실패로 받은 상처가 한이 되어 감정적이 되기 쉬운데다가, 그 책임을 타인에게 전가시키려는 경향이 강해지는데
\vspace{5mm}

젊은이들보고 고생해보라하는 노인분들은 이런 것까지 다 얘기해주진 않는다.
\vspace{5mm}

꼰대들의 이야기도 가려 들을 건 가려 들어야한다. 그게 얼마나 간지나냐가 그런 걸로 따지면 안된다.
\vspace{5mm}
\begin{enumerate}
    \item  그 사람이 돈을 내는가
    \item  그 사람이 정말 실천해주는가
\end{enumerate}
\vspace{5mm}

성인군자인 척 그런 말을 하는 사람들에게 나이먹고 추궁해보니까 반응이야 굳이 적을 필요조차 없을 것이다.
\vspace{5mm}






\section{인공지능이 무서운 점.}
\href{https://www.kockoc.com/Apoc/670160}{2016.03.09}

\vspace{5mm}
\begin{enumerate}
    \item 나이를 먹지 않는다 $-$ 즉 원칙적으로 수명이 무한정.
    그에 비해 인간은 2$\sim$30년 교육시켜서 20년 굴리면 많이 굴리는 것이고 그 외는 사실 쓸모가 없어서리.
    \vspace{5mm}
    
    \item 감정이 없다
    단점도 없지 않겠지만 사실 장점이 더 많습니다. 냉정히 말하면 인간의 감정 때문에 벌어지는 실패와 참사가 더 많은지라.
    \vspace{5mm}
    
    \item 욕심이 없다
    뛰어난 인간은 욕심 때문에 다른 길로 빠지거나 사리사욕을 챙기지만 인공지능이 그럴 일이 없다는 것.
\end{enumerate}
\vspace{5mm}

이것말고도 많을 것 같은데 적어보니까 그냥 '신' 아니면 '부처님'이네요.
수명 무한정이고 감정 문제가 안 겪고 무욕이니.
\vspace{5mm}

그런데 이걸 기계가 인간의 역할을 뺏는다, 즉 대체한다고만 보는 건 단순한 + $-$ 셈법이 아닌가 싶은데.
여태까지는 인간+인간+인간+... + 인간인 덧셈식 발상이었다면
이제부터는 인간 X 인공지능 : 이런 곱셈적 발상으로 가든가 아니면 $\textrm{인공지능}^\textrm{인간}$, 혹은 $\textrm{인간}^{인공지능}$이라는 지수적 발상으로 가야하는 게 아닌가.
인간이 자기 영역을 뺏긴다기보다는 이제 새롭게 진화하지 않으면 안 된다라고 생각해야 할 듯.
\vspace{5mm}






\section{메이드 로봇의 꿈이 멀지 않았다.}
\href{https://www.kockoc.com/Apoc/670931}{2016.03.09}

\vspace{5mm}

걷어차일 때 마음이 아프죠.
\vspace{5mm}

기계음이 나서 그렇지 만약 사람 스킨만 씌운다면야
\vspace{5mm}

4족 보행로봇입니다. '승마'의 개념이 좀 바뀔지도 모르죠
\vspace{5mm}

보다시피입니다.
\vspace{5mm}

일단 여기 결합되어야 하는 것은
\begin{itemize}
    \item[$-$] 3d 프린팅
    \item[$-$] 양자 컴퓨터
\end{itemize}
까지 있어야합니다.
\vspace{5mm}

3d 프린팅이 상용화되지 않으면 로봇사회의 수급을 지탱할 수 없고,
양자 컴퓨터까지 가야만 인공지능이 원활하게 돌아가고 발전할 수 있기 때문이죠.
저것들에게 일자리를 빼앗긴다... 라고 생각하기보다는,
이제 한 개인이 저런 로봇들을 부리며 더 많은 일을 해야하는 시대가 온다 생각해야할 것입니다.
\vspace{5mm}






\section{인공지능처럼 공부하면 된다.}
\href{https://www.kockoc.com/Apoc/672926}{2016.03.11}

\vspace{5mm}

인공지능처럼 공부하면 되죠.
\vspace{5mm}
\begin{enumerate}
    \item 좋은 알고리즘을 갖춘다
    \item 학습패턴을 늘린다(양치기)
    \item 감정과 욕망을 자제한다
    \item 최대한 많은 경우의 수를 고려한다.
    \item 아름답지 않더라도 이기는 길만 궁리한다.
\end{enumerate}
\vspace{5mm}

그에 비해 인간은
\vspace{5mm}
\begin{enumerate}
    \item 나쁜 알고리즘에 집착한다
    \item 양치기를 피하려 한다.
    \item 감정과 욕망을 억제하지 못 한다
    \item 특정한 경우에 집착한다(특히 과거집착)
    \item 아름답게 지는 방법에 도취한다
\end{enumerate}
\vspace{5mm}

개인적으로는 알파고의 2연승을 지지하는 건데
이건 역설적으로 소위 머리좋다고 해서 기득권 해먹으려는 사람을
'소박한 인공지능'이 제대로 엿먹인 케이스라 그렇습니다.
\vspace{5mm}

앞으로도 이런 일은 비일비재하게 발생할 수 있다는 이야기
말이 인공지능이지 실제로는 인지컴퓨터라고 보는 게 맞고
이 인공지능의 발전에는 '인지 과학'의 역할이 매우 큰데,
이건 신성화되고 신비주의화된 학습법을 과학화한 것이죠.
\vspace{5mm}

인류 사회의 진보는 저 인공지능이 담당할 겁니다.
적어도 말도 안 되는 아래와 같은 핑계를 대진 않겠죠.
\vspace{5mm}

\href{http://news.chosun.com/site/data/html_dir/2016/03/11/2016031101156.html}{링크}
\vspace{5mm}

그런데 문제는 인공지능이 인간보다 우월하다는 게 문제가 아니라
인간이 생각하던 이상적인 것들 $-$ 윤리, 아름다움, 진보 등을 정작 인간이 실천 못 하는 데 인공지능은 해버린다는 게 문제고
학습 역시 그렇습니다.
\vspace{5mm}

전에 수업시간에 인공지능에 대해서 배웠을 때 강조되던 게 '학습'이었는데 이것이 원래 인간의 그것을 따온 것이라는 것.
그럼 이런 의문이 들죠. "아니 인간이 했으면 왜 인간은 이걸 실천 못 하는 거야?"
\vspace{5mm}






\section{인공지능 소녀 vs 아이돌}
\href{https://www.kockoc.com/Apoc/672940}{2016.03.11}

\vspace{5mm}

vs
\vspace{5mm}

저 시대가 오면 아이돌도 끝장날 것 같은데(...) 참고로 저 인조인간 소녀는 파이브스타스토리즈의 발란쉐 파티마 휴트랑.
 최약체 주인을 꼬셔서 강자와 승부하는 식으로 주인을 학대하는(...) 
 그 장면은 여기 http://www.gearsonline.net/forum/viewtopic.php?t=2069안 그래도 파이브스타스토리즈에는 
 현실 여성들이 저 파티마들을 극혐(질투)해서 신체 노출에다가 복장까지 온갖 규제를 다 하는 설정이 나오긴 합니다만.
 그렇다고 남성 파티마도 없는 건 아닌데 왠지 비중이 적죠물론 중 최강 기사가 파티마였더라하는 것.
 \href{https://namu.wiki/w/%ED%8C%8C%EC%9D%B4%EB%B8%8C%20%EC%8A%A4%ED%83%80%20%EC%8A%A4%ED%86%A0%EB%A6%AC}{링크}
 
 그런데 지금도 연재가 안 끝나기는 커녕 
 \textbf{1}00년이 지나도 변함없을 것 같은 게 더 공포인데과학 좋아하는 분들은 공부 안 되면 저 만화 소장해보시면 됩니다
 (왜 빌려볼 수 없는지는 직접 보시면 아실 테고)생각보다 꽤 심오한 작품이라서리(짜깁기성이 없는 건 아니지만 짜깁기도 참 예술적으로 해서)투명드래곤급 설정도 이렇게 말이 될 수가 있구나 생각되죠.





\section{금융, 제조, 실업이 궁금하면}
\href{https://www.kockoc.com/Apoc/674055}{2016.03.12}

\vspace{5mm}

ktbook.com 고교 교과서에 보면
\vspace{5mm}

\href{http://www.ktbook.com/Shop/Online/BuyMainList.asp?BookGroup=61005&GroupName=전문교과}{링크} (ㄱ $\sim$ㄹ)
\vspace{5mm}

여기 가보면 다양한 서적들이 있는데
\vspace{5mm}

관심있는 몇 개의 교과서를 주문해보니 내용이 훌륭함.
사실 집필진들도 그 분야에서 내노라하는 분들입니다.
\vspace{5mm}

특목고로 분류되는 마이스터고야 이미 떴지만 공고 상고 등이 더 좋아질 거라고 보는 이유입니다.
현재까지는 학력이 낮아서 기피된다고 하지만 과거 외고의 선례로 보다시피 언제든지 뒤집힐 수 있는 것이고
지금 기성세대들도 '기술'이라도 익혀두면 입에 풀칠이라도 할 수 있지 그 분위기이기도 하지만
무엇보다 교과서 내용들이 대단히 훌륭합니다.
\vspace{5mm}

그에 비하면 사설인강들의 교재라는 건 사실 기출과 잡개념 짜깁기이고(...)
님들 배우는 국영수라는 건 과거에 비해서 많이 가위질당한 것이거니와 전체적으로 하향되어온 것도 부인할 수는 없습니다.
\vspace{5mm}

그리고 이제는 \textbf{"메이커"의 시대}로 가는 추세이죠.
인공지능이 위협하는 건 의사, 변호사, 회계사 등인데 이 직업들의 공통점은
방대한 데이터베이스들을 숙달시켜 '판단'하는 것입니다.
바꿔 말해서 아직까지 인공지능은 이 정도 밖에 못 하고 사실 지금도 그 정도 밖에는 하지 못 합니다.
\vspace{5mm}

역설적으로 인공지능이 침해하지 못 하는 최후의 분야는 '종교'입니다. 왜냐면 종교는 비합리적이기 때문.
그건 '제조' 역시 마찬가지입니다. 새로운 걸 설계하고 제작한다는 것의 초기 과정 자체는 \textbf{낭비}이기 때문입니다.
\vspace{5mm}

그 점에서 지금 실업계는 저평가되었죠.
\vspace{5mm}

다수의 2, 30대들이 자기들 명문대에 갔으니까 현재의 5, 60대들처럼 떵떵거릴 시기가 언젠가 올거야라고 기대하는 건데
과연 그런 시대가 올 수 있을지. 5, 60대야 고성장시대니까 자리가 많아서 문제였긴 하지만
지금은 학벌 뿐만 아니라 '빽'이 없으면 들어갈 수 없고, 꼬우면 자기가 창업하는 수 밖에 없는데 창업을 배운 적은 없잖습니까.
\vspace{5mm}






\section{신용}
\href{https://www.kockoc.com/Apoc/675112}{2016.03.13}

\vspace{5mm}

추한 것을 혐오하지 않는 자는 아름다운 것을 칭찬할 수 없고
나쁜 것을 증오하지 않는 자는 좋은 것을 사랑할 수 없으며
화내지 않는 자는 사실 싸우지 조차 못 한다.
\vspace{5mm}

나이 처먹으면서 소위 어른들 말씀이 얼마나 옳은가 그것도 검증해보았는데
"좋은 게 좋은 거야"하는 것은 잘 가려듣지 않으면 오해할 수 있다.
저기서 좋은 게 좋은 거야... 라는 것은 적당히 눈치 보아서 '굳이 싸울 필요 없는 것'은 싸우지 말라는 이야기 정도로 그쳐야지,
저걸 양비론적인 판단을 정당화하는 걸로 가거나 본인의 호불호를 버리는 걸로 가리는 말라는 이야기다.
\vspace{5mm}

말이라는 것도 그런데
인간관계 손해를 보면서도 약속에 연연해야하는가,
아니면 인간관계가 좋은 게 좋은 거야 하면서 약속을 저버리는가.
\vspace{5mm}

전자가 미학적으로도 그럴 듯 해보이지만 사실 '실리적'으로도 좋다는 판단이다.
상대에게 좋다고 해서 약속을 어기거나 싸바싸바하고 가면 상대가 결국 나를 우습게 보게 되어있다.
왜냐면 약속을 어기는 순간 그건 '신용'이 날라가는 것이고, 신용이 떨어지는 사람의 가치는 낮아지게 되어있기 때문이다.
아무리 사람이 돈이 많다고 해도 소용없다, 그 사람이 신용이 없는 사람이면 언제든지 여포처럼 배신해먹을테니까.
\vspace{5mm}

어떻게 하면 좋은 사람을 만나냐 질문하는 데 사실 하나마나한 질문.
본인부터가 자기가 한 말에 대해서 일단 책임을 지는 자세를 지고, 특히 중요한 것은 약속을 지키며
상대가 약속을 어기면 칼같이 잘라버리거나 불이익을 주고 하는 것만 지키면 된다.
본인이 신용을 갖추면 모여드는 사람들도 신용있는 사람들이니 서로 협조해나갈 수가 있다.
반면 본인이 신용을 버리면 신용있는 사람들도 사라지고 신용없는 인간들끼리 모이니 서로 통수나 먹이게 되어있다.
\vspace{5mm}

그런데 중요한 건, 이것은 가족간에도 칼같이 지켜야 하는 것이다.
가족관계이기 때문에 신용은 신경쓰지 않아도 된다... 라는 잘못된 통념이 퍼져있다.
그래서 실제로 부모가 자식 통수를 때리고, 자식이 부모 통수를 먹이고 하는 경우가 생각보다 많다.
가령 금전적 지원을 받는 n수생이면 공부만 하는 것이 신용을 지키는 일인데 그런 사람이 몇이나 될까.
정반대로 자녀에게 뭔가 약속하고 그걸 어겨먹거나 그저 자녀를 체면의 도구로 보는 부모들도 많다.
\vspace{5mm}

신용을 지키리면 잘라버릴 건 잘라야 한다.
하다 못해 상대가 부모라고 할지라도 약속을 어기면 화를 내야 하고, 반면 자기가 약속을 어기면 그 대가는 치러야 한다.
여기서 좋은 게 좋은 거야... 라고 하다간 그 다음부터는 자기도 모르는 사이에 계속 추락해나가는 것이겠고
어그로 끌자면 지금 자기 처지가 개막장이라고 하는 사람들,
아마 가족들도 신용이 있지도 않겠지만 본인도 과연 신용이 있느냐 하면 그건 아닐 것이다.
자기 길을 잘 가는 사람은 단순히 노력을 하는 케이스가 아니다, 말이 천금의 가치가 있는 사람,
즉 그 사람의 말이 법이나 다름없는 사람이지.
\vspace{5mm}

약속을 지키지 않으면서 자기 처지만 호소하는 사람이면 $-$ 특별한 사정이 없다면 $-$ 내 경우는 일단 멀리할지 않을까 싶은데
인정에 호소하지만 신용이 없는 사람은 그런 인정에 호소하는 마음으로 타인을 엿먹이기 좋고 실제로 많이 당해보았다.
자기가 얼마나 불우한가 비참한가 하소연하는 사람들을 안 겪어보앗겠냐만 문제는 그래놓고 통수치는 경우가 많았단 것이다.
통수를 치지 않는 사람들은 반면 필요한 말만 하고 약속을 꼭 지킨다. 그런 사람들은 정말로 무섭다, 한마디가 법이니까.
하지만 어떤 친분을 강조한다고 해도 약속을 어기는 사람들은 그 말이 결국 '거짓말'인데 내가 좋게 볼 수 있을 리가 없다.
\vspace{5mm}

이십대 후반이 되기 전까지는 정말 인정에 휩쓸렸고 참 많이도 얻어맞았는데
위와 같은 가치관대로 가면서 사람들과 거리를 두면서 자를 사람은 잘라버리고 신용있는 사람에겐 조심하고 하니
그 다음부터는 정말 갈등이 줄어들어 해피해졌다.
\vspace{5mm}






\section{M갈리안 현상}
\href{https://www.kockoc.com/Apoc/675258}{2016.03.13}

\vspace{5mm}

이른바 M갈리안 현상 $-$ 한남충 비하라는 것에 충격을 먹는다거나
그것들이 새롭다라고 하는 것은 그 남자들이 순진하게 살아왔다라고 말하는 것이나 다를 바 없다는 것.
(이런 남자들은 와타나베 준이치라는 의사이자 작가의 수필을 읽어보길 권함(통찰력이 꽤 좋은 사람이다))
\vspace{5mm}

얼핏 생각하기에 남자 > 여자인 것 같지만
생존력, 공격력을 포괄해 소위 독하다라는 개념으로 치면 사실 \textbf{남자는 여자에게 상대조차 되지 않는다}.
무엇보다 남자들은 카사노바가 아닌 한 꾸미거나 거짓말을 칠 줄 모른다, 이건 선량하기보다는 무능에 가까운 개념,
그러나 여자들은 기본이 일단 자신을 꾸미고 보는 것이다.
거기다가 조직 차원으로 갔을 때 온갖 권모술수와 정보능력, 소위 정치에 해당하는 것도 역시 남자가 여자를 따라갈 수 없다.
\vspace{5mm}

M갈리안 현상은 '뭐 당연한 것 아냐'라고 할 수 밖에 없는 게
일본 만화나 애니를 단속하고 게임을 공격한 YWCA 아줌마들에게서 이미 드러났다.
미용실 계모임(...)이야 말할 것도 없고 당연히 엄마들이 주류일 수 밖에 없는 학부모 모임만 보아도 그렇다.
여자들이 결혼하고 엄마가 되면 모성애 때문에 지독한 게 아니냐... 그런 게 아니다.
그냥 원래 저게 \textbf{본모습}이었던 것이다.
그리고 우리는 여자들이 얼마나 무섭고 똑똑하냐를 알 수 있는데
월스트리트 금융 엘리트들을 능가하는 게 우리나라 복부인이라는 것을 아시나.
일선 여교사부터 어린이집 원장에다가 여교장, 거기다가 약사나 공무원 등이 밤새 부동산 토론하고 스터디하는 복부인 집단,
우리나라 최고의 두뇌 집단이라고 할 수 밖에 없다.
우리나라의 부동산 관련 규제법률이 결국 복부인들과의 싸움인 것이다.
\vspace{5mm}

이런 것을 알고 나면 페미니즘은 참 공산주의만큼이나 허황된 것임을 알 수 있다.
사실 잘 나가는 여자들이 페미니즘에 관심이 있거나 신경쓰기조차 하느냐는 건 사실 의문,
학생 때야 좀 관심 갖지만 나중에 사회물 먹고 본격적 돈벌이하면 그런 때가 있었지... 라는 건 공산주의와 똑같은 취급을 받는다.
오히려 불편한 진실은 진짜로 남녀평등에 관심이 있기보다는 그냥 '공격용 무기'로서 페미니즘을 써먹으려 했다... 라는 건데
세상은 의외로 공평해서 여성들이 차별받는 분야도 있지만 반면 남자들도 역차별받는 분야, 즉 여자들이 이득만 보는 분야도 많다.
물론 남자들이 자기가 이득보는 분야를 말하지 않듯, 여자들도 그런 걸 인정할 리가 없는 것이다.
\vspace{5mm}

원래 여자들이 남자보다 강하고 무섭다라는 건 여자들도 잘 알고 있다. 내색하지 않는 것이고
그럼 결론 : 그냥 이런 걸 모르고 엄마가 해주는 밥이나 먹고 살아온 \textbf{마마보이 남자들이 병신이다} 그 이야기지.
\vspace{5mm}

거기다가 IMF 이후부터는 가부장주의가 한물가버린 데에다가 가정에서 남자가 위축되는 게 두드러지고 엄마들 목소리가 커진다.
그걸 보고 자라는 아들 딸이 어떻게 바뀌었는가 하는 것도 생각해봄직한데
그래서인가 연령대가 어려질수록 여자애들은 정말 강해지는데 남자애들은 소심해지기 시작한다.
일종의 '섬세함'이라고 하는 게 오히려 남자애들에게서 자주 발견되는 것을 알 수 있다.
남학생 중에서 '희소한 마초'라고 할 수 있는 애들 $-$ 즉 남자답고 씩씩한 경우는 10명 중 1명 꼴인데
역시 그 아버지가 권투나 격투기를 한다거나 정말 '형님' 소리가 나오는 그런 산적같은 분이더라.... 는 케이스.
\vspace{5mm}

물론 이런 글의 진 취지는 "그러니까 남자들이나 자기 일이나 잘 하고 능력과 힘을 키우자"가 되는데
꼭 이런 글 보면 '너 여혐이지, 우리나라가 얼마나 여자들이 살기 어려운 줄 알아'라는 시비가 걸리곤 하는데
나야 뭐 '입으로만 그러지 말고 그럼 실천하거나 살기 좋은 곳으로 이민 가세용'이라고 하거나
'오전 시간 대 스타벅스나 백화점 가면 돈 쓰는 건 누구게욧'이라고 얘기하면 그만.
(냉정히 말하면 세계에서 우리나라도 여자들에게 천국인 곳이고, 남자들이 월급통장 맡기는 경우도 참 유례없을 텐데)
\vspace{5mm}

그리고 애초에 남자가 여자랑 논쟁하는 것도 참 쓸데없는 짓이다. 논쟁해서 결론이 날 것 같냐 하면 절대 아님.
논쟁 자체가 남자와 여자가 지향하는 바가 그냥 다르기 때문에 끊임없는 평행선을 그을 뿐.
남자로서 다른 남자들에게 권하고 싶은 건 '대화가 주제에서 벗어나려고 할 때에는 과감히 컷하라'하는 것만 권하는 것일 뿐.
가령 탕수육 논쟁을 한다고 했는데 그게 갑자기 떡볶이로 바뀐다거나 우리가 어디서 탕수육 먹었나느냐하는 걸로 싸움질하기 시작하면
신기하게도 우주의 온갖 삼라만상을 논하게 되는 것을 경험하게 된다.
\vspace{5mm}



\section{인공지능에 대한 모순적인 태도}
\href{https://www.kockoc.com/Apoc/675934}{2016.03.14}

\vspace{5mm}

인공지능을 우려하는 사람들이 답답한 것 :
우리가 이미 전기로 돌아가는 기계문명이 없으면 살아갈 수 없고
특히 컴퓨터 없이는 사회생활이 거의 불가능해졌다라는 기본적인 현상조차 망각하고 있다.
\vspace{5mm}

인공지능보다도 더 현존하는 위험은 바로 원자력 발전소, 하지만 그 덕분에 저렴한 전기로 살아갈 수 있는 것이 아닌가.
인공지능이 법조인, 의료인, 교사를 대체할 것인가 말 것인가. 이것도 공리공론 나열할 바 없이 다음과 같이 정리하면 된다.
\vspace{5mm}

\begin{itemize}
    \item[] ⓐ 기능 : 적어도 지금 추세로 보자면 인공지능이 훨씬 낫다라는 걸 부인할 수 없다.
    \item[] ⓑ 비용 : 당연히 인공지능이 낫다
    \item[] ⓒ 윤리 : 그렇게 따지면 우리는 자동차도 타지 말고 인류를 위협하는 원자력부터 날리고 밤에 호롱불 밝히면 된다 .
\end{itemize}
\vspace{5mm}

여러가지 관점에서 검토해보자면 대체는 확정적이다. 물론 '인공지능을 설계하고 통제하며 보완해줄 수 있는' 사람은 일자리를 잃지 않을 것이다.
\vspace{5mm}

하지만 그 이전에 우리는 가장 기본적인 것을 망각하고 있다.
이미 우리는 인터넷의 '집단지성'에 의존하고 있지 않나, 뭘 해도 뻑하면 \textbf{검색질해서 그걸로 대답하는 시대 아닌가.}
다시 말해서 우리가 직접 경험하고 깨닫는 건 사실 얼마 되지도 않는다.
하다 못해 강의도 무조건 현강으로 들어야지, 고작 강의를 녹음한 인강으로 감히 배속수 조절하는 것부터가 문제있는 게 아닌가?
\vspace{5mm}

즉, 이미 우리는 인공지능에 의존하는 시대에 살아가고 있는데 뭘 일자리를 잃느니 아우성치는지 모르겠다.
물론 해당 전문직은 매우 기분이 나쁘겠지만, 그렇게 따져보면 우리가 값싸게 애용하는 농산물과 공산품도
'자동화'로 일자리 잃은 사람들의 한숨이 담겨있지 않나.
\vspace{5mm}

그리고 부인할 수 없는 건 그런 분야를 인공지능이 담당하는 세상이 지금보다는 나을 수 밖에 없다.
무엇보다 인공지능의 그런 속도가 아니면 진정한 의미의 지구촌이 될 수도 없고, 인류가 우주로 나아가기도 힘들어진다.
\vspace{5mm}

어제 이세돌이 첫승을 거둔 것은 매우 뜻깊어졌다. "이미 대단해진 인공지능을 한번 이겼기 때문"이다.
다시 말해서 알파고가 3승하면서 이미 인공지능과 인간지능의 관계가 역전되기 시작했다.
그래서 이세돌의 첫승은 '이미 인공지능이 우월해졌다'라는 걸 전제하는 셈이 된다. 인공지능이 반신(半神)으로 격상된 것이다.
\vspace{5mm}

그래도 인간이 잘하는 것은 그거 아닌가, 바로 학습과 적응.
인공지능이 우월해서 세상을 지배한다고 해도 거기서 '살아남을' 것이다.
그리고 우리가 잊고 있던 게 있지 않나? 지구가 이 모야 이 꼴이 되고 인류멸망의 가속이 촉진된 것은
바로 인간이 지구의 실질적 지배자가 된 이후라는 것.
다시 말해 인간이 그냥 인공지능의 말을 충실히 따르며 보좌한다면 그 인류멸망의 주범인 인간=주인이란 전제가 무너진다는 얘기다.
물론 인공지능이 지배하는 것도 또 다른 비극을 초래할지는 모르겠지만, 이 덕분에 지금 예정된 암울한 시나리오가 수정될 수는 있는 것이다.
\vspace{5mm}




\section{금수저의 시대}
\href{https://www.kockoc.com/Apoc/678638}{2016.03.16}

\vspace{5mm}

오늘 모 사건이 있었습니다.
그 인물의 이력서를 보건대 거참,
자녀 이력을 그럴싸하게 만드는 모범적인 교과서가 아니라고 할 수 없겠더군요.
더 소름끼칠 것도 없는 건 본인이 금수저인데 흙수저 문제에 공감한다는 식으로 얘기했던 건데.
\vspace{5mm}

정작 흙수저들은 열심히 공부하고 일하느라 정신없는데
본인들이 금수저이거나 금수저에 가까운 사람들이 흙수저를 살려야한다고 말하고 있죠.
서민 복지를 확충해야한다거나 하는 사람이 알고보니 기득권이라는 건 지겨운 패턴이라서리.
실제로 입시전형이 다양화되면서 간접적으로 음서제를 용인하는 결과를 가져오면서 금수저의 시대가 터진 것인데
여전히 바보들은 일제 잔재가 청산이 안 되었다거나 이게 군사정권 독재 때문이다라고 헛진단을 하고 있죠.
그 음서제 용인이 2000년대에 벌어진 일입니다, 2000년대에 터졌던 교육과 부동산 문제가 바로 2010년대의 수저론으로 지적되고 있죠.
\vspace{5mm}

일제잔제청산이나 군사독재비판은 필요하죠, 그러나 지금 이것이 현안과 큰 관계가 있을지는 의문.
그러나 정작 지금의 문제와 관계없는 과거사 언급이란 오히려 '현안'을 은폐하기 위한 고도의 공작인 경우가 많죠.
부동산과 교육에서 기득권 챙긴 자칭 진보인사께서 군사독재를 비판해서 여론을 거기에만 골몰하게 하면 자기가 공격받는 일을 피할 수 있죠.
그리고 자기 기득권이 공격받으면 '너 친일파지?'라고 받아쳐서 문제의 핵심을 흐리면 되는데 실제로 이렇게 다들 속아넘어갔죠.
\vspace{5mm}

현실적으로 생각해본다면 일제잔재와 수시입학 비중은 정말 1:1000의 문제입니다.
일제잔재의 경우는청산을 현실적으로 할 수가 없습니다. 일제 잔재가 나쁜 것만은 아니기 때문입니다요.
(엄밀히 말하면 1945년 이후 대한민국시스템이 그냥 일제가 남기고 간 시스템을 그대로 써먹고 있는 것일텐데 말입니다)
그러나 수시입학 비율은 사회적 계급을 결정한 문제고 이건 학업성취와 독립적으로 말하면 부의 세습 비중을 늘려줍니당.
정시도 사교육비가 들어간다는 건 마찬가지이겠지만 적어도 '기회의 평등' 면에선 달라집니다요.
\vspace{5mm}

이 경우 그 위선자들이 문제인지, 아니면 위선자들에게 속아넘어간 대중이 문제인지...  답하기 곤란하겠지만.
결국 그 대가는 \textbf{대중들이 치릅}니다.
\vspace{5mm}






\section{그 분들도 피해자들입니다}
\href{https://www.kockoc.com/Apoc/679752}{2016.03.17}

\vspace{5mm}

http://m.cafe.naver.com/sayalang/1612
\vspace{5mm}

그 분들이 또 일을 저지르신 모양.
그 한심한 분란짓에 대한 답변이 매우 훌륭한지라.
\vspace{5mm}

우리나라 남자들도 하는 짓이 하도 막장짓이 많기 한데
$-$ 3만 코피노, 소위 '영업'이라고 뻥치는 접대, 가정폭력 $-$
문제는 소위 '미러링'이라고 핑계대는 걸로 해서 똑같은 폭력을 저지른단 것임.
\vspace{5mm}

그런데 알고보면 저 분들도 알고보면 피해자들임,
일단 남성우월주의나 가부장주의를 극복하자는 건 좋은데
그 윗 선배들이 주창한 페미니즘에는 \textbf{'어떻게 극복하느냐'라는 방법}은 전혀 나와있지 않죠.
\vspace{5mm}

이건 희대의 떡밥쟁이 마르크스의 자본론과 똑같습니다.
자본론에는 자본주의 사회에 대한 날카로운 비판이 있습니다, 그리고 지금도 눈여겨볼만한 중요한 설명들이 있죠.
그런데 그 \textbf{'공산주의' 사회를 어떻게 설계하고 구현할 건데} 라는 말은 전혀 나와있지 않습니다.
그 결과 세계의 절반 정도가 정말 50년동안 삽질을 하며 살아야했죠.
\vspace{5mm}

페미니즘 책들을 읽어보면 나름 그럴 듯한 개념들은 있는데, 그래서 "어떻게할건데"라는 말은 나와있지 않죠.
아무리 정교한 이론도 결국 그러니까 "남자들이 여자들에게 많이 퍼줘야한다"로 귀결된다라고 해도 지나친 얘기가 아닙니다.
백마디 말보다도 여성들끼리 어떻게 해서 기존의 남성우월주의나 가부장주의와는 다른 대안이 가능한지 보여줬어야죠.
그런데 제가 아는 한 이건 단 한건도 없습니다.
\vspace{5mm}

방법이 나오지 않은 채 공산주의 혁명을 일으킨 소련, 중국, 그리고 북한이 어떻게 되었죠?
현실에서의 모순이 발생하면 그걸 해결하긴 커녕, 정치적 탄압과 숙청을 일삼았죠.
그나마 소련과 중국은 뒤늦게 개방이라도 했는데 북한은 결국 주체사상이라는 희대의 사이비 종교국가로 가서 삥뜯고 있게 되죠.
그 공산주의를 "어떻게" 구현할지 마르크스는 이야기하지 않았거든요.
사실 요 양반도 글빨만 좋았지 직접 땀흘리고 일한 사람이 아닙니다. 씀씀이가 헤픈 금수저였죠.
게다가 친구 엥겔스가 보내주는 생활비도 어마어마한 수준이엇죠. 가정부를 범해서 사생아 낳고 입양보낸 것도 흑역사고.
\vspace{5mm}

실제로 페미니즘 투사들이란 사람들도 '싸우는' 건 좋지만 과연 실천하신 분들이 계시긴 하는지 가히 의문입니다.
그래서 이런 사상에 감화받으신 건 좋은데 '그럼 어떻게 해야하나'라는 대안이 없으니 이 분들도 인터넷 폭력으로 가는 것이죠.
분명 페미니즘 사상이 맞는 것 같은데 \textbf{대안은 없다 $\rightarrow$ 이게 다 한남충과 거기에 의존하는 여자들 때문이다 $\rightarrow$ 혁명!}
2000년대초인가 페미니즘이 등장했을 때 괜히 까인 게 아닙니다. 그 분들 입장에서는 그게 마초의 곡해나 탄압으로 보였겠지만.
문제점 비판은 좋다 그건데 그래서 '그 좋은 사회를 어떻게 만들 거냐'하는 대안이 없으면, 이건 폭력을 정당화하는 선동에 불과하거든요.
\vspace{5mm}

자, 그럼 우리는 여기서 왜 헬조선인들이 천조국을 빠는지 알아봅시다.
단지 이게 헬조선이 조그만 반도국이고 천조국이 세계를 지배하는 경찰국가라서 그런가요?
\vspace{5mm}

헬조선은 입으로만 떠들지 실천은 못 합니다. 어른이고 아이이고 목소리는 큽니다, 비판은 그럴싸하게 해요.
그러나 직접 나서는 거물은 없습니다, 외려 비판만 하면 되지 대안을 왜 요구하냐라고 화를 냅니다.
천조국은 그들의 200년 역사를 보면 항상 실천이 따라옵니다. 공부 이전에 일단 '일'을 하라고 하며 노동의 중요성을 설파하죠.
이들은 그 노동으로 환경을 극복하고 바꿔나가기 시작했고 그 결과 200년 이후에 전세계를 사실상 지배합니다요.
\vspace{5mm}

물론 중국과 일본에 끼인 신세와 드넓은 아메리카 대륙이 펼쳐진 상황이니 바로 비교할 수는 없지만
기본적 가치관에서 엄청난 차이가 있다는 것입니다.
한국의 대학은 기본적으로 공부를 열심히 해서 '관리'(혹은 대기업 사원)가 되는 마인드를 깔고 갑니다.
그에 비해 미국의 대학은 학생이 스스로 창업을 할 수 있도록 도와줍니다.
\vspace{5mm}

부모님들 마인드도 그렇죠. 우리나라는 너 열심히 공부해서 좋은 데 들어가 승진하렴이라고 얘기하죠.
그런데 지금 대한민국 현실에 그런 조언이 먹히나요?
\vspace{5mm}

중국에는 문혁 세대, 일본에는 전공투 세대, 그리고 한국은 386 세대.
현재의 결과는 좀 다르지만 공통점은 '대안 없는 메시지'에 현혹된 적이 있었다는 것입니다.
더군다나 그들이 생각하던 이상향이라는 건 허구에 불과했으며
그들 스스로가 젊은 날의 자신을 '배신'합니다. 바꿔 말하면 철들었다(?)라고 할 수 있을지 모르지만요.
\vspace{5mm}

초기 페미니스트들이 그냥 메시지가 아니라 정말 '실천'하는 모습을 보여주었다면 저런 피해자들이 양산되진 않았겠죠.
요즘 저 분들 하는 행동에 말이 많은데 어차피 시간 지나면 사그라듭니다. 그 분들이 뭔가 생산할 수 있는 구조가 아니라서리.
문제는 그동안 낭비한 시간을 보상받을 수가 있겠냐는 건데 그럴 리가 있겠습니까.
\vspace{5mm}

사실 불편한 대목은 자본주의 사회에서 소비를 담당하는 여성고객에서 저 분들이 차지하는 지분이 크다는 것인데.
소비자로서는 그렇다 치고 그럼 '생산자'로서는 어떻냐라고 물어본다면 여기서 모순이 생기죠.
소비가 나쁜 건 아닙니다. 오히려 장려되어야하는 것이죠, 단 전제는 그 소비자는 경제 시스템에서 '생산자'여야한다는 것입니다.
그런데 그 분들의 소비는 생산에 비해서 과도한 편입니다.
이런 모순을 결국 \textbf{"능력있는 남자"와 결혼하는 걸로 해소하려 하죠}.
초기 페미니즘은 이런 결혼에 대해서도 비판적이었는데, 요즘 문제되는 그 분들은 '돈없고 키작은 한남충'을 까지
돈많은 남자를 비난하진 않습니다(물론 이혼이나 결별시에 적지않은 위자료를 뜯어내고 재산분할한다는 걸 전제하죠)
그런데... 이게 발상만 그렇지 현실적으로 가능하긴 하겠습니까.
철없는 대머리 아재가 2d 미소녀와 현실 미연시를 한다는 것만큼이나 터무니없는 이야기인데 말입니다.
\vspace{5mm}

진짜 나쁜 사람은 저 분들을 이용해먹는 드라마 작가나 업자들이겠습니다만.... 이건 나이먹고 나서야 다들 깨달으실테고.
\vspace{5mm}




\section{먹방이 뜨기 시작한 이유}
\href{https://www.kockoc.com/Apoc/681435}{2016.03.18}

\vspace{5mm}

과거 90년대에
"무궁화 꽃이 피었습니다", 소설 동의보감(드라마 허준), 한권으로 보는 조선왕조실록 등이 대박난 적이 있다.
그에 대한 출판시장의 분석은 명료했다고 알고 있다.
산업화 과정에서 미친 듯이 일하면서 소득이 늘어서 여유가 생기고보니까
"자존심"과 "정체성" 문제에 대한 욕망이 생겼다는 것.
\vspace{5mm}

하지만 요즘은 저런 민족주의나 국뽕이 안 먹힌다.
이제는 먹고살기 팍팍해졌기 때문이다.  그래서 뜬 것이 '먹방'이다.
김준현이나 이국주 같은 푸짐해보이는 연예인들이 인기를 모은 것도,
각종 배우들의 연기에서 '먹방'이 필수요소가 된 것도, 그리고 요리 프로그램이 뜨기 시작한 것도 그렇다.
과거보다 가난해지고 배고프지만 그래도 최소한의 품위나 자존심은 살리고 싶다
여기서 먹방이나 소위 요리/셰프쇼만한 것도 없는 것이다.
\vspace{5mm}

우리들은 정신이 순수한 독립적인 것이라고 착각하지만 알게 모르게 '물질'의 영향을 받고 있다.
\vspace{5mm}

수험서로 돌아가보아도 그렇다. 도저히 이 책이 정말 좋은 책이 맞나... 하는 책이 역설적으로 많이 팔린다.
사실 시장에서 많이 팔리는 건 품질과 관계없는 경우도 있냐하는데 그래도 합리적인 이유를 알고 싶을 것이다.
그리고 우리는 무안단물부터 시작해 검증 안 된 암치료제들도 많이 팔린다는 것,
특히 "부적"이라는 것은 고가에 팔린다는 것에 비춰보면 이런 현상은 가볍게 설명된다.
아, 지금 수험생들은 정말 불안하구나. 그러나 서점에서는 원래는 부적을 팔지 않잖아,
그러므로 부적을 대용하는 뭔가 필요하다. 더군다나 그런 부적을 파는 사람들은 적절한 시점의 교주 코스프레도 해주고 있다.
(그리고 그 부적들은 너무 위대한 나머지 효능을 보는 사람도 선택해준다. 공부 잘 하는 사람들만 선택해주는 신묘한 부적들)
\vspace{5mm}

개인적 의식이든 집단적 의식이든 정말 사소해보이는 것 $-$ 즉 물질적인 것에 거의 좌우된다고 해도 틀린 이야기가 아니다.
능력에 비해 돈이 많이 벌리는 사람들은 거만해진다.
능력에 비해 돈이 적게 벌리는 사람들은 홧병을 앓거나 겸손해지거나 그 중 하나다.
\vspace{5mm}

수험에서 정신론만큼 허망한 것도 없는 이유다.
\vspace{5mm}






\section{그들이 원한 건 평등이 아님.}
\href{https://www.kockoc.com/Apoc/681774}{2016.03.18}

\vspace{5mm}

말하지만 '평등'을 요구하는 것은 거의 95$\%$는 "우월"을 요구한다고 보면 된다.
\vspace{5mm}

이런 논리가 평등을 요구하는 것 자체를 차단한다는 점에서 위험하다고 할 수 있으나
역사적 사례를 보면 적어도 20세기 후반부터는 평등을 요구한 결과가 역차별을 낳는 '우월주의'인 경우가 왕왕 보임.
\vspace{5mm}

평등은 같은 것은 같게, 다른 것은 다르게임.
그리고 이건 기회의 평등과 결과의 평등으로 구분하면서 뭐가 같고 다른지 세세히 따지면 끝도 없는데
이런 복잡한 것을 생략하고 무조건 '같게'라고 하는 건 그게 \textbf{자기들에게 이익이 되기 때문}임.
가령 10년 죽어라 공부한 사람과 10년 논 사람은 똑같이 명문대에 입학시키자하는 게 평등이라고 볼 수는 없음.
\vspace{5mm}

모처에서 '민주화'라는 것이 비추기능로 쓰이는 데 이 유래가 되는 걸 목격한 적이 있죠.
이x루스에서 현재 여당을 지지하는 모 블로거 글이 계속 차단, 삭제당했음.
사실 삭제당할 이유도 없는 합리적인 논거의 글이어서 그 신고자에 대한 비난이 있었는데 변명이 가관이었음
\vspace{5mm}

\textbf{"민주화를 위해서입니다"}
\vspace{5mm}

그 이후로 민주화라는 게 그런 식의 내로남불을 비아냥대는 표현으로 쓰이기 시작했죠.
\vspace{5mm}

한국사회도 저런 민주주의적인 이념을 핑계로 결과적으로 역차별을 하거나 이득을 누리는 케이스가 적잖게 많고
본인이 남을 부양할 필요가 없는 학생 시절에나 왜 그런 것을 까느냐 민주주의 시민이라면 관용으로 이해해줘야하지 않냐고 하지만
정작 자기가 책임을 져야할 나이가 되면 바로 \textbf{우회전}해버리고 말지요.
그 때부터는 아무 것도 안 하고 민주주의를 핑계로 남의 피 빨아먹는 존재가 확실히 느껴지니까.
\vspace{5mm}

인터넷에서 이런 이야기하면 자기가 $\sim$ 할 테이니 $\sim$ 하지 않느냐 하는데
제 답변은 간단함. 미래형으로 얘기하지말고 현재완료형이나 현재형으로 얘기했으면 좋다는 것임.
다시 말해 입으로만 어쩌구 하지말고 본인들이 그에 필요한 것을 내놓으시면 됩니다.
그런데 그렇게 안 함, 평소에 정의 어쩌구 하는 사람이 다 자기 돈 걸리면 그 때는 \textbf{싹 입장 바꿈.}
\vspace{5mm}

이번에 M갈리안이 또... 나오는데 그 사람들이 씹치남이니 갓양남이니 그런 거야 알 바는 아니지만
하지만 데이트를 하건 뭘 하건 자기가 먹은 밥값을 자기가 내는 게 당연한 건이 아닌가 하는 지적은 해야겠음.
그게 거지근성이지 그럼 뭘 잘 했다고 그러나. 말로만 평등하지 말고 자기몫은 자기가 책임지시면 됩니당.
물론 남자가 밥값을 다 낼 수는 있는데, 적어도 더치페이파들은 남자가 사주면 고마워하지만
더치페이하지말자파는 고마워하지 않는다는 점에서 이건 인간성과 가정교육에서 걍 문제가 있는 것임.
\vspace{5mm}

복지 문제도 그런데 이거 증세해야한다고 하면서 님들 세금 거둔다고 하니까 싹 입장 바꾼 모 사이트 사례도 있어요.
상위 10$\%$ 증세해야한다고 했는데 월 200 이상이 상위 10$\%$에 해당한다고 하는 걸 뒤늦게 알고 입을 다물어버린 것이죠.
그럼 도대체 그런 것도 모르고 북유럽식 복지 주장했다는 이야기인데,
남의 선동이나 받아쓰기하는 사람이 어떻게 돈버는지 그게 이해가 안 가죠.
\vspace{5mm}

우리나라가 너무 인터넷이 잘 보급되어서 행동 이전에 키보드를 두들기길 좋아서 그러는데
평등이 어쩌고 소수자의 인권 어쩌구 하는 사람들이 정말 길바닥에서 고생 직살나게 하고 돈버는 게 얼마나 어려운지 경험하면
진짜 소신있는 영웅캐릭터 빼고 나머지는 싹 돌아서버릴 겁니다.
그 이야기는 다시 말해서 인터넷에서 몽상주의적인 이야기하는 애들 보면
'아 얘들은 고생 안 해보았구나'라고 추론해도 별로 틀리는 게 없어요.
본인들이 정말 독립해서 자기 수명 깎은 돈으로 밥벌이하는 사람들은 저런 주장 안 합니다.
\vspace{5mm}

요즘와서 엽기범죄에다가 그냥 집단적으로 맛이 가는 사례가 왜 생겼냐 해보니 저 '민주화' 때문이라는 생각이 듭니다.
나쁘다는 걸 나쁘다, 옳지 않은 걸 옳지 않다라고 분명히 얘기해주고 꾸짖을 건 따끔하게 꾸짖어야하는데
그걸 지적하는 순간 \textbf{'혐오'}라느니 \textbf{'차별'}이라는 식으로 나와버리니까 이게 흐지부지되는 것이죠
\vspace{5mm}

아무튼 그걸 떠나서 이제 민감한 이야기를 하는데 이건 내가 묻고 싶음.
\vspace{5mm}

\begin{enumerate}
    \item \textbf{ 의대에 가서 인술을 베풀고 봉사하고 살고 싶어요}
    \vspace{5mm}

    이거 전에 소감문으로 지겹게 들었음.
    그 분들 어디 가시나. 그냥 수도없이 접하는 사례는 다 돈을 어떻게 버나 고민하는 케이스던데.
    따라서 거짓말.
    \vspace{5mm}

    \item \textbf{ 이제 여성이 주도하는 화목한 가정.}
    \vspace{5mm}

    이게 가장 궁금함,  한번도 못 보았기 때문임.
    물론 그 분들 말로는 여자가 돈을 벌어주고 남자가 가정에서 일하면 된다고 하는데
    정작 그럴 능력을 갖춘 여자들은 자기보다 더 능력이 좋은 남자들과 결혼하심(...).
    그래서 저런 사례가 생기는 경우는 남자가 일자리를 잃는 경우인데 이 경우 남자는 이혼위기에 처해짐(...)
    아니 뭐 결혼부터가 이미 남자는 2억 집해오고 여자는 3천만원 혼수해오면 장땡인 시대임, 걍 '거짓말'.
    \vspace{5mm}

    그리고 개인적으로 접해본 가정들은 잘 돌아가는 경우는 유감스럽지만 남자가 더 능력이 좋은 '보수적인 가정'이었습니당.
    \vspace{5mm}

    \item \textbf{ 불체자들이 한국에서 고통을 겪고 있어요}
    \vspace{5mm}

    그런데 왜 자꾸만 그 고통스러운 한국으로 불법으로 들어오시나, 가장 이해가 안 가고 있음.
    게다가 그 고통을 겪는다는 분들께서 수원역 앞 정육점(...) 앞에 주말이면 줄서고 계시고
    로데오거리에서 여학생들에게 수작거는 건 아시나.
    명목상 저임금이지만 본국 기준으로 하면 우리나라에서 의사가 버는 수준으로 벌어가고 있음 이 분들께선.
    거기다가 결혼전략으로 다문화가정 하면 또 쏠솔한 지원금 받으심
    그러므로 이와 관한 이야기도 거짓말. 고통 겪는다면 그냥 본국으로 돌아가시면 되잖아(...)
    \vspace{5mm}

    \item \textbf{ 우리나라는 서울대가 문제}
    \vspace{5mm}

    '댁 자식이 서울대에 합격해도 취소시키겠군요'라고 하면 어떤 답변이 나올까
    대학교에서 일하는 강사나 교수들은 등록금 문제에는 침묵함,
    그렇게 기득권 까대던 분들이 자기 자식이 서울대에 가거나 자녀가 판검사급이 되면 싹 침묵한다는 걸 발견함.
    그리고 그 때 대는 핑계 : "난 실제로는 보수적인 자유주의자였습니다"
    \vspace{5mm}
\end{enumerate}
...
\vspace{5mm}

쓰다보니 거짓말쟁이들이 짜증나서 더 적기도 그렇고.
아무튼 입으로만 $\sim$ 하겠다, $\sim$ 할 테니 대우받고 싶다... 라는 건 일단 걍 무시함.
모든 미래형은 처음부터  거짓말임, 즉 그걸 실천해야 참말이 되는 거지 실천 안 하는 이상 거짓말인 것임.
콕콕에서도 열심히 공부하겠다 $-$ 이거 일단 거짓말부터 한 것임, 본인들이 열심히 해야 비로소 거짓말쟁이에서 벗어나는 거지.
\vspace{5mm}

그래서 키배를 뜨건 소통을 하건 상대가 "저 $\sim$ 할 거거든요"라고 하면 한숨부터.
$\sim$ 할 게 아니라 \textbf{당장 하면 되잖아}.
\vspace{5mm}

사람답게 대우받고 싶습니다.. 라고 하기 전에 자기가 그에 부합하는 일을 하든 공부를 하면 되는 것일텐데
부모님이 자꾸만 제 꿈을 가로막아요... 라고 할 자격은 일단 자기가 죽어라 그 공부를 하고 있는 걸 전제하는 것일텐데
기본적으로는 세끼 밥 자기가 책임지지 않는 이상은 발언권은 걍 없다고 보는 게 명확할 것임.
\vspace{5mm}






\section{목소리만 내면 뭔 소용이 있나.}
\href{https://www.kockoc.com/Apoc/690820}{2016.03.23}

\vspace{5mm}
\begin{enumerate}
    \item 드라마나 영화에서 '선역'을 연기하는 배우들이 실생활에서는 악역인 경우가 있다.
    xxx가 실생활에서 개차반이라는 것을 아는 사람들이 영화나 드라마에서 xxx가 착한 주인공을 연기하는 것을 보면서 어떻게 할까.
    더 중요한 사실은 xxx도 입으로는 나쁜 놈을 깠을 거란 이야기다.
    \vspace{5mm}
    
    \item 최근 뉴스에서 아이를 남몰래 죽이거나 버린 부모들이 등장하고 있다.
    이들도 역시 다른 뉴스나 드라마에서 아이를 학대하거나 못 살게 구는 타인의 소식에 혀를 차지 않았을까.
    더군다나 이들 역시 악인을 비판했을 것이다.
    \vspace{5mm}
    
    \item 자기가 을(乙)이라고 이야기하면서 헬조선은 왜 젊은이들을 못 살게 구느냐고 하는 사람들도 마찬가지다.
    사실 이들이 얘기하는 건 별로 믿을 필요가 없다. 정작 그들이 갑(甲)이 되면 어떻게 행동하느냐가 더 중요하기 때문이다.
    한 때 기성세대에 비판적인 사람들이 10년 지나고 보니 \textbf{더 해먹더라는 게} 현실이다.
    \vspace{5mm}
    
    \item 학교에서 교사의 체벌이 아이와 학생들에게 상처를 준다고 하던 사람들이 체벌이 줄어든 이후
    학생들끼리 폭행을 넘어 엽기범죄까지 저지르는데다가 일진놀이를 하는 데에는 침묵하는 경우가 많다.
    사실 뭐 이런 분야 운동가들이 알고보니 자기가 사적 영역에서 은밀한 폭력에 중독되어있더라가 까발겨진 경우도 적지 않다.
    \vspace{5mm}
    
    \item 다시 드라마나 영화 이야기를 하면 $-$ 작중에서 분명 작가는 주인공의 입을 빌려 돈만 아는 세상을 비난한다.
    그런데 정작 뉴스 등을 통해 확인해보면 그런 작가들은 엄청난 인세를 챙기고 있다.
    황금만능주의를 까면서도 '거액'을 벌어도 되는구나... 그런데 이렇다면 까일 대상은 없지 않나?
    \vspace{5mm}
\end{enumerate}

쓰다보면 참 끝도 없을 것 같다.
어린 시절에는 착하게 살아라, 인간성이 좋아야 한다.... 라는 말을 들었는데
그 당시에 난 이렇게 반박했다. "지금은 나쁘게 살래요. 그리고 좋은 대학 가고 난 다음에 착해지면 되겠네요"
\vspace{5mm}

지금 생각해보니 내가 한 말이지만 저게 정답이었다. 그리고 내 말을 내가 배신했다가 대가를 혹독히 치러야 했다.
정말로 사람 말을 믿고 착하게 살면 되겠지라고 '의심'을 안 했다가 데였거든.
그리고 인간성 타령하던 어르신들은 종적을 확인해보니 인간성이 좋긴 커녕 막장의 행보를 보여주더라는 현실.
\vspace{5mm}

끝까지 배신하지 않는 건 '문서'와 '공부'다.
\vspace{5mm}

업자들은 말과 행동의 괴리를 이미 전제하고 있다. 그래서 그들은 늘 거짓말을 하는 걸 당연시 한다.
업자들의 권력을 호구들의 무지에서 나온다. 호구들은 말과 행동이 분리될 수 있다고 생각하지 않는다.
호구들은 3년간 정치인들이 병신짓을 하다가 막판에 쇼를 하면 거기 감동해버린다.
반면 업자들은 정치인들이 3년간 잘 하더라도 막판에 쇼를 하는 순간 '너 불합격'이라고 매겨버린다.
\vspace{5mm}

한가지 예만 들어도 그렇다, 피라미드 계통 장삿꾼이든 영업맨들은 '인간미'에 호소하는 전략을 편다.
영업맨들과 대화나누는 호구들은 물건을 사지 않으면 미안해서라도 거액의 돈을 쓰게 되지만
업자들도 씨익 웃다가도 막판에 '곤란합니다, 지금은 바쁘니 다음에 얘기하죠'라고 차단해버린다.
설명을 신나게 듣지만 정작 물건과 서비스의 품질은 본인들이 판단하는 것이다.
\vspace{5mm}

진보적 메시지야말로 더 많은 호구들을 늘리는 짓이다.
사실 사회가 진보하는 해답은 여기에 있다. '메시지'는 그냥 무시해버리면 된다.
정치인이 뭐라고 치장하던 개무시하고, 그가 여태까지 실제로 뭘 달성해왔나 어떤 문제해결을 했나 그것만 따지면 된다.
유권자들이 꼼꼼하게 정치인들의 주가를 신경쓰면 그게 무서워서라도 정치인들을 일을 열심히 하고 딴 짓을 안 한다.
그런데 현실은 어떠나, 선거철만 되면 프로야구경기처럼 패갈라 싸우다가 선거 끝나면 늘 통수에 당한다.
\vspace{5mm}

현재 20대들이 헬조선이라 힘들다 한다. 뭐 다 각자 힘든 건 마찬가지이겠지만
중요한 건 어차피 이들이 소리쳐보았자 그건 무시당할 거란 사실이다. 왜냐면 순전히 메시지에 불과하니까.
사회 전체는 이제 '대가를 치르지 않는, 실천이 없는 메시지'는 그냥 무시해 버리기 때문이다.
\vspace{5mm}

사회적으로 고발하면 해결될 것이다... 라고 믿던 순진한 시대가 있었다.
그러나 현 시대는 고발을 해도 바로 '명예훼손'으로 맞받아칠 수 있다.
네티즌들이 거기에 동요되어보았자 한 일주일 정도만 공격질하다가 질려버린다. 떡밥이 사라지면 재미가 없어지니까
모 드라마로 모 지역 범죄 사건으로 당시 가해자들을 응징하자는 여론이 있었지만 어땠나? 알파고 떡밥에 끝나버려지.
\vspace{5mm}






\section{질적 교육}
\href{https://www.kockoc.com/Apoc/692503}{2016.03.24}

\vspace{5mm}
\begin{enumerate}
    \item  천재로 태어났는데 정글에서 자랐다
    \item  평범하게 태어났는데 영재교육을 받았다.
\end{enumerate}
\vspace{5mm}

둘의 결과는 어떨까 가정하면 끝나는 일이라서리.
물론 유전은 무시 못 하죠. 뇌가 문제가 있으면 나이먹어도 유아 수준에 머무르니까.
1번은 선천적으로 동물적 감각이 뛰어나면 정글에서 생존왕으로 살아남겠지요.
그런데 '문명적인 것'은 유전적인 것과는 다소 거리가 있습니다. 그러니까 '교육'이라는 게 있는 게 아니겠습니까.
\vspace{5mm}

남들이 이미지 프레임 하나를 처리할 때 열가지 이상을 동시에 떠올린다거나
손으로 써서 계산해야하는 것을 암산으로 끝내버린다거나
하나의 명제를 보면 그와 관련있는 일곱가지 이상의 명제를 떠올려서 바로 개요를 짤 수 있는 것은 분명  교육으로 가능합니다.
다만 이게 현행 교육과정에 있느냐 하면 그게 아니라서리.
공교육의 목표는 호구를 키우는 것이지 천재를 키우는 게 아니죠.
\vspace{5mm}

이 문제는 일찌감치 제기되어왔습니다. 그래서 "창의성을 키우는 교육"을 해야한다.... 라고 얘기하죠.
늘 그렇지만 그럼 "어떻게"라고 질문하면 답을 못 합니다.
그렇다고 어떻게가 답이 없는 건 아니지요. 그냥 어른들이 하는 과정을 적당히 요약하면서 뭔가 만\textbf{드는} 과제 던져주고
팁을 가르쳐주고 두뇌를 단련시키면 되는 것이라고 말할 수 있겠으나, 이런 것들은 드는 \textbf{비용이 매우 큽니다.}
\vspace{5mm}

이래서 파더마더 쉴드가 참 중요한데 $-$ 부모가 애초에 지적수준이 높은 가정에서 태어난다면
사실상 그 분야 무상교육을 받을 수 있어서입니다.
물론 부모가 지적수준이 높아도 자녀에게 무관심하거나 신경을 쓰지 않으면 결과는 기대만큼은 아니겠습니다만
\vspace{5mm}

그리고 지금이 이게 명백히 나타나는 시기입니다.
\vspace{5mm}
\begin{itemize}
    \item 고대 : 선택된 왕족과 귀족만 제대로 교육받았습니다.
    \item 중세 : 종교인과 상공업자들까지 확장
    \item 근대 : 일부 노동자들에게까지 확장
    \item 현대 : 여성을 포함해 모든 사람까지 적용
\end{itemize}
\vspace{5mm}

사회진보의 척도는 교육 대상의 확대라고 말할 수 있죠. 그만큼 '격차'라는 것의 문제를 인식하기 시작합니다.
과거에 신분제가 합리적(?)이었을 수도 있는 이유는 특권계층이 교육을 독점했으므로 능력과 문화를 그들이 책임질 수 있었기 때문입니다.
왕의 자질은 왕족만 갖고 있다라고 믿을 수 밖에 없습니다. 왜냐면 교육을 그들이 독점했으니까요.
이건 지금도 마찬가지입니다. 어느 순간 교육을 다 없애버리고 강남애들만 교육받는 것이 10년 이상 이어지면
사회의 모든 중책을 강남 애들이 해먹고 나머지는 그냥 잉여질이나 하고 살자라는 걸로 체념해버리겠죠.
\vspace{5mm}

다만 교육의 양적인 면은 확대되었어도 과연 \textbf{질적 수준까지는 담보하느냐.}.. 라고 하면 그건 아닐 것입니다.
저기서 질적이란 말이 중요하죠. 저출산 문제를 해결하기 위해 이민을 대거 받아들이자... '양적 인구'로는 맞는 말입니다.
그러나 '질적 인구'라는 점에서는 아웃입니다.
사람들이 원하는 교육은 자식을 천재로 만드는 질적 교육입니다. 그리고 그건 분명 존재합니다.
그러나 국가는 거기까지는 책임지지 못 하죠. 교육과 복지는 (최저 수준)이란 수식어가 교묘히 숨어있으니까요.
\vspace{5mm}







\section{인공지능이 나왔으니 공부를 안 해도 된다?}
\href{https://www.kockoc.com/Apoc/694058}{2016.03.25}

\vspace{5mm}

역사의 진보란 교육받는 객체의 확대입니다.
\vspace{5mm}

20세기가 원칙적으로 모든 사람이 교육의 권리를 받는다면
21세기는 이제 사람을 제외한 동식물부터 인공지능까지 교육받는 시대라고 할 수 있을지도 모릅니다.
\vspace{5mm}

옛날에는 귀족이 아니면 사람 취급을 못 받았죠. 여러분들이 두들기는 컴퓨터 역할을 해주는 게 바로 노예였음.
그 때에는 교육이 즉 계급이었습니당.
노예나 노비들이 들고 일어나보았자 사람 취급 못 받습니다. \textbf{아는 게 없기 때문}입니다요.
아는 게 없으니까 소통할 수 없고, 소통하지 못 하니 '나'를 규정할 수 없습니다.
\vspace{5mm}

그런데 지금은 컴퓨터가 교육받는 세상입니다(어떻게 보면 사람들이 인터넷 활동으로 인공지능을 키워주고 있는 셈)
이제는 인간이 안 하면 인간취급을 못 받을 수도 있습니다.
전문직 죽는다 어쩐다 해도 "해당 직무의 인공지능에 관여할 권한이 있는 자"로 수정하면 그만입니다(...)
언론에서 인공지능이 직업을 대체한다라는 것의 전제가
'일반인들이 인공지능을 이용할 수 있다'라는 것, 즉 접근권의 보편성인데  그럴 리는 없잖습니까.
\vspace{5mm}



\section{노력하겠다는 의지를 표시하는 경우는 불길하다.}
\href{https://www.kockoc.com/Apoc/698871}{2016.03.27}

\vspace{5mm}

"열심히 하겠습니다"라는 다짐도 빨주노초파남보로 분리되는 햇빛처럼 분리해보아야한다.
\vspace{5mm}

정말로 잘 하고 있는데 겸손을 가장해 열심히 하겠습니다 소리치기
슬럼프였는데 그걸 극복하고 공부하는 맛을 알아서 열심히 하겠습니다 다짐하기
\vspace{5mm}

그런데 이건 5$\%$도 될까말까하고
\vspace{5mm}

대부분은 열심히 하겠습니다라는 말은 \textbf{'궁지'에 몰렸을 때} 하는 말이다.
정말로 공부를 하고 있으면 열심히 한다라는 말은 별로 안 한다.
\textbf{"힘들어죽겠다", "피곤하다", "더 편한 길이 없느냐"} 이러지.
\vspace{5mm}

자본주의 사회에서 노오력 해보았자 소용없다는 건 맞는 말일 수도 있다.
그런데 그게 금수저 때문에? 아니면 기득권층이 워낙 돈이 많아서?
그렇게 말하는 것부터 그 사람이 노답이라는 얘기다. 그 사람은 그저 인터넷과 매스컴이 떠드는대로 반복하는 앵무새이니까.
상속으로 부자가 된 사람이 많다면 마찬가지로 상속으로 거지가 된 사람도 있기도 하지만
자본주의 사회에서는 그런 식으로 횡재의 기회가 있다고 긍정적으로 생각해 볼 수도 있다.
돈을 많이 버는 경우는 일종의 '곱셈 효과'로써 성공한 것인데 이건 누구에게나 분명 기회는 오게 되어있다.
거기다가 나중에 논하겠지만 '곱셈'은 \textbf{"같은 덧셈의 반복"}이다.
\vspace{5mm}

하지만 그럼에도 불구하고 노오력만 해서 힘든 이유는
진짜로 성공한 사람들은 "머리"로 돈을 벌기 때문이다.
노력보다는 않고 머리를 쓰면서 돈을 벌 수 있도록 시스템을 만들고 조정한다.
이들과 대적... 아니면 모방하기 위해서라면 우리도 결국 배우고 공부해야 한다.
즉, 단순 노오력만으로 되는 건 아니란 이야기다. 그렇게 머리를 쓰기 위한 더 색다른 '학습'을 해야한다.
\vspace{5mm}

공부도 마찬가지다.
본인이 현명한 수험생이라면 자기가 변절할 거라는 현실을 미리 인정할 것이다.
자기가 극단적으로 게으름을 피우거나 놀아제끼는 일이 벌어지더라도 대비할 수 있는,
즉 학습을 계속 할 수 있는 \textbf{환경}부터 만들어놓았을 것이다.
\vspace{5mm}

그런데 열심히 하겠습니다... 라는 사람들은 사실 입으로만 떠드는 경우가 대부분이지만,
이 사람들은 그저 자신을 학대하다보면 공부가 되는 걸로 '착각'한다(그거야 공부를 안 해보았으니까, 아니 뭐라도 성사시켜본 적이 없으니)
정말로 실무적인 사람이라면 최악의 상황에 대비할 수 있도록 좋은 환경부터 찾아서 시스템을 만든다.
어디에 진지를 구축해야 유리한가, 그리고 무기나 식량은 다 잘 되어있는가, 병사들의 사기는 괜찮고 앞으로 날씨는 어떤가.
그렇게 하면서 가능하면 손실을 접게 입는 지구전으로 가려고 할 것이다.
이런 걸 하지 않고 그저 열심히 하겠습니다... 라는 사람들은 그냥 '우라' 돌격하다가 고기방패되는 경우나 다를 바 없다.
그리고 더 무서운 사실은 이런 실패하는 패턴을 수험에서만 반복해먹는 게 아니란 것이다.
\vspace{5mm}








\section{만우절 기념 폭론}
\href{https://www.kockoc.com/Apoc/705958}{2016.04.01}

\vspace{5mm}

오늘은 만우절입니다.
\vspace{5mm}



\section{스포츠와 게임}
\href{https://www.kockoc.com/Apoc/707345}{2016.04.01}

\vspace{5mm}

최소한의 운동을 하는 이유를 설명하라면 '균형' 때문입니다.
\vspace{5mm}

우선 현대문명이 편리해졌다라는 건 프로세스 컷, 혹은 행위 삭제로 설명할 수 있습니다.
과거에는 물건을 사러 직접 상점에 가서 고르고 결제한 뒤 가져와야 했는데
지금은 스마트폰으로 간략히 결제하면 배송이 옵니다.
\vspace{5mm}

그런데 이런 변화가 우리 일상에서는 균형을 깨뜨리기도 하죠.
비만이 많이 먹어서만 일까요. 열량만큼 몸을 움직인다면 그런 게 생기지도 않죠.
다들 이제 온라인 중심으로 생활합니다(이거야말로 근본적인 혁신이라 하겠습니다)
그런데 문제는 우리 몸은 대략적으로 낮밤 주기에 맞게 진화되어 왔으므로 여기서 불균형이 생긴다는 것입니다.
저를 포함해서 적지 않은 콕콕러들이 콕챗을 하면서 그 불균형의 노예로 전락한지 오래입니다만.
\vspace{5mm}

이 신체의 불균형을 잡기 위해 운동하라는 건 전문적으로 스포츠 수준까지 하라는 것이 아닙니다.
그냥 가볍게 걸어주고 광합성을 하는 정도로 족하다 그 이야기입니다. 뭔가 싸이클에 맞춰 움직이라 그 이야기죠.
\vspace{5mm}

스포츠의 경우 전공자도 아니기 때문에 말하기 조심스럽습니다. 이 글로 다른 분이 더 자세한 이야기를 할 수 있을 수도 있는데
일반적으로 스포츠 = 건강 이란 것이야말로 고정관념이 아닌가 합니다. 조금만 실증해보아도 뒤집혀질수 있는.
\vspace{5mm}

만약 스포츠를 해서 몸이 정말 좋아진다면 왜 '스포츠 마사지'라는 분야가 발달했을까 생각해 볼 수 있습니다.
마사지를 하는 이유가 신체 속의 피로요소를 제거하고 혈액순환을 활발하게 하기 위해서인데,
그건 거꾸로 말해서 스포츠 분야의 사람들이 그런 걸로 고생한다고 얘기되는 것이니 스포츠=건강이라는 통념은 붕괴됩니다.
\vspace{5mm}

스포츠=건강으로 인식된 것은 어떤 운동이건 모두 햇볕에 탄 건강한 피부, 야무진 역삼각형 신체, 가슴이 두근거리는 말근육... 일 것인데
정작 사람들이 요절하거나 노년에 고생하는 건 근육 때문이 아니라
뇌혈전, 뇌출혈, 심장마비, 심근경색, 심부전, 간경변, 각종암.... 등의 '내장' 문제입니다.
이런 내장은 근육보다는 '신체 싸이클'과 관련이 있다고 알고 있습니다.
\vspace{5mm}

게다가 스포츠 선수들의 신장이 일반인에 비해 안 좋다거나(소변 검사)라는 건 수십년 전부터 나온 얘기고
그들의 전성기가 끝난 뒤에는 일반인들보다 더 골골해 사는 것부터 시작해 평균수명이 높은 편도 아니죠.
오히려 지나친 신체활동에다가 휴식을 취하기 어려운 빡빡한 스케줄로 노폐물 축적이 심하고 그걸로 내장 부담이 가중된다는 게 중요.
\vspace{5mm}

다시 원점으로 돌아오면 "건강을 위해 운동해야 한다"라는 말은 "신체 싸이클을 회복하기 위한 수준의 부담없는 운동을 한다"가 적절할 듯.
\vspace{5mm}

이런 식의 분석을 하면 마찬가지로 '게임 역시' 저격당합니다. 사실 게임도 이제는 스포츠의 영역에 들어서지 않았나 싶은데
스포츠가 건강에 좋고(그리고 춤바람난 아재 아줌마들도 춤이 건강에 좋다는 말도 안 되는 변명... 교미가 더 좋아서겠지)
특히 두뇌에 좋다는 말을 많이 하고 게임도 그렇지 않나 하지만
\vspace{5mm}

정작 사회에서 스포츠나 게임으로 성공한 사람은 '상업성'이 있는 분야에서 탑을 보여주는 소수 뿐이고
나머지들은 들러리들로서 얻는 것이 그리 없다... 라 하겠지만 무엇보다 그래서 정말 엄청난 건강이라거나 엄청난 두뇌수준을 보여준다...
그 정도까진 아니라고 생각하고 있습니다.
\vspace{5mm}

가령 알파고로 각광받은 바둑도 사실 그걸 잘 둔다고 그 사람이 정말 머리가 좋아 학술적 발견을 이끌어내거나 크게 성공한 사례는 찾아보면 글쎄요.
그것도 팬들이 있어서 돈이 오갈 수 있으니가 그런 것이지.
생각해보면 수학 역시 입시과목이 아니라면 과연?
이런 식으로 접근해보면 우리가 아무 생각없이 참이라고 생각하던 명제들이 하나하나 붕괴되기 시작하죠.
\vspace{5mm}

공부하다 휴식하고 싶으면 운동을 해야한다, 혹은 게임을 하면 된다... 어느 쪽이든 사실 별로 도움이 안 되는 것입니다.
해답은 뻔합니다. 그냥 먹고 싶은 것 적당히 먹고, 구경하고 싶은 것 적당히 구경하고, 아니면 걍 '자면' 됩니다.
자기가 좋아하는 취미활동하거나 그런 게 없으면 수면을 취하는 게 합리적인 휴식이란 것이죠.
\vspace{5mm}

다소 비약일지 모르지만 역사적으로 확실히 우수하다고 파악된, 그것도 유전자보다는 역경과 교육으로 우수해졌다고 보이는 유대인들.
그들의 풍습이 바로 4000년 전통의 '안식일'. 즉 일주일에 하루 쉬고 정말 아무 것도 안 하는 건데
그들의 우수성이 여기서 비롯되었다는 설도 있는데 '휴식의 효용'을 생각하면 터무니없는 이야기는 아닌 것 같습니다.
유대인이 인류사회에 공헌한 게 일주일에 하루 휴일을 만든 것이죠.
\vspace{5mm}





\section{트럼프가 인기를 모으는 이유}
\href{https://www.kockoc.com/Apoc/711025}{2016.04.04}

\vspace{5mm}

여러가지가 있지만 가장 근본적인 것은
개드립을 칠 망정 \textbf{'거짓말'}을 안 하기 때문.
더 정확히 말하면 대중들은 "정치적 올바름"이라는 거짓말에 질려버렸음.
그럴 바에는 개드립이 낫다는 것임. 이건 우리나라에서 여당 지지율이 높아진 이유와도 비슷함.
기독교가 보수꼴통이라고 하면서 이슬람은 평화의 종교다 하면서 다문화에 대해서 편견을 갖지 말아야하는 분들
그 사람들이 최근 유럽 난민 사태부터 IS 에 대해선 침묵하거나 말 바꾸기를 하는 것이 좋은 예임.
사실 이슬람의 역사를 보면 보수적인 개신교보다 더한 폭력성이 있었음. 이걸 싹 빼고 편견 갖지 말아야 한 결과가 현재임.
\vspace{5mm}

인종차별은 하지 말아야겠지만 인종별로 학력, 소득, 범죄율 차이가 나는 것을 부인할 수는 없음.
왜 대기업만 지원하죠? 중소기업도 살려야죠라고 하는 누리꾼들이 폰이나 컴퓨터를 어디 것을 사겠음?
정치적 메시지와 소비행위가 엇갈려버린 이상, 즉 언행이 일치하지 않는 순간 그 말은 거짓말이 되어버린 것임.
거짓말을 한 사람들은 그게 거짓말이 아니다라고 주장하기 위해 자기들과 노선이 다른 쪽을 강경하게 배척함.
\vspace{5mm}

그러다보니 오히려 정치적 소수가 시끄러워지고, 정치적 다수는 침묵해버림.
그러니 인터넷에서 진보 쪽이 더 많아 보이는 것으로 보이지만 선거철만 되면 결과는 정반대로 나타남.
과거에는 군사정권 독재의 잔재가 남아있었거니와 아직까지 사람들이 공권력 무서운 줄 알아서 몸사린 게 있었음.
그 때야 정부가 탄압하지 말아야 할 대상도 탄압하고 희생양으로 만드니 다양성을 추구하자라는 메시지의 실효성이 있었음.
그러나 지금은 너무 정반대로 달려가면서 역차별 현상이 발생하고 있고, 소수를 무조건 보호하자 하다가 선량한 사람들이 피해를 입고 있음.
예컨대 불체자들의 인권을 보호하자는 사람들은 불체자들이 저지르는 범죄나 그들로 인한 임금정체에 대해서 싹 침묵해버림.
게다가 그런 진보적인 쪽을 팔아드신 정치인이나 학자 나으리들은 정작 자기들은 보수적으로 행복하게 살아감.
\vspace{5mm}

이걸 보는 대중들의 시선은 싸늘해지는 것임. 그런데 이게 우리나라만의 현상이 아닌 것임.
일찌기 미국에서부터는 히피, 여피를 겪으면서 이에 아주 냉소적이 되어버렸고, 일본은 전공투 세대에 대한 성찰이 지금도 이어져내려옴.
유럽에서 극우파 정권들이 부활하고 심지어 콧수염 히씨의 중2병적 책이 인기 모으는 것도 이와 무관하지 않음.
정치적 약자를 가장한 소수들이야말로 강경한 꼴통들로서 우리의 삶을 침략해오는데 정치적 올바름을 주장하는 사람들은
왜 우리들에게만 인내와 희생을 강요하느냐, 빡치지 않을 수 밖에 없음.
\vspace{5mm}

트럼프가 대통령이 될 수 있을지 없을지 모르지만 한낱 광대로만 보였던 그가 무섭게 치고 올라온 이유가 이것 때문임.
그의 발언 하나하나는 적어도 미국의 백인 중산층들이 하고싶은 솔직한 이야기임.
그의 주한미군이나 동북아 안보에 대한 발언도 사실 미국 입장에서라면 틀린 이야기는 아님,
주한미군이 미국 이익 때문에 한국에 남아있는 것이다라는 건 사실 우리들이 하는 주장일 뿐이지,
여태까지 주한미군의 수혜를 톡톡히 보았던 건 사실임. 그러나 우리가 미국에게 고마워하나? 양키 고홈이라고 까기나 했지.
알고보면 하나도 잘 해준 것은 없는 북한에 대해서는 같은 민족이라고 관대하게 대하고,
세계사적 측면에서 한국에 잘 해준 게 훨씬 많은 미국에 대해선 적대적으로 굴어야한다는 비일관적 태도를 취했음.
\vspace{5mm}

그나저나 트럼프는 꽤 대단한 인물임. 대통령 후보로만 기억하는 분이 많지만
이미 그 전에 부동산 재벌로써, 그리고 "거래의 기술"이란 책에서 보다시피 그 분야는 참 천재적인 인물임.
예능쇼를 잘 해서 그런가  대중들이 뭘 원하는지 잘 알고 있음.
작년까지만 해도 트럼프가 하는 개드립 때문에 사그라들거라는 사람들이 지금 현상은 뭐라고 설명할 수 있음?
20세기의 온갖 정치학적인 수사가 이미 안 먹히고 있는데도 여전히 20세기의 담론에 빠진 사람들 입장에서는 트럼프가 광인으로 보였을 것임.
\vspace{5mm}

만약 트럼프가 대통령이 못 된다 하더라도 그에게 환호보내던 유권자들의 선호도가 바뀌는 것도 아니죠.
이제는 더 이상은 정치적 올바름을 가장해서 그냥 수사적으로만 그럴싸한 메시지 얘기해보았자 먹히지도 않는다는 얘기죠.
확실히 시대가 또 바뀐 것입니다.
\vspace{5mm}

+
제가 미국인이었으면 샌더스 지지했을 것입니다. 세계적으로 찾아볼 수 없는 언행일치론자라서리.
다만 미국 역사상 케네디 이후로 또 암살당하는 대통령이 될 지도(...) 그 나라는 뭔가 근본적인 개혁을 하는 정치인이 암살당해서리.
\vspace{5mm}

++
미국도 이득을 취한다고 할지 몰라도 그건 거래가 아니라 우리가 입는 수혜이기 때문에 고마워하는 것은 인지상정이 아닌지.
만약 미국 안보우산이 아니면 군복무기간도 현재의 2배는 늘려야 하고, 국방비가 차지하는 비중이 엄청나게 증가해서
세금은 세금대로 내지만 누릴 것은 못 누리는 일도 일어나죠.
혹자는 이렇게 얘기하죠, 대신 우리가 미국의 무기 사주고 있지 않냐.
맞는 말입니다만, 그럼 \textbf{자주국방한다고 해서 미국 무기 구입 안 합니까?}
\vspace{5mm}

단지 대상이 미국이기 때문에 무조건 까고보자하는 논리는 사실 그렇습니다.
한미 FTA에는 그렇게 시위해대더니 한중 FTA에는 침묵,
위험이 극도로 과장된 광우병에는 아주 떠나갈듯하게 시위하더니, 멜라닌 분유나 중국 어선들의 불법행위에는 다들 침묵하죠.
이 정도가 되면 합리적인 척 하면서 미국 까는 논리가 어느 진영에서 생산해댔는지 물어볼 필요도 없습니다.
\vspace{5mm}

지금 고작 나온다는 얘기가 트럼프가 저 발언 하니까 "미국이 이익이 되는데 왜 철수하겠어 ㅋㅋ"라는데, 그거야말로 대착각이죠.
미국 입장은 이제는 주한미군을 빼내서 인계철선 부담 없애고 대신 한국 보고 자주적(?)으로 북한, 중국 상대하라 그 이야기입니다.
자기들도 책임은 안 질 테니 알아서 하라 그건데, 이걸 막는 것이 바로 "의리론"이 아닐까 싶은데.
무조건 미국이 우리에게 도움 준 것 없다 하면 미국에서야 좋아하죠.
한국 사람들이 자기들을 싫어하는데 뭐하러 그 짓을 하냐 하면서 빼면 그만이죠.
\vspace{5mm}

물론 이게 미국을 찬양하지 말라, 미국도 이익이 있어서 그런 거라고 하는 건 맞는 말일 수도 있는데
냉정히 손익득실 따지면 이거 사실상 일방적인 원조라고 해도 지나치지 않을 정도로 우리가 이익보고 있는 게 맞습니다.
트럼프가 그걸 알고 벌써부터 수사 취하고 있는 거죠. 그리고 저게 틀린 말은 아니라서 우리도 가만히 선거를 지켜봐야합니다.
\vspace{5mm}

반미감정이야 군복무기간이 늘어나고 국방비로 내야 할 세금 항목이 신설되거나 그러면 단번에 날라가지 않나 싶은데.
원래 누리기만 하던 우리나라 사람들이 그 당연해보이던 게 상실되면 어떤 입장 취할지야 뭐.
\vspace{5mm}









\section{성격이 급한 사람의 문제}
\href{https://www.kockoc.com/Apoc/715622}{2016.04.07}

\vspace{5mm}

그 테스트는 '시간감도'에서 해보시길 바랍니다만.
\vspace{5mm}

성격이 급한 사람의 문제는
\vspace{5mm}
\begin{enumerate}
    \item 과욕을 부린다 $\rightarrow$ 터무니없는 스케줄을 짠다 $\rightarrow$ 자폭한다
    \item 사소한 것에 민감하다 $\rightarrow$ 스트레스를 잘 받는다 $\rightarrow$ 빨리 GG 친다
    \item 필수 단계를 생략한다.
\end{enumerate}
\vspace{5mm}

하나만 있어도 치명적인데 보통은 3가지가 있죠.
성격 급한 사람은 독학재수를 하기 좋아한다는 것도. 그래서 \textbf{망하기 딱 좋습}니다.
이런 사람들이 학원에 가면 다른 친구들과 보조를 맞추므로 자기 성격이 급한 것을 보정할 수 있을텐데.
학원이 싫다면 최소한 도서관이나 스터디를 해도 되는데,
자기가 무슨 달려라 하니도 아니고 빨리 달리고 싶다보니 n수의 끝까지 달려라 콕창이 되어버립니당.
\vspace{5mm}

성격 급한 건 하루아침에 안 고쳐집니다(제가 그러니 잘 알죠)
1번의 경우는 그래서 계획을 안 짜는 걸 권합니다. 그냥 과제 하나 정한 다음 \textbf{그걸 천천히 끝내는 게 낫습니다.}
어차피 빨리 끝낸다고 해도 대충 할 게 뻔하거니와, 당사자가 정말 잘 지칩니다.
2번의 경우는 둔감력을 키워야합니다.
사소한 것도 그냥 대범하게 넘기고 고통을 덜 느끼는 식으로 가야해요.
3번의 경우는 사소한 행위라도 5$\sim$10단계로 세분화해서 실천하는 연습을 해야합니다.
어떤 행위를 할 때에 메모지에 뭘 해야할지 하나하나 적고 그걸 따라가는 것도 좋습니다.
\vspace{5mm}

이런 조언조차도 성격이 급한 사람에게는 귀찮은 걸로 보이겠으나,
저대로만 한다면 그 사람은 기존에 저질렀던 실패들을 반복하지 않으며, 적체된 부채들을 조기에 청산할 수 있습니다.
\vspace{5mm}






\section{혼인율 최저}
\href{https://www.kockoc.com/Apoc/715867}{2016.04.07}

\vspace{5mm}

시골이 도시보다 뒤떨어지는 이유는 하수구가 없기 때문이다.
\vspace{5mm}

시골에서는 오폐수가 그대로 티가 난다. 폐기물이 발생하면 그대로 지저분해지고, 폐수가 생기면 하천이 더러워진다.
사람들이 얼마 없고 다 아는 사이인데다가 노인들의 입김이 강하다.
그래서 도덕적으로 살아야하며(예외적인 야만 행각이 없는 건 아니지만)
스트레스를 받더라도 욕망을 해소할 수 있는 소비의 기회가 없다.
그야말로 시골은 자기만족에 빠진 나머지 자유가 제약당하는 그런 공간인 것이다.
\vspace{5mm}

반면 도시가 시골보다 발전하는 건 하수구가 있기 때문이다.
\vspace{5mm}

일정한 수준만 지키면 각자 뭐하고 놀든 간섭하지 않는다.
그래서 욕망을 마음껏 해결하고 돈만 있으면 더럽게 놀더라도 내일의 해가 뜨면
언제 그랬냐는 듯이 정상적으로 돌아가는 게 도시다.
다시 말해서 도덕에 꽉 매이지 않아도 자유롭게 즐길 수 있다.
자유가 있으니 새로운 것들이 늘 고안, 실천된다. 기존의 가치관 구속이 심하지 않아 좋은 것이 바로 채용된다.
환락을 즐긴 찌꺼기는 하수구에 흘려보내고 다음 날 깨끗이 샤워하고 멀쩡하게 다녀도 되는 곳이 도시다.
\vspace{5mm}

아무리 대자연이 좋다고 한들 시골 인심이 최고이죠라고 한들 이건 다 거짓말인 것이다.
그 누가 답답한 곳에 가서 살려하나.
그래서인가 원래 도시에서 살던 사람들은 그 도시의 생리를 알기 때문에 적당히 욕망을 충족하고 자기들만의 룰을 지키지만,
촌사람들이 도시에 오면 자제력이 없어 막 나가는 경향을 보여주기도 한다. 하수구가 무한하다고 믿는 것이다.
\vspace{5mm}

역대 혼인율이 최저라고 한다. 물론 내 경우도 별로 결혼은 생각이 없는데 지식인과 언론에서 정말 중요한 문제에 침묵하고 있다고 생각한다.
\vspace{5mm}

출산율이 낮고 혼인율이 떨어진 게 정말 \textbf{집값 때문이겠나}.
그런 논리라면 과거 베이비붐 세대에는 단칸방에 살아도 애 둘 셋은 낳고 살았다.
과거에는 열심히 살면 부자가 될 수 있잖아요... 라는 헛소리는 하지 말자. 어느 시대에나 희망 못지 않게 절망은 있었으니까.
\vspace{5mm}

불편한 진실은 그것이 아닌가?
이제 결혼이 거추장스러워졌기 때문이지.
\vspace{5mm}

결혼을 하지 않더라도 남녀 공히 성욕을 채울 수 있다. 아니, 결혼을 하지 않으면 더 많이 다양하게(?) 채울 수 있지 않나?
유명한 인기인들이 뒤늦게 결혼하는 이유가 뭐나. 즐길 건 다 즐기고 이제 막판에 '꼰대' 노릇을 하고 정착하겠다는 얘기다.
그건 생전에 수많은 살인을 저지른 콘스탄티누스 대제가 죽기 전에야 세례를 받아서 '깨끗한' 상태로 천국에 가겠다는 것과 비슷하다.
적어도 90년대까지는 프리섹스에 대해선 상당히 보수적이었고 혼인 전의 관계나 임신에 대해선 정말 가혹해서 결혼이 강제되었다.
그러나 지금은 그런 게 있나?
\vspace{5mm}

거기다가 자녀에 대한 가치관도 달라졌다. 많이 낳는 것보다는 1$\sim$2명 나아서 집중적 투자를 하는 게 낫다.
딱히 자녀를 낳지 않아도 어차피 남의 자식들이 일해주는 걸로 생기는 복지서비스를 누릴 수 있는 세상이다.
(거기다가 자녀를 안 낳은 걸로 페널티를 입으면, 국가보고 왜 차별하느냐고 시위하면 된다)
그렇다면 원없이 연애질을 즐기다가 한계선에서 "집안이 좋은 배우자"를 만나서 M$\&$A를 하면 된다.
\vspace{5mm}

가장 중요한 건 결혼을 바라보는 시각의 차이를 모른다는 것이다.
과거에는 결혼이 \textbf{욕망을 실현하는 수단}이었다.
유교적 가치관 내에서는 남녀간의 연애라거나 성적인 것들이 터부시되었고 결혼을 하지 않고 이런 걸 즐기면 제재받았다.
그래서 그런 것을 즐기고 싶으면 결혼을 하면 되었다. 즉, 결혼이 바로 하수구 역할을 해주었던 것이다.
물론 결혼해놓고도 바람을 피우는 경우도 적다고 할 수 없지만, 아무튼 결혼이 바로 욕망실현의 수단이었다.
\vspace{5mm}

그러나 지금은? 그 결혼이 더 이상 하수구 역할을 해주지 못 한다. 욕망을 방해하는 족쇠가 되었다.
결혼=시골, 미혼=도시와 같은 구조가 완성되었다.
사람들이 선망하는 건 다양한 사람들과 즐기는 프리섹스, 아니면 자기가 원하는 분야에서 극강의 자아실현을 하는 것 등이다.
현재의 일부일처제 혼인제도는 이 어느 쪽이든 \textbf{도움이 되지 않는다}.
결혼을 하지 않아야 더 많은 자유를 누릴 수 있고 더럽게 놀 수도 있다(이제는 더럽게 놀아도 제재받지 않는다)
\vspace{5mm}

시대가 이렇게 변했는데 왜 점을 아무도 지적하지 않나.
사실 일부일처제가 인류 탄생 때부터 있던 제도가 아니다. 역사적 관점에서 보자면 이 역시 과거의 신분제도와 비슷할 수도 있다.
\vspace{5mm}

그런데 이 대책에 대해서 '경제적 문제'만 언급하고 있으니 답답하기 짝이 없는 것이다.
물론 이 사회의 많은 것들이 유물론적으로 설명되는 건 사실이다.
그런데 마르크스의 사상이 왜 틀렸나. 그는 인간의 정신적인 면을 너무 단순하게 생각했기 때문이다.
인간의 정신 활동을 물질적인 것에 종속된 관념으로 명쾌히 정리해버렸다. 그걸 맹신한 공산주의자들은 수많은 인민들을 도탄에 빠뜨렸다.
저출산 해결을 위해서 복지를 강화하자, 혼인율을 높이기 위해서 집값 문제를 해결하자. 이건 그냥 대책부터 정하고 원인분석을 하는 게 아닌가.
\vspace{5mm}

10년 내에 두가지가 사라진다고 보고있다.
하나는 취업, 그리고 다른 하나는 혼인이다.
\vspace{5mm}

물론 극단적으로 저게 zero가 된다고 할 수는 없을 것이다. 다만 그런 것들이 사라질 때는 포장지는 있는데 속은 텅 비어버린다.
예컨대 취업의 경우 분명 취업들을 형식적으로 한다, 그런데 그 다음 3년 내에 다수가 퇴사해버릴 수도 있다.
이미 비정규직을 뽑는 것 자체가 '개인 대 개인'의 계약으로 가고 있는 경우다.
혼인의 경우도 그렇다, 이제는 혼인이 이혼의 예비절차가 되어버렸다.
게다가 간통죄가 사라진 이상 외도를 해도 별 문제가 되지 않는다. 단지 변호사를 통해 돈을 얼마나 주고받느냐 문제가 되었다.
\vspace{5mm}

저출산 현상이 기득권이 다 해먹어서?
과거 봉건제 시대에 그래서 저출산이 있던 건 아니지 않은가.
첫째는 피임이 가능해서이고, 둘째는 자식을 낳기보단 개인의 자유를 추구하기 위해서이다.
확실한 해법이 뭔지는 안다. 하지만 그건 '역사'를 퇴보시키는 것이기이에 선택할 수가 없다.
저출산의 원흉은 \textbf{"자유"}다.
세계적으로 저출산 현상은 자유로운 민주주의 국가에서 벌어지고 있다.
반면 무슬림 국가들이나 살기 힘든 곳이 오히려 출산율만큼은 높다(물론 사망율은 예외다)
\vspace{5mm}

그럼 앞으로 세상은 어떻게 변하게 될까.
위 영상처럼 되어서 점점 아이큐가 낮아질 수도 있다(어떻게 보면 대자연의 조화가 아닌가)
\vspace{5mm}

그런데 그 전에 가장 유력한 것은
\vspace{5mm}

http://book.naver.com/bookdb/book$\_$detail.nhn?bid=6192691
\vspace{5mm}

이 책에서 경고한 바처럼, 출산율이 높은 종교인들이 득세하면서 근본주의 시대로 다시 돌아갈 가능성이 높다는 것이다.
근본주의적인 종교를 믿는 사람들은 신앙심 때문에 애를 낳는다. 이게 사소한 차이로 보이더라도 시간이 흐르면 지수함수적 변화라는 것.
특히 저출산 재앙을 목도한 종교인들은 사회를 중세시대 수준으로 유지해야한다고 할 것이고 이건 정말로 잘 먹힐 것이다.
\vspace{5mm}

이렇게 되면 결혼은 다시 상식적(?)인 수준으로 돌아올지도 모른다. 물론 지금의 민주주의나 자유지상주의는 물건너가겠지
\vspace{5mm}




\section{자전 확대}
\href{https://www.kockoc.com/Apoc/724850}{2016.04.12}

\vspace{5mm}

https://www.hankookilbo.com/v/e59f2a6643204521a60fcd49cdba8d70
\vspace{5mm}

대학이 신성한 상아탑이라는 메시지와 관계없이, 저출산의 영향력이 강해지고 있죠.
게다가 수시나 정시의 변별력이 떨어지는 것도 무관치 않습니다.
그럴 바에는 뽑고 난 다음에 갈라버리겠다는 것이죠.
10여녀전인가 대학이 무슨 취업하는 학원이냐... 라는 지금 생각하면 매우 배부른 이야기 나왔는데
이제는 취업만 시켜준다면 갓학으로 인정받습니다.
\vspace{5mm}

대학 구조조정에 대해선 역시나 수구 기득권 자본의 음모... 하더니만
구조조정 못 한 채로 현 상황까지 와버리니까 다 침묵.
대기업들도 노조의 입김이 강한 곳이 유감스럽지만 오늘 뉴스 뜬 거제도 꼴이 나버립니다.
구조조정을 하지 않으면 결국 구조조정 \textbf{당해}버리지요.
\vspace{5mm}

그건 그렇다 치고 저래버리면, 대학 과정까지도 선행학습해버리는 케이스가 생겨나겠네요.
입학하자마자 유리한 과목과 좋은 학점으로 '돈되는' 분야으로 갈 것이 뻔하니.
과거에야 학과별로 모집해서 경쟁을 줄면서 계급장 노릇을 했는데 이제는 그것도 먼 옛날 이야기가 되어버릴 듯.
다시 말해서 상위권 고딩들이 대학교 과정 미리 준비한다고 대학국어(...)를 미리 공부하고
김x종 교수님 미적분 책을 보고 대학 물리, 화학 선행하는 것도 일도 아닐 듯(영어야 뭐 이미 그 정도는 지금 다 선행하고 있으니)
\vspace{5mm}






\section{취업과 결혼}
\href{https://www.kockoc.com/Apoc/724863}{2016.04.12}

\vspace{5mm}

생각해보면 조선 말기에 '양반 제도'가 감히 사라질 거라고 믿었던 사람도 없겠고
말 그대로 일제시대에 태어난 사람들이 1940년대에 일본이 전쟁에 져서 물러날 거라고 본 사람도 그리 많지는 않았을 겁니다.
여담이지만 더 이야기하면 당시 조선인들은 일본에서 독립하는 것보다는,
잘 나가는 일본 밑에서 조선인들의 세력을 키워서 크게 한건 해먹자라고 하는 경우도 많았고
그래서 잘 나가는 조선인들이 자기보다 못 한 일본인들에게 "이제 너희가 물러날 때가 된 것 같다"라고 하기도 했따는데.
현재와 같은 취업제도는 역사가 사실 100년도 될까말까이죠.
레디메이드 인생이라는 소설에서 보다시피 조선인들이 취업과 실업으로 고민했던 게 일제시대였지 조선시대는 아니었음.
그리고 지금과 같은 연애결혼도 역사가 50년은 되었는지도 모르겠습니다.
연애결혼이라는 것도 산업화된 1970년대에야 본격화되었고 그 전까지는 농경사회였으니 부모가 점지어주거나 선보는 식으로.
\vspace{5mm}

현재의 실업이나 미혼에 대한 진단의 문제는, 이게 근본적인 변화라는 이야기를 하지 않는다는 것입니다.
산업화 시대에야 법인이 사람을 고용해서 착취해먹는 게 가장 효율적이니까 고용이 늘었겠으나
정보화 시대에는 일시적으로만 부려먹거나 아웃소싱하는 게 효율적입니다. 그럼 정규직을 많이 뽑는 건 미친 짓이 되어버립니다.
씁쓸하지만 결혼 역시 마찬가지인데 이거 하나하나 뜯어보면 상당히 시대착오적이 되어버립니다.
프리섹스가 보편화되었고 간통죄가 사라진 이상 배우자에게 종속될 이유가 없고
맞벌이하기 때문에 이제는 자녀도 돈으로 키우고 있죠. 거기다가 결혼의 관행이라는 건 불만족스러운 게 많죠.
\vspace{5mm}

가장 잔인하고 비참한 선택이 해답이죠.
이미 사회가 변했다, 그리고 저것들은 시대착오적이다라고 생각하면 현재의 비정상이 알고보니 '정상'인 것이죠.
저것들을 비정상으로만 보면 그냥 일시적인 문제로 진단하게 됩니다
그러나 저게 정상이라고 본다면 그럴 수 밖에 없는 필연적인 이유를 발견할 수 있고, 다른 현상까지도 예견할 수 있는 것이죠.
\vspace{5mm}

그리고 이런 관점에서 보자면 1980년대$\sim$2000년대는 정말 배부르게 살았구나... 라는 진단을 할 수 있습니다.
그게 요즘 586 비난론의 일부이기도 한데 비난하느냐 마느냐를 떠나서, 적어도 이 586식 가치관이나 당위적 명제는 지금 봐도 시대착오죠.
다만 지금 미래를 어떻게 대비할지는 답이 나오지 않고 있습니다.
늙은 세대는 여전히 박정희 시절 환상에 젖어서 그 쪽 정치세력을 지지해주면 그 시대의 호황이 올 것이라고 단단히 착각하고 있고,
젊은 세대는 지금 먹고살기 힘든 게 다 기득권 때문이다라고 하면서 간판만 진보지 한물가버린 쪽을 밀어주고 있죠.
(그런데 가장 불편한 진실은 그 늙은 세대가 1980년대 민주화를 달성한 세대인데 지금은 정반대로 달려가고 있다는 것입니다)
\vspace{5mm}

이 와중에서도 일단 빚을 내고 취업과 결혼으로 가면 중산층 이상으로 살 수 있다는 환상이 있는 것 같은데.
그건 취업과 결혼이 정말 그만큼의 수익을 보장해주느냐를 검증해보아야 아는 것이죠.
그래서 대기업 갔다가 때려치우고 공무원 시험에 몰리며, 연애도 '소득이 많거나 집안이 좋은' 배우자를 골라서 정략적으로 하게되는 것입니다.
겉으로는 시장주의를 경멸하는데 현실에서는 철저히 시장주의를 추종하며 살아가는 것이지요.
\vspace{5mm}

아무튼 이런 시대흐름을 그나마 이야기한 사람이 공병호(...)입니다.
거의 책을 찍어내다시피 하는 분이라서 책의 희소성은 떨어지지만 메시지 자체는 지금 보면 그리 틀린 건 없었죠.
유하 감독의 영화 "결혼은 미친 짓이다"도 지금 보면 참 소름이 끼치네요.
처음 보면 정말 이런 로맨스물이 있구나, 감우성씨 엉덩이 탄탄하구나... 그랬는데 지금보니 남녀의 복잡한 심리라는 게 참,
유부녀가 애인과 남편을 오간다.... 는 건 당시 웬 미친 발상이냐고 했는데, 지금은 어, 이런 경우는 현실에도 널렸어(...)라는 게 인식이고
\vspace{5mm}

그 다음은 무엇이 해체될까요?
웃자고 하는 소리가 아니라 정말 제임스 카메론 감독의 "아바타"가 그 다음을 잘 시사해준다고 믿고 있습니다.
\vspace{5mm}






\section{롯데월드타워}
\href{https://www.kockoc.com/Apoc/724878}{2016.04.12}

\vspace{5mm}

졸릴 때 보면 좋습니다.
\vspace{5mm}

저것도 지을 때 상당한 풍자글이 많았는데.
소위 문인들은 토목공사를 비판하면서 그게 시대정신인양 이야기하지만
경부고속도로, 63빌딩, 남산타워, 청계천, 그리고 롯데월드타워는 결국 "인식의 수준"을 바꿔버리지요.
\vspace{5mm}

청계천도 왜 자연하천이 아니냐 하는 별로 현실적이지 않은 비판이 많았으나
일단 그게 뚫리고 나서는 다들 서울 한가운데 인공 녹지가 있는 것을 당연하게 여기게 되었죠.
\vspace{5mm}

과거에는 서울에 상경한 촌사람들이 63 빌딩을 보고 입이 떡 벌어졌다지만 지금은 에이 시시해하여도
롯데월드타워를 밑에서 올려다보면 하늘이 저렇게 높았나를 실감하게되죠.
그리고 앞으로 이런 빌딩들이 더 많이 들어서겠구나라고 하면 어떤 미래가 펼쳐질까.
\vspace{5mm}

그런데 \textbf{마천루의 저주}가 여기에도 먹힐지가 궁금하긴 합니다 $-$$-$
\vspace{5mm}








\section{국가 신용등급}
\href{https://www.kockoc.com/Apoc/730632}{2016.04.15}

\vspace{5mm}

\href{http://news.naver.com/main/read.nhn?mode=LSD&mid=sec&sid1=101&oid=014&aid=0003634715}{링크}
\vspace{5mm}

호불호가 갈릴 지점이긴 하지만 저걸 무시 못 하죠.
\vspace{5mm}

선거에서 수권 정당이 누가 되느냐 그런 건 큰 차이를 낳지 못 합니다.
IMF 다음부터는 국가는 거의 회사나 다름없이 변해서리.
IMF 이전까지는 정말 경제, 경영 관념이라는 게 없었음. 으싸으싸하면 경제가 다 살아난다 믿었죠.
IMF 터지고나서야 알았던 것이죠. 위대한 한민족은 걍 허상이고 제삿상 시루떡마저 \textbf{'달러'}라는 것을
현 야당이 집권할 때에도 무디스, S$\&$P, FITCH 눈치 보는 경제정책 폈죠.
\vspace{5mm}

우리가 열심히 일하면 된다.... 는 건 이제 먹히기 어려워요. 전세계적으로 단순노동가치는 낮아지고 대체되는 추세라
인정받는 건 \textbf{고급노동} 뿐입니다. 그리고 그 나머지는 정말 \textbf{'자본'}을 얼마나 확보할 수 있느냐 문제죠.
\vspace{5mm}

아니 이 꼰대가 왜 그러냐 하는데 이건 기시감 때문서입니다.
아이엠에프가 터진 뒤에 노동법 날치기 통과 반대나 노동투쟁이 아이엠에프가 원인이란 주장이 비웃음을 산 적이 있었죠.
그 당시 주류설이 이게 다 개발위주 독재, 분배를 안 한 것 때문이다라고 했고 저도 그걸 믿었는데
\vspace{5mm}

나중에 이것저것 공부하고 느낀 건 저 이야기가 개소리만은 아니었다는 것입니다.
개발위주 경제가 원인이 되긴 했죠. 개발을 안 해서 아프가니스탄 수준이었으면 아예 망할 리도 없으니까. 성장해야 망하죠(...)
당시 아이엠에프도 유탄에 맞아서 치명상 입은 게 컸고, 당시 정치권이 잘 대처하면 잘 넘길 수도 있었는데
결과적으로는 당시 노동자들 기득권 지킨다고 했다가 다 망해버렸고, 구조조정을 안 하니까 구조조정을 \textbf{당해버렸죠}.
그러나 그 가운데에서 부자들은 오히려 더 벌어댔습니다.
공적자금이야 회수 못 하고 증발댄 것도 많았고 외국계 자본은 헐값에 알짜기업을 먹었죠
현재의 건물주들도 그 당시 헐값된 땅, 건물을 구입한 사람들이고
\vspace{5mm}

지금이 저 정도까지인지는 모르겠는데, 어떻게든 노동개혁(?)은 하긴 해야할 겁니다. 그걸 안 하면 외국계 자본이 나가는데 별 수 있음?
그럼 부자들이 세금 더 내라, 대기업이 다 부담하라 하겠지만 그럼 부자들은 외국으로 토끼면 그만이고 대기업도 외국계 자본 나가면 걍 망하죠.
삼성은 엄밀히 말하면 외국인 회사입니다.
\vspace{5mm}

국민들의 선택은 위대하다... 라고 하는데 그럼 망하는 경제가 복지확대로 늘어날지는 가히 의문입니다.
복지 확대를 하면 빚이 늘어나는 거지 그걸로 생산성이 증가하거나 소비가 느는 건 아니죠. 한국경제는 내수경제로 버틸 수 있는 구조도 아니고?
지금 그나마 버티는 게 중국 덕분이라서 알아서 중국에 박박기고 있는 현실인데 저러다가 중국 쪽도 여의치않으면?
\vspace{5mm}

노동개혁(?)이라는 게 진행되지 않으면 그 때는 그냥 각자도생하는 걸 기대하는 게 나을지도 모릅니다.
노동개혁은 되겠죠. 우리가 하든가, 아니면 \textbf{당하든가}.
그런데 하면야 중장년층 울상이지만 청년층은 살아도, 당하면 둘 다 답이 없겠죠
\vspace{5mm}

아마 야당도 이걸 안다면 결국 은근슬쩍 여당이 하던 걸 좀 소프트하게 가지는 않을까 싶지만
하필 내년이 대선이라서 어찌될지는 모르겠습니다. 개인적으로는 정말 1997년대와 비슷하단 느낌이 들고 있어서리.
생각해보면 머리좋은 사람은 더 벌어댈 수 있는 구조이긴 하니 그 사람들이야 환호하겠네요.
\vspace{5mm}

돌아가는 것 보시면서 금전적인 분야에 신경쓰시길 바랍니다.





\section{[인물 vs 인물 001] 제갈량과 사마의}
\href{https://www.kockoc.com/Apoc/736095}{2016.04.19}

\vspace{5mm}

경영에 있어서는 미국식 합리주의, 일본식 품질주의, 그리고 중국식 다면주의가 있다.
이 근거가 뭐냐고 한다면 그거야 절반은 뇌피셜이다.
미국식 합리주의는 테일러리즘, 즉 시스템을 잘 설계하고 그걸 밀어붙여 수율을 높인다, 다만 인간에 대한 통찰이 없다.
일본식 품질주의는 잇쇼켄메이 정신에다 도요타 시스템으로 재고를 최소화하는 전제 하에 근로자가 달인이 되어 품질을 극대화,
이게 미국에 lean system으로 역수입되긴 했다만 이 역시 불확실성에는 대응할 수 없다.
\vspace{5mm}

중국식 다면주의는 그 실체가 존재하느냐 하는 문제가 있다. 여기서부터가 바로 뇌피셜의 영역이다.
가령 삼국지의 유비는 천하를 눈물을 흘려서 집어 삼킨다.
초한지의 항우만큼 인간적이고 능력좋으며 효율적인 지도자도 없다. 그러나 그 항우는 무능하고 비정하며 곡선적인 유방에게 패한다.
그런 중국인들의 역사적 노하우가 허세만이 아닌 건, 화교를 포함한 중화권의 성장을 보면 된다.
중국인들이 역사적으로 강조한 건 시스템보단 사람이었다. 그것도 사람의 이면과 본성을 통찰해온 것이다.
\vspace{5mm}

삼국지에서 제갈공명과 사마중달을 비교하면 다수가 제갈공명이 한수위라고 평할 것이다. 다만 천운이 따라주지 않은 거라는 주석을 붙이면서
\vspace{5mm}

그러나 실질로 보자면 제갈량은 사마의보다 못 한 인물이었다.
유교적 관점에서나 혹은 일본식 스타일로 보자면 제갈량이 더 비장하고 극적인 인물일지도 모른다.
그러나 제갈량은 저 중국식 다면주의 측면에서는 낙제생이었다.
그건 바로 '사람'을 통찰할 줄 몰랐으며 자존심이 강했단 것이다.
\vspace{5mm}

우선 삼고초려 얘기부터 보자. 이게 허구인지 아닌지 떠나서 이 얘기부터가 제갈량이 자존심이 강하다는 것을 보여준다.
역시 허구이지만 오나라에 가서 키배를 뜨는 장면도 제갈량이 매우 자존심이 강한 인물이라는 것을 암시한다.
그리고 이것은 나중에 오장원에서의 승부에서 족쇄로 작용한다.
\vspace{5mm}

제갈량이 마지막 북벌을 하면서 위의 사마의에 어그로를 끌어지만 사마의는 철저히 반격하지 않는 전략을 세운다.
여자옷을 보내고 분을 보내 도발하여 다른 장수가 화를 내도 사마의는 아무런 미동을 보이지 않았다.
즉, 사마의는 자존심 따위는 승리를 위해 버릴 수 있던 인물이었단 얘기다.
실제로 제갈량 사후 사마의가 조상과 권력다툼을 할 때 일부러 정신나간 척 치매걸린 척 했던 것도 자존심이 없기에 가능했던 일이다.
만약 사마의가 제갈량처럼 자존심이 강해서 심리전에 휘말리는 냄비였다면 삼국지의 결말은 매우 크게 바뀌었을 것이다.
\vspace{5mm}

이건 실제로 그들의 승진 과정만 보아도 그렇다
\vspace{5mm}

제갈량은 요새로 치면 융중에서 책이나 읽으며 백수짓을 해도 굶주릴 게 없는 나름 금수저였다.
그리고 자신을 초빙하러 온 유비를 두번이나 쫓아냈으며 세번째 만남에서 등용되지마자 바로 요직에 앉아 권력을 행사한다.
이건 바꿔 말해 제갈량은 낮은 직급의 일이나 자존심이 상하는 일 따위는 할수 없다는 이야기였다.
그에 비해 사마의는 애초에 후한의 군역소에서 일하던 하급관리였다.
낮은 데서부터 출발해 일하다가 조조에게 인정받아 산전수전을 다 겪었던 인물이다.
\vspace{5mm}

공명과 대진할 때도 마찬가지다.
공명이 사마의를 떠보기 위해 사자를 보냈을 때, 중달은 다른 건 궁극의 말돌리기 기술로 회피하면서 핵심적인 것만 묻는다.
공명의 잠자는 시간, 공명이 뭘 먹는가.
바보같은 사자는 자기가 존경하는 공명의 일상이라고 자신만만하게 대답한다.
"우리 승상은 아침에 일찍 일어나고 밤에는 늦게까지 컥챗 군사일에 힙쓰고 계십니다. 작은 일도 다 챙기고 식사는 적게 하십니다"
\vspace{5mm}

사자가 돌아간 후 사마의는 회심의 미소를 지으며 부하들에게 다음과 같이 얘기한다.
\begin{itemize}
    \item[] "공명은 식사 양이 적고 할일이 많다. 이건 공명이 부하들에게 큰일을 맡기지 못하는 걸 말해준다"
    \item[] "이제 공명의 목숨은 얼마 남지 않았다"
\end{itemize}
\vspace{5mm}

공명이 사망한 후에도 사마의는 공명이 이끌던 촉군을 추격하려고 하다가 도중에 중단하고 뒤쫒지 않는다.
이것이 죽은 공명이 산 중달을 쫓았다라는 문구의 진상이지만, 이건 사마의가 감정에 휩쓸리지 않고 미끼도 물지 않는 냉정한 인간임을 보여주는 것이다.
그런 사마의의 손자가 진을 건국한 것도 무리가 아니다
\vspace{5mm}

한편 공명이 정치적으로도 공명정대했다라고보기만도 어렵다.
첫째로 마속을 중용했던 것. 만약 사마의였다면 마속 같은 인간을 중용하지도 않았겠지만, 중용했어도 브레이크 장치는 준비했을 것이다.
그러나 공명은 선주 유비가 마속을 쓰지 말라고 유언했음에도 이를 무시하고 자기 수제자라고 중책을 맡긴다. 보다시피 말아먹는다.
둘째로 위연을 홀대했던 것. 물론 나중의 일이지만 장안을 공략하는 건 위연의 방법이 훨씬 더 나았다.
그러나 공명은 위연을 터무니없이 홀대했고 끝까지 푸대접을 한다.
\vspace{5mm}

즉, 공명은 단지 백성들을 부리는 정치는 잘 했을지 몰라도, 정말 중요한 인재들을 다루는 데에서는 젬병이었다는 것이다.
선주인 유비가 떠돌이 생활을 하면서도 그 놈의 말빨과 눈물로 각 지역의 인재들을 쓸어왔던 것과는 대조적이다.
파촉 땅이 위보다 작아서 어차피 질 수 밖에 없지 않느냐하겠지만 사실은 틀린 얘기다.
원래 촉이 차지하던 곳은 중국에서 꽤 비옥했던 곳이다. 문제는 바로 그 놈의 인재풀이었다.
유비가 근거지가 없어 사방을 떠돌아다니고 자존심도 죽였기 때문에 본인의 매력으로 인재들을 쓸어올 수 있었서 촉이 버틴 것이다.
\vspace{5mm}






\section{관상술과 사주팔자의 문제.}
\href{https://www.kockoc.com/Apoc/736121}{2016.04.19}

\vspace{5mm}

음양오행과 관련된 동양적인 것들의 문제는 이것들은 아리스토텔레스로부터 기원된 서유럽의 철학, 과학과는 다르단 것이다.
\vspace{5mm}

우리가 배우는 서유럽의 수학, 과학, 철학은 참과 거짓을 명쾌히 나누려고 한다.
그 참과 거짓을 가리지 못 하거나 모순이 생기면 기존의 패러다임이 붕괴되고 다른 패러다임으로 간다.
그리고 이것이 서양이 세계를 재패한 이유다.
\vspace{5mm}

그런데 동양학문이라는 것들은 $-$ 사실 학문이라고 보기는 무리일지 모르지만 $-$ 이런 서유럽식 접근방법으로 가면 낭패를 본다.
말하지만 이것들은 True나 False로 명쾌히 구분되지 않는다.
음양이란 말부터가 모든 사물에는 이면이 있단 얘기이다
오행부터가 대상 하나를 개별적으로 고찰할 수 없고 반드시 관계측면에서 바라보아야한다는 얘기다.
즉, 서양학문은 참이냐 거짓이냐 명쾌히 따진다. 그리고 거기다 확률, 통계적인 기술과 장치를 마련해둔다.
그에 비해 동양학문이란 것들은 애초에 참과 거짓을 따지지 않는다.
\vspace{5mm}

그렇기 때문에 동양 것들은 사이비다라는 말은 틀린 이야기도 아니다. 왜냐면 애시당초 참과 거짓을 구분하기도 힘들지 않나.
달의 위상변화를 보아도 그렇다.
서양적 패러다임으로 치면 저 달은 모월 모시에 보름달이다라고 정확히 말한다.
동양적 패러다임은 '달은 차니까 기운다'. 이렇게 보는 것이다.
\vspace{5mm}

사람에 따라서는 동양적인 것들이 심오하다고 할 수도 있겠지만 생각해보면 이건 죽도 밥도 안 되는 것이다.
뛰어난 요리사라면 퓨전푸드겠지만, 평범한 요리사라면 개밥이 되는 것과 똑같은 원리다.
그렇기 때문에 동양학문을 직관의 영역이라고 말하는 것이다. 다시 말해서 직관이 아니면 별로 쓸모가 없기 때문이다.
\vspace{5mm}

그래서 관상술이나 사주팔자를 '참과 거짓이 뚜렷이 구분된다'고 믿고 여기에 미쳐있다 망한 사람이 많다.
풍수지리만 예를 들어보자. 고전적인 풍수지리는 배산임수에다가 남향을 선호한다. 북으로는 산, 남으로는 강
실제로 조선시대 건축물들이 이런 셈이다.
그러나 지금은 어떤가? 강남이 이 고전적인 풍수지리 틀에 맞춰져있나?
그런데 생각해보면 이건 방위 개념을 절대화시켰기 때문에 그렇다. 북쪽방향이 반드시 그 북쪽일 이유가 없는 것이다.
\vspace{5mm}

사주팔자의 경우도 그렇다. 아마 한번 정도는 소문난 곳에 가서 죄인이 된 심정으로 자기 사주를 물어보고 과거맞추는 것에 놀랄 것이다.
그런데 재밌는 건 미래는 못 맞춘다. 이걸 가지고 혹자는 콜드리딩이라고도 하는 데 틀린 이야기는 아니다.
그러나 더 근본적인 건 그렇다. 우리가 평가하는 과거는 현재의 시점에서 재구성된 것이다.
과거 당시에는 매우 잘 나간다 혹은 좋았다 느껴진 게 지금 생각하면 최악이고, 그 역도 성립하는 것이다.
우리는 미래에 시험에 합격하는 것이 좋은 것이라고 생각한다. 그런데 더 먼 미래로 가면 시험에 떨어지는 것이 차라리 나았다라고 볼 수도 있다.
그런데 이것이 서양적인 OX로 명확히 구분되나?
하지만 사람들은 그렇게 참과 거짓으로 재단하려하지 않나.
\vspace{5mm}

아무튼 이 분야도 참 골치아프기 때문에 나름 책을 읽고 정리해보고 느낀 건
사람들의 접근방법 자체가 애초에 틀렸다는 것이다.
사주관상 그런 것들부터가 검증 안 된 사이비가 많은 것도 있지만, 더 중요한 건 '사유의 틀'부터 다르다.
예컨대 시간 개념을 보자. 서양식 시간 개념은 시분초로 미세하게 나눠서 한 점이라는 걸 따지려고 한다.
즉 정오라고 한다면 12:00:00로 한 점을 가리키는 것이다.
그에 반해 동양적 시간 개념은 한자 풀이 그대로 時間 $-$ 즉 시와 시 사이를 중시했다.
보통 '오시'(午時)라고 한다면 그건 자정 개념이 아니다. 어렴풋이 11시와 13시 사이를 말하는 것이다.
\vspace{5mm}

그래서 사주가 좋다 나쁘다라고 얘기하는 것도 잘못된 접근이다. 그건 바로 서양적인 인식이기 때문이다.
오히려 동양적 사유로 간다면, 좋은 것 이면에 나쁜 것이 있고, 나쁜 것 이면에 좋은 게 있다.
그렇다면 어떻게 하면 그 이면을 통찰해 흐름을 잘 탈 수 있을까하는 것으로 가야지, 뭔가 단정하고 가는 건 이미 접근방식이 틀린 것이다.
\vspace{5mm}

어떤 미녀를 보고 뿅반했다. 그녀가 좋아졌다.. 라는 건 실제로는 까놓고 말해 교미하고 싶다고 한다.
그런데 운명을 읽을 줄 아는 사람이 당신의 운은 바람직하지 않겠네요라고 얘기한다.
그러나 당사자는 그런 건 없어 하면서 노력을 해서 그 여자를 쟁취한다. 그리고 결혼하고 늘 그렇듯 초반에만 좋다가
사랑과 전쟁을 찍고 이혼해버린다. 그리고 본인은 그 전보다 더 안 좋아진다...
자, 이걸 좋다고 할 것인가 나쁘다고 할 것인가.
그래도 본인이 노력해서 적어도 교미까지 가고 인생 화끈하게 산다는 점에서 보자면 좋다고 할 수도 있고
최종적으로는 상처만 남고 화평하지 못 했으니 나쁘다고 할 수도 있다.
그럼 어느 쪽이든 사실 단정할 수는 없는 것이다.
\vspace{5mm}

어떻게 보면 이 역시 스킬보다 개념이 중요하다는 게 먹히는 영역이다.
직관으로 쭉 바라보는 거라면 가장 중요한 건 잡다한 이론이 아니라, 이러한 현상들을 관(觀)하는 본인의 마음이 맑아야한다는 것이다.
즉 욕심을 버리고 아주 차분하고 온화하게 자신의 비극까지도 응시할 수 있어야 한다.
그러나 보통 관상, 사주, 풍수에 혈안된 사람들은 욕심과 불안에 휘둘리고 있다. 그러니 \textbf{똑바로 볼 수가 없다}.
그래서 똑바로 볼 수가 없으니 사기꾼에 휘둘리기도 하고 지나치게 운명론적인 데 빠져서 스스로 접싯물에 코박고 익사해버린다.
\vspace{5mm}




\section{성과 계급}
\href{https://www.kockoc.com/Apoc/746405}{2016.04.25}

\vspace{5mm}

이건 20대 초에 지겹게 보았던 스테레오 타입의 명제 중 하나가
\vspace{5mm}

\textbf{성(sex)은 계급이다.}
\vspace{5mm}

그런데 납득이 가게 설명하는 사람은 없었다. 그저 동어반복에다가 네가 부르주아 마인드에 젖어서라나(...) 그래서 그냥 무시했던 명제다.
그리고 세월이 흐르고 다양한 사례들을 보고 나서야 귀납적으로 납득이 가기 시작했다.
\vspace{5mm}

과거 봉건영주의 초야권으로 설명하거나
귀족 이상부터는 소위 처녀성을 따지지만 그 이하는 신경쓰지 않는다던 것
그리고 동성애자들의 은밀한 속어에서 드러나는 권력관계.
\vspace{5mm}

혈기방자한 시절에야 성은 신비이고 호기심이며 욕망부터 부추기니 그런 게 보이지 않는다.
하지만 이제 이런 게 너무 과잉이 되어버려서 초딩들도 알 걸 다 아는 세상에야 비로소 보이지 않는 게 보인다.
그건 바로 \textbf{권력.}
\vspace{5mm}

성과 관계된 온갖 변태적인 것들이 사실은 "권력"적인 것들의 요소를 강화시킨 것에 불과하다.
권력이란 말이 납득이 안 가는 이유는 두가지다.
하나는 권력은 적어도 20대 후반 정도는 되어야 피부로 느끼는 것이고
둘째는 공교육에서 사람들이 평등하다고 주입당했기 때문에 실제로는 불평등하다는 것을 가슴으로 받아들이기 어렵기 때문이다.
\vspace{5mm}

포르노, 즉 야동을 보는 사람들이 정말 성욕 때문에만 그걸 본다고 하는 건 오산인 것 같다.
보는 목적이야 제각각이지만 실제로 야동에서 강조되는 것은 바로 '권력'이다.
그 권력이 명쾌한 논리가 아니라, 신화화된 기호로 표상되어 있다. 이 말이 너무 현학적이면 "원초적 이미지"로 그려져있다 보면 된다.
그런데 이런 야동들을 즐기는 사람들은 권력자들이 아니다.
이 대목부터가 중요하다. 야동을 보는 사람들은 이 사회의 '호구'들이다.
그래서 야동을 보면서 몰입(...)하면서 자기가 권력자라도 된 것과 같은 대리만족을 느끼는 것이다.
\vspace{5mm}

곰곰히 생각해보자. 포르노에서 평등한 관계가 나오나?
물론 거기 나오는 등장인물(...)들이 공과 수가 바뀌어서 결과적으로 0가 되는 경우는 있어도 \textbf{실제로 대등한 경우는 없다}.
사람들이 포르노를 보는 심리는 이 민주주의 사회에서 '한물 간 고대, 중세' 시대를 다룬 사극을 보는 것과 같다.
사극의 낭만이라고 해보았자 반민주적인 봉건주의에다가 무식한 전쟁으로 인적자원이나 낭비하는 광경,
혹은 귀족이나 왕족이라는 코스프레를 뒤집어 쓰고 평민들을 농락하거나 혹은 농락당하는 코스프레쇼일 뿐이다.
그런데 사람들이 여기 열광하는 이유는 간단하다, '불평등하기' 때문이다.
\vspace{5mm}

챗방에서 자주 나오는 얘기가 "나쁜 남자들이 여자들이 많이 꼬인다".
\vspace{5mm}

그건 당연하다.
첫째로 대중매체에서 강조하는 사랑이란 말은 사실 실체가 모호한 일종의 선동에 가까워서이다.
발끈하는 사람들에게 물어보자. 그럼 사랑이 뭐라고 정의할 수 있나? 그리고 사랑한다는 사람들이 왜 깨지고 망하나?
\vspace{5mm}

둘째로 그 나쁜 남자들은 바로 사람들이 소망하는 불평등한 권력관계를 생산하기 때문이다.
결국 눈이 맞거나 심쿵한다는 건 상대와의 관계가 불평등하여 전류가 흐를 수 있는 것이다.
\vspace{5mm}

그럼 그 불평등은 무엇인가? 그게 바로 계급이 아닌가.
이과 수학으로 얘기하면 불평등이 벡터라면 계급은 바로 시점과 종점이다.
문과 수학으로 얘기하면 불평등이 두 점 사이의 거리라면 계급은 각 점들의 좌표인 것이다.
\vspace{5mm}

이렇게 정리하면 성이 계급이다... 라는 말의 의미가 비로소 오게 된다.
\vspace{5mm}

보통 이맘 때 쯤에는 대학교 신입생들이 연애하는 사람들이 많다. 물론 대다수는 깨진다.
군대 때문이라고 하지만 실제로는 금방 질려서인데 이유야 간단하다. "평등"하니까.
평등한 관계에서 권력은 생겨나지 않는다. 권력이 없으니 결국 성적인 것도 무의미해지기 시작하는 것이다.
그런 것 말고 그냥 매력이 있어서.... 라고 하는 순간에도 움찔할 것이다. 매력부터가 이미 타고난 것의 불평등을 전제하지 않나?
\vspace{5mm}

사람들이 까대지만 실제로는 궁금해하는 건 바로 사주팔자, 그 다음이 픽업아티스트술(...)이다.
픽업아티스트에 관한 것도 여러가지 책을 읽고 거기 나온 대사를 연구해본 결과는,
\textbf{'감언이설'로 불평등한 권력관계를 만들어내는 것,} 이 하나로 정리된다.
(여담이지만 이 분야가 의외로 과학적이다 $-$$-$ 바로 진화론을 철저히 써먹고 있기 때문이다. 진화론과 국어의 만남)
\vspace{5mm}



\section{소수자의 딜레마}
\href{https://www.kockoc.com/Apoc/751570}{2016.04.29}

\vspace{5mm}

소수자들이 배려받을 수 있는 건 어디까지나 그들이 소수자이기 때문.
그러나 그 상태로 모두가 행복했습니다라고 끝날 리가 없다.
배려받는 것도 잠시 뿐이고 그 소수자라는 컴플렉스 때문에 다수의 상식을 깨고 자기들이 주류임을 인정받고싶어한다.
\vspace{5mm}

논란의 여지가 많은 동성애 문제는 어떻게 될 것인가.
과연 다양성의 존중 문제로만 해결될 수 있을 거라면 이미 진작에 해결되지 않았겠나.
동성애를 반대하는 애들은 옳지 않다라고 단정하는 순간부터 뭔가 석연치않은 딜레마가 생긴다.
그 옳지 않다라는 단정도 의사표현을 억압하는 일종의 폭력성을 간접적으로 띠고 있기 때문이다.
\vspace{5mm}

난민 사태 이후 유럽의 무슬림들을 존중과 배려로 바라보는 시선은 사라졌다. 솔직히 말해 싸늘해졌다라고 보면된다.
어떤 존재건 소수자일 때는 그 진면목을 알 수 없다. 그들이 소수자임을 벗어날 때에야 비로소 진면목이라는 게 보이는 것이다.
A가 억압받고 차별받을 때는 A는 정말로 선량한 존재인데 핍박받는 것으로 보인다.
그러나 그 차별이 끝나는 걸 넘어서 A가 권력을 쥐면서 갑질을 하게 되는 순간은 어떻게 될까.
재밌는 건 그 순간에도 A는 자기가 약자(들)이라고 강조할 것이다. 이것만큼 편리한 도구가 없기 때문이다.
\vspace{5mm}

어떤 메시지건 정말로 그 누드 메시지 $-$ 즉 '솔직한 욕망'이라는 걸 정확히 읽어내야 한다.
예컨대 한국여자들이 서양남자들과 만나다니 이건 불결하다... 라는 말은 그것이 옳고 그르고를 떠나서
그 서양남자들을 부러워하는 심리가 담겨 있을 수 있다.
B라는 애가 여자들은 조신해야한다고 한다면 그 B는 새디스트
C라는 애가 여자들이 더 적극적으로 활동해야 한다고 외치면 그 C는 마조히스트.
다소 극단적인 표현일 수도 있지만 실제로 이런 게 맞아 떨어지는 경우가 많다.
\vspace{5mm}

이렇게 보자면 "착한 인디언은 죽은 인디언일 뿐이다"라는 식의 극단적인 메시지가 참이냐라는 반론에 날라올 것이나
유감스럽지만 역사적으로 선량한 존재들도 없고 진정한 의미의 이타주의적인 실천을 하는 경우도 찾기 힘든 것으로 보인다.
\vspace{5mm}

학벌을 폐지하자는 주장은 진정한 이상 사회 구현보다는
학벌로는 해먹기 힘드니 다른 걸로 출세해야 하니 이 학벌 게임을 무효로 하지 않을래... 라는 걸로 해석하는 게 더 타당하다.
x대생이나 x고생은 별 것 아니었습니다, 노력하면 이길 수 있습니다... 라는 메시지가 실은 컴플렉스 덩어리인 게 보이지 않나.
실제로 x대 나오건 x고를 나오건 신경쓰지 않는 사람은 x대나 x고를 형식적으로라도 추켜세워줄 것이다.
\vspace{5mm}

개인 경험을 일반화시키고 싶지는 않다.
그런데 내가 기억하는 그 소수자들의 옹호자들은 지금 아는 한, 오히려 더 한 갑질을 해댈 가능성이 있는 쪽으로 고속후진해버린 것으로 안다.
지금 생각해보면 이들이 소수자들을 옹호하고 변명해주는 것은 정말로 그 취지에 공감해서가 아니라
자기보다 약하거나 밑에 있는 그 소수자들이나 약자들을 보호해주면서 반사적으로 '권력자'가 되는 효과를 나오기 때문.
그리고 시간 지나보니 주류 쪽에서 더 큰 권력을 누릴 수 있다면 고속후진 정도야 아무 것도 아닌 것이다.
\vspace{5mm}






\section{양적축적은 질적변환을 초래한다.}
\href{https://www.kockoc.com/Apoc/751576}{2016.04.29}

\vspace{5mm}

모든 것을 설명할 수 있는 이론은 사실 아무 것도 설명하지 못 한다라는 말 다음으로
양적인 축적은 질적인 변환을 초래한다... 라는 말은 매우 즐겨쓰는 말이고 실제로도 그렇다.
\vspace{5mm}

운이라는 건 개인적으로 수년동안 그 관련한 점성술이나 팔자 책을 읽어보았는데 지금 정리해보면 한가지 문제가 있다.
\vspace{5mm}

서양적 사고에서 가장 핵심이 되는 건 \textbf{"모순"}의 발견이다.
이 모순이 발견되어야 비로소 기존의 것들이 전복되면서 새롭게 발전하게 된다.
\vspace{5mm}

반면 동양철학 $-$ 정확히 말하면 음양오행에 따른 어떤 기(氣)의 설명은 상생 상극의 순환을 따라 조화로워보이지만
여기 어디도 \textbf{'모순'}이라는 게 없다.
\vspace{5mm}

물론 성급한 절충론자는 그럼 모순과 순환을 둘 다 챙기면 되겠군요... 라고 할 것이다. 물론 그것이 가장 얄밉지만 현명한 태도다.
그러나 굳이 경중을 두자면 순환은 모순에 비하면 쨉도 되지 않는다.
\vspace{5mm}

그럼 이걸 우리의 가치관으로 가져오자.
\vspace{5mm}

운이나 환경 타령을 하는 사람들의 문제는 자기도 모르는 사이에 저런 동양철학적 사고를 하고 있단 것이다.
좋은 운이 오면 나쁜 운이 오겠지, 나쁜 운이 지나면 좋은 운이 오겠지라는 건 그럴 듯 하지만 사실 그렇게 순환한다는 100$\%$ 증거는 없기도 하지만
이것 자체는 결국 그런 운이라는 패러다임을 무조건 믿고 보면서 그 틀 안에 자신을 가둬버리면서 더 이상 생각하지 않게 된다는 치명적 약점이 있다.
또한 수저타령하는 환경론도 마찬가지다. 이건 사실 역사로도 반례가 많다.
한국만 하더라도 20세기 100년동안 정말 적지 않게 변했지만 금수저를 물고 태어나서 다 잘 된 것은 아니다.
\vspace{5mm}

운이나 환경을 극복하는 방법은 간단하다. '모순'에 이르는 것이다.
모순에 이르게 되면 한 세계의 붕괴를 맛보게 된다.
상술하자면 우리가 보는 세상의 틀이 붕괴된다는 것이다.
\vspace{5mm}

노오력을 하는 이유는 별 게 아니다. 성과를 맛보기 위해서? 노오력이라도 안 하면 답이 없으니까.
그것들은 부분점수를 받을 수 있는 답일지도 몰라도 근본적인 답은 아니다.
노오력을 하는 이유는 노오력하다가 그 양적축적이 질적변환을 초래하는 순간에 비로소 우리는 \textbf{기성 세계관의 모순에 도달할 수 있기 때문}이다.
어떤 노예가 죽어라 일하면 좋은 세상이 온다 믿고 죽어라 일했다. 그런데 죽어라 일해도 좋은 세상은 오지 않았다.
그럼 그가 헛되게 일한 것인가? 적어도 그는 죽어라 일해보았자 좋은 세상이 오지 않는다는 걸 알았다. 그리고 그런 가치관의 모순에 부딪치게 된다.
노오력해보았자 소용없다라는 것을 아는 순간 그가 어떤 허위와 허상에 사로잡혀있는가를 전신으로 느끼게 되면서 새가 알을 깨고 나오게 된다.
\vspace{5mm}

여기서 성미가 급한 사람은 노오력 뿐만이 아니라 그냥 삶 전체가 특정한 모순에 도달해나가는 과정이라고 현학적 이야기를 할 것이다.
그런데 사실 이게 맞는 이야기다. 관념적인 삶과 죽음이나 생과 사는 그저 '닫힌 세계'이다.
그러나 본인이 실존적으로 경험하는 삶은 열린 세계에 좀 더 가깝다.
살아가는 과정에서 그가 믿고 있던 세계가 붕괴되는 것을 운좋게 경험할 수 있다.
알다시피 중세시대 사람들은 대부분 자기가 믿고 있던 종교에서 거의 벗어나지 못 했다.
바꿔말해 우리들도 이 자본주의적인 세계관이 전부 진리인 줄 알고 거기서 벗어나지 못 한다.
자본주의를 반대한다는 사람들의 메시지도 거의 쓸모가 없는 건 그들이 정말 치열하게 살면서 모순에 부딪친 것이 아니기 때문이다.
(자본주의가 무너지고 공산주의가 온다는 이야기는 오히려 동양철학적인 순환론에 가까운 이야기가 아닌가)
\vspace{5mm}

거창한 이야기 그만두고 수험으로 돌아오면 간단하다.
공부하는 것이 실력을 키우기 위해서라는 건 맞는데 더 정확히 말하면 죽어라 공부하는 건 기존의 나로부터 벗어나기 위해서이다.
공부가 힘든 건 당연하다. 기존의 내가 모순에 부딪쳐나가는 과정, 즉 나를 부정하는 과정이다.
힘든 공부를 하려면 기존의 나로는 어렵다. 그래서 새로운 차원으로 이행해나가야하는데 이것이 보통 어려운 게 아니다.
만약 본인이 양적축적 $\rightarrow$ 모순의 발견 $\rightarrow$ 질적 변환이라는 싸이클을 빨리 탄다면 이만한 행운도 없을 것이다.
\vspace{5mm}






\section{지식}
\href{https://www.kockoc.com/Apoc/751991}{2016.04.29}

\vspace{5mm}

일반적으로 지식이 정보보다 우월하다고 알려져 있다.
지식은 정보를 가공해서 체계화한 것이므로 정보에 없는 가치를 담고 있다가...그 골자인데
그럴 듯 하지만 실제로는 틀린 소리다.
\vspace{5mm}

지식의 문제는 그 자체로는 \textbf{모순이 없다는 것}이다.
아니 모순이 없어야지 뭔 소리냐라고 하는 사람들이 그냥 수험생이라고 생각하면 넘어갈 수 있는 문제다.
그러나 어떤 분야건 활동해본 사람은 알 것이다. 어느 분야건 이론과 실무는 일치하지 않는다. 모순의 머리카락 정도는 보이고 있단 것이다.
그럼 그 모순은 어디서 배태되는가? 세상이 모순의 집합이어서 그럴지도 모르지만, 근본적으로는 우리의 앎이 \textbf{불완전}하기 때문이다.
이론을 외치는 사람들이 아무리 그럴 듯 하게 말해도 실무에서 발리는 이유는 모순을 다루지 못 하기 때문이다.
실무상 문제는 대부분 그런 모순과 관련된 것이다.
\vspace{5mm}

예를 들어 경제분야로 가보자. 분명 경제학 이론상으로는 재정거래(아비트리지)는 성립할 수 없고
공짜 점심이란 존재할 수 없다. 시장은 균형으로 돌아가기 때문이다.
그러나 실무적으로는 아비트리지는 지금도 실시간으로 벌어지고 있으며 시장의 모순을 파악한 투기꾼들은 엄청 벌어들이고 있다.
거기다가 시장은 매일 붕괴되지 않을까 할 정도로 불안하다.
수험분야로 가볼까. 분명 교과서만 열심히 읽고 기본적인 걸 충실히 하면 잘 나온다는 게 원리인데 실제 성적은 그렇지 않다.
그런데 또 웃긴 건 그렇게 성공한 친구들은 교과서가 중요하다고 말을 하니 어느 걸 따라야하는지 알 수가 없다.
\vspace{5mm}

정보는 지식과 달리 아직까지 모순의 '원석'을 품고 있다.
그렇기 때문에 지식과 정보는 같이 가져가야 한다. 실시간 정보를 확인함으로써 기존의 지식을 뒤짚어 엎을 수 있는 모순을 파악하기 위해서이다.
또한 지식은 반드시 실천과 결부되어야아만 비로소 지혜로 탈바꿈한다(지혜=지식+실천)
상술하자면 본인이 실무 과정에서 그 지식을 어떻게 가감하면서 조심스럽게 다뤄야하는 과정에서 지혜가 키워지는 것이다.
\vspace{5mm}

모순을 보아야하는 이유는 무엇일까. 그 모순이야말로 변화의 씨앗이기 때문이다.
미래를 보는 건 세 가지라고 정리할 수 있다.
\vspace{5mm}

첫째는 예언
미래에 어떤 일이 벌어지겠습니까... 라는 것을 지식으로 처리하려고 하는 것이 바로 예언에 의지하는 걸로 나타난다.
왜냐면 예언이야말로 미래에 어떤 일이 전개된다하는 '모순없는' 진술이기 때문이다.
그런데 예언을 믿을 수 있느냐가 관건이고 무엇보다 이런 경우 당사자는 의존적인 병신이 되어버리고 만다.
\vspace{5mm}

둘째는 예측
과거의 자료나 패턴에 근거해서 앞으로 어떤 일이 벌어질 것인지를 시나리오 및 확률로 계산하여 선택한다.
적어도 예언보다는 적극적이라는 점에서는 바람직하다, 그러나 이 역시 치명적인 오류가 있다.
과거 일이 반드시 반복된다는 보장은 없으며, 정말로 중요한 사건은 "늘 새로운" 것이다.
\vspace{5mm}

셋째는 모순의 응시
미래의 변화는 현재 시스템이 품고 있는 그런 모순에서 변증법적으로 비롯되는 것임을 알고 그 모순을 먼저 파악하는 과정이다.
이치적으로 말이 되지 않지만 현실에서 너무 당연하게 먹히고 있는 것이나, 수요와 공급이 불균형을 향해 발산해가는 것.
혹은 호황일색에다가 100$\%$ 긍정과 찬양으로써 수상한 것들부터 보는 과정이다.
다만 이것은 아직까지도 암묵지적인 것에 속한다.
\vspace{5mm}

이 세가지를 모두 버릴 필요는 없다. 그러나 예언과 예측의 문제는 당사자가 특정한 시나리오에 '사로잡혀'버린다는 것이다.
예언은 그렇다 치고 예측을 보자. 저출산 고령화 때문에 한국사회가 암울하다는 예측을 들은 사람은 정말로 그렇게 되는 양 착각해버린다.
앞으로 로봇기술이 발달해 노동력을 대체해줄 수도 있고, 오히려 인구가 줄어서 1인당 자본이나 일자리가 늘어나 안정이 될 수도 있다.
애당초 저출산이 되는 것 자체가 원래는 정상이었고 과거와 같은 다출산이 비정상이지 않았나하는 식으로 통찰하다보면 정반대가 될 수도 있는 것이다.
\vspace{5mm}






\section{제2의 IMF}
\href{https://www.kockoc.com/Apoc/752253}{2016.04.29}

\vspace{5mm}

1997년 당시인가 S모씨께서 경제위기가 온다는 식의 글을 썼다가 설렁탕 드시러 가셨다는 훈훈한(?) 이야기가 있습니다만.
독극물일수록 달콤한 향기를 내고 위험한 곳일수록 절경을 뽐내며 사기꾼일수록 좋은 사람인 척 하지요.
지금 세상 돌아가는 게 얼핏 보면 전세값이 폭등하고 물가가 다 오르는 것 같아 호황인 것 같지만 실제 민심은 선거 결과에서 드러났다시피입니다.
어떻게 보면 가장 위험한 것이죠. 분명 뭔가 이상하게 굴러간다라고 보이긴 하는데 미심쩍은 부분이 보이니까요.
\vspace{5mm}

원론적인 이야기입니다만 집안에 빚이 있으면 어떻게든 갚는 쪽으로 가는 게 낫고
$-$ 극단적으로 말해서 수익성이 없는 대학교 학과에 진학할 바에는 그 돈을 킵하는 게 낫다는 이야기 $-$
본인들도 철저히 허리때 졸라매고 아껴쓰는 방법 외에는 답이 없습니다.
$-$ 치킨이나 피자도 끊고 편의점 음식도 가능하면 금지하고 지출 내열 철저히 적고 통제하시갈.
\vspace{5mm}

1997이나 서브프라임 모기지 사태는 금융적인 문제였지만
이번이 위험하다면 그건 근본적으로 경쟁력이 상실되어버린 상황이어서입니다.
공부 잘 하는 학생이 컨디션이 안 좋다거나 하필 시험장 뒷자리 녀석이 조폭이어서 망했다하는 게 전자라면,
정말로 실력이 없고 다른 애들에 비해서 실적이 안 나와서.... 라고 하는 게 후자이겠죠.
\vspace{5mm}

경쟁력 상실도 상실이지만 이걸 정부, 시장, 시민들이 해결할 수 있느냐라고 물어보면 대답은 글쎄요.
결국 \textbf{누군가 희생해줘야} 합니다. 그런데 희생해주는 사람은 늘 약자였습니다.
\vspace{5mm}

역설적으로 생각해보면 모두가 다 정체되어있으니. 즉 다들 질주하다가 멈춘 상황이니 루저라고 생각하는 사람들로선 부활의 기회가 되겠죠.
지금 잘 나가는 사람들이 레버리지 효과를 누렸던 경우 양날의 검에 베여 오히려 몰락할 수도 있습니다.
하지만 아무 것도 없는 사람이라면 잃을 것도 없으니 고효율에 절약형으로 자기 분야를 준비해나가면 희망이 있겠죠.
\vspace{5mm}

+ 검색하다 읽은 흥미로운 글
\vspace{5mm}

\href{https://vivitelaeti.wordpress.com/2016/01/24/%EC%9D%B4%EC%8A%88$-$%EB%B8%8C%EB%A6%AC%ED%95%91$-$17$-$%EC%A0%9C2%EC%9D%98$-$imf%EB%A5%BC$-$%EC%A4%80%EB%B9%84%ED%95%98%EB%A9%B0$-$%ED%98%84%EA%B8%88$-$%EB%8A%98%EB%A6%AC%EB%8A%94$-$%EC%82%AC%EB%9E%8C%EB%93%A4/}{링크}
\vspace{5mm}

++ 흔히 경제위기 와서 집값, 건물값 떨어지면 주워담는다... 하는데 그건 현금으로 10억 이상이 있을 때나 할 수 있는 개드립입니다.
떨어지는 칼날을 맨손으로 잡는 짓인데 그게 정말 금덩어리인지 아니면 참치캔뚜껑급 칼날인지 모른다면 누구나 그걸 할 수 있는 게 아니죠.
1997 IMF는 그래도 일시적인 걸로 끝난 셈이고 서브프라임은 그나마 잘 대처했기 때문에 끝났지만 이번 건 그렇게 끝날지 장담할 수 없습니다.
의치한은 잘 나간다, 그래도 건물이 최고다... 이거 조금만 생각해도 반론이 나옵니다.
의치한도 환자가 와서 돈을 써주지 않으면 답이 없고, 건물도 들어오는 임차인이 없으면 부질없습니다.
다시 말해서 국민들 다수가 가난해지는데 나 홀로 돈번다 그런 일은 없다는 것이죠. 돈의 가치는 그 경제활동의 흐름 속에서 창출되니까요.
\vspace{5mm}





\section{목숨값 : 마이크로모트}
\href{https://www.kockoc.com/Apoc/763162}{2016.05.06}

\vspace{5mm}

스탠퍼드 대학교의 로널드 하워드 교수는 '마이크로모트'라는 개념을 고안했다.
마이크로모트는 우리가 위험한 행동을 할 때 감수해야하는 비용의 단위다.
마이크로모트는 사망가능성의 100만 분의 일이다.
우리는 1천만 달러를 받고 인생 전체를 팔진 않겠지만, 10달러를 받고 그 100만분의 일을 파는 건 허락할지 모른다.
그렇다면 우리의 마이크로모트는 한 단위에 10달러인 셈이다.
\vspace{5mm}

우리가 자동차를 할 때 추가 비용을 지불하고 더 안전한 차나 비행기를 탈 것인지는 현실적인 결정이다.
예컨대 미국에서 경차는 해마다 100만대당 109명의 사람들이 사망하고, 중형차의 경우 100만대당 53만명이 사망한다.
우리의 마이크로모트를 단위당 10달러라고 가정하면
경차로 1마일을 가는 비용은 0.0109 마이크로모트인 반면
중형차로  1마일을 가는 비용은 0.0053 마이크로모트이다.
우리가 10만 마일을 자동차로 달릴 때 경차의 안전비용은 10900달러, 중형차의 안전비용은 5300달러이다.
\vspace{5mm}

출처 : 알라딘 중고시장에서 님들이 찾으삼
\vspace{5mm}

신뢰성 100$\%$가 아님에도 우리가 현대문명의 이기를 '위험'하지만 누리는 건 과학이 아니라 경제적인 걸로 설명한다.
교통사고가 나서 사람이 많이 죽으니까 이것이 인간의 존엄성을 저해하니 자동차를 없애자라고 하면 절대 과학적으로는 반박할 수 없다.
왜냐면 정말 그걸로 \textbf{사람이 안 죽는다라고 보장할 수도 없기 때문}이다.
이 글을 읽는 사람 중에서도 설마 하겠지만 사망 원인이 자동차로 기록될 사람도 있을 것이다.
그럼에도 불구하고 자동차를 없앨 수 없는 건, 이 자동차 때문에 사실 삶이 쾌적하거니와 더 많은 목숨을 구할 수 있기 때문이다.
사회의 교통과 수송 기능을 포기하면 그로써 사라지는 목숨이 많아진다. 당장 구급차도 그렇지만 이로써 사회적 기능이 정체되는 것만 보아도 그렇다.
그리고 이걸 그나마 제대로 설명해주는 개념이 위에서 상술한 마이크로모트이다.
우리는 목숨 전체를 내놓진 않는다. 단지 그 일부를 떼어주면서 살아가고 있는 것이다.
\vspace{5mm}

사실 이건 수험이든 노동이든 뭐든 사회 전분야에 적용된다.
우리는 살아가면서 목숨의 일부분을 조금씩 매매하고 있다.
부자들은 일부를 팔고 나중에 더 많이 수거하는 반면,
서민들은 사람들은 많이 팔고 일부만 수거한다고 해도 지나친 이야기가 아니다.
하나만 예를 들면 왜 대기업에서 월화수목금금금하던 사람들이 공무원으로 옮겨타는지도 이걸로 설명된다(목숨값을 버는 거니까)
\vspace{5mm}

상품과 서비스는 과학적으로 엄격히 따지면 모두 위험하다. 그럼에도 불구하고 그 생산과 소비를 하는 건 '싸기' 때문이다.
엄밀히 따지면 건설도 양자역학(...)에다가 상대성 원리까지 적용해야할지도 모른다.
그럼에도 불구하고 그렇게 하지 않는 건 경제적으로 타당하지 않기 때문이다. 만약 그걸로 건설을 중단한다면 손실이 더욱 커진다.
그리고 이 경제원리를 따라간다는 건 '어른'이라면 누구나 알고 있다.
그러나 이걸 사회적으로 떠들 수는 없다. "그 선택이 몇몇의 목숨을 위협할지도 모르지 경제적이니까요"라고 말할 수 있겠는가.
그걸 돌려서 말한 게 '전문가들이 검증한 과학'이라는 수사다.
\vspace{5mm}

예컨대 후쿠시마 방사능이 한반도에 영향을 안 미친다, 이건 편서풍 덕분입니다... 라는 건 구라로 밝혀졌지만
다들 이건 침묵한다. 왜냐면 그걸 떠들어서 한일관계 경색이 되어보았자 이득볼 게 없기 때문이다.
중국산 미세먼지가 한반도를 급습해도 반도민들은 침묵해야 한다. 우리가 중국에서 경제적 이득을 보는 것만 천문학적이다.
하지만 '경제' 때문에 이렇게 여론을 조작한다고조차 말까지 할 수 없다.
그래서 미세먼지 일부가 한반도에서 나온 것이다라는 과학적 수사로 넘어가는 것이다.
\vspace{5mm}

마이크로모트의 지불자는 정부가 아니라 '본인'이다.
어떤 사건이 터져도 그건 전문가에게 맡기면 된다, 검증을 확실히 하면 된다라는 건 좀 문제가 있는 발언일 수 있다.
첫째, 그건 정부나 전문가 카르텔이 만든 일방적인 메시지에 불과하다 그런데 그걸 무비판적으로 받아들이고 있지 않나
둘째, 발언자는 자신의 마이크로모트, 즉 목숨값을 전혀 생각하지 못 하고 있다.
\vspace{5mm}

아니 그 전에 가장 중요한 이 목숨값이라는 개념을 중고 교육에서 가르치진 않는다.
중고 교육에서는 과학이 얼마나 위대한지 그리고 전문가들의 조언을 들어야한다고만 얘기할 것이다.
생각해보면 위험천만한 얘기다. 그건 결국 권력자들의 말을 듣는 충실한 '종속자'들만 양산하자는 것이 아닌가.
\vspace{5mm}

이과짱임이라고 해보았자... 그 이과짱들도 교수든 연구자든 '대기업'과 '정부'가 주는 돈에 기대야 한다.
그리고 대기업과 정부는 개개인의 목숨값을 아주 저렴하게 매긴다.
우리나라의 교수들이나 연구자들이 과연 소신껏 연구할 수 있는가에 대해선 물어볼 필요도 없을 것이다.
그리고 사건이 터져보았자 피해자들은 제대로 된 보상을 받을 수 없다. 우선 죽은 사람에게 돈이 무슨 소용이 있겠나.
그리고 대기업들은 이런 것도 빠져나갈 구멍이 많다.
\vspace{5mm}

사실 이 논쟁을 하면 우리 모두가 병신이라는 것을 깨닫게 되는데
우리는 실제로는 정말 우리가 요구해야 할 권리를 못 챙기고 있다라는 결론이라서리.
\vspace{5mm}

그리고 선의의 시도조차 실은 돈을 좆는다라는 것을 보이는 게
\vspace{5mm}

가령 설탕세를 부과하겠다 하면서 비만과의 전쟁을 벌인다고 하는 건 '건강보험 재정' 문제와 관련이 있는 것이고
복지를 강화하는 것은 $-$ 그게 실효성이 떨어져도 $-$
저출산 문제를 해결해서 인구수를 늘려야 역시 장기적인 돈문제가 해결되기 때문이다.
그러나 그것 역시 100$\%$ 추진되지는 않는다.
언론조차도 대기업 스폰서가 없으면 영위될 수 없고, 정부 입장에서도 세금을 많이 내는 대기업은 조져도 적당히만 조져야 한다.
\vspace{5mm}

어떻게 보자면 합리성이라는 환상으로 사회가 안정적으로 돌아갈 수 있을지 모르나
개인의 입장에서는 목숨값을 착취당하고 있다는 것만큼은 전혀 무시될 수 있는 것이 아니라는 것이다.
\vspace{5mm}

옥시 사태야 그렇다 치더라도
그럼 전자파나 첨가물의 해악은 어떤가? 간혹 TV 방송에서만 공포마케팅급으로 조장되지만 결국 없던 일처럼 지나가지 않나.
물질은 아니지만 딸 키우는 사람들 입장에서는 TV에서 성적 대상으로 여중생, 여고생들이 소비되는 것만큼 위험한 일도 없지만 과연?
\vspace{5mm}

이런 걸 곰곰히 생각만 해보아도 어디든 부조리는 가득차 있다.
다만 그걸 하나하나 해결하기 힘들고, 우리 개인은 개인의 문제만 해결하기도 벅차지만
문제는 그 개인에게 매우 중요한 문제조차도 정부나 전문가의 말을 지나치게 신뢰한 나머지 자기 권리를 포기할 경우도 생긴다는 것이다.
이에 한몫하는 것이 결국 "담론" 형성인데, 이 담론 형성을 담당하는 자들이 바로 지식인들이다.
하지만 이 지식인들조차도 중요한 건 침묵한다. 그런 건 자기들의 일자리와 소득에 악영향을 미칠 수 있는 것들이다.
\vspace{5mm}

원칙적으로 말해 일반인들도 검증해야 한다. 만약 검증 주체에서 일반인이 빠진다는 건 '주권포기행위'나 맞먹는 것이다.
직접 할 수 없다하더라도 대리인을 선임하거나 위임할 수 있다. 그러나 한국 사회에서 그런 시도조차 몇이나 있었느냐는 의심스럽다.
개인적으로는 지나치다고 생각하지만 세월호 유족들의 다소 선을 넘어보이는 것과 같은 활동이 이 점에서는 긍정적인 면도 없지는 않다.
이 사람들이 막 나가지 않았다면 정말로 조용히 묻히고 넘어갔을 수도 있다.
다시 말해서 거의 갑질 비슷하게 나갔으니까 그나마 그 정도로 보상받고 이슈가 된 것이지 그렇지 않았다면 어땠을까하면
이들의 행동이 부정적인 것만은 아닌 것이다. 문제는 그럴 자격이 있는 사람들이 그렇게 큰 목소리를 내지도 못 했단 것이지만.
\vspace{5mm}






\section{역산적 사고법}
\href{https://www.kockoc.com/Apoc/764294}{2016.05.07}

\vspace{5mm}

서른살이었을 때 무엇을 후회하고 있을까,
마흔살이었을 때 무엇을 후회하고 있을까.
죽기 직전에 어떤 것을 후회할까.
\vspace{5mm}

일단 이런 마인드로 접근하는 게 가장 정확한 것 같습니다.
즉, 이상적인 미래상을 정한 다음에 그 미래상을 미래의 현실 시스템과 결부지어 교정한 다음
그걸 완성하기 위해서 지금 어떤 과정을 밟아야하나 보면 되는 건데.
\vspace{5mm}

한데 상당수가 "과거지향적"인데다가 "누가 잘못했나"하는 식의 데카르트 사고의 노예가 된 경향이 있습니다.
단지 과거에 실패하지 않았다면 지금 훨씬 더 잘 나갔을 텐데... 라고 생각하죠.
그래서 정작 자기가 어떤 미래로 나아가야할지 그런 게 없어서 결국 모든 것에서 탑이 된다는 망상에 빠져있습니다.
다수 수험생들도 그렇지만 콕콕러들도 이런 경향이 없는 게 아니죠.
\vspace{5mm}

예컨대 대학과 자격증에 관한 생각도 뭔가 잘못되어있습니다.
대학도 '창업', '취업'을 위한 발판일 뿐입니다(미국에서는 창업을 취업보다 더 쳐주는데 우리나라도 이제 그렇게 되어가는 분위기죠)
만약 본인이 창업, 취업에 자신이 있고 그만한 역량이 있다면 대학과 자격증은 그냥 악세사리일 뿐입니다.
먼저 말해서 자기가 마흔살에 어느 직장에 어떤 부서 직책을 맡고 있을 것이다라고 가정하고 역산해보면 답이 나옵니다.
그런데 이 생각을 못 하니까 재수, 삼수까지 하면서 명문대에 꼭 가기만 하면 된다라고 보는데 이게 오히려 망가지는 길일수도 있습니다.
당장은 모르지만 10년 뒤에도 과연 학벌만 가지고 버틸 수 있다고 볼 수는 없기 때문입니다.
\vspace{5mm}

게다가 자기가 서른, 마흔살이 되었을 때 한국사회가 어떻게 변할 것인가도 정확히 보아야죠.
10년 전만 하더라도 신의 직장(공무원 포함)이나 대기업 때려치우고 고시친다는 사람들 많았습니다. 변호사가 훨씬 더 많이 번다고 얘기하면서요.
그런데 지금은 어떻습니까. 다들 신의 직장이나 대기업에 못 들어가서 안달이죠.
절대 변호사가 망할 리 없어라고 했습니다만 미래 예측서에서는 의사나 변호사를 오히려 몰락 1순위로 보는 경우도 있습니다.
설마 의사가 라고 하는데... 절대 그럴 리 없어라는 게 부뚜막에 먼저 올라간 고양이처럼 바뀌니 이런 데 주의할 필요가 있습니다.
\vspace{5mm}

역산적 사고는 미래를 개척하는 유일무이한 사고법입니다.
수학도 이 역산적 사고로 굴러가죠. "해가 있다면" 이라고 가정하면서 "x=?"라고 잡는 것이 그것이니까요.
이러한 x를 쓰지 않으면 우리는 때려맞추기를 해야합니다.
마찬가지로 역산적 사고를 쓰지 않으면 그냥 자기의 운명에 수동적으로 순응해버리고 맙니다.
\vspace{5mm}

정말로 현명한 친구는 국영수를 잘 하는 친구가 아니라,
현실을 꽤 정확히 예측하면서 자기가 원하는 미래상을 정확히 잡고 거기에 매진하는 친구가 아닐까 합니다.
고딩에게 현실예측을 바라다니... 라는 것이 안 먹히는 이유는 어차피 이 분야는 연령성별불문하고 다들 모릅니다.
어떻게 보면 참 얄궂은 블루오션입니다. 그냥 한국 사람들은 자기가 성적대는 대로 대학과 전공과 직업 정하고 살아가는 사람들이 다수입니다.
자기가 열심히 공부했으니까 알아서 좋은 자리가 들어오겠지라는 걸 너무 당연시 여기고 있는 것이지요.
그렇기 때문에 이 분야는 중학생이 어른보다 훨씬 더 나을 수도 있습니다(어른들은 돈에 감염되어버렸거든요)
\vspace{5mm}




\section{붕괴}
\href{https://www.kockoc.com/Apoc/765536}{2016.05.08}

\vspace{5mm}

\begin{itemize}
    
    \item[$\#$] 일본의 빈집
    \href{http://news.naver.com/main/read.nhn?mode=LSD&mid=sec&sid1=104&oid=023&aid=0003171704}{링크}
    \vspace{5mm}
    
    \item[$\#$] 테슬라 예약주문 40만대
    \href{http://news.naver.com/main/read.nhn?mode=LSD&mid=shm&sid1=103&oid=020&aid=0002970286}{링크}
    \vspace{5mm}
    
    \item[$\#$] 알파고 시대 우리아이 알파백수로
    \href{http://news.naver.com/main/read.nhn?mode=LSD&mid=shm&sid1=105&oid=023&aid=0003170802}{링크}
    \vspace{5mm}
    
    \item[$\#$] 우버의 확장
    \href{http://news.naver.com/main/read.nhn?mode=LSD&mid=shm&sid1=104&oid=001&aid=0008384770}{링크}
\end{itemize}
\vspace{5mm}

매년 초마다 발간되는 박영숙 교수의 유엔미래보고서를 완전히 무시하기만은 힘들 것 같다.
이 책도 짜깁기라고는 생각하지만 짜깁기도 정말 꼼꼼히 된 짜깁기다. 풍부한 자료에서 쓸만한 것만 추려낸다.
터무니없다고 생각되던 것들이 저렇게 기사화된다.
\vspace{5mm}

그 전에 무엇보다 우리가 알던 자본주의 틀이 무너지고 있다.
공산주의, 사회주의가 색다른 형태로 실현되고 있다는 느낌이 온다.
그건 기업들이 '이윤추구' 이전에 '사회적 공헌'을 내세움으로써 실현된다.
이제는 사회적 공헌을 하지 못 하는 기업은 살아나지 못 한다.
비즈니스 모델은 우리가 알던 것들 90$\%$가 거짓말로 판명났다고 보면 된다.
\vspace{5mm}

우선 기사들을 보자. 일본의 빈집이 시사하는 건 우리가 알던 부동산 시장의 붕괴다.
이 글을 읽은 사람들은 혹시 부모님이 부동산에 환장하셨다면 바로 말리길 바란다. 철저히 현금을 확충하고 빚을 줄여야 하는 시대다.
지금이 제2의 IMF라고 말을 안 해서 그렇지 실제로는 그 상태로 접하고 있다.
1997년 외환위기가 미분불가능한 뾰족점이었다면 지금은 미분가능한(...) 접점이라는 차이 뿐이다.
이 상태에서 어디 하나라도 무너졌다간 아작이 나버린다.
그것도 그렇거니와 우리나라도 빈집 문제가 부각되고 있는 중이고 무엇보다 일본을 10년 늦게 따라간다.
괜히 전셋값이 높은 게 아니다.
\vspace{5mm}

테슬라나 알파고가 시사하는 건 우리 사회 후진성이다.
현기차 등이 테슬라를 따라잡을 수 있을까. 누가 더 가격이 싸고 품질이 좋고 그런 문제가 아니다.
테슬라 생산자나 소비자 입장에서는 기성 생산/소비는 중세시대 수준으로 보일 수 밖에 없다.
그런데 한국은 기존의 이권 때문에 그런 중세시대적인 것을 지키고 있다.
2018년에 코딩 교육을 하고 문이과통합을 한다고 하지만 이것도 늦은 게 아닌가.
우버의 경우도 우리나라에서는 통과되지 않은 걸로 안다. 물론 카x오 쪽이 그 수요를 담당하고 있다.
하지만 이 역시 부족하다. 과격한 혁신이 아니고서는 절대로 생존할 수 없다.
\vspace{5mm}

그동안 제시되던 패러다임들이 거짓말이 아니라는 것은 확실해졌다.
이럴 때는 \textbf{"갈아타라"}는 게 답이다. 우리가 알던 기성의 교육이나 산업은 화석이 되어갈 건 분명하다.
물론 기성 시스템에서도 안 되는 사람은 갈아타지도 못 할 게 뻔하다.
\vspace{5mm}





\section{인간관계}
\href{https://www.kockoc.com/Apoc/765542}{2016.05.08}

\vspace{5mm}

개인적 경험에서 오래가는 순
\vspace{5mm}

\textbf{(오프없는) 온라인 >>>> 중고등학교 친구 > 그 외}
\vspace{5mm}

믿거나 말거나이지만 과거에 하이텔, 나우누리 시절에 번개(...)라는 걸로 정말 시대를 앞서가는(...) 문화양식을 즐기고 다양한 사람을 만났다.
그런데 그게 오래 가는 게 없는 이유가 왜 그런가 생각해보니
오프에서는 철저히 외모와 재산이라는 걸 보기 때문이다.
\vspace{5mm}

이게 어느 정도 웃기냐면 채팅방에서 영혼을 나누었다는 남녀가 오프를 트고 서로의 외모에 실망하니까 어떻게 핑계를 대고 안 만날까.
이런 궁리를 하던 게 그 당시 PC 통신 시절의 이야기다.
그런데 지금은 그게 덜한 이유는 이미 온라인에서 \textbf{끼리끼리} 놀기 때문일 수도 있다.
\vspace{5mm}

진짜 오래가는 친구는 힘든 시절을 공유한, 다 같이 굶어 뒈지면서도 서로의 오줌을 나눠먹는 그런 사이라는 말도 있지만
여기저기 듣고 관찰하고 경험하고 상담하다보면 현금 다발에 서로 배신하거나 생까는 게 당연하다라는 것이고
따라서 본인이 만약 잘 나갈 때에 다가오는 사람들은 본인이 못 나가면 그대로 떠날 수도 있다고 여겨야하는데 사람 심리가 그렇지 못 하다.
그걸 무의식적으로 알기에 잘 나가던 사람들은 더 외모에 신경쓰고 빚을 내서 더 화려한 소비를 하려는 걸 보여주려고 하다가 자취를 곧 감춘다.
\vspace{5mm}

순수하게 오래 가고 싶으면 온리 온라인으로 하는 게 낫다는 게 경험적인 판단이다.
저기서 중고등 친구조차도 일년에 1$\sim$2번 만나는 수준이니까 오래 가는 것이지 그 이상 가면 현실적인 이해관계상 힘들어진다.
요즘 같은 시대에는 '돈 빌려줘', '좋은 보험이 있어'라고 무리한 부탁을 하는 건 당연하다.
아무리 친구가 잘 나간다고 하더라도 경제적 부담은 주지 말아야 하며 일부러라도 만나는 횟수를 줄여주는 것이 현명한 처사다.
그에 비해 순수 온라인은 서로가 거리를 분명히 확정지어놓았기 때문에
그 이상은 안 넘어선다는 불문율을 지키니까 더치페이처럼 편한 것이다.
그래서 역설적으로 가장 현실적인 문제에 대해서 솔직하게 이야기할 수 있는 점이 있다.
\vspace{5mm}

지금은 연락을 안 하지만 (그런데 언제라도 트면 또 얘기할 수 있을지 모르지. 중고딩 친구니까)
정말 인간관계에서는 너무나도 환상적이어서 양다리가 아니라 문어다리 수준으로 엽색행각을 벌인 친구가 진리를 얘기한 적이 있는데
사람 사이를 오래 유지하는 비결은 \textbf{"거리를 유지하는 것"}이라는 이야기였다.
그리고 내 경험상은 저 말은 진리이다.
물론 거리를 유지해도 오래 가지 않는 관계도 없지 않는데 그건 어차피 '비즈니스' 관계인 것 뿐이다.
얼치기들은 비즈니스 관계를 진정한 관계로 오해한다.
예쁜 여자가 갑자기 생글생글 웃으면서 잘 대해주면 정말 자기를 좋아하나보다라고 착각하는 남자들이 좋은 예다.
비즈니스 관계는 그냥 비즈니스로 쿨하게 끝내야 한다.
\vspace{5mm}

입시를 치르고 대학에 들어간 친구들이 겪는 건 사람 사이의 배신이다. 그건 누구나 겪을 수 밖에 없다.
그런 인간관계가 자산이라도 되는 줄 알고 일부러 잘 꾸미거나 친절한 척 하는 사람들도 있는데 그건 정말 낭비다.
하나의 이미지를 구축해놓으면 그걸 유지하기 위해 많은 비용을 들일 수 밖에 없다.
외제차나 명품백에 거금을 쓸 수 밖에 없는 것도 그런 이유다. 그런 데 돈을 쓰지 않으면 무시당하니까.
차라리 무시당하고 걍 손절해버리면 되지 않냐 하지만 사람 심리가 그렇지 못 하다. 여기서 또 말리는 것이다.
\vspace{5mm}

하나 재밌는 예만 들면 어디든 커뮤니티가 망가지는 길은 간단하다.
모 사이트에서는 자기가 여자인 걸 드러내는 걸 금지시켰는데 그거 경험과 관찰에서 우러나온 썰일 것이다.
보통 잘 나가는 커뮤니티는 일반적으로 마니아 덕후 남자들이 완성해놓는다. 이들이 열정적으로 자료 올리고 행사 진행하면서 사람을 모은다.
그래서 어느 정도 규모가 활성화되면 하나의 작은 사회가 완성되는데....
재밌는 건 이런 데가 있으면 귀신 같이 냄새를 맡고 들어오는 새로운 여성회원이 늘 있다.
온갖 애교에다가 착한 소녀 기믹으로 오빠 어쩌구 하고 들어오는데 문제는 바로 '오프모임'에서 터진다.
그 때까지만 해도 유비 관우 장비의 의형제 정신으로 버텨오던 남자들이 오프에 참여한 여자가 에쁘더라하면 그 때부터 심각해지는 것이다(...)
자기들은 초연한 척 하지만 결국 그런 예쁜 애가 슬그머니 갑질을 시작하고 남자들이 서로 견제하고 싸우고 자기가 잘났다 경쟁한다.
거기에 질린 다른 여성회원들은 점점 활동을 중단하기 시작하고 갈등이 표출되면서 딴 살림차리러 나간 사람들은 독립해서 나간다.
\vspace{5mm}

저렇게 망가진 커뮤니티가 한둘이 아니다.
이런 의문을 제기할 수도 있는데 $-$ 그럼 여성들만이 모인 사이트가 괜찮겠네요 $-$ 라고 하는 것.
검색 사이트에는 모르지만 알만한 사람은 아는 그런 곳들이 없는 건 아니다.
xx 오빠 팬클럽이라거나 드라마 덕후질 같은 경우야 꽤 오래 간다. 그런데 이런 곳은 남자들이 보기엔 '군기'가 잡혀있다(...)
빠심이나 팬심 아니면 버틸 수 없는 곳이다. 그리고 최근에 추가되는 것은 증오심(남혐) 정도.
이걸 빼고 나면 사실 여성사이트들은 별로 재미가 없다. 그래서 그 회원들은 다시 남성들이 모이는 사이트 공략을 시작한다는 뻔한 패턴.
\vspace{5mm}

