
\section{[중2론 001] 원인이냐 결과냐}
\href{https://www.kockoc.com/Apoc/710622}{2016.04.04}

\vspace{5mm}

어설프게 아는 것만큼 위험한 것은 없다(물론 이 글을 쓰는 작자도 거기에 해당하는 것임을 부인할 수 없다)
가령 경제학 개론을 대충 훑으면서 독과점 시장의 메커니즘을 논하지 못 하면서 인과관계와 상관관계나 얘기한다든지
헌법 이야기를 하면서 직접 민주정과 간접 민주정의 차이를 말하지 못 하면서 위임민주주의가 뭔지 괴상하게 얘기한다든지.
\vspace{5mm}

대충 책을 읽은 사람들도 문제지만, 그 지식이 현장에 어떻게 적용되는가 검토하지 못 한 경우가 많다.
이런 사람들은 메시지는 거창하지만 실제 문제해결능력은 기대할 수 없다, 오히려 개인으로서는 마이너스 성장을 하고 있는 경우다.
차라리 그런 메시지를 구사하지 않았으면 현시창인 것을 알고 반성이라도 하겠지만
\vspace{5mm}

현실의 중2병이란 이런 것이 아니겠나.  중2병이 정말 무서운 건 평생 갈 수 있단 것이다.
그 상태로 나이를 먹으면 "자살"이라고 정말 진지하게 외치거나, 오늘 아침 뉴스처럼 염산테러를 저지를 수도 있다.
그리고 이 중2병은 나도 걸려본 적도 있고 지금도 걸려있을지도 모르기 때문에 논하지 않으면 안 되는 주제라고 여기기에 적어본다.
\vspace{5mm}

중2병 증상 1 : "\textbf{원인을 찾아보자고 해놓고 과거에 집착하고 타인을 원망한다."}
\textbf{원인과 결과 중 무엇이 중요?}   다수는 원인이라고 이야기할 것이다. 그리고 과정이 결과보다 중요하다고 바로 기계적 대사가 튀어나올 것이다.      그런데 생각해보라. 우리가 노력하는 이유가 뭔가? 좋은 결과를 내기 위해서이지.   그리고 원인 분석을 하는 이유가 무엇인가? 다음에 좋은 결과를 내기 위해서이지.   아울러 과정이 중요한 것? 그거야 앞으로의 많은 시도에서 역시 좋은 결과를 내기 위해서 아닌가?   조금만 생각해보면 원인과 과정도 역시 결과에 종속된다는 것을 알 수 있는데 다수는 이상하게 \textbf{원인과 과정이 중요하다고 얘기}한다.   본인들은 자기들이 독자적으로 이성적 사고를 하고 독립적으로 행동한다고 착각하지만 그렇지 않은 예가 바로 그것이다.   물론 그런 메시지를 주입한 현자들께서는 겉보기 결과에 집착하지 말고 근본적인 원인을 탐구하라... 는 좋은 말씀을 하고 싶으셨을 것이다.   그러나 현실에서는 이러한 '원인' 분석이란 "누군가에게 책임을 돌리기 위한 것"으로 전락하기 좋다는 것을 도외시 하고 있다.   입시에 실패한 n수생은 그 원인을 찾는다 해도 자기 탓을 하겠나? 결국 가족이나 친구나 담임 탓을 하겠지.    원인 분석이 외부나 타인에 대한 책임전가로 이어진다면 이건 하나마나 못하게 된다.   과정에 대한 강조가 역시 그렇다. 입시라면 일단 붙는 게 우선이지, 무슨 사이비 교주들의 강의나 교재를 완벽히 듣는다 이게 중요한 것은 아니다.   그러나 n수생들을 보면 반드시 아무개의 xxx 를 봐야한다는 정신나간 소리를 하는 경우가 있다. 이게 바로 과정을 강조하는 사고의 결과물이다.   허영심으로 점철된 완벽주의 그 이상 그 이하도 아닌 것이다.   실패하는 사람들은 주로 이렇게 생각한다.   "내가 시험에서 망하는 이유는 무엇일까, 옛날에 잘 했는데 지금은 왜 이럴까. 교재 때문일까? 아무개 강의를 안 들어서? 평가원 탓일까"   이처럼 안 되는 원인을 생각한다고 하지만 실제로는 진짜 원인도 찾지를 못 한다. 결론은 일본을 공격한다 식으로 외부와 타인을 탓하는 걸로 바뀐다.   자, 이런 태도가 문제해결에 도움이 되겠는가   실제로 원인이 파악된다고 치자. 공부하려고 할 때마다 술먹고 와서 밥상 집어던지는 아버지가 문제였다 치자.   그런데 그걸 안다고 해서 무엇이 바뀌겠나? 우선 과거의 문제인 이상 타임머신이 없으면 해결할 수 없다.    지금 아버지 멱살을 잡고 과거를 돌려내 라고 한다고 배상금이라도 받아낼 수 있나?   문제 해결을 하려면 바로 그 때 했어야 한다.    반면 성공하는 사람은 이런 질문들을 한다.   "어떻게 하면 점수를 더 높일까", "어떻게하면 지금 반수하는 상황을 극적인 기회로 바꿀 수 있을까"   "현실적으로 지금 만점을 불가능하지만 국영수 -1 선방하는 결과는 낳으려면 뭘 보충하면 될까"   잘 되든 안 되든 이런 사람들은 현재 상황보다 더 나아지기 위해 '어떻게 해야 하는가?'에 초점을 두고 질문한다.   어차피 안 되는 원인을 파악해도 바꿀 수 없다면, 처음부터 결과지향적인 사고를 하는 게 맞다.   둘 다 시간과 노력이 든다면, 결과 쪽에 초점을 두는 편이 나은 것이다.   그런데 우리 중2병들은 자기가 원하는 결과조차 뭔지도 모른다. 아니, 입시를 계속 치는데 왜 치는지도 모르는 것이다.   미래 언제 어디서 어떤 수준의 삶을 누리고 있을지, 무얼 하고 살고 있을지, 그리고 어떤 사람들을 만나고 있을지 물어보면 답을 못 한다.   그들은 '터무니없는 망상'이라고 반박한다.   하지만 어차피 알아도 별 소용이 없는 원인에 집착해 외부와 타인을 원망할 바에는,   터무니없는 상상이 현실이 되도록 노력하고 방안을 모색해보는 게 더욱 생산적이지 않나?   그럼 왜 어른들은 \textbf{원인과 과정이 중요하다고 얘기}할까.   그거야 당연하지, 그 사람들도 남에게 들은 이야기로 현자 코스프레하는 경우가 대부분이어서이다.   실제 성공한 사람들이 그런 이야기를 하겠나. 물론 실제 성공한 사람들도 거짓말을 한다.   그런데 그들의 이력을 보면 원하는 결과를 얻기 위해서 심지어 부도덕한 시도까지도 한다.    이게 윤리가 없어서일까, 아니면 원하는 결과를 맞추기 위해 심지어 소신까지 포기하는 건지 해석하는 건 자유다.   이게 바람직한 건 아닐 것이다. 다만 눈여겨볼 건 이들은 철저히 결과중심적인 사고를 하는 것이다.   실패하는 사람들은 브로커를 동원해 거액을 주고 좋은 대학에 입학한다라는 것을 애당초 거부한다.   하지만 성공하는 사람들은 그게 뭐 어때서... 라고 먼저 생각하고 고려할 것이다. 미국에서는 실제로 기여입학도 있기 때문이고   한국도 사실상 입학사정관제도나 수시입학제도가 부분적으로 그런 특징이 없다고만은 볼 수는 없어서이다.   다만 성공하는 사람들은 이것도 불가능하지 않다라고 처음에 고려하되, 결국은 안 된다라고 하면 다른 방안으로 대체할 것이다.   거액을 주고 입시 프로들에게 자문을 구하고 효율적인 사교육 코스를 밟아나가는 것으로 대체하는 식으로 나가는 것이다.   그래서 이들은 운신의 폭이 넓다.   생각해보면 부도덕해보이는 것이면 처음부터 무서워라고 내빼는 것보다는   어, 그런 것도 고려할 수 있겠군이라고 검토해보다가 결국은 안 되겠군이라고 X표 치는 게 더욱 능동적인 윤리관이 아니겠는가?   실패했던 과거는 그냥 내가 왜 실패했는가 참조하기 위한 학습기간으로 검토해보아야 한다.   그리고 사실 한번 실패해보았다면 성공하기는 쉬워질 수도 있다. '실패한 것과 반대방향'으로 가면 성공할 확률이 높아지기 때문이다.   그런데도 실패를 반복하는 건, 그 실패한 과정 자체에 집착하기 때문이기도 하지만,   이 사람들은 원인, 과거, 과정 자체에 집착하기 때문에 벌어지는 것이다.   다시 말해 A란 코스로 실패했다고 치자. 정말 합리적인 사람이라면 일단 결과를 위해서 A를 버리고 B라는 과정을 서슴없이 선택할 것이다.   하지만 양민들은 자기가 모자라서 부덕하다고 하면서 \textbf{다시 A로 성공해보자}라고 노오력을 한다.      원인과 과거에 대한 집착은 '자기중심적인 사고'에서 비롯된다.   그 원인과 과거가 이미 자아의 일부, 아니 핵심으로 느껴지기 때문에 그걸 못 버리는 것이다.   이 때에는 "내가 만약 xxx이라면? " 이라고 하면서 성공한 사람의 예를 들어서 행동해보는 게 낫다.   가령 "내가 만약 오바마라면"이라고 해보자. 밑바닥에서 놀다가 미국 대통령까지 올라간 능력자라면 지금 상황에서 어떻게 하겠는가.   그걸 객관적으로 기술해본 다음에 바로 그대로 하는 게 답이다.      이렇게 자기보다 훨씬 강하고 믿음직스러운 모델을 불러내는 것을 '모델소환술'이라고 해보자.   이 모델소환술은 비루하고 엉터리인 자기 자신에게서 탈출할 수 있는 디센터술의 일종으로 매우 유용한 것이다.   유감스러운 일이지만 콕콕에서 상담해주는 사람들도 이런 중이병에 빠진 분들이 없지 않다.   과거에 집착하거나 어떤 과정 자체로 성공할 수 있다는 걸 보여주고 싶어한다.    이런 분들은 마인드를 바꿔 결과중심적인 사고로 바꿔야 한다.






\section{[중2론 002] 먼 산일수록 작아보인다}
\href{https://www.kockoc.com/Apoc/712608}{2016.04.05}

\vspace{5mm}

먼 산일수록 작아보인다.
작아보이니까 만만해보이고, 만만해보이니 \textbf{쉽게 오를 수 있다 착각}한다.
\vspace{5mm}

초보자, 초심자일수록 시험을 너무 만만히 보는 이유가 이거다.
아는 게 없으니 만만해보인다. 그래서 준비를 소홀히 하거나 시간을 방만히 쓴다, 그러니 실패할 수 밖에 없다.
\vspace{5mm}

산을 멀리서 바라보는 것과, 직접 오르는 건 정말로 다른 문제라는 것을 그들은 모른다.
그리고 이건 중2병 환자들을 키워내는 우리 교육과 무관하지 않다.
한국의 교육은 실천과는 전혀 거리가 멀다.
창의성을 강조한다고 말로만 그러지, 실제로는 '결론과 정답'을 정해놓고,
그것 밖에 없다고 강조를 해대는 게 우리나라 교육이다.
그래서 학생들은 메시지는 정말 그럴싸하게 소리친다, 하지만 "직접 해결해보라"고 하면
"자기가 그걸 하면 안 되는 이유"나 거창하게 얘기한다.
그러다 루저가 되는 거지.
\vspace{5mm}

자, 그럼 여기서 사장과 알바의 태도를 나눠보자.
\vspace{5mm}

똑같은 시간 10시간이 있다.
알바는 두가지다.
\vspace{5mm}

하나는  대충 10시간동안 놀고 시급 챙기는 것이다 - 물론 오래 가지 못하고 잘린다.
다른 하나는 박카스에 레드불을 마시면서 10시간동안 졸라 일한다 - 인정받고 그 다음으로 승진한다... 는 옛말이고 더 착취당한다.
\vspace{5mm}

이 둘의 공통점은 뭔가. 대충 시간을 흘려보내거나, 아니면 그 시간동안에 자기 학대를 하려한다.
어른들은 후자를 강조하지, 물론 '알바'를 못 면한다면 나도 후자를 강조할 것이다.
\vspace{5mm}

반면 사장은 어떠나?
\vspace{5mm}

사장은 바로 얘기한다. "10시간 가지고 뭘합니까, 1000시간"은 있어야지
아니 그럼 시간을 어떻게 늘려? 무슨 상대성이론도 아니고
사장은 어이없다는 듯이 말한다. "돈을 주고 시간을 사면 되지 않습니까. 알바 100명 고용하면 되겠네요"
\vspace{5mm}

여기서 고용 경제학적인 건 침묵하고 사장의 태도를 보자.
사장들이 그 시간동안 열심히 일하지 않는 건 아니다.
다만 이들은 그 10시간 가지고는 택도 없다는 걸 안다. 그래서 시간을 늘리려하는 것이다.
\vspace{5mm}

그럼 여기서 똑같은 수험에 대해서 알바와 사장의 태도를 보자.
게으름 피우는 알바는 대충 공부하다가 수험 말아먹을 것이다. 그리고 +1수하면서 롤이나 하고 있겠지.
자기학대적인 알바는 대학도 다니고 반수한다고 할 것이다. 학점, 수험, 건강 중 최소한 하나는 아작나는 건 분명하다.
\vspace{5mm}

하지만 사장은?
자기가 얼마나 많은 시간을 확보할 수 있나부터 계산할 것이다.
대가리가 금강석이더더라도 시간을 확보하면 해결된다는 수험 함수 정도는 알고 있다.
중요한 건 그 함수의 적분값을 구하기 위해서는 최소한 연속이어야한다는 것 정도 역시 알고있다.
\vspace{5mm}

작년 11월에 내가 +1수를 종용하는 낭만주의자로 오해받은 적이 있을 건데 그 때야 그렇게 권한 건 당연하다.
바로 결단 내려서 +1수를 하는 게 낫기 때문이다. 그런데 만약 내 주장에 반론이 있다면, 그건 그 사람 나름대로 정당한 이유가 있는 것이고,
+1수론과 논쟁해는 과정에서 자기 확신을 얻을 수 있어서이다.
가장 곤란한 건 결국 이것도 저것도 아닌 채로 어영부영 시간을 보내는 것이다.
\vspace{5mm}

현 시점에서 +1수를 내가 권할 리는 없지. 지금 공부 시작하면 승산이 없지 않나. 가용시간 100일 정도인데 뭘 할 수 있지?
이 시기에 죽어라 일하면 된다 하는 사람은 그냥 알바 뛰어서 돈이나 버는 게 낫다.
\vspace{5mm}

다른 데 돌아보니까 결국 11월부터 4월까지 날려먹고 이 시점에 공부하겠다고 부지런히 결제들 하는 사람들 보인다.
그러니까 그들이 호구인 것이다.
\vspace{5mm}





\section{[중2론 003] 전부 아니면 꽝}
\href{https://www.kockoc.com/Apoc/722596}{2016.04.10}

\vspace{5mm}
\begin{enumerate}

    \item \textbf{ 전부 아니면 꽝}
    \vspace{5mm}

    이건 여러차례 얘기할 주제다.
    특히 수험사이트에서 주로 발견되는 것들이라서 그동안 흥미롭게 보았는데 슬그머니 다룰 때가 되어서 적어본다.
    \vspace{5mm}

    한국인(혹은 동아시아인)의 주된 멘탈리티 중 하나는 전부 아니면 꽝이라는 것.
    \vspace{5mm}

    \begin{itemize}
        \item[$-$]  서울 의대 아니면 그냥 재수
        \item[$-$]  금수저 아니면 자살
        \item[$-$]  목숨걸고 죽어라 하고 망하면 걍 자살.
    \end{itemize}
    \vspace{5mm}

    반쯤은 농담일 수도 있지만, 그건 절반은 진담이 된다는 얘기다.
    전부 아니면 꽝이라는 사고를 보면, 그 사람은 철이 덜 들었거나, 어떤 안전한 환상 속에서 살아왔다가 보면 되지 않나 싶은데
    그렇게 본다면 아직 한국사회의 정신적 성숙도는 낮지 않나 싶다.
    \vspace{5mm}

    전부 아니면 꽝의 문제는 이성적 사고를 마비시킨다는 것이다.
    10번의 승부가 있는데 1번 지면 나머지 9번의 승부도 포기해버린다(실제로는 쫄거나 하기 싫어서이다)
    명문대에 가지 못 하면 인생이 아작나는 줄 알고 자포자기한다(그리고 자신의 쓰레기짓이 이유있는 양 정당화한다)
    \vspace{5mm}

    그것 뿐만 아니라 직업관도 그렇다.
    xx대나 xx과에 가지 못 한 나머지 직업은 망하나?
    실제로 n수를 종용하거나 반드시 xx과에 가야한다는 것은 그걸로 벌어먹고 살고자하는 업자의 논리가 개입되어있다.
    실제로 더 잘 벌 수 있는 건 많은 사람들이 쳐다보지 못 하는 블루오션일 것이고 그걸 발견하기 위한 공부가 더 필요할 것인데도
    마치 xx대나 xx과에 가지 못 하면 망한다고 하는 것 역시 전부 아니면 꽝의 사고방식대로다.
    여기까지의 객관적 기술을 보면 알 것이다. 과연 전부 아니면 꽝에 사로잡힌 사람은 어느 길을 가도 잘 될 수 없다는 것.
    \vspace{5mm}

    \item \textbf{ 평균적 사고}
    \vspace{5mm}

    이 역시 합리성을 가장한 그럴싸한 사이비 판단이다.
    에컨대 특정 직업의 연소득을 제시하면서 특정 직업이 아니면 살 가치가 없다라는 식으로 단언하는 것인데
    재밌는 건 정작 그런 이야기를 하는 사람들은 적어도 내가 보기에는 인생 경험(=즉 고생)을 해본 적이 없는 사람들이다.
    \vspace{5mm}

    평균은 말 그대로 평균일 뿐이다. 그 이야기는 다시 말해 잘 나가는 사람도 있는 반면 못 나가는 사람도 있다는 이야기이다.
    평균 말고 고려해야 하는 건 편차다. 연소득 평균만 보자면 왜 사람들이 소득이 적은 공무원을 현실적으로 선호하는지 알 수가 없다.
    공무원들은 말 그대로 '안정적'이고 '편차'가 적기 때문에 그 적어보이는 소득이 의미가 있다.
    그러나 다른 직업들의 소득은 그 평균 외에도 편차와 시계열 자료를 보아야하는데, 이상하게 이런 이야기는 하지 않는다.
    \vspace{5mm}

    그 뿐만 아니다. 평균적 사고는 어디까지나 과거의 추세나 의미가 있다. 평균이 미래를 예측해준다?
    그건 단순한 선형적 외삽법에 따른 결과인데 요즘 같이 변화가 빠른 세상에 이게 적중할 리는 없다.
    통계자료에만 의존한 사고의 문제는 과거의 패턴이 미래에도 반복된다라고 '믿는다'는 것인데, 이 믿음은 그냥 버리는 게 좋다.
    과거에는 미래였을 현재의 흐름을 정확히 예견한 사람이나 기관이 있던가? 단지 추상적인 트렌드만 읽어냈을 뿐이다.
    아니 더 정확히 말하면 그 예언은 자기실현적일 수도 있다.
    미래는 정보화 사회가 된다는 예언은 인력과 자본을 정보화에 쏟아부은 자기실현에 가까웠다라는 것이 더 타당한 이야기일 것이다.
    \vspace{5mm}

    평균적 사고는 모든 게 다 정해져있다고 가정하는 안일한 사고다.
    이 역시 전부 아니면 꽝과 마찬가지로 이런 사고에 사로잡힌 것부터가 이미 감점 대상이다.
    \vspace{5mm}

    \item \textbf{ 한계수익적인 사고}
    \vspace{5mm}

    빵을 1개 먹을 때는 +20, 2개 먹을 때는 +10, 3개째 먹을 때는 +5, 4개째 먹을 때는 0, 5개 먹을 때는 -5.
    이런 경우 빵을 4개까지 먹는 것이 바로 한계수익(혹은 한계효용적)인 선택이다.
    전부 아니면 꽝, 평균적 사고에 사로잡힌 중2병 환자들이 탑재해야할 것은 바로 한계수익적인 사고다.
    \vspace{5mm}

\end{enumerate}
어떤 행위의 결과를 미리 예단하지 말자, 자기가 꿈꾸던 목표가 물건너간다거나, 자기가 생각한 평균치에 미달하더라도 실망하지말자.
\textbf{다만 그 행위를 하면 지금보다 조금이라도 나아진다면 그 선택을 하는 것이다.}
\vspace{5mm}

예컨대 올해 수능시험이 망할 것 같다는 예감이 든다. 그런데 지금 공부를 하면 어제보단 똑똑해지는 것 같다.
그럼 수능공부를 하면되는 것이다. 쓸데없이 놀거나 방황하는 것보다는 책 한자 더 읽고 문제 하나 더 푸는 것보다 낫기 때문이다.
반면 더 나은 선택안이 있어서 - 자기가 보기에는 걍 장사하면 더 많이 버는 게 분명하다 - 수능공부 대신 장사를 해야한다.
그런데 기존에 해둔 공부가 아깝다... 라고 하면 서슴없이 장사를 하면 된다, 어차피 기존에 한 수능공부는 매몰비용이기 때문이다.
\vspace{5mm}

한계수익적인 사고는 상식에 어긋나보인다. 그러나 사실 정확히 판단만 한다면 매일매일이 '플러스'가 남는 선택이다.
남들이 xx가 좋다고 하던, xx 계통이 암울하다고 하든, 그런 건 생각하지 말고 매일매일 조금이라도 이익을 거두는 사람이 승리하게 된다.
정반대로 전부 아니면 꽝이라고 샤우팅을 하거나, 자기 선택이 평균보다 높다고 자위하거나 아니면 낮다고 좌절하는 사람들은,
하루하루가 플러스가 되지 않고 마이너스가 되기 때문에 결과적으로는 뒤처진다.
더 무서운 사실은 그 마이너스를 알면서도 자기가 선택을 잘하면 \textbf{'평균이 높아지니' 상관없다고 착각}하는데
하루하루 플러스도 못 하는 사람들이 횡재를 바라는 꼴이다.
\vspace{5mm}

즉, 어제보다 더 나은 하루를 살면 된다.
어제보다 조금이라도 퇴화한다면 심각한 대책회의를 열어야 할 것이다.
자기 진로가 어떻든 한계수익적인 사고에 따라 어제보다 나은 하루를 사는 걸로 1년이 지나면 무섭게 자라있을 것이다.
\vspace{5mm}







\section{[중2론 004] 선택은 기회비용}
\href{https://www.kockoc.com/Apoc/724006}{2016.04.11}

\vspace{5mm}

\textbf{기회비용} : A를 선택했기 때문에 \textbf{포기}한 것 중에서 가장 가치높은 것을 \textbf{포기}한 비용
\vspace{5mm}

가령 수능을 실패했는데 서슴없이 군대가라고 조언하는 케이스는
어차피 군대를 안 가고 공부해보았자 2년 허송세월할 게 너무나도 뻔해보여서입니다.
이 경우는 군대를 가는 대신 공부를 2년간 유보해도 잃을 게 없으니 기회비용이 적습니다.
그러나 이 친구가 군대에 안 가고 공부를 한다면 기회비용이 역으로 커지겠죠(일찍 갈수록 좋으니까요)
\vspace{5mm}

선택장애를 치유하는 방법 중 하나가 이 기회비용적 접근입니다.
A를 선택할까, B를 선택할까 하면서 A의 가치, B의 가치를 논하는 건 자기가 고려하기 어려운 C, D, E... 등을 누락시키는 문제가 있죠.
그 경우에는 A를 선택했을 때 포기해야만 하는 것의 가치, B를 선택했을 때 포기하는 것의 가치를 고려하시면 되겠습니다.
\vspace{5mm}

철수라는 친구가 대학교를 포기하고 쌩n수를 한다고 칩시다.
쌩n수를 하면 이 친구는 대학교에 다니는 것 이외에도,
그냥 일함으로써 벌 수 있는 근로소득, 혹은 공무원 시험 공부하기, 아니면 창업해서 대박내기 등도 고려해보아야합니다.
그러나 우리 철수는 쌩n수해서 성공했을 때의 가치만 생각합니다.
그런데 이런 접근은 상당한 문제가 있죠. "n수해서 실패해도 본전"이라고 착각한다는 겁니다.
자기가 성공해서 +100이라는 효용을 얻으면, 실패해도 제로가 아니겠느냐... 라고 하는 것이죠.
그런데 그가 특정한 사업을 벌여서 +10000000 의 가치를 얻을 수 있었다고 하면 실제로는 -1000000이 됩니다.
이런 극단적인 경우라면 합격하더라도 손해긴 하겠지만요.
\vspace{5mm}

물론 이런 접근은 다음과 같은 의문을 불러일으킵니다.
"그럼 자기가 선택할 수 있는 것들이 어떤 것이 있는지 정확히 알 수가 있나"
조금만 생각해보면 이 역시 자기 책임이라는 결론에 도달하죠. 무엇인지는 자기가 스스로 조사하거나 공부해서 파악해야하니까요.
가치라는 게 주관적이라는 점에서, 그래서 어른들이 \textbf{"좋아하는 것을 하라"}고 이야기하는 게 정답일지도 모릅니다.
최고의 가치는 일단은 자기가 좋아하는 분야부터 매겨지기 때문입니다.
\vspace{5mm}

그런데 여기서 눈치빠른 사람은 n수의 문제를 다시 깨닫게 됩니다.
그건 바로 n수의 기회비용이 역설적으로 줄어든다는 겁니다.
재수할 때에야 아직 젊고(!) 게다가 꿈이 있기 때문에 가치를 매길 수 있는 선택안들이 많습니다.
그래서 재수의 기회비용이 커집니다.
그러나 삼수, 사수, 오수... 가 되면 나이는 먹을대로 먹고 본인의 몸값이 낮아지므로 선택안이 줄어듭니다.
그래서 1년 더 공부한다고 해도 잃을 게 없다고 느껴지기도 하고 실제로도 그렇습니다.
\vspace{5mm}

그런 의미에서 보자면 특정 선택을 함으로써 잃는 게 크다는 것 자체가 역설적으로 '희망'인 것이죠.
그렇다고 할지라도 n이 늘어날수록 나아지겠네요... 라는 바보같은 질문은 없길 바랍니다.
n수생은 결국 그 막대한 기회비용을 치름으로써 결과적으로 자산이 줄어들어 이미 마이너스를 찍은 상태라서 잃은 게 없는 것입니다.
그럼 거꾸로 금수저라면? 슬프지만 이 경우라면 n이 늘어나도 상관없을지도 모르죠.
\vspace{5mm}







\section{[중2론 005] 돈을 버는 방법}
\href{https://www.kockoc.com/Apoc/746336}{2016.04.25}

\vspace{5mm}
\begin{enumerate}
    \item 수입을 늘린다
    \item 투자한다.
    \item 지출을 줄인다.
    \item 부잣집 사위 혹은 며느리가 된다.
    \item 전생에 나라를 구하고 부자집에서 태어난다.
    \item 한건하는 범죄를 저지른다
\end{enumerate}
\vspace{5mm}

뭔가 이상한 게 있지만 넘어 가면, 돈을 번다는 건 결국 저축한다와 똑같은 의미고, 저축하기 위해선 이 3가지 외에는 방법이 없다.
\vspace{5mm}

\begin{itemize}
    \item 첫째 수입 늘리기.
    \vspace{5mm}
    
    회사원 - 출세, 야근수당 이직등
    부업 - 알바, 책 쓰기, 인터넷 사이트 제휴광고
    창업 - 치킨과 편의점
    \vspace{5mm}
    
    인간의 노동 가치가 점점 낮아지고 있어서이다. 당장 외노자를 쓰거나 기업을 해외로 옮겨도 인건비 절감이 된다.
    전문직으로 가면 되지 않냐 하지만 그건 그 시장의 수요와 공급을 보면 된다. 초과수요가 발생하는 곳은 어지간해선 찾기 어렵다.
    부업은 잘 된다는 보장도 없고 무엇보다 이 역시 자기 몸을 망가뜨린다. 창업은 설명할 필요조차 없을 것이다.
    \vspace{5mm}
    
    \item 둘째 투자하기
    \vspace{5mm}
    
    주식투자, 부동산투자 등
    \vspace{5mm}
        
    공부 안 하고 들어가면 판돈을 잃거나 돈이 묶여버린다.
    기본 종잣돈과 공부가 필요할 뿐만 아니라 자칫하면 자기 돈을 날려버릴 수 있다. 게다가 스트레스도 와장창 받는다.
    투자에 성공한 사람들은 수능 최상위보다도 적다.
    제가 아는 사람들은 많이 벌었는데요.... 그거 얼마나 걸렸나 물어보라, 많이 잃다가 운이 좋아 대박 한번 터뜨려 만회한 경우다.
    이 때 들어간 시간자원으로 따지면 투자가 아니라 노동이란 말이 맞을 것이다.
    \vspace{5mm}

    \item 셋째 절약하기
    \vspace{5mm}

    의식주 비용, 유흥비와 온갖 소모품, 집세, 통신료 등
    \vspace{5mm}

    한달에 100만원을 벌고 50만원을 쓰는 사람과, 한달에 2000만원을 벌고 2500만원을 쓰는 사람 중 누가 낫나.
    지출을 줄이는 건 누구라도 실천할 수 있다. 거꾸로 말해 누구라도 실천 안 할 수도 있지만 이건 마음의 문제다.
    이것이야말로 가장 확실한 돈버는 방법이다.
    \vspace{5mm}

\end{itemize}
돈을 버는 건 고전적으로 저 3가지 틀을 못 벗어난다.
그리고 저 중에서 가장 확실하고 노력이 더 들어가는 건 절약하기, 즉 지출을 줄이는 것이다.
남들이 xx 샀다고 해서 부러워할 게 아니라, 그만큼의 지폐를 침발라 세어놓은 뒤 밤마다 만지면서 지출을 억제하면 된다.
갑자기 충동적으로 구매욕이 드는 순간 바로 '일주일 지나서 구입하자'라고 장바구니만 등록해놓고 기다리는 게 바로 돈버는 길이다.
\vspace{5mm}

그런데 실제로 지금 20대들은 어떤가. 정말 뼈빠지게 아끼고 허리띠 조이면서 헬조선을 외치는 사람이 몇이나 될까.
나라 사정 어렵다는 말을 안 믿는 이유는 그렇게 힘들다는 사람들이 휴일에 해외여행을 다녀오거나 비싼 커피숍을 애용한다.
우리 때는 이랬다... 하는 건 치사한 논법이라는 소리 들을지 몰라도, 내가 아는 한 지금 노년 세대가 바로 전셋집부터 시작하는 케이스는 아니었다.
그 분들 세대는 단칸방에서 출발해서 아끼고 또 아끼 모으고 또 모으니 돈이 눈덩이처럼 불어난 것이다.
그에 비해 젊은 세대들은 '한방'을 노린다. 그 한방이 로또를 지르는 것과 뭐가 다른지는 모르겠다.
\vspace{5mm}

가계부를 소프트하게 쓰면서 자기가 얼마나 쓰는지 기록하고 지출을 억제할 것.
스트레스를 받으면 돈을 현금으로 모아두면서 그 지폐와 동전을 만지작거리는 걸로 대리만족할 것.
이건 소설이 아니라 실제로 숨겨진 부자라고 하는 사람들의 습관이 저런 걸로 알고 있다.
진짜 알부자들은 평소에 허름하게 하고 다니면서 바겐세일만 찾아다니고 식료품도 마감 직전 마트에서 구입하는 아줌마들인데
이런 분들의 낙이 자기 전에 금고에서 두둑한 지폐/수표 다발을 꺼내서 한장한장 세고 후루룩 넘겨보고 스킨쉽(?)해보는 것이라고 한다.
그 순간 현란한 광고에서 나오는 '돈 쓰세요'라는 저주가 사라진다나.
\vspace{5mm}

그렇게 보면 사실 현금카드나 신용카드는 매우 위험한 도구다.
신용카드나 전자화폐는 돈이 숫자이다. 느낌이 없기 때문에 소유욕이 생기지 않는다. 그래서 느낄 수 있는 '소비'를 하려고 한다.
심지어 마이너스 통장은 그 마이너스 액수가 자신의 피가 빨리는 것인데도 그 액수가 늘어나면 뭔가 성취한다는 착각을 준다.
자기가 지출이 심한 사람은 돈을 계좌에 입금하기보다는 숨겨진 비밀 장소 같은 데에 지폐와 돈을 차곡차곡 보관해보고 만져보는 걸 권하고 싶다.
무슨 그런 터무니없는 소리가 어딨냐 하겠지만 이건 수학문제를 직접 손으로 풀어야 감이 생기는 것과 같다.
지폐와 동전으로 실존적(...) 체험을 하는 사람들은 돈의 후각이 발달하고, 후각이 발달하기 때문에 돈의 흐름을 읽어낸다고 한다.
\vspace{5mm}

경제학 하나 배워본 적이 없는 복부인들이 부동산 투기를 잘 하는 이유는 다른 게 아니다.
집에 미쳐있기 때문에 이 분들은 집값이 어떻게 오를지 그리고 돈이 어떻게 흘러가는지 여성 특유의 직감으로 읽어내는 것이다.
반면 경제학에 빠삭하다는 이론파들이 망하는 건 다른 게 아니다. 오직 머리로만 느끼기 때문에 실체에 접근하지 못 하는 것이다.
\vspace{5mm}

이에 비해서 젊은 사람들은 돈을 모으는 모든 행위가 디지털적이다. 그래서 느끼지 못 한다.
이런 불감증에 메마른 자아는 촉촉한 소비에 유혹당해버린다. 그래서 빚쟁이들로 전락하고 만다.
서울 번화가 - 강남 신촌 등에서 여유있게 소비하는 남녀들을 보고 절대 부러워할 게 아니다. 태반이 채무자들이다.
빚을 져도 부모에게 기대도 된다, 아니면 애인이나 미래 배우자에게 전가시키면 된다고 하는 쓰레기들도 널렸다.
\vspace{5mm}

우리가 할 수 있는 건 사실 저축 뿐이다. 즉 아껴쓰는 것이다.
부자처럼 살고싶다고 하는데 쓰기만 한다.... 남자는 농담 아니고 새우잡이배이고, 여자는 결국 결혼의 형태든 매춘의 형태든 자신를 팔게 된다.
우리가 할 수 있는 건 아껴쓰고 또 아껴쓰면서 그로써 살려낸 현금을 만지면서 자랑스러워하고 자기가 구두쇠가 되어간다는 걸 자랑스럽게 여기는 것이다.
심하게 말하면 타인의 소비를 내심 경멸해야 한다(내심 경멸하지 겉으로는 아무 말도. 사실 간섭할 권리는 없는 것이다)
철저하게 아껴쓰고 그 시간에 도서관이나 헌책방에서 좋은 책을 구해 읽으며 자기 뇌를 강화하면 그 때부터 희망이 보인다.
\vspace{5mm}

"\textbf{있는 놈이 없는 척 하고, 없는 놈이 있는 척 한다."}
\vspace{5mm}

적어도 돈 문제에서는 이에 버금가는 진리는 없다.
\vspace{5mm}








\section{[중2론 006] 노오력은 어디까지 해야하나}
\href{https://www.kockoc.com/Apoc/748594}{2016.04.27}

\vspace{5mm}

운은 비와 같다.
비가 내려도 그걸 담을 그릇이 없으면 아무 소용이 없는 것이다.
누구나 운이 좋을 때가 있다. 그런데 그 운을 담을 그릇이 없다. 그래서 그 운이 지나가버린다.
그릇을 만들지 않고선 자기가 억울하다고 하소연하는데 어쩌란 말인가.
\vspace{5mm}

운이나 노오력이 소용없다 이야기할 때는 그 화자를 반드시 본다.
그리고 정말 노오력을 죽어라 해서 인정할만한 사람인가하면 개인적으로는 단 한명도 없다.
패시브 독설 스킬이 있어서 자제하지만 지적하려면 얼마든지 지적할 수가 있다.
\vspace{5mm}

정말 노오력하는 사람들은 내가 아는 한 운타령을 안 한다. 운을 얘기하는 건 단 하나, 자기가 성공해도 \textbf{겸손떨기} 위해서이지.
노오력한다는 게 어느 정도냐면 비가 내리는 데에도 기우제를 지내 신이 질려버리게 하는 정도다.
이들은 대개 괴물들이라고 불린다. 운타령사주팔자타령하는 사람도 이들 앞에서는 찍소리 못 한다.
\vspace{5mm}

운이 있다고 해도 간단한 거다. 노오력을 해서 그릇이 그랜드캐년 수준이니 운이 좋으면 극대화된다.
운이 나쁘더라도 노오력으로 커버쳤기 때문에 잃을 거나 망할 것도 별로 없다. 그래서 결국 운이 별 역할을 못 끼치는 것이다.
\vspace{5mm}

시험을 보기 위해서 이 정도만 공부하면 되지, 취업하기 위해서 저 정도만 공부하면 되지... 이걸 가지고 노오력이라고 하지 않는다.
행운은 100배로, 불행은 1/100배로 하는 것도 물론이지만, 그 자체가 \textbf{타인을 감동시켜야} 노오력이다.
자기가 노오력했다고 하는 사람들은 공부하는 걸로 남들을 공포에 빠뜨리거나 감동시켜보았지 차분히 복기해보면 된다.
그런 경우는 그런데 100명당 1명꼴도 나오지 않을 것이다.
\vspace{5mm}

그것도 다 \textbf{기득권} 탓이라고 하는 사람들은 정말 본인들이 직접 그 \textbf{기득권}을 경험해보았나?
사실 10대 후반이나 20대 초반들은 그럴 일이 거의 없다. 뭔가 조직생활을 하고 착취당해보아야 피부로 느끼는 것이다.
그럼 반론은 꼭 경험해보아야하나요... 라고 하면 도대체 그 기득권이 그래서 누구냐... 라고 하면 뻔한 대답 나올 것이다.
이러니까 무시해도 좋은 것이다.
그런데 개나소나 기득권 드립친다. 그럼 그게 어디서 주입된 것이겠나?
\vspace{5mm}

그럼 타인을 감동시킬 정도로 노오력한 게 배신으로 돌아온 경우는?
없지는 않을지도 모르지. 그런데 적어도 내가 그걸 목격한 적은 단 한건도 없다.
그건 기득권 드립도 마찬가지다. 기득권을 경험한 사람들이야 처음에는 기득권을 욕하지만
곧 자기들이 그 기득권만 된다면 그래도 어느 정도 꿀빨 수 있구나, 기득권이라는 사람들도 나름 속사정이 있구나를 알면서 \textbf{간사해진다}.
지금 갑 행세하는 사람들이 과거에 갑 욕 안 했을 것 같나.
며칠 전에 모 시사 프로에서 폭력을 비판하는 프로의 진행자가 과거에 폭력을 행사했더라는 게 제보된 것과 똑같은 얘기다.
\vspace{5mm}

이렇게 하나하나 따지고 분석해보면 세상은 의외로 공평한 것이다.
본인들이 인과관계를 따지지 못 했으니까 다 운빨인 걸로 착각하지만 가만히 보면 필연적 인과관계라는 건 분명히 있다.
단지 그걸 본인들이 못 보거나 알면서도 외면하는 것이다.
특히 운이 최고다하는 사람들은 \textbf{'노오력해보았자 소용없다'}라고 하고 싶어서 그런 경우가 많다.
물론 나는 그런 사람들이 내 편이 아닌 이상 냅둔다. 그래서 망해도 그건 나랑 상관없지 않은가.
\vspace{5mm}

사실 자본주의 사회가 서로 경쟁자라면 다른 사람들이 노오력해도 소용없다 운이 최고야 하는 걸 오히려 부추기는 게 좋다.
\vspace{5mm}

+ 운이 좋은지 나쁜지 알아보았자 소용없는 게
그건 대개 과거형이다. 운이 좋았다 나빴다라고 평가하는 게 대부분 아닌가.
운이 좋을 것이다 나쁠 것이다라고 하는 미래형은 바로 점성술이나 사주팔자. 그런데 이걸 알아도 별 소용은 없다.
운이 좋다고 노력을 안 할 것인가. 그럼 운이 나쁘다고 포기할 것인가.
\vspace{5mm}

++ 내가 수험생이었다면 노오력해야해요 그딴 드립은 안 쳤다.
모두가 운으로 좌우된다, 노오력해보았자 소용없다라고 선동해서 경쟁자들 \textbf{떨어뜨렸지}.
실제로 중고딩 때 그런 일환(?)으로 만화책과 게임을 신나게 유포했으나 정작 공부하는 녀석들에겐 안 먹혔다는 슬픈(?) 후일담이.
청년들보고 거리로 나가서 시위하라고 하시는 분들은 자기 자녀들은 어떻게 했을 것 같나라고 하면 이건 좀 흠좀무한 대목이지만.
\vspace{5mm}




