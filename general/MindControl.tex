


\section{[세뇌론 001] 시작하면서}
\href{https://www.kockoc.com/Apoc/499247}{2015.11.18}

\vspace{5mm}

운명이 정해져 있다, 바꿀 수 있다라는 이야기만큼 매력적이지만 골치아픈 주제는 없을 것이다.
다만 이 주제에 대해서는 전문가 수준은 아니더라도 그동안 경험하고 개인적으로 조사한 것을 정리해보고 싶었는데
\vspace{5mm}

인간이 걸어갈 때 장애가 되는 건 비단 '물리적'인 벽만이 아니라는 것,
정신적인 벽이 가장 중요하다는 것을 깨닫는데는 꽤 오랜 시간이 걸렸다.
\vspace{5mm}

죄수가 있다, 그 죄수를 가두는 건 1차적으로는 시멘트벽이다.
하지만 시멘트벽을 부수고 탈옥하는 것은 불가능하지 않다. 이걸 아는 당국은 어떻게 할까?
끈임없는 교육과 세뇌로 너희는 절대 탈출할 수 없다라고 이야기한다.
시멘트벽의 이미지는 곧 "빠져나갈 수 없는 감금 상태"라는 추상화된 기호로 학습된다.
그걸 학습한 죄수는 나갈 수 없다라고 생각하기 때문에 탈옥 자체를 시도할 수 없다.
이는 즉, '기호적 언어'가 그 죄수를 가두는 것이다.
\vspace{5mm}

범죄의 발생 - 체포 - 구형 - 형의 실시.
\vspace{5mm}

이런 뉴스의 생산과 유포야말로 안전한 사회를 만든다는 것.
저런 범죄뉴스를 학습한 사람들은 자기도 모르는 무형의 벽에 갇혀 살게 된다.
정확히 말하면 의식과 무의식을 지배하는 언어의 벽을 넘지 못 하는 것이다.
\vspace{5mm}

우리는 고정관념을 깨자라고 얘기를 하지만 사실 이건 기만에 가까운 이야기다.
우리가 살아가는 세상은 고정된 기호의 견고한 완성물이다.
머릿 속에 관념적으로 세겨진 그 기호의 감옥을 파괴할 줄 아는 사람만이 자유롭게 살아간다.
매트릭스가 온라인 세상만을 의미하는 건 아니다. 아니 사실 인터넷은 디지털로 입력된 기호의 세계일 뿐 그 전부는 아니다.
우리가 수학을 배우는 것도 사실은 "그 기호와 법칙으로 이뤄진 수학적 가상현실"을 머릿 속에 구축하는 과정일 뿐이다.
\vspace{5mm}

그런데 이게 운명론과 어떤 관계가 있나요.
\vspace{5mm}

우리가 사는 세상이 우리가 학습한 기호들로 이뤄진 세상이라면
운명이라함은 \textbf{그 기호들만으로 만들어지는 사건이다.}
그리고 여기서부터 운명을 바꿀 수 있는 그나마 냉소적이면서 현실적인 자유를 찾을 수 있을 것이다.
\vspace{5mm}






\section{[세뇌론 002] 인생은 곧 세뇌}
\href{https://www.kockoc.com/Apoc/499298}{2015.11.18}

\vspace{5mm}

학습과 교육을 적분한 결과가 인생이라면, 곧 그건 세뇌 과정이라고 할 수 있다.
사실 넓은 의미건 좁은 의미건 우리는 특정 기호들'을 의심하지 않고 주입받은 것에 갇혀 살아간다.
\vspace{5mm}

우리는 우리가 보는 세상이 완전하다고 착각하지만
실제로는 매우 불완전한 세상을 불완전한 감각에 의존해 일부 이미지만 조합하고 있으며,
이 역시 우리들에게 주입된 기호의 물리적 작용에 따라 편향적인 방향으로 보고 있다.
그러면서도 그 편향성을 만들어나가는 기호들의 필연적인 관계가 바로 우리들의 '운명'을 결정한다.
\vspace{5mm}

실제로 운명을 결정한다면 이건 두가지로 설명해볼 수 있다.
\vspace{5mm}

첫째는 인간이란 기호를 해체하고보자면 결국 원자덩어리들이고
그 원자덩어리들이 예정된 운동을 하는 것이라고 가정한다면 운명이란 결국 필연적인 우주 움직임의 해석이라고 볼 수 있다.
우리가 공부를 잘 해서 좋은 대학에 가는 것부터 먹고자고싸고하는 것들을 원자, 분자의 움직임이라고 본다면
이건 물질들의 필연적인 움직임이 있다는 전제 하에서 운명이 있다라고 얘기할 수 있는 것이다.
다만 이 설명은 매우 치명적인 문제가 있다. 그 원자나 분자들의 '기계적인 움직임'도 하나의 가정에 불과하다는 것.
애초에 우주에 대한 기계론적 설명부터 역시 하나의 주관적 학설이고 주장에 불과할지도 모른다는 것이다.
완벽해 보이던 뉴튼의 기계적인 물리 이론조차도 실은 틀리지 않았나.
\vspace{5mm}

둘째는 바로 기호들의 필연성이다. 우리가 수학에 매료되는 것은 다양한 기호들의 조합이 필연적인 결론으로 이어진다는 것이다.
삶이 무수히 많은 기호들의 조합이고 인생이 그 기호들로 적분된 결과라면
우리 인간은 어떤 특정한 현상을 "우리의 기호 체계"로서 주관적으로 해석한 것을 인생으로 받아들이면서 여기서 운명 관념이 생긴단 것이다.
가령 죽음에 매우 민감한 사람이 있다.
너무 민감하다보니 모기가 죽는 것, 고양이가 죽는 것부터 시작해 자기가 먹는 야채, 생선, 쇠고기까지도 죽음으로 바라보게 되었고.
그리고 그 죽음의 패턴이 숫자 7과 관련있다는 걸 발견했다고치자.
그럼 이 사람에게는 숫자 7과 죽음은 다양한 이야기를 만들어내면서 이게 하나의 운명으로 형성되어버린다.
\vspace{5mm}

사실 이 둘째야말로 운명론을 결정하는 가장 중요한 이론일 것이다.
숫자 7과 죽음이 뭔 관계냐고 비웃을지 모르지만, 사실 이 글을 쓰는 사람이건 읽는 사람이건
하나하나 다 회의해보지 못 한 기호체계들을 '패턴화'시킨 상태라는 것을 자인하지 못 하고 있다.
우리가 쓰는 언어체계에서 ㄱㄴㄷㄹ ... ㅈㅌㅋㅌㅍㅎ부터 시작해 국어문법부터 어휘 중 필연적인 것은 아무 것도 없다(자의성)
다만 그렇게 학습하고 반복함으로서 이 모든 것이 진실이고 절대로 바꿀 수 없다고 느끼고 있는 것이다.
\vspace{5mm}

머리가 좋은 사람은, 아 바로 이걸 넘어서면 운명이 바뀌곘군요라고 하지만
이것조차 우리가 모르는 정체불명의 기호들로부터 도출된 뭔가 수상쩍은 결론이란 점에서 끝까지 기호에 사로잡힌다라는 문제가 생긴다.
아니 애초에 운명을 개척하고 넘어선다는 것 자체도 우리의 자유의지가 아니라, 실제로는 운명이란 걸 만든 기호들의 음모라면 어쩌겠는가.
\vspace{5mm}






\section{[세뇌론 003] 교주들의 등장}
\href{https://www.kockoc.com/Apoc/499321}{2015.11.18}

\vspace{5mm}

애초에 종교의 시작은 수상쩍은 데가 많다.
가장 수상한 것은 그것이다. 원시 샤머니즘 수준이 아닌 본격적인 종교의 시작은 \textbf{'문명' 이후}라는 것이다.
\vspace{5mm}

하나만 예를 들면 샤머니즘은 엄밀히 말해 종교라고 할 수가 없다.
샤머니즘에서는 추상화된 신이 존재하지 않는다. 그저 인간이 자연과 하나로 몰입하여 자연의 메시지를 애기한다.
몰입한 무당들은 의식이 없다. 인간의 의식이 있다면 자연과 하나가 될 수 없기 때문이다.
\vspace{5mm}

샤머니즘과 달리 현대 사회를 지배하고 있는 종교들의 특징은 "언어"를 정말 중시한다는 것이다.
그 말씀이나 경전은 사람들에게 이 세상이 어떤지 주입을 시킨다.
그리고 종교의 권능은 이 세상의 현상을 그 경전에 따라 해석함으로써 발휘된다.
실제로 사람이 죽은 뒤에 어떻게 되는지는 아무도 모른다.
그러나 기독교의 해석으로는 천국에 가있고, 불교의 해석으로는 윤회의 고리를 타고 간다.
이는 자연과학적인 해석과 살짝 충돌하거나 비켜지나가지만
그 종교적 기호에 세뇌된 사람들에게는 "필연적인 현상"이며 이로써 그 기호는 실체가 되어간다.
\vspace{5mm}

실제로 천국이나 윤회가 있느냐 없느냐.... 존재하긴 할 것이다. 바로 \textbf{"언어" 속에} -
만약 언어가 없었다면 천국이나 윤회에 관한 생각은 없었을 것이다. 추상화가 불가능하기 때문이다.
추상화가 불가능하다면 일반화시킬 수 없을 것이며, 일반화시킬 수 없으면 기호들이 우리를 강하게 사로잡을 수가 없다.
\vspace{5mm}

어떤 종교가 사이비인지 아닌지 이 역시 절대적으로는 구분할 수 없다.
그러나 분명한 사실은 교주들이 등장한 건 구어든 문어든 "언어"가 보급된 이후라는 것이다.
현대사회에서도 살아남은 종교들은 '말씀'으로 이어지고 있으며 신흥종교들도 역시 말씀을 통해 이어지고 있다.
고대 그리스 로마 신화나 게르만 신화가 종교가 되지 못 하거나 중국의 도교나 일본의 신토가 그리 강력하지 못 한 이유도 그렇다.
사실 이것들은 이미지로만 치자면 무수히 많은 수익을 거둘 수 있는 영감들 투성이다.
그러나 그것 뿐이기 때문에 문제(?)다. 이것들은 결국 '말씀'의 차원으로까지 이어지지 못 했다.
\vspace{5mm}

이미지가 없고 황량한 말씀은 매우 빈약하고 허술해보인다.
그러나 그 '말씀'을 통해서 추상화된 기호는 한번 학습되면 지워지지 않는다.
애당초 3대 종교 교주들이 자기들이 창시한 종교가 이토록 세계를 지배했을 것일지.
그 중 지구가 둥글다라는 걸 알았던 사람도 몇이나 있었을까도 그렇지만, 자기들이 남긴 말씀의 위력을 사실 \textbf{아무도 몰랐을 것}이다.
이들을 무시할 수 없는 건 어찌되었든 현대 사회의 형성에 이들의 말씀이 주요한 역할을 했기 때문이겠지만
한편으로는 문명이라는 건 결국 '근거없는 말씀의 주입'으로만으로 형성될 수 있었다는 점에서 이 문명의 허술함에 통탄하지 않을 수 없으며
광신도 집단에 불과할지도 몰랐던 ISIL이 생각보다 오래 가거나, 혹은 그들의 목적을 완성시킬 수도 있다는 가능성을 무시할 수 없다.
\vspace{5mm}

이런 거창한 담론을 넘어서서 사소한 일상이나 과거를 되돌이켜보아도
나나 주변인들의 삶을 지배하는 건 언어와 기호였다.
사주팔자나 타롯카드의 장점은 이것들은 어떤 것이든 그럴싸하게 설명할 수 있는 참 탁월한 기호란 것이다.
거창하게만 생각하지 않는다면 자신의 운명에 의미를 부여하는 차원에서 쓸만하겠지만
그걸 넘어서 그 언어에 절대성을 부여하는 순간 우리의 이성은 석기시대 이전으로 돌아서버리고 만다.
\vspace{5mm}






\section{[세뇌론 004] 마법사들의 암약}
\href{https://www.kockoc.com/Apoc/499345}{2015.11.18}

\vspace{5mm}

그렇게 인간이 살아가는 세상이 기호의 세상이라면
그 기호를 움직이는 건 두가지이다.
하나는 물리적인 실체가 정말로 바뀌는 것이고,
다른 하나는 물리적인 실체를 해석하는 본인의 기호가 변하는 것이다.
\vspace{5mm}

이런 걸 깨달은 자들이 있다 - 유감스럽지만 이들의 직업이나 스킬은 그리 유형화되지는 않은 것 같다 -
단지 말로만 가지고 사기를 치는 사람들도 있지만, 이 경우는 기호를 조작하는 수준까진 아니다. 단지 거짓말을 한 것이기 때문이다.
\vspace{5mm}

하지만 기호 자체에 영향을 줄 수 있는 사람
우리의 의식과 무의식을 넘어서 변화를 줄 수 있는 마법사들이 있다.
그들은 사이비 종교의 교주일수도 있고, 인터넷 강의를 찍는 강사일수도 있고, 일개 블로거일수도 있다.
사기꾼에게 속은 사람들은 자기가 당했다는 걸 알고 길길이 뛰면서 경찰서에 달려가지만
마법사들은 그들에게 속았다거나 영향을 받는다고까지 생각하지 못 하는 것이다.
끝까지 단물을 빨아먹히고 착취당하더라도 끝까지 그 마법사의 말이 옳다고 생각하며
그 마법사를 공격하는 어떤 시도에 광신적인 반응을 보이게 된다.
\vspace{5mm}

하지만 이 마법사들이 오래 가는 경우는 드물다.
이 마법사들에게 착취당한 사람들이 또 다른 피착취자를 영속적으로 생산한다든가
적당히 해먹고 중간에 건전한(?) 사업으로 빠지면서 양지에 발을 걸친다면 오래 갈지도 모르지만
언젠가는 수익율이 떨어지기 마련이고, 작년 세월호 침몰 사건 같은 것이 터질 경우에는 마법사도 속수무책이기 때문이다.
그 마법사들이 자연과학적 법칙을 거스른 적은 단 한번도 없으며,
어느 나라건 그 나라의 사법체계 시스템에서 꼼수를 부려 도망간다고 하더라도 그 자체를 거역한 적은 단 한번도 없다.
즉, 기호를 조작하고 세뇌하는 마법조차도 짧게는 수백년, 길게는 수천년의 역사를 가진 '법'을 완전히 장악하지는 못 한다는 이야기다.
\vspace{5mm}

하지만 사회 전체가 성숙해진다고 하여 개인이 성숙해지는 건 아니다.
콕콕 사이트의 수험생들만 봐도 느끼지만, 개인의 기호 시스템은 매우 취약하다. 언제든지 부서질 수 있는 것이다.
공교육이든 사교육이든 이걸 제대로 가르치는 경우는 드물다. 물론 최근에 픽업아티스트라고 하여 여자들의 기호를 공략하는 건 성행한다.
그러나 이것은 정식으로 교육되거나 가르치지 않기 때문에 저 마법사들의 먹이가 된 호구들이 사라질 수가 없다.
\vspace{5mm}

오늘은 여기까지.
이 글을 읽은 사람들은 마법사라는 기호가 주입되었고
아울러 그 마법사들이 어떻게 개인 세계를 농락하는가에 대한 이미지를 형성할 것이다.
그럼으로써 세뇌가 어떻게 운명에 영향을 미치는지에 대한 입문에 성공할 것이다.
\vspace{5mm}






\section{[세뇌론 사례 01] 외모와 인기}
\href{https://www.kockoc.com/Apoc/499377}{2015.11.18}

\vspace{5mm}

딱히 세뇌론 사례라고 붙일 건 없지만 응용가능하다고 생각되어서리.
적어도 음흉한 제가 보기엔 인터넷만 돌아보아도 우울증에 걸릴 사람들이 많다고 보는데
그건 SNS 때문입니다.
\vspace{5mm}

첫째, 얼굴을 드러내는 경우
둘째, 잘 생겼다 혹은 예쁘다와 같은 외모에 연연하는 경우
셋째, 인기.
\vspace{5mm}

지금은 아직 검증될 시기가 아닙니다만 10년 정도 흐르면 이와 관련해서 환자들이 생겨날 겁니다.
이유는 간단하죠. 우리는 \textbf{나이를 먹기 때문입니다. 그리고 노화되어버리지요}.
\vspace{5mm}

그런데 문제는 자기가 한창 인기있는, 속칭 \textbf{리즈 시절이라고 할 때의 젊고 잘 생기거나 예쁜 시절만 기억한다는 것입니다.}
거기에 세뇌되어버린 이상, 나이을 먹으면서 추해지고 늙어가는 모습을 '부정하고' 싶어집니다.
아무리 곱게 늙어간다고 해보았자 젊은 게 좋은 거죠.
그럼 애시당초에 이걸 비교하지 않으면 차단할 수 있는데
\vspace{5mm}

문제는 인터넷으로 자기 얼굴을 알리면서 유명해진 경우 - 그 사람들에게 전파된 꼴이니 이걸 수습할 수도 없거니와
자신의 외모로 유명해졌다는 그 시절에서 벗어나는 건 매우 괴롭단 것이죠.
\vspace{5mm}

그래서 연예인들이 먹고사는 것 때문도 있지만  한동안 은퇴해서 성형수술을 하는 것이지요.
거금을 들여서 \textbf{다시 젊어질 수 있다면 다시 원래(!)의 자기로 돌아오기 때문}입니다.
하지만 성형의 결과야 뭐. 그거 오래 갈 수가 없죠. 늙어가는 자기를 인정하고 과거의 자기를 버리지 못 하면 괴로울 수 밖에요.
\vspace{5mm}

연예인들이야 이런 게 반세기는 되었지만
일반인들이 인터넷으로 자기 얼굴 드러내면서 한 게 싸이월드 때가 처음인가. 아무튼 10여년 정도 밖에 되지 않았습니다.
\vspace{5mm}

과거의 이미지를 유지하지 못 하게 되면 사형선고 받은 것과 똑같을 것입니다.
본말이 전도되어서 과거의 그 얼굴이 진짜 얼굴 역할을 하는 셈이기 때문에 생기는 비극이라고 할 것이죠.
\vspace{5mm}

외모를 가꾸는 건 중요합니다만 우선순위를 거기에 두지 마십시오.
정신이 소중해서가 아닙니다. 어차피 우리가 뭘 하더라도 그 외모를 유지할 수는 없습니다.
투자를 하려면 본인의 지식, 교양, 기술, 실력에 투자하십시오. 벽에 똥칠할 때가 아니면 이건 티가 나긴 커녕 더욱더 좋아지니까요.
\vspace{5mm}

발상을 바꿔서 자기가 노화된 모습을 떠올려본 다음 거울을 보면 이보다 행복할 수 없을 테고
또 어떻게 늙어가야할지 방향잡을 수도 있을 것입니다.
\vspace{5mm}



\section{[세뇌론 005] 모두가 세뇌당하고 싶어한다}
\href{https://www.kockoc.com/Apoc/499770}{2015.11.19}

\vspace{5mm}

우리의 고정관념 중 하나가 부정적인 것은 사람들이 하기 싫어한다는 것
그러나 유감스럽지만 그렇지 않다. 부정적인 것이 정말 부정적인 것이면 진작 일소되었을 것이다.
술, 담배, 마약, 지나친 섹스 등이 나쁘다고 교육되면서도 지금도 고민하는 이유는 실제로는 사람들이 그걸 원하기 때문이다.
\vspace{5mm}

덧붙이면 과거에는 저런 것들을 금기시라도 해서 균형을 이루었지만
더 많은 돈을 벌고 싶어하는 음모가 "다양성"과 "상대주의"를 핑계로 그러한 금기까지 무너뜨려
10대들까지 마수를 뻗친다. 제제까지도 망사 스타킹을 신고 욕망하는 미친 세상이다.
\vspace{5mm}

그렇다면 세뇌는 어떨까
세뇌당하기 싫다고 생각하기 쉬울 것이다. 다들 자유를 원하고 세뇌를 거부할 것이라고
그러나 유감스럽게도 이 역시 그렇지 않다.
세뇌당하고 싶은 사람이 이 세상에 존재한다.... 가 아니라 사실은 \textbf{대부분}이다.
과장해서 말할 것이 아니라 100명 중에 99명은 세뇌당하고 싶어하고 있으며 이 글을 쓰는 자 역시 그렇지 않다고 부인하기 어렵다.
또한 세뇌당하고 싶어하는 사람은 학력이 낮거나 가난하거나 비윤리적인 사람만 그런 것도 아니다.
고학력자에다가 부자인데다가 윤리적인 사람들조차도 \textbf{세뇌를 원한다.}
\vspace{5mm}

불교에서 말하는 열반은 산스크리트어로 니르바나를 예기한다.
이는 번뇌가 완전히 사라진 절대 자유, 절대 상태를 가리킨다.
그런데 재밌는 건 교회를 다니건 사찰을 다니건 뉴에이지 명상수행을 하건
심지어 자신이 무신론자라고 주장하면서 과학을 신봉하는 사람들은 똑같은 용어만 쓰지 않지
모두가 저런 니르바나를 원하고 있다.
\vspace{5mm}

자, 그럼 여기서 브레이크를 걸어보자.
니르바나 혹은 너바나라고 했을 때 이것이 정말로 좋은 것인가 이 글을 읽는 사람은 의심해 본 적이 있었을까.
아, 물론 여기 똥글 읽는 사람들이야 이 색기 또 유도심문하네 하면서 "아, 나 있어요"라고 개뻥을 깔 것이 뻔한데
100명 중 100명, 1000명 중 1000명. 그 누구도 니르바나와 너바나를 의심하거나 부정해 본 녀석은 단 한명도 없을 것이다.
\textbf{왜냐면 니르바나, 너바나의 문제를 그 누구도 지적한 적이 없기 때문이기도 하지만}
\textbf{니르바나가 뭔지 이 글을 읽고 나서야 안 사람도 많을 것이다.}
\vspace{5mm}

그런데 신기한 것은 니르바나라고 하면 모두 좋은 걸 연상한다는 것이다.
현대인들은 특정 기호의 의미를 생각하지 않고 주입받기 때문이다.
대중문화에서든 종교에서든 니르바나는 안 좋은 맥락으로 쓰인 적이 단 한번도 없다.
만약 니르바나가 뭔지 의심해보았으면 이건 단지 니르바나를 부정하는 걸로 끝나지 않았을 것이다.
그 순간 그 사람은 정말로 새로운 세계로의 퍼스트 펭귄이 되어서 쇼펜하우어보다 더한 염세론자가 되었을 것이다.
\vspace{5mm}

니르바나를 원하는 사람들은 세뇌당하고 싶어하는 사람들이다.
다른 말로 하면 종교가 시키는대로 해서 천국에 간다거나
아니면 수행을 통해 참자아를 완성시킨다거나 하는 사람들,
아니 더 넓은 의미로 남의 말을 듣는 사람들, 특히 종교의 세계관을 의심없이 받아들이는 사람들은
세뇌당하고 싶어하는 사람들이다.
\vspace{5mm}

그렇다면 이런 질문을 할 것이다.
우리 어머니는 아무 욕심없이 xx에 다니면서 기도하고 선행했다
너 색기는 우리 어머니를 세뇌당하고 싶어한다라고 하는데 이거 싸가지없는 소리가 아니냐.
\vspace{5mm}

\textbf{싸가지없는 소리인 것은 맞을 것이다. 그러나 그래도 '세뇌당하고 싶어한다'라는 진실이 바뀌는 건 아니다.}
\vspace{5mm}

그럼 나는 거꾸로 반문하겠지.
천국, 천당 얘기하는데 정말 가본 사람이 있고 그게 실증이 되었냐고
참자아의 완성이라건 모든 욕망으로부터의 해탈이라고 하는데 그럼 단 하나라도 그런 사례가 있냐고.
참스승이라거나 참종교인이라고 하는 사람들이 나중에 드러난 것은 어째서 돈과 섹스, 그것도 아니면 권력 냄새가 지독하게 나냐고.
\vspace{5mm}

실증해보지 않은 것 - 즉 허구의 것이 추상화 과정을 넘어서 사람의 무의식에 뿌리를 내리고 실체처럼 행동한다,
이게 바로 세뇌의 요체인데 그럼 그것이 세뇌가 아니고 무엇이란 말이지?
\vspace{5mm}

...
\vspace{5mm}

컬트 Cult 는 종교적인 숭배를 말한다. 바꿔 말하면 일부 집단에 의한 열광적인 지지.
우리나라에서는 특정음악마니아를 지칭하는 말로 쓰이기도 한다.
대중매체에 드러나 있지 않아서 그렇지 저런 컬트에 빠진 사람들은 온갖 것을 다 내주면서 니르바나를 찾아 수행하려고 한다.
\vspace{5mm}

현대인들에게는 괴로움이 존재하지 않는 열반의 상태는 천국처럼 들린다.
아무 생각도 하지 않아도 되고 마냥 행복할 테니까.
\textbf{전혀 욕심없이 교주나 지도자가 시키는대로 움직이는 피세뇌자의 상태} 그대로이다.
세뇌당하고 싶은 사람들은 고생하지 않고 손쉽게 열반에 도달하고 싶어하는 사람들이다.
\vspace{5mm}

고생없이 열반에 도달하고 싶은 사람일수록 세뇌라는 주술에 걸려 자유를 박탈당한다.
그리고 그 과정에서 마음이 조정당하면서 돈과 시간을 뺏기고 심지어 성폭력의 객체가 되기도 한다.
결국 인격이 붕괴되면서 폭력을 당하는 상태에서도 행복감을 느끼는 노예로 전락한다.
\vspace{5mm}

여기까지 진술은 제3자를 바라보는 것이라 무난하니까 카메라를 돌려보자.
이 글을 읽는 당신들이 바라는 미래상과 희망.... 이것들이 실제로 조작되고 주입된 이미지라는 생각은 해보았는가.
당신들의 일상, 공부습관부터 시작해서 "나답다"라고 생각하던 것들이 정말로 나다운 것들인지
아니면 만들어진 것인지 한번이라도 의심해본 적이 있나?
\vspace{5mm}

이런 질문을 진지하게 던져보면 답이 나오기보다도 화나거나 짜증나면서 이 색기 왜 이래 그런 욕설이 튀어나올 것이다.
당연하지. 저 질문은 당사자들의 게슈탈트를 흔들어버리고 이 역시 인격까지 해체시켜버릴 수 있는 위험한 질문이기 때문이다.
'의심해본다' 빼고는 세뇌와 적대하는 것은 아무 것도 없다.
\vspace{5mm}

아무튼 모두가 세뇌당하고 싶어하고 지금도 그러고 있다.
수험생들조차도 실제로는 세뇌당하고 싶어하고 지금도 부지런히 자신들을 세뇌시키는 존재들이다.
\vspace{5mm}

+
\vspace{5mm}

눈치빠른 사람들은 내가 왜 인강이나 교재 추천을 병적으로 싫어하는지 알았을 것이다 -
인강 거의 듣지 않고 교재 양치기를 하라는 이유는 수험도 수험이지만 그래야만 본인들이 직접 고생해서 세뇌 상태를 벗어나기 때문이다.
특정 교재나 특정 강사를 숭배하는 것은 이미 수험을 넘어 그냥 컬트다,.
\vspace{5mm}






\section{[세뇌론 사례 02] 우상화, 동양사상}
\href{https://www.kockoc.com/Apoc/499998}{2015.11.19}

\vspace{5mm}

\href{https://namu.mirror.wiki/w/%EB%82%B4%20%EC%99%B8%EB%AA%A8%EC%97%90%20%EB%B0%98%ED%95%B4%20%ED%98%B8%EA%B8%B0%EC%8B%AC%EC%9C%BC%EB%A1%9C%20%EC%A0%84%ED%99%94%ED%96%88%EB%8B%A4%EA%B0%84%20%ED%81%B0%20%ED%98%B8%ED%86%B5%EC%9D%84%20%EB%93%A4%EC%9D%84%20%EA%B2%83%EC%9D%B4%EC%95%BC}{나무위키 링크}

\vspace{5mm}
\begin{enumerate}
\item  우상화
\vspace{5mm}

북한은 공산주의 국가를 넘어서 개인우상화를 어떻게 해야하나 보여주는 나라다.
같은 민족이라는 이미지에 세뇌되면 통일을 찬성해야할지도 모른다
그러나 저기는 국민 다수가 마약과 성병에 찌들고 주체사상이라는 종교에 세뇌당한 사람들이다.
차라리 북한 정권이야말로 영악하기 때문에 합리적(?)일지 모른다, 무서운 건 그 주민들이다.
\vspace{5mm}

인터넷 까페나 블로그에 가보면 별 것도 아닌 사람들이 얼굴이나 패션 사진을 올리는 경우를 보는데
실소가 나올 수 밖에 없지만 - 10대라면 이해나 가지만 - 곰곰히 생각해보면 저것도 꽤 무서운 장치들이다.
그럼으로써 그 사람의 이미지가 반복학습되면서 어느 사이에 실체화되기 때문이다.
\vspace{5mm}

미운 사람도 자주 보면 정든다... 라고 할지 모르지만 뛰어난 업자들일수로 자기 이미지를 반복학습시키는 게 괜히 그런 게 아닌 듯.
그 이미지가 부정적이더라도 실제로 그 이미지를 학습한 사람은 실물이 등장하면 가슴이 두근두근해질 수 밖에 없고
그럼으로써 평소의 이성을 잃고 비합리적인 판단을 하는 경우가 많아서임.
\vspace{5mm}

남북정상회담 때인가. 그 때 김정일 찬양까페가 생기고
심지어 김정일이 매우 합리적인 지도자인데 우리가 몰라보았다(?)라거나 미국의 음모다 해서 호감(!)을 품은 분위기란 게 있었다.
물론 그 진실이 뭔지는 우리는 알고 있지만 분명한 건, 부정적인 이미지로 알려진 사람에게조차도 대중들은 컬트적 찬양을 한다는 것이다.
최근에 비윤리적인 처신을 했던 방송스타들도 대중들의 그러한 면모를 알고 있기 때문에 더 뻔뻔하게 나서는 것이다.
\vspace{5mm}

개인 우상화를 우습게 보는 경우가 많다.
예를 들어서 당신들은 그럼 3대 종교 교주부터 시작해서 세계의 온갖 위인을 다 비판하고 까발길 각오가 되어있나?
아니, 심지어 자기 가족과 부모조차도 혈연은 혈연이지만 인간 대 인간으로서 비판하고 회의할 수는 있나?
이런 비판이 지나치면 그것이 다른 컬트가 된다.
특정인을 숭상한 것이 바로 중국의 문화대혁명인 것이다.
\vspace{5mm}

국영수를 잘하면 뭐하나. 저런 우상화에도 휘둘리면 어차피 끝인데.
사이비 종교라고 하면 사실 일본의 오움진리교가 거명된다. 오움진리교와 IS의 차이는
전자는 반란에 실패했지만 후자는 성공했었단 것이 아닌가 싶은데
당시 오움진리교에 엘리트들도 많았다(이걸 취재하고 도쿄대를 비판한 사람이 다치바나 다카시)
그런데 우리나라에서조차도 적지않은 사이비 종교들이 있는데 거기에 명문대, 사자 돌림, 심지어 고위 관료들도 있는 건 아나.
\vspace{5mm}

\item   서양 vs 동양
\vspace{5mm}

까놓고 말하자. 서양이 우월하고 동양은 미개하다고 외쳐!
갓양인 만세! 똥송하옵니다.
\vspace{5mm}

.... 라는 건 슬픈 자화상인데 약간은 농담(그럼 나머지는 진담이라고 해야하나)
\vspace{5mm}

진지빨고 말하면 그렇다.
소위 의심하고 회의해본다, 확실한 근거가 없으면 말할 수 없다라는 풍토는 서양에 있었지 \textbf{동양에는 없었다.}
동양에 구장산술 있었잖아욧.... 헛소리다. 동양에 있던 건 산수였지 수학이 아니었다.
산수는 그냥 숫자를 계산하는 것이고, 수학은 말 그대로 모든 걸 의심하고 해체하고 '참'인 것만 골라 이론적 틀을 확고히 한다.
\vspace{5mm}

왜 우리나라에서 수학을 배워야하나. 그거야 당연하지, 수학을 안 배우면 전체가 다 미개해지니까.
그런데 다만 지금과 같은 입시수학이면 이건 좀 곤란한 측면도 없지 않다. 지금의 입시수학이 \textbf{'근대인'을 양성할 수 있나,}
결국 수학사교육에 세뇌받아서 문제 풀이 패턴화에 미친 노예들만 양성한다는 비판은 맞는 소리 아닌가?
\vspace{5mm}

우리나라에서 동양의 신비주의는 전두환 때로 돌아간다. 전두환 정권 시대에 유명한 키워드가 3S.
(군사정권이란 이미지와 달리 문화적으로는 개방되었다고 할 수 있다. 아마 문화로만 보자면 민주적(--)이었을지도 모르겠다)
하지만 3S보다 더 유명한 건 국풍8\textbf{1}이다.
\vspace{5mm}

\href{https://namu.wiki/w/%EA%B5%AD%ED%92%8D81}{나무위키 링크}

\vspace{5mm}

그런데 이걸 계기로 해서 재야에 은거했다는 온갖 동양 신비주의 고수들이 진치기 시작하고
특정 출판사의 온갖 동양사상 상품과 함께 서비스들이 IMF 직격타를 맞기 전 흥청망청 시절에 정말 잘 팔리기 시작한다.
동양의학, 사주팔자, 주역, 풍수, 침술, 요가, 단전호흡 ... 그리고 이런 것들이 성장하면서 나중에는 동호회 차원을 넘어 기업화되기 시작하는데.
\vspace{5mm}

이 당시 철이 없던 나는 저런 책들을 자주 탐독해보아서 알지만,
쟤들이 공통적으로 말하는 게 물질적인 서양사상은 끝났고 정신 위주의 동양사상의 시대가 온다는 것이다.
\vspace{5mm}

물론 그 때는 철없던 시대라 열광했지만 세월이 흐르면서 하나하나 다 검증해볼 수 있었는데
저 중에 맞은 건 한 10개 중에 2~3개? 그런데 이것도 어떻게 보면 대충 말해도 맞는 수준이라고 보아서.
게다가 맞았다 틀렸다가 문제가 아니라, 당시 정신을 강조한다는 분들이 알고보니 돈과 여자에 미쳐있더라는 것은 꼭 하나씩 드러났거니와
사이비 종교화된 곳일수록 도대체 민족정신을 강조한다는 분들께서 그 조직과 경영조차 일본의 오움진리교를 베끼는 행태를 보여주었는지 참.
\vspace{5mm}

여담이지만 저런 동양적인 건 난 완전히 부정하지 않는다.
한의학은 과학적으로 검증될 수 있다고 보고 있으며(물론 상술 차원에서는 문제가 많은 곳도 많다.)
사주팔자는 개인 성향에서는 유의미한 이야기, 주역은 잘만 업그레이드하면 라이프니츠 모나드 이론,
풍수 역시 현대 지리이론 등으로 바꿀 수 있다. 요가는 그 동양 뽕만 빼고 실전적으로 익히면 이보다 좋은 운동은 없거든.
\vspace{5mm}

이거 실제로 문헌조사만 한 게 아니라 그런 종교단체와 부딪친 적도 있다(...) 조직망 정말 쩌는 곳이더구만.
그것도 꽤 좋은 수업이 되지 않았나 싶었고 시간나면 저런 곳들의 뒷배경이나 철학 같은 것이 어디가 원산지이나 분석해보았는데.
대체로 비슷한 패턴을 보인다던 것?
\vspace{5mm}

일단 사이비 종교는 믿을 이유는 없다. 왜냐면 그들의 이야기가 맞으려면, 우선 그런 종교단체는 '하나'여야하지 않나.
우리나라만 해도 사이비 종교가 큰 곳만 추려도 두자리는 되는데, 일본과 중국에도 더럽게 많다.
전세계에 있는 석가모니불의 사리만 다 모으면 코끼리 몇마리분이 나온다던데(석가모니불이 무슨 진격의 거인도 아니고)
이런 이야기나 비슷하지 않나?
\vspace{5mm}

그래도 종교집단이 맞는 건 가끔 무속인들을 불러서 예언을 한다는 건데 이건 신빙성이 있을 수 있지만 사실 문제가 없는 건 아니다.
이 역시 오컬트적인 것인지라 나름 찾아보고 조사해보았다는 것.
개인적으로는 귀신 같은 건 없다고 할 수는 없지만, 과연 우리가 생각하는 것처럼 파워가 세다거나 인간을 말아먹는다... 정도는 아니란 것이다.
무당들이 사람들의 과거를 잘 맞추는 건 콜드 리딩도 있지만 또 다른 것은 일종의 뇌파감지(?) 같은 건데
다음에 있는 모 점술까페 가서 점사 검증보면 알지만 제대로 맞춘 곳은 정말 없다(...)
사실 귀신에 시달리는 빙의증세 보이면 가장 확실한 건 "이사"를 가는 것이다. 터가 나쁘다 무슨 사연이 있다 할 필요없이 떠나면 된다.
아울러 잘 먹고 운동 꾸준히 하면 된다.
\vspace{5mm}

그런데 동양 vs 서양의 문제는 서양의 저런 '의심하는 습관'을 부정해버리는다는 것이다.
우리가 서양학문에서 배워야하는 건 엄격한 논증도 논증이지만 "그래서요?"라고 하면서 무조건 의심하고 확실하지 않으면 버리는 습관이다.
가설이지만 아마 동양사상 장삿꾼들은 이걸 어떻게 해야하나 고민했을지도 모른다.
왜냐면 자기들이 하는 건 제대로 검증된 건 단 하나도 없거든.
서양의 저런 논리를 붕괴시킬 수 없다면 차단이라도 해야 고객들이 늘어나는데 어떻게 해야하나
이 때 호소하게 된 것이 바로 민족주의이고, 또한 일부 기독교에서 깽판치는 것도 매우 좋은 소재였을 것이다.
\vspace{5mm}

이런 서양 동양 대등론은 요즘 와서는 주장되지는 않을 것 같다.
"두유노우김치", "두유노우강남스타일" 같은 유머는 그간의 지나친 국뽕에 대한 자학 개그이기도 하지만
인터넷을 통해 전세계의 기호들을 유입받으면서 우리가 그동안 우물 안 개구리였구나 깨달은 바도 있고
IMF 이후 공중부양이든 열반이든 "금융자본" 앞에서는 깨갱하는구나를 경험하고 나서부터 다들 합리적으로 변해간 덕분이기도 하다.
\vspace{5mm}

\end{enumerate}






\section{[세뇌론 006] 보잘 것 없는 운명관}
\href{https://www.kockoc.com/Apoc/500505}{2015.11.19}

\vspace{5mm}

각자가 생각하는 자기 인생이란 참 보잘 것이 없다.
그건 인간의 삶이 초라해서가 아니다.
\vspace{5mm}

개인이 생각하는 자기 인생의 txt는
소설보다도 매우 빈약하고 초라하기 때문이다.
\vspace{5mm}

소설 = 픽션, 인생 = 논픽션 ... 이니까 픽션이 더 허약하다는 편견이 있지만 실제로는 그렇지 않다.
소설이 읽히기 위해선 치밀해야 하고 설득력을 갖춰야 한다.
그렇기 때문에 많이 읽히는 소설은 호소력이 높은 기호들이 조합되어서 그 나름으로 타당한 결론으로 유도되는 경우가 많다.
일부 망작이 없는 건 아니지만 어지간해서 많이 읽히고 팔리는 작품들은 설득력이라는 게 있다.
\vspace{5mm}

그러나 개인의 인생관은 어떨까.
대단히 허술하다.
콕콕 남자를 예로 들어보자.
명문대 들어간다, 의치한 간다, 그리고 젊고 예쁜 글래머 여자를 만난다.
평범하게(?) 강남 50평 아파트에서 산다...
보통 이런 식으로 대단히 막연한 인생관을 가지고 있다.
본인들은 자기 인생이 소중한다고 생각하지만 실제로는 그렇게까지 치밀하게 계획해놓은 사람은 없다.
이는 거꾸로 말해서 각자가 자기의 인생 게임을 미리 시나리오 짜놓고 소설 수준으로라도 플레이하면 성공할 수 있단 얘기다.
\vspace{5mm}

그렇기 때문에 각자의 운명관이란 사실 상당히 단순한 것이다.
크게는 행복한가 불행한가.
그리고 학업, 취업, 금전, 연애, 질병.....
생각해보면 운명에 들어가는 항목은 별로 없다.
저게 언제 어디서 터질 것이냐 그런 차원까지만 얘기해도 대단하다고 하겠지만
실제 이런 경우 예언가들이 던지는 '말씀'이란 항목분류를 해보면 대부분 한정되어 있다.
\vspace{5mm}

사실 그렇기 때문에 정말 예언이라는 게 과학적인가 아닌가 정말 맞는가 아닌가를 떠나서 쓸모없는 것이다.
그 마법사들의 예언은 대부분 내담자들의 관심사에만 집중되어 있다.
예컨대 작년의 세월호 사건이라거나 올해 있었던 온갖 굵직한 사건들에 대해서 \textbf{누가 한명이라도 제대로 얘기한 적이 있었던가.}
간혹 성지순례라고 하여 "질병이 돈다" = "메르스 유행"이라고 알려지기도 하지만, 이건 우연의 일치에 불과한 경우가 많다.
하차해버린 모 PD의 점술가 검증인가 하는 프로그램 말미에도 거기 출연한 사람의 운명을 얘기한 경우가 있는데
이 경우도 딱히. 왜냐면 후기들을 보면 별로라는 평이 많다.
\vspace{5mm}

이 경우면 그럼 가짜 점쟁이 행세는 가능할 것인가.... 그게 가능하다는 게 문제다.
왜냐면 실제로 미래는 미래 일이기 때문에 당장 닥치지 않으므로 어떤 말이든 던질 수가 있는데,
문제는 내담자가 그 말에 사로잡히는 경우 예언의 자기실현이 가능해져버린다는 것이다.
실제로 너는 시험을 망칠 거야... 라는 예언을 듣는다고 하면 그 경우 망칠 가능성이 크다.
만약 예언을 듣고 가볍게 어, 그냥 극복해보일 거야하는 사람은 사실 별로 없고 그런 사람이라면 그런 예언도 걍 흘려듣거나
생산적인 쪽으로 삼는다. 그럼 더 열심히 공부해서 합격해야겠네.
하지만 예언에 귀 기울이는 사람들은 유리멘탈인 경우가 많다.
그렇기 때문에 시험을 망칠 거라는 예언을 들으면 \textbf{정말로 '망쳐야 한다'는 것을 자기 운명으로 받아들이는 게 문제}다.
그래서 실패하면 그걸 또 자기 운명이라고 생각하면서 지배되는 것이다.
\vspace{5mm}

그래서 망한다고 했는데 정작 잘 보면?
그 때야 "운이 좋았나보다", "내가 기도해서 그런 겨", "부적의 효과가 좋았지"라고 덕담(?)하면 되는 것이고
만약 이 경우 따지면 "잘되었는데 왜 야단이냐"라고 대꾸해버리지 않나.
\vspace{5mm}

오해살 것이 아닌게 난 이 분야를 굉장히 좋아한다. 그리고 빙의라는 것도 일정 부분은 그런 현상이 있다 보며
그 비슷한 것을 체험했다.
그러나 관심이 많다고 체험했다고 해서 이 모든 걸 믿는다면 그게 사람같이 사는 건가.
인류가 기존의 통념에 지배당했다면 지금도 길가는 육식동물에게 잡혀먹히고 있는 신세였을 것이다.
인류의 진보는 '부정'에서 시작된 것이다. 의식을 갖춤으로서 자연과 분리되고, 의심함으로써 종교와 분리되면서 지금 상태에 올 수 있는 것이다.
운명이 있든 없든 귀신이 있든 없든 이런 걸 무조건 신봉하지 말고 회의하고 따져보아야 진보가 있지 그게 없으면 걍 '노예'다.
\vspace{5mm}

한가지 예만 들어보자면 잠을 잘 때 가위눌림을 자주 겪곤 했다. 소위 루시드 드림.
이것도 정말 영적인가 아닌가라는 건 지금도 해답을 못 내리지만 중요한 건 그게 아니라 일단 짜증나는 경험이 문제가 아닌가.
어렴풋이 잠을 들면 환청과 환각이 보일 때는 기분이 뒤숭숭해진다.
그런데 1년간은 그게 보이지 않는다. 비결은 간단했다 - 온갖 잡것들이 등장하자 디씨에서 보았던 무시무시한 형님(누군지 말하지 마라)를
그 꿈 속에서 소환시켜보았다. 그리고 ~ 형님 믿습니다라고 외치니까 거짓말아니고 그 모든 환청과 환각이 사라져버렸다(...)
중학생 때부터 그런 것들을 겪을 때마다 예수님 공자님 부처님 맹자님 외쳐도 소용없던 것이
디씨 합필갤에서 우스꽝스럽게 만든 짤방과 음원을 등장시켜버리니까 소리없이 사라진 것이다.
그럼 이것이 그냥 정신작용에 불과했는지 진짜 잡귀들이 등장했나 그건 모르겠지만
이 경험으로 치면 나는 예식장 사업하다가 말아드시고 흉악한 범죄를 저지른 그 형님을 믿는 종교의 신자가 되어야할까?
\vspace{5mm}

운명이란 것이 뭔지 회의해보자면 정말 비참해진다.
우리는 사실 우리 인생에 대해서 명백한 주관식 논술답안을 작성하고 있는 상태가 아니다.
1번 밖에 못 산다는 점에서 치자면 수백장은 나와야하건만 실제로는 자소서 쓰라고 해도 한장도 채울 수 있을까.
SNS 상에서야 셀카 올리고 나 예뻐~ 그런 거나 잘하지 실제로 우리는 소설주인공만도 못 한 삶을 살고 있는 것이다.
그리고 이렇게 운명이라는 것도 애매모호한데, 재밌는 건 운명관이라는 건 더럽게 잘 믿는다는 것이다.
그럼 왜 그런 운명관을 믿게 될까?
그거야 간단하지. 뇌가 그렇게 하고 싶으니까.
그럼 왜 뇌는 운명관에 의존할까? 그거야 \textbf{생각하기 귀찮으니까} 그렇지.
\vspace{5mm}

정작 10대 때를 돌이켜보면 그 때 누구도 스마트폰을 끼고 다니는 환자가 있을 거라 예측한 경우는 없다.
사실 그 당시에도 미래의 과학기술이라고 해서 2010년대의 외계인 같은 삶에 대한 픽션이 있었는데
정보화의 경우는 오히려 그 픽션보다 지금이 더 진보했다. 설마 컴퓨터를 손바닥에 넣고 다닐 거라는 얘기까진 나오지 않았지.
다만 로봇이 일을 대신 해준다.... 이건 너무 섣부른 예측이었던 것이다.
그리고 무엇보다 온라인 게임. 이 예측은 들어본 적도 없다.
\vspace{5mm}

우리가 무슨 시인이라도 되는 양 운명을 부르짖을 때 저런 기술의 진보나 생활의 개선은 왜 하나도 언급하지 않는 걸까.
그 점에서 우리의 인생, 운명관이라는 건 정말 조잡한 기호들의 나열이라는 것이 더더욱 분명해진다.
우리가 한달 뒤에 원양어선을 탈지, 대국가적 재난에 처해있을지, 아니면 강도와 인질극을 벌일지, 병원에 누워있을지.
이거 정확히 알 수 있나?
\vspace{5mm}

그러나 돌이켜보면 중요한 사건이란 늘 예기치않게 찾아온다.
전혀 찾아올 거라고 여기지 않은 사건일수록 그렇다.
처음에는 황망해서 그럴 리 없다고 하지만 인간은 또한 적응의 동물이라서 수시간이 지나면 그 사건에 순응해버린다.
가령 올해 수능만 하도 그렇지 않나? 영어 쉽게 나올 거라고 다들 믿고 어렵게 나올 리 없어라고 했지만 결과는 어땠더라.
수능 치고 물수능 얘기 쏘옥 들어가고 다시 불수능 이야기 오가는 거, 이게 천박하고 참 박약한 인간들의 지성이란 것이다.
그래도 복기는 해야지, \textbf{왜 과거에는 그럼 '잘못된 예측'에 빠졌던 걸까 하는 것.}
\vspace{5mm}

하지만 더욱 답답한 건 한번 세뇌된 사람들은 그런 현실을 겪고도 여전히 정신을 못 차린다는 것이다.
A를 믿는 사람이 있다. A가 B라는 결과를 내놓는다, 그런데 현실은 C다. 그래도 그 사람은 A를 믿는다.
왠지 아나? 그렇게 학습해버렸기 때문이지. 그리고 거기서 벗어나긴 정말로 힘들다.
우선 세뇌라는 게 그렇게 만만한 게 아니다. 무의식에다가 말초신경까지 다 지배당한 상태다.
결과가 틀렸기 때문에 A를 까기보다는, A의 예언이 빗나간 건 다 그런 이유가 있다고 생각하거나
아니면 C도 그림 살짝 그리면 B가 된다는 식으로 과거 황우석 사태와 같은 인지부조화 증세를 드러낸다.
시리즈에서 논하겠지만 이것이 바로 '앵커링'의 결과이다.
몸 속 깊숙히 박힌 갈고리침을 제거하는 것만큼이나 매우 고통스럽고 힘겨운 일이다.
아프니까 걍 박고 살테니 그냥 건드리지 말라는 게 거부반응은 너무다도 당연한 것이다.
그리고 제거한다고 쳐도 다시 박으려고 하는 경우도 있다. 앵커링이 한번 되면 이걸 빼는 것은 정말 어렵다.
1+1=2가 아니라 1+1=-10이라고 얘기하면서 이걸로 수학을 풀라고 하면 풀겠는가?
\vspace{5mm}

마법사들은 인간의 심리가 이렇게 초라한 걸 알고 있다.
그래서 그들은 그걸 이용해 돈을 번다.
여성 심리를 대상으로 한 것이 바로 픽업아티스트들이고
물욕에 관한 심리를 이용한 것이 바로 피라미드 업체들이다.
자기가 대단한 줄 알지만 그 내용물이 보잘 것 없는 호구들을 구워삶는 건 간단하다.
적당히 띄워주고, 자기가 그 내용물을 채워주고(세뇌), 그리고 적절할 때 꾸짖고 훈육하고, 다시 달래는 것이다.
생각보다 인간을 세뇌시키는 건 그리 어렵지 않다.
\vspace{5mm}

석가모니가 이런 말씀을 하셨던가. 인생은 호흡지간이라고. 즉, 숨을 쉴 때 그 순간 뿐이라고.
사실 이것만큼 정확한 이야기는 없다. 석가모니가 원래 수학을 잘 했다는 설화들이 많은데(게다가 '인도의 왕자님' 아닌가. 인도수학은 뭐)
인생은 호흡지간이라는 건, 즉 인생의 순간변화율이고, 우리는 여기서 석가모니가 인생을 '미분할 줄 알았구나'라는 걸 알 수 있지만... 절반은 농담.
아무튼 가장 좋은 건 운명관이라는 것도 결국 만들어진 것이고, 이건 우리가 어떤 언어를 집어넣느냐에 따라 바뀔 수 있다는 걸 아는 것이다.
한 때 유행했던 시크릿인가... 라는 책이 이걸 겨냥했던 것이고, 최면기법에서 파생된 NLP 역시 이걸 전제로 하는 것이다.
\vspace{5mm}

궁금한 사람은 일주일동안 일기를 적으면서 자기와 영 상관없어보이는 분야에서 자기가 어떻게 성공, 실패하나 소설을 써보시길.
지금 입시를 시작하는 사람은 수험문학이라고 해서 내년동안 겪을 온갖 사건들을 미리 가정해서 자기가 어떻게 공부하고 실패해나갈 건지
개연성있게 적어보면 된다. 그리고 놀랍지도 않지만 그거 대부분 적중할 것이다.
일단 이건 자기가 자기 스스로에 하는 예언이기도 하지만, 우리의 무의식은 말을 안 해서 그렇지 주인이 어떤 인간이지 잘 알고 있다.
\vspace{5mm}






\section{[세뇌론 007] 원론}
\href{https://www.kockoc.com/Apoc/501415}{2015.11.19}

\vspace{5mm}

1-1
\vspace{5mm}

세뇌된 상태는
주관적으로는 넋을 잃고 \textbf{몽상공간을 혼이 떠다니는 것과 같은 상태다}.
동시에
객관적으로는 \textbf{치밀하게 계산된 허구의 세계에 감금된 상태다.}
\vspace{5mm}

이런 상태에 완전히 빠진 사람은 밤낮은 가리지 앟고 힘든 일도 마다하지 않는다.
또한 교주의 무리한 명령에도 아무 생각 없이 절대 복종하는 가하면
자기 자신을 완전히 버리고서 절대자의 의미에 분리되는 것처럼 보인다.
그리고 이 몰두의 비결은 행위 자체가 본인의 육체적, 정신적 Ecstacy에 연결된 것에 있다.
\vspace{5mm}

이는 매우 위험한 상태이지만 동시에 수험생이 달성해야 할 상태란 점에서 부정할 수만은 없다.
저기서 세뇌를 '집중'이라고 하고 '허구'를 공부할 텍스트라고 하며 '교주'를 선생님으로 고치면 완벽한 입시의 경지이기 때문이다.
정말 공부에 미친 놈들은 10시간 이상 공부해도 고통을 못 느끼는데 이것 역시 Ecstacy라는 점이 그렇다.
\vspace{5mm}

여기서 중요한 건 세뇌된 사람의 정신은 '현실과 분리된 가상공간'에서 노닐고 있다는 것이다.
마찬가지로 공부에 미친 사람들은 공부한 텍스트들로 구축된 학습공간에서 노닐고 있다(즉, 이 경지까지 공부해야한다)
\vspace{5mm}

이렇게 현실과 분리된 가상의 공간으로 인생의 무대를 옮기면
누구라도 생각이 정지되고 끝없는 쾌락을 좆아 달려가게 된다.
여기서 캐락은 굳이 성적 흥분, 권력 놀이, 금전욕, 혹은 나르시시즘일 필요가 없다.
쾌락을 느끼는 방법은 단 하나 - \textbf{논리적인 사고의 정지}이다.
\vspace{5mm}

바꿔 말해서 세뇌된 것과 유사한 상태의 학습몰입은 이런 점이 위험하다.
적지 않은 인강을 들어대면서 트랜스 상태에 빠져 고득점이 나오는 학생들이 수능 시험 당일에 맛가는 이유.
본인은 문제를 많이 풀어서 어떤 것이든 척척 풀어내는 패턴에 세뇌된 상태로 쾌감을 느끼지만,
그 쾌감의 원천은 '논리적인 사고의 정지'를 전제한 것이고, 따라서 논리적으로 접근할 문제가 나오면 풀지 못 하게 된다.
\vspace{5mm}

1-2
\vspace{5mm}

마법사들은 자기들의 호구들이 자기를 위해 봉사하는 것에 엑스터시를 느끼도록 바꿔치기할 수 있다.
바꿔 말해 위험한 사교육강사라면 학생들이 공부에 엑스터시를 느끼는 걸 넘어, 강사 자신을 위해 공부하도록 기호조작을 할 수 있단 얘기다.
아닌게 아니라 수험사이트들을 돌아다녀보면 그런 술수가 읽히는 경우가 많다(이걸 아는 사람은 별로 없을 것이다)
\vspace{5mm}

이런 마법에 걸려든 사람들은 자기 존재 전부를 걸고서라도 세뇌하는 사람, 즉 마법사를 위해 진력을 다 한다.
자기 희생, 금욕, 수행, 고차원의 학습 등 본래대로라면 고통을 수반하는 일조차 '고상하고 숭고한 행위'로 형상화된다.
즉, 마법사들은 간단한 조작만으로도 피세뇌자들에게 '유사' 니르바나 상태를 만들어대는 것이다.
\vspace{5mm}

이 점에서 인강이 문제될 수가 있다. 인강은 그 자체가 강사가 원하든 원하지 않든 보급형 세뇌나 다름없기 때문이다.
혼자 책을 읽는 건 불편하지만 인강은 불편하지 않은 이유는 무엇일까.
혼자 읽을 때는 논리적으로 사고를 해야한다, 그래서 쾌감이 끊기게 된다.
반면 인강을 들을 때는 사고할 필요가 없다. 낭랑한 가성 목소리에만 귀기울이고 생각이란 걸 안 해도 된다. 그러니 쾌감을 느낀다.
\vspace{5mm}

물론 이런 걸 악용하는 강사는 별로 없을 것이다. 그러나 본인 의지든 아니든 그 인강의 인터페이스 자체가 지닌 문제가 있다.
인강을 잘 듣는 방법은 간단하다. "생각을 멈추고 오로지 강사의 말에만 귀기울이는 몽롱한 상태에 빠지면 된다"
논리적 사고를 하며 듣는 순간 인강은 매우 불편해지고 번거로워진다.
하지만 생각을 하지 않고 강사시키는대로 하게 되면 인강만큼 편리하고 쾌락을 주는 건 없다.
\vspace{5mm}

1-3
\vspace{5mm}

학습이란 결국 내가 나를 세뇌하느냐, 아니면 남에게 세뇌당하느냐
이걸로 나뉜다는 것이 지금 드는 생각이다.
작년 말에 온갖 음해에 시달렸습니다(...)와 관계없이 인강을 줄이고 기본교재를 철저히 하라고 했고
그래서 온갖 수험사이트에서 온갖 인강을 다 들은 경우보다 성적이 좋은 경우(물론 반대도 있다)는
오히려 그동안의 인강이 독이 된 케이스다.
\vspace{5mm}

인강이 안 맞는 사람들은 사실 잠재력이 매우 좋을 수도 있다.
학습의욕이 넘치지만 본인의 읽기, 사고하는 속도가 남들과 다르거나
무엇보다 본인이 논리적인 사고를 하는 게 익숙한 사람들은 인강이 안 맞을 수 있다.
이런 사람들은 논리적으로 설명된 개념서를 본인 페이스로 읽고 문풀만 신나게 하다가 고난이도를 정리해보는 게 낫다.
\vspace{5mm}

그럼 반대로 인강을 들으면 안 되나? 그건 아니다.
본인이 공부할 수 있는 상태가 아니면 타율적인 세뇌가 약이 되는 케이스가 있다.
상담 게시물에서 내가 EBS만 따라가라고 한 케이스가 그런 케이스다(근거없이 이래라저재라하지는 않는다는 것)
본인이 사고력을 더 개선해야한다거나 혼자서 공부할 수 없으면, 다른 시스템에 올라타거나 혹은 신탁통치를 받는 것도 한 방법이다.
\vspace{5mm}

물론 어느 쪽이든 최종적으로는 자기가 정리해야한다는 것만큼은 변함이 없을 것이다.
\vspace{5mm}






\section{[세뇌론 008] 간단한 테크닉}
\href{https://www.kockoc.com/Apoc/505389}{2015.11.22}

\vspace{5mm}

\href{https://namu.wiki/w/%EC%84%B8%EB%87%8C}{나무위키 링크}

윗 소개대로 세뇌의 시초는   중국이 6.25 전쟁 포로를 상대로 사용한 심리적 테크닉을   에드워드 헌터라는 저너리스트가 Brain Washing이라고 소개한 것에서 유래한다.
세뇌는 우리의 신경 수준에서 정보, 신호를 처리하는 단계에서
약간의 개입적인 조작을 보탬으로써
그 사람의 생각, 행동, 감정을 마음대로 통제하려 하는 것이다.
\vspace{5mm}

$\#$ 의식하는 수준과 무의식 수준의 차이
\vspace{5mm}

크게 잡아서 세뇌의 테크닉은 2가지이다.
하나는 피세뇌자가 그걸 의식하는 데 하는 것, 다른 하나는 피세뇌자가 의식하지 못할 때 하는 것이다.
상대방이 세뇌를 의식하는 경우는 심리적 저항이 따른다. 본인은 싫어한다.
그런데 이도 강제로 따라오게 하는 기술이 있다.
첫째는 독방에 감금하기
둘째는 일종의 사회적 역할을 부여하면서 주종관계를 강요해 고정시켜버리는 것이다.
\vspace{5mm}

본인이 저항하더라도 강제력을 행사해서 어느 순간에 상대방을 컨트롤해버리는 방식인데
이는 수험생들도 자기도 모르게 당하고 있는 세뇌다 - 그건 바로 학교와 학원.
이에 관해서는 스탠퍼드 교도소 실험 참조
\href{https://namu.wiki/w/%EC%8A%A4%ED%83%A0%ED%8D%BC%EB%93%9C%20%EA%B5%90%EB%8F%84%EC%86%8C%20%EC%8B%A4%ED%97%98}{나무위키 링크}
\vspace{5mm}

감투는 효율적인 세뇌방법 중 하나라는 것을 알 수 있다.
\vspace{5mm}

한편 무의식하의 세뇌는 상대방에게 심어놓고자 하는 명제를 '숨기면' 된다.
예를 들어 "님들을 부자로 만들기 위해 제가 책을 팔겠습니다"라고 한다.
그 다음 그 교재에 링크된 주소를 통해 들어온 사이트에 저자의 온갖 이미지를 도배하면서 긍정적인 평가를 유도한다.
그렇게되면 그 교재를 보고 들어와 저자의 이미지를 평가하는 사람들은 그 저자를 '숭배'하는 예속적인 태도를 취하게 된다.
\vspace{5mm}

이 경우는 당연히 독자들을 자기의 신도로 만드는 게 목표지만, 처음부터 "교재평가를 바란다"라고 구라를 까야 설명할 수 있다.
이와 같은 수법은 자기계발 세미나, 수험 세미나, 그리고 여러가지 상담에서 이용되고 있다.
사실 수험계에도 흔한 바가 아닌가 싶기도 한데
이 경우 콕콕이 안전한(?) 이유가 이거다. 우선 허혁재군은 아무리 미화시키더라도 세뇌가 불가능한.... 에 해당하기 때문이다.
\vspace{5mm}

개인적으로 인강과 교재를 대단히 비판적으로 보고
심지어 선량한 목적이더라도 그러한 게시물에 대해서 공격적인 이유는 간단하다.
\vspace{5mm}

그렇다 글이나 소통이 저 두번째의 목적에 이용되는 경우가 많아서이다.
대체로 성인들의 경우라고 하더라도 - 여성들의 경우 - 외모가 반반하고 잘 빠진 남자 주인공에 홀라당 넘어가는 경우가 많지만
수험만 아는 학생들의 경우는 - 특히 독서를 안 해서 문과적 소양이 부족한 지금은 더욱 심각하다 - 그냥 대놓고 잡수세요... 케이스가 많다.
\vspace{5mm}

실제로 상담이란 것을 해보면 도저히 보면 안 되는 교재인데도 왜 샀느냐 물어보면 이유를 대답 못 한다.
A 교재나 B 인강을 안 보면 망한다... 이렇게 메시지를 반복하는데 그럼 풀어보았느냐하면 안 풀어본 경우가 많다.
이 경우 왜 그런가 추적해보면, 그 학생이 하라는 공부는 안 하고 결국 교묘한 세뇌전략에 brain-washed 되어버린 케이스다.
여기서 난감한 건 그 세뇌된 메시지를 고발하고 자네가 세뇌당햇다네... 라고 하는 건 '아 고맙습니다'란 반응보다는
'뭐여 지금 시방 나를 욕하는 겨'라고 공격적으로 나오기 좋다는 것이다.
\vspace{5mm}

세뇌는 단지 메시지의 주입으로 끝나지 않는다. 피세뇌자가 세뇌자와 동일시되거나 그 종을 자처하는 수준으로 간다는 것이다.
나중에는 결국 자기 입시를 위해서가 아니라, 그 교재를 푸는 것을 목적으로 하는 황당한 케이스까지 보는데 이거 설명할 게 따로 있나?
\vspace{5mm}

이런 경우는 두가지다. 피상담자를 걍 무시하기
아니면 가망이 있는 경우 믿지 못 하는 피상담자에게 커리 추천해주면서 반드시 밟아보아라하면서
치고박고하는데 그런 과정을 거치다보면 피상담자가 자기가 무엇에인가 사로잡혔다 느끼면서 정신차리기 시작한다는 것이다.
\vspace{5mm}

이런 경험을 한 사람으로서는 교재추천을 하려고 하는 모든 시도에 대해선 삐딱한 시선을 안 보낼 수 없다.
이 세뇌가 결국 해당 수험생의 인생을 처절하게 말아먹는다는 것까지 생각하면 그런 경우.
게다가 나는 이런 걸로 이해관계를 딱히 추구하지 않는다(물론 세뇌된 사례들을 탈세뇌시키는 것은 매우 좋은 인생의 경험이 된다)
\vspace{5mm}

적어도 수험시장에 있어서는 누구든 일단 까고보아야지 절대 신뢰해서는 안 된다. 그건 나에 대해서도 마찬가지이다.
강사든 교재든 자기들이 돈 주고 이용하는 건데 왜 xx님 하면서 숭배해야 하나?
말하지만 업자가 보여줄 건 하루종일 공부하고 연구하느라 거무튀튀해진 피부에다가 홀라당 벗겨진 대머리
그리고 피로감을 호소하는 것이지, 그 정반대의 '교주적'인 이미지가 아니다.
\vspace{5mm}

수험생은 돈을 지불한만큼 양질의 정보와 더 좋은 문제를 훈련할 기회만 가지면 되는 거다.
그런데 내가 보는 입시판은 \textbf{"xxx님을 따르지 않으며 망해"} 라는 사이비 종교 메시지가 횡행하고 있다.
\vspace{5mm}








\section{[세뇌론 사례 03] 긍정적인 활용 : 일지}
\href{https://www.kockoc.com/Apoc/505768}{2015.11.23}

\vspace{5mm}

상대방이 세뇌를 의식하는 세뇌는 긍정적으로 쓰일 수 있는 방법이 있다.
\vspace{5mm}

가령 수학을 못 하는 A군이 있다 치자,
이 친구가 수학을 못 하는 건 다른 건 다 좋은데 문제를 해석할 때 잘못 해석해서 그렇다.
\vspace{5mm}

이 경우 다른 사람들은 "야, A. 너는 왜 이렇게 문제를 못 읽니"라고만 탓할 것이다.
하지만 나는 미소를 흘리면서 그 A에게 "오늘부터 콕콕에 수학문제를 올려주고 쉽게 풀어주는 선생님은 님입니다"라고 역할을 던져줄 것이다.
처음에 A는 장난하냐고 하겠지만 내 말을 들으면 네 머리칼은 다 내 것이라고 하면서 말 안 들으면 온갖 저주를 가할 거라고 얘기한다.
\vspace{5mm}

투덜대던 A군은 내가 시키니까 억지로라도 수학문제를 해설하는 RPG를 하는데...
\vspace{5mm}

에이, 저거 가정에 불과한 것 아니에요?
저거 이미 \textbf{작년말에 시작해서} \textbf{올해 성공 사례 만들었는 걸요.}
굳이 해설할 필요는 없다고 보인다.
일지가 뭔 효용이 있느냐 하는 사람은 세뇌를 공부하지 않은 케이스다.
상원 선발과 역할 부여도 마찬가지이다.
물론 전부가 입시에 성공한 건 아니다, 그러나 마인드 면에서 변화를 주었기 때문에 내 경우는 매우 긍정적으로 생각하고 있다.
\vspace{5mm}

한편으로 나에게 일부러 상담을 요청하는 경우도 있다.
이건 답변 자체보다도 '저 사람에게 상담받으면 뭔가 된다'라고 자가세뇌하는 경우다.
불안하기 때문에 그러는 건 이해가 가지만, 이 경우는 난 무조건 매몰차게 대한다.
내가 장삿꾼이면 일일히 답변해주었을 것이다. 그래야 내가 돈을 벌기 때문에
그러나 내가 원하는 건 수험생 개개인이 자기 힘으로 극복하는 경우다.
위의 RPG는 자기 힘이다, 게다가 저 RPG에 넘어간 사람들은 \textbf{나와 맞장뜨고 놀고 있다(....)}
그러니 나에게 무의식적 세뇌를 당하진 않는다.
\vspace{5mm}

하지만 나에게 질문해서 뭔가 의존하고만 싶어하는 사람은 무의식적 세뇌를 받고 싶어하는 케이스다.
내가 수험가 장삿꾼들을 경멸하는 이유가 그거다. 실력이 아니라 무의식적 세뇌 - 교주로서의 추앙을 받길 원한다.
이런 케이스는 결국 자기마저 파멸시키는 것이다.
\vspace{5mm}

콕콕에서 일지와 칼럼을 쓰는 건 단순한 기록으로만 생각하면 무의미해보일 수도 있다.
반대로
일지를 쓰면서 공부한 기록을 알리는 나,
칼럼을 쓰면서 수험방법론을 개척하고 알리는 나.
라고 역할을 부여한다면 그 때부터는 "일지와 칼럼을 위해서 공부하지 않을 수 없는" 내가 된다.
\vspace{5mm}

한편으로 인강을 찾는 서글픈 이유도 여기에 있다.
수험생은 자기 역할이 없다고 믿고 있다.
\vspace{5mm}

\textbf{xxx 선생의 인강을 듣는 다거나 xxx 저자의 책을 소비함으로써}
\textbf{xxx 선생이나 xxx 저자 왕국의 일원이나 서포터가 된다는 것으로써 쿠크다스같은 자아에 포장지를 입히는 것이다.}
\textbf{인강이나 교재가 좋아서라기보다는 자기가 기대하고 지탱할 수 있는 집단을 찾고, 거기서 역할을 부여받고 싶어서 그런 것이다.}
\vspace{5mm}

올해 내가 방법론을 제시해서 성공시킨 케이스가 과연 순수히 독학을 했기 때문'만' 성공한 건 아니다.
이들은 자기 역할을 부여받았기 때문에, 그리고 콕콕을 통해 좌표를 확인함으로써 흔들리지 않고 자기 공부를 할 수 있었던 것이다.
이들이 나에게 고마움을 표하는 건 기쁜 일이지만 한편으로는 매우 경계하는 일이다.
내가 써먹고자하는 세뇌라는 건 '자기 자신이 스스로 일어나는 방법'으로서 결국 보조기를 떼어버려야하기 때문이다.
\vspace{5mm}

그리고 이런 식의 긍정적인 세뇌가 먹힐 수 있다는 건, 훨씬 전에도 검증해보았지만 이번에 다시 검증해보고 틀림이 없다...
물론 이런 것을 더 어떻게 써먹을 것인가하는 건 허혁재군 등이 더 생각해보아야하는 것이다.
\vspace{5mm}

방법론 측면에서는 이제 여기 일반회원들끼리도 정보를 교환할 수 있다.
(그렇기 때문에 난 교재추천해달라하는 질문은 화를 내고 볼 수 밖에 없다. 그건 '교조화'를 부추기기 때문이다)
\vspace{5mm}

내가 오히려 신경쓰고 싶은 건 자아가 약한 수험생들이 자기 위치를 어떻게 부여받고, 어느 지점을 향해 운동할 것인지에 대한 진로고민이다.
만약 +1수를 하고 싶다거나 공무원 시험을 쳐야 한다거나 하는 건 기쁘게 소통해볼 수 있다.
이건 내가 악랄한 마음을 먹더라도 세뇌시키긴 힘든 분야이기 때문이다.
\vspace{5mm}









\section{[세뇌론 사례 04] 나르시시즘}
\href{https://www.kockoc.com/Apoc/507013}{2015.11.23}

\vspace{5mm}

이게 가장 위험한 케이스 중 하나다.
나르시시즘에 빠진 사람은 그 사람이 성공한 사람이건 실패한 사람이건 걍 멀리하는 게 좋다.
\textbf{어느 쪽이건  망하거든.}
여기서 주의 :  여자의 나르시시즘은 나르시시즘으로 치진 않는다.   여자가 자기 미모를 빛내는 건 대단히 실용적이기 때문이다.   그러나 남자의 나르시시즘은      그게 얼굴이든, 패션이든, 목소리든, 정신상태든, 자기자랑이든 내가 보기엔 걍 아주 망하려고 작정한 것 같다.   블로그든 뭐든 자기 얼굴 사진을 상습적으로 올리는 사람들은 개인적으로는 사실상 사이코로 분류했다. 이런 사람은 내가 걍 피해 간다.   일단 남자가 자기 얼굴 보고 히히힛거리는 건 아무리 보아도 제정신이 아닌 것 같아서.   그런데 이 나르시시즘의 극단이 요즘에는 줄었을지 모르는데   서울대 - 고시 하다가 망한 케이스에서 정말 지겹게 보였던 걸로 기억한다.   이 양반들이 공부 잘 한다고 주변에서 칭찬해서 그 맛으로 서울대까지 왔는데   고시하다가 시험이 안 되니까 우울증에 빠지고 자존심이 무너지고 주변에서도 호응을 안 해주니까.   \textbf{그걸 메꾼다고 자기 외모에 치중하거나 아니면 철학자라도 되는 양 인생과 세상을 논하다가 나중에 정말 미쳐버린다(....)}   그런 상태에서 공부가 될 리는... 있겠냐. 그렇게해서 폐인이 되었다가 고향에 내려가거나 노가다 뛰러 가는 거지.   어제인가 모 분이 자기 고찰을 한다고 햇는데 내가 그거 말렸다.   간단하지, 그 모 분이 그런 글쓰기에 맛들이는 순간 저 장수생 코스 딱 밟는 것이거든.   장수생들의 공통점이 뭔지 아나? 공부방향이 딱히 틀린 것도 아님, 노력도 상당히 많이 했음, 그리고 수험 모르는 게 없어.   그런데도 시험이 안 되지.    그런데 대화하다보면 "자기가 어떤 사람인지" 궁금하지도 않는데 정말 강조해대는 것은 아나?   그거 본인이야 모르지. 그런데 주변 사람들은 듣기에는 왜 자꾸만 자기를 노출하려고 하나 그 생각이 든다 그거지.   자기를 강조하고 사랑하게(...) 되면 가장 위험함. 그 때부터는 공부를 해도 머리에 안 들어가거든.   자기가 참 보잘 것 업고 병신 같다는 걸 깨닫고 그 환경에서 벗어나려고 아둥바둥해야 공부를 하는 건데   자기를 관찰하고 사랑하게 되면 그 환경에서 벗어날 생각을 안 하게 되거든.   게다가 나중에 자기 방법이 다 옳다고 착각하면서 균형감각을 잊어버림.   그런데 나 혼자 보면 외로우니까 남들에게 계속 자기를 드러내려 함(...)    처음에는 주변 사람들도 호응해주지. 그런데 나중에는 슬슬 멀리한다, 아니 고추달린 남자가 저게 뭐하는 짓이냐 생각도 있지만.   무엇보다 나르시시즘에 빠진 사람, 수년 째 계속 자기가 잘 생겼냐 얼마나 생각이 깊냐 너희들이 나만큼 고민해보았냐 이딴 소리나 하니까.   또 쫓겨날지 모르지만 혀혁재군도 살짝 디스해보자. 오수썰도 나르시시즘이 과연 없을 것 같나?   내가 이 글을 읽으면서 장점과 단점을 다 추려보면서 고찰해보았다. 어찌되었든 대학은 갔으니까 해피엔딩(?)인 것이지만   오수썰을 딱 읽으면서 느낀 게 인간성이 괜찮은 남자조차 장수를 하면 \textbf{나르시시즘을 필연적으로 겪는 건 어쩔 수 밖에 없다란} 생각이었다.   나르시시즘에 빠지면 이게 처음에는 좋을 것 같지.... 나중에는 진짜 병신같이 되어버린다.     이거 치유하려면 그냥 식민지배당하면서 근대화... 그러니까 타율적으로 공부해서 합격할 수 밖에 없다.   그리고 정상루트로 돌아가봐야 자기가 당시에 얼마나 병신같이 살았나 깨닫고 고쳐갈 수 밖에 없다.   물론.... 성공하고 나서도 나르시시즘을 더 발전시킨 답없는 케이스도 있다만 그건 특수한 케이스다.   N수생 이상 대화를 하다보면 남자 녀석들은 예외없이 나르시시즘 끼를 보인다.    그러나 경악스런 사실은 이거 \textbf{여학생들은 없다(...)}. 여자들이 나 예뻐 어쩌구하는 게 살아남기 위한 실전적인 진화임을 알 수 있다.   이 나르시시즘 증세는 동물 세계의 짝짓기가 금지된 문명 세계와는 맞지않는 도태의 증거가 아닐까.   여학생들이 상담하면 그 변덕 때문에 까다로울 것 같은데 실제로는 아니다(...)   나르시시즘 그런 걸 겪지 않음, 캐리만 잘 해주면 일단 자기를 봐준다 신용해준다는 걸로 정말 잘 따라온다.   그런데 남자녀석들은 이거 안 먹힘, 중중 나르시시즘 걸리면 충고해도 안 들어먹음.   남자들 나르시시즘은 그거 백해무익한 거니까 걍 버리는 게 좋음.    서울대 출신 고시생들이 인생 날려먹은 게 이 때문이라니까.    남자가 군대, 직장 등의 조직생활하면 철든다는 얘기 왜 듣는지 아나? 조직에서는 나르시시즘을 허용해주지 않으니까.   조직에서는 남자 개인의 인격을 억제해버니까 나르시시즘이 방지되어서 역설적으로 긍정적인 결과가 나오는 것이다.   여풍의 원인은 여러가지가 돋보이지만 이것도 한몫할 거다 레알.








\section{[세뇌론 사례 05] 언행일치}
\href{https://www.kockoc.com/Apoc/509842}{2015.11.25}

\vspace{5mm}

"xx니까 xx이다"라는 게 한동안 유행했다. 안 좋은 의미로 -
정작 그런 구호를 외친 사람이 정작 xx본 적이 있는가.
청년들보고 고생하라고 하는 사람이 그럼 자기도 고생을 하느냐.
\vspace{5mm}

자기 자식을 미국 국적주고 카투사 보낸 사람이 반미 외친다면 위선자라고 불러야한다.
외제차 몰고다니면서 부유하게 사는 사람들이 청춘들의 희망을 외친다면 강아지라고 얘기해줘야한다.
물론 메시지만 보면 되지 왜 메신저를 보냐고 공격해대는 한심한 자들도 있다. 이런 사람들은 끝까지 속을 것이다.
\vspace{5mm}

휴머니즘적인 글은 솔까 얼마든지 \textbf{주작}해낼 수가 있는 시대다.
그리고 무엇보다, 청년층을 속이고 착취하는 자들이 인생의 희망을 외치면서 거짓말하는 것, 지겹게 본다.
입으로는 자기가 올바르게 산다 노력한다 정신적 가치를 추종한다라고 하는데 정말 그래본 적이 있을까.
그런데도 저들이 거짓말을 하는 이유는 간단하다. 그래야 그 글에 낚인 호구들이 열심히 \textbf{'사주기'} 때문이다.
\vspace{5mm}

그런데 이제는 언행일치라는 게 주목받기 시작하자 저런 거짓말도 안 먹힌다.
연예인들조차도 과거에는 안 하던 고소미를 시전하기 시작한 모양이다. 당분간은 입을 틀어막을 수는 있겠지.
그게 정말 모욕감을 얻어서인가? 아니지. 자기 장사에 불리하지 않게 여론을 조작하기 위해서이지.
하지만 그럴 수록 대중들도 똑똑해지기 시작하는 건 아나?
\vspace{5mm}

청년들에게 희망이 있나? 글쎄. 난 노력은 해야한다고보지만 희망은 없다보는데?
노력하라고 하는 건 그냥 지껄이는 이야기가 아니다. 이건 내가 느끼기 때문에다. 장기적으로 노력한 건 분명 보답을 하거든
그런데 희망이라는 것이 있긴 하나. 이런 소리를 나에게 지껄이는 어른들은 적어도 내 경험상 모두 위선자였던데 말이다.
그리고 그 희망이 의미하는 게 뭔지 아나? \textbf{"노력해서 내 자리를 위협하지 마. 알아서 굶어 뒈지든가"}
\vspace{5mm}

유감스러운 사실은 요즘 10대나 20대는
\textbf{"노력하지 않아도 돼. 사회를 바꾸면 돼. 그럼 희망이 있을 거야}"라는 한물간 것에 세뇌당했다는 것.
이게 적게 먹고 운동하는 고통 없이 다이어트가 가능하다라는 이야기와 뭐가 다른지 모르겠다.
운동도 싫고 적게 먹는 것도 싫은 사람이 택하는 게 결국 수술일 건데
그럼 자기 노력 안 하고 걍 사회가 다 뒤집혀야한다는 것도 뭐가 다른 이야기일까.
저런 이야기를 하던 사람이 정작 자기 자녀들은 열심히 공부시키더라는 걸 알고는 있을까.
\vspace{5mm}

\textbf{희망은 없지만 뒤처지지 않기 위해 노력한다.}
그냥 이 한줄로 충분할 것 같은데.
\vspace{5mm}

추가로 진실을 더 적어볼까?
\vspace{5mm}

다른 사람들이 노력하지 않으면 더 좋은 것 아닌가?
라이벌들이 알아서 죽어준다면 눈물을 흘리면서 "앙, 이 사회 참 절망적이야"라고 맞장구 적당히 쳐주고 우린 몰래 노력하면 되지.
사실 안 좋은 교재나 강의도 굳이 비추할 필요가 있나?
"앙, 그거 참 좋아"라고 하고 열심히 공부하라고 해준 다음에 자기는 탈출하면 되는 것 아닌가?
옛말에 미운 자식 떡 하나 더 주고 예쁜 자식 매 한 대 더 때린다라는 말 그대로다. 이런 종류의 지혜는 절대 바뀌지 않는다.
\vspace{5mm}

간혹 가다가 이런 종류의 얘기로 시비거는 사람들도 있는데 한심하다고 본다.
자기들이 노력하기 싫으면 하지 말고 그냥 스랙처럼 살면 되지 왜 그걸 남에게 하소연하나.
그리고 사회가 xx해서 문제다 하면 군소리 말고 걍 이민가버리든가, 아니면 자기들이 혁명을 일으키든가.
왜 입으로만 그러고 앉아있지?
그 사람들도 하는 짓 보면 사례 04의 나르시시즘 그대로.
그러다가 술이나 까먹고 나 잘났다 자랑질하다가 나중에 취업 어딜해 허둥지둥 그러는 거 한두번 보았나.
\vspace{5mm}

수험도 마찬가지이다.
난 강의나 교재 물어보는 것도 짜증나지만(그 시간에 풀면 되잖아),
도대체 강사 아무개가 어쩐다 교재 저자 저무개가 저런다 말하는 사람들 보면 '세뇌되었네 이 양반' 하는 생각에 이제는 지겨울 정도다.
돈 쓰는 사람이 갑 아닌가? 그럼 학생이 갑이고 강사는 을인데, 정작 돈은 다 바치면서 강사님 믿숩니다... 이게 사이비 종교가 아니면 뭐냐.
그리고 강사가 가르친 게 좋은지 안 좋은지는 기출분석해보면 나오지 않나? 그거 일주일이면 다 하는데 그럴 시도는 해보았나.
강의나 교재가 좋냐 안 좋냐 하는 건 그냥 본인이 비교해보면 되는 거다.
그럴 능력 자체가 없다는 것도 역시 세뇌당한 케이스인걸 부정할 수 있나?
\vspace{5mm}

이번에도 보면 뭔 놈의 저자가 쓴 교재가 좋다 하더니 역시나 수능 끝나고 다들 침묵이다.
저자들도 뭐 자기가 쓴 거에서 다 나올 거라고 하던데 일부 빼고는 걍 꿀먹은 벙어리고,
학생들도 그런 거나 능동적으로 분석해서 누가 실제로 도움이 되었나 파악하면 될 것을 참 엉뚱한 데 신경쓰고 앉아있다.
자기들이 세뇌당해서 레밍즈질하는 것도 모르고 "이 사회가 흉악해서 내가 노력해보았자 소용 어쩌구저쩌구" 한탄이나 하고 앉아있겠지.
\vspace{5mm}

수험의 비결은 간단하다. 자기가 판단해서 입시에 도움이 되는 걸 선택해서 그걸 밟으면 되는 거다.
그럼 스스로 부딪쳐서 공부해보아야 한다, 그러니까 문풀을 많이 해보라는 이야기지.
무슨 인강만 들으면 시간이 단축된다? 그럼 뭐하나. 자기가 스스로 공부할 줄 몰라서, 자기 공부가 어떤지 알 수가 없는데.
시간이 걸리더라도 스스로 문풀하고 책읽으며 깨져보면서 나아가야만 자기가 어딜 보충해야할지 알 수 있는 거다.
그런데 뭐 이거 상담글이든 질문글이든 보면 수험생이 수년째 공부해도 본인의 공부가 뭔지 모른다.
그저 어느 인강이 좋나요? 교재가 좋나요? ... 이런 스바 옴마니반메훔 구절 외우는 것도 아니고
\vspace{5mm}

이런 것도 하여간 깔 건 까야겠다. '좀 친절할 수 없냐'라고 할 수도 있겠지.
그런데 내가 친절히 대해주고 일일히 맞장구쳐주다간 나도 병신될 것 같다는 게 딱 보이더라는 것이지.
일단 특정 강사나 특정 교재(적어도 내가 보기엔 저런 것도 팔리는구나 생각되던 것) 강조하는 인간이 잘 나가는 것 못 보았다.
이것도 사실 검증이 된다. A라는 교재로 다 해먹으면 그 A 좋은 게 바로 누설되어서 대세가 되어버린다.
대세가 된 A 교재가 그럼 유의미한 효과가 있나? 결국 모두가 다 볼 것인데.
그럼 대세가 된 A교재를 평가원이 저격질하면 A만 본 사람은 다시 망하는 거지.
그렇기 때문에 어떤 만능교재라는 건 사실 존재할 수가 없다. 이건 인강도 마찬가지일 것이다.
\vspace{5mm}

계속 의심해보고 자문자답해보면서 검증해나가면 사소한 세뇌도 비켜나갈 수 있는데 그조차 안 하는 사람들이 너무나도 많다.
사실 그런 사람들이 좋은 대학에 가도 무슨 소용이냐... 라는 생각이 들지만 이건 답답한 것이 아닌가?
하나 예를 들어볼까.  잘나가던 영어교재 Z가 있다.
그런데 어느 순간 갑, 을, 병, 정이라는 강사들이 Z를 비난하기 시작하면서 자기들만의 교재 G, H, I, J를 낸다.
그럼 우린 어떻게 생각해야할까?
\vspace{5mm}

일단 Z를 보겠지. 왜냐면 갑, 을, 병, 정이 Z를 정말 잘 비난한다면 그건 그 사람들 실력이 Z로 키워졌단 이야기거든.
우선 Z를 독학한 다음에 G, H, I, J의 샘플을 보고 거기서 선택하겠지.
Z라는 책도 단점이 없을 리는 없으니 맹신하면 안 된다. 하지만 G, H, I, J가 모두 Z를 보완해준다고 보장할 수도 없지.
그러니까 Z부터 빠삭히 봐서 저 갑, 을, 병, 정의 수준을 따라잡고 비평하려하겠지.
\vspace{5mm}

그런데 현실은 G, H, I, J 하나 보고도 낚인 애들이 나중에야 Z를 보고 어라 강의내용이 여기 다 있었네... 이러는 경우가 비일비재하다는 것.
\vspace{5mm}







\section{[세뇌론 009] 반복은 덮어쓰는 과정}
\href{https://www.kockoc.com/Apoc/513169}{2015.11.27}

\vspace{5mm}

수학적 사고는 미신을 제거해주지요.
논리의 출발은 \textbf{"부정"}입니다. ~p가 거짓이니 p가 참이라고 얘기하는 것이죠.
고대 그리스 사람들의 사유가 다른 문명권과 달랐던 게 이것입니다.
그래서 국가적인 수학교육은  '근대적인 사고를 할 수 있도록' 백신을 대량접종하는 것이라고 하겠습니다.
어떻게 보자면 모든 학생을 근대인으로 세뇌시키는 과정이라 하겠지요.
개인에게 수학은 대입을 좌우하는 짜증는 과목 그 이상 그 이하도 아니겠습니다만,
국가입장에서는 원시시대나 중세시대 사람들이 늘어나는 것을 막기 위한 과정이라고 하겠습니다.
\vspace{5mm}

우선 아래 학습공학과 관련해서 이야기해봅시다.
\textbf{\textbf{반복}만 가지고 정신적인 문제가 해결되지 않는다...}. 그건 공부가 뭔지 아직도 모르기 때문일 겁니다.
세뇌의 기본은 반복이지요\textbf{. 반복없는 세뇌는 없습니다}. 그리고 세뇌의 효과는 대단히 강력합니다.
약물이 정말 불가피한 경우라거나 신경이 손상되거나 하는 경우가 아니라면
보통은 본인들이 어떤 경험으로 특정 트라우마에 사로잡혀 살아왔는가 하는 케이스는 일종의 반복경험으로서 바로 잡을 수가 있습니다.
거꾸로 말하면 잘못된 세뇌를 당하면 정신병자처럼 행동할 수 있다와 똑같은 이야기이지요.
악몽에 시달리는 사람이 우선 시도해야할 것은 그 악몽을 지우고 좋은 이미지로 덮어쓰는 과정을 반복하는 것입니다.
그 악몽을 지우기는 매우 어려운 일이나 그걸 흐리게 하거나 작게 만들 수는 있얼 것입니다. 시간이 걸리더라도 말이지요.
\vspace{5mm}

전문적인 최면(참고로 문학에 관심있는 사람은 밀턴 에릭슨 참조)까지 논할 필요 없이
우리가 하는 공부 자체가 \textbf{일종의 세뇌이자 간접 최면이기까지하다}는 것을 간파할 필요가 있습니다.
사실 다년간 특정 과목들의 도그마들을 반복학습하여 그걸 숙달시키는 과정은 여타 어떤 세뇌보다도 매우 강력하다고 하겠죠.
이걸 엿볼 수 있는 예는 가령 특정 인강을 들으면 그 학생은 그 강사를 업자를 넘어서 부모, 스승으로 여기는 경향이 있다는 것입니다.
(직접 만나본 적도 없는데 말이죠. 우리가 TV 드라마를 볼 때 배우나 PD에게 그런 감정을 품는가요?)
독학을 권유하는 이유이기도 합니다. 적어도 자가세뇌가 덜 위험하니까요(물론 매우 좋은 강의라면 그런 강의로 세뇌되는 게 낫습니다만)
하지만 무엇보다 본인이 스스로 공부, 아니 자가세뇌를 해서 인간이 바뀔 수 있다면 그건 운명을 스스로 개척하는 것이지요.
운명 이야기가 나와서 적는다면 만약 삶이라는 게 주관적 심상의 연속체라면 세뇌로 운명이 바뀐다라는 건 틀린 이야기가 아닙니다.
온라인 게임이나 하고 야동이나 보던 청년이 감금되어 국영수탐구 공부를 강제당해 무수히 반복하다
사고 프로세스가 바뀌면 사람이 달라진 것이고 그럼 결국 운명이 바뀐 것이지요.
\vspace{5mm}

공부를 어느 정도까지 해야하느냐? \textbf{사이비 종교가 사람을 세뇌시키는 정도는 넘어서야} 합니다.
그 정도까지 가야 떨린다거나 긴장한다든가 하는 것도 아무 것도 아닌 문제가 되는 겁니다.
단순히 공부를 한다... 수준이면 여전히 뇌는 공부를 거부해버립니다. 문제를 못 푸는 게 아니라 실제로는 '안' 푸는 것이죠.
수학공부를 한다 수준으로는 곤란합니다. 아예 \textbf{수리적 사고에 세뇌당해버려야합니다.}
소위 수학을 잘 한다는 사람들의 특징은 머리가 좋은 게 아닙니다. 제가 보는 그들은 \textbf{세뇌당한 '광신도'}들입니다.
아래 학습공학에서 침착해지는 건 실성하는 것이라고 얘기드렸죠. 광신도는 이미 '실성' 속성은 갖고 있음.
\vspace{5mm}

차분히 공부해서 합리적으로 이해하고 정리한다.... 까페에서 여유부리면서 총칼 든 강도와 적대할 수 있단 애니메이션을 너무 많이 본 것이죠.
이미 시험문제부터가 제정신들이 아닌 것들입니다. 제 시간 내에 그것들을 풀어내려면 정상적 사고로는 불가능할 터인데이요.
물론 문제를 독해하고 의도를 파악하는 건 차분한 이성적 태도로 가야합니다.
그러나 그런 관리자는 검은 양복 입은 깍두기들이 지켜주고 있어야하죠. 즉, 해당 과목에 세뇌당한 것을 기본으로 합리적 사고를 해야하는 것입니다.
사실 우리는 한글만 하더라도 ㄱㄴㄷㄹ.... 제자 원리 제대로 알고 쓰는 것도 아니죠. 이 역시 사실 우리가 국어네 세뇌당한 것입니다.
\vspace{5mm}

다시 앞으로 돌아가 수학적 사고의 세뇌라고 했을 때 조심해야할 게 있습니다.
여러가지 유형과 패턴으로만 세뇌당하느냐, 아니면 그 패턴을 해체하고 참거짓을 따지는 수준까지로 세뇌당하느냐.
전자와 후자의 차이는 매우 큽니다. 사소해보일지라도 단지 문제를 풀기만 하는 것과, 그 문제를 이루는 모든 개념을 의심하고 검증하는 건 다르지요.
유감스럽지만 전자를 강조하는 업자나 학생들이 매우 많습니다. 그리고 실제 시험에서는 털립니다.
심지어 문제를 많이 풀었다고 하는 경우조차도 전자에 머무른 경우라면 안 되지요.
xx 문제면 xx하게 푼다.... 라고 흔히 강의하거나 야매교재에 쓰여있는데로 가면 망합니다.
\textbf{xx 문제에 xx이 나왔는데 xx의 교과서적 정의는 무엇인가, xx는 식, 그래프, 개념으로 어떻게 표상되는가, 그리고 이게 어떻게 변형되는가...}
\textbf{이런 식으로 하나하나 형식적인 검증을 밟으면서 출제자가 어떤 루트를 따랐을까 읽어보는 방식으로 세뇌되어 있어야합니다.}
\vspace{5mm}

그리고 이 점에서 맨 앞문단에서 말한 수학적 사고의 혜택이 드러납니다.
저런 식의 사유를 하면 본인의 문제까지도 인과관계를 파악하면서 내가 왜 불안해하거나 그런 충동이 드는가 하는 것도 분석되면서
매우 심각하게 여겼던 문제조차도 사소하게 생각하거나, 해결가능한 실마리를 찾게 될 수도 있습니다.
현대수학이 아닌 근대수학의 학습은 본인의 정신에 매우 긍정적인 효과를 낳을 수 있다고 봅니다.
물론 저 근대수학에만 세뇌된 것도 문제가 없는 건 아니겠습니다만...
근대수학도 세련된 '도그마'이기 때문입니다.
\vspace{5mm}

대화를 하다보면 성적이 높은 친구와 낮은 친구는 말하는 수준이 다릅니다만.
그건 머리가 좋다거나 그런 게 아니라, 각자가 어떤 형식에 세뇌되어있나에 의한 차이라고 할 수 있겠고.
개인적으로는 그런 형식들을 하나하나 따져보면서 그 형식을 해체시키는 방향의 대화를 걸고 일종의 무질서를 초래해보면서
그런 건 유전이라기보다는 반복학습되어 세뇌단계에 이르게 된 양식에 불과하다라는 확신에 이르게 되었습니다.
\vspace{5mm}

여기서 심각한 질문을 던진다면 님들이
\textbf{"나"라고 생각하는 것들은 감각과 연결된 일종의 형식논리의 반복에 불과할지도 모른다.}
이걸 크게의 심해보셔야 할 것입니다.
세뇌론에서 언급하는 마법사들은 개인과 집단의 삶이 그렇게 연약하고 단조롭고 취약한 형식의 구조물이라는 걸 깨닫고 장난질을 치고 있죠.
이게 뭔지 깨달으면 왜 인문학이 매우 비인간적이고 위험하며 강력한 학문인지 깨닫게 되실 것입니다.
"나", "가족", "사회", "국가"라고 하는 것들이 얼마나  부스러지기 쉬운 레고블럭인지 깨닫는다면 그건 제대로 공부, 즉 세뇌된 것이겠죠.
\vspace{5mm}







\section{[세뇌론 010] 교주와 교단이 갖춘 무기}
\href{https://www.kockoc.com/Apoc/694250}{2016.03.25}

\vspace{5mm}

일반적으로 교주들은 대단히 따뜻하고 인자한 면모를 보인다.
이 점에서는 북쪽의 김씨 왕조 수장이 초절정 고수일 것이다. 남몰래 정적을 숙청하면서도
그 자신은 인자한 아버지상을 표출했기 때문이다. 말을 잘 들으면 따뜻하게 해주겠지만 안 그러면 죽어.
\vspace{5mm}

사람은 사상, 행동, 감정의 세가지 축 중 한가지만 무너져도 인격이 붕괴된다
생각만큼 사람은 강하지 않은데 잘 생각하면 컬트집단들은 이 셋 중 하나를 흔들 수 있다.
첫째, 기존 윤리와 철학을 포섭할 수 있는 매우 정교한 교리가 있다 → 사상 공략
둘째, 집단이기 때문에 실천이 그리 어렵지 않으며 쇼하는데도 탁월하다 → 행동 공략
셋째, 무엇보다도 피세뇌자의 어두운 감정을 다독일 수 있다 → 감정 공략.
\vspace{5mm}

이걸 버텨낼 사람들이 있나?
견뎌내기 위해서는 논리적 부정을 생활화하며 뭐든지 의심해보아야 하는데
그 순간 본인은 매우 매몰차고 냉혹하며 감성이라고는 없는 '악인'으로 비치게 되는데
이런 악한의 이미지를 뒤집어쓰더라도 진실만을 향해 나아갈 용기가 없으면 십중팔구 세뇌당해버린다.
그러나 우리사회는 이런 '비관주의적인 회의론자'를 치유 대상으로 여기고 있지 않나.
\vspace{5mm}

게다가 똑똑한 사람들이 세뇌당하기 쉬운 이유는, 세뇌자들은 항상 \textbf{가치있는 '정보'}를 던져주기 때문이다.
그런 모임들이 가상과 허구의 세계에서만 노는 것이 아니다. 오히려 그들은 가상과 허구가 얼마나 허망한지 잘 알고 있다.
피세뇌자들을 요리할 수 있는 현실론자들답게 어떤 정보가 가치있는지 알고 있으며,
그런 가치있는 정보를 던져주면 회의론자들도 중독되어서 자기들의 희생물이 될 수 있음을 알 수 있다.
그래서 실제로 금전적인 사고를 꼼꼼히 한다고 하는 아줌마들이 종교인들의 노예가 되는 일이 벌어지는 것이다.
\vspace{5mm}

약도 지나치면 독이 된다. 가치있는 정보라고 할지라도 자기가 원하는 것 이상을 넘어서면 과감히 컷해야하며,
자기가 구축해 온 지식에 포섭해버려야 흔들리지 않는다.
술 한잔이라면 모를까 수십잔을 마시면 고주망태가 되어 상대방의 속셈대로 되는 것이 아니겠나.
\vspace{5mm}

정보에 중독된 고학력자들일수록 세뇌자들의 '정보조작'에 당하기 좋다.
이건 돈이 모이는 판에서는 자연스럽게 벌어진다. 수험판이라고 다를 게 없는 게 아니라, 사실 수험판이 이런 점에서 매우 재밌는 곳이다.
그리고 여기서 자기가 가치있는 정보를 찾다가 n수해버린 사람들은 상당히 찔릴 것이다.
\vspace{5mm}

상품을 파는 업자들은 슬그머니 자신의 신상, 외모를 강조한다. 그러면서 '우상'이 되기 위한 절차를 거리낌없이 저지른다.
객관적으로 보면 상품과 관계없는 행각이지만 사실은 이런 행위야말로 고도의 세뇌술이라고 할 수 있다.
이제부터 호갱들이 구입하는 건 '품질이 나쁘면 선택 안 해도 되는 상품'이 아니다. 바로 돈을 내서 교주님을 모시게 되는 것이다
품질이 기대 이하인 상품이 드디어 나왔습니다... 하면서 아무개님의 정성이 들어갔다고 치자.
이 경우 호갱들은 그 상품이 좋은지 나쁜지 신경쓰지 않는다, 오로지 그 아무개님을 배알하는 신자의 마음으로 돈을 쓰게 되어있다.
\vspace{5mm}

이것이 세뇌술의 가장 원시적인 것임을 모르는 사람들도 많다. 그런데도 쓰는 작자들이 많은 것이다
그걸 쓰는 사람들의 연령이나 전공을 보면 정신분석이나 심리, 종교적인 것과는 거리가 있으면 두가지이다.
남이 하는 걸 따라하거나, 아니면 비공개적 루트로 고객을 신자화하는 테크닉을 익혔단 이야기가 된다.
\vspace{5mm}






\section{[세뇌론 011] 본인이 강하지 않으면 극한으로 공부하지 말 것.}
\href{https://www.kockoc.com/Apoc/694274}{2016.03.25}

\vspace{5mm}

실제 세뇌란 말은 중국어인 'husi nao(스이 나오)'에서 파생된 것이다.
\vspace{5mm}

20세기의  공산주의자들은 각 진영의 전도사들은 상대를 포섭하기 위해 애를 썼다.
마르크스주의에 전염된 사람들이 실제로는 마르크스 사상이 뭔지 모른다.
\vspace{5mm}

세뇌를 할 때에는 상대가 극한 상태에 있어야한다. 극한 상태에서는 정상적인 사고가 되지 않기 때문이다.
과거 중국, 즉 중공이나 과거 러시아, 즉 소련에서 세뇌할 때에는 감금해놓고 당근과 채찍을 반복해놓는다.
일주일 내내 어둠 속에 지내게 하거나 일부러 식사/수면 시간을 교란한다. 초인이 아닌 이상 정상적인 사고가 무너져버린다.
정상적인 사고가 무너진 사람들은 현실감이 사라지며
그 상태에서는 \textbf{뭔가 고백하지 않으면 안 되는 기이한 상태}에 빠진다.
자기가 잘못하지 않았는데도 "고백"하고 용서받으려하는 것인데 이건 자아를 부정하면서 자존심까지 저버리는 것과 똑같다.
이렇게 모든 입구가 뚫리면 세뇌자가 원하는 대로 사상/감정/지식/정보 조작을 할 수 있다.
\vspace{5mm}

이걸 수험으로 연결시켜보면 수험생들이 꽤 위험한 길을 걷고 있다고 느끼는 대목이 있다.
실제로 상담하다보면서 느낀 것인데, 실패한 사람일수록 이상하게 '극한'을 좋아하는 경향이 강하단 것이다.
매우 극단주의적으로 한쪽에 치우친 선택을 하는 건 그렇다 치고, 공부를 할 때에 자신을 막 내던져 '극한'으로 간다는 걸 선호한다.
물론 공부할 때에는 극한으로 결국 가게 되어있다. 그러나 이 경우는 성적이 좋고 실적이 있어 자존심도 세졌고 자아가 튼튼해졌을 때이다.
반대로 \textbf{시험에 실패하고 자존심이 사라져 "고백"하고 싶어하는 게 본능이 되어버린 상태에서 극한을 달린다?}
\vspace{5mm}

대충 이 글을 읽는 사람들은 이런 패턴들을 경험해왔다는 것에 소름이 끼칠 수도 있다.
그리고 이런 것을 몰랐던 사람들이 대부분이겠지만, '아뇨, 실제로는 알고 있었습니다'라고 대답할지도 모른다.
다만 하나 분명한 건 수험에서의 극한 상태라는 건 불가피하다 할지라도 사실 이것만큼 위험한 것은 없으며
그래서 수험생활을 길게 겪은 사람이 살인충동을 겪거나 변태적인 분야로 빠지거나 맛가버리는 것도 이상할 게 없다는 것이다.
다시 말하지만 극한의 상태에 도달하기 위한 전제는 \textbf{"자존감을 회복한 이후"}다.
\vspace{5mm}






\section{[세뇌론 012] 집단세뇌}
\href{https://www.kockoc.com/Apoc/694295}{2016.03.25}

\vspace{5mm}

강의는 사실 세뇌와 많이 겹친다.
선생님이라고 하면서 그 사람이 하는 강의에 집중하고 귀기울이는 건 좋게 보면 학습이지만,
만약 강사가 엉뚱한 걸 가르치거나 불순한 의도의 코드를 집어넣는다고 하더라도 받아들이는 일종의 종교집회와 똑같다고 할 수 있다.
\vspace{5mm}

학습론에서 혼자 공부하지 말고 도서관이나 학원에 가라고 하면서 혼자 공부하기는 어렵다라고 했지만
이건 사실 위험한 것을 이용한 고육지책이라고 하지 않을 수 없다.
\vspace{5mm}

원래 인간은 혼자서는 세뇌당하기 힘들다. 혼자인 경우는 더 많이 긴장하기 때문에, 그리고 일대일 대화를 하며 비판을 하기 좋다.
그러나 집단 속에서는 달라진다. 우선 집단에 속한 이상 우리의 본능은 생존을 걱정하지 않으므로 \textbf{경계심을 죽이게 된다}.
거기다가 자기가 속한 집단과 다른 방향으로 가기 꺼려진다. 그래서 집단 속에서 강의를 들으면 \textbf{그 강의에 집중하게 되는 것}이다.
\vspace{5mm}

그냥 mp3로 들으면 별 것도 아닌 롹 공연을 실연으로 들었을 때에는 감동의 해일에 빠지는 원리와 비슷한 것이다.
인강으로 들었을 때에는 그냥 그런 메시지인가 하면서 다소 흘려듣거나 반사시키며 듣는다.
세뇌의 위험도는 줄어들 수 있어도 대신 학습효율이 떨어져 버린다.
반면 수십, 수백명의 열기가 가득한 강의실에서는 그 집단적 흐름을 따라 맹목적으로 집중하게 된다.
세뇌의 위험성은 커지나 그만큼 학습효율이 높은 것이다.
\vspace{5mm}

어느 쪽이 낫다고 할 수 없지만 장점을 살리고 단점을 죽이려면 이런 특징은 알고 있어야한다.
공부가 잘 된다는 것은 그래서 두가지이다. 그게 세뇌를 통한 것인가 아닌 것인가.
양적 측면 때문에 그리고 효율성 때문에 우리는 세뇌적 학습을 안 할 수는 없다.
그러나 자기 스스로 생각하고 검토하고 비판하는 것이 없으면 그 세뇌적 학습의 약빨은 한계가 있는 것이다.
\vspace{5mm}





\section{[세뇌론 013] 분노의 에너지와 악의 저주}
\href{https://www.kockoc.com/Apoc/696312}{2016.03.26}

\vspace{5mm}

스트레스가 쌓이면 두 가지 현상이 생긴다.
\vspace{5mm}

하나는 \textbf{분노의 에너지}
다른 하나는 \textbf{악의} \textbf{저주.}
\vspace{5mm}

소위 성격이 안 좋아진다라는 것이 저걸 대충 표현한 것이다.
수험에 실패하는 사람일수록 세뇌당하기 쉬워지는 이유는 저것으로부터의 도피일 수도 있다.
\vspace{5mm}

뭣도 모르는 어른들이 화를 가라앉히라고 하지만 엉터리 조언이 아닐 수 없다.
경쟁 상황에 놓이면 분노의 에너지는 필연적으로 쌓인다. 적절히 \textbf{'공부 에너지'로 활용하라}는 게 좋은 조언이다.
똑같은 핵분열도 통제를 잘 하면 저렴한 원자력 전기가 되지만, 실패하면 체르노빌이나 후쿠시마, 의도적으로 폭발시키면 히로시마 나가사키.
분노 에너지를 공부 에너지로 적절히 전환할 수 있는 '환경'이라는 게 그래서 중요해진다.
\vspace{5mm}

다른 하나는 바로 악의 저주다. 자신의 실패나 상처를 처음에는 긍정하려하지만 결국에는 참지 못 한다.
인간이 재밌는 건 감정적 기복을 심하게 겪으면 그걸 예술 - 노래, 그림, 시로 표현하려 하는 것인데
수험생의 경우는 자신의 좌절감을 가지고 \textbf{"나는 안 되는 놈이야"}라고 자기 비하를 하는 예술을 하는 게 문제다.
그 예술이 소위 저주가 되어서 그 자신을 옭아맨다. 심하면 정말로 지하철 선로에 뛰어내리거나 한강에 다이빙하는 것이다.
이것 역시 그런 싸구려 저주에 속지 말라는 식으로 조기에 가치관을 바꿔주거나 따끔하게 비판하는 게 적절할 것이다.
\vspace{5mm}

이전 글에서 집단에 속하는 것이 위험하다고 하면서도 독학이 위험한 이유를 적지 않았는데 바로 위 두가지라고 할 수 있다.
혼자 공부하는 사람들은 분노 에너지로 초사이어인이 되어있는 데다가 저주의 언어를 불경처럼 외우고 다니고 한다.
이 사람들은 상담해달라고 하면서도 자신의 불행을 상대가 동감해주길 바라는 걸 넘어, 상대가 불행해지길 원하는 경향이 있다.
물론 본인들은 이걸 보면서 찔리면서 '아 나는 그런 적 없어'라고 할 것이다.
그만큼 분노의 에너지와 악의 저주는 무시무시한 것이다.
\vspace{5mm}

그런데 이건 정말이지 디씨 고갤이든 ㅇㅂ 대팍도 마찬가지이지만 수험생 사이트, 그리고 콕콕에서도 자주 접한다.
분노의 에너지나 악의 저주에 사로잡히면 심하게 말하면 5년 이상 계속 같은 패턴을 반복하며 시간을 날리게 되어버린다.
그럼 5년 이후에는 왜 치유되느냐. 그거야 나이를 먹고 자신도 외형적 변화를 거치다보니까 허영심이 사라지는 탓이다.
거꾸로 말해 독학을 하던 사람들이 학원에 가면 저런 병이 완화된다.
"나랑 똑같은 처지에 있는 애들이 많구나, 아니 나보다 심한 사람이 있구나"라고 느끼는 순간 분노의 에너지가 정화되고
악의 저주가 허무맹랑하게 들리는 것이다. 동병상련을 느끼거나 자기보다 열등한 사람을 보며 우월감을 갖는 것이 진통제 역할을 해준다.
\vspace{5mm}

이 효과를 맛본 사람들은 그래서 '학원에 가는 게 낫다'라고 충고하게 된다.
물론 자기들이 얼마나 끔찍한 분노의 에너지나 악의 저주 덩어리인지는 싸악 망각해버리고 말이다.
\vspace{5mm}

분노에 사로잡히고 저주에 걸린 채로 극한을 추구하는 것이 나을까.
만약 그런 사람이 있다면 1년에 쇼부보지 말고 2~3년 길게 가면서 자신을 정화한 다음 천천히 가는 게 낫다.
시험에 붙을 확률도 낮지만 어쩌다가 운이 좋아 대학에 간다고 해도 정신이 엉망진창인 채 대학생활을 해낼지도 의문이기 때문이다
\vspace{5mm}



