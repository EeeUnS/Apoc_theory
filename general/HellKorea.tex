

\section{[헬조선 001] 로스쿨}
\href{https://www.kockoc.com/Apoc/758436}{2016.05.03}

\vspace{5mm}

정치적 이야기하자면 20대들이 지지하는 모 정당이 알고보니 금수저만 좋은 일 해주었다는 확정적인 흑역사.
\vspace{5mm}

조선이 망한 이유야 여러가지가 있지만 결국 망한 건 '신분사회' 때문이다.
신분사회에서는 능력이 있는 사람이 그 능력을 펼 수가 없다.
과거 시험이 있다고 하지만 이것도 양반 자제가 아니고서는 현실적으로는 응시하기 힘들었다.
그러니 그들만의 리그가 되어버린 것이다.
\vspace{5mm}

적어도 일제시대가 조선보다 나은 것이 이것이다. 일제시대에는 조선총독부 공무원 시험에 양반집 자제라고 딱히 우대는 안 했으니까.
\vspace{5mm}

3·1 운동의 불씨가 남아 있던 1922년 순사직 경쟁률은 약 2.1대 1 수준에 불과했으나 문화정치가 본격화한 1920년대 중반 이후부터 그 경쟁률이 10대 1을 웃돌았다. 1926년에는 856명 모집에 9천193명이 지원, 약 10.7대 1의 경쟁률을 보였으며 1932년에는 854명 모집에 1만 6천193명이 지원해 19대 1로 경쟁률이 수직으로 상승했다. 순사 등에 대한 19.6대 1로 정점을 찍은 1935년 이후 순사에 대한 선호는 1936년 14.1대 1, 1937년 10.9대 1로 내리막길을 걸었으나 1920년대 중반부터 태평양 전쟁 이전까지 경쟁률이 10∼20대 1에 이를 정도로 순사직이 높은 인기를 구가했다.이를 두고 연세대학교의 장신은 "순사는 조선인 사회에서의 좋지 못한 이미지에도 불구하고 법률로 보장된 권한 탓에 해마다 높은 지원율을 보였다."라고 지적하면서 "관리의 최말단인 까닭에 지원자의 학력 수준은 보통학교 졸업자가 80$\%$ 정도를 차지했다."라고 분석했다.
\vspace{5mm}

일제시대를 거쳐 한국전쟁이 끝난 뒤에 고속성장할 수 있었던 건 뼈아프지만 "일본식 교육 제도"가 반도의 병폐를 씻어주었기 때문이다.
그 혜택을 입고 정신나간 지식인들이 일제 잔재이니 고등고시를 청산해야한다 어쩐다 하지만
이런 줄세우기 시험이야야말로 헬조선의 악습을 막아주었던 것이다.
(그리고 일본 순사 너무 까지말자. 금수저, 흙수저론으로 보자면 나름대로 출세하려고 열심히 공부한 사람들이구만)
\vspace{5mm}

로스쿨은 가장 한국적인 제도다. 겉으로는 공정한 척 하면서 결국 부모 빽이 먹힌다는 것이 우리의 고유한 전통 음서제 그대로가 아닌가?
물론 이런 시도는 로스쿨이 처음은 아니다. 그 전에는 기여입학제가 있었으니까.
대학교 재정이 부족하니 돈많은 사람들에게 2$\sim$30억 받고 입학시켜주자는 것,
그러나 이 발상은 그들이 그렇게 욕하던 꼴보수정권 시절에도 통과되지 않던 것이,
기어이 가장 개혁적이고 서민적인 분 밑에서 어이없이 통과되었다.
더 빡치는 문제는 이런 문제가 분명히 터질 거라는 걸 다 알고 있었는데도 강행했다는 것이다.
\vspace{5mm}

적어도 그 전까지는 입시비리가 의심되면 예체능을 제외하고는 정말 국민여론이 장난이 아니었다.
예전에 이런 일이 터졌다면 장관 모가지가 날라갔을 것이다.
그런데 지금은 그렇지도 않는다. 부정이 있지만 입학취소 불가요 $\sim$ 법이 그런 걸요.
예체능은 왜 예외냐고 하지만 그건 간단하다. 이 분야는 걍 실속이 없으니까, 그리고 우리의 삶에 영향을 미치는 게 덜 하니까.
하지만 의대나 법대는 우리의 삶에 치명적인 악영향을 끼치는 분야다. 거기에 부정입학자가 있으면 파급효과는 이루 말할 수 없는 것이다.
\vspace{5mm}

이 로스쿨은 그것만으로 끝나지 않는다. 정시비중이 줄어든 지금의 입시제도 역시 그렇다.
극단적으로 말해 학생들이 공부는 덜 하고 스펙쌓기만 한다는 평가가 나오고 있다.
정시 문이 줄어드니까 뒤늦게라도 공부하겠다는 친구들은 좌절해서 공부를 포기한다.
어린 시절부터 부모 서포트로 스펙 쌓아온 친구들은 자기가 뭐라도 된 양 귀족처럼 군림한다.
헬조선 미래가 너무 밝아서 눈이 멀어버릴 정도다.
\vspace{5mm}

이른바 일제 잔재 청산하고 개혁한다는 결과가 결국 이렇게 나타났다.
물론 그들은 취지는 그런 게 아니었다... 변명하겠지만 그렇게 따지면 취지가 안 좋은 게 어딨나.
\vspace{5mm}

정치 얘기하자면 헬조선하면서 특저 정당 지지하는 친구들이 이해가 안 가는 게 그렇다능.
저 로스쿨 제도 그 특정 정당이 강행하고 지금도 지켜주고 있다. 이게 로스쿨로만 끝난 게 아니지.
교육제도 전반을 금수저에게 유리하게 바꿔버렸다능. 수능 등급제나 등록금 폭등은 어휴.
역설적으로 그들이 비난했던 '먹방의 달인' 모 대통령이 장학금과 등록금 문제는 흙수저를 배려한 정책을 폈으나... 욕먹는다.
이 친구들은 자기 삶의 문제를 보기보다는 누군가에게 주입받은대로 호불호를 나뉘고 정의관을 주입당한다라고보는 강력한 증거다.
\vspace{5mm}

지금이라도 그냥 일제시대 스타일로 가는 답이겠지만 그렇게 해줄 리는 없지.
그들이 신경쓰는 건 이 부와 권력을 어떻게 자녀들에게 세습시켜주느냐.... 인 것이다.
이건 한편으로 공부를 열심히 하는 흙수저 녀석은 절대 올라오지 마라...는 것. 그래서 그런 시험제도를 하나하나씩 없애버리는 것이다.
\vspace{5mm}






\section{[헬조선 002] 도서정가제}
\href{https://www.kockoc.com/Apoc/759676}{2016.05.04}

\vspace{5mm}

\url{http://www.newsis.com/ar_detail/view.html?ar_id=NISX20160501_0014057073&cID=10701&pID=10700}

\vspace{5mm}

출판시장 정상화.... 는 커녕 제대로 출판사들을 엿먹였다.
안 그래도 살림살이가 어려운 판에 책값을 정가로만 받겠다는 데 사람들이 호주머니를 열 일이 없었고
그로써 판매량이 급감하면서 중고서적 시장으로 수요가 몰리면서 알라딘 헌책방만 잘 나가고 다른 서점도 허겁지겁 중고책 매매에 뛰어들게 되었다.
\vspace{5mm}

도서정가제를 밀어붙인 모 국회의원이나 어느 출판계 거물께서 경제학을 조금이라도 공부했으면
이런 일이 벌어질 것이라는 건 분명히 알 수 있었을 텐데... 아니 알면서 인터넷 서점 밀어주려고 일부러 그런 게 아닌가 의심스러울 정도다.
\vspace{5mm}

사실 정가제로 개인적으로 피해본 것은 없다.
수학 참고서나 일본서적 구매 빼고는 새책을 구매하지 않거든.
북코아 아니면 알라딘에 나온 헌책 중 절판되어서 구하기 힘들거나 알고보니 명저인 것만 구하는데
신간 서적 사는 15,000원이면 과거의 좋은 책을 5권 이상 구매할 수 있다. 그러니 새 책에 연연할 필요가 사라졌다.
수학 참고서도 이제 구매할 건 별로 없고 웬만한 건 지역 도서관에서 빌려보거나 신청하면서 기다리면 된다.
과거에 교보문고나 반디앤루니스에 자주 갔다면, 이제는 깨끗이 정돈된 알라딘 헌책방에 간다.
보지 않는 책을 잘 선별해 팔아도 생활비를 아낄 수 있다.
\vspace{5mm}

고객 대부분이 나랑 같다면 출판사들은 어떻게 되나. 죽어나는 거지.
이런 이상향을 구현하신 출판 관계자 분들은 여전히 뽕에서 못 깨어나셨는지 완전 정가제에다가 중고시장 규제까지 하려 하시는 모양인데
그런 주제에 무슨 독자들에게 책의 향기니 하신지 모르겠다.
멋대로 가격규제해버리면 부작용 일어난다는 것이야 상식 중 상식인데 그걸 몰랐나.
단통법이야 통신 시장이 워낙 비탄력적이니까 효과가 있다 치자(그래보았자 대기업 배불리기겠지만)
안 그래도 미디어 소비 스타일이 바뀌어서 책값 낮춰도 모자랄 판에 정가제 실시하면 사람들이 책을 읽겠니.
\vspace{5mm}

어설픈 정의관 토로하지 말고 그냥 냅두었으면 시장은 나름 균형을 찾았을 텐데
출판사 살리겠다고 하다가 시장교란시키고 대형서점만 배부르게 한 책임은 누가 지겠나.
헬조선에서는 그딴 거 책임지는 사람은 없다. 저 뉴스에서처럼 말도 안 되는 이야기나 하신다.
\vspace{5mm}

그냥 정가제 철폐하면 되잖아.
\vspace{5mm}

정말로 웃긴 건 현재 가장 이득을 본 알라딘은 도서정가제를 강력하게 반대하는 입장이었단 것이다(...)
\vspace{5mm}






\section{[헬조선 003] 공대의 황혼}
\href{https://www.kockoc.com/Apoc/759903}{2016.05.04}

\vspace{5mm}

\url{http://news.chosun.com/site/data/html_dir/2016/05/04/2016050400298.html}
\vspace{5mm}

공돌이가 그만큼 많아지면 '공급'이 늘어나니 몸값은 싸진다는 건 당연.
거꾸로 생각해보면 만약 의치한이나 로스쿨 정원 늘린다고 했으면 바로 시위했을텐데.
정원을 늘린다고 해도 시위도 못 하고 아 그런가보다라고 얻어맞는 불쌍한 공돌이들.
\vspace{5mm}

사실 지금도 이미 많은데 이걸 더 늘린다.... 이제 공돌이면 취업 잘 된다라는 건 옛날 이야기가 되어버렸음.
겉으로는 수험생 배려하는 것 같지만 실제로는 대학교들이 정원감축 안 하면서 먹고 살려는 꼼수.
\vspace{5mm}






\section{[헬조선 004] 과학만능주의}
\href{https://www.kockoc.com/Apoc/760048}{2016.05.04}

\vspace{5mm}

간단히 말해서 과학은 만능일 수가 없다.
추상적이고 100$\%$ 완결성을 꿈꾸던 수학조차도 그렇지 않은데 과학까지도?
\vspace{5mm}

옥시 사태로 정부탓만 하는 사람들이 있다. 그런데 그들은 가장 중요한 과학만능주의를 은폐한다.
과학은 완벽할 수가 없다. 오히려 불완전하다는 것을 인정하면서 그걸 줄여나가는 것이 과학적인 태도가 아닌가?
그런데 정부만 탓하는 사람들은 정작 과학을 종교화한 케이스는 말하지 않는다.
\vspace{5mm}

현실 속에서의 과학은 자본에 놀아난다
학계는 돈을 대주는 자본에 불리한 이야기를 할 수 없다.
연구 결과 조작이라거나, 조작이 아니더라도 편향적인 방향의 연구 자체는 전문가 빼고는 검열할 수 없는 것이다.
이미 이 사태는 황우석 교수의 줄기세포 조작 사건에서 드러났다.
사진 복붙인 것을 다른 전문가들이 눈치까고 제보했다. 처음에는 주류언론과 정부조차도 음모라고 일축했으나
그 껀수가 '해외대학'에 제보될 정도가 나서야 그 공범이 시인해버렸던 것이다.
믿거나말거나 난 당시에 황우석이 사기꾼이라고 생각했는데 주변 사람들은 그럴 리 없다라고 얘기했다.
물론 사건이 터진 이후에는 그 사람들은 바로 태도 바꿔 그럴 리 없다라고 한다.
이 패턴이 옥시 사태에도 반복된 것이다.
\vspace{5mm}

전문가의 말은 신뢰할 수 있다. '경제적 이해관계나 종속관계'가 없는 평행 세계의 전문가 혹은 죽은 전문가라면 말이지.
장사해야 하는 현실 속의 전문가라면 우선은 의심하는 태도를 취해야 한다.
그러나 그 전문가들은 여전히 "과학만능"을 주장하면서 그래도 전문가를 믿어야 한다, 이게 다 정부탓이라고 말한다.
아니, 그럼 당시에 문제의 제품을 승인한 정부관료들이 전문가 말을 그럼 안 들었겠나, 그리고 옥시 업자들이야말로 전문가 아니었나?
\vspace{5mm}

과학이라는 미명 하에 자사 영업에 불리한 담론을 차단하는 것 자체가 문제지 지금 비과학이 문제겠나.
한국사회에서의 과학은 판타지이자 스테레오타입 그 자체인데?
역으로 광우병은 어땠나. 사람들에게 생소한 온갖 전문용어로 범벅되어서 광우병의 위험을 과장해댄 담론이 무한증식하지 않았나.
그 당시 광신도들이야 자기들이 그렇게 했으니까 안전했다고 변명하는데 헛소리다.
그 사람들이 그런 관심을 당시 가습기 살균제에 기울였다면 어땠을까.
\vspace{5mm}

옥시 제품의 문제를 발견하는 전문가는 그 자본의 눈치를 보지 않아도 되는 전문가였다. \textbf{이게 중요한 것이다.}
그건 바꿔 말해 이 문제를 누군가는 이미 알고 있었지만 덮었을 가능성이 높다. 책임지기 싫으니까.
\vspace{5mm}

그리고 여기서부터 논란이 되는 이야기하자면
\vspace{5mm}

고교과정만 배우면서 '과학이 만능'인 줄 안다거나
어떤 분야의 실무건 그건 사실 불가능한 이야기다.
실무가들은 자기 하는 일도 바빠서 그걸 판단할 겨를도 없고 그럴 이유조차 없다.
대학원생 이상급이 되면 언행이 달라지는 게 있다. 정말로 전문분야로 가면 매우 소극적이 된다는 것.
반드시 $\sim$ 하다라고 얘기하는 게 안 들어맞는 경우가 많고 이론이 설명하지 못 하는 예외적 현상이 많다.
전공자들이 밥벌이할 수 있는 분야는 사실 바로 그 예외다.
\vspace{5mm}

다른 걸 떠나서 옥시만 문제겠나.
지금 언급되지 않지만 '전자파' 공해 문제도 부각 안 되어서 그렇지 이것도 털면 재밌는 결과가 나올지 모르는데.
애시당초 흡연의 암유발도 과거에는 그럴 리가 없다고 부인하다가 미국에서 뒤늦게 인정했는데
그건 이미 미국의 담배회사가 해외로 눈돌린 이후였다(즉, 뽕뽑을만큼 뽑고나서야 정의를 실천했다는 미담이 되겠다)
\vspace{5mm}






\section{[헬조선 005] 전문가}
\href{https://www.kockoc.com/Apoc/762604}{2016.05.06}

\vspace{5mm}

현실에서는 그 전문가들이 바로 악의 세력인 경우가 많아서리.
사태 터질 때마다 반복되는 논리가 "이게 다 검증 잘못해서이다, 전문가에게 맡기면 된다"라는데
그런 사태가 전문성이 없어서 터졌다면 맞는 말입니다만 실제로는 전문성이 없어서가 아니죠.
\vspace{5mm}
\begin{enumerate}
    \item 황우석의 줄기세포 : 이 때 전문가들이 바로 황우석 집단이었죠. 국내에서는 검증 사실 불가. 해외에 알려져서야 실토당했습니다.
    \item 후쿠시마 원전 : 도쿄전력 인간들이 전문가들이 아니었나요?
    \item 옥시 : 그 전문가들이 주범이었고 전문가 중 전문가인 교수들이 돈먹은 정황도 포착되엇는데요?
\end{enumerate}
\vspace{5mm}

그럼 이건 전문가들의 문제. "그래도 전문가들에게 검증만 맡기면 된다"라고 하면 전혀 설득력이 없는 이야기죠.
이런 문제는 주체를 다각화하는 수 밖에 없습니다.
전문성이 떨어져서가 아니라 해당 지식의 접근 권한부터 시작해 권력을 산학연정만 갖고 있어서 그런 것이잖아요.
이 문제도 해당 산업과 이해관계가 떨어지는 의사들이 발견한 것입니다.
\vspace{5mm}

그런데 여기서 주류학문은 죄가 없다... 이게 신기하지 말입니다.
인문 사회과학이 자연과학이나 공학보다 나은 건, 적어도 '분식 가능성'이나 '거짓말'은 인정해요.
하지만 자연과학이나 공학은 그걸 별로 다루지 않죠. 과학자의 윤리나 공학자의 윤리 그 정도로나 이야기하지 그 외에는 싹 침묵.
생명과학의 잡다한 유전을 배우는 사람은 많아도 '리센코' 사건 등을 배우는 사람은 없죠(모르는 사람이 많을 듯)
\vspace{5mm}

옥시 사태는 검증을 잘못한 게 아니라 전문가들이 의도적으로 속이고 은폐한 겁니다.
이런 일이 이제 한두번 수준도 아닌 것 같은데 그래도 주류만 믿으면 된다라는 식으로 말하면 곤란하지 않나 싶고.
\vspace{5mm}

사실 거창하게 말할 필요없이 우리 주변의 전문가들도 다 알게 모르게 소소하게 남겨먹고 사기치는 경우가 많아서리.
업자-고객 정보가 비대칭적인 상황에서는 거짓말해먹는 게 일상임.
재밌는 건 자기들도 그게 문제라는 걸 알면서도 먹고살기 위해 사기칠 수 밖에 없다는 것.
\vspace{5mm}

어제 인상깊게 읽은 뉴스가 있어서리.
\vspace{5mm}

http://www.nocutnews.co.kr/news/4589174
\vspace{5mm}

세무조사 관련 내부 비리를 폭로해 시민단체로부터 상까지 받았던 대전의 한 세무서 간부가 뇌물수수 비리로 구속됐다.충남지방경찰청 지능범죄수사대는 부동산 매매업자에게 세무조사 관련 정보 등을 가르쳐 주고 금품을 받은 대전의 한 세무서 간부 공무원 한모(59) 씨를 뇌물수수 혐의로 구속했다고 4일 밝혔다.
이번에 구속된 한씨는 지방국세청 감사계장 시절이던 지난 2002년 말 기자회견을 열어 "국세청 상부의 압력으로 대기업 4곳에 대해 추징한 세금이 부당하게 면제되는 비리가 저질러졌다"고 내부 고발을 했던 인물이다.한 씨의 유죄 여부는 법원의 재판과정에서 드러나게 되겠지만, 현재까지 드러난 경찰 수사결과만을 볼 때는 영웅의 추락이라는 탄식을 듣기에 충분하다는 평가가 나오고 있다.
\vspace{5mm}

영웅의 추락 어쩌구 하는데 그건 알 건 없고 (젊은 시절 영웅 아니었던 독재자가 있으랴)
세무서 간부면 전문성과 그 분야 권력을 쥐고 있으니 저렇게 되는 건 너무 당연하지 않냐 싶은데 말입니다.
\vspace{5mm}










\section{[헬조선 006] 관료주의}
\href{https://www.kockoc.com/Apoc/769156}{2016.05.11}

\vspace{5mm}

앞서 과학에 대한 글을 썼을 때 그것이 실은 '관료주의'라고 지적하는 게 더 낫지 않았나 합니다.
전문가만 믿으면 된다는 그럴싸합니다만 더 깊은 함의가 있습니다. 그건 '관료주의'를 공고히 믿는다는 것이지요.
현실 세계에서는 결국 '직위'가 높은 사람들의 권력대로 따라가게 되어있습니다.
전문성을 담보로 그들을 비판하거나 견제하지 못 하면 그들 뜻대로 따라갑니다. 이로써 보이지 않는 관료주의가 진행되는 것이지요.
\vspace{5mm}

과학이란 말이 나오면 주의해야 합니다. '과학'이 아니라고 주장하지 않는 분야가 없기 때문입니다.
정말로 과학이냐고 얘기하려면 정말 그게 과학적이냐, 과학적인 사고방식과 성찰이 담보되는 것인가 얘기해야합니다.
그렇지 않으면 그건 눈속임으로서의 형식화된 과학에 불과합니다.
상대방의 반론을 차단하는 스테레오타입화된 과학이 되는 것이지요.
부정적 이미지가 강한 관료주의가 '과학'이란 양가죽을 뒤집어 쓰고 다니는 것도 그리 놀라운 일은 아닙니다.
그런데 과학적 사고나 성찰은 관료주의와 사이가 좋지 않습니다.
\vspace{5mm}

한데 이야기해보면 본인들이 관료주의적인 것에 빠졌다는 것을 모르는 경우가 많더군요
\vspace{5mm}

검증이라는 것은 전문가가 인정했으니 닥치고 믿으라하는 게 아닙니다. 비전문가도 납득할 수 있는 정보공개와 소통이 있어야합니다.
물론 그 정보공개와 소통까지도 거부해버리는 음모론자들이 없는 건 아니겠지만요.
관료들은 정보공개를 싫어합니다, 정보가 공개될 수록 자신들의 권한과 권력이 줄어들기 때문입니다.
과학 말고 민주주의라는 것도 역시 거죽인지 아닌지보려면 정보공개가 되어있는가 그걸 따지면 됩니다.
\vspace{5mm}

거창한 담론을 떠나서 한국에서 공부를 잘 한다는 건 '관료주의 수혜자'가 되는 것과 동급이다라고 해도 틀린 말은 아닙니다.
수입적 측면 이외에도 사람들이 공무원, 변호사, 의사에 몰린 이유는 가장 '관료주의적인' 직업이기 때문입니다.
첫째로 서열화가 되어있다, 둘째로 정보가 폐쇄적이다, 셋째 그들만의 리그가 가능하다.
이러니 충분히 갑질이 가능하다는 것입니다. 바꿔 말해 일반 양민들을 속이면서 착취할 수 있다는 특성이 따라옵니다.
\vspace{5mm}

여기서 한국 현실과 교육의 괴리가 드러납니다.
현재의 역사적 흐름은 관료제를 '붕괴'시키는 방향으로 이뤄지고 있습니다. 관료제적인 것이 완전히 사라지는 일은 없을 것입니다.
그러나 점진적으로 그 관료제가 무너지고 있는 건 사실입니다. 대신 시장의 힘이 강해지고 있어서 다수의 눈치를 봐야하는 시대가 오고있죠.
하지만 교육은 여전히 '관료'를 키우는 방향으로 이뤄지고 있습니다.
등수를 매기는 것부터 시작해서 학력의 서열화 자체가 예비 관료를 키우는 것입니다.
\vspace{5mm}

이런 흐름을 예고하고 공부한 사람들이 졸업하자 관료주의적인 것의 붕괴에 맞닥뜨리게 됩니다.
우리 교육은 시장(market)을 제대로 가르친 적이 없고 특히 메이커가 되는 것 역시 준비한 게 없습니다.
국어, 영어, 수학을 우리가 왜 공부하는가 돌이켜보면 이건 '상위 관리자'를 키우기 위한 과목입니다.
시장을 공부하려면 경영, 경제, 컴퓨터, 온갖 공학 기술미술, 제2외국어 등을 어린 시절부터 배워야하는데 이상한 게 아닌가요?
물론 국영수는 관료만 키우는 게 아니죠. 그런 관료주의적인 틀을 그 피지배자들에게 주입시키는 역할도 하죠.
\vspace{5mm}

나이처먹고 나니까 더 삐딱하게 봅니다만... 전문성을 강조하는 그 분들의 양심도 일부 없지는 않지만 본심은
\vspace{5mm}
\begin{itemize}
    
    \item[-] \textbf{복종하라}
    \item[-] \textbf{내놓아라}
    \item[-] \textbf{의심하지 마라}
\end{itemize}
\vspace{5mm}

사실 이 3가지입니다. 저 중에서 '문제가 되면 내가 책임져주겠다'라는 건 없습니다.
저 자리에 올라가기 위해 이 헬조선에서는 많은 청년들이 노오력을 하고 있고 그로써 저 관료주의가 단기적으로는 지탱되는 것이지요.
\vspace{5mm}

열심히 공부해 출세해 올라간다... 라는 발상 자체를 버려야하지 않나 합니다.
아니, 저건 또 아적아... 가 아니라 저 발상 자체가 관료주의 시작이어서입니다.
이제는 금수저들이 스펙놀이로 관료주의적인 것을 유지하려한다면 흙수저들이 할 수 있는 건 관료주의가 덜 먹히게 가는 수 밖에 없습니다.
\vspace{5mm}

여기서 논란이 되는 걸 적는다면 그 점에서 '신자유주의 비판'은 흙수저들의 발등찍기라고 할 수 있죠.
신자유주의적인 것이야말로 관료들이 정말 싫어합니다. 개방하고 경쟁시키면 관료제의 단꿀이라는 건 점점 줄어들기 때문입니다.
그러나 대한민국에서는 이런 것이 악으로 포장되면서 경쟁을 줄이자라는 식으로 역으로 관료주의적인 것을 부활시키고 있습니다.
젊은이들을 위하는 척 하면서 자본의 음모를 막는다느니 이 사회가 희망의 씨앗을 뿌려야한다느니 라는 사람들이
어떤 조직에서 어떤 권한을 갖고 있나 그걸 유심히 보는 게 정확할 겁니다. 그 사람들은 절대 자기들에게 불리한 말은 하지 않습니다.
\vspace{5mm}





\section{[헬조선 007] 민낯}
\href{https://www.kockoc.com/Apoc/774887}{2016.05.15}

\vspace{5mm}

과거총정리 : \url{http://onepageinfo.tistory.com/74}
\vspace{5mm}

문예상 :  \url{http://news.donga.com/3/all/20070215/8407888/1}
\vspace{5mm}

공부를 잘해 선생님의 귀여움을 많이 받았고 명문대를 나왔지만 너무 일찍 돌아가신 아버지, 생계를 위한 직장 생활에 지쳐 있던 어머니, 친척집을 전전하던 어린 시절은 스스로 도저히 치유할 수 없을 거라고 여겼다는 것.
그러던 어느 날 영화 ‘바그다드 카페’를 보고 나서 행복의 조건은 외적인 것에 있는 게 아니라, 다른 사람에게 따뜻한 관심을 보여 주는 것만으로도 세상을 달라지게 할 수 있다는 깨달음을 얻었다고 한다.  법관 정기인사로 서울중앙지법으로 발령을 받은 최 판사는 14일 통화에서 \textbf{“세상이 바뀌었다고 하지만 외적인 조건이나 돈보다 귀한 가치가 있다고 믿는 사람이 더 많다는 걸 보여 주고 싶어 이 글을 썼다”}고 말했다.
\vspace{5mm}

개원 : \url{https://www.lawtimes.co.kr/Legal-Info/Print-Legal-Info?serial=9458&type=Trends&tab=1}
\vspace{5mm}

그리고
\vspace{5mm}

체포 당시 경찰관의 얼굴을 할퀴고 팔을 물어뜯는 등 격렬히 저항했던 것으로 드러났습니다
\vspace{5mm}

돈에 환장한 사람과 그렇지 않은 사람, 특히 위선자는 잘 걸러낼 필요가 있다.
똑똑하든 인격이 좋건 그건 상관이 없다.
첫째로 돈이 얼마나 소중한지, 중요한지, 그리고 무서운지 알아야 한다.
둘째로 돈을 잘 다스리되 절대로 돈에 영혼이 팔리지 말아야 한다.
하지만 이건 말만 쉽지 매우 어려운 일이다.
\vspace{5mm}

개인적으로는 그런 위선자들은 경멸하는 걸 넘어서 그냥 아무 말 없이 관찰한다. 관찰기간은 1$\sim$2년이 아니라 10년이 넘는다.
그러면서 느끼는 것이지만 옛말은 정말로 틀린 건 없다는 것이다.
어떤 사람이 돈을 많이 번다면 그걸 부러워하는 척만 해야한다.
그 무대의 2편에서 그 사람은 돈 때문에 몰락하는 장면이 그려질 수 있어서이다.
\vspace{5mm}

흔한 경제, 경영 퀴즈에서는 이런 이야기만 한다. 10$\%$의 100억이냐, 100$\%$의 5억이냐. 그리고 전자를 택해야한다고 말한다.
계산만 보면 그렇다. 그러나 이런 퀴즈의 문제는 항상 그렇듯 숫자 이외의 것을 무시하는 것이다.
일단 저 퀴즈만 가지고는 정답을 낼 수 없다. 문제는 도박으로 100억이냐 안정적으로 5억이냐 그게 아니다.
그걸 선택하는 사람이 그런 배포와 기량이 있으며 그릇이 크냐, 그리고 정말로 돈에 끌려다니지 않느냐 하는 게 중요한 이야기다.
\vspace{5mm}

돈은 일종의 흐름이다. 돈을 많이 버는 사람들은 자기가 능력이 좋아서라고만 착각하기 쉽다.
그러나 유능해도 돈을 못 버는 사람, 무능해도 돈을 잘 버는 사람이 있다. 중요한 건 \textbf{'어떤 흐름'을 타고 있느냐}는 것이다.
돈의 액수만 신경쓰면 나쁜 흐름이 뭔지 알지 못 한다. 막말로 돈을 많이 번다고 해서 오피스텔에서 일하는 소녀는 없을것이다.
그러나 막상 오피스에서 일하는 여자들은 그만두지 못 한다. 그 돈이 너무 달콤한 나머지 스스로 빠져나올 수 없기 때문이다.
그래서 나쁜 흐름을 탈 때에는 조기에 빠져나오지 않으면 답이 없다.
권력과 돈과 섹스의 흐름을 잘못 탄 사람들은 정말 헷가닥 맛이 가버린다.
자기들은 절대 그럴 리가 없다고 자신할 것이다.
\vspace{5mm}

돈을 무시하라는 이야기는 아니다. 그러나 나쁜 흐름을 타면서 많이 버는 사람을 경멸하지 않으면 '똑같아'질 수 있다.
\vspace{5mm}








\section{[헬조선 008] 혐오발언은 계층화의 시작(메갈의 탄생)}
\href{https://www.kockoc.com/Apoc/782301}{2016.05.19}

\vspace{5mm}

본격적으로 여혐 증후군이 터진 것은 1999년 군가산점 논쟁이었습니다.
한가지 말하면 그 당시 헌소를 제기했던 이대 학생 편에 섰던 '이석연 변호사'조차 군가산점 폐지에 대한 실질적 정책이 있어야한다면서
이걸 아쉬워하고 있다는 점에서 결과적으로는 뭔가 잘못된 처리이긴 한데 그것보다 더 중요한 것은,
그 때부터 여자들도 공무원 시험 응시율이 높아지는 등 남자들의 일자리를 위협하기 시작했고,
지금 젊은 분들은 모르겠지만 그 전까지는 여자들이 남자 조건 따지면서 결혼하는 풍토가 덜 했는데
IMF 이후부터 여자들도 남자들의 스펙을 따지기 시작하면서 짝짓기하지 못한 '루저남'들이 생겨납니다.
\vspace{5mm}

그 과정에셔 여혐이 보편화된 것이죠. 된장녀 논쟁이 그렇게 터지고 최근에는 김치녀 담론으로 표준화(?)되었습니다.
\vspace{5mm}

그런데 이제는 메갈을 통해 남혐이 보편화되기 시작합니다.
정말 소수 멧퇘지(...)들의 소행만은 아니죠. 다 은밀히 공감해주고 있으니까 세불리고 여론전에서 한몫하는 게 아님?
그럼 남혐이 왜 생겼냐, 위 공식대로입니다.
\textbf{여자들도 이제 루저들이 생겨나기 시작했다} 그것입니다.
예전에는 몸만 오면 받아준다는 남자들이 여자들의 외모, 키, 사이즈 뿐만 아니라
집안, 학력, 소득을 따지기 시작하면서 루저녀들도 생겨납니다.
골드미스니 그런 건 환상이고 이제 여자들도 30대 건어물녀들이 생겨나기 시작했단 것이죠.
여자들이 바라는 이상적인 신랑감들은 모조리 20대 초반 여자들에게만 관심있지 20대 후반 루저녀는 관심이 없죠.
육체나 실력 면에서 좋은 남자에게 채택받을 수 없는 루저녀들이 어디다 화풀기 시작하겠습니까?
\vspace{5mm}

어떻게 본다면 여혐하는 남자들 입장에서는 메갈의 탄생을 좋아(?)해야할지도 모르죠. 여자들도 똑같이 당하는 것이니까.
\vspace{5mm}

기존의 정사갤, 코갤, 그리고 일베의 루저남들 주장은 그거잖아요.
\textbf{여자들은 결국 우리 루저남들에게 순종해야 한다.}
이건 메갈도 마찬가지입니다. 우리 여자들이 차별받으니 사회적 약자이니 더 많은 혜택을 누려야 한다. \textbf{걍 남자들이 호구가 되어라}.
물론 조금만 생각하면 저 주장이 터무니없는 걸 떠나서, 현실화되기 힘들지요.
\vspace{5mm}

사회 현실은 어떨까요.
위너남은 위너녀랑만 결혼을 하려 합니다. 위너남 입장에서 루저녀는 노골적인 표현을 쓰면 '1회용품'입니다.
루저남이 위너녀랑 맺어질 가능성은 제로로 수렴합니다(이 환타지가 일본에서 히트쳤던 드라마 '전차남'이지요)
그리고 루저남이 루저녀랑 맺어진 결혼 생활은 불행해질 가능성이 높습니다.
\vspace{5mm}

그럼 이 자본주의 사회현실을 개혁하자... 는 것도 판타지고
20대 초반이신 분들은 열심이 노오력해서 루저가 되지 않는 길 밖에 없다... 그게 답입니다.
위너들끼리야 서로 하하호호할 터이니 혐오발언이 생기겠습니까.
위너남이 위너녀에게 더치페이타령하겠습니까.
\vspace{5mm}

그러니 루저가 되기싫으면 헛소리하지말고 공부나 하라 6평 틀린 갯수에 1000을 곱한 만큼 뛰고 와라하는 교훈적 결론이 되겠습니다.
하라는 공부는 안 하고 쓸데없는 것에 관심갖는 사람들보면 정말 한숨이 나오는 게 아니라 태풍이 불 지경임.
공부할 수 있는데 공부 안 하고 루저가 되려는 분들은 사람들로 안 보임.
\vspace{5mm}

+
과거에는 몸만 가면 받아주엇냐... 1990년대 중반까지는 그랬음.
한 남자가 한 여자와 첫관계하면 무조건 결혼 확정임. 뭐 예외도 없지 않았겠지만 그러는 게 너무나도 당연한 분위기였음.
어떻게 보면 미개하다가 할지 모르지만, 이렇게 가면 남자와 여자가 서로 스펙을 따질 수가 없죠.
스펙 따지는 것도 연애를 많이 하고 그 후보들을 비교해봐야 가능한데 처음 만난 이성과 19금 가고 생식코스로 가는 데 그럴 겨를이 있음?
그런데 지금은 '연애'란 이름 하에 각자 시제품을 이용해보지요. 그리고 꼼꼼히 따져보니까 남녀가 위너, 루저로 갈라지는 겁니다.
2000년대까지는 여자들이 유리했음. 왜냐면 남자들이 권력을 잃어가던 시기였고 여자들의 권리가 신장되는 흐름이었으니까.
그런데 2010년부터 이게 역전됩니다. 여권신장도 거의 끝났고 남자들도 영악해졌음.
게다가 서로 눈맞아 어쩌구한다... 는 건 냉정히 말하면 남자들만 유리하거든요. 그래서 루저녀들이 더 생겨남요.
경제력있는 남자들은 한참 연하만 바라보고 있음. 그래서 그 와중에 버림받은 세대녀들이 메갈리안 쪽으로 활동하게 됩니다요.
\vspace{5mm}

++
남혐을 하건 여혐을 하건 메시지 뿐인데 그게 뭔 상관있다고 그러시는지들.
그거 별로 소용없습니다. 그리고 전 남혐 발언도 걍 어 그런가보다라고 생각하고 경청할 건 경청함.
그런 극단적인 발언들일수록 오히려 진실된 측면도 없지는 않거든요.
루저들은 혐오발언 해보았자 사실 달라지는 건 없고 자기 인생만 낭비된다는 것을 알지 못 하죠. 그 시간에 공부하는 게 백배 나을 건데.
사실 메갈에서 활동하는 그 코어 분들이야 10년 전이야 젋었으니까 아무 노력없이 '젊은 여자'라는 이유만으로 먹고 들어갔겠죠.
그러나 세월이 흐르면... 세상 참 잔인한 겁니다.
어차피 남자 여자 모두 나이먹고 노화하니 열심히 공부해서 실력을 쌓아 그걸로 버텨야하는데
그런 건 안 하고 자기 몸뚱이만 믿고 여성들의 공산주의 페미니즘에 지나치게 경도되었다가 정신차리고보니 뭐.
\vspace{5mm}

+++
그러고보니 흠좀무한 게 성형수술 유통기한도 끝나갔다는 것도 참.
2000년대에 여자들이 그렇게 실리콘밸리에 투자해서 몸값 높여서 갔는데 그것도 약빨이 떨어지기 시작할 시점임.
반면 그 때 루저들이라고 까인 남자들 중에서 죽어라 공부해서 뒤늦게 올라간 사람들은 한창 어린 여자 찾기 시작할테고
걍 돌고도는 자본주의 사회 순환 법칙임.
\vspace{5mm}

++++
그리고 사실 생각해보면 순결이란 게 저런 스펙싸움을 막아주는 역할을 역설적으로 해주었다는 것도 재미있네요.
왜냐면 지금 기준에서 예쁘고 키크고 사이즈 좋은 여자들은 연애경험이 많을 수 밖에 없고 그러니
순결론자 입장에서는 slut이라고 까이면서 강제적인 '평등'이 이룩되는 건데.
지금은 그런 걸 따지지 않으니까 저런 여자들이 연애경험이 많든 적든 잘 나가는 것이죠.
그런데 이걸 지적하고 예측하신 분은 아무도 없었음.
\vspace{5mm}

+++++
대략 지금 20대 초반까지 여성분들은 윗 세대와는 다를 겁니다. 세대 차이가 확실히 느껴지거든요.
그 윗세대부터가 문제지. 앞으로 심각한 사회문제화될 거라고 여겨졌는데 이미 되었다고 생각함.
그 세대가 메갈 등을 중심으로 아랫 세대들도 같이 엿먹여보자고 선동하고 있는데 그 아랫 세대도 호락호락할 리는 없죠.
\vspace{5mm}

20대 후반부터 30대 후반까지야 '여권신장'의 흐름을 타고 올라간 세대지, 자기 능력으로 올라갔다고 보기는 힘들 거든요.
그래서 실제로 실력도 없으니 지금 도태되는 거지. 경쟁하지 못 하니까 여자들은 약자거든요.. 라고 하면서 받아먹으려고 함.
그런데 그 아랫 세대 여성들은 남자들보다 더 적극적으로 교육받으면서 경쟁에 익숙해졌음.
경쟁을 당연히 하고 실력이 있으니까 정정당당하게 권리 쟁취를 하지 윗 세대처럼 여자들이 약자 타령은 별로 많이 하지 않을 것임.
\vspace{5mm}

그러니 기승전공입니다, 공부나 합시다.
\vspace{5mm}








\section{[헬조선 009] 산부인과 참관문제}
\href{https://www.kockoc.com/Apoc/788842}{2016.05.23}

\vspace{5mm}

이건 입장에 따라 달라지겠지만.
대학병원이 갑자기 의대생들이 들어와서 환자 몸을 사실상 볼 수 있다는 것을 모르는 사람들이 훨씬 많습니다.
사람들이 무턱대고 다 대학병원에 가는 것도 아니니까요.
\vspace{5mm}

그런데 이 논리는 좀 아닌 듯.
\vspace{5mm}

\textbf{"동의하라고 하면 다들 어차피 동의 안 할텐데 그럼 교육은 어찌하냐"}
\vspace{5mm}

저 얘기부터가 환자들이 수치심을 가진다라는 것을 본인들도 인정한다는 건이고 그럼 문제가 있다는 걸 알지 않습니까.
그리고 의사가 무슨 무급봉사직도 아니고 교육문제가 심각하다 싶으면 그건 자기들이 알아서 해결해야죠.
하다 못해 그 의사들부터 그럼 자청해서 교육을 위해 헌신하는 모습을 보여주든가 하는데 제가 듣기로는 전혀(...)
다시 말해 의사 사모님들부터 그 학생들의 참관 대상이 대든가 해야죠.
\vspace{5mm}

그리고 \textbf{만약 환자가 VIP라거나 상류층 여성이더라도 참관 시킬지}는 의문이지 말입니다.
인공지능이라면 하겠지만 인간이라면 저럴 가능성이 적죠. 일반인들도 수치심을 느끼는데 VIP나 상류층이 잘도 그러겠습니다.
\vspace{5mm}

그래서 사이트들 돌아보면 \textbf{"의사님들 마누라, 여동생, 따님부터 교보재로 쓰세요"}라는 댓글에 대답은 전혀 못 함.
그저 한다는 이야기가 동의 받게하면 교육이 안 되니까 정상적인 의료가 힘들다...
\vspace{5mm}

솔직히 말해서 이건 특권의식이라고 밖에 보이지 않네요.
다른 직종조차도 당연히 교육이 필요합니다만 이 정도로 동의 안 받고 하는 경우는 제가 아는 한 없습니다.
그리고 의사님들이 힘들다 어쩐다 하는데 그건 다른 직종도 마찬가지이죠.
문제는 의사님들이 일 힘든 건 알지만 '무급봉사'라고 하는 건 좀 말이 안 되는 이야기라서리.
\vspace{5mm}

이 문제는 의사님들 마누라 여동생 따님부터 VIP, 상류층 여성, 심지어 대통령 장관 재벌 여성들도 '교보재'가 되어주면 깔끔히 해결됩니다.
그런데 그랬다는 이야기는 전혀 듣지 못 해서 말입니다.
\vspace{5mm}








\section{[헬조선 010] 시체팔이}
\href{https://www.kockoc.com/Apoc/789064}{2016.05.23}

\vspace{5mm}

시체팔이의 근원은 유교이지요.
\vspace{5mm}

유교에서 뭐라고 하던 그 핵심은 "장례"이죠.
어떻게 상을 치를 것인가, 그리고 그 족보를 짤 것이냐 하나 가지고 정치경제문화를 모두 쌈싸먹으니까요
물론 모든 시체를 평등하게 취급하는 건 아닙니다. 혈통이 좋거나 비싸게 팔릴 수 있는 시체만 취급해오지요.
\vspace{5mm}

우리나라에서 볼만한 문화재라는 건 유감스럽지만 산 자보다는 죽은 자를 위한 것에 가깝습니다.
온갖 왕릉과 사찰도 그렇고 병풍 족자 그림 대부분이 무덤과 관계없는 게 없습니다.
수원성조차도 융건릉을 위한 것이었죠.
\vspace{5mm}

죽은 자 중심으로 정치 경제 문화가 돌아가니 조선이나 중국이나 정체될 수 밖에 없는 것이죠.
\vspace{5mm}

그럼 왜 망자 중심으로 돌아갈까요.
시체팔이는 매우 기막힌 사기캐입니다. 비판이나 반론을 원천차단하거든요.
죽은 자를 추모한다는 명목 하에서 상궤를 벗어난 일을 해도 제제받지 않습니다.
그리고 그런 것이 너무하지 않느냐하면 고인모독이라고 탄압받습니다.
\vspace{5mm}

조선이 백성을 위한 나라였다고요? 이미 썩어버린 왕족과 양반들의 시체를 위한 나라였습니다.
당시 엘리트들이 몰두했던 음양오행과 풍수지리는 조상을 좋은 데 잘 모시면 자손들이 발복한다라는 믿음 하에 죽은 자에 치중해있었고
그래서 묏자리를 둘러싼 산송문제가 정말 지겹게 일어났다 하죠.
풍수지리도 만약 산 자를 위한 쪽으로 나갔다면 그건 우연히 자연과학적인 것으로 재편되었을 수도 있습니다.
그러나 애당초 '죽은 자'를 대상으로 하다보니 관념론으로 흐르고 그래서 별 이상한 것에 신경쓰기 시작했죠.
\vspace{5mm}

죽은 자를 위한 나라가 근대화 할 수 있었을까.... 그건 무리수입니다.
한반도가 일제의 식민지배와 6.25까지 겪은 건 비극이긴 합니다만 역사적으로 보자면 긍정적인 것이 없지 않습니다.
왜구라고 비하하던 일본에게 지배당하고 거기다가 사람이 어이없이 죽어나가는 한국전쟁을 겪고 나니
'시체팔이'로 돌아가던 유교적 지배체제가 사실상 무너졌기 때문입니다.
\vspace{5mm}

물론 북쪽의 어느 나라는 공산주의 한다면서 \textbf{'죽은 자'를 위한 나라}로 되돌아가버렸죠.
\vspace{5mm}

요즘 와서 대한민국도 다시 그런 아름다운 전통으로 돌아가려 합니다.
논리적으로는 자기들이 하는 게 말이 안 된다는 걸 깨달으니까 그런 식으로 '망자'를 팔아먹는 작업으로 돌아서는 것입니다.
그리고 유감스럽지만 꽤 성과(?)가 좋습니다.
갓 쓰지 않고 도포만 입지 않았지 "죽음"을 이용하면 막 나가는 행패를 부려도 대우받을 수 있다는 걸 그들은 알아버린 거죠.
그동안 기쓰지 못 하던 유교 바이러스가 이렇게 변종화되어 창궐하기 시작합니다요.
\vspace{5mm}

살아있는 사람은 평등해야한다고 한다면 죽은 사람도 평등해야 할 터인데
재밌는 건 산 사람들의 인권을 중시한다는 사람이 "죽은 사람"은 차별한다는 것이 아닐까 싶고
\vspace{5mm}

그 점에서는 기독교가 지배했던 유럽이 나았던 건
적어도 얘들은 산 자여도 죽은 다음만큼은 평등했기 때문이 아닐까 합니다.
물론 시신을 어디에 모시느냐 차이는 있어도 죽은 다음에는 똑같이 심판받는다 라고 하니 동아시아처럼 그렇게 극성을 부리진 않았죠.
그렇기 때문에 당대 엘리트들이 예송논쟁 따위 하지 않았고 살아있는 사람들의 '자본증식'에 신경쓸 수 있었던 것입니다.
\vspace{5mm}

그런데 조선은? 유럽 애들의 사정권에 들어산 시기에 개혁군주라는 정조가 막대한 예산으로 자기 아버지 묘에나 신경씁니다.
수원성? 지금 보면 꽤 유명한 문화재이지만 어떻게 보면 그가 꽤 반동적인 군주였다는 얘기가 됩니다.
하지만 사람들은 다수가 정조가 개혁군주라고 착각하고 있죠.
\vspace{5mm}

그런데 지금 우리나라도 할 말은 없습니다. 죽음으로 장사하려는 사람들이 널려있거든요.
\vspace{5mm}

+ 그런데 조선시대 노비들이 죽은 뒤에는 과연?
들개나 물고기의 단백질 공급원이었다고 하면 지나친 표현인지도 모르겠지만요.
\vspace{5mm}

++ \textbf{수입 담론 +  시체 +  폭력사태 유도}
이것만 있으면 우리나라에서는 갑질도 가능합니다. 이게 정말 한두건이 아니라는 것이 중요
구체적으로는 얘기 안 하겠습니다. 이런 유형이 아닌 걸 찾기가 더 힘들어서리.
\vspace{5mm}

쓸모있는 담론이야 설득력이 있고 현실적중력이 높으니 시체도 필요없습니다.
쓸모없는 담론들은 감정에 호소해야하고, 그 점에서 '죽음'과 '폭력' 만한 떡밥이 없습니다.
시위할 때 대부분은 시위대가 폭력을 조장하죠. 그래서 폭력사태 유발을 해서 시체가 생겼으면이라고 유도합니다.
(그런 걸 바라지 않는다면 애당초 폭력시위는 하지 않았지)
\vspace{5mm}

물론 경찰이 죽은 건 취급 안 합니다. 자기들 시체가 아니거든요.
\vspace{5mm}





