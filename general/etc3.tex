
\section{커뮤니티 활성화 방법 $-$ 자존심 상처입히기}
\href{https://www.kockoc.com/Apoc/766833}{2016.05.09}

\vspace{5mm}

\textbf{키배 조장 $\rightarrow$ 자존심 환자 발생 $\rightarrow$ 게시판 헤게모니 욕망 $\rightarrow$ 악순환(?)}
\vspace{5mm}

커뮤니티 활성화를 위한 양적완화술 중 하나.
게시판에서 키배가 벌어지면 거기서 패한 사람이든 이긴 사람이든 자존심에 사로잡힘.
자존심에 사로잡힌 사람들(특히 남자들)은 게시판 승부에 환장함.
\vspace{5mm}

타인이 보기에 보잘 것 없는데도 특정 커뮤니티에 집착하는 사람들이 많음.
왜냐며 그 사람들은 그 커뮤니티에서 상처를 입었기 때문임.
그래서 과장해 말해 그걸 인생의 목표로 삼고 온라인 활동을 하기 시작합니다.
\vspace{5mm}

이거 남의 문제로 보지 마셈. 일부가 아니라 \textbf{대부분}이 그렇습니다.
게시판 뿐만 아니라 과거 싸이월드도, 블로그도, SNS도 다 그렇습니다.
이 맛에 중독되어버리면 그 이후로는 답이 없어집니다. 중독의 배경은 다름아닌 자존심이지요.
당연히 그런 건 본인 인생을 위해 그만둬야 하는데 그런 건 근절되지 않죠.
\vspace{5mm}

본인이 사이트 열어서 장사치되고 싶다하면 이 방법 쓰면 됩니다. 다만 칼부림 날 가능성이 있죠.
참 별 것도 아닌 것 같은데 키보드 몇자로 마음이 지옥과 천국을 오가는 겁니다.
이건 게시판 관리자의 역할이 매우 큽니다.
게시판에서 서로 의견 다른 걸 존중하고 그냥 토론은 토론으로 끝나게 하고 그 이상 가면 무조건 컷시켜야
작성자들도 하면 안 되는 것임을 알고 그러지 않는데
그런 것을 방치해버리면 그 다음부터는 만인의 만인에 대한 투쟁으로 바뀌는 것이죠.
\vspace{5mm}

게다가 저런 상태에서는 어느 커뮤니티에서든 핵심적인 역할을 하는 침묵하는 능력자들이 질려서 떠나버리지요.
활동 회원을 보는 척도는 그 회원의 "공헌도"와 "비판도"를 종합해서 따져야합니다.
\begin{itemize}
    \item[] 공헌도가 높고 비판도가 높다면 실천자
    \item[] 공헌도가 높고 비판도가 낮으면 종속자
    \item[] 공헌도가 낮고 비판도가 낮으면 방관자.
    \item[] 공헌도가 낮고 비판도가 높으면 \textbf{파괴자}.
\end{itemize}
\vspace{5mm}

당연히 실천자를 늘려야합니다. 파괴자는 언뜻 그럴 듯 해보이지만 그냥 암세포입니다.
그렇기 때문에 자존심 환자들을 양산하는 전략은 피해야 합니다.
\vspace{5mm}






\section{고민의 절반 이상이 과거에 집착하는 겁니다.}
\href{https://www.kockoc.com/Apoc/766840}{2016.05.09}

\vspace{5mm}

아래 적은 역산적 사고법을 참조해주시고 그에 따라 행동하길 바라고.
\vspace{5mm}

과거는 전혀 도움이 되는 것이 없습니다.
과거에 사둔 땅이 올랐다거나 주식이 배당금을 준다는 것도 과거가 아니라 '현재'에 약속된 것입니다.
엄밀히 말하면 모든 결과는 '선택'과 '실천'할 때 이미 결정되는 것입니다.
강령술한다고 시체가 살아 움직이는 것도 아니죠. 간혹 죽었다고 착각한 케이스가 있을지는 몰라도.
\vspace{5mm}

사람을 망치는 게 자존심과 과거 집착입니다. 이게 도움이 되는 경우는 단 한건도 없어요.
흔히 하는 착각이 자기가 죽어라 하면 과거의 실패조차 성공을 위한 자산이 된다고 하는 것인데 헛소리입니다.
그건 어디까지나 성공했으니가 재평가하는 윤색에 불과하지, 정말 그게 자산이라면 지금 도움이 되어야죠.
\textbf{나는 삼수, 사수했으니까 다음에는 입시 성적이 좋을 것이다라는 게 가장 흔한 미신}입니다.
그런데 바꿔 말해 3번, 4번 실패했으니까 실패할 확률도 크다라고 하는 게 더 정확한 이야기죠.
\vspace{5mm}

과거는 과거일 뿐이라고 생각하되, 현재 지금 자신의 상태를 정확히 직시하는 것이 역전을 위한 최소한의 기본조건입니다.
자신의 단점은 자존심을 인질로 삼고 있습니다. 그 이야기는 자존심을 버리면 단점도 처리할 수 있단 이야기죠.
그러나 자존심을 버리는 사람은 \textbf{정말로 몇 없습니다}.
자존심을 버리는 사람은 없지만, '포기당하는' 사람은 물론 많죠. 정말 제대로 망해버린 뒤라면 말이죠.
\vspace{5mm}

서른이나 마흔이라고 가정하고 지금 어떤 걸 했어야 덜 후회할까 하는 걸로 판단하시면 됩니다.
이게 이해가 안 가면, 자기가 \textbf{지금 가장 하고싶지 않은 선택이라거나 자기 자존심상 허락할 수 없는 것을} 택하면 됩니다.
그게 알고보면 가장 현명한 선택인 경우가 많습니다.
\vspace{5mm}



\section{1년 뒤 일은 고민해보았자 무의미하다.}
\href{https://www.kockoc.com/Apoc/769099}{2016.05.11}

\vspace{5mm}

한국의 주력 산업이 중국에게 털리고 있을 거라는 예측은 10년 전에 겨우 나올까 말까였죠.
짱깨라고 업신여겼지만 지금은 주권을 내주면서까지도 중국인에게 굽신거려야 할 시절이 와버렸습니다.
\vspace{5mm}

그런데 이런 정말 중요한 걸 10년 전에 알고 예비한 사람이 몇이나 있을지.
\vspace{5mm}

개인 차원에서 1년 뒤에 어떤 일이 벌어질지는 아무도 모릅니다.
그 변동성을 줄이기 위해 조직에 들어가거나 학업을 밟습니다만 그래도 궁극적으로는 변동을 피할 수 없습니다.
미지의 변동에 대비하도록 늘 준비할 수 밖에 없죠.
\vspace{5mm}

그런데 \textbf{하라는 준비는 안 하고} 쓸데없는 고민을 하는 사람들이 많습니다.
수험사이트 학생들을 보면 미래에 일어날 일을 너무 정확히 알면서 '고민'하고 계십니다. 그런다고 준비 따위 하는 것도 아니지만요.
준비는 안 하고 \textbf{어떤 직업이 좋나 서열놀이나 하고 앉아있죠}. 쓰레기라고 불려도 별로 억울할 것 같지 않습니다.
\vspace{5mm}

1년 뒤는 커녕 사실 내일, 아니 1시간 뒤에 무슨 일이 일어날지도 아무도 모릅니다.
그나마 계획을 세워 통제한다고 하더라도 예측불허의 사건은 늘 터지게 되어있습니다.
예측불허의 사건이 안 터지는 방법은 우리가 전지전능해서 미래를 모두 알고 있어서 피하면 되는 것이죠.
그러나 우리는 전지전능하지 않죠. 그러니 미지의 사건과 조우할 수밖에 없습니다.
\vspace{5mm}

저렇게 미래를 훤히 안다는 것의 문제는 '자기예언 실현의 함정'에 빠진다는 것입니다.
"나는 뭘 해도 공부할 수 없어, 걍 편의점 알바로 살아야겠다"라고 좌절하면서 그 고통을 즐기는 사람은 정말 편의점 알바만 합니다.
의도적으로 그렇게 자기 시야를 좁히고 노력을 안 하죠. 예, 뇌가 그렇게 시키는 겁니다. 그게 편하거든요.
\vspace{5mm}




\section{일본 로스쿨 정원미달}
\href{https://www.kockoc.com/Apoc/772439}{2016.05.13}

\vspace{5mm}

\href{http://news.naver.com/main/read.nhn?mode=LSD&mid=sec&sid1=104&oid=001&aid=0008395996}{링크}   

\vspace{5mm}

우리나라는 정확히 일본을 따라가죠.   전여옥의 일본은 없다라는 책에서 깠던 일본의 온갖 음란한 풍습, 반도민들이 더 해먹었음.   원조교제한다고 일본 애들 문란하다고 깠지만 한국은?   성상품화한다 어쩐다 하는데 여학생의 성을 상품화한 아이돌은 더 잘 팔아먹고있죠.   

\vspace{5mm}


세대간의 특성이나 직업문제도 그렇죠.   재밌는 건 15년 전만 하더라도 여자가 무슨 바깥 일이냐, 살림이나 해야지라고 하면서 전업주부가 암울하다가 했는데   일본에서는 오히려 전업주부야말로 여성의 상위계급이라는 이야기가 미우라 아츠시의 [하류사회]에서 나와서 기묘하다 했건만   지금도 여자들은 전업주부 이제 아무나 할 수가 없죠.   저것도 정확히 따라갈 겁니다.   한 때 많은 사람들이 보았던 드라마 [공부의 신]도    원래 일본 원작에서는 주인공이 변호사지만 돈을 벌 수 없어서 학교 재생사업에 나선 것이었죠.   (여담인데 이 작가 만화야말로 현대사회를 살아가는 훌륭한 교본인데 번역이 잘 안 되고 있죠)





\section{안중근 의사 논란}
\href{https://www.kockoc.com/Apoc/772476}{2016.05.13}

\vspace{5mm}

개인적으로는 아이돌은 별 관심이 없어서.
그런데 안중근 의사 모른다고 마녀사냥당할 것까지는 없을 것 같은데
\vspace{5mm}

그렇게 따지면 여기서 '\textbf{원태우} 열사'아는 사람 몇이나 있을지.
원태우 열사 모르지? 그럼 갈굼당해봐라하면 다들 억울하다 하겠죠.
\vspace{5mm}

사실 안중근은 세컨드죠. 퍼스트는 원태우 열사.
안양역 에스컬레이터에 부조로 새겨지신 분이죠. 그 근처 철로에서 이토 히로부미가 탄 열차를 전복시켜 죽이려 했던 대담한 분인지라.
쫄아버린 동료가 철로에 놓은 돌을 치워버리니까 짱돌로 이토가 탄 객실에 크리를 먹여 얼굴에 유리조각이 박히는 부상을 입힌 분입니다.
고문당해서 고자(...)가 되신 분이지만 조국의 광복까지는 보고 돌아가신 분입니다.
\vspace{5mm}

안중근은 모를 수 있느냐라고 하는 분에게는 그럼 원태우는 아느냐 따지면 되는 것이죠.
그래도 안중근이 유명하다... 라고 하는 건 그건 본인들이 역사를 적극적으로 알려는 게 아니라 수동적으로 배웠다는 것 밖에 되지 않죠.
\vspace{5mm}

사실 안중근도 꽤 논란이 많은 인물입니다. 이토 히로부미를 사살하긴 했지만 '반일'은 아니었어요.
오히려 안중근은 냉정히 말하면 '친일파'에 가까운 인물입니다.
재판 받을 때 자기가 '일본 천황을 위해서 간신 이토'를 죽였다라고만 이야기했지, 일본 제국주의를 까지는 않았거든요.
즉, 일본은 서양의 침략에 맞서 조선을 지켜주었고 그래서 러시아와 싸우지 않았느냐,
그런 착한 일본이 한국을 먹어치울 리 없고 일본 천황도 그럴 분이 아니다(참고로 안중근은 '일왕'이라고 하지 않고 '천황'이라고 대놓고 얘기합니다)
다만 이토 히로부미가 농간 부려서 그런 거다, 이토만 죽이면 모든 게 다 정상으로 돌아갈 것이다....
그래서 일본 천황이 자기를 죽이지 않고 살려줄 것이다라고 얘기하기도 했다고 하죠.
\vspace{5mm}

지금 보면 이만저만한 착각이 아니죠.
\vspace{5mm}

오히려 이토는 조선을 강제병합까지 시키려는 사람은 아니었습니다. '보호국'으로 관리하길 선호했죠.
사실 조선왕조 입장에서도 그 정도까지 타협을 보려고 했던 것 같다고 봅니다. 이토가 죽자 고종이 매우 애석하게 생각하고 예우를 기렸거든요.
어차피 나라도 작살나는 판이라면 왕조라도 존속시키는 게 남는 일이었는데
\vspace{5mm}

이토가 살해당하자 오히려 강제병합이 빨라졌죠. 이토가 살해당하자 오히려 일본의 강경파들이 더 마음대로 할 수 있었기 때문.
조선왕족은 일본 황족에 편입되었고(사실 조선왕조가 고려왕족들을 어떻게 살육했는가 반추해보면 관대한($-$$-$) 조치일수도)
그냥 깔끔하게 병탄당해서 사직이 무너져버렸죠.
\vspace{5mm}

사실 이토야 우리 입장에서는 원수이지만 객관적으로 보면 역설적으로 '진보'에 가까운 인물입니다(...)
메이지 헌법의 초안을 만들었거니와 무엇보다 '인권'을 강조했으며 전쟁보다는 외교를 선호했죠.
당시 조선의 엘리트들이 이토에게 괜히 넘어간 게 아닙니다. 원래 동양평화론도 안중근이 아니라 이토가 원조였어요.
이토가 원조 동양평화론으로 한중일이 합쳐서 서양으로부터 아시아를 보호하자라고 하면서 구워삶자 하니까 다들 헷가닥 넘어가버린 것이죠.
그러나 조선은 무조건 일본에 흡수되어야한다는 것보다는 '보호국'이라도 가는 게 낫다는 건 그 당시 엘리트로서는 할 수 있는 합리적 판단인데
이토가 죽자 이게 물건너가버립니다.
\vspace{5mm}

그렇다고 안중근 의사의 이토 사살이 폄하당할 건 아니지만 이게 과연 '치밀한 계산'에 따른 합리적 판단인가하는 건 다소.
안중근이 이토를 죽이면서 발표한 15가지 이유를 보면 당시 정치적, 국제적 상황을 정확히 알고 있다고 보기 어려운 점이 많죠.
예컨대 메이지 천황의 아버지 고메이 천황이 암살당했다는 루머를 그대로 담은 것도 그렇죠.
하지만 무엇보다도 거기에는 직접적인 반일은 없습니다. 모든 원인을 그저 이토 히로부미 개인에게로만 돌린다는 게 치명적입니다.
일본 제국주의 첨병이 바로 메이지 천황인 걸 간과하고 있었으니 말이지요.
\vspace{5mm}

일본인은 고대사 컴플렉스가 강하고, 중국인과 한국인은 근대사 컴플렉스가 강하단 말이 있죠
그래서 일본이 얘기하는 고대사, 한국 중국이 말하는 근대사는 걸러들어야합니다.
흔한 반박이 역사는 승자의 기록이니라고 하지만 그건 객관적 근거가 없는 한 하나마나한 이야기죠.
동아시아 역사 기술은 '패배' 컴플렉스를 화장하는 데 치우쳐져있고, 이건 공자의 춘추필법으로 거슬러 올라갑니다.
우리나라의 근대사는 고교 과정이라면 입시니까 어쩔 수 없이... 라고 쳐도, 입시가 아니라면 냉정하게 파악할 필요가 있어요.
\vspace{5mm}

+ 보호국화하더라도 어차피 일본이 조선을 먹기는 마찬가지였냐 아닌가 싶은데... 이것도 약간 국뽕적 시선이 있다고 봅니다.
이토도 성격이 좋아서 보호국으로 관리하겠다가 아니었어요. 이유는 간단합니다.
\textbf{한반도는 직접 관리하기에 수지타산이 안 맞는 곳이어서입니다.}
국토의 70$\%$가 산지고 평야가 적고 자원도 종류만 많지 물량은 별로입니다.
외교를 선호하는 파 입장에서는 한반도를 굳이 직접 관리해 돈들일 필요 없습니다. 협조만 받고 비옥한 만주나 중원으로 가는 게 낫죠
반면 전쟁을 선호하는 강경파 입장이야 다르죠. 호전파들은 보호국화한 조선이 통수칠 수도 있으니 \textbf{병참기지화}하는 게 답입니다.
그리고 그게 현실이 되었죠.
\vspace{5mm}

++ 또한 일제시대 수탈을 이야기할 때 조심할 게 많습니다. 가령 토지조사사업으로 농민들이 땅을 잃엇다라는 기술이 교과서에 되어있는데
이건 원래 조선에서 농민들이 양반들에게 종속된 상태였다라는 걸 은근히 은폐하고 있죠. 사실 달라진 건 크게 없어요.
수탈했다고 하는데 합병 이후 농지들을 독식하던 지주들은 잘 나갔습니다. 그래서 쌀도 일본 본토에 팔아먹어서 일본농민들을 울상에 빠뜨리죠.
교과서에서는 쫓겨나다시피 만주로 갔다고만 나오는데 이것도 딱히. 그 당시 만주는 드넓은 평야에 무주공산의 기회의 땅이었으니까 간 것이죠
일본이 만주를 지배하면서 중국을 침략할 때 조선인들도 한몫하면서 실리를 챙긴 건 사실입니다(...)
\vspace{5mm}

+++ 이토가 온건파건 아니건 뭔 차이가 있느냐하면서 기존의 역사관을 유지하려는 분들도 많습니다만... 그래선 발전이 없죠.
이토가 장기적으로 치밀하게 식민지화하려 했으니 더 잘 죽인 것 아니냐하는 논법은 두가지 문제가 있죠
첫째는 그건 이미 식민지화될 거라고 기정사실화하는 것입니다. 말이 좋아서 장기적인 것이지 이게 결코 쉬운 게 아닐텐데요
둘째는 우리 측도 일본에 완전히 복속당하지 않기 위한 \textbf{'시간'을 벌 수 있다}는 걸 간과한 겁니다(이게 가장 중요한 것일텐데)
\vspace{5mm}

그리고 일본의 노엄 촘스키인 마루야마 마사오부터가 이토가 죽어서 일본이 강경노선을 걸었다고 얘기했고
이에나가 사부로라고 태평양 전쟁을 반대한 사학자 역시 이토의 죽음을 베트콩이 베트남전 반대 운동가를 죽인 것이라고 얘기했죠.
극단적으로 말하면 안중근의 저격이 이런 나비효과를 불러왔다라고 볼 여지가 아예 없는 것도 아니란 것이죠.
그런데 이런 사실을 지적해도 그럴 리 없다라고 눈을 감는 사람들도 많죠.
그 눈감은 사람들의 양산이 바로 우리나라의 한국사 교육이라서 사실 한국사 교육은 그냥 안 시키는 게 낫다고 생각될 정도.
\vspace{5mm}

++++ 독립군을 너무 추켜세워주는 것도 문제인데.
6.25, 즉 한국전쟁의 전초가 이미 일제시대의 독립운동가들 사이의 '분열'에서 비롯되고 있다는 것도 역시 가르치지 않죠.
윤치호 일기에서 지적했던 바 당시 독립운동가들이 기호파와 서북파로 나뉘어 분열상태였고
그 와중에서 사상적으로도 민족주의파와 사회주의파로 갈라져 있다는 것.
이 분야를 공부해보면 이승만을 무시할 수 없다라고 생각되는 게, 그나마 국제정세를 가장 잘 읽은 사람이 이승만입니다.
심지어 일본이 설마 미국을 치겠냐라고 생각조차 하지 않았던 당시에 JAPAN INSIDE OUT을 써서 일본의 해악을 경고한 사람이 이승만.
그 외 나머지는 사실 객관적으로 따지면 정세를 읽었다 보기 어렵죠
\vspace{5mm}

\section{국가입장에서는 전문연 이제 해줄 이유가 없어요.}
\href{https://www.kockoc.com/Apoc/777839}{2016.05.17}

\vspace{5mm}

논쟁거리가 될 수 있는 글이지만 그냥 핵심만 적겠음.
\vspace{5mm}
\begin{enumerate}
    \item 전문연 자체가 "이공계 인력을 굴려서 고성장 시대에 과실 따먹는" 시대에는 유효한데 지금은 고성장 시대가 아니죠.
    \item 이미 '국내' 이공계 석박은 넘치기 시작했으며, 차후 교육과정에서 이과 문과 구분은 사실상 사라집니다. 초딩들은 소프트웨어 교육 의무화
    \item 정부에서 관심있어하는 대상은 해외 석박입니다.
    \item 기업에서 관심있어하는 대상은 국내 석박 싼값에 굴려먹기인데 \textbf{병특 없어도 가능해졌}습니다(그 때야 저런 데 취업 안 하려고 했죠)
    \item 원래 이 제도는 \textbf{특혜}였지 당연히 보장된 권리가 아니었죠.
\end{enumerate}
\vspace{5mm}

댓글보니 그냥 해외로 나간다고 하는데 사실 해외로 가는 걸 정부는 더 좋아할지도 모릅니다(...)
해외라 보았자 미국 일본 유럽인데 어디가든 일자리가 부족해지고 배타적이 되면 최상위 아니면 다 국내로 되돌아오니까요.
그리고 무엇보다 국민여론이 중요한데 이해관계자들 빼고는 반대 안 할 것 같은데(...)
소위 국익 이야기하면 사실 한류스타 빼줘야 한다는 딜레마 그 사람들 1인이 벌어오는 돈이 훨씬 더 많다는 부조리.
정부에서는 사실 특례 주고 싶으면 '\textbf{중동가는 젊은이}' 주고싶은 심정일지 모르겠고
\vspace{5mm}

이해관계가 조금이라도 있으신 분은 그냥 신속히 손절하거나 다른 방식으로 해결하는 게 좋아보입니당.
조심스럽게 적는다면 이거 만약 언론에서 취재 조사 들어가면 당사자들에게 더 불리하면 불리해졌지 유리할 건 없어보여서(...)
\vspace{5mm}






\section{전문연 논란}
\href{https://www.kockoc.com/Apoc/779846}{2016.05.18}

\vspace{5mm}

\href{http://orbi.kr/bbs/board.php?bo_table=united&wr_id=8435509}{오르비링크}
\vspace{5mm}

글이 어그로성이긴 하지만 딱히 틀린 이야기는 아니고
댓글들이 더 문제있다고 여김, 검색하면서 찾아볼 생각을 해야지 인신공격을?
그리고 이게 문제가 없는 제도만은 아니었다고 여기죠.
사적으로 보고듣고하는 것도 없지는 않기도 하지만 기사화된 적이 있습니다.
\vspace{5mm}

$\#$ 세계일보 단독보도 출·퇴근 멋대로…'군복무' 신분 잊은 연구요원들
\vspace{5mm}

국방부의 갑작스런 발표대로만 진행되지 않겠지만 이 문제에 대해서 사람들이 정말 놓치는 게 그것이죠. "정말 문제없이 잘 돌아가는 제도"였나.개인적으로는 국방부 발표는 언론플레이성이 다소 없지 않고 아울러 유예기간도 꽤 비현실적이라고 보기는 하는데그것과 별개로 저 제도가 정말 문제없이 잘 돌아가는 것인가라고 물으면 정말 마땅한 대답이 나올지는 의문.별로 비리가 없지 않나 하기에는 한국은 발견된 비리는 빙산의 일각 뿐이고 어디든 털면 그냥 먼지가 나오는 게 아니라 히말라야 산맥이 초융기하는 수준인지라그리고 내부자들은 같이 연루되면 그 죄의식이 사라집니다요.사건사고로 등장하는 용의자나 범죄자들이 다른 범죄자들을 안 욕하고 살았을까요.내로남불이지.다른 것 떠나서 저 제도에 대해서 만약 언론이 뭔가 떡밥을 물고 들어가면 과연 어찌될 것인가.공보의나 전문연을 완전히는 없애긴 힘들지 않을까 싶지만 국방부 정책과 별개로 만약 저 실태라는 게 언론에서 캐고 들어가면 그 때 결과는 사실 알 수 없습니다.95가 잘 한다고 해도 나머지 5가 문제가 있다면?이와 별개로 흥미로운 건 이걸 수험생들이 관심갖는다는 건데(참 쓸데없는 짓)현재 대학생 2$\sim$3학년까지는 몰라도 수험생들이 저 제도에 관심갖는 건 그냥 안 하는 게 좋을 것 같은데.그 때는 정말 없어져있을 가능성이 높기 때문입니다. 그렇다고 하더라도 국방부 발표는 좀 문제가 있는 게적어도 현 대학생 1학년까지는 그래도 구제될 수 있도록 유예기간을 두었어야하는데 지나치게 짧게 잡았네요.아마 유예기간은 더 늘어나지 않을까 싶긴 합니다만 저렇게 되면 전문연 바라보고 공부한 분들은 신뢰가 깨지면서 피해를 보는 거죠.거기다가 공보의 폐지는 사실 비현실적이기도 하고.




\section{남혐 여혐을 막을 필요가 없음.}
\href{https://www.kockoc.com/Apoc/781211}{2016.05.18}

\vspace{5mm}

어느 쪽이든 다 그렇게 주장할만한 타당한 이유가 있어서리.
어차피 말 뿐인 주장은 의미가 없죠. \textbf{실천}이 중요하기 때문에 말입니다.
\vspace{5mm}

상대 진영이 뭘 주장하느냐 그건 별 의미가 없는데 일일히 거기 대꾸하고 하는 것도 바보같은 짓이죠.
\vspace{5mm}

예컨대 남자들이 ㅁㄱ을 위시한 여성 진영이 뭘 하든 그걸 두려워할 필요가 없어요.
정말 두려워하는 건 여자들이 정말 노오력해서 실력으로 남자 자리 차지하고 가모장 행세하고
자기들 말대로 갓양남(...)을 국내에 데려온다면 모르는데 아직까지는 그냥 메시지 뿐입니다.
\vspace{5mm}

그리고 남자들이 아무리 여혐한다고 하더라도 실제로는 현실에서는 호구짓하고 있는데 사실 누굴 까리.
더치페이하고 싶다 하면 눈높이 낮춰서 그런 여자 만나면 됩니다.
그러나 현실은 예쁜 여자 만나고 싶다고 다들 그러고 있잖음, 그래서 그만한 대가를 결국 지불한다고 보면 손해보는 것은 아님.
\vspace{5mm}

가만보면 세상 은근히 공평함.
\vspace{5mm}

가장 극혐인 건 남혐 여혐 부질없다 하면서 기득권을 욕하자(그 기득권이 누군데 그래서) 남자 여자 억지로 화해하자 그건데
갈등이 생기면 싸울 건 싸우고 투쟁할 건 투쟁하는 게 낫지, 괜히 억지로 화해시킨다고 해서 그거 억누르다간 나중에 더 큰일납니다.
여자들도 여혐을 계속 당해야 뭔가 바꾸고 남자들도 남혐을 접해야만 반성하고 바꾸지
무슨 이 사회가 혐오사회냐 그런 것이야말로 철없는 짓입니다.
\vspace{5mm}

저 개인이야 남자지만 사실 여자가 남혐 어쩌구 하더라도 메시지가 일리있으면 고개 끄덕이고 어 그래 ... 그러고 할 것임.
가장 중요한 건 그 여자가 뭐라고 하는 게 아니라 \textbf{실제로 어떤 행위를 하고 어떤 실적을 얼마나 거두며 자기가 돈을 얼마나 쓰느냐} 그거라서리
부모님께 말로만 사랑해요 하는 것보다 저축해서 선물세트라도 하나 바치는 게 실질적 효도인 것과 똑같은 것입니다.
그런데 누리꾼들이 말로만 그렇지 실천하는 것 보기나 했음?
\vspace{5mm}

상대집단을 격하해서 혐오한다고 해보았자 자기에게 돌아올 콩고물 없어요.
뭘 하든 능력자만 대우받게 되어있어서리.
남혐한다는 여자들이 그래서 실제 잘 생기고 유능한 남자 오면 거부하고 여혐한다는 남자들이 쭉빵미녀도 깔 것 같음?
\vspace{5mm}

실제로는 남혐 여성들은 '여성 아이돌' 까기 정신없고, 여혐 남성들은 엄친아 남자는 못 건드리면서 왜 여자들이 후궁노릇하느냐 울고 있는 게 현실임.
그런데 그 까는 애들이 과연 왜 까겠음? 뭐 타당한 이유가 없지 않겠지만 동기는 걍 '컴플렉스'임.
이쯤해서 언질을 던지면, 뭔가 비판하는 분들은 정말 자기의 명분이 천명인지 아니면 컴플렉스인지 그건 구분해보시길요.
이번에 설현 같은 애들이 정말 안중근 의사 몰라서 까였겠습니까. 자기들 컴플렉스 자극하는 예쁜 애라고 까인 거지.
\vspace{5mm}

다시 말하지만 입으로만 뭐라 하는 건 사실 무서워할 것도 없음.
\vspace{5mm}






\section{마이크로소프트 입사시험 마지막 문제}
\href{https://www.kockoc.com/Apoc/782086}{2016.05.19}

\vspace{5mm}

\href{http://news.chosun.com/site/data/html_dir/2016/05/18/2016051801367.html}{링크}
\vspace{5mm}

시사하는 바가 매우 큽니다.
\vspace{5mm}





\section{그래서 남혐 여혐의 대안이?}
\href{https://www.kockoc.com/Apoc/783930}{2016.05.20}

\vspace{5mm}

남혐 여혐은 둘 다 하든 말든 상관은 없는데
그래서 대안이 뭔지 궁금하지 말입니다(...)
\vspace{5mm}

예컨대 남자 무서워 못 살겠다 그건 이해는 갑니다만.
이 논리는 한가지 딜레마에 부딪칩니다.
그런 어떤 식으로의 요구도 '남자'에게 하지 말아야한다는 결론이 도출되죠.
다시 말해서 남자 경찰관이라거나 남자 경호원도 믿을 수가 없다고 봐야 일관된 이야기이기 때문입니다.
오히려 이 경우라면 여성들이 호신술을 배우거나 하다 못해 군사훈련(...)이라도 받아서 자기 몸을 스스로 지키는 게 대안일 터인데.
비꼬는 게 아니리 진지하게 그렇습니다. 이거 남혐 백날 해보았자 소용없어요.
언제 어디서 나타날지 모르는 미친 놈들은 본인이 막는 수 밖에 없거든요.
\vspace{5mm}

마찬가지로 여혐도 딱히... 착한 여자 만나고 싶다는 남자들이 그래서 착하지만 못 생긴 여자 찾는 경우는 없더구만요.
말이야 한국여자들은 어쩌구하면서 결국 이런저런 남자들 후리는 여우같은 경국지색 여인들에 홀려다니더구만.
그리고 우리나라 여자들이 돈만 밝힌다... 라고 할 게 아닌 게 사실 그렇게 따지면 여자들의 미모도 '금전'으로 환원할 수 있는지라.
\vspace{5mm}

한국사회는 참 재밌는 게 문제제기는 잘 하는데 그래서 \textbf{'어떻게 해결할래}'라고 하는 이야기는 하나도 나오지 않는다는 것.
그래서 이번 강남 묻지마 범죄는 그럼 어떻게 해결할 거냐 그 이야기 나와야하는데 이게 여성혐오증 없앤다고 사라질 것 같지는 않은데.
그리고 왜 다들 이런 질문은 안 하는지 모르는데 그 범인이 '여자들에게 무시당했다'라고 말했다던데
그게 진실이든 아니든 간에 실제로 여자들이 남자들을 무시하는 발언을 하는 경우는 왜 문제 삼지 않는지 그건 좀 지금 이해가 안 간다능.
즉 사건의 본질을 보기보다도, 먼저 "이 사회는 여혐이얏"이라고 결론을 정해놓고 사건을 꿰어맞추는 분위기?
\vspace{5mm}

\textbf{천하제일 추모대회}(...)라고 해도 딱히 이상할 게 없다는 지적입니다.
이것도 유구한 동아시아적 전통이죠. 고인의 장례의식에 집중하고 그 재해석에 골몰하지, 정말 산 사람을 위한 대안을 세우지 않는 것.
남혐 여혐할 시간이 있으면 도대체 이런 사건을 어떻게 막을 수 있느냐 그런 데 집중해야하는데 정작 그런 이야기는 없어요.
다시 말해서 포스트잇 붙이는 사람들도 사실 그 사건 자체는 별로 관심이 없어보입니다(...)
\vspace{5mm}

자녀들이 학대당해서 부모에게 살해, 유기당한 게 1년도 안 지났습니다. 그런데 이걸 '자녀혐오'라고 하던가요?
이것도 사실 자녀혐오라고 해도 크게 틀리진 않을 것 같은데 차이가 있다면 그것이죠. 이런 걸 노리는 세력이 없다는 것.
\vspace{5mm}

남자들이 문제야라고 하는 메시지는 이해는 가는데 "그래서 어떡할래"라고 할 때부터 답답해집니다.
그럼 추모를 할 게 아니라 정부나 지자체로 하여금 치안확보를 주장한다거나 아니면 호신술 교육기회를 달라 그래야하지
"한국남자가 문제야" 백날 그래보았자 바뀌는 것 하나도 없죠.
아니, 이 모든 걸 남자 탓으로만 돌리는 것부터가 사실 가장 위험한 '남성종속적인 사고'로 비칠 수도 있습니다.
\vspace{5mm}

어떤 문제는 결국 자기가 해결해야하고 그럴 의지를 보이는데 사실 그런 집단적 의지는 보이지는 않는다고 생각하고 있습니다.
그냥 이걸 포스트잇이나 화환으로 추모대회 하는 것이야말로 "예송논쟁"에다 "당쟁"으로 허송세월하던 조선시대 관행 그대로가 아닌가 하는.
\vspace{5mm}






\section{[영화] 님포매니악 1, 2 (스포 주의)}
\href{https://www.kockoc.com/Apoc/783977}{2016.05.20}

\vspace{5mm}

님포매니악 = \textbf{색정광}
라스 폰 트리에의 이 경악스럽지만 매우 철학적인 영화에서 등장하는 색정광은 여주인공 조만이 아니죠.
이상은 스포성인지라.
\vspace{5mm}

색정광 증세를 못 이기고 온갖 성관계를 하다가, 나중에는 잊혀진 오르가즘을 되찾으려고 중독적 행위를 하는 조에게
상담해주는 모쏠(...) 샐리그먼이야말로 색정광이었다는 사실.
샐리그먼은 조에게 상담해주면서 사실은 그 이야기에 자신의 지적인 자신을 덧붙여 추상적인 오르가즘을 느끼려 합니다.
직접 경험하지 않고 자신의 현학주의적인 추론과 상상을 통해서 대리만족하려고 하는 것이지요.
\vspace{5mm}

하지만 조의 색정광 인생은 쾌락에서 고통으로, 과잉에서 상실로 바뀌어 온 고통스러운 과정입니다.
그녀의 섹스는 결국 중독이었다라는 걸 샐리그먼과 상담하면서 깨닫게 되죠.
그래서 엔딩 직전에 금욕주의를 선언하면서 섹스중독자들을 치유하겠다고 마음먹습니다만...
\vspace{5mm}

샐리그먼은 여기서 실망해버리는 것이고 결국 상담을 마치고 그는 조를 덮치다 그만(...)
\vspace{5mm}

영화 제목이 시사하는 바는 달리 "마음의 평정"이 중요하다는 교훈을 일깨워주면서 위선적인 지식인을 까는 메시지를 담고 있다고 하겠습니다.
샐리그먼이 상징하는 건 르네상스부터 현대에 이르는 서유럽의 지성 그 자체겠죠.
성경을 근거로 마녀사냥을 실시해 여자를 잡아와 온갖 고문을 자행하고 마침내 시신을 해부하여 백과사전적으로 완성한 지식의 총집합.
이 영화의 메시지는 그런 서구적 사고에 대한 반박이자 해방 선언이라고 볼 수 있을 것입니다.
조의 권총 한방보다 충격적인 건 "어차피 넌 수천명의 남자와 잤는데 나랑 자도 상관없지 않느냐"라는 샐리그먼의 위선적인 발언.
거기에는 사실 공감대라는게 없죠. 그가 많은 책을 읽었든 지적이든 간에 그건 다 소용없었던 것입니다.
\vspace{5mm}

어떻게 보면 정신적 위안 $-$ 즉 평정심이라는 건 하늘의 햇님과 같습니다.
흐린 날 빼고는 늘 볼 수 있고 손만 뻗으면 잡을 수 있지만 사실 영원히 잡을 수 없습니다.
영화를 보다 보니 기억 속에 묻고 있던 잊어버린 사람들 $-$ 중독자라고 할 수 있는 사람들이 떠올랐는데
그 사람들도 어떻게 보면 평정심이라는 것을 잡지 못 해서 그토록 방황하지 않았나 하는 생각이 듭니다.
지나치게 예민한 사람도 있고 반면 느끼지 못 해서 괴로운 사람도 있지만 무엇보다 불안한 자기 마음을 추스리긴 정말로 힘듭니다.
불안하니까 중독에 빠지나... 중독은 결국 신경을 마모시키고 더 강한 자극이 아니면 느낄 수 없게 모든 것을 파괴해버립니다.
\vspace{5mm}

그런데 개인적으로는 그 위선적인 샐리그만에 가깝게 살아가고 있는지라 사실 영화를 보면서 내내 찔렸다는 게 포인트입니다.
샐리그먼은 조의 경험에 실제로는 황홀해하면서 자기만의 해석을 덧붙이고 코멘트하는데 히익 저건 내가 하는 짓이잖아.
\vspace{5mm}

그러므로 샐리그먼과 달리 2차원에 정사영이나 해야겠다는 잘 나가다가 삼천포로 빠지는 결론으로 마무리$\sim$
\vspace{5mm}

+ 샐리그먼에 가까운 인물은 플라톤이겠네요. 거기에다가 데카르트가 살짝 혼합된?
\vspace{5mm}






\section{[논란주의] 여혐 프로파간다의 이유}
\href{https://www.kockoc.com/Apoc/784229}{2016.05.20}

\vspace{5mm}

말머리상 이건 보는 사람에 따라 불쾌할 수도 있습니다만
그냥 저 나름대로 시장분석을 한 것이니 그렇게 받아주시길. 사실 저는 걍 그런가보다라는 방관자에 가까운지라
\vspace{5mm}

결혼시장의 변화가 중요한 것 같은데(...)
현재 남자와 여자 결혼할 때
\vspace{5mm}

\textbf{남자는 최소 1억 5천만원의 전셋값을 마련하고 여자는 3000만원의 혼수를 준비한다}
\vspace{5mm}

이게 핵심이라능. 그리고 이건 진짜 다들 침묵하고 있죠.
\vspace{5mm}

지금 남자들도 의식이 많이 바뀌었습니다. 이전처럼 여자들에게 퍼주기 그만하자,
그리고 자기들도 여자들처럼 정략적으로 결혼하고 사귀겠다라고 바뀌고 있는 것이죠.
\vspace{5mm}

사실 조금만 생각해보면 여혐 지적이 말이 안 된다고 느끼는 게, 그럼 과거에는 여혐이 정말 없었겠습니까.
IMF 이전에는 여혐이 없어보일 수도 있습니다. 왜냐면 그 때에는 여자들이 그냥 결혼이나 하고 살림이나 해야한다 취급받았으니까.
사실 그 때가 더 심각한 겁니다. 한반도 역사상 여자들이 \textbf{가장 살기 좋은 때가 현재라는 사실을 아무도 말을 안 합니다.}
그런데 재밌는 건 지금 이 때 여혐이 문제다라고 이야기한다는 것이죠.
물론 이는 이제야 여자들도 정치적 발언을 할 수 있기 때문이다라고 풀이할 수도 있을 겁니다.
하지만 여혐이라는 건 뭔가 핀트가 안 맞다는 것이죠.
\vspace{5mm}

아랫 글에서 적었지만 이제는 여자들도 'give'해야하는 시기기 왔거든요.
그녀들도 경쟁해야 하고 살아남아야 합니다. 2000년대처럼 권리신장하는 시기는 끝났기도 했지만
저도 그렇지만 남자들이 비혼으로 가는 추세는 점점 커질 거라고 보이고 있습니다.
왜냐면 현 시점에서 우리나라의 \textbf{결혼}문화는 확실히 거품이 많이 껴있거든요.
집값이 하락할 일은 당분간은 없어보이지만(집값이 떨어질 거라고 10년째 주장하는 평론가들 안습)
제 생각에는 5년 내에 결혼(과 취업)\textbf{은 유명무실화될 가능성}이 높다고 보고 있습니다.
\vspace{5mm}

그 근거는 사실 꽤 많은데
유럽 쪽만 하더라도 결혼을 하기보단 '동거'를 하다가 헤어진 뒤 동남아 쪽으로 가서 딸, 손녀뻘 여자를 품는 남자들이 늘어나질 않나(...)
우리나라도 이미 프리섹스가 상용화되었고 피임에다 낙태까지 자유자재로 해서 결혼의 의미가 너무 많이 퇴색되어버렸으며
커플도 연령차가 커지고 있어서 굳이 결혼을 빨리 해야 할 필요성이 사라지고 있습니다.
\vspace{5mm}

그런데 이렇게 되어버리면 재밌는 일이 벌어지죠. 결국 저런 거품성 거래가 지속될 거라고 신뢰한 사람들이 상실감을 느낀다는 것입니다.
그래서 생긴 게 메갈이라고도 얘기했지만, 사실 지금도 여혐을 문제삼는 다수의 여성들은 이런 시장의 변화에 위기감을 느끼고 있다는 것이죠.
그나마 10대나 20대 초까지는 공부할 시간도 많고 혼인적령기와 거리가 머니까 상관없을 건데 거기 도달했거나 넘어선 여자들은 힘들어집니다.
공부에다 자기계발을 열심히 해서 \textbf{혼자 먹고 살 수 있는 여자들이라면 사실 결혼을 늦추거나 안 하려고 하죠.}
혼자서 자유롭게 취미 생활하고 살 수 있으니까.
그러나 그렇지 않은 여자들은 남자들이 퍼다주는 결혼하는 걸 노리고 있었는데 이게 점점 힘들어지는 것이지요.
\vspace{5mm}

남녀평등이 반드시 좋은 건 아니라는 것이지요.
기존에는 여성들이 약자였다고 하니까 남자들이 생각 외로 양보를 많이 했던 편입니다(이걸 인정 안 하려는 분들도 많지만)
한국에서 남자들만 군대 끌려가는 것도 그렇고, 위에서 말한 결혼비용도 그렇습니다.
그리고 사실상 결혼을 하면 통장은 여자가 관리했죠.
그래서 결혼하면 시댁 돈은 내 돈.... 이라는 게 관행이었는데
\vspace{5mm}

남녀평등으로 가면 이게 깨진다는 것이지요.
\vspace{5mm}

자, 눈치 빠른 분들은 이제 느끼셨을 겁니다. 왜 그들이 \textbf{'여혐'}이라고 주장하는지요
\vspace{5mm}

'여혐'이라고 하면 다시 여자들이 \textbf{'약자' 대우를 받을 수 있기 때문}입니다.
다시 말해 이걸 남자 여자의 정치공학적인 문제로 환원해서 이게 다 남자들이 여자를 혐오하기 때문이라다고 하면서
여자들이 약자라는 프로파간다를 강조하면, 남녀평등이라는 미명 하에 사라지기 시작하는 여자들의 과거 권리를 지킬 수 있다고 믿기 때문이죠.
아울러 인터넷에서 유포되거나 혹은 과장되는 여성들의 문제점이라는 채무도 일소할 수 있습니다.
\vspace{5mm}

즉, 이건 더 비약해 말하면 \textbf{"우리는 이제 남녀평등이 싫어요. 그냥 과거로 돌아가면서 대우받을래요"}라고 바꿔 말해도 무리는 아닙니다.
이건 다른 여성 분들이 심각하게 비판할 수 있겠지만 저는 답변을 안 합니다. 시간이 흘러서 그 분이 어떻게 사나 그리고 어떤 결혼을 하나 보면 되니까요.
실제로 20대 초에 이런 걸로 다른 여성분들과 토론한 적이 있었는데 답이 안 났지만, 시간이 흘러 보니
그렇게 보수적인 남성과 마초를 비난하시던 분이 정작 "시댁 재산이 많고 소득도 괜찮은 마초"들을 만나 여자로서 행복히 살더구만요(...)
다시 말해서 남녀평등 주장하던 분들이 결혼은 절대 평등하지 않습니다.
그런데 그거야 그 분들이 전생에 애국이라도 하셔서 한미모하셔서 그런 것이고(그런데 페미니즘은 그 미모팔아먹는 것 까지 않나?)
그렇지 못 한 분들이야 뭐...
\vspace{5mm}

게다가 저 분들도 막차는 잘 탄 것이죠. 지금 같이 제2의 IMF가 온 시점에 저렇게 정신나간 결혼을 할 남자들은 별로 없죠.
이제는 각자 플레이인지라 정말 남자나 여자나 열심히 공부해서 각자 능력 키우고 대등하게 가야합니다.
위에서 말한 대로 20대 초반까지는 별 문제는 없다 보이지요. 그러나 그 이후는 어떻게 될까요?
\vspace{5mm}

콕콕에 들어오는 여학생분들도 많으실 건데 저거 보고 휘말리지 마시고 님들 공부 죽어라 하세요.
우리 사회가 이제 부모님 세대처럼 남녀가 화합해 결혼하다는 건 개뿔이고, 이제는 정말 개인으로 살아가야 할 시점이 온 것입니다.
소위 정상적인(?) 결혼은 위너남과 위너녀 말고는 허용되지 않습니다.
외모도 사실 한 때이고 결국 돈발라서 관리 잘 받아야 젊음 오래가는 것인데 이것도 본인이 졸라 공부해서 고소득자가 아닌 이상은 힘듭니다.
공부 안 하고 외모타령해보았자 별풍선이나 타먹는 아프리카 BJ녀들 빼고 답이 있겠습니까.
\vspace{5mm}

여성들은 전성기가 일찍 오기 때문에 $-$ 즉 미모가 꽃이 피는 20대 초중반까지, 남자의 눈에는 진화심리학적으로 '생식'하기 좋은 시기 $-$
별 노력을 안 해도 꿈같은 삶을 살 수 있다라고 착각하기 쉬운데 그거 20대 후반부터 아작납니다.
그래서 그 때 선배들이야 주가 떨어지기 전에 빨랑 결혼해서 호구(...) 잡자 하는 식으로 결혼하는 사례들도 많았던 것이죠.
그런데 이제 이게 앞으로 먹히겠어요?
제가 보기에 여자들이 남자들보다 밀리는 건 신체적 파워 이전에 중년, 장년층이 되었을 때에 실력과 자본이 없으면 재기가 힘들단 겁니다.
하다 못해 남자들은 망하더라도 노가다판을 띠거나 온갖 궂은 일을 하면서 버틸 수 있는데 여자로선 그걸 하기도 힘들죠.
\vspace{5mm}

지금 선동하는 메갈언니들이 그 세대니까 반면교사 삼으시면 됩니당.
\vspace{5mm}



\section{[논란경고] 여자가 밤에 안전하게 다닐 수 있는 권리}
\href{https://www.kockoc.com/Apoc/784431}{2016.05.20}

\vspace{5mm}

그게 추상적으로는 당연하긴 한데 현실적으로는.
정작 \textbf{한국(은 일본과 더불어)은} 밤에 술마시고도 안전하게 다닐 수 있는 몇 안 되는 나라라는 게 아이러니.
우리가 떠받드는 천조국만 하더라도 뉴욕 밤거리에 여자가 지나가보셈, 거기 남자들이 어떤 반응 보이나
\vspace{5mm}

정작 치안이 개판인 다른 나라에서 이런 사건 터져도 별 문제 삼지 않았을 거란 이야기(...)
아니 이건 약간 좀 비꼬아서 말하면 우리나라에서 왜 여자들보고 집에 일찍 들어가라고 하느냐 문제삼는 분들이
\textbf{치안이 개판 5초전인 다른 나라에 가서도} 살도 그렇게 대자보 붙이고 시위할 수 있을지는 좀 의문이네요
아니, 정말 그 정도로 위험한 나라면 강남역 10번 출구라고 해도 그 자정에 포스트잇 붙이는 게 가능하긴 할까.
\vspace{5mm}

그렇다고 그런 권리를 주장하지 말라는 건 아니지만 '현실 인식' 측면에서는 괴리감이 느껴지는 건 사실입니다.
그런 치안을 구축한 건 일단 '\textbf{남자}'들이 아님?
국방도 그렇고 경찰도 그렇지만 자정 넘은 시각 편의점에서 알바하는 사람들 대다수가 남자들이죠.
(사실 치안 측면에서는 24시간 편의점이 정말 긍정적인 외부효과를 가져왔다고 보고 있습니다.)
결과적으로 우리나라 정도면 다른 나라 여자들이 환호할 정도로 우수한 측면인데 그럼 어떻게 해야하는지 그게 참 궁금
\vspace{5mm}

이번 사건 같은 것 막는 거 "추모 메시지" 해보았자 의미없다니까요.
확실한 해결책을 세워야지
그런 차원에서 과거 군사정권 때 있었던 "통금조치"를 하면 어떨까 하면 돌맞아죽겠죠? (그걸 해제한 게 아이러니하게도 전두환 정권 때)
남녀공용화장실 없애고 화장실을 쪼갠다.... 뭐 그건 그렇다 치더라도
아예 그런 곳은 어떤 건물이건 들어가는 모든 사람은 신분조회한다... 이것도 확실한데 이건 또 반대하겠죠.
\vspace{5mm}

피해자가 잘못했다 그런 이야기는 아님요. 그러니까 도대체 이런 일이 재발 안 하려면 어떤 해결책을 내세워야하느냐는 거지.
그런데 현실은 참 말도 안 되는 생트집 잡는 것입니다. 그러니가 여혐하지 말라는 건데 그건 핀트 잘못 잡은 게 아님?
결국 이 문제는 '정신병자'나 '사이코패스' 같은 애들 범죄 어떻게 막느냐... 그러는 건데 여혐 때문이다라고 하면 어이구 답이 안 나오지요.
혹시 원하는 게 "인터넷에서 여혐 글 차단 먹이고 그런 글 쓰는 놈들은 처벌하라"고 하는 것이라면*(어 이건 가능성 있어보이네)
자기들이 알아서 파시스트 사회 만들겠다 그런 얘기죠.
\vspace{5mm}

솔직히 피해자 추모와는 관계없이 그냥 화풀이하는 것 같은 데 정말 우리나라 치안 좋은 걸 모르고 그러시는가 그런 생각이 듭니다.
그럼 다른 나라들은 여혐이 성층권을 뚫고가서 강력범죄가 빈발하고 여자들이 밤에는 거리를 걷지도 못 합니까.
\vspace{5mm}






\section{[논란경고] 한국의 치안지수}
\href{https://www.kockoc.com/Apoc/784560}{2016.05.20}

\vspace{5mm}

\href{http://www.numbeo.com/crime/rankings_by_country.jsp}{링크}
\vspace{5mm}

여기 가서 확인하셈, 혹시 1위에 왜 한국이 없냐 하는 반응 보이면 답 없습니다(...)
서양 사람들이 우리나라에 눌러앉는 경우가 치안이 좋아서라는 게 큰 이유가 되죠.
그리고 걸핏하며 이민 간다는 사람들, 말로만 그렇죠. 타국이 치안이 좋은지 따져보기는 하셨나.
\vspace{5mm}

이 현실 보면 저걸 가지고 '여혐 범죄'라고만 소리치는 사람들은 과연 생각이 있는 건지는 좀 의심이 갑니다.
거꾸로 말해서 한국이 치안이 안 좋았으면 이런 논쟁도 안 터집니다.
치안이 안 좋은 나라면 강남역 술마시는 사람들도 없고 뜸하니 그 정도 범행은 일어나지 않죠.
밤 8시 이후면 아예 못 다닐테니까.
\vspace{5mm}

냉소적으로 말해서 여자라서 두렵다 무섭다 하는 분들 심정은 이해갑니다만
다른 나라에 가서도 그런 말씀을 하실 수 있는지도 다소 회의적입니다.
물론 우리가 치안이 안 좋은 나라 수준에 맞출 이유는 없겠지만,그렇다면 그 반응은 과잉이라는 것입니다.
\vspace{5mm}

그래서 한국이 싫으면 이민 가겠다... 그럴 리야 없죠.
그럼 통금조치 하자, 당연히 반대하겠죠.
호신술 교육하고 군사훈련 받아서 몸 지키도록 하자, 이건 아예 시위하겠죠.
\vspace{5mm}

가장 큰 문제는 그게 실제로 여혐이더라고 하더라도 본인 몸은 스스로 지켜야하고
사회적으로 요구할 건 분명히 구체화시켜 요구해야하는 데 그런 게 없다는 것입니다.
\vspace{5mm}

결국 저것도 남자들에게 뭔가 요구하는 식으로 '의존적 행태'로 간다는 게 가장 심한 문제죠.
그런데 남자들이라고 해서 남자들이 여자들에게 성범죄 저지르고 수작 부리고 폭력 휘두르는 걸 모르는 건 아니거든요.
개인 차원에서는 그러지 말자고 윤리적인 행위만 할 수 있을 뿐, 다른 나쁜 남자가 어떻게 하는지 그걸 막을 수도 없죠.
게다가 성폭행 범행현장에서 범인과 싸우다 죽은 남자 장레식에 정작 도망간 여성피해자는 나타나지도 않았다죠?
여자란 이유만으로 희생할 이유 없듯 남자도 마찬가지입니다.
\vspace{5mm}

커뮤니티 돌아다니다보니까 재미있는 게 "여자친구와 $\sim$ 하게 대화했다", "강남역 여혐 범죄사고로 헤어졌습니다" 이건데.
비위 맞추는 수준이라면 몰라도 거기서 여자 말을 들어야할지는 의문입니다.
말하지만 문제 해결이라는 건 남자라고 특별히 깎고 여자라고 특별히 봐주고... 그런 것은 없습니다.
오빠가 지켜줄께... 라는 치킨순살 멘트 원하는 여자라면 걍 ㅄ이니 그냥 헤어지라고 오히려 권하고 싶습니다..
'남자들이 문제가 많아'라거나 '우리 여자들은 불안해 미치겠다'라면 뭐 그런 반응이면 모르지만.
남녀평등 주장할 거라면 일단 도와달라면 모를까 자기 몸은 자기가 지켜야지
매드맥스 보고 퓨리오사에 환호할 때는 언제고.
\vspace{5mm}

하여간 이 광경은 다음 짤방으로 요약됨
\vspace{5mm}

매우 적절하다




\section{주식할 때 망하는 패턴 중 하나가}
\href{https://www.kockoc.com/Apoc/784654}{2016.05.20}

\vspace{5mm}

물타기입니다.
사놓고 떨어지니까 수익율 높인다고 그 주식에 또 돈을 퍼붓는 것이죠.
그런데 예측과 달리 또 떨어지고(...). 그래서 조금이라도 원래 가격보다 오르면 나오는 수익으로 손해 메꾼다고 빚까지 지다가(...)
\vspace{5mm}

이게 공부도 마찬가지임요.
\vspace{5mm}

자기 능력을 과대평가하는 게 망하는 지름길.
자기 능력은 비관적으로 평가하는 게 올바른 길입니다.
\vspace{5mm}

6월달부터 빡세게 하면 된다.... 체력과 지구력은 돈과 같습니다, 그럼 그 돈을 어디서 조달하건지?
체력은 한정되어 있습니다. 그리고 점점 더 떨어집니다.
그리고 모의고사 치고 나면 자신감이 떨어지는 경우가 훨씬 더 많으니 이것도 잘 추스려야합니다.
자기가 잘 나간다고 생각하면 마이너스지만, 원래 못 했는데 과거보다 나아지고 있다고 생각하면 손해볼 게 없습니다.
\vspace{5mm}

이제 시간은 2배 빨라지는데 해야할 일은 생각한 것보다 최소 2배 이상 늘어납니다.
인내하면서 하루하루 무리하지 않게 버텨나간다는 자세로 가야하는 것이죠.
실력과 점수는 머리카락과 같습니다. 자기도 모르게 늘어나는 것이지 늘어나라 한다고 확 늘어나지 않습니다.
\vspace{5mm}

언제 벼락치기 하라고 하지 않았냐 하는데 그거야 더워지기 전까지입니당(...) 이제 더위가 시작되었으니까 관리모드 가야함요.
그리고 냉정히 봐서 자기가 목표하는 대학이 힘들다 하면 내년까지도 염두에 두시길요.
어차피 공부하는 자세가 안 되어있으면 대학 가도 소용없습니다. 서울대도 은근히 학업 못 따라가서 휴학, 유급, 자퇴 코스 가는 사람 없지 않거든요.
앞으로 10년 공부해야하는 것 밑바탕 잘 잡는다라고 마음먹고 제대로 하시기들 바랍니다.
\vspace{5mm}






\section{인공지능의 김대식 교수 썰}
\href{https://www.kockoc.com/Apoc/784899}{2016.05.20}

\vspace{5mm}

\href{http://news.naver.com/main/read.nhn?mode=LSD&mid=sec&sid1=105&oid=079&aid=0002831975}{링크}
\vspace{5mm}

◆ 김대식> 학자도 아니고 유발 하라리 분은 정말 훌륭한 역사학자이십니다. 제가 유발 하라리라면 짜증이 났을 것 같아요. 아니, 나는 역사학자이고 호모사피엔스 역사에 대해서 정말 재미있는 책을 썼는데 인공지능이라는 얘기는 600장 책 맨 끝에 한 장에 들어 있는데 기자들이 물어보는 질문의 100$\%$ 또는 95$\%$가 인공지능 시대에 어떻게 살아야 되냐고 물어보는 거예요. 재미있는 건 파키스탄에서 오는 손님들한테 그런 질문을 우리는 당연히 안 합니다. 다시 말해서 제가 또 한 번 느끼는 건 ‘이야, \textbf{대한민국 국민 머리 또 언론인들 머리 안에 상당히 깊게 박혀 있는 지적인 사대주의가 여기서도 나오는구나’}. 우리가 100년 전부터 결국은 세상을 모르는 상태에서 너무 많은 접근을 하다 보니까 \textbf{우리가 항상, 우리가 궁금한 것이 있으면 외국 사람들한테 물어보면 됐죠.} 어떻게 산업발전 할까요? 민주주의는 어떻게 만들까요? 환경보호는 어떻게 할까요?
\vspace{5mm}

◆ 김대식> 결국 우리가 100년 전부터 했었던 것은 이 지구에서 일어나는 문제를 우리가 스스로 습득하고 이해하고 질문을 던진 것이 아니고 타인이 이미 경험한 문제들을 우리는 겪다 보니까 \textbf{먼저 경험한 사람들한테 항상 물어보고 압축성장으로 빨리 배운 거죠}. \textbf{그런데 재미있는 것은 인공지능이라는 것은 이 세상 그 누구도 답을 모릅니다. 아무도 경험을 못 해본 것이기 때문에.}
\vspace{5mm}

◆ 김대식> 연구분야에서 좀 앞섰겠지만 인공지능 사회는 아무도 경험을 못 해본 거죠. 우리가 기술적인 건 물어볼 수 있겠지만 인공지능이 예를 들어서 직업의 50$\%$를 대체하고 이런 사회는 존재하지 않는다는 거죠. 그런데 우리는 신기하게도 알파고 덕분에 어떻게 보면 유럽의 대부분 나라들보다 인공지능의 이 문제성은 \textbf{우리가 먼저 본 거거든요.}
\vspace{5mm}

◆ 김대식> 저는 이걸 사실은 약간 \textbf{역사적 행운}이라고 당시에 생각을 했었어요. 이야, 우리가 250년 전에 산업혁명이 처음 나왔을 때는 아무 것도 모르다가 당했지만 이번만큼은 우연의 결과로 우리가 먼저 눈을 뜨고 봤기 때문에 먼저 무언가를 할 수 있겠구나라고 생각을 했었는데.
\vspace{5mm}

◆ 김대식> 이 세상을 남의 답을 통해서 세상을 보는 것이 아니고 우리도 이제는 세상을 우리 눈으로 좀 봐야 되지 않을까. 적나라하더라도 위험하더라도 무섭더라도. 우리는 여전히 세상의 모습을 부모님들, 우리보다 좀 더 큰 어른들 눈을 통해서 대신 보려고 하는 약간 좀 어린 아이 같은 생각을 하고 사는 것이 아닐까. 거기에서 우선 좀 벗어나야 하지 않을까 싶네요.까지 표현하셨는데 답답하단 말이에요.
\vspace{5mm}

◆ 김대식> 당연히 그런데 재미있는 건 우리가 기술적인 것에 대해서는 서로 물어볼 수 있겠지만 결국 인공지능이 지배하는 사회의 사회구조를 어떻게 만들어야 되고 일자리를 어떻게 만들어야 되고 사회복지는 어떻게 해야 될지는 사실 우리가 직접 해결해야 되는 문제라는 거죠\textbf{. 결국 우리는 문제를 해결하려고 하지 않고 자꾸만 정답을 바란다는 거죠. 그런데 어떻게 보면 이게 우리나라 역사상 상당히 새로운 경험일 수도 있는데 그 누구도 답을 알지 못하는 거예요, 지금으로서는.}
\vspace{5mm}

◆ 김대식> 물론 미래 예측은 불가능합니다. 10년, 20년 후 세상이 어떻게 될지 아무도 모르겠지만 적어도 10년, 20년 후에 지금 10대들이 직업을 선택해야 될 나이에는 다른 건 몰라도 \textbf{기계가 국영수를 우리보다 잘할 거라는 건 우리가 예측할 수 있습니다. 그 정도는. 그렇다면 지금 10대 이하들이 학교에서 열심히 국영수를 배운다는 것은 불도저가 등장하는 시대에 열심히 삽질을 잘하는 방법을 배우고 있다는 거예요. 경쟁력이 없습니다}, 사실은. 지금 10대 이하들은 나중에 커서 인류 역사상 처음으로 기계하고 경쟁해서 직업을 얻어야 하는 친구들이잖아요. 그런데 우리가 지금 이 친구들한테 기계보다 더 잘할 수 있는 것을 하나도 안 가르쳐주고 있다는 거죠. 기계가 우리보다 당연히 더 잘할 것들을 열심히 지금 가르쳐주고 있다는 게 가장 큰 문제 중의 하나겠죠.
\vspace{5mm}

◆ 김대식> 우리가 지금 돈을 내고 취미생활로 하고 있는 것이 불과 100년, 200년 전에는 우리가 꼭 했었어야 하는 행위들입니다. 100년 전, 200년 전에는 웬만한 성인 남자는 하루 종일 벽돌을 짊어지고, 무게를 짊어지고 건물을 올라갔다 내려갔다 했었는데 사실 그 당시 사람들한테 퇴근하고 네 돈 내고 어디 가서 무거운 짐을 2시간씩 드세요. 누가 했겠습니까? \textbf{지금은, 예전에는 꼭 했었어야 할 노동적인 행위를 우리가 돈을 내고 취미로 한다는 거예요.} 똑같은 행동을. 그렇다면 결국 우리가 지금 인간이 하고 있는 대부분 육체적인 노동과 지적인 노동을 기계가 하는 순간 인간이 손을 놓을 필요 없이 지금은 먹고 살기 위해서 하지만 어떻게 보면 30년, 40년 후에는 그게 우리의 취미생활이 될 수 있다는 거예요.
\vspace{5mm}

◆ 김대식> 기계인 척을 하는 사람보다는. 그래서 \textbf{가장 먼저 그만둬야 할 것은 반복성이 있고 내가 볼 때도 내가 하는 일이 기계적이라면 무슨 일이 있어도 거기에서 나와야 한다는 것이고 두번째는 아까 현실을 말씀하셨는데 이 현실과 인공지능을 보면 재밌는 현상이 하나 일어날 것 같아요. 그건 뭐냐 하면 인류역사상 인간은 항상 이기는 자 쪽으로 붙게 돼 있습니다.} 그렇다면 20년, 30년 후에 기계들이 인간의 일자리를 지금 대체하고 더 잘나가고 더구나 동시에 기계는 인간이 가진 단점들을 안 가지고 있잖아요. 죽지도 않죠. 밥도 안 먹죠. 잊어버리지도 않고 더구나 이세돌 9단은 우리가 복사할 수가 없습니다. 알파고는 구글이 원하기만 하면 100만번 복사할 수 있다라는 거예요. 거기다 zero marginal cost. 아무 돈이, 추가비용이 안 드는 상태로. 그렇다면 인간의 심리상.
\vspace{5mm}

우리가 가진 가치관의 99$\%$를 버려야할지도 모른단 생각이 듭니다(...)
여태껏 접한 인공지능 썰 중 그나마 유용한 이야기네요. 나머지야 그냥 무속 수준의 얘기였으니
\vspace{5mm}

사실 가장 찔리는 건 한국인들은 문제를 해결하려고 하지 않고 무턱대고 정답부터 찾는다.
이건 수험에서도 마찬가지인 듯. 정답에만 집착하면 킬러는 못 풉니다
\vspace{5mm}






\section{일본의 풍토 : 곤카쓰}
\href{https://www.kockoc.com/Apoc/785611}{2016.05.21}

\vspace{5mm}

\href{http://news.naver.com/main/read.nhn?mode=LSD&mid=shm&sid1=102&oid=469&aid=0000122058}{링크}
\vspace{5mm}

알고리즘
\vspace{5mm}
\begin{enumerate}
    \item 남자들의 결혼기피
    \item 걸스푸어현상
    \item 전통적인 남녀역할로 복귀
\end{enumerate}
\vspace{5mm}

일본에서 만난 젊은 남성들은 결혼에 대해 크게 걱정을 하지 않는 분위기였다.오히려 남자들이 결혼을 ‘필수’에서 ‘선택’으로 여기자 타격을 받은 것은 경제적 기반이 취약한 여성이었다. 이 같은 현상은 되레 여성들이 ‘결혼 활동’에 더 적극성을 띄게 만들었다.
특히 30대 자녀를 둔 부모세대가 은퇴하면서 수입이 급감한 2000년대 후반부터 결혼 문제가 사회적 이슈로 떠올랐다. 경제적으로 독립하지 못한 ‘파라사이트 싱글’ 자녀에게 부모가 더 이상 지원을 해 줄 수 없게 된 것이다. 부모의 경제력이 약해지자 젊은 비정규직 여성들은 다시금 결혼이라는 전통적 해법을 찾아 나섰다. 청년 스스로의 욕구가 아니라 부모에 의해 등 떠밀려 결혼 상대를 찾아나서는 형국이 됐다.
\vspace{5mm}

2012년 일본 내각부 조사에 따르면 20대 여성의 44$\%$가 ‘아내가 가정을 지켜야 한다’고 응답했으며 이는 3년 전보다 16$\%$포인트 증가한 것이다. 후쿠시마 미노리 도코하대 교수는 “여대생을 포함해 전업주부를 꿈꾸는 20$\sim$30대 여성이 증가하고 있다”며 “이는 ‘일벌레 남편’과 ‘가정주부 아내’라는 부모세대의 젠더화된 생존 전략이 다시 부상하고 있음을 보여주는 현상”이라고 말했다.
\vspace{5mm}

한국 시나리오.
\vspace{5mm}
\begin{enumerate}
    \item 재산에서 주택 비중이 높은 586 세대의 몰락 : 집값 하락
    \item 캥거루 가족을 지탱하기 어려워짐. 남자는 비정규직으로 일하는 게 당연해짐, 여자들이 일자리 찾기 힘들어짐.
    \item 경제력 있는 남자에 대한 한국판 취집 현상 늘어남.
\end{enumerate}
\vspace{5mm}

핵심은 경제력입니다.
지금 이미 여성들이 경제력에서 밀리기 시작했고 이들을 지탱해주던 5,60대 부모들의 지원도 힘들어집니다.
이 와중에 일베와 메갈이 생겨나서 그러한 빈곤을 인정하지 못 하고 "혐오"로 세몰이합니다.
물론 아무 것도 생산하지 못 하는 세몰이므로 흑역사로 남을 것입니다.
\vspace{5mm}

일본과 완전히 똑같지는 않을 겁니다. 아직은 성비불균형이 있어서리.
다만 서울의 경우는 여초 지역이니 저런 일이 벌어질 가능성은 높습니다.
\vspace{5mm}

한국도 이 길을 따라하고 있다고 보는 증거가 근래의 메갈 활동이죠.
만약 여자들이 속편하게 앉은 자리에서 아무 준비없이 시집갈 수 있다면 저러진 않았을 겁니다.
원래 메갈의 주본거지가 디씨의 아이돌, 연애인 갤러리인데 그러면 딱히 짐작가는 것이죠.
저런 걸 따라할 필요가 없습니다. 그 사람들과 다른 길로 가야지 똑같은 길을 갈 이유가 없어요.
\vspace{5mm}

그럼 왜 혐오활동을 하느냐. 그 혐오 행위에서 연대감을 느끼면서 공동체의 소속감을 확인하기 때문입니다.
그 이야기는 혐오활동을 하는 사람들은 소속될 공동체가 없고 거의 외면받기 시작한 루저들이라는 이야기입니다.
메갈의 경우도 기성 사회에 대한 반항이 투철한 여성들의 모임이죠. 이들의 결속은 한국남자를 혐오하는 것으로 유지됩니다.
또한 그 결속력을 강화하고 공동체를 넓히기 위해 한국남자들에 대한 증오를 재생산하는 것입니다.
\vspace{5mm}

하지만 이것이 색다른 걸 창출할 수는 없지요. 일단 이들은 루저들이니까요.
일베가 기발한 컨텐츠를 제작해보았자 한계가 있는 것과 마찬가지입니다.
\vspace{5mm}

SNS 문화도 한몫하죠.
페이스북의 핵심은 그 사람이 어떤 기관이나 조직에 소속되어있느냐 하는 '스펙'입니다.
구경하다보면 그런 게 없는 사람이 '유령조직'을 만들어서 자기가 그런 데 소속되어있다라고 뻥카치는 경우도 봅니다만.
가족보다도 SNS 친구와 더 많이 대화하는 경우가 보편화되고 있고, 그래서 이런 인맥에 소속감을 보여주기 위해서
뭔가 '사진'으로 증명해야합니다. 그래서 '보여주기'로 치중하는 추모대회를 여는 것이지요.
그 사람들이 정말 진정성이 있어서 그렇다면 정말 점잖게 '조의금'을 냈을 것입니다.
\vspace{5mm}






\section{강남역이 시사하는 것}
\href{https://www.kockoc.com/Apoc/786600}{2016.05.21}

\vspace{5mm}

저걸 보고 절망했다는 남자도 철이 없긴 마찬가지이다. 원래 \textbf{여자들 저런 거 몰랐다}는 거냐
\vspace{5mm}

여학교에 환상을 품지 말라고 하거나(여학교가 돼지우리보다 더럽다라는 이야기) 아줌마들 집단이 무서운 거 다 헛소리가 아니다.
\vspace{5mm}

바꿔 말하면
\vspace{5mm}

그만큼 여성고객 다루는 건 인간관계 중 최고의 기술을 요하는 것이다.
\vspace{5mm}

그런데 동영상의 저 남자는 매우 순진하게 '설득'할 수 있다고 믿은 모양이다.
\vspace{5mm}

굶주린 식인종 부족에게 가서 성경을 읽어주면 다들 기독교도로 개종할 수 있다고 믿는 풋내기 선교사나 다름 없다.
\vspace{5mm}

현실은
\vspace{5mm}

긍정적으로 생각해본다면
\vspace{5mm}

저런 여자들을 설득할 수 있는 사람은
\vspace{5mm}

어디서든 살아남을 수 있고 모래알로 황금덩어리를 만들어낼 수 있다.
\vspace{5mm}

사실 돈줄은 대부분 여자들이 쥐고 있기 때문에 이런 분야를 알지 못 하면 굶어죽기 딱 좋다(...)
\vspace{5mm}

다른 사람들은 메갈과 워마드에 화가 날지 모르지만 나는 그다지... 그게 메갈과 워마드만 그런 게 아니라 \textbf{원래 여자들이 그런 거 몰랐어}?
\vspace{5mm}

남자들은 성욕만 없으면 초식동물에 가깝지만, 여자들은 성욕이 없어도 기본이 육식동물이란 사실을 나이처먹고 나면 알게 되는 것 몰랐냐.
\vspace{5mm}

무엇보다 저 린치당하는 남자는 경험이 없는지 말하는 게 참 어리버리.
\vspace{5mm}

연애하는 사람들이라면 오빠 그러는 것 속지 말고 집단 속에서 그 여자가 어떤 완장차나 꼭 보아야한다(...)
\vspace{5mm}

물론 적극적인 것이 좋은 경우도 있다. 적어도 그런 여자가 멍청하지 않다면 남편까지 먹여살릴지도 모른다.
\vspace{5mm}

남자들이 여자들 본능을 저 정도로 몰랐느냐... 라는 게 더 의아스러울 정도다.
\vspace{5mm}

이성과 논리 그게 먹히는 줄 알았나. 그냥 저 자리 갔으면 딱 완장차기 좋은 여자 지목해서 연설을 시켰어야 한다.
\vspace{5mm}

여자들이 스스로 눈물 흘리면서 다 털어놓게 만들면 저 남자는 호감도가 올라갔을 것이다.
\vspace{5mm}

다시 말해 저 남자는 자기가 말할 게 아니라, 여자들보고 떠들라고 했어야한다.
\vspace{5mm}

여자들이 왜 모였겠냐? 평상시에는 여자라는 강요 하에 얌전하게 억눌려살아야했는데 저 장소가서 분출하고 싶어서 그런 거지.
\vspace{5mm}

그런데 이걸 모르고 "선교"가능하다고 믿는 남자들도 한심하기 짝이 없다.
\vspace{5mm}

진짜 영악하고 똑똑한 사람은 퀸베충(...)인 듯. 그거 하나로 인지도 올리면서 팬들을 모았으니까.
\vspace{5mm}

남혐이 문제가 아니라 여자들이 깔깔대고 떠들어댈 수 있는 기회나 장소가 부족하다는 게 문제가 아니었나.
\vspace{5mm}

그리고 저 정도 동영상 가지고 린치라고 하기에는 오히려 그녀들이 불쌍하다. 비아냥거리는 게 아니라 실제로 그렇다는 것.
\vspace{5mm}






\section{그리고 떠들게 냅두지 왜 끼어드는지.}
\href{https://www.kockoc.com/Apoc/786879}{2016.05.22}

\vspace{5mm}

주장이 타당하면 다 납득할 것이고, 주장이 이상하면 그 정도 밖에 안 되었느냐... 그 정도일 터인데 굳이 막을 필요가 있을까.
"남자들은 잠재적 가해자예요"
라고 소리치면 "아, 그래요"라고 하고 걍 가볍게 대꾸해주면 지나가면 그만이다.
사실 말도 안 되는 주장이기 때문에 그냥 냅두면 된다.
\vspace{5mm}

그리고 무엇보다 그거 주장하는 여자들이 과연 현실을 바꿀 힘이 있는 능력자일까.
담론과 시위. 그거 어느 쪽이든  힘이 없으면 정말 쓸모없는 것이다.
여자들이 한국남자 못 되었어요, 다 거세해버려야한다고 소리친다고 한들 이게 무슨 소용이 있나.
막말로 거기 이부진이나 이명희가 참석한 것도 아니다.
\vspace{5mm}

거기 참석한 여자들이 '여성 인권'을 위하여 가난한 남자도 바보온달처럼 키워주고 결혼하고 뒷바라지할 수 있는 사람들이라면 모를까
어차피 시간 지나면 왜 그런 바보같은 시위를 했을까 하면서 잊혀질 권리 주장하면서 돈많은 남자나 찾으려 할 터인데 뭘 무서워하는지.
역으로 거기 대꾸하는 일베 애들이 힘이 있나.
이제 천하제일 丙申 대회 개최하고 그냥 노는 것이지.
\vspace{5mm}

보루토 보고 히잉 나루토 죽은 거야라고 소리쳐보았자 그거야 나루토팬들이나 광분할 일이지 현실에 영향 미치는 것 없다.
막말로 거기 참석한 사람들이 정말 '물질적'인 것을 내놓거나 자기 '시간'을 바쳐서 실제로 뭘 해내느냐 그런다면 몰라도
그냥 모여서 쓰잘데기없는 이야기나 하면서 한풀이한다면 걍 냅두면 되는 것이다.
이건 초창기 촛불시위 당시 미군도 알았고, 광우병 촛불시위 당시 이명박 정부도 알고 있었다.
\vspace{5mm}

사람들이 정말 이념이나 정의를 위해 목숨을 바칠 수 있다고 보나?
그런 사람은 정말 없다. 배부르고 할 짓 없을 때나 눈물 흘리며 쇼하지, 배고프면 밥 어디있나 어슬렁거리는 건 다 똑같은 것이다.
사람은 먹고 싸고 호흡하지 않으면 죽는다. 현학적인 담론보다 국밥 한그릇이 더 임팩트가 있단 것이다.
\vspace{5mm}

다만 한가지 짜증나는 건 이 때문에 '옥시 사태'가 묻히고 있다 그 정도다. 그러니까 다들 병신들인 거다.
우리 몸에 어떤 식으로 해악을 끼치는 상품은 신경쓰지 않는다. 그저 "같은 X자로서 불안해요" 이러고 있는 건 정말 한심한 짓이다.
그 정도 열정으로 옥시제품 불매운동 대대적으로 벌이고 대기업에 항거하면서
자기들이 환장하고 남을 명품 백, 스카프 같은 것 불태웠으면 여성 고객이 무서운 줄 아는 옥시 쫄아서 걍 엎드렸지.
\vspace{5mm}

아무튼 저는 상호비존중입니다. 남혐 하는 것은 안 말리고, 당연히 여혐 하는 것도 안 말림.
담론이든 시위든 그거야 나름 지겹게 겪어서 걍 무시함. 인간에 대해서 딱히 신뢰도 안 해요
하라는 공부 안 하고 와서 이 뻘글 읽는 학생들만 봐도 한숨 나오는 데 무슨, 당장 가서 공부나 하시면 좋겠고
\vspace{5mm}

+
여성에 대한  입장도 위 주갤선언문의 깔끔한 그대로입니다.
남성들 X 잡고 반성하라하는 여자 떠들라 하면 그만임. 다만 그런 여자는 걍 비존중하고 무시하면 그만이라서리
내 한 몸 건사하기도 힘들어죽겠는데
우리집 강아지도 오야오야 키워주엇는데 손등 물어뜯었을 때 원효대사의 스켈레톤 워터 마신 깨달음 그대로였음.
절대로 처음부터 잘 해주면 안 된다. 통수 맞는다 $-$
\vspace{5mm}

++
믿거나말거나인데 그들이 말하는 것처럼 여성에게 수작 건 적도 없고 당연히 성희롱, 추행은 진짜 먼나라이야기인데
제가 안 하니 역으로 여자 쪽에서 시도해오는 재밌는 경험을 한 적이 있습니다. 이거 어떻게 설명할 거임?
연애하면 좋다느니 보드라우니(...) 그건 개뿔이고 이거 돈먹는 하마인데다가 깨진 사람들 상담하느라 노이로제 걸린 것만 생각해도 크윽
그리고 뭐가 예쁘다고. 다 나이먹으면 어렸을 때 보던 옆집 아줌마임. 안 늙는 건 부르마 누나랑 아스카랑 레이였음.
\vspace{5mm}

+++
그리고 이건 걍 떡밥인데 던지면 지하철이든 버스든 여성들 있으면 불편함. 시선처리 잘못 하면 오해사기 딱 좋아서(...)
제가 표정 잘못 지었다가 순간 파렴치한으로 몰리고 촬영당해서 경찰서 가서 형사랑 농담 따먹기하고 있을 가능성이 정말 없을 것 같음?
이거 나름 심각합니다. 그나마 봄까지야 괜찮았지 이제 노출의 계절이 오면 썬그라스라도 껴야하나 생각 중인데
썬그라스 끼면 영락없는 중국산 깍두기이고 그래서 이어폰 끼고 더러운 원서 하나 끼고 다닐까 생각 중입니다. 오해 안 사려고
\vspace{5mm}






\section{[만화] 보이즈온더런}
\href{https://www.kockoc.com/Apoc/787126}{2016.05.22}

\vspace{5mm}

만화 다 보고 나면 위 짤방이 극혐이 되죠(다 보고나시면 압니다)
\vspace{5mm}

만화야 구글링해서 찾으면 쉽게 보실 수 있을 겁니다.
작가 분이 찌질한 남성들의 심리를 참 잘 파악하는 작품을 쓰시는데(르 상티망, 보이즈온더런, 아이엠어히어로)
그 중 압권이 보이즈온더런입니다.
콕콕하는 남학생들이 아직까지 숫기 없고 철 없으면 이 만화부터 읽어보라고 싶을 정도.
\vspace{5mm}

가치관이 송두리째 바뀌기 좋은 몇 안 되는 작품입니다.
정의는 언어가 아니라 내 주먹(보이즈온더런)과 총(아이엠어히어로)에 있다.
그리고 여타 일본만화와 달리 찌질한 남자 심리나 그런 남자 속이는 여자 심리가 정말 리얼하게 잘 묘사되어 있습니다.
작품 끝까지 카타르시스가 해소되는 건 아닌데 마지막 장을 넘기고 나면 '아' 하는 탄식과 함께 이런 게 철드는구나를 경험하게 된다능.
\vspace{5mm}

진짜 남자라면 키배도 시위에 집착할 게 아닙니다.
싸우거나 일해야지
매일매일 노동으로 피로해지거나 정말 목숨걸거나 다치는 것 각오하고 싸우거나 그러는 거지.
오늘 강남역 영상이라는 걸 보니 노동과 싸움과는 거리가 먼 찌질함 그 자체였습니다. 이게 대한민국의 미래죠
\vspace{5mm}






\section{[논란주의] 동영상}
\href{https://www.kockoc.com/Apoc/787211}{2016.05.22}

\vspace{5mm}

$\#$ 안중근, 윤봉길의 독립투쟁을 이은 백색테러
\vspace{5mm}

$\#$ 한국 민주주의 쾌거를 보여주는 강남역 시위
\vspace{5mm}

$\#$ "재기하라"며 격려하는 아름다운 한국여자들
\vspace{5mm}

이런 민주주의 여러번 하면 후진국화되어 모두가 평등하게 살 수 있어 개꿀.
제 딴에는 다 자기가 하는 얘기가 옳고 그걸로 설복시킨다고 생각하지만 현실은 '카니발'
내 살아 생전 저런 추모는 처음 보았다. 창조경제는 간데없고 창조추모 $-$ 저 분들 죄다 상조 취업하면 대박이겠다.
\vspace{5mm}

그래도 하나는 긍정적이네
저 정도 패기부리는 여성 분들이라면 범죄 피해자가 될 리는 없잖아.
범죄자가 접근하기도 전에 암에 걸려 사망할 걸?
그리고 이런 영상이 돌면 여성 판타지가 사라져 성범죄자들의 의욕이 낮아지는 긍정적 효과.
\vspace{5mm}

시위하는 사람은 알 것 없고 누구라도 내 앞에서 시위가 좋다느니 담론 최고라느니 하면 걍 진짜 저주한다.
내가 그런 것 참가 안 해본 것도 아니고 담론질이면 고딩 때부터 했는데 이거 '소용'없음.
저런 짓 해보았자 남는 건 없거든. 아주 광기에 빠져있음, 파트리크 쥐스킨트의 향수에 나오는 그 궁극의 향수 뿌리면
강남역 10번 출구는 소돔과 고모라에 나오는 아주 난잡한 광경이 벌어진다고 해도 이상할 게 없을 거다.
\vspace{5mm}

저러면서 다 밥은 먹고 다녔겠지
\vspace{5mm}

그것보다 강남역 지가는 떨어질까 안 떨어질까.
애초에 저런 시위 같은 것 안 했으니까 비쌌지, 저렇게 뚫린 걸 보니까 강남역도 걍 하향평준화 망할 분위기인 것 같다.
\vspace{5mm}

\textbf{$\#$ 강남역 말말말}
\vspace{5mm}

재기해! 재기해! (뛰어내려 죽어버려!) 01:12야! 사진찍지마! 1:33웃지마 씨발!! 1:40야! 사진찍지마! 사진찍지마! 사진찍지마! 사진찍지말라고!!! 1:54사진찍지마!에 해병대 빨간모자와 충돌 1:50그럼 너네도 죽이고 고소하면 되겟네? 4:34재기해$\sim$! (뛰어내려 죽어버려!) 30:57재기해$\sim$! 한남대교가서 재기해! (뛰어내려 죽어버려! 한남대교가서 뛰어내려 죽어버려!)32:02소추소심! 소추소심! (고추가 작으면 마음도 작다) 32:10빻았다 빻았어 33:04소추소심! 소추소심! (고추가 작으면 마음도 작다) 35:06숨쉴한! 숨쉴한! (한국남자는 쉼실 때마다 한번씩 패야한다!) 36:23재기해! 재기해! (뛰어내려 죽어버려!) 36:30울지마 울지마 36:49야 뭐라고? 고추가 작아서 안들려!!! 37:22하 시발 좆까고 저리가라 개새끼들아! 39:58애기 집에가$\sim$ 40:07여기 멍석 깔려있다 아무나 나와라$\sim$ 42:46창민아$\sim$ 팬티내리자$\sim$! 42:59
\vspace{5mm}



\section{희생자 오빠의 반응}
\href{https://www.kockoc.com/Apoc/790615}{2016.05.24}

\vspace{5mm}
\begin{enumerate}
    \item \textbf{피해자 오빠의 반응}
    \vspace{5mm}

    \href{http://news.donga.com/BestClick/3/all/20160524/78280176/1}{링크}
    \vspace{5mm}

    \item \textbf{그들의 반응}
    \vspace{5mm}

    \href{http://mlbpark.donga.com/mlbpark/b.php?m=search&p=1&b=bullpen2&id=5113437&select=sct&query=%EC%97%AC%EC%8B%9C&user=&reply=}{링크}
    \vspace{5mm}

    \item \textbf{ 결국 기사화}
    \vspace{5mm}

    http://news.nate.com/view/20160524n19555
    \vspace{5mm}

    \item \textbf{SNS 반응}
    \vspace{5mm}

    \textbf{유가족도 씹는 패기}
    유가족도 씹는 패기

    \vspace{5mm}
 
    \item \textbf{잡설}
  
    호랑이 양담배 피다 폐암걸려 가죽을 남겼는데 알고보니 점박은 사자였더라 그럼 친아빠는... 이라고 묻던 시절에
    우리나라 여성들은 박해받기 때문에 잠재력을 발휘 못 하고 있다 이게 다 남자 때문이다라는 논리가 꽤 잘 먹힌 시절이 있었다.
    \vspace{5mm}
  
  
    유교 때문에 한 집안의 엄마, 누나, 여동생이 희생하면서 아들만 밀어주니까 아들이 잘 나가는 것 아니냐.
    여자도 얼마든지 똑똑하니까 그런 불합리한 게 사라지면 사회가 훨씬 좋아지지 않겠느냐.
    \vspace{5mm}
   
  
    세월은 흐른다
    \vspace{5mm}
  
  
    공산주의는 진작에 무너졌다.
    유럽의 진보는 '무슬림 혐오'로 바뀌었고 미국에서는 트럼프가 인기다.
    그리고 한국사회가 많이 좋아지긴 했다. 다만 그 여성해방이라는 건 기대 이하였다.
    다들 강남역만 바라보겠지만 잊고 계신다. '청와대'부터가 현재 주인은 여성이다.
    \vspace{5mm}
  
  
    공산주의는 '착한 자본가'와 '사악한 노동자' 혹은 '노동자를 선동하는 사기꾼'은 말하지 못 한다.
    유교에서는 사악한 군주, 위선적인 사대부, 폭력남은 언급하지 않는다.
    페미니즘의 여성은 '악녀'나 '찌질한 여자'가 없다.
    \vspace{5mm}
  
  
    이러니까 저런 사상들은 정말 걸러들어야한다니까. 예외 처리가 안 되어있는 코드니까 에러가 터질 수 밖에 없지.
    \vspace{5mm}
  
  
    개인적으로는 남자니까 이래야 한다 여자니까 저리야 한다... 라는 구태의연하고 케케묵은 관념 따위는 필요없다.
    그냥 남자 여자 관계없이 \textbf{어떤 실적을 보여주느냐만 따지면 되는 것이기 때문에.}
    \textbf{그 사람이 뭘 하느냐가 정말 중요하지}
  
    \vspace{5mm}
  
    여자는 약자고 착하니까 보호받아야 한다와
    여자도 남자랑 똑같이 속물이니까 같은 건 같게, 다른 건 다르게 취급해서 대등하게 대하자 중 어느 게 평등일까?
    \vspace{5mm}
   
  
    그리고 남자 입장에서는 어리둥절할 수 밖에 없는 게
    모든 남자가 다 성범죄나 폭력을 저지르는 건 아니란 것이다.
    그런 식으로 따지면 여자들도 할 말 없는 사례는 정말로 많다.
    이런 걸 하나 얘기하면 마법의 키워드가 나오지. "\textbf{찌질해}"
    그런데 원래 인간은 찌질하니까 별로 소용없는 말이다.
  
    \vspace{5mm}
  
    자기들은 여자니까 보호받아야 한다고 생각하겠고 이미 유족들마저 무시한 포스트잇이 추모라고 착각하겠지만
    정말 멀쩡히 살아가는 남자들 입장에서는 저런 식으로 메시지는 설득은 커녕 "무고한 사람을 죄인으로 몰아가는" 점에서 짜증만 날 뿐이다.
    남자들이 마초라서 포스트잇이나 저런 시위를 싫어하는 게 아니지
    핑코나 여중생 폭행에다가 유가족 모욕에서보다시피 본색은 절대 추모가 아니라는 걸 원래 알고 있었거든.
  
    \vspace{5mm}
  
    물론 이와 별개로 저런 시위는 좀 계속하면 좋겠다. 그래야 호구 남자들이 정신을 차릴 수 있으니까
    적어도 저게 일부 급진론자들의 소행으로만 알던 사람들이 이번에 좀 제대로 깨달았으면 좋겠다는 것.
    이슬람도 그들이 소수일 때야 평화의 종교지 IS와 난민사태 겪고 나서야 사람들이 무슨 평화는 평화하면서 싹 입장 바꾸듯,
    남녀간 문제도 마찬가지다. 이번이야말로 '여성들도 잠재적 폭력배'라는 걸 보여준(그들의 논리대로) 최초의 사례이니 매우 긍정적이다.
    \vspace{5mm}
  
  
    그리고 잠재적 가해자... 뭐 좋은 논리인데 그렇게 따지면 일부 여자들이 돈많은 나쁜 남자 선호하는 것도 일반화시켜도 되겠지 뭘.
    (사실 이건 일반화시켜도 된다고 보던데... 자 논리는 예외없이 처리해야지?)
    남자들도 걍 자기들이 잠재적 가해자라는 걸 인지하고 남성으로서의 의무를 포기하는 게 좋아보인다.
    \vspace{5mm}

    여자들을 보호할 이유도 없고(요즘 같은 험한 시기에 보호한다가 인생 아작나기 좋다)
    데이트 비용 분담할 필요도 없고(그거 하나 하면 요즘 5만원은 기본 아냐? 그렇게 먹고살기 좋냐)
    아울러 굳이 자국여성 고집할 필요 없다. 한국녀는 갓양남 만나고 한국남은 외국녀 만나면 되는 거지 뭘
    마찬가지로 여자들도 남자들 밥해준다 가사노동해준다할 것도 없다. 맞벌이해서 똑같이 벌어오고 가사분담 육아분담 철저히
    시집살이... 요즘 그딴 게 있나 각자사는 거지. 다만 결혼하려면 집값 절반은 여자들도 분담해야지.
    \vspace{5mm}
   
  
    그런데 이렇게 명쾌한 해답이지만 정작 실천할 리...가 있나
    인터넷으로만 떠들고 다 그러고 끝나겠지.
  
    \vspace{5mm}
  
    그런데 다들 하는 이야기지만 남자 입장에서는 자녀만 있으면 딱히 결혼할 이유가 없다는 게 정말 진지한 생각임.
    가사노동이야 어차피 본인도 혼자 하고 거의 다 기계에 의존하며 10년 내에 로봇이 나와 해줄텐데
    애들 양육도 어린이집. 전업주부조차도 애 맡기고 커피숍가서 수다떨고 백화점 가는 게 현실인데 무슨.
  
    \vspace{5mm}
  
  
  
   +
  
   그리고 쟤들이 일부라는 옹호는 하지 맙시다. 극단적 일부의 소행이었으면 그 집단에서 사죄하고 그 일부를 제재했어야죠.
   \vspace{5mm}
   
  
   ++
  
   적어도 현 대통령이 실정을 저질렀어도 '여자 대통령'이라고 남자들이 까지는 않았을 터인데 말입니다.
   \vspace{5mm}
  
  
   저는 문제가 있으면 그냥 깝니다. 한국 시위역사상 가장 ㅄ 같은 시위라고 여겨서 이 글 썼음.
   그 전까지는 광우병 시위, 용산 참사를 깠죠. 광우병은 정말 선동 자체였고 용산 참사는 신나와 화염병을 자기들이 준비했거든요.
   그런데 그건 이유라도 명쾌했지 이건 뭐 딱히 -- 대안이 있는 것도 아니고 자기들이야말로 혐오부추기다가 유족까지 '한남충' 공격 으휴.
   이런 사람들이 뼈빠지게 먹여살려주고 키워준 자기 아버지도 '한남충'이라고 보고 있을 게 뻔하고.
   무엇보다 저 집단 애들이 정말 땀흘려 일하고 고생하면 저딴 ㅄ 시위는 하지도 않았겠죠.
   \vspace{5mm}
  
  
   정말 가난한 사람들이 절박한 심정에서 그런다면 이해는 가는데 이번 것은 아닙니다.
   여혐 공박한다면서 자기들이 남혐 부추기고 있고, 남자들 잠재적 가해자라고 하더니 자기들이 폭행 저지르고 다니며
   추모한다더니 갑질 하지도 않은 유족들 까고 있고 참.
   이런 것 얘기한다고 '너 여성에 대해 분노하는 찌질이냐'하면 전 더 상찌질이가 되겠습니다. 말은 바로 해야할 것 아냐
\end{enumerate}






\section{도서정가제}
\href{https://www.kockoc.com/Apoc/793149}{2016.05.26}

\vspace{5mm}

책(특히 인문서)의 가치를 지나치게 높이 평가한다는 것.
\vspace{5mm}

책 한권 잘 만드는 것은 매우 어렵다(새움출판사에서 나온 [출판 24시] 참조)
그러나 그 책이 독자에게 효용을 주느냐는 건 다른 문제다.
정가제 찬성자들의 문제는 출판사와 종이책의 가치를 고평가하고 있다는 것인데
현재는 역사서 한권이 \textbf{'나무 위키' 페이지 하나만도 못 하다라는 게 현실이}다.
\vspace{5mm}

위키 이야기가 나와서 얘기하면 사실 위키는 출판사들이 먼저 선점하면서 저작권 문제를 분명히 했어야 한다.
그러나 출판사 관계자들은 시대의 흐름을 읽지 못 했다. 그리고 지금 밀리고 있는 것이다.
\vspace{5mm}

도서정가제 찬성자들이 왜 무리한 주장을 하는 걸까 하면 그 배경에는 '인문학'이 있다.
그리고 이게 사람들이 인문학을 점점 경시하는 이유다.
인문학을 공부하면 세상 돌아가는 데 달통한다면 누구라도 공부할 것이다.
그러나 현실은 인문학에 몰두한 사람들은 시대 흐름에 뒤떨어지는 '광신도'라는 인상이 강하다.
\vspace{5mm}

그럴 수 밖에 없는 이유야.
진짜 깨인 사람이라면 '반례'가 되는 사례가 나타났을 때 기상 패러다임을 과감히 버릴 수 있어야 한다.
하지만 수년간 특정 학문을 공부한 사람보고 그걸 버리라고 하는 건 매우 잔혹한 이야기다.
\textbf{수년간 시간과 돈을 들여 특정 이론을 공부한 사람은 그 이론이 쓸모없고 틀리다는 걸 발견하더라도}
\textbf{그 이론을 버리기보다는 현실을 부정하기 시작한다(인지부조화)}
\vspace{5mm}

그리고 사실 이것이 좌파든 우파든 교육문제에 개입하는 이유다.
어린 시절부터 심어진 특정한 가치관은 매우 충격적인 경험을 하지 않는 이상 어지간해서는 사라지진 않는다.
골수 시장만능주의자든 공산주의자든 선비(...)나 페미니스트들이 가치관을 바꾸긴 매우 힘들다.
가치관을 바꾸는 순간 그간의 세월이 \textbf{무상}해지기 때문이다.
예컨대 통일을 한다 쳐도 골치아픈 게 북한 사람들이 정말 그 주체사상을 포기하지는 못 할 거라는 사실.
인생 전체가 그 주체사상에 저당잡혔는데 순간 부인당한다, 죽고싶어질 것이다.
\vspace{5mm}

개인적으로는 이념을 전파하는 것을 안 좋게 생각하는 이유다.
내가 경험한 그런 $\sim$ 주의는 종교와 같다. 그게 실제로 현실에 도움이 되는가... 자기들이야 그렇게 얘기하겠지만 실제로 그런 건 없다.
오히려 위에서 얘기한대로 인지부조화에 빠져서 현실을 부정하기 시작한다.
이데올로기를 공부한 사람이 현실을 바꾸긴... 현실이 바뀔 거라고 자위하다가 궁해지면 그 사상을 팔아먹으려고 노력한다.
혹은 현실이 안 바뀌면 자기들이 바꾸면 된다라고 정신승리한다. 이게 '통진당'의 에너지원이다.
\vspace{5mm}

그 사람들은 긴 말 할 필요없이 그래서 자기들이 어떻게 현실을 바꾸었나 그리고 정확히 예견했나 그것만 보여주면 된다.
\textbf{그걸 하지 못 하기 때문에} 말이 많아지고 쓸데없는 현학적 논의를 덕지덕지 붙이는  것이다.
실증을 못 하면 결국 반론자의 입을 봉하기 위해 폭력까지 가한다.
\vspace{5mm}

그런 건 관계없이 세상에 기여하기 위해서 그런다?
아니 그러지 말고 그런 쓸데없는 데 연연하지 말고 그냥 \textbf{돈 주면 된다니까}.
어떤 이념을 공부해서 세상을 행복하게 만든다? 그러니까 그 성공모델이 누군지 제발 가르쳐주셨으면 좋겠다.
그럼 세상이 어떻게 진보한 건데요. \textbf{그거야 기술이 발달하고 생산력이 높아져서 물질적 여유가 생겨서 그런 거지.}
\vspace{5mm}

극단적으로 말하면 xx 주의 같은 건 공부하지 않아도 된다. 이거 공부 안 하는 게 오히려 낫다는 게 내 판단이다.
그런 걸 공부해서 정말 좋아지면 내가 지금도 공부하고 있을 것이다. 그런데 내가 느낀 거, 그런 것은 \textbf{정말 과장된 것}이라는 사실.
그런 게 정말 심오했으면 그걸 열심히 공부한 사람들이 현실에서 파워를 갖고 세상을 긍정적으로 바꾸었어야지.
\textbf{그런데 이런 질문을 하면 꼭 그 사람들은 답을 제대로 못 하면서 너는 속물이니 왜 물질적인 것만 따지니 세상 그렇게 각박하게 살지마}
\textbf{조용한 데에서 열심히 일하는 그런 사람들이 있어서 세상이 나아지는 거야라는 소설을 써댄다.}
솔직히 이게 사이비 종교의 논법과 뭔 차이가 있나.
그냥 대답만 하면 되잖아.
\vspace{5mm}

도서정가제는 실패했다. 그런데 그걸 밀어붙인 사람들은 여전히 현실 부정을 하면서 완전 정가제를 해야한다 라고 얘기한다(...)
그 옹고집은 어디까지 갈까. 그거야 쓸데없는 책은 안 읽어도 된다는 걸 사람들이 인정하면서 다른 방식으로 지식을 습득할 때까지 지속되겠지.
그리고 그 끝까지 가면 그걸 밀어붙인 사람들은 "실제로 본 뜻은 그게 아닌데 왜곡된 것이다"라고 말 돌리기 시작할 것이다.
\vspace{5mm}

대학에 들어가서 배워야하는 건 \textbf{"자본가"로} 살아가는 법이다.
이게 뭔 소리인가 하는 사람들은 시행착오해보면서 그럼 공부해보시면 안다. 이건 내 말이 맞다고 자신한다.
심오한 이데올로기를 배운다 치자. 그래보았자 \textbf{"그래서 어떻게 문제를 해결할 것인데"}에는 전혀 답하지 못 한다.
문제해결을 위해서 필요한 것들이 결국 \textbf{'자본'}으로 수렴된다.
\vspace{5mm}

도서정가제를 주장한 사람들은 왜 책이 안 팔리는가부터 시작해 사람들이 책을 다시 보게 하는 방법은 무엇일까 생각은 안 하고
정가제로 돌아가면 유통이 정상화되어 출판사가 수익이 늘어서 더 좋은 책을 만들 수 있다 라는
형식적으로는 그럴싸하지만 현실적으로 들어가면 너무나 빈틈이 많은 판타지에 의존했다.
그 사람들은 그래서 중고서점이 더 확장되는 현실을 예측할 수 있을까.
지금 한다는 게 왜 대형서점이 중고거래에 진출하느냐 분개(...)하는 건데 정가제 아니었으면 중고서점거래가 늘지도 않았다는 점에서
자기들이야말로 중고서점의 친부라는 걸 모르는 것 같다.
\vspace{5mm}






\section{갚으면 된다}
\href{https://www.kockoc.com/Apoc/793885}{2016.05.26}

\vspace{5mm}

윤리에 호소한다는 건 그다지
\vspace{5mm}

바람난 남녀는 자기의 간통을 '로맨스'라고 바라보고 있음.
사실 그건 핑계대려면 얼마든지 댈 수 있습니다.
심지어는 배우자가 제대로 사랑해주지 않아서 죽을 것 같아 바람폈다... 라는 개드립도 가능하죠.
\vspace{5mm}

사건사고 뉴스에 나는 범죄들이 정말 범죄자들이 자기가 비윤리적인 걸 자각하고 저지르는 건 아니죠.
다 나름대로 핑계를 대고 변명할 여지를 마련하고 있습니다.
친딸을 xx한 파렴치한 친부나 계부들도 어쩔 수가 없었다 핑계를 대지요.
\vspace{5mm}

보통은 이렇게 요구하죠. "사과해"
\vspace{5mm}

그런데 그것도 말도 안 되는 이야기입니다.
사람을 죽여놓고 사과하면 그만?
돈을 안 갚고 사과하면 그만?
\vspace{5mm}

사실 가장 확실한 건 \textbf{"갚는 것"}입니다.
다만 살인이나 손발 자르는 건 그것 자체가 또 다른 사회적 피해를 낳기 때문에 금하는 것이지
원래대로라면 원상복구 아니면 응보형이 정답입니다.
이걸 전제한 다음에 '인권' 이야기를 하는 거지, 갚는 걸 전제하지 않는 인권타령은 '책임회피'입니다.
\vspace{5mm}

만약 이런 글이면 보통 "네가 뭐라든 나는 xx 할 건데 왜 시비냐"라고 하지만,
미래형은 원래 거짓말입니다. 실천이 있어야 참말이 되지요.
취업하면 엄마아빠한테 잘 할게요... 이게 참인 것 같죠? 본인이야 그렇게 생각하고 부모에게 뜯어먹겠지만
산전수전 겪으면 부모님들은 그게 거짓말인 것 잘 알고 계실 겁니다.
\vspace{5mm}

다 윤리타령하지만 현실이야 뭐. 결국 돈문제 가면 대부분 다 치졸하죠.
\vspace{5mm}

지금하고 있는 공부들이나 열심히 합시다. 지금 하는 공부도 지독하게 안 하면서 미래에 $\sim$ 하겠다라고 하려면 먼저 입술에 침부터 바르시고.
누구나 발언권은 있죠.
그리고 흔히 이런 말을 하죠.
소수의 의견을 존중해야 한다.
\vspace{5mm}

예, \textbf{존중만 하고} 채택 안 하면 그만입니다.
흔한 남녀갈등 떡밥도 그렇습니다. 여자가 돈을 많이 내면 여자 말이 갑입니다. 그런데 지금은 남자가 더 많이 분담하잖아요.
젠더 문제니 성차별이니 그런 이야기하지말고 여자가 더 돈을 많이 내면 여자 말대로 해도 됩니다. 그러니 돈을 내시면 되는 겁니다.
돈을 적게 내고 주장해보았자 존중만 받고 끝날 뿐이죠.
수험생들도 공부 안 하는 사람들 말은 역시 존중만 하지 그냥 무시해버리면 그만입니다.
\vspace{5mm}

이렇게 정리하면 발언$-$책임을 대응시킬 수 있어서 무질서가 줄어듭니다.
길게 말 할 필요 없이 부모 말이 마음에 안 들면 부모에게 그만큼 돈을 갚으시면 되겠고(성년자라면 필수)
상대와 얘기할 때 자기 발언의 실효력을 확보하고 싶으면 실적을 보여주시면 됩니다.
특히 돈문제가 걸리면 거래는 분명 대등히 해야합니다. 그리고 자기가 더 권한 갖고 싶다? 그럼 돈을 더 내시면 되지요.
\vspace{5mm}

이렇게 하면 말끔히 정리되는데 현실은 돈을 내지 않는 사람이 이상한 이데올로기 끌어와서 아주 천외마경을 만들고 있죠.
며칠 전 강남역 문제도 포스트잇 타령하지 않고 자기들이 돈내서 피해자 구호를 한다든가 실질적으로 뭔가 했으면 되는 것이지요.
돈은 내기 싫다, 그런데 남자들은 사과하라.... 이러니까 짜증만 유발시키는 겁니다.
\vspace{5mm}

적어도 이 글을 보는 분이면 고길동이 정말 천사라는 사실은 알고들 계시겠죠.
혐오감정 어쩌구저쩌구 그거. 그냐 서로 혐오하시는 데 그러니까 서로 의존들 안 하면 됩니다.
남혐도 여혐도 다 근거가 있음, 뻔한 이야기니까 다시 설명 안 해도 되죠.
그런데 물질적으로 의존하는 것까지 무시하고 정당화하지 맙시다. 진짜 이것까지 부인하면 그건 인간이길 포기하는 것입니다.
한남충이라고 비하하는데 금전적 지원은 받아야겠다, 김치녀라고 까지만 돈은 뜯어내도 상관없다... 이건 미친 것이지요.
\vspace{5mm}

\section{금전거래는 부모자식형제도 정확히 해야한다.}
\href{https://www.kockoc.com/Apoc/795089}{2016.05.27}

\vspace{5mm}

이건 부모님들의 잘못이 크다.
부모님들의 고민 중 하나가 자식들끼리 불화가 생기면 어떡하냐... 인데 결국 불화는 생긴다.
왜냐면 '분배' 문제를 정말 공정하지 않게 하기 때문이다.
\vspace{5mm}

가령 물려줄 재산이 9천만원이라고 하자. 그런데 형제가 셋이 있다.
그냥 삼등분하면 되지 않느냐 하겠지만
알고보니 장남이 대학을 포기하고 경제활동하면서 기여한 게 있고
차남은 경제활동을 안 하는 대신 대학에 가서 등록금 지원을 받았으며
막내는 아직 학생이라 뭔지 모른다.
\vspace{5mm}

이런 경우에 3등분이 가능하느냐 하면 그건 아닐 것이다. 가령 9천만원 중에 5천만원을 장남이 기여한 것이라면?
하지만 차남은 이걸 인정하지 않으려고 한다. 그리고 자기가 대졸해서 대기업 가면 갚을 수 있다고 말할 것이다.
그리고 막내는 재산관리 능력이 없어서 누군가 관리해줘야하는데 형들을 믿을 수 있을까.
\vspace{5mm}

게다가 부모라고 해도 자식들을 차별하는 게 많다. 예쁜 자식과 미운 자식이 엄연히 나뉘기 때문이다.
예쁜 자식에게 더 주고싶고 만약 법만 아니라면 몰빵해주고 싶은 부모도 있다. 다만 현실에서는 엄연히 유류분이 있다.
이렇게 되면 이야기가 매우 복잡해진다.
\vspace{5mm}

결론적으로는 부모자식간이라도 금전거래는 분명히 해야한다.
그런데 우리나라는 이게 상당히 물러터져 있다. 그래서 터지는 비극이 많다.
이런 걸 보다보면 돈문제에 대해서 대충 넘어가는 사람들에 대해선 참 곱게 보기 힘들다.
\vspace{5mm}

도박묵시록 카이지에서는 '목숨보다 돈이 중하다'라는 말이 나온다. 사실 반례도 없는 건 아니지만
그만큼 돈이 생명에 필적해나간다는 의미로 받아들인다면 생각할 거리는 많다.
여기 친구들이 공부하는 것도 좋은 대학에 가는 것도 학문 때문이 아니다. 바로 \textbf{돈 때문}이지.
\vspace{5mm}

기본적인 금전거래감각도 제대로 안 하면서 추상적 담론을 이야기한다... 잘못되어도 한참 잘못된 것이다.
돈문제에 있어서 매우 비굴한 자세를 취하면서 거창한 메시지를 이야기한다... 미성년이면 이해는 가지만 성년이라면 이야기는 다르다.
일반화시킬 수 없는 개인적인 경험이라고 하지만 그런 사람들이 어떻게 변절했는가 똑똑히 알고 있다.
(더 적자면 돈과 이성 문제가 지저분한 사람은 그냥 조기에 차단해버리는 게 낫다)
\vspace{5mm}

시민단체들에 대해서 곱게 볼 수 없는 이유가 있다. 뭘 모르는 사람들이야 네가 뭘 아느냐 공부는 해보았느냐.
그런 것 필요없다. 그냥 그 단체 재정구조를 보아서 돈을 어떻게 충당하느냐, 그리고 일하는 간사들은 임금을 어떻게 주느냐 보면 된다.
정부지원금을 많이 받고 있는데 간사들은 저임금 받고 있으며 착취당한다면 그 쪽에서 하는 얘기가 거창하든 말든 걍 무시해도 된다.
정말 올바른 곳은 회원들이 알아서 회비를 내며 일 시킬 때 임금을 정확히 준다.
시민단체가 정부지원금을 받는다면 결국 정부나 특정 정당 입맛에 맞는 활동을 하게 된다.
\vspace{5mm}

우리사회에서 소위 아버지라고 하는 사람들은 불륜을 저지르고 마누라 자식을 폭행하고 하는 등으로 많이 까인다.
물론 그건 분명히 까야할 건 많다. 그런데 왜 그러면서도 마누라와 자식들은 헤어지지 못 하는 건가도 생각해보아야지.
그건 그 가부장이 \textbf{돈을 벌어오기 때문}이다.
물론 돈조차 벌어오지 않는 더한 막장도 있지만 이 경우는 바로 이혼해버린다.
그러나 남편이 잘못된 걸 알면서도 헤어지지 못 하는 아줌마들이나 아버지라고 하면 이를 바득바득 가는 친구들도 결국 의존한다.
그거야 '돈버는 것'은 그리 쉽지 않기 때문이다.
\vspace{5mm}

이에 대해서 적절한 대응은 뭘까. 사실 돈버는 것만큼은 인정해줘야 한다. 그게 아니면 본인들이 \textbf{벌면 된다.}
그리고 콕콕에서도 알바 뛰어본 친구들이 많겠지만 이게 보통 어려운 게 아니라는 것도 알 것이다.
나이가 어릴 때는 돈버는 게 쉽다고 생각하면서 어른을 불신하게 되지만, 자기가 그 어른이 되면 정말 '아버지의 진심'이 뭔지 알게 된다.
현대사회에서 \textbf{진심의 척도는 돈이다.}
\vspace{5mm}

그런데 재밌는 건 돈을 지불해주는 사람을 까면서, 자기에게 사실상 한푼도 주지 않은 사람들 말\textbf{'만'} 믿는 사람들이 많다.
이게 인터넷 세상의 세뇌효과일지도 모른다.
부모 자식간의 대화는 줄었다. 그 시간에 다들 인터넷이나 SNS를 한다.
자기랑 대화하는 사람 말에 빠지는 것이다.
\vspace{5mm}

재밌는 건 그런 사람들에게 '네가 받은 만큼 갚아보라'하면 결국 변명하게 된다는 것이고
나 역시 그 점에서는 자유롭지는 못할 것이다.
다만 아무리 부모자식간에 원한이 있더라도 차분하게 부모가 나에게 얼마를 썼느냐 계산해보면
그리고 판 같은 데 올라오는 주작인지 아닌지 모르는 막장썰을 읽어보면 '내가 행복한 케이스구나'를 느끼는 경우가 많다.
\vspace{5mm}

다른 건 떠나서 give$\&$take 안 하고 자기들이 받는 걸 당연하게 생각한다면 난 그 친구들을 쓰레기 취급할 것이다.
무협지의 흔한 코스가 원수에게 가족을 잃은 주인공이 무림고수에게 갔을 때 '궂은 일'부터 하는 것이다.
세상물정 모르는 친구는 당장 필살기나 가르쳐주지 일부터 시켜먹네 답답하다... 라고 느낄 것이지만
경제관념이 있는 친구들이라면 "수업료를 저렇게 받는구나"라고 생각할 것이다. 세상에 공짜가 어딨나?
\vspace{5mm}

대학에 들어가는 것보다 중요한 건 이런 관념을 갖는 것이다.
심지어 좋은 대학에 들어가서조차도 이런 경제관념이 없으면 정말 답이 없어진다.
\vspace{5mm}

+
\vspace{5mm}

여담이지만 공정거래 금전거래만 똑바로 한다면 본인이 뭘 하든 누구든 말릴 자격이 없다.
하지만 자기가 하고싶은 일을 $-$ 그것도 허황된 것으로 벌이겠다는 사람들이 자기가 돈 벌어서 하는 게 아니라
결국 부모에게 기대기 때문에 부모 입장에서는 그걸 허락하지 못 하는 경우가 많다는 게 트루다.
\vspace{5mm}

최근에 대작으로 걸린 모 가수만 하더라도 사실 자유로운 영혼이었다지만 그 배후를 보자
대작이든 아니든 그림이 고가에 팔려나간다... 이것이 바로 그 자유로운 영혼의 '진면목'이었다는 것이다.
거래는 어찌되었든 거래 아니냐 라면 할 말은 없다. 다만 그게 불공정거래로 보인다는 게 문제지.
\vspace{5mm}

아무튼 주변 사람들에게 민폐 안 끼치고 자기가 돈 벌어서 자기가 하겠다는 게 법이나 윤리에 어긋나는 게 아니라면 누구든 말릴 자격이 없다.
자기가 번 돈을 쓰는 시도라면 진심으로 노력하게 되고, 진심으로 노력한다면 실패하더라도 그 과정에서 얻는 게 많다.
N수하는 사람들이 실패하는 이유야 셀 수 없이 많지만 '자기 돈'으로 안 하는 경우가 많다는 것도 큰 이유다.
자기 돈으로 하는 거라면 매우 절박하니까 죽기살기로 하게 된다. 돈나가는 것에 피말리니까 손해보기 싫으니까.
그런데 보통은 자기 돈으로 하는 경우가 드물다.
\vspace{5mm}

물론 돈쓰지말라는 건 아니다. 그런데 차용증은 쓰고 하라는 것이다.
학원비를 고가로 지불해도 좋다. 그런데 그걸 자기가 부모에게 갚는다고 약속하고 공부해야 한다는 것이다.
이건 비단 N수 뿐만 아니라 대학에 들어가서도, 그리고 대학에서 졸업한 이후에도 필요한 가치관과 습관의 정립을 위해 필요하다.
이런 관념없이 그냥 부모가 지원해주는 데로 N수하고 운좋아 대학에 들어갔는데 등록금도 부모가 다 대준다...
하지만 차용증이 없으니 갚을 필요가 없다고 치자... 이런 친구가 열심히 살 것 같나? 인간은 간사한 존재인데?
\vspace{5mm}

과거에는 저런 게 없어도 먹고 들어갈 수 있다. 기성 체제는 공부만 잘하면 '일자리'가 보장되었으니까.
그런데 지금 결혼과 취업이 사라지는 시대다. 이제는 일자리에 집착하지 말아야 한다. \textbf{본인이 일거리들을 물어와야한다.}
남녀간도 프리섹스를 자유롭게 하는 시대라면 배우자를 독점하는 걸 전제로 한 결혼도 무의미해진다.
결혼제도가 의미가 있다면 이제 '계약혼' 정도일 것이다. \textbf{상호 행위, 재산, 그리고 자녀에 관한 계약 전반.}
신랄하게 말하면 사랑이라는 건 3년이 지나면 감가상각되는 권리금에 지나지 않는다.
\vspace{5mm}

이런 변화에 맞는 가치관을 정립하는 게 중요하다. 그 사람이 N수를 하는데 N이 얼마냐 그건 사소한 문제일 것이다.
그러나 아직까지 이런 가치관이 정립되지 않은 친구들도 많다. 돈을 정말 사소하게 생각하는 걸 넘어 기본적인 거래관 자체가 없다.
저런 거래관을 갖춘 사람들에게는 금전감각이 물러터진 사람들이 조선시대 사람들로 보일 것이다.
\vspace{5mm}






\section{[뉴스] 올해 가장 더운 여름 확률 95$\%$}
\href{https://www.kockoc.com/Apoc/795983}{2016.05.28}

\vspace{5mm}

\href{http://news.naver.com/main/read.nhn?mode=LSD&mid=sec&sid1=104&oid=016&aid=0001053552}{링크}

\noindent\fbox{%
    \parbox{\textwidth}{%
    미국 항공우주국 NASA가 2016년의 여름이 사상 최고로 더운 해가 될 것이라는 전망을 내놓았다.
    미국 항공우주국 NASA 관계자는 “올해 들어 4월까지 기온을 미루어 볼 때 올해가 가장 더운 해가 될 확률이 95$\%$ 이상이다”고 전했다.
    \vspace{5mm}
    
    또한 미국 국립해양대기청도 “137년간 기상 관측 이래로 지난 4월은 가장 온도가 높은 달을 기록했고
    이는 지난해 5월 이후 한 달도 빠짐없이 가장 더운 달을 이어가고 있다”고 밝혔다. 
    
    \vspace{5mm}

    }%
}

알아서 잘 대비하시길
