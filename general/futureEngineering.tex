


\section{[미래공학 001] 다보스포럼의 일자리 경고}
\href{https://www.kockoc.com/Apoc/606699}{2016.01.23}

\vspace{5mm}

\href{http://news.mk.co.kr/newsRead.php?no=51908&year=2016}{뉴스링크}
\vspace{5mm}

인공지능, 바이오 등 하이테크놀로지로 대표되는 4차 산업혁명이 본격화되면 전문 기술직에 대한 수요는 늘어나는 반면 단순직 고용 불안정성은 더욱 커진다. 보고서는 미래에 일자리가 감소할 것으로 전망되는 산업군으로 사무행정직군, 제조업생산, 건설채광업 등을 꼽았다. 사무행정직에서 470만개, 제조업생산 160만개, 건설채광업 50만개의 일자리가 사라질 것으로 전망했다. 반면 재무관리(50만개) 매니지먼트(41만개) 컴퓨터수학(40만개) 등은 일자리가 새로 만들어질 것으로 내다봤다.
\vspace{5mm}

라고는 하는데 저걸 곧이 곧대로 믿을 필요는 없습니다.
저건 기술 발달로 인한 대체 가능성을 이야기하는 것이죠.
\vspace{5mm}

기술로만 직업을 논하면 상당히 많은 오류에 빠지게 됩니다.
그 논리대로라면 지금 인기있는 직종이 프로그래머여야 하는데 현실은?
과거에 의대가 비인기였던 이유 중 하나도 그것도 역시 대체될 수 있다고 믿어서인데 지금은?
물론 의치한도 앞으로 매우 불안해질 수 있다고 보는데
우리나라에서 의사의 고소득은 '기술'이 아니라 '제도'가 간접적으로 보장해주고 있기 때문입니다.
의대 정원수 통제, 건강보험 시스템, 거기다 전문의 분과.
정부에서는 만약 경제 활성화를 위해서라면 의사들의 권리를 조금씩 박탈해갈 수도 있고, 사실 지금도 그런 움직임이 엿보이고 있죠.
\vspace{5mm}

인간은 기술을 지배할 수 있습니다. 그러나 마음을 지배하지 못 하죠.
그 마음을 간접적으로 지배하는 것은 바로 제도입니다, 그런데 그 제도는 법률의 통제를 받게 되고
법률은 결국 여론을 따라가게 되어있습니다.
독재가 되지 않는 한은 결국 표심에 따라 모든 게 결정되고, 그건 결국 다수의 의사를 좆게 되어있습니다.
예, 여기서 눈치 까셔야죠. 소위 전문직이든 고학벌은 '소수자'에 속한다는 것입니다.
저기서 언급한 재무관리, 매니지먼트, 건설 수학 등은 '공부'를 안 하는 사람은 할 수가 없습니다.
그러나 거꾸로 보자면 그만큼 어느 사회에서든 '소수'만이 담당한다는 이야기인데, 이 소수들은 \textbf{정치적으로는 약자에 속합}니다.
\vspace{5mm}

수험사이트에서는 지나치게 의사가 최고다라는 신앙이 강합니다만.
정작 의사들은 '권력'이나 '명예'가 있는 직업을 부러워하고 있고 (사실 돈번다고 해도 자기들이 다 쓰는 것도 아니고)
명예를 갖춘 판검사들은 겉으로나 그렇지 실제로는 격무에 시달리고 있어서 뭔가 힐링이 되는 것을 원하고 있어서
다 상대적이지 않냐 하겠습니다만.
\vspace{5mm}

아무튼 이건 간단한 이야기는 아닙니다.
공부를 잘 하면 당연히 의치한에 가는 게 '안전'하다라고 얘기할 수 있을 지언정 정말 그게 좋은 길인가하는 건 회의적입니다.
고소득이 목표라고 하면 - 돈이 좋아서라고 하면 괜히 인술 베풀고 싶어서 의대간다 그럴 필요 없이
돈에 관한 전공을 밟는 게 나은 것이고
마찬가지로 정치를 하고 싶다라고 하면 위선적으로 남을 위해 봉사한다거나 사회학을 공부하고 싶다... 그럴 필요 없이
적당한 전공 밟아서 스스로 자기 이름 알리고 시민으로서 운동하다가 정당에 들어가 경쟁하면 되는 것입니다.
\vspace{5mm}

예컨대 돈의 속성을 모르면 상당한 오류에 빠지는데
가령 의사들이 월 천 정도는 가볍게 번다에서 간과하는 것은 현재 강남 아파트 가격 - 대충 8억대라고 잡으면 뭔가 착잡해질 수 있습니다.
강남 부동산이 폭등한 게 바로 2000년대 초중반이었는데(그리고 집집마다 부부싸움이 잦았다죠)
그 이전만 하더라도 강남 아파트 가격이 평균적으로 3억대 정도 했고(이게 틀리면 지적해주시길 바랍니다)
그 때에도 의사들은 현 수준 정도는 온갖 합법적, 탈법적 수단으로 벌어들었기 때문입니다.
제가 서초구에 살 적에 자주 다녔던 모 피부과의 경우는 강남이 별 게 없었을 때 동네 수준(?)으로 개업했는데
성실히 모아서 투자한 부동산 폭등으로2010년 초에 100억대까지.
\vspace{5mm}

그럼 이 예를 왜 드느냐.
현금으로만 계산하는 경우 인플레를 감안 안 하기 때문에, 즉 실질 가치를 생각하지 않기 때문에 오류에 빠질 수 있단 것입니다.
부동산이 폭등해서 집값이 2억에서 5억으로 올랐다는 건 부동산의 실질 가치가 올라갔다기보다는
반대로 현금 가치가 그 정도로 형편없이 떨어졌다고 해석하는 게 맞습니다.
실제로 2000년대 초중반 부동산 폭등의 배경에는 시중에 엄청난 자금이 풀린 탓이 컸죠.
그 자금이 풀린 건 당대 정치권력의 의도가 있었습니다.
\vspace{5mm}

저런 걸 읽어낸 사람은 누워서 수십억을 벌었고, 그걸 모른 채 부동산이 폭락할 거라는 걸 믿은 사람은 앉아서 수억을 잃었죠.
사실 지금의 수저론을 낳은 빈부격차가 그 때 확정되었다고 보는 게 맞을 것입니다.
\vspace{5mm}

아무튼 다보스의 경고대로 일자리가 바뀔 건 사실이라고 쳐도
저건 어디까지나 '기술'로만 평가한 것이니 너무 신뢰하지 않는 게 좋으며
특히 공부하는 입장이라면 당장 돈벌이 이전에 앞으로 인생에 도움이 되는 것을 공부하는 게 좋습니다.
예컨대 저기 법률이 내려간다.... 다소 회의적입니다만 법을 잘 안다는 것부터가 힘인데 과연 법조인이 그리 쉽게 망할까요?
법조인이 직업이 아니더라도 법을 공부해두는 것 자체가 삶에 많은 도움을 줍니다.
\vspace{5mm}

그리고 정말 부자가 되고 싶다면 위에서 말했다시피 '돈'을 다루는 그런 전공을 가는 게 백번 낫습니다.
\vspace{5mm}






\section{[미래공학 002] 의료인의 정원통제}
\href{https://www.kockoc.com/Apoc/607568}{2016.01.24}

\vspace{5mm}

\href{http://www.newspim.com/news/view/20150916000272}{뉴스링크}
\vspace{5mm}

살펴보면 2008년 국내 병원의 CT와 MRI보유대수는 인구 100만명당 각각 36.5대와 17.5대 수준이었고, 병상수도 인구 1000명당 7.7병상이었다. 이에 반해 의대 졸업자수는 2008년 9.1명보다 오히려 줄었다. 임상의사수는 0.3명늘긴했지만 평균에 크게 못미친다. 더구나 임상의사는 한의사를 포함한 수치다. 한의사 수가 지난 10년간 40$\%$이상 증가한 것을 감안하면 실제로는 늘지 않은 셈이다.
\vspace{5mm}

한의사수까지 포함해보자면 시장에 비해서 '공급 통제'가 되고 있기 때문이다라고 한다면
저건 언제든지 어떤 정권이 들어서 어떤 정책을 펴느냐에 따라서 뒤집힐 수 있는 것입니다.
따라서 의사가 많이 버는 것이 정말 의료 쪽이 블루오션이고 비전이 좋아서인가.... 라기보다는
수요가 확실하고 건강보험체제까지 갖춰져 있어서 수입을 보장해주는데 정원이 통제되어있기 때문이다...
이렇게 보는 것이 맞는 것이지요
\vspace{5mm}

바꿔 말해서 공대의 경우 만약 기계, 전기, 화학에 대해서 공대 석사까지 마치지 않으면 면허를 받지 못 하게 되어있고,
그런 면허 소지자가 아니면 대기업에 입사할 수 없다라고 한다면
아울러 연구분야에서 일하면 해당 기술에 대해서 특허까지도 보장된다고 한다면
의대 저리가라일 것입니다. 물론 현실은 공대는 그 어떤 권리도 보장받지 못 하고 있습니다.
\vspace{5mm}

그리고 여기서부터는 매우 정치적인 사건이 되어버립니다.
왜 공대인들은 권리를 보장 못 하는가. 공대졸업자들의 권리가 보장되면 대기업의 입지가 약화되어버리니까요.
그런데 이건 마찬가지로 우리나라 대기업들이 의료 분야 쪽으로 먹거리를 창출하게 되는 경우 의료인들에게도 일어날 수 있는 일입니다.
현재 의대가 잘 나가는 것에 대해선 IMF다 경기불황이다라고 하지만
가장 직접적인 사건은 아랫 기사에서 설명된 것입니다. 바로 \textbf{'의약분업'}이죠.
\vspace{5mm}

\href{http://www.donga.com/docs/magazine/weekly_donga/news249/wd249ee020.html}{뉴스링크}
\vspace{5mm}

결과적으로는 - 적어도 현재까지는 어설프다고 할지라도 상호윈윈을 보장받은 구조입니다.
사회현실을 모르는 사람들이야 의사들이 많이 번다 이게 다 수학을 잘 해서라고 자뻑하겠지만
문명권의 현대사회에서 먹고사니즘의 갈등은 모든 게 정치로 시작해서 정치로 끝납니다.
사실 저런 것이야말로 본질적인 정치인데도 이런 것은 '복잡하고 어려우니까' 외면하는 사람들이 대부분이니 말입니다.
\vspace{5mm}

그럼 의대정원이 앞으로 늘어날 일은 정말 없다고 장담할 수 있나.
\vspace{5mm}

적어도 제가 하소연 듣고 얻는 정보로 치면 그 세계도 수저론에서는 전혀 자유로울 수 없고
오히려 그 쪽 사람들은 상류층들이 몰려있기 때문에 평범한 사람들의 세계보다 더 빡센 감도 없지 않지만
법률 쪽만 하더라도 로스쿨로 다량배출을 승인해준 것이 기성 세대가 법인화, 조직화로 거대자본화하면서
신규 인력들을 싸게 부려먹고 실질적으로 경쟁자들을 줄이는 효과를 노린 것이라고 한다면 실제로 여기도 장담은 못 하는 것입니다.
5년 전만 하더라도 쭉 연금 잘 받고 갈 수 있던 공무원들도 지금 점진적으로 불이익을 받는 방향으로 가고 있죠.
\vspace{5mm}

과연 앞으로 철밥통이 존재할 수 있느냐라는 일반론으로만 가더라도
의치한 가면 잘 먹고 잘 살 수 있을 것이다라는 건 현재의 현상에 기초한 기대에 불과한 것이며
이것이 정확한 미래를 담보하지 못 하는 한 얼마든지 붕괴될 수 있는 판타지일 수 있다는 것도 생각해보아야겠지만.
결국 권력을 현재 누가 쥐고 있으며 앞으로 누가 신흥강자가 될 것이냐고 생각해보면
현재 우리가 기대하는 것과는 매우 다른 결론이 나올 수도 있다고 생각해볼 수도 있습니다.
\vspace{5mm}

기존의 제국주의적 식민수탈이나 선진국의 후진국 착취는
자본을 앞장세워서 저렴한 노동력을 착취하고 자원을 수탈해나간다로 요약되었지만
현재와 미래의 자본주의가 \textbf{1}순위로 꼽는 것은 바로 '인구'입니다.
이제는 선진국들이 발목이 잡힌 게 바로 저출산이며, 이건 이민으로도 해결될 수 없다는 것이 뚜렷해졌지만.
일단 물건을 만들더라도 시장이 있어야 개인이든 법인이든 활동할 수 있습니다.
그래서 중국이 각광받았던 것이고, 현재는 출산율이 높은 무슬림들의 눈치를 봐야하는 시대가 오고 있는 것이지요.





\section{[미래공학 003] 돈의 악취}
\href{https://www.kockoc.com/Apoc/608899}{2016.01.25}

\vspace{5mm}

의대 가면 좋다 어쩌구하는 것에 대한 기시감
풋내기 시절에 들었던 "공대가면 놀아도 취업하고 꿀빤다"와 똑같음.
현재 의대에 관한 환상과 전망이 당시 공대에 관한 그것과 똑같다고 하면 에이 그럴리가요 했지만
\vspace{5mm}

\textbf{"대기업이 무너질 리가 없다"}

\textbf{"쭉 이대로만 태평성대. 대학만 가면 무조건 잘 먹고 잘 산다"}

라던 사람들이 그와 정반대 현실이 오자마자 소신(?)을 바꿔버리는 일이 벌어져버린다.
\vspace{5mm}

2000년대에는 좋은 데에 취업하고 싶어서
신의 직장이라는 xx은행 같은 데를 때려치우고 고시공부를 하러 나가는 사람들도 많았고
한의사가 많이 벌 수 있다 하면서 빡세게 공부해서 한의대 다시 가는 사람들도 있었지만?
지금은 \textbf{신의 직장}만 다닌다면 전문직도 필요없다... 라는 분위기가 아닌가?
돈을 적게 받아도 좋으니 안정성 있는 직업을 갖고 싶어서 과장이 아니고 늘그막에 공무원 학원을 다니는 사람들도 널렸다.
\vspace{5mm}

여기서 얻은 지혜는 하나.
돈냄새가 싸구려 향수의 악취처럼 처럼 풍기기 시작하면 \textbf{일단 거기선 무조건 탈출해야한다는 것}이다.
지금 의사들이 상대적으로 잘 나간다고 해도 지금 레지까지 마치고 나가는 사람들이 '돈냄새' 맡고 의대가진 않았다.
역설적으로 돈냄새를 맡았다면 공대에 갔어야할 것이다. 당시에는 그게 더 돈을 많이 버는 길로 인식되고 있었으니까.
금전의 악취가 풍긴다는 것은 수익성을 보장 못 하는데도 그 분야로 '개미'들이 푼돈을 털어넣기 시작했다는 이야기이다.
\vspace{5mm}

실제로 2000년대에 돈을 많이 번 사람들은?
IMF 때와 서브프라임 때 주식이 개폭락했을 때 슬그머니 매집해서 지금은 투자의 현인으로 칭송되는 사람들,
동일한 원인으로 부동산으로 올인해버린 사람들, 무엇보다 유학을 포기하거나 공부를 못 해서 그 돈을 부동산으로 돌린 사람들.
다들 망할 줄 알고 손해보기 싫다고 손절매할 때 미래를 내다보고 과감히 베팅한 사람들이 성공했다.
\vspace{5mm}

지금 의치한 열풍의 문제는 "돈냄새" 보고 간다는 것인데.
실상은 어디를 가도 정말 이런 데에까지 의원, 한의원, 치과가 들어서 있다는 데 놀란다.
그런데 지금은 다시 인구 절벽 현상으로 지방이 몰락하고 지방의 어여쁜 처자들조차 살아남기 위해 수도권으로 올라가지 않나.
하지만 가장 중요한 건 바로 "구매자"들의 존재다. 의료서비스 공급이 늘어나는데 수요도 그만큼 늘어난다, 아니 실질 구매력이 담보되나?
혹자는 보험을 이야기할지 모른다. 그런데 이건 정치에 종속되는 것이라고 실토하는 것이다.
다들 무관심할지 모르지만 이번의 보육사태가 의미하는 게 무엇이겠나.
\vspace{5mm}

콕콕에서 몇명은 소신껏 공대 지원을 한다. 사실 칭찬해주고 싶은 사람들이 이 친구들이다.
자기가 좋아하는 공부를 우선 하는 게 맞기도 하지만, 무엇보다 지금은 공대에는 돈의 악취가 나지 않는다는 것도 그렇다.
마찬가지로 고득점 맞은 친구가 취업과 관계없이 인문학이 좋아서 인문대에 가서 술 안 처먹고 공부한다면 아마 달리 볼 것이다.
그런 사람들은 전공에 관계없이 뭘 해먹어도 크게 해먹을 수 있다.
남들이 비웃거나 가난하다고 깔볼 때에 비현실적이라고 생각하던 것을 현실적으로 노력하고 경주하던 사람이 나중에 큰일내지
그저 돈 많이 번다고 그거 따라간 사람들은 진짜 문자 그대로 평범하거나 그 이하로 밖에 살지 못 한다.
\vspace{5mm}





\section{[미래공학 004] 인구 없는 화폐는 휴지}
\href{https://www.kockoc.com/Apoc/609990}{2016.01.26}

\vspace{5mm}

화폐에 대한 가장 간명한 정의는 빚문서
이영애가 주연한다는 신사임당이 그려진 누런 종이조각은
\textbf{"소유자에게 정부는 50,000원 어치의 빚을 지고 있다"}
\vspace{5mm}

사실 이것 이상 이하도 아니지만
이것의 함의는 중대하다.
\vspace{5mm}

빚은 갚는 사람 마음인 것이다.
채무자에게 채권자가 가혹하게 나설 수 밖에, 아니 나서야하는 건 당연하다.
채무자가 갚지 않으면 채권자는 개털이기 때문이다.
채무가 5억인데 채무자의 재산이 5000만원에 불과하면 열심히 털어도 그것 밖에 받지 못한다.
그럼 앉아서 4억 5천만원을 떼먹히게 된다.
그래서 채권자들이 사용할 수 있는 온갖 법률적, 사실적 수단들이 많지만,
이건 역설적으로 그만큼 빚 받아내기가 쉽지 않다는 걸 이야기한다.
\vspace{5mm}

그리고 눈치빠른 사람은 여기서 파악할 것이다.
개인 대 개인인 채권채무 관계도 어려운 데
하물며 개인이 어떻게 정부로부터 빚을 제대로 받아내지?
개인과 정부가 지닌 힘은 비교할 수도 없지 않나?
거기다가 사실상 화폐 발행과 유통도 정부가 담당하고 있다.
\vspace{5mm}

오르한 파묵의 "내 이름은 빨강"에 재밌는 대목이 있다.
예니체리(=오스만 투르크의 정예군)들이 급료로 받은 악체화를 물 위에 던졌더니 둥둥 떠서(...)
바로 솥을 뒤엎고 반란을 일으키는 장면이 나온다(이와 비슷한 원리가 바로 아르키메데스의 왕관)
국가들이 멸망하기 전에 꼭 등장하는 것이
화폐에 넣는 금의 양을 인위적으로 줄이는 것이니 이렇게
정부에서 돈 가지고 장난치는 건 매춘, 사기와 더불어 오랜 역사를 지니고 있는 셈인 것이다.
\vspace{5mm}

그렇다면 A란 사람이 화폐로만 10억을 갖고 있다치자.
무엇보다 현금이 안정적이다라고만 믿고 갖고 있어도 그 재산을 보존할 수 있을까.
물론 가능성이 낮다고 하지만 정부에서 의도적으로 인플레이션을 조장하고 금리를 낮춘다면?
컵라면 1개를 1000원에 사먹다가 나중에 5000원을 주고 사먹는다면 10억의 가치는 2억으로 떨어져버린다.
\vspace{5mm}

이런 게 2000년부터 현재까지 지속되어온 것이다.
월 천을 버네 2천을 버네라는 특정 전문인들조차도 직접 물어보면 자기들은 가난하다고(!) 한다.
이게 뭔 배부른 투정이여... 라고 하겠지만 사실 완전히 틀린 것도 아니다.
그만큼 현물가격도 미친 듯이 높아졌기 때문이다.
특히 저 정도 고소득자들 입장에선 나름 체면(?)이란 것을 유지하기 위해 들어가고 싶은 동네, 그리고 거기 생활비 등은
그들의 소득으로는 감내하기 어려운 경우도 많다.
\vspace{5mm}

그런데 여기까지는 아주 고전적인 이야기다. 현대적인 이슈는 바로 인구절벽이다.
\vspace{5mm}

이 수치의 변화에서 파생되는 결론들을 논하면 끝도 없겠지만
\vspace{5mm}
\begin{itemize}
    \item 첫째, 몇몇 시군구는 영원히 사라질 수도 있다. 임계선을 넘어 인구가 줄어들어 제 기능을 못 하면 행정 서비스도 줄이고 통폐합당할 수 있다.
    \item 둘째, 공무원, 교사 인원도 대폭 감축되는 건 필연적이기도 하지만 그 전에 보험, 연금의 근간도 흔들리고 만다.
    \item 셋째, 개인의 몸값은 높아질 수 있지만 대신 기업은 인건비 부담이 가중되어버린다. 분명한 건 소비시장은 줄어든다는 것이다.
    \item 넷째, 노인들도 일해야 한다.
\end{itemize}
\vspace{5mm}

....
\vspace{5mm}

등이 있지만 가장 중요한 건 원화든 부동산이든 그 가치는 형편없이 떨어질 수 있다.
돈 자체는 값어치가 없다. 돈의 가치는 그 화폐시스템에 지배당하는 노동력으로부터 나오는 것이다.
마찬가지로 토지나 건물도 사람이 이용하지 않으면 아무 쓸모가 없다. 그런데 인구 자체가 줄어들면 어떤 쓸모가 있는가.
가격은 곧 교환, 사용가치를 계량한 것이고, 교환과 사용은 그 행위주체인 사람들이 있어야 의미가 있다.
그래서 관료들은 다문화라는 이슈 하에 이민을 장려하려 했으나.... 최근 난민사태에서 보다시피 이건 대안이 못 된다는 게 뚜렷해지고 있다.
\vspace{5mm}

그럼 앞에서 말하는 의치한 환상과 결부지어보자.
대학의 입결이라는 건 그 대학의 특정 전공을 함으로써 벌 수 있는 평생소득을 반영한 것이다.
이 평생소득이라함은 앞으로 사회가 ~ 할 것이니 ~ 한 직업이 ~ 하게 벌 수 있다라는 모형에 기초한다.
그런데 만약 이 모형이 허구라는 게 명백해지면 입결은 떨어지게 되는 것이다. 그럼 그 기대소득은 달리 산정해야하기 때문이다.
물론 입결이 미래의 모든 변동까지 감안할 것이라는 주장도 가능하지만, 정말로 그게 적중을 했냐하면 또 그것도 아니다.
정말로 중요한 사회적 변동은 '예측되지 못 하는' 데에서 나온다.
2000년대에 의치한에 가던 사람들이 외치는 건 고령화이기만 했다, 저런 인구절벽까지 고려한 것은 아니었다.
\vspace{5mm}

게다가 더 중요한 사실, 과연 인구절벽만이 유일한 리스크인가?
\vspace{5mm}







\section{[미래공학 005] 기계와의 결합 시대}
\href{https://www.kockoc.com/Apoc/617674}{2016.01.30}

\vspace{5mm}

\href{http://news.naver.com/main/read.nhn?mode=LSD&mid=sec&sid1=101&oid=023&aid=0003133206}{뉴스 링크}
\vspace{5mm}

기계가 못 하는 활동에는 한 가지 공통점이 있습니다. 바로 아이디어 떠올리기(ideation), 즉 훌륭하고 새로운 아이디어나 개념을 떠올리는 행동입니다.
단어 같은 기존 요소들의 새로운 조합을 만들어내도록 컴퓨터 프로그램을 짜기는 아주 쉽습니다. 어떤 상황에서 시나리오별로 확률을 계산하는 것도 어렵지 않죠. 하지만 그런 조합이 어떤 의미를 지니는지를 판단하는 것은 여전히 사람 몫입니다.많은 사람이 저에게, 앞으로 기술이 발전해도 가치를 잃지 않는 인간의 기능과 능력은 무엇인지 물어봅니다. 그리고 대부분 기계가 할 수 없는 영역, 인간만이 할 수 있는 일이 무엇일지 고민하고 찾으려 하죠. 하지만 저는 굳이 로봇과 경쟁해야 한다는 편견을 버리라고 조언하고 싶습니다. 오히려 인간만이 가진 창의성은 기계와 만났을 때 더 빛날 수 있다고 생각합니다. 앞으로 세계는 기술을 제대로 활용할 줄 알고, 이를 통해 참신한 전략을 짤 수 있는 인재들이 지배할 것입니다. 미래학자 케빈 켈리는 이렇게 말했어요. '앞으로 로봇과 얼마나 잘 협력하느냐에 따라 연봉이 달라질 것'이라고."
\vspace{5mm}

이런 것을 '모라벡의 역설'이라고 부릅니다. 인공지능·로봇공학 연구자에 따르면 고등 추론에는 연산 능력이 거의 필요 없는 반면, 낮은 수준의 감각 운동 기능은 엄청난 연산 자원이 필요합니다. 35년간 인공지능 연구가 주는 중요한 교훈은 '어려운 문제는 쉽고, 쉬운 문제는 어렵다'는 것이죠. 즉 앞으로 인공지능 로봇이 진화하면서, 애널리스트, 가석방위원회 위원, 공학자, 회계사, 의사, 운전자 등 \textbf{관리직 혹은 전문 기술이 필요한 직업은 기계로 대체될 가능성이 큽니다}. 그들이 하는 일을 컴퓨터 소프트웨어로 만드는 것은 현재 기술로 어렵지 않기 때문이죠. 하지만 배관공, 정원사, 안내원, 요리사, 가정부, 간호사는 앞으로도 수십 년은 직장을 지킬 것으로 보입니다."
\vspace{5mm}

저 이야기를 단순히 직업이 사라진다라고 받아들이면 곤란하다.
단지 특정 직능만 가지고 먹고살기 어려다고 보면 된다.
모라벡의 역설은 그래도 주목할 만하다.
실제로 대체되거나 사라지고 있는 건 '고등 수준의 지적 작업'이다.
사실 이건 좀 어이없는 자충수 때이기도 한데.
\vspace{5mm}

고등 추론은 그 자체가 매우 깔끔하고 잘 정리되어있어서 단순화할 수가 있다.
단순화할 수 있기 때문에 컴퓨터로 대체할 수 있는 것이다.
반면 낮은 수준의 단순노동은 실제로는 단순하지 않으며 이런 건 기계로 대체하는 건 비싸게 먹힌다.
\vspace{5mm}

그렇기 때문에 가령 의학이 발달한다 하면 이건 오히려 기계로 대체될 수 있는 쪽에 가깝다고 보아야 한다.
발달한다는 건 더 상위 차원에서 '간단히 정리'될 수 있다는 이야기가 되기 때문이다.
예컨대 약사란 직업이 필요없고(?) 자동판매기로 충분하단 것도
제약회사서 워낙 약을 잘 만들기 때문에, 즉 이 분야 매뉴얼이 너무 잘 정리되어있기 때문에 역설적으로 벌어지는 일이다.
\vspace{5mm}

한 직업이 송두리째 사라진다거나 몰락하는 일은 생각 안 해도 된다. 그러나 \textbf{헤게모니는 분명히 바뀔 것이다.}
\vspace{5mm}






\section{[미래공학 006] xx에 가라고 하는 어른들 얘기를 믿어야할까}
\href{https://www.kockoc.com/Apoc/623414}{2016.02.04}

\vspace{5mm}

195, 60년대 젊은이들은 쓸데없는 생각 말고 농사지으라는 이야기나 들었을 것이다.
197, 80년대 젊은이들은 기술 배워서 뭐하냐 대기업 사무직이나 가라, 공무원은 연봉이 짜잖아.
그리고 1990년대에는 게임과 만화해서 뭐하냐 천박하게 살 건데라는 소리를 들었을 것이다.
\vspace{5mm}

당연히 1950, 60년대에 농사나 지었다간 농산물 개방에 막걸리나 마셨을지 모른다.
1970,80년대 젊은이 - 지금 장년층, 노년층들은 기술을 갖춘 사람이나 공무원 연금받는 분들이 그나마 낫다.
그리고 1990년대에 게임과 만화를 즐기는 걸 넘어 이걸 공부하고 생산하려 한 사람들이 지금 부가가치를 창출하고 있다.
\vspace{5mm}

어른들 이야기가 다 맞는 것은 아니다. 가려들어야할 것은 가려들어야 한다.
특히 주의할 건 지금의 40대 이상은 유례없는 고성장에 '중독된' 분들이다.
1년 지나면 빚내서 구입한 아파트 가격이 올라있고 인서울 대학만 졸업하면 대기업 취업이 가능했고
남녀가 사랑만 하면 잘 살 수 있다는, 이제는 환상이나 다를 바 없는 패턴을 학습한 분들이다.
유례없는 저성장을 이해하는 사람은 현 60대 언저리이다. 물론 이 60대들도 박정희 시절의 고성장에 뽕맞았긴 마찬가지이다.
\vspace{5mm}

지금 10대들이나 20대들은 돈과 취업이 보장되는 과에만 가려고 하는데
문제는 정말 자기들이 졸업할 때에는 그걸 '누릴 수 있느냐'이다.
지금 우리나라 어른들이 강조해대는 그런 분야는 미래 전망서에서 마이너스로 치부하는 경향이 있다.
사실 게다가 어떤 분야건 영원히 잘 나갈 수는 없다(하지만 물론 자기들은 예외라고 생각할 것이다)
\vspace{5mm}

과연 입시 한방으로 생계가 보장되지는 않으며
오히려 남들이 가지 않는 길(물론 미래에 수요가 있다)로 가는 게 승률이 좋았다고 보는 게 맞다.
현재의 의치한 대접은 IMF가 터지기 직전 대기업 급인 것이다.
여기서 주의할 건 그 분야가 정말 절대적으로 좋은 분야인지 아니면 상대적으로 나은 분야인지 따지는 것이고
'기업'에게 좋은지 '개인'에게 좋은지도 따져보아야한다는 것이다.
\vspace{5mm}

실제로 절대적으로 좋은 분야는 없다, 다 상대적으로 나은 것 뿐인데
상대적인 것은 가변적이라서 조금씩 바뀌기 시작한다.
\vspace{5mm}

하지만 가장 중요한 건 기업에게 좋은지 개인에게 좋은지인데
가령 공대의 경우는 기업 CEO 입장에서야 물고빨고할 게 있지만 개인 입장에서는 별로라고 볼 수 있고
정반대로 의대의 경우는 기업 입장에서는 시큰둥하지만 개인 입장에서는 좋다고 할 수 있는 것이다.





\section{[미래공학 007] 젊은이에게 빚을 지우는 사회}
\href{https://www.kockoc.com/Apoc/623428}{2016.02.04}

\vspace{5mm}

긴 말 필요없고 사실상 '미국식 개혁'을 한다고 해서 저딴 식으로 만든 것인데
현 20대들에게 빚을 지운 세대는 정작 자기들이 빚 요구를 받은 적이 없다.
이제 586이라고 할 수 있는 그들은 이미 기득권이 되었고 젊은이들이 부동산 폭등의 노예가 되어주길 바란다.
세대간의 갈등은 2000년대에 조심스럽게 제기되었고 이제 가시화되어버린 것이다.
\vspace{5mm}

단적으로 말하면 지금 10대나 20대들은 - 어떻게 보면 부모세대들이기도 한 저 세대들의 말을 안 듣는 게 생존하는 지름길이다.
소위 지금 꼰대들은 조금만 노력해도 성과가 잘 나오는 '고성장' 시대에 살았고,
현재 20대들은 스펙 경쟁은 기본이고 '잘 태어나지 그랬냐'라는 소리까지 듣고 살아야 한다.
\vspace{5mm}

자기 세대의 채무를 후세에 전가하려는 게 한계에 부딪친 게 현재다.
\vspace{5mm}

사실 이런 문제에 대해서 독재나 친일을 까는 건 좀 의아스럽다고 생각하는 것도 그런데
현재 20대들이 부딪치는 버거운 문제들은 저 독재나 친일 쪽 세력과는 좀 거리가 있기 때문이다.
(오히려 독재나 친일이라고 욕먹는 쪽들은 최소한 '공부하는' 학생들은 우대해주려고 했던 쪽이다)
지금 헬조선이라고 하는 것들은 오히려 그 독재나 친일을 욕하는 사람들이 칼자루를 쥐고 있을 때 시작했던 걸로 기억한다.
그 때 온갖 룰을 어겨가면서 흥청망청 잔치를 벌였고 이제 돈갚으세요라는 지불요구서가 애꿎은 20대들에게 전달되는 것이다.
\vspace{5mm}

저출산을 고민하는 건 좋다. 그런데 그러려면 기성세대들이 흥청망청한 것을 후세에 전가시키진 말아야하지 않나.
요새 젊은이들은 근성이 부족하다하는 건 얼핏보면 맞는 이야기인데 항상 그렇지만 '메신저 분석없는 메시지'는 무의미하다.
젊은이들의 근성을 탓하신다는 분들이 하나 같이 \textbf{내 집값 언제 오르냐 타령하는 것 자체가 이중적이지 않나?}
게다가 원칙적으로 말하면 근로자의 권리 뭐고 이전에 철밥통 영위하시는 분들만 나가주셔도
청년들의 숨통이 틀 수 있다. 그런데 이런 건 또 신자유주의 반동이니 뭐니 슬그머니 막으시더라.
사실 북유럽식 복지니 그런 건 쓸데없는 논쟁이다. 자유롭게 경쟁하기만 해줘도 된다.
\vspace{5mm}

하지만 가장 심각한 건 저런 문제에 대한 해결책이라는 것도 저 위선적인 인간들에게서 나오고 있다는 것이다.
\textbf{사회를 개혁해야 해, 하지만 내 부동산은 건드리지 마, 교육개혁은 해야 혀, 하지만 내 아이는 '면접'만으로 좋은 데 가야해}
이게 그들의 진심 아닌가?
\vspace{5mm}

더 심각한 건 이런 문제의 본질을 슬그머니 "그러니까 xx당을 지지해야 한다"라는 괴상한 정치적 문제로 환치하는 것이다.
말하지만 그럴 시간이 있으면 차라리 청년들에게 경제, 금융, 법률 공부를 더 시키라고 하면 된다.
\vspace{5mm}





