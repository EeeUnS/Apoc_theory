


\section{[학습공학 001] 공부는 하는 게 아니라 하게 되는 것이다.}

\href{https://www.kockoc.com/Apoc/339548}{2015.09.23}

\vspace{5mm}

샐러드에 소스를 넣는 건 그건 야채 자체는 더럽게 맛이 없기 때문이죠.
몸에 안 좋은 것을 알면서도 라면을 먹는 것은 중독성 강한 수프 때문이죠.
\vspace{5mm}

인간의 행동은 합리성과는 거리가 멉니다.
케인즈의 그 유명한 명언을 바꿔하면 우리 모두는 장기적으로 '중독자'입니다.
\vspace{5mm}

공부를 잘 한다라는 말 자체는 상당한 오해를 불러일으키죠.
성적이 높은 사람들은 공부를 잘 하는 것이 아니라, 공부를 잘 \textbf{'하게 되어버린'} 것입니다.
하는 것과 하게되는 것의 차이는 친절하게 대화로 해결하는 대인배와 사소한 것도 고소미 먹이는 소인배만큼이나 크지요.
\vspace{5mm}

공부를 하는 사람은 마치 드래곤볼의 베지터처럼 심각한 인상을 쓰고 나 공부할 거야라고 합니다.
그러나 의지라는 건 길어보았자 하루 가나요. 그리고 의지만으로는 절대 스트레스와 고통을 이겨낼 수 없습니다.
그래서 머지않아 딴짓을 하게 되죠. 그리고 다시 "나 공부할 거야"라고 재도전, 또 딴짓, 재도전, 딴재. ㄸㅈ ∞
찔리는 분들의 숫자만 상용로그로 취하면 지표가 최소 3 이상은 나오려나요.
\vspace{5mm}

실제 우등생들은 공부를 한다는 의식이 별로 없습니다. 다만 그들은 숨쉬듯이 공부를 하게되는 것이지요.
그리고 공부를 하게 되는 비결은 "환경"입니다.
\textbf{지적 자극을 매일 받는 동시에 지적인 작업에 몰두할 수 있는 \textbf{환경}에 노출됨으로써 마침내 공부중독에 이르게 되어버린 것}이죠.
\vspace{5mm}

이런 이론이 설명하는 건 왜 여러분들이 인강을 들으면 공부가 잘 되느냐일 것입니다.
그건 간단합니다. 인강은 '조리료'에다가 '진통제'까지 갖추고 있어서입니다.
입담 좋은 강사의 썰을 듣고 있는 것이 혼자 책읽는 것보다 스트레스가 덜 하고
하라는대로 필기하다보면 최소한의 공부라는 것을 하게 됩니다.
그래서 수십강 정도 듣다보면 혼자 공부할 거야와 달리 일정량의 학습량이 완성됨으로써 마이크로한 중독상태에 빠지는 것입죠.
즉, 인강은 돈주고 구입하는 가상의 '공부환경'인 것입니다.
\vspace{5mm}

현재 이 시점에서 성과가 미미하다는 분들은 절대 자신의 의지 따위는 믿지 않는 게 좋습니다.
저도 제 의지 따위는 그리 믿지 않습니다.
믿어야 하는 건 오직 자기가 속한 환경입니다.
님들이 공부하는 공간적, 시간적, 인적 환경이 정말 공부하지 않을 수 없게 되어있는가
단 하나라도 예외가 없는가 하는 걸 마음에 안 드는 예비 며느리의 국을 마시는 시어머니처럼 점검해야합니다.
\vspace{5mm}

머리? 유전?
동아시아의 빈민국이나 아프리카 남미의 10대들은 그럼 머리가 다들 나빠서 그런 건가요.
인강? 교재?
분명 이건 공부환경의 일부입니다. 그러나 일타강사들을 소비한 그 많은 학생이 전부 다 성공했습니까?
\vspace{5mm}

이 시점 되어서 수능 걍 포기하자 혹은 올해는 올림픽 정신으로 치고 내년에 재응시하자 별별 분들이 다 계실 겁니다.
그리고 제 무성의한 답변을 짐작하면서도 쪽지보내는 분들이 계시겠지만 그 전에 제가 말씀드릴 건 그겁니다.
먼저 환경부터 점검하십시오, 본인이 술이나 연애에 쩔더라도 공부할 수 밖에 없는 그런 환경을 만드시길 바랍니다.
예를 들어서
\vspace{5mm}

\begin{itemize}
    \item 책상에 앉으면 맛폰 컴퓨터는 일체 차단된다
    \item 어떤 사람도 방해를 하지 않는다, 그리고 공부 빼고는 할 수 있는 게 없다
    \item 필기구와 연습장, 그리고 지우개는 절대 학습에 지장을 주긴 커녕, 그걸 쓰는 쾌감이 크다.
    \item 공부하면서 쌓이는 스트레스를 일주일에 한번 풀 수 있다.
    \item 모르는 문제나 내용을 언제든지 물어보거나 요청할 수 있다.
    \item 하루 순공부 6시간이 확보된다.
\end{itemize}
\vspace{5mm}

적으려면 끝이 없겠죠. 아무튼 환경이 가장 중요합니다.
하지만 대부분의 학생들은 환경을 경시하죠. 그래놓고 망해놓고 원인을 엉뚱한 데 찾는 게 현실입니다.
환경을 갖추려고 고심하는 데는 일주일 정도만 걸립니다. 아무리 길어도 한달?
돈의 문제가 걸린다고 하지만 그것도 대부분은 익스큐즈 가능하죠.
\vspace{5mm}

다시 강조하지만 공부는 하는 것이 아니라, 내가 싫어도 '하게되는 것'이고
내가 싫어도 하려면, 달리기 싫어도 뛸 수 밖에 없는 런닝머신과 같은 환경을 만들면 됩니다.
환경을 만드는 건 "공부하기 싫어"라는 핑계가 안 먹히겠죠.
\vspace{5mm}






\section{[학습공학 002] 한번이 아니라 여러번 해야한다.}
\href{https://www.kockoc.com/Apoc/345565}{2015.09.25}

\vspace{5mm}

그렇다면 학습환경의 기본 조건은 무엇일까.
\vspace{5mm}

우선 여기서 우리는 편견을 깨야한다.
\vspace{5mm}

친부모는 아버지 어머니 각각 한분일 수 밖에 없고
가능하면 배우자도 한명이면 좋겠고
무엇보다 우리의 인생도 알려진 것으로는 한번이다.
\vspace{5mm}

하지만 수험은
어떤 과목이든 한번에 끝낼 수도 없고 아니 끝내서도 곤란하다.
한권으로 정리하거나 완성한다는 건 거짓말이고 그렇게 해서 성공한 케이스도 없다.
\vspace{5mm}

다수 수험생들이 실패하는 가장 큰 이유는 노오력이 부족해서가 아니라 잘못된 수험상식을 갖고 있어서인데
그 중에 대표적인 것은 바로 \textbf{"한번에 모든 걸 끝내겠다"}이다.
\vspace{5mm}

학습의 원천은 반복이다. 토하고 싶어질 때까지 반복해야 지식과 기술은 우리 것이 된다.
그럼 인간은 왜 반복을 하지 않으면 학습이 되지 않습니까라고 질문할 수도 있다.
그 이유는 매우 심플하면서도 철학적이다.
간단하다. 반복을 하지 않고 단 한번만 접촉한 것 감각대로 학습되어버리면 우리는 자아를 잃어버릴 것이기 때문이다.
인간이 인간다운 이유는 어느 환경에서든 자기다움을 유지하는 것이고, 자기다움을 유지하는 것은 함부로 학습하지 않는 데 있다.
함부로 학습해버리면 어떤 자극에도 조정당하는 좀비로 전락한다.
나답게 살아갈 수 있는 건 함부로 학습하지 않기 때문이다.
\vspace{5mm}

반복학습의 방어막이 없으면 어떻게 될까. 술, 섹스, 담배, 마약에 쩔어버린 케이스로 충분히 설명된다.
그것들은 반복과 노력을 요구하지 않는다. 단 한번의 접촉만으로 우리는 이성을 잃고 끌려다녀버리게 된다.
생각해보면 그것들도 그리 나쁜(?) 것은 아니다. 문제는 우리가 자제력을 발휘하기 어려워서 나다움을 잃어버린다는 것이다.
\vspace{5mm}

요컨대 일부 영역을 제외하고는 '반복' 없이는 함부로 학습하지 않기 때문에 우리는 인간답게 살 수 있는 것이다.
\vspace{5mm}

문제를 푼다면 적어도 3$\sim$4번은 다시 반복해서 풀어보아야 한다.
개념서를 읽더라도 한번 읽고 끝낸다는 생각은 버려야 한다. 반복을 꾸준히 해야한다.
하수들은 그 반복이 매우 지루하고 무의미하다고 착각한다.
고수들은 반복이 매번 똑같지 않으며 한면 할수록 즐거워진다는 것을 잘 알고 있다.
\vspace{5mm}

가령 500문제를 1번 푼 것과, 100 문제를 5번 푼 것은 얼핏 초딩 산수로 보면 똑같아 보일지 모른다.
하지만 실제 질적인 것은 달라진다.
100문제를 5번 풀게 되면 처음 볼 때에 몰랐던 행간의 내용까지 파악하게 되고
반복하기 때문에 뇌에서도 어쩔 수 없이 이것들을 학습하여 자아를 바꾸게 되며
틀린 문제들을 다시 보고 각성함으로써 오답률이 현저히 줄어들게 된다.
하지만 500문제를 1번 풀고 던지면?
뇌에서는 그것들을 중요시 하지 않게 여기고 잊어버릴 가능성이 높다.
여기서 학생들은 머리가 나빠서 그것들을 까먹린다라고 착각하는데 그건 거짓말이다.
이건 뇌가 일부로 잊어버리는 것이다. 중요하지 않다고 판단하는 동시에 스트레스를 받기 때문이다.
500문제를 1번만 푼다면 스트레스도 상당할 것이고 괴로운 마음도 들 것이다. 그런 정보를 뇌에서 바로 학습하겠는가.
하지만 100문제를 5번 푼다면 정답률은 현저히 올라갈 것이고, 아는 것을 더 자세히 알게되니 순탄해지거니와
기존에 안 보이던 것도 보이게 되어 즐거워진다. 이럼 뇌에서도 경계심을 풀고 그 내용을 학습해버린다.
\vspace{5mm}

그렇기 때문에 가장 어리석은 수험생들은 이렇게 예시할 수 있을 것이다.
\vspace{5mm}

- 한권으로 모든 걸 끝낼 수 있다고 믿거나 그렇게 자처하는 교재를 보는 경우
- 시험이 다가오는데 기존에 풀었던 것을 오답정리하지 않고 저 실모 좋다면서 교재 늘리거나 다른 인강만 듣는 경우
\vspace{5mm}

학습의 핵심엔진인 반복, 그리고 3편에 언급할 그 반복에 의한 복리효과를 누리지 못 하는 이들이 성적이 올라갈 수 있을까.
미소짓는 건 오직 장사치일 뿐이다.
\vspace{5mm}






\section{[학습공학 003] 잡몹 죽여서 렙업하기}
\href{https://www.kockoc.com/Apoc/366443}{2015.09.29}

\vspace{5mm}

경험치 10,000을 달성해야 한다.
\vspace{5mm}

한마리 죽일 때마다 100 주는 잡몹을 100마리를 죽일 것인가
한마리 죽일 때마다 5,000 주는 보스몹 2마리에 도전할 것인가.
\vspace{5mm}

속칭 버스 타지 않고(즉, 잘 하는 파티에 섞여서 경험치 나눠먹기 : 물론 이것이 소규모 가애 혹은 학원행이겠지만)
혼자서 사냥해야하는 경우라면 무얼 선택할 것인가 고민할 필요도 없다.
\vspace{5mm}

100 주는 잡몸은 죽일 때마다 그 경험치가 내 것이 되고, 그 와중에 렙업도 이뤄져버리고, 잡는 패턴도 숙달되어서 속도가 빨라진다.
하지만 5000 주는 보스몹은 내가 잡히기 좋은 데다가 이걸 잡아야지 하면서 시행착오로 시간과 돈을 날리고 나중에는 자존심 문제가 걸린다.
\vspace{5mm}

앞의 칼럼에서 공부에 관한 진실 두가지를 언급했다.
\begin{itemize}
    \item 첫째, 공부는 하는 게 아니라, 할 수 밖에 없는 환경을 만들어야 한다는 것. 우리는 공부하기 싫어한다.
    \item 둘째, 우리가 인간답게 사는 비결이 바로 함부로 학습하지 않는 데 있다는 것. '학습'해야하는 걸 납득시키기 위해서 반복횟수를 늘리라는 것.
\end{itemize}
\vspace{5mm}

\textbf{세번째 진실은 공부 못 하는 애들의 특징은 "과식"을 하려고 한다는 것이다.}
일단 공부를 못 하는 녀석들은 무조건 거창한 계획을 잡는다.   1년동안 열심히 해서 서울대에 갈거라는 둥, 일주일 내에 문제집 10권을 다 끝낸다는 둥.   하지만 계획은 바로 실천가능한 게 아니면 '몽상'일 뿐이라는 것을 그들은 모른다.   애당초 실천에 도움이 되기 위해서 계획을 짜는 건데, 못 하는 사람들일수록 현실적으로 불가능한 목표를 세우고 막상 실천할 때 포기해버린다.   그리고 그런 포기를 '학습'해버림으로써 구더기 수준으로 낙하해버리는 것이다.   잘 하는 친구들은 절대 과식은 안 한다. 정해진 \textbf{단위시간 내}에 100문제와 10문제를 선택하라고 하면 당연히 10문제를 택한다.   첫째로, 100문제는 다 풀지 못 해서 많은 애로사항이 꽃피겠지만, 10문제는 어찌되었든 풀 수 있기 때문이다.   둘째로, 다 풀지 못 한 100문제는 데미지는 심하게 입혔지만 킬을 못 해서 경험치에 가산되지 않지만   다 풀고 오답정리한 10문제는 비록 양이 작을지라도 바로 나의 경험치에 가산되기 때문이다.   유식하게 말하면 확실히 내 것으로 만든 소량의 학습량은 "재투자 가능한 자본"이 된다.   그러나 다 끝내지 못한 미완의 학습량은 "재고"가 되어버려서 처치곤란해져버린다. 경험치 가산이 되지 않거니와 좌절감만 안겨준다.   그리고 이것이 일주일, 한달 그렇게 넘어가버리면 엄청난 차이를 발생시킨다.   만약 700문제를 푼다고 하자. 기한은 일주일   A는 하루에 700문제를 풀고  6일간 놀겠다고 하고, B는 하루에 140문제를 푼다고 하면 어떤 일이 벌어지냐면   A는 이틀간 한 게 400문제. 그리고 바로 지쳐버려서 3일 째에 놀다가 다시 벼락치기 모드로 가서 악순환이 벌어진다.   B는 140 140 갑자기 사건 사고 140 140 휴식 140. 이런 식으로 해서 어찌되었든 끝낼 수는 있다.   거기다가 A와 B는 중대한 차이가 벌어지는데   A의 경우는 처음에 허겁지겁 풀다가 오답정리도 못 해서 푼 문제가 자기 것이 되지 못 하지만   B는 목요일부터는 앞에 풀었던 280문제가 자기 것이 되기 때문에 학습 수준이 높아셔 문풀 스피드나 정확도가 높아진 상태라는 것이다.   바로 이것이 학습복리의 마법이다.      보통 수학에서는 $an=a+a+a+...+a+a$(n번 더하기)로 나오므로 사람들은 실제로 학습도 저럴 거라고 착각을 한다.   그러나 실제 학습에서는 $an<a+a+..+a$ 이다. an을 한번에 하지 않고 n 회로 쪼개면 $a(1-(r)^n)/1-r $과 같은 등비수열의 합형태로 나타나는 것이다.   그렇기 때문에 한번에 많이 하려는 것보다는, 장기간에 걸쳐 매일 적정량 꾸준히 학습하는 것이 훨씬 많은 보답을 해준다.   고수들이 느긋한 건 폼잡는 것이 아니다. 더 많은 것을 얻으려면 잘게 쪼개서 학습의 복리효과를 누려야한다는 것을 알고 있는 것이다.   그에 비해 하수들은 마음이 급하기 때문에 처음부터 터무니없는 계획을 세우기만 하고 사실 공부를 하지 않는다.   정말 자신없는 과목에 접근하는 방법은 별개 아니다.   초반에는 공부를 한다는 생각은 버린다. 탐색한다고 생각하고 책을 끝까지 주욱 훑는다. 지루한 건 당연하고 이해 안 가야 정상이다.   하지만 이렇게 함으로써 뇌에서는 이거에 대한 두려움을 줄일 수는 있다. 적어도 지루하고 재미없다는 걸 알았으니 무서운 건 없는 것이다.   그 다음에는 매우 쉬운 기출문제집을 가져온다. 그리고 하루에 12개씩으로 못 박는다(그 이상 가면 스스로를 꾸짖는다)   가장 쉬운 기출문제를 매일 12개씩 보고, 그 해설을 읽는다. 그 다음 그 기본서에서 해당 내용을 찾아 읽어본다.   이렇게 해서 일주일이 가면 84개는 풀게... 가 아니라 사실 100문제를 풀게 된다.    인간의 숫자 심리는 반올림 욕망이 있어서 조금만 더 하면 100 달성 가능하잖아 라고 하면서 더 공부하게 되어서이다.   그럼 일주일동안 당사자는 무려 100문제를 풀었고, 기본서와 대조해보았고, 그래서 어떤 식으로 출제되는지 어떤 게 우선순위인지 알 수 있다.   그런 식으로 하면서 25일간 조심스럽게 진도를 나가면 그 다음부터는 문풀량을 늘릴 수 있고, 기본서가 훨씬 더 잘 읽히게 되는 것이다.      학습도 확실히 '소식'을 해야한다. 물론 반드시 '소화시킬 수 있는' 것으로 소식하는 것.   절대 욕심은 부리지 말고, 바로 내가 착수하고 완료할 수 있는 수준으로만 학습량을 잡고 매일 실천해나가면 된다.   대략 50일이 지나면 가장 못하는 과목도 3$\sim$4등급에 비빌 수 있다.





\section{[학습공학 004] 자존심 환자들}
\href{https://www.kockoc.com/Apoc/391905}{2015.10.05}

\vspace{5mm}

보통 30일을 앞두고 낙담하는 이유는 두가지로 구분할 수 있다.
\vspace{5mm}

\begin{itemize}
    \item 첫째는 공부가 절대적으로 부족한 케이스이다.
    이런 경우 해결책은 "올림픽 정신으로 치른다"라고 다짐하고 그 다음 해를 기약할 것,
    그리고 절대 11월부터 2월까지 놀리지 말아야한다. 오히려 이 기간동안은 하루 9시간 공부를 해야한다.
    \vspace{5mm}
    
    \item 둘째는 공부가 상대적으로 부족하다고 느끼는 케이스이다.
    이건 마치 온도가 높아지면 포화수증기압이 높아져 절대습도가 낮아지는 것과 비슷하다고 할 수 있다.
    공부하면 할수록 실력이 늘어나는 동시에 '기대치'가 늘어나면서 자기 공부량이 형편없이 보이는 것이다.
\end{itemize}
\vspace{5mm}

위 두가지는 양립가능하다. 그러나 사실 위 두가지는 알아도 그만이고 몰라도 그만이다.
어찌되었든 수험은 학습량이 많아질수록 확률적으로 유리해지는 도박이니 결과 신경쓰지 말고 그냥 공부하는 게 답인 것이다.
그러나 왜 불안할 수 밖에 없냐면 이건 간단하다.
이맘 때쯤 \textbf{어떤 문제를 풀어도 다 만점이 나오고 수험사이트에서 자랑질할 걸 계획}했기 때문이다.
\vspace{5mm}

자존심은 윤리적인 분야를 제외하곤 쓸모가 없는 것이다.
내가 바른 일을 했고 선행을 했다는 데에는 자존심을 얼마든지 가져도 좋다.
하지만 돈을 많이 번다든가 외모가 뛰어나다거나 하는 것에서 가지는 건 무의미하다. 그런 건 언젠가는 사라진다.
\vspace{5mm}

그런데 "수험생활" 자체에 자존심이나 자부심을 가지는 사람들이 있다.
이들은 게임의 승패보다는, 어떤 마우스나 키보드를 쓸 것인지 신경쓰는 부류이기 때문이다.
실제로 이들의 문제는 "목표"라는 게 없다는 것이다.
아니 서울대에 가는 게 목표가 아니냐 하겠지만, 틀린 이야기이다.
"서울대에 갈 거야"라는 것과 "내 수준에 맞춰 현실적으로 서울대에 가기 위한 점수를 따는 것"은 분명히 다르다.
\vspace{5mm}

다시 말해 똑같아 보이는 목표라도 추상적이고 허황된 목표와, 구체적이고 현실적인 목표는 설탕과 소금만큼이나 다른 것이다.
\vspace{5mm}

전자의 문제는 본인이 스스로 무중력 상태에 있다는 착각에 빠져있다는 것이다.
그래서 수험을 할 때에도 온갖 수험사이트를 돌아다니면서 가장 최고인 커리를 골라 그걸로 열심히 공부하는 척 하려고 한다.
이 사람들은 시중에 유명하다는 교재들은 다 갖고 있다, 게다가 교재들도 너무나도 깨끗하다.
그러다가 100일 정도 다가오면 어 이거 아니라 느끼다가 50일 정도 오고 나서야 계산해보고나서 도저히 올해는 불가능하다고 여긴다.
\vspace{5mm}

이런 일이 왜 벌어졌을까.
그들은 수험생으로서의 미학에 사로잡혔기 때문이다.
이 사람들이 실제로는 합격을 간절히 바라지 않는다.  \textbf{'아름다운 수험 '}으로서 남들에게 자랑질할 걸 생각하고 있다.
적어도 이 사람들 중에 말이 적은 사람이나 침묵을 지키는 사람은 별로 없다는 걸 지적해두고 싶다.
그리고 실제로는 합격할 생각도 없는 것이다. 본인은 인정 안 하지만 수험생 생활이 가장 편하다고 몸이 인식하고 있다.
정말로 배고프고 헐벗어서 당장 합격 못 하면 나가 죽을 거라면 연초부터 게으름을 피우면서 수험생 코스프레질은 안 했을 것이고
어떤 교재가 좋아요 그딴 질문은 당연히 하지 않았을 것이다.
정말로 수험에 매진하려면 '2년'간의 기간을 둬야한다. 그리고 정말 '죽기 직전'까지 해야 한다.
저승사자조차도 대학에 합격하기 어려운 시대다.
\vspace{5mm}

정말로 합격이 절실하다면 목숨이 걸렸다면 시간을 아꼈을 것이다
교재는 뭐가 좋아요 하지 않고 그냥 닥치는대로 다 미리 풀었을 것이다.
한마디로 수험에 미치면 어떻게 신이 내려오는지 그것도 보여줄 것이다.
\vspace{5mm}

다시 말해 진짜 합격을 향하는 수험생이라면 미학 따위는 없다. '추해져도 좋으니 합격만 한다'라는 마인드로 달려가는 것이다.
그리고 역설적이지만 그런 진흙탕과 땀내의 지저분함이야말로 역설적인 아름다움인 것이다.
이런 데 빠져든 사람들은 나중에 합격 불합격도 이제는 잊어버린다. 왜냐면 그 경지를 넘어서 대가를 이루기 시작하기 때문이다.
\vspace{5mm}

지금 이 시기는 사실 한참 미쳐있어야 한다. 물론 그걸 콕콕러든 누구든 강요하고 싶지는 않다.
그러나 이미 그렇게 공부하는 사람도 있고,  충분치 않지만 그걸 일부나마 맛본 사람들도 없지는 않다.
하지만 분명한 건 합격이냐 불합격이냐 따지는 사람들이 떨어질 가능성은 당연히 높지만,
별로 공부가 기대 안 되었지만 꾸준히 하면서 과목이 재밌다고 느끼는 걸 넘어 스스로 품질관리까지 가는 사람은 대박거둔단 것이다.
설령 출발점이 늦은 사람일지라도 한번 더 도전하면 승률은 매우 높아져있는 상태다.
\vspace{5mm}

앞으로 시간나면 여러개 적겠지만 수험생들의 문제는 교재나 인강이 아니라, 지나치게 잘못된 가치관에 사로잡혀있다는 것이다.
가령 교재 따지는 것을 보자. 나쁜 교재라고 배척하고 그럴 것인가?
물론 내 경우는 쓰레기 같은 교재들은 가능하면 안 팔리는 게 좋다고 본다. 그 저자들도 인성이 의심될 정도인 케이스도 있다.
하지만 수험생은 일단 교재 구입에 있어서는 돈을 아끼면 안 된다.
나쁜 교재라도 자기에게 도움이 된다면 활용할 때는 활용하는 것이다.
수험 초기에는 돈을 많이 지불해서라도 '정보'를 갖춰야 한다.
이게 수험에서 가장 절실한 자원인 "시간"을 아끼기 위한 것이라면 합리적인 선택이다.
\vspace{5mm}

그런데 수험생들은 정작 1만원, 2만원 이런 걸로 교재 어느 게 좋냐 타령하면서 수어시간을 헛되게 낭비한다.
그리고 교재를 사서 좋다 나쁘다 평가... 하는 것도 비웃을만한 게. 적어도 10회독은 해야지 뭔가 아는 건데
그 정도까지 하지 않고 마치 자기가 평론가인 것처럼 나선다. 그리고 이런 케이스는 다음 해에도 수험생 시즌 2를 시작한다.
소름끼치는 건 이 사람들은 타인들이 자기와 똑같은 행동을 하면 비웃는다는 것.
그리고 이것이 잘못된 거라고 지적하면 "내가 바보란 말이예요?" 반응을 보인다.
\vspace{5mm}

만약 내년을 대비하시는 분이라면 쓸데없는 자존심, 수험생의 미학, 그런 건 버려야 한다.
서재에 손도 안 댄 교재들(특히 실모)이 많다면 그건, "자존심 때문에 소비한 결과"라고 생각하시길 바란다.
정말 미친 듯이 공부하는 사람들이 자존심과 미학이 어딨나?
\vspace{5mm}

지금 불안한 사람들은 실제로는 시험 결과가 두려운 게 아니다.
이 시점 오고 나서야 뭔가 자기 실상이 보이는데, 그게 연초에 기대했던 자신의 수험생 미학과 모순되는 것이다.
그래서 자존심이 확 상해버린 것인데... 곰곰히 생각해보면 이건 매우 쓸데없는 망상이다.
\vspace{5mm}

시험날 결과는 아무도 모른다. 그냥 가서 실력대로 치르고 오면 된다.
요행 따위는 기대 안 하는 게 좋다. 요행은 내 의지와 관계없이 작용하는 것이기 때문이다.
실력대로 치르고 결과를 묵묵히 받아들인 다음, 그 결과대로 갈지, 다른 길로 갈지, 다시 시작할지 빨리 결정짓고 시간낭비하지 말아야 한다.
\vspace{5mm}

물론 이 글을 읽더라도 안 되는 사람들은 내년 3월까지 그 자존심 때문에 귀중한 시간을 날려먹는다는데 머리털을 건다.
\vspace{5mm}






\section{[학습공학 005] 행복}
\href{https://www.kockoc.com/Apoc/406900}{2015.10.11}

\vspace{5mm}

한국은 행복한 국가
그러므로 우리는 행복해질 수 없다.
그럼 무인도에 가면 됩니까?
아니, 한국인은 절대 혼자서 행복해질 수 없다.
\vspace{5mm}

이게 무슨 헛소리냐 하면서도 어차피 끝까지 읽을 거니까 반론하지 말고 들어보자.
행복은 부등식이다. 적어도 한국인의 행복은 평정심이라거나 아가페와는 거리가 멀다.
인삿말인 '안녕하십니까'부터가 살기힘들었던 현실을 반영, "밥은 먹고 다니냐"도 굶고 다녀야했던 선조들의 현실
그래서인지 모르지만 우리의 행복은 \textbf{"남보다"}라는 관용구가 없이는 성립될 수 없다.
\vspace{5mm}

수험도 그 점에서는 마찬가지이다.
재수부터 힘들어지는 이유는 그 때부터는 자기보다 잘 하는 애나 잘 나가는 케이스와 '비교'당하기 때문이다.
그리고 끊임없이 비교하면서 나는 불행해, 아니 난 행복해라는 명제를 도출해낸다.
현역에 서울대 합격하고 연애 잘 하고 유명해진 친구가 있으면 그 친구만 떠올려도 n수생이고 모쏠이며 은둔형인 자기가 원망스러워진다.
\vspace{5mm}

이런 심리를 실험해보기 위해 제가 직접 게임을 해보았습니다... 가 아니라
가끔 대전형 온라인 게임을 할 때 신기한 건, 딱 한판만 해보자라고 생각했는데 열판째하고 있다는 것이고
그게 왜 그런가 싶다면 이기면 이긴대로 즐거워하게 되고, 지면 지니까 열받아서 만회한다고 하게 되는 것이다.
이건 도박판에서 100만원을 잃고 벌충한다고 1000만원 꼴아박았다가 또 잃어서 다시 만회한다고 1억 퍼붓고하는 케이스와 비슷한데
이런 것의 '승패'는 사실 무의미한 것이다만, 사람이 게임에 몰두하게 되는 건 아무래도 '사냥'을 하던 본능 때문이 아닐까 싶다.
\vspace{5mm}

타인과 비교해야 행복해지고 수험생 신분이면서 열등감에 시달리는 것도 이와 비슷한 이유일 것이다.
그렇게 하지말라고 해도 하게 되는 것은 그것들의 미묘하게 우리들의 본능과 불륜관계여서인 것이다.
\vspace{5mm}

어찌되었든 이성적으로는 적어도 자기가 열등감을 느끼는 모델을 한명 정한 다음 그 모델을 따라가는 건 매우 괜찮은 전략이다.
그러나 감성적으로 열폭할 게 아니라, 그 모델의 '노하우'라는 걸 철저히 분석해야 한다.
하루에 몇시간 공부했느냐, 어떤 교재를 몇회독했느냐, 어디서 공부했느냐, 그리고 어떻게 스트레스를 풀었는가.
이런 걸 하나하나 다 분석한 다음 그대로 '흉내내는' 걸 권한다.
대신 그 사람보다 2등 아래라고 생각하자.
\vspace{5mm}

왜 2등 아래로 생각하냐는 건 간단하다. 그 사람을 흉내내면서 좋은 것만 받아먹으면서 뒤를 졸졸 쫓다보면
상대도 사람인지라 언젠가는 추락할 때가 오기 때문이다. 만약 내가 그 사람을 앞선다면 내가 먼저 추락했을 것이고
같은 등수이거나 1등 아래라고 한다면 동반추락할 위험이 높다. 그러나 2$\sim$3등 아래라면 나까지 추락하려면 시간이 남지 않았겠는가.
\vspace{5mm}

그리고 무엇보다 날린 시간은 생각하지 말아야 한다.
과거에 날린 건 후회한다고 해도 절대 1원도 받아낼 수가 없다. 게다가 후회하는 시간동안에 더 많은 것을 할 수 있다.
천천히 간다고 생각하고 잘 나가는 모델 3명 정도를 추린 뒤 이 사람들이 어떻게 공부하고 장단점이 어떤지 추리고 가면 된다.
\vspace{5mm}

그리고 이런 콩라인들이야말로 꿀을 잘 빤다.
사실 1등은 영예로울 것 같지만 시선도 부담스럽거니와 가장 먼저 힘든 일을 겪는데 대가는 보잘 것 없는 경우가 많다.
1등을 하면 더 이상 올라갈 수도 없고 내려가야하는 스트레스에 시달려야 한다.
그러나 2등은 올라갈 수도 있고 내려갈 수도 있다. 나름 상위권이면서도 주목을 덜 받지만 실속은 많이 챙길 수 있는 것이다.
\vspace{5mm}

그것도 그렇거니와 당사자들이 굳이 감정적으로 비교하라면 과거의 자신과 비교하셨으면 한다.
\textbf{과거보다 나아졌느냐 아니면 떨어졌느냐.}
이것이야말로 절대 기준이다.
작년 수능에 비해서 올해 수능을 쳤는데 떨어졌지만 1등급씩은 올랐다라고 하면 결코 헛되이 보낸 건 아니다.
과거보다 올라갔다는 건 시간이 걸릴 뿐 목표는 성취할 수 있단이야기이기 때문이다.
\vspace{5mm}

불안함도 긍정적인 신호다. 불안하지 않는 사람은 '공부하지 않고 시험을 포기한 사람' 빼고는 없다.
최상위권은 그럼 행복한가. 매일매일 자기가 떨어질까봐 노심초사한다. 당연히 안 그런 척 꾸밀 뿐인 것이다.
좋은 대학에 가려는 것은 가난하지만 행복한 삶을 위해서가 아니라, 불행하지만 가난하지 않은 삶을 살기 위해서이다.
위로 올라간다는 건 그만큼 더 많이 불안해지기 위해서, 그 불안함을 먹고 사는 직업인이 되기 위해서이다.
\vspace{5mm}

그럼 질문을 바꿔보자. 굳이 행복해질 필요가 있는가?
남과 비교하고 열폭하며 불안해하는 거야말로 살아있단 증거가 아니겠나.
욕심을 버리고 심신의 안정을 취하며 착하게 살아라.... 헛소리가 아닌가.
결국 구질구질하게 남들 호구짓이나 해주는 식물인간처럼 살라는 이야기가 아닌가.
욕심이 많아야 인생이 재밌는 거다. 욕심이 많으니 늘 불만이 쌓이고 불안할 수 밖에 없다.
그래서 앞으로 전진하려고 노력하니 심신이 안정되긴 하겠나.
그리고 조금이라도 내 것을 많이 불려야하고 남보다 앞서야하는데 착하게 살아야한다고?
\vspace{5mm}

시험을 앞둔 이상 불안한 건 당연하다. 그러나 아이러니한 것은
긍정적 신호인 불안을 부정적 신호로 받아들여 자포자기하는 일들이 벌어진다는 것이다.
다시 말해 공부를 열심히 했기 때문에 이제야 수험생다운 탐욕을 부리면서 생긴 불안감이
내가 공부를 안 해서 힘들구나라고 잘못 해석되는 해프닝이 벌어진다는 것이다.
\vspace{5mm}

불안해지지 않는 방법은 딱 하나 그냥 포기하면 된다.
\vspace{5mm}

"그럼 실력자가 된다는 건"
"매일 눈 감으면 파산, 몰락, 지구멸망 등의 상상은 하는 거지"
\vspace{5mm}

다시 말해서 모평 전부 만점받는 사람일지라도
실제 본 시험에서 어이없는 실수로 후두득 비가 내리는 것 정도까지 상상하고 감수해야
실력자라고 할 수 있는 것이 아닐까.
\vspace{5mm}

그런데 더 중요한 건, 최악의 케이스까지도 담담하게 준비하고 받아들이는 케이스가 최악을 겪을 가능성은 매우 낮다는 것이다.
\vspace{5mm}



\section{[학습공학 006] 순서}
\href{https://www.kockoc.com/Apoc/427063}{2015.10.18}

\vspace{5mm}

누군가 논리가 뭔가 물어보면 난 간단히 이렇게 얘기할 것이다.
\vspace{5mm}

"납득할 수 있는 \textbf{순서}"
\vspace{5mm}

이해와 암기는 여기서 하나가 되어가는데.
한 예를 들자면 잘 정리된 그림이나 도표가 줄글로 이뤄진 텍스트보다도 기억의 효율이 떨어지는 경우가 있다.
이게 왜 그럴까 하면서 오랫동안 고민하다가 한 일본인 저자가 쓴 책을 읽고 나서 깨달았다.
우리가 기억할 수 있는 건 "기호의 수열"이다.
\vspace{5mm}

사실 인지심리학 인간공학적으로 더 상세히 정의할 필요가 있지만
공부에 있어서 가장 중요한 건 '의미' 기억이며, 이 의미기억은 \textbf{'순서'를 대단히 중시}한다.
사실 이미지 기억조차도 '시간'과 '공간'이라는 순서를 따라간다는 점에서
기억의 핵심은 좋은 머리, DHA, 정체불명의 머리좋아지는 약이 아니라
\textbf{"납득할 수 있는 의미들을 가장 올바른 순서"대로 반복하느냐에 달린 것}이다.
\vspace{5mm}

그리고 이 점에서 책을 혼자 읽는 건 매우 위험하면서도 중요한 작업이 되어가는데.
강의가 책보다도 소화가 빠른 이유는, 탁월한 강사라면 \textbf{그 의미들을 더 '잘개' 쪼개서 책보다도 타당한 순서대로 정리}해주기 때문이다.
즉, 강의란 더 흡수율이 좋으며 호소력을 갖춘 의미들의 '재배열'이라고 보면 되는 것이다.
그런데 이걸 바꿔 말하면 탁월한 reader라면 책읽기에 있어서 저자의 고리타분한 문장 그대로 따라가지 않으며
반드시 그 의미들을 본인이 이해하기 쉬운 순서로 재배열할 것이란 사실이다.
\vspace{5mm}

노오력을 해도 되지 않는 공부의 문제는 '논리'가 없다는 것이고, 논리가 없는 이유는 '순서'를 갖추지 못해서이다.
순서를 갖추지 못한 건 본인들이 그런 순서의 '프레임'을 알지 못해서이다. 그리고 이렇게 가다보면 결국 초기 교육과 독서로 귀결된다.
우리가 배우는 그 고리타분한 예의범절, 이런 걸 지키는 사람들이 계속 상위계급에 위치해있을 수 있는 이유.
그건 바로 \textbf{'순서'대로 하는 습관}이 들어있기 때문일 수도 있다.
\vspace{5mm}

공부시간이 많은데도 성과가 없는 이유는 여러가지가 있다 - 사실 인간은 학습하면 안 되는 동물이라서 공부를 싫어하고 성과가 적어야 당연하지만.
공부를 잘 하는 학생들의 경우는 어떻게든 '순서'의 스키머가 있는 반면,
공부를 못 하는 학생들은 그런 걸 갖추지 못 했다라는 걸 개인적으로 확인하고 있다.
머리가 좋건 나쁘건 그건 사실 알기도 힘들고 오히려 머리좋다고 여긴 놈들이 딴 길로 빠지는 것도 적잖게 보았지만,
그 부모가 매우 보수적이고 형식을 중요시하는 경우라거나, 어린 시절부터 독서의 틀을 잡아준 경우는 매우 괜찮다고 보고있는 중이다.
\vspace{5mm}

순서 하나 잘 지키면 이해와 기억까지 모두 확보하고 이것이 연쇄적 반응을 일으켜...
반대로 순서를 지키지 못 하면 그만큼 많은 것을 잃게 된다. 빈익빈 부익부는 이렇게 예정된 것이었을까.
자신들이 깨어있는 척 하던 자칭 참교육자들이 강조하는 것이 '생각하는 방법'이라는 것인데
사실 생각하는 방법이란 것도 결국 인류의 역사에서 확립된 보편타당한 논리적 순서를 따라가는 것 그 이상 그 이하도 아니다.
\vspace{5mm}

국어, 수학, 영어에서 배우는 것도 결국 '순서'
컴퓨터는 0 1 이진법으로 입력해야 한다. 우리의 뇌가 기억하고 받아들이는 것도 '순차적인 의미'들일 뿐이다.
모든 작업도 순서가 있다.
\vspace{5mm}

그런데 우리는 일상 감각의 완전함을 믿은 나머지 그 순서를 종종 무시하곤 하는데...
사실 수능 시험도 "더 효율적이고 타당한 순서대로 입력된 기호들을 더 압축된 순서로 인출할 수 있느냐"로 결정남.
\vspace{5mm}

가령 1번문제에서 A라는 단서를 준다.
이 문제를 푸는 사람은 Z를 연상해낸 사람이다.
그런데 이 경우는 2가지로 나뉜다. A$\rightarrow$Z인 경우와 A$\rightarrow$B$\rightarrow$ ... $\rightarrow$Z인 경우.
\vspace{5mm}

보통은 A$\rightarrow$Z를 선호할 것이다. 당연히 쉬운 문제는 이렇게 풀린다.
그러나 어려운 문제를 풀려면 시간이 걸리더라도 중간의 B, C, ... X, Y까지 연상해내야한다.
실제로 수학 4점이든 탐구 3점은 많은 논점을 도출해낼 수 있느냐가 득점을 좌우한다.
그런데 A$\rightarrow$Z로만 학습한 사람이라면? 당연히 못 풀어낸다.
\vspace{5mm}

이건 비단 수험 뿐만은 아닌 것 같다.
일반적인 대중은 A에서 Z만 연상하고 나머지는 몰라라하지만
전문가라 할 수 있는 사람은 A에서 Z를 도출하기 위해 B,... Y까지 다 유도해내고 거기서 쓸 수 있는 건 없나 빛의 속도로 떠올리며
그것도 모자라 가나다라마사부터 일본의 가나문자까지 동원해대기 때문이다.
\vspace{5mm}

어떤 강의가 좋냐 한다면 이런 걸 가르쳐주는 사람이 아닌가 싶다.
교과서에서는 A$\rightarrow$Z로 나와있더라도 이걸 ABCDEFGHIJKLMNOPQRSTUVWXYZ로 쫙 풀어 설명해주는 강사.
그것도 알파벳송을 붙여서 뮤지컬스럽게 가르칠 수 있으면 더할나위없이 좋다.
\vspace{5mm}

그런데 대다수가 이걸 알리는 없지 않나.
우리의 기억은 '순차적'으로 이뤄지고, 또한 회상과 연상 역시 '순차적'으로 이뤄질 뿐이다.
이건 매우 간단하지만 중요한 진리이겠지만 이걸 알아먹을 사람은 별로 없을 것이다. '경험'해보면서 직접 검증해봐야 알 수 있기 때문이다.
\vspace{5mm}






\section{[학습공학 007] 고정관념}
\href{https://www.kockoc.com/Apoc/427079}{2015.10.18}

\vspace{5mm}

제목만 보면 또 이 꼰대가 뭔 잔소리를 하느냐.... 전혀 오해마셨으면. 뻔한 이야기는  쓸 필요가 없기 때문이다.
우리는 종종 인과관계와 상관관계를 혼동하기도 하는데
보통 공부를 열심히 한 사람들은 피골이 상접하고 스트레스를 많이 받는다... 라고 하여
\textbf{공부를 많이 하고 잘 하기 때문에 몸이 피곤하고 스트레스라는 부작용이 따라온다라고 믿고있다.}
그런데 학습공학 앞선 글에서 논했지만   인간은 자아정체성을 유지하기 때문에 함부로 학습하는 걸 꺼리는 동물이다.   닥치는대로 학습한다면 그건 공기 중에 노출된 나트륨처럼 대책없이 연소해버리는 것이기 때문이다.   공부가 힘든 이유는 간단하다, 공부한다고 얘기해도 뇌에서 그걸 함부로 받아들이려고 하지 않기 때문이다.   콧대 높은 미녀에게 밀당을 잘 하고 선물, 편지 공세에다가 온갖 심리전을 써야 그나마 데이트가 가능하듯   여러번 회독수를 높이고 강의도 듣고 문제도 많이 풀고 깨져보는 경험을 해야 그나마 뇌에서 에라 모르겠다하면서 흡수해서 공부가 되는 건데.   뇌는 쾌감을 주는 자극도 잘 받아들이지만(음식, 게임, 성 기타 등)   \textbf{무엇보다 '생존'과 관련된 자극도 매우 민감하게 받아들인다.}   그리고 여기서 우리는 왜 지칠 때까지 공부한 사람들이 공부를 잘 하는가하는 걸 맥빠진 결론으로 파악할 수 있다.   A라는 학생이 공부시간을 늘려서 공부가 잘 된다고 생각하고 미친 듯이 한다.   당연히 스트레스를 와장창 받게 되고 피로감이 쌓여서 미칠 지경이다, 게다가 타인과 경쟁하고 비교하는 맛에 들린다.   그런데 이것이 "내 목숨이 위험하다"라는 신호로 뇌에서 받아들인다면?   당연히 뇌에서는 그 학습을 위험한 경험이라고 여기며 공부하지 않는 쪽으로 움직이려 할 것이다.   그래서 공부를 12시간 해서 우왕 잘 된다 하는 애들이 자기도 모르는 사이에 놀아제끼꺼나 허송세월을 보내는 것이다.   그런데 그것도 안 되고 계속 공부해야하는 상황이 온다고 치자. 이럼 재밌는 일이 벌어진다.   첫째, 뇌에서는 그렇게 학습해야 하는 상황을 체념적으로 받아들인다. 그래서 문을 열고 학습하려는 정보를 받아들일 준비를 한다.   둘째, 정신없이 공부해서 몽롱해지고 지쳐버리면 '일상감각'이 사라진다. 그 때문에 그 공부하려는 내용이 직접 무의식으로까지 들어오게 된다.   다시 말해서 공부를 많이 해서 결과가 좋고 힘들어진다.... 라는 통념과 달리   공부를 많이 하면서 스트레스에 눌리고 피로감이 생기며 제정신이 아니다보니까   뇌에서는 학습하지 않으려던 기존의 쉴드를 유지하지 못 하게 되면서 '학습'해버리는 상태에 도달,   그래서 비로소 공부가 되는 것이라는 이야기.   지금 적으면서도 이게 말이 되나... 싶기도 하지만    개인적인 경험이든 관찰이든 그동안 설명되지 않던 학습현상이 이걸로 설명된다는 것이다.   가령 학습법에 관한 책을 읽고 집중한다 하는 친구들이 정말 공부를 잘 하나, 유감스럽지만 단 한건도 없다.   나 역시 그런 집중법에 대해선 꽤 많은 것을 알고 있고 실습해보았다, 당연히 모두 실패로 돌아갔다.   어떤 시험이건 성과를 잘 거두는 친구들은 그런 효율성 신경쓰지 않고 아주 무식하고 피곤하게 공부한 녀석들이었다.   절대 편하게 공부한 녀석들은 성적이 좋지 않았다.   우리의 통념은 \textbf{신체적 피로와 스트레스가  학업에 지장을 준다}는 것인데.   이거야말로 사실 가장 위험한 착각이 아니었을까.   오히려 피로와 스트레스가 뇌를 항복시켜 학습을 하지 않을 수 없는 상태로 몰아간다고 본다면   효율성이든 뭐든 신경쓰지 않고 무식하게 공부하는 친구들이 결국 잘 되는 것이 합리적으로 설명될 수 있다.   정반대로 편한 환경에서 유유자적하게 마치 자기가 제갈량인 듯 한 포즈로 공부하는 친구들이 정작 결과는 개판인 이유도 설명이 된다.   수험에 대해선 이것저것 만물박사급으로 알고 있지만 성과가 미진한 똘똘이 스머프류들도 설명할 수 있다.   오히려 학습효과를 늘리려면 스트레스를 와장창 받고 신체적 피로도 적당히 느껴야 '뇌'를 협박하여 공부가 되는 것이 아니겠는가.   졸렬한 건 비단 손오공도 나뭇잎 마을도 아니다. 바로 우리의 '뇌'이다.   뇌가 우리의 지적행위와 감정 등을 총괄한다고 생각해서 우리의 뇌는 순결하다능, 깨끗하다능, 공부하기 좋아한다능... 이라고 생각하지만   사실 이건 아무런 근거가 없는 이야기이다. 뇌가 원래 그렇게 완전무결하면 막장드라마가 현실에서 벌어질 리는 없지 않나.   오히려 뇌는 학습하기 싫어하고 말초적인 쾌감을 추구하는 등 우리를 소인배로 몰아가려고 한다,   따라서 우리가 어떤 강제를 해서라도 뇌가 학습을 하지 않으면 안 되게끔 해야한다라고 보는 것이 합리적이지 않나.   그렇게 보자면 '스트레스 받지마라' 하거나 '신체적 피로를 피하라'하는 건 오히려 위험한 이야기가 아닌가.   학습을 하게끔 하려면 생존의 위협을 느끼거나 강한 압박에 시달려야한다는 측면에서 보자면, 절대 공부환경은 편해서는 안 된다.   매일 눈물바다에다가 짜증도 부리고 악도 지르며 주먹질도 기물파손도 해야 하면서   이걸 공부하지 않으면 더 큰 고통을 맛보게 된다라는 걸 뇌가 납득해야한다는 이야기로 간다면   그동안 '학습이론'과 맞지 않는 현상들의 딜레마가 명쾌히 설명될 수 있다.   그렇다면 매일매일 꾸준히 공부하는 게 낫다라는 것도 역시 배격해야 할 고정관념일수도 있다.   사람에 따라선 차라리 이틀사흘 밤샘하거나 잠 못자는 식으로 벼락치기로 공부해보는 게 나을 수도 있단 이야기이다.   그 절박한 경험을 해야만 뇌를 움직일 수 있다라고 한다면.







\section{[학습공학 007] 황금의 3개월}
\href{https://www.kockoc.com/Apoc/432268}{2015.10.21}

\vspace{5mm}

공부의 단계를 ABCDEFGHIJKLMNOPQRSTUVWXYZ라고 합시다.
\vspace{5mm}

우선 기본서는 저걸 다 커버하긴 합니다, 그런데 조심할 게 있음.
눈에 보이는 문자는 딱 ABC까지만 나타냅니다.
나머지 EFGHIJKLMNOPQRSTUVWXYZ는? 행간에 숨어있습니다요.
행간에 숨어있단 의미는? ABC를 철저히 마스터해서 이걸로 알아서 추론해야한다는 이야기임요.
그럼 왜 애당초 저자들이 저걸 싣지 않지? 그건 두가지 이유가 있을 겁니다.
첫째, 그걸 다 싣는 것 자체가 불가능하고 다 싣는다 하면 교재 분량이 10배로 늘어납니다.
둘째, DEF의 영역부터는 책에 싣는 건 어려운 '경험적', '실무적' 방법이나 절차여서입니다.
\vspace{5mm}

그럼 기출은? 대략 QST의 단계입니다.
기본서의 ABC와 기출의 QST 사이를 메꾸는 것이 학습이겠지요.
혹자 왜 기본서를 보느냐, 기출과 동떨어졌는데. 맞는 지적입니다.
ABC와 QST는 동떨어져도 너무 동떨어졌으니까요.
하지만 알파벳은 QST부터 시작하는 게 아닙니다. 저 QST는 ABC로 시작되는 체계에서만 온전한 의미를 갖추는 것이죠.
\vspace{5mm}

그럼 문제집은? 대략 DEFGHI순입니다.
그리고 인강은?
초급이냐 중급이냐 고급이냐에 따라서 달라집니다만
대략 A$\sim$T까지 상당히 왔다갔다 거리는데 문제는 이게 불완전하다는 것입니다.
중요한 건 커버 범위는 넓다하더라도 제대로 가르치는 건 3개입니다.
ABC냐 DEF냐 GHI냐, JKL이냐. 이런 식으로요
\vspace{5mm}

학습이란 결국 ABC와 QST 사이를 잇는 DEFHGIJKLMNOP까지를 본인이 ABC와 QST를 열심히 암기, 이해하고 문풀하면서
채워나가는 과정이다라고 할 수 있을 겁니다.
이 영역에서 인강은 도움이 될 수 있습니다만 '책임'은 절대 지지 않습니다.
C에서 Q 사이는 본인이 채워가야 유효합니다. 그래야 T 다음의 UVWXYZ까지도 도달할 수 있죠.
\vspace{5mm}

보통은 A$\sim$X까지만 해도 최상위권에다가 괴수 인정받을 수 있을 것입니다.
실제로는 상당히 불완전한 상태에서 문자 10개 수준으로 와리가리거리지만요.
다만 뛰어난 강사라면 QST를 설명하면서도 이게 어떻게 A,B,에서 연유하는지를 이야기하겠죠.
\vspace{5mm}

그런데 제목은 황금의 3개월인데 이 녀석은 뭔 알파벳 타령을 하고 있느냐일 건데요
이 시점에 오면 3개월만 더 있었으면 하는 친구들이 있습니다.
이 친구들은 우선 ABCDE까지, 그리고 OPQRST까지는 어느 정도 되었습니다. 즉, 공부할 맛이 생겼단 것이죠
그러나 FGHIKLMN은 아직 미완성이거나 불완전합니다. 그래서 여기서 점수가 안 나오는 것이죠.
이 영역은 학이 아니라 습(習)에 속합니다. 본인이 문제를 풀고 기본서를 읽고 깨져보고 생각하고 고뇌하고 울어보고 소리질러야 달성됩니다.
\vspace{5mm}

하지만 시험이 끝나면 다시 공부하는 친구들은 ABC와 QST를 다시 시작합니다. FGHKIKLMN을 채워야하는데 그걸 모르고 삽질하는 거죠.
이 친구들이 더 일찍해서 3개월동안 FGHIKLMN을 스스로 채웠다면 그 때부터는 고수의 반열이 되는 것입니다.
시장진입에 겨우 성공해서 매출확대하여 브랜드를 알리는 단계라고도 할 수 있는데 보통은 시험 직전에야 이 단계에 도달해버리죠.
\vspace{5mm}

이 이야기는 간단합니다, 수험에서 가장 중요한 비용은 '시간'이란 것이지요.
1m만 더 파면 금맥에 도달하는 건데라는 교훈으로 인식되죠
제가 보는 수험은 결국 본인이 저 FGHIKLMN까지를 직접 채우냐 못 채우냐 하는 OX로 결정납니다.
이걸 채우고 공부한다면 가치투자가 되는 것이고, 이걸 못 채우고 ABC와 QST에서 맴돈다면 도박이 되어버리지요.
\vspace{5mm}

뭔가 수험생을 위하는 척 하면서 장사질하는 수험고수들은 QSTWXYZ만 강조합니다라는 게 중요한 얘기가 되겠죠.
뭘 하더라도 가장 중요한 건 ABC입니다. 그래야 그 사이를 채울 수 있으니까요.
\vspace{5mm}

그리고 FGHIJKLMN을 채우는 과정이 힘든 건, 이게 바로 결과가 나타나진 않아서입니다.
기출이나 모의 대부분은 QST 이상에서 나오기 때문에 QST 이상의 과정을 외우기만 하더라도 어느 정도 점수는 나오지요.
그러나 그 이상으로 가려면 반드시 FGHIJKLMN을 스스로 채워야합니다. 그래야만 QST 이상의 내용이 더 이상 '암기'가 아니게 됩니다.
이 스스로 채워야하는 영역은 강사들이나 고수들이 거의 언급을 안 합니다요.
섹시 이미지로 수억씩 벌어대는 미녀들이 자신들의 성형수술을 공개하지 않듯이 말입니다.
\vspace{5mm}

올해 시험을 치르고 불만족스러우면 일주일만 신나게 놀고, 바로 다시 '문풀'로 들어가시길 바랍니다요.
님들 시험 망했다고 아무도 놀리진 않음. 요즘 수능은 도박의 성격이 매우 강하다고 생각해서입니다.
그러나 그 도박을 확실성의 영역으로 바꾸는 건 '가능합'니다.
나올 수 있는 모든 문제들을 풀어보고 그 수험과목에 대해서 출제자들 머리 꼭대기에서 노는 수준까지 도달하는 것은
수능이란 시험에 응시한다는 것 이상으로 꽤 가치있습니다. 무엇보다 그 과정에서 교활, 사악해집니다(...)
\vspace{5mm}

+ 할머니심, 아니 노파심으로 말씀드리면
시험 직전에야 '아, 진작 이렇게 공부할 걸'이라고 깨닫는 이유는 무엇일까.
이건 저도 겪고 후회하는 감정이기에 적습니다만, 우리들은 비겁해서 '시험 직전'에야 \textbf{\textbf{절박}하게 공부}해서이죠.
예, 공부라는 별로 자연스럽지 않은 지적행위를 하기 위해서는 엄청 절박해야합니다.
하지만 늘 절박하다간 우울증 걸려 사망할 테니 이걸 강요할 수는 없겠지만.
기왕 절박한 감정을 누릴 거라면 진짜가 좋겠죠, 지금 님들이 누리는 그 절박한 느낌을 내년에 재도전한다면 그대로 가져가야합니다.
다른 경쟁자가 3$\sim$5월에 강사만 따라가면, 교재만 충실히 풀면 되겠지 안심할 동안에, 님들은 11월의 쌀쌀한 가을바람 속에서 공부해야합니다.
\vspace{5mm}

공부할 때 드는 감정은 성취감이 아닙니다. 성취감이야말로 부자연스러운 감정이죠.
수억 버는 사람이 만족스럽게 사나요? 아니죠. 1억 버는 사람은 왜 2억이 아닐까, 50억 버는 사람은 왜 100억이 못 될까 늘 불만입니다.
성취감은 순간적인 쾌감일지 모르겠죠. 절대 오래 못 갑니다. 만약 강사들이 성취감을 얘기한다면 그건 사이비입니다.
\textbf{만점이 나오더라도 늘 불안할 수 밖에 없고, 1권을 풀고나면 다른 2권을 더 풀어야만 잠이 올 것 같아라고 해야 정상입니다.}
그러니 인간성은 개차반이 되고 여기저기 성질 다 부리고 다니고 심지어 주변 사람들을 경멸하고 다 때려죽이고 싶고(...)가 정상(?)이란 것이죠.
대중매체는 흔히 공부 잘 하는 사람들을 인간승리로만 묘사하는데 그건 대중들의 입맛에 맞게 꾸며댄 결과입니다.
공부 잘 하는 사람들은 그냥 악마들이지요.
천사로 살면 도저히 공부 못 합니다. 혹자 자기가 천사라고 하는 수재 양반은 남에게 악마로 보인다라는 것을 모를 뿐이죠.
단, 악마로 안 보이는 방법은 같은 악마들 품에 들어가면 되는 것입니다.
\vspace{5mm}

상담받는 사람 중에서 성격이 매우 더러워지고 깐깐해지고 불안해하는 사람들이 있습니다.
올해 시험은 안 될 수도 있겠죠. 그런데 그 사람들은 정말 바른(?) 길로 가고 있는 것입니다.
이긴다라는 건 단순한 경쟁을 넘어서, 사람의 탈을 쓰고 공부하는 다른 악마, 악귀, 귀신, 사탄 등과 두뇌로 승부하는 게임이죠.
\vspace{5mm}

++ 망하는 지름길은 딱 하나입니다. '만족'하는 것이지요.
흔히 욕심, 탐욕을 경계하라 합니다.
그런 말은 '의식주'도 책임 못지면서 마누라에게 일 다 시켜먹으며 첩질하는 선비들이
현재 북한과 같은 국가에서 기득권 유지하던 동아시아 고대, 중세 국가에서나 나온 위선적인 이야기이죠.
\vspace{5mm}

주부의 낮잠, 침 질질 흘리고 게걸스럽고 우웩거립니다.
이걸 "누님의 오수"로 바꾸면 저작권 인정 안 된다는 일본의 창작물스러운 분위기가 물씬 나죠.
\vspace{5mm}

사람들은 욕심에 대해선 거부감을 느끼는데, 운명을 개척하라에 대해선 오오$\sim$ 그렇지요.
그런데 개척의 전제 자체가 '욕심'이자 '탐욕'이 아닌가요? 그리고 개척의 결과가 바로 빈부격차 형성과 자연파괴 아닌가요?
그런 것 없다라고 말장난할 게 아니라, 실제 역사를 보면 그렇게 나오지 않나요?
많은 사람들이 '의미'를 생각하지 않고 그저 그 말의 멋에만 신경쓰는 경향이 있습니다.
만족하며 살면 가난해진다라는 말이나, 개척 안 하고 멈춰있으면 빈곤해진다라는 말은 똑같은데 전자는 비난먹고 후자는 찬양받죠.
\vspace{5mm}

위선적인 건 질색이죠. 여기 수험판만 하더라도 적잖은 장삿꾼들이 수험생을 위한다 그렇죠.
하지만 실제로는 '인세'나 '수업료'를 더 바란 것 아닌가요?
좋은 문제 공유? 기실은 왜 더 많이 안 팔리느냐, 쓸데없는 실○가 쏟아진다하던 게 그들의 내심이 아닌가요?
좋은 의도라는 것도 그게 현실에서 '금전거래'와 결부되면 일단 의심받는 게 당연하죠.
\vspace{5mm}

이 글 읽는 수험생 분들은 법을 어기지 않는 범위에서 이기적인 방향으로 가세요.
이런 생각도 하겠죠, 내가 윤리적이지 않으면 세상 안 굴러가는데.... 나중에 더 많이 경험해보시면 아십니다.
이 사회가 무서운 것이 - 더 정확히 말하면 이 사회를 설계한 문과계열인들이 무서운 건,
이기적이고 탐욕적이며 게걸스러운 탐욕에다가 발바리스러운 성욕을 지닌 인간이란 짐승들이
그나마 평화롭게 살 수 있는 설계도를 짜놓았단 것이죠. 이게 너무 자연스러워서 우리가 공기의 소중함을 모르듯 못 느끼고 있는 겁니다.
\vspace{5mm}

항상 자기 처지에 불만을 느끼고 더 많이 얻을 수 없을까, 더 효율적인 시스템을 훔칠 수 없을까 고뇌해야합니다.
그래야 진보합니다.
\vspace{5mm}






\section{[학습공학 007-2] 읽는 법에 대해서}
\href{https://www.kockoc.com/Apoc/432733}{2015.10.21}

\vspace{5mm}

그냥 보는 것과 읽는 것은 다름.
본다는 건 그 그 이미지로 느끼는 것이고
읽는다는 것은 그것을 우리가 알고있는 기호로 분설한 다음 의미를 읽어내는 것임.
\vspace{5mm}

의미를 읽어내면 추상적 명제들로 나눌 수 있고
그 추상적 명제들의 조합으로써 그 읽은 문장의 전제와 결론을 추론해볼 수 있고
그런 추론 과정에서 과거를 추적하거나 미래를 예견해볼 수 있는 것임.
\vspace{5mm}

수학에서 배우는 것이 그것인데
수학문제를 잘 푸는 친구들은 주어진 문제들을 분설해 읽은 뒤
그 조건, 공식, 그래프 등이 뭘 의미하는가 위에서 말한 과정에서 읽어내어서
해답을 구하기 위한 풀이과정을 '순서'대로 나열, 즉 논리적 풀이를 해나가는 것임.
물론 대부분은 이런 과정을 번거롭다 거치지 않고 자기가 암기한 패턴대로 풀어댐.
당연 올해 치를 수능시험의 21, 29, 30은 그렇게 내지 않겠죠.
\vspace{5mm}

우리가 음식물을 그냥 먹는다고 영양분이 되는 게 아니지요. 소화를 시켜나아가야지.
꼭꼭 잘 씹고 적당한 양을 적시에 섭취해야 소화기관에서 기계적, 화학적 소화를 시켜 건강해지는 것이지요.
'읽는다'는 것도 마찬가지입니다.
\vspace{5mm}

강의라는 건 양날의 검이지요.
강의가 책보다 잘 들어오는 이유는, 보통은 강사가 '소화를 시켜주거나' 아니면 '소화가 잘 되도록 도와주기' 때문입니다.
강사는 교과서나 참고서를 가지고 실제 시험문제를 풀 수 있게 자기만의 해석을 전달하는데
그 해석을 그대로 따라가는 게 득점에 도움이 되는 경우가 많지요. 그래서 강의를 선호하게 되는 것입니다.
\vspace{5mm}

그런데 이걸 알아두셔야 함, 저건 강사의 해석이지 님들의 해석이 아니라는 것.
교과서나 참고서가 아무리 풍부한 내용을 담았다고 한들 그걸 본다고 실력이 올라가는 게 아닙니다.
읽어야 비로소 내 것이 되는 것이지요.
그리고 한번 읽는다고 되는 것도 아닙니다. 주어진 내용을 기억할 건 기억하고 나에게 맞는 의미로 소화시켜 숙달시켜야죠.
게다가 읽을 '관점'과 '방법'도 여러가지가 있지요.
초기에는 정독이 무리라서 속독으로 대충 읽으면서 지도를 파악하겠죠.
그러나 가랑비에 옷젖듯이 반복해 읽다보면 어떤 내용인지 감이 오게 되고
그 다음부터 내용이 숙달되기 시작하면서 '정독'을 하면서 질문을 던지고 문제를 스스로 풀면서
"왜 그런 결론이 나오나", "xx한 문제는 어떻게 준비해야하나"라고 생각해보는 겁니다.
이런 과정을 거쳐야 실력이 오르는 것이죠.
\vspace{5mm}

개인적으로 교재 추천해달라 어떤 교재 좋냐 하는 사람들을 강경하게 취급해대는 이유가 그겁니다.
그 사람들은 저런 단계는 도달도 못 해보았단 이야기이거든요. '읽는다'의 경지까지 가면 교재간 차이는 - 적어도 시중교재만 보면 무의미해집니다.
중요한 건 내용이 아니라, 어떻게 읽느냐이고, 또 어떤 읽기 과정을 통해서 우리가 쓰는 일상어처럼 자유자재로 쓰느냐여서입니다.
이런 걸 모르니까, 즉 읽는 걸 못 하니까 더 새로운 내용이 있을지 모르는 교재를 찾거나 다르게 해석해주는 강의를 찾는 거죠.
국가로 비교하자면 제조업 자체가 없어서 농업에 의지해 선진국의 식민지로 전락한 상태나 마찬가지입니다.
\vspace{5mm}

평소에 규칙적 운동을 해서 건강한 사람이야 그냥 소박한 백반만 먹어도 날라다니겠지만,
비실비실거리는 멸치나 파오후거리는 뚱보는 현기증난다고 무턱대고 보약지어달라하는 것과 똑같다니까요.
\vspace{5mm}

수능은 '\textbf{신속하고 정확하게 출제자의 의도를 읽어낼 수 있느냐'}가 중요한 시험입니다(논술시험도 마찬가지이겠지만요)
님들이 문제집을 푸는 건 '읽는 과정'을 훈련하기 위해서입니다. 답을 함부로 보지말라하는 것도 그것이죠.
스스로 문제를 읽고 답에 도달하기 위한 추론과정의 틀을 해설에서 배우긴 합니다만 일정단계에서는 직접 해설을 만들어야합니다.
국어, 수학, 영어, 탐구에서 변별력을 좌우하는 킬러문제의 공통점은
그런 읽기-해석을 10계단 정도 밟아야하거니와 아주 정확히 논증해야한다는 것입니다.
강의에서 킬러 공략법을 내세워보았자 소용없는 것은 이런 것들은 결국 매우 정확한 논리를 요구하기 때문입니다.
\vspace{5mm}

올 1년동안도 쭉 관찰을 하고 교차해보았지만
교재 가리지 않고 그냥 하나 딱 정해서 빨리 시작하는 사람이 덜 웁니다.
가장 현명한 건 작년 11월 이후에 바로 빨리 시작한 케이스인 것 같습니다(이 역시 시간이 모자라다 아우성이지만요)
\vspace{5mm}

반면 똑똑은 한데 이상하게 시험에서 나가리나는 케이스가 있습니다. 이 경우도 당연히 왜 그런가 생각해보고 탐구해보지만
첫째, 공부량이 자기가 생각하는 것보다 부족함. 연습이 덜 되어있음.
둘째, 특정 인강이나 특정 교재에 대한 의존도가 심하고 절대 거기에서 안 벗어나려함(이른바 '닫혀있는' 케이스)
\textbf{셋째, "읽는 능력" 자체가 부족함.}
\vspace{5mm}

다시 말해 본인은 매우 똑똑하고 아는 게 많습니다
그런데 실전에서는 망가짐 - 그리고 특히 이런 케이스가 이공계에 많음.
왜 그런가 하면서 내세울 수 있는 가설은 그겁니다. 지나치게 특정한 틀에 매여있는 것도 그렇거니와
"읽는다"는 걸 하지 못 합니다.
\vspace{5mm}

\section{[학습공학 008] 흙수저의 공부법}
\href{https://www.kockoc.com/Apoc/437242}{2015.10.23}

\vspace{5mm}

뱁새가 황새따라한다라는 속담도 있죠.
흙수저의 문제점은 흙수저답게 공부해야하는데 자꾸만 금수저를 따라한다는 것입니다.
아마 10대 분들은 '자본'이나 '가정환경'에 대해선 아직까지도 관념적으로만 알지, 이걸 체감상 느끼기는 힘들 것입니다만.
어렸을 때부터 교수나 전문직급 부모 아래에서 독서도 많이 하고 해외여행도 다니면서 명문학교 간 학생과
맞벌이 부모 밑에서 울며겨자먹기로 뽀로로나 보다가 pc방에 맡겨져 그냥 평범한 학교간 학생은 정말 '같은 인간이 맞나'할 정도로 차이가 큽니다.
\vspace{5mm}

더 아이러니한 것은 저 명문학교 간 학생이 "인간은 평등하다"란 말을 외친단 것이죠 ㅋ
인권적 메시지조차도 실은 상류층이 독점하고 있습니다.
흙수저 입장에서든 저런 걸 보는 제 입장에서든 인간이 정말 평등한 것 맞아 라는 생각이 들겠습니다만.
\vspace{5mm}

수험사이트에서 최상위권 학생이 만들었다는 교재들은 흙수저를 배려한 것이 아닙니다.
그런데 아이러니컬하게도 그걸 가장 많이 구매하는 건 흙수저입니다.
똥배나온 아줌마들이 드라마의 여주인공들이 입는 옷이나 걸치는 장신구를 구매하는 것과 똑같죠.
반면 정말 흙수저에게 필요한 책들은 수준이 낮다고 까이죠.
\vspace{5mm}

그 비극은 흙수저들의 컴플렉스 때문입니다.
실상은 흙수저지만 나는 절대 흙수저가 아니야라고 소리치고 싶죠.
창조경제 어쩌구 이전에 '마케팅'의 강력한 무기는 상대방의 컴플렉스 자극이라고 하죠.
하위권이기 때문에 1등 교재에 집착하고, 상위권들이 모인다는 수험사이트의 글을 열심히 읽습니다.
정말로 뱁새가 황새 따라한다라는 이야기가 되는 것입지요.
\vspace{5mm}

흙수저가 정말 흙수저를 벗어나려면 흙수저의 장점을 살린 공부를 하는 수 밖에 없습니다.
이해보다는 암기를 해야하고, 스킬에 집착하기보단 양치기로 가야하며,
나는 저 금수저보다 머리가 10배는 나쁘다, 그 대신 엉덩이도 10배는 무겁다라고 가며
너희 금수저놈들이 이 진흙탕에 들어오면 내 먹이다, 나는 나만의 '노예 방식'으로 너희들을 압살해주겠다라고 하며 가야합니다.
그러나 다수의 흙수저들은 금수저들에게 홀려서 금수저에게나 먹히는 방식으로 가다가 더 좌절해버리죠.
\vspace{5mm}

다이아몬드나 흑연이나 성분은 똑같죠.
흑연수저가 더욱 흑연을 많이 모아 고온고압 상태에 가면서 다이아몬드 수저가 되는 수 밖에 없지 않겠습니까.
노오력이라고 하면 무조건 거부감부터 드는 분들도 계시겠지만,
그럼 노오력 말고 다른 길이 있느냐 물어보면 단 한명도 답을 못 하덥니다.
혹자 이 사회를 바꾸면 된다 하는데 그래보았자 그런 걸 원하는 머리좋은 사람의 하수인으로 전락할 뿐이죠.
\vspace{5mm}






\section{[학습공학 009] 야수 길들이기}
\href{https://www.kockoc.com/Apoc/450912}{2015.10.29}

\vspace{5mm}

극단적 상황을 가정해보자.
낙서가 많이 된 5년 묵은 오탈자 투성이 EBS 교재를 던져주고
사흘동안 공부해서 2등급이 나오지 않으면 목숨을 잃는 상황이 온다라고 해서 공부하면
과연 성적이 오를까 안 오를까.
\vspace{5mm}

이게 뭔 소리요 하는 사람들이 많겠지만
실제로 공부를 못 하는 사람들이 많은 이유는 너무나도 간단하다.
\vspace{5mm}

엄밀히 말하면 공부는 \textbf{'뇌가 알아서 하는'} 것이지, 우리가 의식적으로 하는 것이 아니기 때문이다.
우리가 의식적으로 할 수 있는 건
강의를 듣는다거나
책을 펼친 뒤 안구를 굴린다거나
사실 그것 밖에는 없다.
\vspace{5mm}

우리가 할 수 있는 것이란 "뇌를 특정환경에 놓아 자극을 받도록" 관리하는 것 빼고는 없는 것이다.
다시 말해서 능동적으로 할 수 있는 것이란, 뇌가 알아서 공부하도록,
즉 공부가 되도록 하는 환경을 준비하고 유지하는 것 빼고는 없는 것이다.
\vspace{5mm}

그래서 공부를 열심히 해야지라고 결심할 필요없다.
그래보았자 30분도 안 지나서 인터넷에 접속하는 것보다는,
주변을 청결히 하고 공부할 책만 갖다놓으면서 진도가 얼마나 나갔나 체크하면서 공부할 수 밖에 없는 환경을 만드는 게 낫다.
그래서 피트니스 클럽에서 배에 왕자를 새기거나 섹시한 하반신을 만들 듯이
뇌를 단련시키는 수 밖에 없는 것이다.
\vspace{5mm}

뇌가 자극받아 변하고 난 다음에야 공부의 맛을 알기 때문에 뇌는 더욱 더 많은 정보와 지식을 원하게 되고
그래서 그 다음부터는 공부에 중독되는 것이고, 이렇게 상위권으로 올라가는 것이지
실제로 성적이 잘 나오는 친구들이 공부를 열심히 한다... 라는 건 사실과 부합하지 않는 얘기다.
그런 친구들이 실제로 의지가 강하다거나 패기를 발휘한다거나 그런 건 아니다.
그렇다고 손모 선생이 얘기한대로 유전자가 특별히 좋아서도 아니다(유전자만큼 불확실한 것도 없는데 그렇게 자신있게 얘기할 수 있나)
단지 그 친구들은 뇌가 그렇게 공부의 맛을 아는 상태에 도달할 수 있는 환경에 장시간 놓여있었을 뿐이다.
\vspace{5mm}

다시 말해 공부는 능동태가 아니다. 그냥 뇌가 알아서 하는 것이다.
그럼 우리가 할 수 있는 건?
공부보다 더한 쾌감을 제공하는 인터넷, 게임, 드라마, 야동 등을 멀리해야 한다.
공부하다가 30분도 안 되어서 접속한다면 그냥 인터넷을 잘라버려야 한다.
우리가 능동적으로 할 수 있는 것이라고는 뇌가 다른 마약에 중독되지 않고 공부의 참맛을 느낄 수 있는 환경 조성이다.
\vspace{5mm}

자기는 공부 열심히 하려 하는데... 라고 다짐하면서 2년 이상 뻘짓하는 인간들의 문제는
우선 본인이 직접 밥벌이도 안 하고 고생해보지도 않으면서 그저 엄마가 주는 세끼밥이나 챙겨먹으면서
언젠가는 자기가 공부를 잘 해서 이 모든 것을 만회하겠다는 전혀 현실성 없는 착각을 하고 있단 것인데
난 진실만 말한다. 100명 중 1명 정도만 가능하나? 나머지는 그냥 허송세월하다 집안 말아먹고 사건사고 기사에 날 가능성이 높다.
공부한다고 핑계대면서 노동을 회피하고 잉여질하는 상태만큼 '뇌'가 좋아하는 게 없다.
그 상태에서는 공부한다는 의지나 각오조차도 뇌의 핑계 수단으로 전락할 뿐이다.
이런 친구들은 가차없이 식민통치 당하는 수 밖에 없다. 군대도 좋은 약일지도 모른다.
\vspace{5mm}

자, 공부라는 건 뇌가 하는 것이라고 했다. 그렇다면 우리가 하는 건,
'공부하기 싫어하면서 마약, 게임, 섹스'나 좋아하는 뇌를 혼내고 길들이는 것이다.
그런데 뇌는 교묘히 우리의 이성을 잠식해서 쾌락을 추구하는 방향으로 움직이려한다.
인강과 교재를 선택하는 것조차도 기실은 \textbf{야동을 골라보는 것과 똑같다}.
공부가 더 잘 된다는 건 개뿔이고, 실제로 뇌는 덜 고생하고 덜 공부하는 방향으로 움직이려 한다.
\textbf{그래서 인강과 교재를 선택한다는 핑계로 그 기간동안은 '공부'를 안 해도 되기 때문이다.}
이렇기 때문에 그냥 생각없이 양치기하는 게 가장 효과가 좋은 것이다.
양치기만큼 뇌를 굴복시키는 합리적인 시스템이 없기 때문이다.
\vspace{5mm}

그럼 수능시험으로 가보자. 지금부턴 뭘 해야하나
실모나 직모를 푼다?
난 이것만큼 어리석은 게 없다고 본다.
\vspace{5mm}

어려운 문제를 풀려면 그건 올해 여름, 늦어도 9월까지이다.
왜냐면 이 때까지는 처절히 깨져보아야하고 뇌를 재조립해야하기 때문이다.
그리고 이 때까지는 실패경험을 해도 좋다, 왜냐면 10월에 만회할 수 있어서이다.
\vspace{5mm}

무슨 이야기냐고 하는 분은 다시 윗 주제를 보자. 공부는 우리가 하는 게 아니다.
\textbf{공부는 뇌가 하는 것이고, 우리는 뇌의 조련사일 뿐이다.}
뇌가 시험에 최적화되도록 달래거나 설득하거나 협박하거나 억지부리거나하는 게 우리의 몫이다.
\vspace{5mm}

시험이 10일도 남지 않은 지금은, 오히려 '성공 경험'을 해봐야한다.
그 점에서 좋은 건 '쉬운 문제'를 시험형식으로 제 시간에 맞춰 풀거나, 아니면 과거에 풀었던 기출을 다시 보는 것이다.
만약 이런 것들도 점수가 안 나온다고 하면, 내년 수능을 치르는 게 낫다(이건 실력이 태부족이기 때문이다)
시험문제를 다 맞거나 아니면 1$\sim$2개로 선방하는 그런 체험을 해봐야 뇌는 '고득점'의 의욕을 불태우게 되고
그래서 시험 당일날에도 컨디션을 발휘하는 것이다.
이 글을 보는 사람들이 어려운 문제를 풀고 있었다면
극단적인 비율로 줄여보시길 바라고, 전에 말한대로 아침 5$\sim$6시에 기상해서 시험시각에 맞춘 과목실전연습을
\textbf{적당한 난이도의 문제로 하시길 바란다. 그래야 뇌가 시험을 좋아하게 된다.}
\vspace{5mm}

다시 말해서 이 시점에서 어려운 문제 푼다고 실모 간다거나 직모 가는 건 점수가 잘 나오는 최상위권 아니면 비추다.
그런 문제들을 풀었다가 좌절하는 경험을 해보았자 좋을 게 하나도 없기 때문이다.
그런 문제들이 실제 시험에 나온다는 보장도 없고(만약 적중한다면 적중하는대로 사회적으로 큰 문제가 되어버린다)
사실 그걸 내는 사람들은 응시자들이 시험이 망하건 말건 책임질 일은 없기 때문이다.
\vspace{5mm}

과거 기출이나 교육청 기출 등을 똑같은 시간 내에 분량을 1.5배로 늘려서 본인들이 실전연습을 해보는 게 낫다.
시험 당일날 가장 중요한 건 문제푸는 체력이다. 중간에 기진맥진하거나 지쳐서 문제를 못 푸는 경우가 많다.
마라토너들이 트레이너들의 관리로 당일 뛰기 위한 포도당을 비축해놓듯, 수험생들도 당일을 위한 리허설을 해둬야한다.
\vspace{5mm}

시험이 어떻게 나올까 그건 역시 우리가 걱정할 게 아니다.
시험문제는 뇌가 알아서 푸는 것이다. 우리가 하는 것이라곤 뇌가 실수하지 않게 보조하는 것 뿐이다.
본인들이 단련한 뇌를 믿는 수 밖에 없고, 뇌가 불안해하면 다독이면서 '시험쾌감'을 주는 경험을 하는 것이 이 시점에서 할 일이다.
\vspace{5mm}

'제 뇌는 못 믿겠는데요'하는 사람이야 쿨하게 올 시험 걍 올림픽 정신으로 치르고 다시 시작하는 수 밖에 없지 않겠어?
\vspace{5mm}




\section{[학습공학 010] 명문고의 방식}
\href{https://www.kockoc.com/Apoc/466731}{2015.11.07}

\vspace{5mm}

노력보다 습관이 중요하고
목표보단 성격이 강조된다.
\vspace{5mm}

그럼 이를 위해선 어찌해야하나, 바로 환경을 바꿔야하는데
수험생에게 환경이란
\vspace{5mm}

\begin{itemize}
    \item 공간 : 공부와 체육과 과외활동에만 집중할 수 있게 해준다
    \item 시간 : 규칙적인 기상과 취침, 제 때에 하는 식사, 그리고 뇌를 위한 효율적 스케줄링
    \item 도구 : 매우 효율적인 교재, 책걸상, 필기도구 등.
\end{itemize}
\vspace{5mm}

이런 걸 잘 종합한 게 바로 명문고 - 딱히 표현할 말이 없어서리 - 가 아닌가 싶다.
아래 영상은 두고두고 공부하실 분들이 봐야할 영상이 아닌가 싶은데
\vspace{5mm}

척 봐도 공부하는 것에만 쾌감을 느끼게 생긴 학생들인지라 열심히 한다 그런 얘기는 할 필요가 없다.
다만 저기서 눈여겨보아야할 것은 바로 '정체성'이다.
\vspace{5mm}

명문학교가 일단 사기캐로 먹고 들어가는 이유가 여기에 있는데
"너는 명문학교 학생이다", "우수한 녀석이다"라는 자부심이 일단은 고민거리를 확 덜어준다는 것
그리고 라이벌들과 경쟁구도를 유지하기 때문에 싱크로나이징을 계속 할 수 있다는 것.
덤으로 말하면 저기에서 좋은 교재 나쁜 교재도 알아서 검증되기 때문에 쓸데없는 교재선택고민도 할 필요가 없다는 것.
\vspace{5mm}

그런데 이런 특징이 N수생들에게는 정반대로 나타난다.
자기가 N수한다고 생각하기 때문에 뭔가 자존심이 확 상해버리면서 의욕상실이 나타난다는 것.
혼자 공부하는 경우가 많거나, 학원에서의 라이벌 구도라는 건 그다지 강하지 않아서 싱크로나이징이 안 된다는 것.
\vspace{5mm}

그래서 명문학교에 들어간 것과 그렇지 않은 경우 격차는 상상 이상으로 커진다.
다시 말하지만 오해하지 말아야 할 건, 이건 명문학교에 들어간 애가 우수해서만이 아니다.
명문학교의 시스템 자체가 입시에 매우 최적화되어있기 때문에 - 물론 이걸 따라가지 못하는 학생도 많지만 -
저 시스템에서 시간을 보낸 우수한 학생이 저 시스템을 따라가지 않는 우수한 학생을 추월하는 일들이 생긴단 것이다.
\vspace{5mm}

내년 시험을 앞두는 학생들이라면 우선 저런 시스템을 따라갈 수 없을까, 아니면 따라가기 힘들면
저걸 본딴 자기만의 미니 시스템을 가족들의 협조, 혹은 친구들과의 계모임으로서 구현할 수 없을까 고민해보는 게 좋다.
벌써 지금 포기각이 나온 사람도 있고 열심히 달린 사람도 있겠지만 아마 작년에 내가 한 얘기 - "시간이 촉박하다"라는 걸 절감할 것이다.
그나마 학원에 간 사람은 학원에라도 안 갔으면 어찌되었을까, 도서관에 간 사람은 도서관에 안 갔으면 망했을 것이다라고 생각할 것이다.
\vspace{5mm}

공부가 안 되신다거나 지금 다 때려치고 싶다는 분들은 저 영상 한 30분동안 보시길.
공부에 의욕을 주는 영상이나 그림 따로 모아서 게시할까 모으고 있는데 저 영상도 그 중 하나임.
\vspace{5mm}

+ 물론 저거 방송한다니까 좀 과장한 것도 있을 것이다. 방송용 이미지는 따로 있는 법이니까.
\vspace{5mm}

+ 그리고 명문학교일수록 노는 것도 잘 논다. 공부만 한다고 보는 건 무리 - 나만 하더라도 고딩 때 가장 많이 논 것 같다(...)
\vspace{5mm}



\section{[학습공학 011] 생략과 속도}
\href{https://www.kockoc.com/Apoc/474136}{2015.11.10}

\vspace{5mm}

성적이 주로 떨어지는 시기는 중학교 3학년부터 고등학교 1학년 사이다.
단지 고등학교 과정이 힘들어져서 그러는 건가라고 의심할 수 있긴 한데 사실 생각해보면 중학교 과정도 그리 만만치많은 않다.
\vspace{5mm}

어째서 이런 일이 벌어지는 걸까.
\vspace{5mm}

그거 바로 \textbf{"거품"}
\vspace{5mm}

중학생들이 성적을 올릴 때에 받는 유혹은 무조건 '빨리' '많이' 푸는 것이다.
많이 푸는 건 나무랄 데가 없다. 그런데 문제는 '빨리' 푸는 과정에서 우유에 물타는 일이 벌어진다는 것이다.
중학교 수학 출제의 맹점은 \textbf{"논리적 조건을 꼼꼼히 따지거나 개념을 음미하지 않아도 패턴만 암기하면 점수가 잘 나온다"}는 것이다.
반대로 개념을 꼼꼼히 이해하고 논리적인 조건을 다 따지면서 문제를 푸는 경우에는 뒤로 밀릴 가능성이 높다.
\vspace{5mm}

자, 이런 친구들이 고등학교 진학 후 어떻게 맛이 갈지 뻔한 것이다.
\vspace{5mm}

고교수학에서 가장 중요한 건 꼼꼼한 논리다
\vspace{5mm}

\textbf{ABCDEFGHIJKLMNOPQRSTUVWXYZ를 전부 빠짐없이 연상해나가면서 풀어야한다.}
\textbf{사실 이렇게 해야만 나중에 4점짜리 킬러도 문제없이 대처할 수 있다.}
\vspace{5mm}

\textbf{그런데 대다수 학생들은 A F O T W Z 이런 식으로 생략해나가면서 스피드업을 한다.}
\textbf{이래서 단축되는 시간은 사실 그리 쓸만하지 않다. 왜냐면 저런 생략 때문에 놓치는 논점도 많고 실수도 많이 하기 때문이다.}
\vspace{5mm}

사실 고교수학은 생략은 독이다. 오히려 검증체제를 갖춰도 힘든 판에 -
가령 계산실수를 막으려면 효율적인 검산 시스템이 있어야 한다. 숫자 계산을 할 때에도 식 뿐만 아니라 수평선까지 동원하는 경우도 있다.
하나라도 틀리는 경우면 수년간의 공부가 허사로 돌아가기 때문이다.
\vspace{5mm}

자, 그런데 학원에서는 어떻게 가르치지? 저렇게 꼼꼼하게 논리적으로 다 밟으라고 가르칠 수 없다.
한정된 시간에 요점만 전달하는 축사 시스템이기 때문이다.
그럼 인강은 어떤가, 이것도 거의 마찬가지이다. 물론 떠먹여주며 꼼꼼히 가르치는 경우도 없지는 않다.
하지만 꼼꼼히 따지고 논리적인 단계를 밟아나가는 것
교과서에 ABCDE 가 쓰여져있더라도 AA'A''BB'B''CC'C''DD'D''EE'E'' 이렇게 더 심화시켜보는 '습관'이 중요한 것이다.
강사가 떠먹여줘보았자 한계가 있다. 나중에는 자기가 할 줄 몰라서 끝가지 강의에 의존하는 것이다.
\vspace{5mm}

이런 점에서는 야매 교재는 절대 볼 게 아니란 것이다.
야매 교재가 좋은 건 하나다. 출제 패턴에 참 근접하게 써놓았다는 것
그래서 당장은 점수가 잘 나오는데 여기서 익힌 방식으로는 절대 꼼꼼하게 갈 수가 없다.
그래도 점수가 잘 나온 경우? 수저가 금속이거나 다른 사교육을 받는 등 그래도 기본이 되어있는 케이스다.
내가 아는 한 중하위권 이하가 야매교재를 보는 경우는 올라가기 참 힘들다.
더 웃긴 건 \textbf{본인들이 공부를 안 해서 야매교재가 나쁜 걸 모른다(...) 그리고 그렇게 인생을 망친다.}
\vspace{5mm}

수학은 절대 서두르면 안 된다. 세밀화를 그리듯이 꼼꼼히 학습하고 풀기 싫은 것도 풀고 그렇게 거북이 방식으로 가는 게 가장 빠르다.
하지만 중하위권일수록 마음이 급해서 빨리 가려한다. 그래서 결국 더 늦게 가버리는 것이다.
세밀화 방식으로 꼼꼼히 논리적으로 가는 사람, A$\sim$Z를 AA'A''A'''A''''A''''.... A'''''''''''''''''''''''''' $\sim$ ZZ'Z''Z''Z''''Z''''.....Z'''''''''''''''''''' 까지 학습하면
그런 야매교재에서 선심쓰는 척 하는 스킬이나 꼼수가 얼마나 무의미한지 알게 될 것이며
괜히 교과서 강조하는 게 아니라는 것도 알게 된다.
\vspace{5mm}

그런데 상담이라는 것을 해봐도 느끼지만 수험생들 다수는 참 서두른다. 뭔가 지름길이 있다고 착각해대고 하는 것이다.
그래서 서둘러서 결과가 좋은 케이스는? 유감스럽지만 단 한건도 없다.
그렇다고 1년이란 시간을 알차게 쓰나. 어림잡아 5개월이라도 공부하면 다행일 것이다.
교재나 인강 좋은 것 고른다고 하지만 정작 쇼핑해놓고 제대로 듣는 케이스도 별로 없는 것 같다.
결국 문제는 '편한 길로만 빨리 가려는 것' 때문에 결국 수년간 정말 '백수'처럼 편안히 지내는 것이다.
\vspace{5mm}

비단 수학 뿐만 아니라 다른 분야도 그렇다.
어떤 일이든 결국 잘 하기 위해선 사전에 정말 꼼꼼히 준비해야 한다는 걸 느낀다.
주어진 절차나 매뉴얼을 하나하나 꼼꼼히 지키면 실패할 가능성은 기하급수적으로 줄어든다.
뭔 일이 생긴다 하면 그건 결국 집단적으로는 인재, 개인적으로는 내 부주의, 준비부족, 그리고 자만심 때문인 걸로 판명난다.
\vspace{5mm}

수학을 잘 한다는 천재들에 대해서 말이 많다.
문제지를 적게 풀고 놀 건 다 놀면서도 성적이 높은 괴수들.
그런 친구들 절대 성격 급하지가 않다. 여유롭게 호기 부리면서도 스나이핑을 정말 잘 한다.
그럼 스나이핑을 잘 하는 건 유전자가 좋아서? 그럴지도 모르지만 가장 중요한 건 '그런 가정환경'에서 자랐기 때문이다.
기본바탕이 잘 되어있는 교육을 받은 친구들이 수학을 잘 한다라면 그렇지 못 한 환경에 자란 친구들로서는 서운하기도 할 것이다.
하지만 대안은 간단하다, 그런 환경에서 못 자랐다면 본인이 그런 환경을 만들면 된다.
\vspace{5mm}

스피드업을 하기 위해선 '생략되는 게 단 하나도 없어야 한다'라는 조건을 충족시켜야 한다.
\textbf{만약 하나라도 빼먹는다, 절대 스피드업을 하면 안 된다.} 그 순간 인생의 재앙이 시작되는 것이다.
나사 하나라도 잘 안 조여진 우주왕복선을 타는 미친 놈은 없을 것이다. 그런데 우리는 우리 삶에 너무 자만해 나사를 대충 조인다.
아무리 수학문제가 답이 다 맞더라도 본인의 풀이과정에 핵심적인 게 누락된다면 그게 결국 발목을 잡게 되어있다.
이런 잘못된 습관을 학습하는 시기가 바로 중2$\sim$3 때이고 그게 고2$\sim$고3까지의 장기간 침체로 이어지는 게 아닌가 싶다.
\vspace{5mm}




\section{[학습공학 012] 도그마틱}
\href{https://www.kockoc.com/Apoc/495293}{2015.11.17}

\vspace{5mm}

제대로 된 개념서를 보아야한다는 이유는 다른 게 아니다.
만약 A라는 개념을 다룬 킬러문제가 있다면, 출제자는 어디까지 내려고 할까?
A라는 개념을 그대로 내진 않을 것이다. 그럼 너무나도 쉬워진다.
\vspace{5mm}

문제에는 A2나 A3으로 변형해 제시한다. 그건 그래프일수도, 식일수도 있고
혹은 B나 C라는 조건을 정리해보았을 때에만 비로소 A라는 개념이 쓰이는 것임을 알 수 잇다.
\vspace{5mm}

자, 그런데 A라는 개념이 쓰이는 걸 학생이 알아도 그대로 풀게 할까?
그건 아니다.
교과서나 개념서에서는 A라는 개념은
\vspace{5mm}

A의 정의 A'
A의 배경 A"
A의 성질 A'''
A의 공식 A''''
A의 지엽논점 XYZ
\vspace{5mm}

로 제시되어있다. 당연히 출제자는 A를 그대로 내지 않고 저기서 A'와 XYZ를 조합해서 풀 수 있게 낸다.
\vspace{5mm}

자 그럼 이것이 새로운 것인가? 어디서는 이게 무슨 특효약인 줄 알려져있는 모양인데 천만의 말씀이다.
저 방법은 우리나라 고시 2차 시험에서 답안작성할 때 쓰이는 방법이다.
그럼 그 방법을 우리나라에서 개발했나?
아니지. 우리나라 학문은 대놓고 일본 것을 베껴온 것인데 무슨(궁금해서 일본 책도 수입해보았는데 일본 책이 더 잘 나왔더라)
그럼 쪽발이들은 자기들이 다 개발했나?
아니지, 얘들이 수입한 건 독일학문이다. 저런 식의 사고방식이 바로 Dogmatik이라고 하는 것이더라.
그럼 독일 애들은 오딘 신이게 받았겠나.
칸트나 헤겔이나 그 제자들이 발전시켰겠지만 그 이전에 그리스-로마 학문을 발전시킨 결과지.
\vspace{5mm}

이런 것 야매로 해서 돈버는 병신들은 별 신경도 쓰긴 싫다(사실 언급할 가치가 있나)
저게 뭐가 대단한 것인양 숭배하고 찬양하는 광신도들이 있는 것 같은데, 이게 뭐 인터넷 시대의 원시부족 추장놀이도 아니고.
하지만 자기들이 공부하는 것이 어디서 비롯되었나 그건 치열하게 추적하고 근원을 파악해야 하는데
이 놈의 나라는 외국 것을 그냥 무단으로 베껴다가 자기가 개발한 것인양 얘기하면서 교주놀이하는 병신들이 너무나도 많다.
\vspace{5mm}

그럼 수학을 잘 하고 싶은 길은 저런 도그마틱을 철저히 익히고 암기하고 계속 개발시키는 것이다.
내가 야매교재로 분류하는 책들의 문제는, 도그마틱을 구현해도 참 어설프게 구현했다는 것.
그리고 그런 책들을 칭찬하는 친구들의 수험기를 보면 본 책들의 수준이 참 비슷하다.
\vspace{5mm}

도그마틱 수학의 대표가 뭔지 아나? 바로 정석이다.
사실 정석이야말로 '이게 업데이트 되지 않아서' 그렇지 저런 도그마틱 체제를 정말 잘 구현한 책이다.
그런데 정석이 참조(?)했던 것으로 보이는 차트식 수학과 비교해보면 그렇다. 정석은 변화가 없는데 차트식은 정말 진화를 하더란 것.
아무튼 우리나라 수학문제집은 사실 거의 다 저 '뭔가 정체한' 정석에서 비롯된 것이다.
\vspace{5mm}

그런데 문제는 정석을 참조해서 만들어진 것임은 좋은데, 도그마틱의 정신(?)이 망각되기 시작한 거지.
수험사이트의 수학고수들? 걔들이 머리가 좋아서? 웃기는 소리다.
머리가 좋다는 것이 어떤 개념인지는 죽을 날 얼마 안 남은 나도 지금까지도 모른다.
\textbf{다만 수학 잘 한다는 애들은 저 도그마틱이 머리에 잘 박혀있고 그게 업데이트가 되고 있단것}이다.
이것이야말로 진짜로 비밀인 것 같은데?
\vspace{5mm}

궁금하면 올해 수능부터 시작해 역대 수능 킬러문제의 해설
그리고 수리논술 어려운 문제라는 것을 시중 개념서와 교과서의 저런 정의, 성질, 공식과 대응시켜보길 바란다.
그래야 왜 '교과서가 강조되는구나'라는 걸 알 수 있을 것이다.
\vspace{5mm}

이게 웃긴 게 도그마틱을 달달 외운 녀석들은 수학 잘 한다라고 거드름피우면서 정작 자기들 교재에는 도그마틱을 반영해놓지 않았다.
불성실한 건지 올챙이 시절을 모르는 건지 아니면 영업기밀을 누설하기 싫어서인지는 모르겠지만
저런 식의 장사는 하기 싫은 나는 걍 말하지. 잡소리 집어치우고 개념을 처음부터 끝까지 달달 외우라고.
그걸 순서대로 구현할 수 있으면 어려운 문제를 풀 수 있는 것이라고
\vspace{5mm}

그리고 이런 도그마틱 방법론은 수학 뿐만 아니라 서양에서 비롯된 모든 학문에 다 쓰이는 방법론이라고.
심지어 국어, 영어, 탐구도 마찬가지라는 것.
이럼 쓸데없는 돈을 쓸 필요는 없다.
저기까지 가야 아, 자기가 얼마나 병신같이 신비주의적 마케팅에 사로잡혔나 깨닫고 이불킥 성층권까지 시원하게 차올리겠지.
그리고 도그마틱에 맛들이면 그 다음부터 서양의 온갖 학문을 섭렵하기 시작할 것이다.
\vspace{5mm}

물론 이런 방법론은 남들에게 알려줄 건 없고(어차피 이 글 읽은 애들도 새가슴이라서 대부분 또 휘둘리겠지)
그냥 본인들이 열심히 해서 성과 보시길 바란다.
최소한 3년간 내가 보는 수학 사교육은 갓가원을 못 따라잡았거든.
\vspace{5mm}




\section{[학습공학 013] 논리적인 사고란?}
\href{https://www.kockoc.com/Apoc/504605}{2015.11.22}

\vspace{5mm}
\begin{enumerate}
    \item 대상을 기본 단위들로 쪼갠다.
    \item 기본 단위에 순서를 매긴다.
    \item 그 순서대로 생각하고 실천한다.
\end{enumerate}
\vspace{5mm}

고교과정에서 말하는 논리적인 사고는 이걸로 끝.
연역추론이나 귀납추론이라는 건 하나의 기술에 불과할 뿐이다.
\vspace{5mm}

그럼 어느 단위까지 쪼개느냐.
그야 우리가 익숙한 P ⇒ Q 사고가 가능한 기본명제나 기본작업단위까지 쪼개면 된다.
실제로 논리적으로 사고하는 사람들이 논리학에 익숙하느냐 하면 그건 아니다.
그런데도 그 사람들이 합리적인 결론, 결과를 도출할 수 있는 건 간단하다. "순서대로" 하기 때문이다.
\vspace{5mm}

혼자 공부하지는 못 하는데 인강, 학원을 따라가거나 집단으로 공부하면 되는 이유?
그건 간단하다. 대부분의 강의는 '순서'대로 진행되기 때문이다(순서를 지키지 못 하면 전달이 현격히 되지 않는다)
사람들이 공동으로 일을 하면 서로 호흡을 맞춰나가야기 때문에 역시 순서를 지켜야 한다.
그러나 혼자 공부하는 경우는 지켜야 할 순서가 없는 경우가 많다. 자기 멋대로 해도 되기 때문이다.
\vspace{5mm}

책은 논리적이다.
페이지든 목차든 다 번호가 붙어있다, 내용도 '순서'대로 기술되어 있다.
물론 그 순서가 반드시 정답은 아니다. 재배열하는 경우가 좋은 경우도 있다. 그럴 때에 강의가 도움이 된다.
그러나 가장 좋은 건 본인이 재배열해보는 것이다.
\vspace{5mm}

중학교 때까지는 왜 스피디한 풀이가 가능했는데 고등학교부터는 안 먹히느냐?
중학교 때까지는 순서가 그리 필요없다. 모두가 P ⇒ Q 의 원샷원킬로 끝나기 때문이다.
그러나 고등학교의 학습은 A ⇒ B ⇒ ... ⇒ S  이런 식으로 다단계 순서를 정확히 밟아줘야한다.
그래서 학생들에게 텍스트를 천천히 읽으라고 하는 것이며 자잘한 꼼수나 스킬은 일단 무시하라고 하는 것이다.
본인들이 순서를 지키는 것조차 못 하는데 무슨 입시고수의 스킬을 배운다? 주화입마에 빠져버리고야 만다.
\vspace{5mm}

형식과 개요 모두 순서에 속한다.
\vspace{5mm}

어떤 교재를 봐야하느냐는 건 당연히 자기에게 모자란 것을 보충해주는 것이 되겠지만,
개인의 특성을 제외하고 좋은 교재와 나쁜 교재를 구분하면
순서를 지키면서 순서대로 사고하는 법을 가르쳐주는 교재가 좋은 것이고
그렇지 않은 교재가 나쁜 교재인 것이다.
\vspace{5mm}

재밌는 건 수험가에서는 후자가 선호되고, 그래서 망한 학생들도 적지않게 보인다는 것이고
실제로 저자들을 추적해보면 공부라는 것을 하는지 의심스럽거니와 기본 사고나 성격조차 의심되는 경우도 많다.
\vspace{5mm}

교과서를 보라는 이유는 별것이 아니다. 올바른 순서대로 쓰여져 있기 때문이다.
자꾸만 뭘 봐야하느냐 물어보는 사람들이 있는데 그냥 '검증된' 것을 보면 된다.
검증된 것을 보지 않고 과대광고된 교재 보는 사람들이 정말 성과들이 좋았나 확인해보시길, 그딴 것은 없기 때문이다.
\vspace{5mm}

머리가 좋냐 안 좋냐는 건 그다지 중요한 문제는 아니다.
그러나 대화해보면 이 사람이 순서를 지키느냐. 즉 예의를 지키며 형식적인 것을 준수하느냐는 건 정말 차이가 난다.
\vspace{5mm}







\section{[학습공학 014] 지성 vs 야성}
\href{https://www.kockoc.com/Apoc/511886}{2015.11.26}

\vspace{5mm}

섬세한 대신 약하게 준비하느냐
아니면 섬세함을 버리고 강하게 준비하느냐 차인데
승률은 뒤의 것이 더 나아보입니다.
\vspace{5mm}

섬세한 플레이가 유효한 건 명중률 - 즉 적중률이 높을 때를 말하는 건데
최근 3년간 출제를 보면 그런 걸 기대하기 매우 힘듭니다.
다시 말해서 xx 강의나 xx 모의고사만 보니까 잘 나왔다라고 말할 수 없다는 것임요.
\vspace{5mm}

그냥 아무 생각없이 \textbf{구할 수 있는 교재를 다 구해서 풀어대는 게 낫다}는 이야기입니다.
물론 기본 - 유형 - 기출 - 탈패턴 - 다시 교과서로 돌아가기 - 논술급도 손대보기는 당연한 이야기이지요.
\vspace{5mm}

사실 입시정보라는 건 수능이 잘 나와야 의미가 있지, 그 외에는 쓸모가 없죠.
아울러 입시강사들이나 교재저자들은 입시 결과를 책임지지 않습니다.
그렇다면 입시정보나 특정 강사 및 교재에 치중하는 건 바람직하지 않은데도 수험생들은 어리석은 선택을 하는 경향을 보입니다.
저기 들일 돈이 있으면 차라리 이 시점에 운동하는 게 나을 겁니다. \textbf{6월부터는 다시 체력에 허덕이니까요.}
\vspace{5mm}

이 이야기는 그럼 극단적으로
A 강사 커리를 따라가는 것과
그냥 교재 ㄱㄴㄷㄹㅁㅂㅅ를 풀어대는 것 중 후자가 낫다는 이야기냐 할건데
그렇다라고 얘기할 수 있단 것임.
\vspace{5mm}

강사가 말하는 것 대부분은 개념서, 기출 해설에 있습니다.
그 강사만의 특유한 스킬이나 노하우는 사실 여름 정도 되어서 강의 평가 올라올 때 그거 따라들으면 되는 것이죠.
오답에 얻어맞더라도 문풀을 해서 어느 정도 경험치가 쌓인 상태라면 그 이후에 인강 흡수율도 좋지만,
반대로 인강만 따라들었다가 나중에 문풀로 가는 경우 인강을 안 들은 것과 큰 차이가 없어서 다른 인강을 찾습니다.
\vspace{5mm}

대략 관찰해보면 왜 수험에 빠삭한 사람들이 n수를 하느냐.
\vspace{5mm}

\textbf{지성은 과다, 야성이 부족하기 때문이다.}
\vspace{5mm}

라고 정리할 수 있습니다. 물론 야성이 너무 치우쳐져있고 지성이 부족한 경우도 없진 않습니다만.
이건 콕콕 대다수가 다 해당되는 얘기죠.
\vspace{5mm}

야성이 부족하지 않은 사람들은 \textbf{여학생들}이지요.
다시 말해 여자들은 야성이 넘치는데 남자들은 야성이 부족합니다.
\vspace{5mm}

이게 왜 그런가 싶냐 들어가면 진화심리학까지 갈지 모르겠지만 적어보겠음.
\begin{itemize}
    \item 여자는 남자보다 자기 현실을 바꾸고 싶어하는 절박한 심정이 강하다.
    \item 남자들보다 터프하다(* 와타나베 준이치의 칼럼을 읽어보면 여자들이 얼마나 둔감력이 강한지 나와있죠)
    \item 애까지 낳으면서 그 애를 키울 준비도 되어있다.
    \item 여자들의 사랑은 그리 로맨틱하지가 않다
    \item 여자들은 한번 믿으면 끝까지 간다.
\end{itemize}
\vspace{5mm}

남자가 야성이 넘치고 여자가 야성이 없다라는 건 편견이죠. 수험 뿐만 아니라 실제 전분야 보면 여자들이 더 강합니다(...)
그래서 여자들의 수험은 오히려 적절한 조언만 뒤따르고 체력만 뒷받침되면 성공할 확률이 높습니다.
\vspace{5mm}

그런데 남자는 안 그러죠. 자기들이 야성이 넘치고 씩씩하다고 착각을 하는데 천만에$\sim$
섬세하고 우유부단한 케이스가 대부분입니다. 남자이기 때문에 멋지게 보여야한다는 강박 때문에 터무니없는 선택도 하죠.
거기다가 여자들에 비하면 덜 절박감도 느껴서 그런가 그렇게 처절하지 않으며 일관된 공부를 하기가 어렵습니다.
\vspace{5mm}

콕콕에 오면 흔히 5수는 기본이다라는 말이 웃고 넘길 게 아닙니다.
이건 수험사이트가 전반적으로 다 그렇습니다. 수험적 귀납법이 성립하는 이유는 다들 너무 섬세하기 때문입니다.
거기다가 온갖 상술에 휘말리면서 나중에 입시는 잊고 교재와 강의 포트폴리오 짜기에 여념이 없는 광경을 목격하게 되죠.
그리고 그 수험멘토라는 사람들이든 그 사람들이 하는 조언은 전혀 마초적이지가 않습니다. 즉, 야성 결핍증에 시달린다는 얘기죠.
\vspace{5mm}

실패할 것이냐 성공할 것이냐가 아니라 일단 '닥돌'하는 걸 개인적으로 권하겠습니다.
탐구과목 선택이라든가 하는 건 신경써야할지 몰라도, 그 외는 그냥 \textbf{닥치고 문제푸는 게 답입니다.}
그래도 모자라는 게 느껴지 때에나 강의나 Q/A 요청하는 것이지, 그것도 안 된 상태애서 재어보았자 의미가 없죠.
어떤 운동이든 해야 몸이 좋아지는 거지, 1년째 계속 상담만 받기만 한다면 근육이 생길리가 있겠습니까.
\vspace{5mm}

지금 부족한 건 지성이 아니라 야성입니다.
\vspace{5mm}


\section{[학습공학 015] 시스템 올라타기}
\href{https://www.kockoc.com/Apoc/523683}{2015.12.02}

\vspace{5mm}

KTX를 타면 그 안에서 맛폰으로 영화를 보든 책을 읽든 연인과 키스를 하던
그 관성계 내부에서는 가속, 감속 구간을 제외하고는 달린다는 느낌이 들지 않을 것이다.
우리는 돈을 내고 '교통 시스템'을 산다. 그리고 우리의 시공간은 그 시스템에 얹혀 목표한 상태로 도달한다.
피트니스 클럽의 밀쓰레드도 마찬가지다.
일단 가동시킨 이상 안 달릴 수가 없다. 운동이 강제되는 시스템이라는 이야기다.
\vspace{5mm}

학원 (현강)의 장점도 강제성이다. 인강은 본인이 중단시킬 수가 잇다.
그러나 현강은 내 의지가 작용하지 않기 때문에 강제로 따라가야하면서 공부에 방해되는 요소가 억제된다.
얼핏 보면 자유로운 인강이 좋을 것 같지만 그렇지 않다는 역설이 되겠다. '자유'가 공부에 도움이 되지 않는다는 좋은 사례일 것이다.
\vspace{5mm}

우리는 물론 노력을 해야하지만 점점 중요해지는 건 "어떤 시스템"에 올라타 있느냐는 것이다.
과거에는 상품과 서비스를 구매했다. 그러나 지금 우리는 시스템들을 구입하여 그 속에서 살아가고 있다.
그리고 우리가 고민하는 건 시스템의 효율성과 부작용 정도일 것이다.
시스템이 마음에 들지 않으면 자기가 만들기도 한다. 그러나 자기가 시스템을 만들고 제어할 능력이 없다면 실패해버린다.
흔한 독학의 실패 이유다. 학원 시스템이 마음에 들지 않으면 개인의 학습 시스템을 예비해놓아야하는데 그렇지 않은 것이다.
\vspace{5mm}

타야 할 시스템과 타지 말아야 할 시스템이 있다.
어머니들이 치맛바람을 날리면서 아이들 교육에 극성인 과정도 현란한 시스템 교환이다.
비싼 돈을 들여서라도 자기 아이만큼은 좋은 시스템 위에 올려놓는다는 건 정말 유효한 전략이다.
분명 여자들은 남자들이 따라잡을 수 없는 감이라는 게 있다. 천성적으로 우월한 시스템을 알아보고 그걸 이용하고자 하는 감.
\vspace{5mm}

독서실에서는 공부가 안 되는 사람도 교실 자습 - 야자의 교실책상과 의자에서는 공부가 잘 되는 이유다.
여럿이서 하면 잘 되는 건 공부 뿐만 아니라 운동과 노래도 마찬가지이다.
여럿이 비슷한 일을 하면서 거기서 시스템이 만들어지기 때문이다.
개인 시스템보단 집단 시스템이 더 효율적이다. 우선 개인 자유가 줄어드니 변덕을 부리기 어렵다.
그리고 남들이 하는 대로 페이스를 따라가면 된다
\vspace{5mm}

이렇게 본다면 인강도 아프리카처럼 만약 실시간으로 채팅하는 시스템이면 현강보단 나을지도 모른다
(그러나 비용구조상 그러기 어렵지 않을까)
사람은 열악한 시스템보다 우월한 시스템에 더 빠져드는 경향이 있다.
개인이 공부하다가 인터넷이나 게임에 빠져드는 것도 그렇다. 인터넷과 게임은 매우 정교하고 우월한 시스템이다.
\vspace{5mm}





\section{[학습공학 016] 인강듣는 요령 제시}
\href{https://www.kockoc.com/Apoc/529777}{2015.12.06}

\vspace{5mm}

EBS만 해당
\vspace{5mm}
\begin{enumerate}
    \item EBS에서 동영상 파일과 음성파일을 다운로드받을 것
    \item 곰플레이어 등으로 배속 조절하며 보시는 시스템 만들 것
    \item 처음에 1.0$\sim$1.2 배속으로 들으면서 필기를 연습장이나 A4에 해놓을 것
    \item 문풀, 다른 문제를 푸실 것. 그리고 다른 개념서를 보아도 됨
    \item 2$\sim$3일 지난 후에 음성파일만 1.5$\sim$1.8배속으로 들으면서 필기를 교재에 단권화하거나 알아서 추릴 것.
\end{enumerate}
\vspace{5mm}

강의를 한번만 듣고 끝내는 경우가 많은데
많은 강의 들을 필요 없이 일단 자기가 선택한 강의를 여러번 배속수 늘려주면서 들어주는 게 효과가 좋습니다.
처음에는 동강으로 판서를 하고 흐름 잡되, 그 다음에는 음성강의를 병행하면서 책과 문제를 읽어주고
그러면서 배속수를 늘려나가는 게 핵심입니다.
\vspace{5mm}

일단 동영상강의의 경우는 입문을 위해서 쓸데없는 내용도 많이 들어가 있습니다. 이건 한번 보면 그 다음 버려야할 게 많죠.
거기서 필기한 것만 따로 추린 다음에, 그 다음 똑같은 강의를 음성강의로 빨리 들어야 뇌에서 이걸 흡수하기 시작합니다.
그 다음으로는 걸어다니거나 운동하면서 판서노트 없이 음성강의를 1.5 배속 이상 들어주면 됩니다.
\vspace{5mm}

그럼 최소 2$\sim$3회청 이상이 되죠. 이게 쓸데없긴... 오히려 가장 효율이 좋습니다.
1번 들으면 뭥미 하던 것이 반복해서 들으면 강사가 무얼 이야기하는 건지 이해가 되기 시작합니다.
내용이 숙지된 상태에서 헤드폰 끼고 걸어다니면서 들을 수 있다면 더욱 좋죠. 같은 내용이 반복되므로 숙달되는 것입니다.
배속수를 높이는 이유는 시간도 시간이지만 베르니케 중추가 활성화되어서 사고속도가 빨라지기 때문입니다.
\vspace{5mm}

다만 수학과 과학의 경우이거나 판서가 필요한 경우라면 아무래도 교재를 들고다녀야할 것인바인데
이 경우라면 1회청할 때 판서한 것을 경량화시켜 들고다니는 것도 권해볼만하겠지만
아무래도 수학의 경우라면 그냥 걸으면서 듣는 건 무리. 그러나 국어, 영어, 탐구라면 가능할 것입니다.
이 방법은 공짜로 파일을 다운로드가능하게 하는 EBS 강의가 용이하게 먹힙니다.
\vspace{5mm}

독학재수를 하는 분이고 돈이 없다면 이 방법대로 가면 됩니다.





\section{[학습공학 017] 추상과 구체}
\href{https://www.kockoc.com/Apoc/548030}{2015.12.16}

\vspace{5mm}

생각은 크게는 \textbf{추상적인 것(이론, 논리)}와 \textbf{구체적인 것(사례, 이미지)}로 구분.
\vspace{5mm}

그래서 여기서 변환이 일어나는데
\vspace{5mm}

\begin{itemize}
    \item \textbf{ⓐ 추상 $\rightarrow$ 추상}
    \item \textbf{ⓑ 추상 $\rightarrow$ 구체}
    \item \textbf{ⓒ 구체 $\rightarrow$ 추상}
    \item \textbf{ⓓ 구체 $\rightarrow$ 구체}
\end{itemize}
\vspace{5mm}

천재들이 잘하는 건 ⓐ와 ⓓ입니다.
일반인들이 알 수 없는 기호를 가지고 하루종일 골몰하는 것이 바로 ⓐ
소위 직감과 육감으로 승부하는 것이 바로 ⓓ입니다.
ⓐ의 경우는 사례나 이미지 없이 바로 추상적인 이론과 논리에서 다른 추상적인 이론과 논리로 이어집니다.
이 정도면 평범한 인간이 아닌, 전파계이거나 그 분이 오셨네 수준. 흔한(?) 대학수학교재나 철학서를 보시면 됩니다.
ⓓ의 경우는 논리나 이론으로 정리하지 않고 바로 동물적인 감각으로 승부에 임하는 투기꾼이나 도박사나 달인.
본인도 자기가 왜 잘 하는지 모르지만 일단 몰입하기 시작하면 엄청난 결과를 만들어내죠
\vspace{5mm}

일반인들이 택할 수 있는 건 ⓒ와 ⓑ이죠.
제가 칼럼에서 말하고자 하는 기준은 일반인들입니다. 특히 강조하는 건 ⓒ와 ⓑ입니다.
왜 ⓒ부터 강조하느냐, 그거야 일반인들은 추상적인 것들이 박혀있지 않기 때문입니다.
어떤 과목이든 양치기를 하라는 건, "구체적인 문풀"을 하면서 본인들의 추상적인 이데아를 만들어보라는 것입니다.
가령 수학도 개념과 문풀로 승부해서 5000문제 풀다보면 본인만의 '감성'과 '이론'이 뒤섞인 추상적 세계가 만들어집니다.
그럼 그 추상적 객체를 본인 스스로 '사례화'시켜보거나 '이미지'를 부여하면서 구체화시키는 ⓑ 과정으로 가면 응용력이라는 게 생깁니다.
패턴화라는 것 자체가 바로 구체적인 경험을 쌓으라는 이야기고
탈패턴화라는 건 그 구체적 경험들을 반복해 그 중 공통분모만 찾아내 체화시켜 추상적 세계를 완성하라는 겁니다.
\vspace{5mm}

수학에 있어서 이게 왜 중요하냐.
초심자들은 해설을 보고 어떻게 하라 '지시'받길 원할 겁니다. 그리고 일본 것을 무단계수한 책만 보고 이해가 안 가서 헤매겠죠.
하지만 양치기를 해서 ⓒ, ⓑ를 완성한 사람들은 자기만의 추상적인 수학 메뉴로써 '스스로 풀이'를 만들어냅니다.
수학고수들은 해설 같은 걸 일부 빼고는 다 잊어버립니다. 스스로 성질과 정리를 유도하고 풀이도 그 때 그 때 만들어내거든요.
추상적 세계가 완성되었기 때문에 그것으로서 \textbf{구체적 사례나 이미지를 제조할 수 있는 것}입니다.
\vspace{5mm}

일부 교재저자들은 이런 걸 모르고 마치 수학에 비법이 있는양 짜깁기서를 내지만 그건 분수도 모르는 걸 넘어 학습이 뭔지 연구 안 해본 것이죠.
수학을 잘 하려면 본인만의 추상적인 수학세계가 완성되어있어야합니다.
수험수학은 ⓒ와 ⓑ에 숙달되면 극복됩니다.
만약 본인이 뛰어난 학문의 대가로 올라가고 싶다면 ⓐ를 잘 해야겠죠.
\vspace{5mm}

그럼 ⓓ는? 학문의 영역이 아니죠.
본인이 기술이나 장사로 성공하고 싶다거나 승부사로 가고 싶은 경우에 해당하는데 이건 입시에는 안 맞습니다.
중학 수학의 경우 ⓓ로 쇼부보는 애들도 많아요. 보통 '머리가 좋다'라는 안이한 표현으로 설명되나 고등학교에 가서 죽쑵니다.
중학 수학은 구체적 사례나 이미지로도 딜할 수 있지만 고등 수학은 반드시 추상적 세계가 구축되어 있어야 합니다.
이과 수학의 경우는 잘 하느냐 못 하느냐 그게 문제가 아닙니다.
학생 당사자의 머리가 수리적 모델을 구동할 수 있는 추상적인 수리적 세계가 얼마나 잘 깔려있고 진화해나가느냐가 관건입니다.
\vspace{5mm}

똑똑하거나 공부가 뭔지 아는 사람이라면 이 글이 매우 중요하다는 걸 눈치깠을 것입니다.
사실 이 내용만 안다면 이과수학이 어렵다거나 꼭 가애받아야한다거나 하는 것도 터무니없는 소리임을 깨달으실 겁니다.
\vspace{5mm}

+ 여담 +
\vspace{5mm}

기하와 벡터가 힘든 이유는 죄다 ⓓ로 접근해서입니다(...)
이과 기하와 벡터는 사실 추상과 구체를 정말 왔다리갔다리 잘 해야하온데
그림이 주어지니까 중학교 때처럼 보조선 잘 그으면 구체적 이미지로 풀린다하는데 그건 매우 힘듭니다.
주어진 도형들을 명제로 환원(추상화)하고 그 다음 다시 그려야(재구체화)해야합니다.
그런데 보통 공간도형의 명제나 성질을 대충 읽고 넘어가니까 그게 안 되죠.
그 상태에서 xxx 강의만 들으면 된다거나 xxx 교재만 보면 된다고만 하니 무한 루프
\vspace{5mm}






\section{[학습공학 018] 맹목도}
\href{https://www.kockoc.com/Apoc/550110}{2015.12.17}

\vspace{5mm}

안 좋은 풍토가 이것이죠.
\vspace{5mm}

강사를 교주로 믿는다. 아울러 인쇄본들이 형편없다.
\vspace{5mm}

지식도 명시지와 암묵지가 있습니다. 명시지는 기록가능한 것, 암묵지는 기록이 어려운 걸로 생각하시면 됩니다.
책이 아닌 '강의'라는 측면에서 존중받으려면 암묵지여야하죠
A라는 강사가 실제로 사람을 수백명 죽인 암살자라면 그의 '살인론'은 직접 들을 가치가 있겠죠(그리고 나란히 경찰서에 간다거나)
실제로 픽업아티스트(...) 분야라는 어둠의 사교육분야에서는 다수의 여자들과 사귀었다는(...) 강사들이 그걸로 장사질을 한다고 알고 있습니다
\vspace{5mm}

그런데 수험의 경우 그 강사"만"의 독자적인 컨텐츠가 있느냐 하면 그건 아닙니다.
본인이 정말 그 분야 박사급이라서 교과서도 다시 써내고 하는 강사는 다른 시험에는 있습니다만,
적어도 제가 아는 한 수능에서는 찾기 어렵습니다. 탐구라면 예외일지 모르겠지만요(탐구는 지금 교과서만으로 커버가 안 되기 때문입니다)
\vspace{5mm}

강의를 들을 때에는 \textbf{강사 그 사람의 인격이나 개성을 무시해야합니다.}
과학적으로 강의를 듣는 법이라는 건,
그 강의가 좋은 건 알겠는데, \textbf{도대체 강사가 어떤 소스를 근거로 뭘 연구해서 그렇게 전달해볼까 '고민'해보는 겁니다}.
그 강사가 태어날 때부터 그런 것을 알고 있는 것은 아닐테고 그들도 공부해서 전달가능한 상태로 만들어서 썰을 풀어보는 것이거든요.
무엇보다 만약 "내가 똑같은 강의를 하기 위해선 뭐가 필요할까"라고 스스로 고민해보시면 됩니다만.
\vspace{5mm}

사실 수학의 경우만 하더라도 제가 보기엔 정파와 사파가 있사온데
정파라는 건 수학, 수학교육론까지 전공하면서 본인이 수험을 넘어서 아예 학문을 완성시킨 케이스라고 하겠고
사파라는 건 그냥 문제를 많이 풀고 꼼수를 알아내서 그걸 그럴싸하게 가공해 소개시키는 경우라고 하겠는데
당연히 사파강의는 딱 한번 정도 들어주면 그 다음에는 들어줄 이유가 없습니다.
굳이 듣고 싶으면 정파 강의 정도인데 이건 제가 아는 한 '거의' 없고 시중 참고서에서 직접적으로 드러난 경우가 없습니다.
\vspace{5mm}

차이가 있다면 사파는 "$\sim$ 하게 푼다"라는 것만 강조하지만 \textbf{정파는 "왜 $\sim$ 풀이가 나오고 출제자는 $\sim$ 한 문제를 낼까"라는 걸 고민}시키지요.
\vspace{5mm}

물론 수험은 수험일 뿐입니다. 학문을 하면 수험을 할 수가 없죠.
그런데 어떤 통수가 오더라도 혹은 상상을 초월하는 문제가 나오더라도 대비하려면 '학문'을 하고 있어야 합니다.
뿌리-줄기-잎 구조가 완성되어야 바람이 불더라도 날라가지 않지, 잎만 있으면 날라갈 것이고 줄기만 있으면 쓰러져버리지요.
그렇다면 강의를 골라듣는다는 건 저런 '정파' 강의 하나를 잘 들어서 "학문적으로 생각하는 자세"를 갖추는 것이 필수일지 모릅니다.
그런 강의는 첫강만 제대로 들어도 몸과 마음이 떨리고 새로 태어난 듯한 느낌이 오지요.
\textbf{학문을 한다는 게 참 즐거운 것이구나라는 느낌이 정말로 옵니다}.
그런 강의는 한번 들어도 다시 듣고싶어질 정도이고, 시중 전문서를 뒤져보아도 전문서가 그 강의를 못 따라갑니다.
\vspace{5mm}

그런데 이런 강의가 있냐하면 그건 아닙니다요.
저렇게 강의하려면 본인이 정말 20년 넘게 구르거나 아니면 정말 박사급은 넘어가야 하는데 그 정도가 있는지는 참 의문입니다.
이 이야기는 다시 말해서 현재로선 강의에 너무 의존할 필요는 없다, 그 이야기입니다.
강의 들으니까 괜찮던데요 하겠지만 끝까지 들으면 아실 겁니다. 그걸로는 절대 킬러를 대적할 뇌 상태까지 만들긴 힘들어요.
\vspace{5mm}

그럼 1년동안 상담한 결과 넣으면 전 사설 인강은 여름 이후에 문풀 아니면 들을 필요가 없다는 입장입니다.
\vspace{5mm}

까닭 말씀드리지요. 상담하면서 안 좋다고 하는 경우가 \textbf{100$\%$ 사설인강파였기 때문}입니다.
우선 이건 가슴 아픈 이야기인데 그 사람들의 진학실적이 좋은가. 아닙니다.
사설인강을 들은 사람들은 진학실적이 좋은 경우와 나쁜 경우가 골고루 배치되어있습니다.
님들이 바보가 아니라면 확률 단원에서 독립이 무엇인가 배웠을 테니 문제풀 때나 쓰지 말고 여기서 적용시켜보십시오.
그런데 문제는 사설인강파들은 \textbf{"생각을 할 줄 모른다"}라는 것입니다.
\vspace{5mm}

이게 꽤 흥미로워서 컵라면을 먹으면서 제가 좋아하는 타 강사 mp3를 듣다보니까(뭐 수능은 아닙니다) 어느 순간 깨달음이 오더군요.
애들이 강의를 듣는 걸 선호하는 이유는?
머리가 안 아프고 듣기 좋아서이지요. 책을 읽을 때는 불편하지만 강의를 보면 \textbf{기분이 좋아서입니다.}
그럼 \textbf{왜 기분이 좋지?}
\vspace{5mm}

언제 행복한지 묻는 문제와 똑같습니다. 간단하죠 - 행복한 건 우리가 그 때에는 \textbf{'생각'을 안 하고 느끼기만 하면 되기 때문입니다.}
그럼 강의를 들으면 왜 기분이 좋냐도 마찬가지겠죠. \textbf{'생각할 필요가 없기' 때문입니다.}
\vspace{5mm}

이걸 3개월 전에 추론하고 나서야
부모 돈 졸라 까먹으면서 사설강의 들었단 친구들이 왜 아는 것이 많으면서도 기본적인 사고가 안 되었으며
심지어 똑똑하다라고 알려진 친구들조차도 그것은 일종의 흉내냄에 불과한가하는 의문이 다 풀리더군요.
\textbf{생각은 '명제의 부정'을 말하는 겁니다. 부정은 괴로운 과정입니다. 부정을 하려면 느낌을 절단해야하거든요.}
\vspace{5mm}

만약 생각하는 게 행복한 일이라면 사람들은 어리석은 짓을 안 하겠죠. 매순간 생각을 할 테니까.
하지만 사람들은 생각하지 않습니다. 생각하는 건 괴로운 일이기 때문입니다.
\vspace{5mm}

자, 그럼 수학능력시험의 취지는?
더 물어보고 말 것도 없겠죠. 킬러문제라고 하는 말 정정합시다. \textbf{"생각하는 문제'}입니다.
2$\sim$3점은 별로 생각할 필요가 없는 문제들입니다만 4점은 무조건 생각을 해야합니다.
생각해야하는 문제를 킬러 문제라고 환치시키면서 xxx 강의나 xxx 교재 보면 된다라고 하던 거야말로 세뇌 아니던가요?
\vspace{5mm}

고액과외를 한다거나 대치동 강의 들으라 하겠요. 그런데 이건 수십년 전부터 지속되어왔던 것입니다.
그래서 그렇게 좋은 대학 간 사람들이 사회적으로도 출세했는가. 냉정히 따지면 그게 아니던데 말입니다.
면허 주는 의대라면 모르겠는데 이제 면허자들도 늘어난다면 자유경쟁은 치열해지겠고 '생각할 줄 아는 사람'이 살아남겠죠.
\vspace{5mm}

짧은 시간이 수험에 관한 전반적인 걸 리뷰하는 강의라면 괜찮습니다. 그런데 그건 EBS로도 충분합니다.
그게 아니라 정말 생각하는 법을 익히는 강의라면 신중히 잘 골라야합니다. 안 그러면 그 강사의 '노예'가 되면서 그 강사 좋다고만 하겠죠.
이게 소크라테스가 등장할 시절 소피스트들에게 변론술만 주입받는 데 거액을 지불했던 멍청한 시민과 뭐가 다른지 묻고 싶네요.
\vspace{5mm}

환경이 좋아야 성적이 좋다라는 씁쓸한 상관관계도 있습니다만.
질문을 많이 하고 그걸 답해보고 고민해보는 횟수가 많은 학생이 결국 잘 됩니다.
여기서 도입할 지표가 일종의 '맹목도'입니다.
생각이 없이 무조건 $\sim$ 하자 무조건 $\sim$ 들으면 된다 하는 수준인데
맹목도가 높은 사람은 정말 운빨을 따르는 경우고, 맹목도가 낮은 사람은 운이 나쁘더라도 어떻게든 좋은 결과를 냅니다.
하지만 맹목도가 낮은 사람은 50명 중에 1명도 될까말까하다는 게 제 판단입니다.
\vspace{5mm}

+ 뱀꼬리 1 +
심지어 픽업아티스트들조차 - 미국과 일본의 경우는 - 아주 충실한 교재를 냈죠(...)
그런데 한국은 뭐 터무니없는 강의로 팔아먹는 듯.
이 분야도 은근히 마니아(...)들이 있어서 보는데 결국 진화심리학과 사교술을 적당히 융합시킨 듯.
\vspace{5mm}

+ 뱀꼬리 2 +
최초 수능만점자는 질문이라는 걸 아주 잘 했던 걸로 기억합니다. 제가 들었던 질문은 메우 안드로메다성이었긴 한데.
뛰어난 학생이라는 건 사설강의 맹목적 추종하는 노예들이 아닙니다.
기본적인 걸 충실히 하면서 그런 강사들이 답하기 힘든 질문을 만들어서 던지고 스스로 그 답을 찾는 친구들이죠.
\vspace{5mm}

+ 뱀꼬리 3 +
"그럼 한마디 한마디마다 피로도가 높고 생각해야하는 기분나쁜 강의를 들어야하는 거네"
라고 질문하면 좋은 답변이겠죠. 사실 그런 게 좋은 강의입니다. 소크라테스의 산파술을 따로 찾을 필요가 없죠.
그러나 그런 강의는 인기가 없고 안 팔리죠. 다 달콤한 것만 찾으니까요.
\vspace{5mm}





\section{[학습공학 019] 푼다}
\href{https://www.kockoc.com/Apoc/565246}{2015.12.26}

\vspace{5mm}

해설집에서는 A$\rightarrow$Z까지 나와있습니다.
그럼 대부분의 학생은 아, AZ인가보구나라고 그냥 수인합니다. 그렇게 암기하면 그 패턴 문제는 바로 풀 수 있으니까요.
그런데 수능 기출에서는 A에서 Q까지를 물어보죠.
A라는 조건이 나오면 반드시 Z로 풀어야한다라고 암기했고, 그리고 그렇게 문제를 풀어서 좋은 점수가 나오던 학생들은 무너집니다.
\vspace{5mm}

여기서 푼다라는 의미를 다시 짚어봅시다.
보통 학생들은 문제를 푼다라는 것을 '답을 낸다'로 혼용하고 있습니다만, 엄밀히 말하면 이것은 \textbf{혼동}이죠.
사실 조금만 생각하면 알 수 있는데 이 풀다의 한자는 해(解)이고, 해의 용례는 해석, 해체, 해부....
\textbf{쓱싹 자르고 나열하는} 이미지입니다.
그리고 우리가 푸는 수학문제의 답이란 대부분은 '방부등식의 근'입니다.
\vspace{5mm}

많은 학생들이 '근을 구하는 것'과 '문제를 푼다'라는 것을 생각없이 섞어쓰고 있다는 것이죠.
문제를 푼다라는 것은 근을 구하는 것이 아닙니다. 근을 구하는 건 문제를 푸는 과정의 일부인 것입니다.
이미 푼다라는 말이 타락해버린 것 같으니 차라리
\textbf{문제를 '해부'한다}
라고 얘기하는 게 낫겠군요.
어느 콕콕의 미녀칼럼니스트 분께서 저 기록을 쓰시기도 했습니다만
생각해보니 그렇지 않습니까. 수학을 제대로 공부한다는 건 '문제'라는 시체가 주어지면 그걸 해부해나가는 과정입니다.
\vspace{5mm}

해부한다라는 말을 통해서 '푼다'라는 것의 본의미를 잡아야한다는 건 뭔가 서글픈 일입니다만
그래서 도대체 A$\sim$Z까지 모든 과정을 다 하나하나 보는 컨텐츠를 어디서 채워야하는데?
그것은 시중문제집의 양치기 - 그리고 모든 해설을 다 읽어보고 비교해보는 과정에서도 절반이 되겠지만
나머지 절반은 '교과서', '수리논술', '인강'도 그렇지만 이런 것을 통해서 스스로 왜 그런 풀이가 나올 수 밖에 없나하는 것,
즉 우리가 배우고 공부한 것을 "100$\%$ 참이라고 할 수 밖에 없는 원자명제"들로 스스로 해부하고 재조립하는 것에서 얻는 것입니다.
\vspace{5mm}

아까 한 일지러가 도저히 모르겠다고 했습니다만 그 이유는 간단합니다. 그 친구는 아직 그 원자명제들까지 공부를 안 했기 때문입니다.
예를 들자면 어떤 문제에서 a, b, c, x 같은 게 주어지면 무조건 그 미지수가 자연수냐, 정수냐, 실수냐라는 걸 무조건 따져야합니다.
적어도 양식있는 교사들이 쓴 교재는 그런 게 적시되어있습니다.
물론 생략된 경우도 있죠. a>b 같이 주어진 경우라면 굳이 실수라고 쓸 필요가 없다는 걸 실수의 정의를 아는 사람이면 얘기할 수 있습니다.
생각해보자면 우리가 푸는 대부분의 문제라는 것은 '근'을 구하는 것입니다.
근을 구하기 위해서는 문제에 명시적으로 주어진 조건, 암시적으로 주어진 조건, 그로써 파생되는 조건들로
교집합을 구하거나 혹은 여집합을 구하는 식으로 하여 근의 범위를 제약하며
그 와중에서 식, 그래프, 논리로써 최종적인 결론에 도달해나가는 것입니다.
그런 과정을 누가 보아도 납득할 수 있는 명제들로 나열해나가는 게 바로 문제를 '해부'해가는 과정이지요.
제대로 해부된 문제들은 읽어보면 정말 맛있는 생선회와 같습니다.
\vspace{5mm}

하지만 다수의 친구들이 어처구니없게도 "문제를 푼다 = 답을 구한다"라고만 알고 있어서 수년동안 헤매고 있습니다.
답을 구한다라고만 착각하니까 인스턴트 풀이에 집착하고, 그 인스턴프 풀이가 가능해보이는 야매교재를 찾으려고 하죠.
본인이 직접 회쳐야하므로 칼을 갈면서 매일매일 칼질을 해보아야하는데 그렇지 않고 어디 회 만들어주는 기계 없어요 그러는 겁니다.
2점 문제가 그냥 먹어도 되는 멸치라면 3점은 광어, 도다리가 되겠고 4점은 마구로나 복어라고 할 수 있겠죠.
\vspace{5mm}

푼다라는 것은 "자르다"라는 이미지입니다.
사실 여기서부터 솔로깡님이나 허혁재님이 다른 논의를 제기할 수 있는 떡밥이 나오기도 하는데
근대 서양에서 동양을 앞서나간 것들의 공통점은 "자르다"이지요.
담론들을 분해해서 추상적 관념으로 잘라낸 철학과 수학, 외과술이 발달해서 현대의학의 기초를 이룬 것이나
신과 인간을 분리하고 아울러 자연에서 자원을 잘라내 자본으로 축적한 자본주의 등.
반면 한국과 중국은그런 잘라내는 과정을 꺼린 경향이 있습니다(일본은 외과수술, 생선회, 사무라이)
대상을 철저히 자르고 분해하기보다는 음양오행과 성리학의 장막 속에서 '순환론'적인 것을 추구했죠.
그리고 그런 담론의 유전자(?) 같은 게 아직도 남아있습니다.
그래서 문제를 푼다 = 답을 낸다 내지 비기를 구사한다... 라고 오해한 것도 이런 데서 비롯된다고 보고 있습니다.
\vspace{5mm}

수학문제는 보통 AIZ BPY SOX이렇게 개념들을 뭉쳐놓았죠.
이걸 다 잘라서 A, I, Z,  B, Y, S, O, X로 분해하고
우리가 아는 질서에 따라서 ABIPQSXYZ로 나열하고
저기서 빠진 것은 우리의 지식으로 채워나가면서 풀이가 완성되는 것입니다.
\vspace{5mm}

그런데 잘한다는 친구들이 그런데 저렇게 안 하고
\vspace{5mm}

AIZ $\rightarrow$ AIDS
BPY$\rightarrow$ SPY
SOX$\rightarrow$ 19금
\vspace{5mm}

이렇게 연상해버린 다음에 엉터리로 풀이과정을 써나가거나 헤매버리죠
많이 풀고 패턴화된 친구들일수록 그 패턴을 절대시하고 의심하지 않기 때문에 자기가 아는 패턴에만 끼워맞추려 합니다.
평가원의 수능문제는 이걸 매우 정확히 저격질하지요.
\vspace{5mm}

시간이 걸리더라도 교과서나 개념서에 나온대로 해부해나간다... 이것이 바른 방법입니다.
더 빠른 방법이 있지 않느냐. 그건 본인들이 해부해가면서 만들어야합니다.
사실 그런 스킬이나 방법이라는 것은, "불필요해보이는 것을 skip한 단축과정"인 경우가 많죠.
수학에 있어서 모든 공식은 사실 개념들의 논리전개과정을 '압축'한 것입니다.
공부하는 사람이라면 즉 공식을 보더라도 그게 어떻게 증명되고 또 연관된 이야기가 뭐가 있는지 쫙 썰을 풀어낼 줄 알아야합니다.
참치든 소든 잘 해부하는 사람이라면 칼질을 하는 순간 남들이 보지 못하는 그 내부의 뼈, 장기 등을 보고 있겠지요.
문제를 잘 해부하는 사람도 마찬가지입니다. 문제를 읽으면서 동시에 그 문제의 해부도를 바로 연상해서 어딜 잘라야할까 보고 있는 것입니다.
\vspace{5mm}

이건 비단 수학에만 국한된 접근방법이 아닙니다.
국어도 영어도 탐구에도 적용되는 보편적인 방법입니다.
\vspace{5mm}

아직은 황금의 3개월이 1/3이 끝나갑니다만 그래도 "칼을 예리하게 갈고 다듬는" 데 부족한 시간은 아닙니다.
자를 수 없는 단위까지 잘라내보시길 바랍니다. 강사나 책의 지침을 맹목적으로 좆지만 말고
누구라도 부정할 수 없는 100$\%$ 참인 명제들로 대상을 잘라버리십시오.
\vspace{5mm}




\section{[학습공학 020] 암기}
\href{https://www.kockoc.com/Apoc/576404}{2016.01.05}

\vspace{5mm}

원칙적으로 암기는 해야합니다.
그런데 여기서 \textbf{암기를 하는 목적이 무엇인가} 정확히 이야기해야하죠.
\vspace{5mm}

공부가 지식을 입력하는 과정이라면, 시험은 지식을 인출하는 과정입니다.
그러나 이 양자가 대등한가? 그렇지 않지요.
수능에 실패했는데 주변 사람들에게 나 30번 잘 풀어요 해보았자 땡전 한푼 안 떨어집니다.
여담이지만 기왕 그렇게 된 것 사교육으로 나서면 되니까 대학 대충 가도 된다 해보았자 나이먹을수록 절감할 겁니다.
이 분야만큼은 정말 '졸업한 학교 네임'이 중요하단 걸요.
명문대 안 가면 어쩌냐, 내가 수학을 잘 하는 걸... 이건 거꾸로 말해서 다른 친구도 그렇게 수학을 잘 하면서 잘 가르칠 수 있고,
공부 대충 해도 운좋아서 명문대 간 다른 라이벌도 충분히 그럴 수 있으므로 그건 뺏기고 마는 것입니다.
\vspace{5mm}

이렇기 때문에 시험 : 공부의 비중은 대략 7 대 3 으로 잡아야 합니다.
공부를 엉터리로 해도 시험만 잘 보면 되느냐? 예, 슬프지만 그렇습니다.
그럼 운빨 제외하고 시험을 잘 치르려면 어떻게 해야하느냐?
\vspace{5mm}

지식의 인출이 매우 \textbf{'신속 정확'}해야합니다.
\vspace{5mm}

가령 이과 과정에서 미적분 2에 해당하는 삼각함수 심화공식을 써야 할 때 컴퓨터급으로 인출되어야합니다.
고수들일수록 거드름 피우면서 시험장에서 유도하면 된다라고 하지만 개소리입니다. 유도할 시간이면 1$\sim$2문제는 풀어댈 수 있습니다.
물론 문풀에 필요한 유도가 있습니다만, 그런 유도과정 역시 \textbf{암기되어 있어야}합니다.
수능 본시험에서 삼각함수 심화 공식 유도가 뭔지 깨달아서 그나마 위안이 되었다는 웃지 못 할 이야기도 있습니다.
\vspace{5mm}

다른 과목도 그렇지만 \textbf{수학이 암기다}라는 말은 이런 의미에서 나오는 것입니다.
개념, 정의, 성질, 공식 등은 아주 정확히 암기되어있어야 합니다.
심지어 문풀 패턴조차도 거기에 매몰되지 않고 탈패턴화된 상태에서 암기하고 있는 것 역시 필요합니다.
자기가 아는 패턴이 아니라는 판단이 드는 순간 긴장하면서 풀기 때문일 것입니다.
\vspace{5mm}

즉, 정리하면 "시험에서의 암기는 절대적이다"라고 하겠습니다.
\vspace{5mm}

그런데 이 대목에서부터 중요합니다.
\textbf{시험의 암기와 공부의 암기는 다르다는 것이지요.}
이건 기존의 이해 vs 암기로 이야기되는 것이지만 이건 한쪽이 필수적 배제된다라는 착각의 원인이 되곤 했습니다.
암기와 이해는 둘 다 필요합니다. 다만 적용 영역이 다를 뿐입니다.
\vspace{5mm}

지식을 입력, 즉 공부하는 과정에서는 처음부터 암기하면 안 됩니다.
지식을 입력하는 과정은 그 지식들을 원초적으로 납득가능한 원자명제들로 분해해서 그것들을 흡수해나가는 것입니다.
다시 말해 음식물을 먹고 그것을 소화시켜 기본적인 당, 아미노산 등으로 흡수하는 것과 비슷하다고 하겠지요.
여기서 다시 한자를 봅시다.
\vspace{5mm}

이해(理\textbf{解})
\vspace{5mm}

저번 19번 글에서 언급된 한자입니다. 이 의미는 '해부'라고 보면 된다고 했습니다.
소의 뿔을 잡고 칼로 잘라나간다, 소위 회쳐나간다는 이야기이겠죠.
다들 이해한다고 말은 하는데 그럼 이해가 뭐냐고 하면 중구난방인 경우가 많습니다.
그럼 원래 한자로 돌아가면 되겠죠.
이(理)라는 건 추상적이고 근본적인 앎 자체, 해(解)는 칼로 사정없이 자르고 썰고 쪼갠다.
○○을 이해한다는 것은 그걸 잘개 쪼개서 이(理)에 대응시키는 과정이라고 보면 되겠습니다.
\vspace{5mm}

그럼 암기(\textbf{暗記})는?
어두울 암(暗)이 의미하는 건 여러가지가 있겠지만 \textbf{'보지 않고'}도 쓸 수 있다는 의미이죠.
그럼 암기를 하기 위해서는?
\vspace{5mm}

고전에 보면 흔한 이야기가 있죠. \textbf{통달}(\textbf{通達})
다시 말해서 사물을 잘개 쪼개서 이(理)로 만들고 궁리하면 그것들이 통해서 깨닫는다는 이야기입니다.
이로 쪼개서 깨닫게 되면 그 다음부터는 절로 알게 되고 마음으로서 받아들이게 되는 그 이후부터는 '암기'도 용\textbf{이해}진다라는 이야기이겠죠.
\vspace{5mm}

우리가 혼자 책을 읽건, 강의를 듣건 그 어느 쪽이든 필수목적은 이해입니다.
이해를 시켜주는 책이 좋은 책이고, 이해를 시켜주는 강의가 좋은 강의인 것입니다.
만약 어떤 책이 요란한 광고를 했는데 정작 이해시킬 대목을 교과서나 타 책에 맡긴다면 그 책은 정말 나쁜 책입니다.
어떤 강사가 스킬을 난사했는데 왜 그러느냐 하는 질문에 답 못 하고 근본적인 것을 납득시키지 못 하면 그냥 약장수인 것입니다.
\vspace{5mm}

수험사이트에 보면 xxx를 보아서 점수가 올라갔다거나 ○○○를 들어서 2등급 나왔다라는 글을 보는데 피식 웃습니다.
점수는 올라가기만 하는 것이 아닙니다. 정체될 수도 있고 내려가기도 합니다.
등급은 혼자 공부만 한다고 오르는 게 아닙니다. 정확히 말하면 남들보다 상대적으로 잘해야 오르는 것이죠.
그런데 님이 공부하는 참고서든 듣는 인강은 '독점'하는 게 아니지요. 다른 친구들도 똑같이 이용하고 있습니다.
차별화를 두고 싶으면 "얼마나 제대로 잘 이해했느냐"(공부), "얼마나 신속정확하게 인출할 수 있도록 암기했느냐"(수험)으로 나뉘는 겁니다.
\vspace{5mm}

참고서든 강의든 남들이 xxx 본다 xxx 듣는다 그런 것에 귀기울이지 마십시오.
일단 \textbf{자기가 뭘 모르는지}부터 제대로 알아야합니다.
아무 생각없이 남들 하라는대로만 하면 시험 전날에서야 "내가 xxx을 몰랐구나"를 깨닫는 울지도 못 할 일이 벌어집니다.
황금의 3개월도 절반이 지나가고 있습니다만 일찍 시작하라는 이유도 별 게 아닙니다.
일찍 시작해서 망하든 말든 공부를 해야만 자기가 뭘 모르느니 알 수가 있고, 그래야만 제대로 공부할 수 있습니다.
자기가 뭘 모르는지 알면, 그것을 해결하는 쪽, 즉 자기가 모르는 것을 '이해'하는 데 도움이 되는 책이나 강의를 선택하면 됩니다.
하지만 현실은 자기가 뭘 모르는지도, 아니 왜 공부하는지도 모르면서 남들이 xx 좋다라고 하면 그것 따라가는 병신행각이 아닙니까?
\vspace{5mm}

만약 님들이 자기가 모르는 것을 정확히 알고, 그것을 이해시켜주는 책이나 강의를 골라 성공했다 칩시다.
그런데 재밌는 건 그건 비주류가 의외로 많을 것입니다. 그런데 님들이 성공하면 그 비주류 책이나 강의는 순식간에 주류가 됩니다.
그리고 역시 떠벌이기 좋아하는 사람들은 그 비주류 책이나 강의를 광고하고 있겠지요.
"자기가 모르는 것이 뭔지 안다"를 잘 알고, 그 다음 독학이건 강의건 '이해'를 제대로 하는 긴 여정을 밟고 날면,
스스로 뭘 암기해야하는지 혹은 뭘 버려야하는지 정확히 알 수 있을 것이고, 그 때가 되어서는 하지 말라고 해도 스스로 암기를 합니다.
게다가 이치를 알았기 때문에 어떻게 암기해야할지 남들에게 물어볼 필요조차 없습니다.
\vspace{5mm}

물론 위에서 말했듯이 이해를 안 하고 그냥 암기만 해서 시험에 합격한 운좋은 케이스들도 있습니다.
그러나 이해를 제대로 하고 암기할 때 10이 필요하다면, 그냥 암기만 하는 경우는 1000이 필요합니다.
자기가 암기한 데에서 문제가 나왔다면 더할나위없이 좋겠지만, 그렇지 않으면 '이해'를 한 사람을 따라잡을 수 없습니다.
그렇다고 해도 이해만 어설프게 하고 암기를 안 했다면 시험장에서 죽쑤기는 마찬가자지인 겁니다.
\vspace{5mm}

\textbf{이해 O, 암기 O >> 이해 X, 암기 O >>>>>>>> 이해 △, 암기 X}(이해 O이면 특별한 사정이 아니면 암기도 O이 됩니다)
\vspace{5mm}

지금 공부하고 계신 분들은 3월 정도 되면 자기가 뭘 모르는구나 아실 것이니 이해 O, 암기 O가 될 것입니다.
그러나 3월달에야 공부 시작하신 분들은 사실 5월달부터 또 모평의 싸이클에 휘말릴 것이니 욕심 버리고 이해 X, 암기 O라도 하길 바랍니다.
\vspace{5mm}




\section{[학습공학 021] 공부는 수동태}
\href{https://www.kockoc.com/Apoc/586591}{2016.01.13}

\vspace{5mm}

사실 노오력은 기본이지만 가장 중요한 것, 그리고 제가 지금 아 이건 부족했구나라고 느끼는 건
\vspace{5mm}

"참는 방법"
\vspace{5mm}

공부에 있어서 머리보다는 환경이 중요하다는 것이 일종의 '관찰'과 '종합'에 따른 결론이면
(사실 이건 경제지리학 쪽 관련한 교양서에서 미국, 유럽인들이 연구결과도 자주 내놓았죠)
거기서 가장 중요하다고 할 수 있는 것은 "참는 능력"이라고 생각.
\vspace{5mm}

파티 구성해서 레이드 갈 때에는 탱커는 반드시 있어야합니다.
탱커가 없으면 공격을 하기도 전에 순삭당하죠.
여러분들이 공부를 한다고 생각하지만 사실 주체는 우리만이 아니죠.
공부는 상호과정입니다.
\vspace{5mm}

우리가 책을 읽을 때 책은 우리의 마음을 읽습니다.
우리가 문제를 풀 때 문제는 우리의 지능을 테스트하고 있죠.
우리가 시험을 치를 때 시험은 우리를 판별하고 있습니다.
\vspace{5mm}

다시 말해서 공부=능동=공격이라고만 생각해서 공부를 '하는' 것으로만 생각들 하시지만
실제로는 우리는 공부를 하는 게 아니라 '공부하는 상태로 당하고 있는' 것입니다.
아니 더 엄밀히 말하면 \textbf{우리가 공부를 하는 건 1$\%$ 정도이고 99$\%$는 공부하는 상태를 당하고 있는 것}이죠.
\vspace{5mm}

한가지 재밌는 현상을 설명해볼까요. 강의를 해보거나 가르쳐보신 분은 느끼실 것인데
그럴 때에는 정말 시간이 잘 가고 피로한 게 덜합니다.
반면 강의를 듣거나 책을 읽을 때는 매우 피로하죠.
얼핏 생각하기에는 강의하거나 가르치는 게 더 피곤할 것 같은데 그게 아닙니다.
강의를 듣거나 공부하는게 강의를 하거나 가르치는 것보다 더 피곤하다는 것입니다.
그래서 본인들이 가르쳐보는 사람들은 "야, 내가 적성에 맞는 일을 하나보다"라고 착각하겠지요.
\vspace{5mm}

실제로는 그렇지 않습니다.
원래 공격=능동은 그리 피곤하지가 않습니다. 피곤하고 힘든 건 방어=수동인 것입니다.
남들을 가르칠 때에는 그냥 나의 지식을 전달해가는 과정입니다.
반면 남에게 배울 때에는 그 배우는 내용으로 내 뇌를 바꿔야 하고 모든 것을 다 뜯어고침을 당해야하는 것이죠.
\vspace{5mm}

강사가 힘들다... 글쎄올시다.
우리나라 현실상 웬만한 교재는 거의 다 짜깁기하면서 그게 위대한 양 애들에게 사기구라까고 있는 것도 그렇지만
정작 강의하는 것 자체는 매우 재밌습니다. 혼자 떠들면서 학생들에게 뭔가 주입시킨다는 것에서 쾌감(...)을 느끼는 부류도 없지 않을 걸요.
다른 일 하라고 하면 못 할 겁니다. 왜냐면 다른 일들은 '수동태'인 반면, 가르친다는 건 정말 99$\%$ 능동태이니까요.
물론 '강의'를 위해 "공부하는 것"(즉, 공부하는 상태를 당하는 것)이나 "자료를 준비하는 것"은 매우 힘들고 어려운 일이지요.
\vspace{5mm}

공부한다가 아니라 '공부하는 상태를 당하는 것', 즉 수동태가 공부의 본질입니다.
한편으로 그렇게 한대씩 얻어맞으면서 생각이라는 것을 강요받고 주입, 세뇌당하다가 자기가 스스로 생각도 해보아야하는 것이지요.
\vspace{5mm}

공부의 본질은 사실 저것입니다. 특히 수험공부의 본질은 "얻어맞는 것", 가장 먼저 수동태적인 입장에 취해서 개조당하느냐는 것이지요.
그럼 개조당하기 위해서 필요한 건? 바로 인내심입니다.
힘들고 성과가 안 나오더라도 그 때일수록 참고 버틴다라고 해서 탱커질을 잘 하는 사람이 살아남고 그렇지 않으면 죽습니다.
콕콕 내에서 인평을 할 때에 저 친구는 잘 하겠군... 이라고 볼 때에 1순위는 머리가 아닙니다. '참을성'이지요.
물론 그 참을성이 무조건 수동태로만 전환해버리면 스스로 생각할 줄을 모르니 바보가 되는 문제도 있겠지만
이거야 방법론을 잘 제시하기만 하면 어떻게든 바뀔 수는 있습니다. 탱커가 마비 상태에 걸린 것이지 죽은 것은 아니거든요.
\vspace{5mm}

그러나 이것저것 거창하게 $\sim$ 한다라고 말은 하는데 참을성은 없다
그런 친구는 수능에서 5년 이상 구른다고 해도 별로 이상할 게 없습니다.
\vspace{5mm}

수능현역합격수기(말하지만 뭘 해도 현역>재수입니다. 인정할 건 인정합시다) 같은 걸 읽을 때에는 환경을 가장 많이 보는데
글투든 아니면 수기에 드러나는 공부환경 같은 것을 보면
집안은 보통 중산층 이상이고 부모들은 한가닥하는 직업인 것도 보지만
일단 자녀에게 독서를 시키는 동시에 '인내심'까지 전수한 것까지도 확인합니다.
똑같이 인강을 들어도 대치동 학원을 다녀도 누구는 되고 누구는 안 되는 이유는 간단합니다.
강의 효과를 누리는 친구들은 탱커 능력은 보유하고 있습니다. 그 친구들이 싸가지거 없든 인내심이 바닥이든지요.
자기들이 대놓고 잘한다고 하지도 않는데 결과는 대단합니다. 그리고 이런 친구들이 대졸 이후에도 잘 나가지요.
그러나 입시판의 고수다 뭐다.... 그게 수능판에서만 알아주지 대학과 대졸 이후에도 알아줄까요?
\vspace{5mm}

인내심을 보완하는 방법은 여러가지가 있사온데
자신들이 공부한 시간, 참고서의 문항과 페이지 수를 달력에 표기하면서 피드백해보면 성취감을 느낄 수 있어 답답함이 덜할 것입니다.
일지 쓰라는 것도 그런 맥락이죠. 앞으로 제안하겠지만 꾸준히 일지 쓰는 사람들끼리 서로 칭찬해주고 감상해주는 것도 정례화시키면 좋겠죠.
결국 수험판은 누가 오래 버티느냐.... 이게 아주 중요한 것입니다.
이런 인내심이 있어야 가장 큰 적, 즉 자기 마음의 변덕스러움과 대적할 수 있습니다.
\vspace{5mm}


\section{[학습공학 022] 금단증세와 역금단증세}
\href{https://www.kockoc.com/Apoc/589069}{2016.01.15}

\vspace{5mm}

그냥 야매이론이니 알아서 걸러들으시길.
\vspace{5mm}

달콤한 자극을 정기적(?)으로 받던 뇌가 그 자극을 받지 못 하면
그런 자극을 유도하는 방향으로 움직이는 건 당연.
마찬가지로 괴로운 자극을 받지 못 하던 뇌에서는
그 자극을 회피하는 방향으로 움직임.
\vspace{5mm}

전자가 금단증세면 후자는 역(逆)금단증세.
\vspace{5mm}

뛰어난 고수들이라면 둘 다 극복했을 것이고
그냥 평범하다면 역금단증에세 버벅댈 것이며
막장인생이면 금단증세도 못 이기고 있을 것임.
\vspace{5mm}

습관적으로 자위하던 사람이 이제 그만 둬야지하다가도 어느 새 야동이나 특정 부위에 손이 가는 것은 뇌의 명령.
이건 술담배게임도 마찬가지임. 의식적인 차단이 없으면 끊는다고 하다가 다시 즐기고 있음.
뇌는 원래 자기 정체성을 위해 학습하기를 정말 싫어하지만, 쾌감을 주는 것은 우리가 시키지 않아도 알아서 학습해버림.
그래서 뭔가 안 좋은 걸 끊는 사람은 금단증상에 시달릴 수 밖에 없음.
\vspace{5mm}

거꾸로 공부 역시 일정한 학습량을 하면 어느 순간 \textbf{'나는 왜 이리 공부를 못 할까, 머리가 나쁜 걸까'하는} 우울한 느낌이 오는데
사실 이건 공부하기 싫어하는 뇌에서 '주인님 그거하자'(...)라고 변덕을 부리는 것임.
그래서 여기에 넘어간 당사자는 신나게 놀다가 당연히 후회하지만 한편 뇌는 쾌감을 얻었기 때문에 시치미 떼는 것임.
\vspace{5mm}

n수를 하는 사람들이 눈빛이 맛탱이가 가거나 인격적으로 붕괴하는 게 별 게 아님.
적어도 내가 접한 케이스 중에서 그런 사람들은 뭔가 하나씩 문제가 있음. 저런 금단증세나 역금단증세에서 대부분 좌절.
대학 좋은 데 갈 필요가 없잖아, 공부 안 하면 안 되어요라고 본인들은 진지하게 말하지만
그 때마다 '아, 상대방의 뇌가 이런 식으로 주인을 꼬드겨서 작살냈구나'라는 생각만 하고 있는 것임.
\vspace{5mm}

그런데 재밌는 건 실제로 공부를 하는데도 자살을 외치거나 변태적인 이야기를 하는 경우도 있는데 그럼 이 경우는 나쁜 걸가?
아님, 이 경우는 뇌에서 체념하면서 최후의 단말마를 외치는 광경임.
그런 유혹에 넘어가지 않고 공부를 하면 뇌에서는 더욱 퇴폐적이거나 범죄적인 망상을 유도하면서 스트레스를 가중하는 경향이 있음.
그래서 당사자는 온갖 정신병자적인 이야기를 하면서도 공부하고 있는 기이한 양상이 벌어지는 것임.
이것이 바로 역금단증세의 최종단계임. 그렇게 해서 교재, 강의 회독수가 확보되고 뇌에서 항복해버리면 그 다음부터 수월해짐.
\vspace{5mm}

어떤 강의 보냐 교재 읽냐 그런 건 아무 의미가 없는 것다고 하는 게 이런 이유임.
수험에서 정말 중요한 금단증세와 역금단증세 극복은 말 안 하고 어느 교재나 강의가 좋아요 이 지랄하는 건
그 본인이 공부를 안 하니 역금단증세도 안 거치는 것이고, 자위나 술담배에 항복하는 거니 금단증세도 시달리지 않는 것임.
\vspace{5mm}

그럼 스트레스는 뭘로 푸나요?
공부로 풀면 되는 것임.
\vspace{5mm}

이 색기가 장난하나?
아니, 수능공부가 싫다면 다른 자격증 시험 - 가령 제빵기능사나 조주사도 좋으니까 그런 잡기 같은 걸 공부하란 것임.
그게 아니면 정말로 헬스클럽에 가서 배에 전제군주제를 실천하는 것도 좋음.
부지런히 학습할 수 밖에 없다라는 걸로 자기 뇌를 항복시켜야하는 것임.
\vspace{5mm}

그럼 눈치빠른 사람은 이런 이야기를 할 것임.
\textbf{머리가 좋냐 안 좋냐보다도 더 중요한 건 '공부할 수 밖에 없다라고 뇌를 납득, 항복시키는 것'이네요.}
\textbf{그럼 환경이 좋은 데에서 태어난 친구들은 어린 시절부터 그게 잡힌 것이네요}
\vspace{5mm}

정반대로 환경이 안 좋거나 공부와 거리가 먼 친구들과 대화해보면 이 친구들은 욕망의 절제나 금욕과는 지구반대편에 떨어져있음.
솔직한 건 좋지만 지나치게 욕망에 끌려다니고 있음. 삶의 목표가 이상이 아니라 욕망에 사로잡혀있음.
\vspace{5mm}

자위든 술담배든 하면서 풀어줘도 공부가 되잖아... 라는 친구들에게 묻고싶음.
그럼 거꾸로 그렇게 수험에서 성공한 친구들이 여유있게 스트레스 풀면서 간 게 몇이나 되냐고.
무엇보다 금단증세도 못 이겨내는 사람들이 과연 역금단증세는 초월할 수 있을까?
금단증세는 단지 술담배 등이 하고싶다라는 것인 반면
역단증세는 스스로 '공부해보았자 소용없어, '난 자살해버려야해'라는 식으로 더 고난도로 나오는데?
\vspace{5mm}






\section{[학습공학 022] 결국 성공을 바라는 속물들이 아닌가.}
\href{https://www.kockoc.com/Apoc/604826}{2016.01.22}

\vspace{5mm}

"저, 공부가 적성이 아닌 것 같습니다. 웹툰 그리고 싶은데요, 괜찮을까요"
\vspace{5mm}

보통 이런 식의 질문은 그냥 대화나 하고 싶다, 그리고 상대가 칭찬해주기를 바라는 심리가 깔려있다.
대충 돈받고 상담하는 사람이나 어떻게 구워삶으려는 사람은 끝까지 이야기를 들어줄 것이고
\vspace{5mm}

물론 상대가 정말 심각하거나 자살할 지경이면 이야기를 들어주는 경우가 나을 수도 있고
여성들의 경우는 자기들이 이야기하다가 스스로 문제해결해버리기 때문에(애당초 해결책을 원한 게 아니라 들어주는 걸 원한 경우가 많다)
\vspace{5mm}

하지만 내 경우는 간단하다
\vspace{5mm}

\textbf{"어, 맞는 말씀이네요. 그럼 내일까지 웹툰 그려서 보여주세요"}
\vspace{5mm}

저 웹툰을 '장사', '노래', '운동'으로 바꿔도 다 똑같은 이야기다.
그리고 저런 경우 성과물을 기한 내에 마쳐 보여준 케이스는 여태껏 단 한번도 없없다.
\vspace{5mm}

사실 생각해보면 간단하다.
본인들이 공부가 적성이 아니라 다른 게 정말 간절하다면 \textbf{벌써 성과물을 내놓았지} 그냥 '입'으로만 $\sim$ 하고 싶다라고 하겠나.
자기들이 정말 좋아서 하는 건 금지시켜도 한다.
공부가 좋은 녀석은 공부하지 말라고 몰래 책 들고와 밤새서 공부하고
음악이 좋은 녀석은 알아서 작곡에다 연주질해서 유튜브에 올려 평가받고 있고
만화가 좋은 녀석도 디씨 같은 데 올려서 온갖 막장평가를 받으면서도 행복해한다.
\vspace{5mm}

박사들도 고졸사원 채용에 줄서는 세상에 누가 '공부만 아니면 답이 없다'라고 할 사람이 누가 있나.
중요한 건 실천이다. 열심히 일해서 바로 돈벌어오는 고졸이, 입으로만 의치한 가서 출세할 거야 하는 N수생보단 낫다.
그런데 다 알겠지만 어느 분야든 '수험'만 아닐 뿐이지 미친 듯이 공부하고 조사하고 연구하고 도둑질해야하는 건 똑같다.
\vspace{5mm}

자, 그럼 저런 '도피적' 이야기는 왜 하는 걸까.
\vspace{5mm}

그건 본인들이 성공하고 싶어서이다.
수험에서는 실패했지만 다른 분야에 가면 숨겨진 재능이 일본만화 주인공처럼 드러나서 1등할 수 있을 거라 상상하는 것이다.
그렇게라도 하지 않으면 지금 열불나는 걸 참을 수도 없고 자살하고 싶어진다.
\vspace{5mm}

그럼 본인에게 솔직해지면 된다.
\textbf{"난 성공하고 싶다, 남들 위에 군림하고 싶다, 공부 잘 한다고 주변에서 칭송하는 것 듣고싶다"}
\vspace{5mm}

그럼 그 길로 다시 전력질주하면 되는 것이다.
그런데 왜 노력을 안 하고 다른 길로 도피하려 할까.
그건 '실패'를 두려워하기 때문이다.
그럼 왜 실패를 두려워할까?
잘못된 환경에서 잘못된 교육을 받아 잘못된 가치관을 주입당했기 때문이지.
하나 예를 들면 왜 너는 성적이 안 나오니 하면서 실패를 겪기만 하면 자녀를 갈구고 인격적 모독을 주는 정신나간 부모들.
그리고 무조건 승리만 해야하지 패배는 없다라고 그저 입으로만 자녀들을 갈구는, 정작 자기들이 공부하라하면 핑계대는 부모들.
\vspace{5mm}

그런데 생각해보자. 항상 이긴다는 게 있긴 하나? 백전백승 그런 게 어딨나.
항상 연전연승하던 나폴레옹이든 히틀러든 그 끝은 어땠나?
흔히 드는 이순신도 젊은 시절에는 막장이었으며, 결국 끝에는 전사하지 않았나.
\vspace{5mm}

\textbf{1번 성공하려면 9번 실패는 해보아야한다, 실패의 선물은 바로 "지혜"이고, 거꾸로 지혜는 실패가 아니면 얻을 수 없다.}
이걸 알고 9번 실패할 걸 99번 실패해야지, 그럼 돌아오는 1번의 성공은 대단할 것이다라고 생각하고 도전하면 되는 것인데
이런 걸 모르고 그저 우리 아이는 완벽해야해하는 부모들이니
자기들이 천재이고 너무 머리가 좋아서 대입성적이 잘 나왔다고 하는 허세들이니
그저 이런 데에 영향받아서, 그냥 아무 생각없이 공부만 해도 잘 될 수 있었던 친구들이 휘둘리고 폐인이 되는 경우가 많다.
\vspace{5mm}

10대 시절부터 잘 생긴 엄친아로서 항상 높은 성적을 거두고 어쩌구저쩌구하는 게 우상?
실제로 그런 사람들 중에서 몇이나 20대 후반부터 잘 나가는지 내가 궁금, 그런 사람이 그래서 누군데?
거꾸로 부모 극성으로 실패없이(?) 명문대 들어갔다가 한번 굴러떨어져 재기불능 상태로 빠진 케이스는 꽤 있는 건 아시나?
학벌과 관계없이 안정적으로 성공한 사람들은 남들이 성인기에 겪을 실패를 선행학습한 케이스가 많다는 건 아나?
\vspace{5mm}

일침 가하면 공부가 내 적성 아니여요하는 친구들의 실제 생활을 보면 '방만한 케이스'가 많다는 것도 지적해야할 것이다.
이들은 입으로는 자기들이 공부가 질렸고 적성에 안 맞다... 라고 하지만 실제 어떻게 사는가보면 이미 폐인처럼 사는 케이스도 많다.
엄마가 차려주는 세끼 밥은 꾸역꾸역 먹고 용돈 받아쓰면서도 '노오력? 웃긴다', '헬조선은 탈출이 갑' 이 딴 드립이나 치고 있단 것이다.
\vspace{5mm}





\section{[학습공학 023] 계획술의 핵심은 줄이는 것}
\href{https://www.kockoc.com/Apoc/611403}{2016.01.27}

\vspace{5mm}

계획을 실천하는 사람은 100명 중 1명 꼴도 찾아보기 힘들다.
그건 나도 마찬가지이고, 그렇기 때문에 역설적으로 왜 계획을 실천하기 어려운지 안다.
\vspace{5mm}

우선 계획을 실천하기 어려운 건, 계획이 비현실적이기 때문이다.
계획은 우선 자기가 실천 가능한 만큼만 세워야 한다. 이건 \textbf{'능력'과 '현실'의 문제다}.
하지만 다들 계획을 세울 때에는 자기의 능력을 넘어서고 현실적으로 불가능한 수준으로 세운다.
즉, "당위"와 "이상"으로 세우는 것이다.
당위와 이상으로 세운 계획이면 당연히 실패할 수 밖에 없다.
\vspace{5mm}

통일은 우리의 미래다라고 백날 외쳐보았자 그게 실천이 되나
중요한 건 통일이 왔을 때에도 대처할 수 있게 준비하고 있느냐, 그런 예산을 마련해놓고 있느냐는 것이다.
하루에 100문제도 건사 못 하는 친구들이 계획을 세울 때에는 하루 200문제를 잡는다.
이게 실천 가능했다면 계획 따위도 불필요했을 것이다.
\vspace{5mm}

그럼 계획의 핵심은 무엇일까, 그건 바로 "줄이는 것"이다.
계획을 세우는 건 한마디로 실천하기 위해서이다. 그럼 실천을 잘 하는 요령은?
당연하지. "양"을 줄이는 것이다.
아니, 당신은 양치기를 하라면서 한편으로 양을 줄이라니 뭔 소리인가?
당연히 수능 날까지는 많은 양의 문제를 풀어야 한다. 이건 '결과'로서의 양이다.
그러나 매일 공부할 때에 푸는 양은 줄일 수 있을 만큼 줄여야 한다. '과정'으로서의 양은 적어야 좋다는 것이다.
총 20,000문제를 푼다고 하자. 이걸 10일로 나눠 하루 2000문제를 푸는 건 비현실적인 일이다.
그러나 20,000문제를 200일에 나눠 하루 100문제를 푸는 건 가능성이 있다.
\vspace{5mm}

결국 계획이란 \textbf{"더 많은 시간자원을 확보해 총 공부량을 분할, 단위 공부량을 줄인다"}로 요약된다.
생각해보면 별 것 없지만 이걸 실천하는 사람은 그다지 없다.
아마 다수가 성공한 사람들의 수기를 보면서 나도 허리띠 조르고 저렇게 해야지라고 하다가 올해를 날려먹을 것이다.
(그것도 그 성공한 사람들은 수년 전부터 시작한 사람들이고 환경도 좋다는 걸 간과한 결과다)
\vspace{5mm}

자신이 얼마나 무능한지 깨닫고 그에 맞게 작업량을 줄여주는 사람이 현명하다.
\vspace{5mm}






\section{[학습공학 024] 문제해결의 핵심}
\href{https://www.kockoc.com/Apoc/611483}{2016.01.27}

\vspace{5mm}

모든 문제는
\vspace{5mm}

\textbf{항(=핵심 요소)들의 관계}로 요약된다.
\vspace{5mm}

국어와 영어와 사탐에서는 주로 명제와 표를 쓰고
수학과 과탐에서는 식이나 그래프를 덧붙여 사용한다.
\vspace{5mm}

읽는다라는 것은 항들을 파악하고 - 덧붙여 항들의 갯수인 계수, 항들의 차원인 차수까지 -
더불어 그것들이 속해있는가 포함하는가(집합), 같은가 다른가, 다르다면 어느 게 우월한가(등식과 부등식)
상관, 인과관계를 어떻게 나타나는가(함수) 등까지 파악하게 된다.
\vspace{5mm}

물론 위와 같은 문제해결의 테크닉을 교과과정에서나 인강에서 바로 가르쳐주지 않는다는 게 문제다.
보통은 '패턴화'를 통해서 간접적으로 익힌디.
\vspace{5mm}

교과과정을 떠나 사회 활동에서의 모든 문제는 다음과 같은 해체 및 재구성 과정을 거친다.
\vspace{5mm}

\begin{itemize}
    \item ⓐ 읽기 - 사실관계 파악 - 인문, 사회적인 것이냐, 자연과학적인 것이냐
    \item ⓑ 분해 - 그 사실관계들 중 무엇을 강조하고 소거할 것인가 - 여기서 법률, 경제, 철학적인 기준 발동
    \item ⓒ 환원 - 대안이나 구제수단을 어떤 걸로 결정한 것인가 - "물리적 가능성"(자연과학) "윤리적 타당성"(인문과학), "경제적 합리성"(사회과학)
\end{itemize}
\vspace{5mm}

저런 분해와 환원에는 다양한 각론적 틀이 있고, 이것이 대학에서 배우는 전공과정이라고 할 수 있다.
\vspace{5mm}

여기까지 오면 가장 중요한 건 사실 철학이라는 걸 알게 된다.
사실 서유럽이 득세한 것은 자연과학 이전에 중세시대 종교철학이 성숙되면서 그로서 '생각하는 방법'이 발달한 덕분이다
(사실 철학 없는 수학은 산수고, 철학 없는 과학은 망상일 뿐이다)
문제를 분해하여 환원화는 과정에서는 무엇이 참이고 거짓이냐 하는 걸 세밀히 따져야하는데
이 점에서 인문과학적인 것이 상당히 유용하다.
\vspace{5mm}

위와 같은 접근으로 가는 교육이야말로 가장 낫겟지만 이건 '시간'이 많이 걸린다.
특히 주입을 강조하는 공급자 입장에서는 더욱 그렇다.
\vspace{5mm}

운이 좋아서 저런 틀이 머리에 박혀, 평소에도 저런 틀로 사고하고 행동하는 사람은 머리좋은 사람으로 인정받을 것이다.
반면 운이 나쁘거나 본인이 노력을 안 하여 저런 틀을 익히지 않고, 교조주의적이고 집단세뇌적인 것에 낚인 사람은 평생 바보로 살  것이다.
\vspace{5mm}




\section{[학습공학 025] 공부를 환경으로 바꿀 것}
\href{https://www.kockoc.com/Apoc/616830}{2016.01.30}

\vspace{5mm}

학습의 4차원은 \textbf{이해, 암기, 반복}, 그리고 \textbf{응용}이다.
\vspace{5mm}

이 중에서 뇌를 길들이는 것은 \textbf{이해와 반복}이다.
암기는 뇌를 겁박하는 것, 응용은 뇌가 스스로 하는 것이기 때문이다.
\vspace{5mm}

이해는 뇌를 설득시키는 것이고, 반복은 뇌를 체념시키는 과정이다.
인강을 듣는 것은 뇌를 납득, 설득시켜서 특정 구조를 주입시키는 것이고 (전에 얘기했다시피 뇌는 공부를 싫어한다)
반복을 시키는 것은 그래야 뇌가 그 공부를 '환경'으로 인식하고 받아들이기 때문이다.
뇌는 공부는 싫어하지만 \textbf{환경}에는 적응한다. \textbf{공부를 \textbf{환경}에 수렴시키기 위해 필요한 게 '반복'인 것이다}.
마찬가지로 인강보다 실강이 비효율적인 것처럼 보여도 실제로는 실강의 집중도가 높은 건,
인강은 단순한 멀티미디어 정보에 불과하지만, 실상은 그 자체가 공부환경이기 때문이다.
\vspace{5mm}

공부하는 것보다도 공부가 되는 것이 더 중요한 것이고
인간의 학습에서 가장 중요한 게 환경이라면, 현명한 사람은 공부를 \textbf{'환경'에 수렴시키고자 할 것}이다.
머리가 좋은 친구들은 그 뇌가 '공부'를 거부하지 않는 특징이 있다.
그런 친구들은 \textbf{새로운 정보를 즉석에서 이해, 반복하여 환경화할 수 있는 패턴이 들어가 있다}고 할 것이다.
\vspace{5mm}






\section{[학습공학 026] 어째서 공부할 때 달콤한 걸 먹는 게 좋을까.}
\href{https://www.kockoc.com/Apoc/616850}{2016.01.30}

\vspace{5mm}

파블로프의 개와 같은 원리다.
뇌는 공부를 싫어한다, 하지만 쾌감을 좋아한다.
공부하기 싫어하지만 오르가즘(...)을 원하는 뇌와의 끝없는 싸움이 수험이다.
자위를 가능하면 하지 말라는 이유도 그렇다.
조금만 노력(?)해서 쾌감을 얻을 수 있는 자위를 하지 뭐하러 힘든 공부를 하려 하겠나
특히 공부를 괴롭게 하려는 친구들일수록 상담해보면 "실은요, 밤마다$\sim$"라는 케이스가 많다.
\vspace{5mm}

공부 자체가 즐거울 수는 없다. 그러나 \textbf{뇌를 속일 수는 있다.}
가령 xxx 쵸콜릿을 좋아하는 친구라면 그 쵸콜릿은 반드시 '공부할 때에만' 먹는 걸로 정하자
물론 공부하자마자 먹는 건 안 된다, 100문제라면 적어도 80문제는 풀었을 때 먹으면서 나머지 20문제를 푸는 걸로 잡으면 된다.
이런 과정을 반복해서 공부할 때에는 살찌는 것 신경쓰지 않고 먹어도 된다라고 하면
뇌는 그러한 쾌감을 공부하는 쾌감으로 '인식'할 수 있다.
\vspace{5mm}

이건 음악을 듣는 것도 마찬가지이다. 음악이 집중력을 떨어뜨리거나 후크송화될 수 있다는 위험은 있다.
그러나 본인이 공부할 때 잠이 오거나 멍하다면, 차라리 좋아하는 음악을 틀어놓고라도 공부하는 게 아예 안 하는 것보단 낫다.
다만 이 경우도 공부를 할 때에만 좋아하는 음악을 듣는다는 철칙을 지켜야 한다.
반드시 공부는 엄숙하게 그리고 완벽한 환경에서만 하야한다는 것을 버려야 한다.
물론 공부를 환경에 수렴시키는 것이 중요하다. 그러나 완벽한 환경을 인위적으로 만들 필요는 없다.
여기서 말하는 환경이라함은 '뇌'가 적응하고자하는 대상을 말하는 것이지, 완벽한 독서실 완벽한 교재 완벽한 책상 이런 게 아니다.
\vspace{5mm}

집중한다는 건 기합을 외치는 것도 마음씨가 정화된 것도 아니다.
야동을 보거나 밤에 몰래 자위할 줄 아는 사람이면 '집중'은 충분히 할 수 있는 것이다.
그렇다면 집중은? 그건 다시 말해 자기가 하는 행동에서 쾌감을 느끼기 때문에서 뇌에서 즐길 수 있는 걸 말한다.
공부에 집중하는 방법은 외적으로는 방해가 되는 것들을 치우는 것이지만, 내적으로는 공부를 즐길 수 있게 세팅하는 걸 말하는 것이다.
\vspace{5mm}

하다 못해 정말 야한 것에 빠져서 주체 못 하는 사람이 있다치자.
그럼 차라리 국어나 영어를 그런 걸로 대입시켜 공부하시고
수학이나 과학문제도 그런 것과 결부짓거나 모에화(...)시키는 것이 아예 공부를 포기하고 안 하는 것보단 나을 것이다.
\vspace{5mm}

요컨대 자기가 정말 좋아하고 즐기는 것들을 공부하는 도중에 할 수 있도록 해서 공부=쾌감이라고 인식시키는 것이 중요하다.
\vspace{5mm}




\section{[학습공학 027] 복리효과}
\href{https://www.kockoc.com/Apoc/618752}{2016.01.31}

\vspace{5mm}

\textbf{1.001 : 365일 - 1.440배}
\textbf{1.440 : 005년 - 6.197배}
\vspace{5mm}

\textbf{1.440 : 010년 - 38.404배}
\vspace{5mm}

\textbf{1.440 : 020년 - 1474.903배}
\vspace{5mm}

매일 0.1$\%$만 자신이 발전한다고 가정하면 365일이 되면 1.4배, 5년이면 6배, 10년이면 38배, 20년이면 1474배가 된다.
\vspace{5mm}

그런데 문제는 저 0.1$\%$는 본인이건 타인이건 감지할 수가 없다.
하지만 저 0.1$\%$의 차이가 장기적으로 쌓이면 기하급수적인 결과를 낳는다.
\vspace{5mm}

0.1$\%$의 차이 : 똑같은 공부라도 2년 더 시작하면 2배 차이가 나는 것이다.
이게 5년이면 무려 6배 차이가 난다.
하루로는 변화를 알 수 없다. 그래서 이걸 설명을 못 하니까 "저 녀석은 머리가 좋아서"라는 \textbf{유전자 타령을 하는 것}이다.
\vspace{5mm}

사실 우리가 알 수 있는 건 변화의 결과, 즉 함숫값이지, 실제 변화율, 즉 미분계수를 파악하기란 힘들다.
설사 미분계수를 파악한다고 해도 0.1$\%$ 가지고 뭘 하겠어라고 비웃으면수 결국 수학에게 비웃음당한다.
\vspace{5mm}

그리고 눈치빠른 사람은 \textbf{갑자기 자살하고 싶어질 것}이다.
\vspace{5mm}

0.1$\%$씩 늘려서 저런 결과라면
0.1$\%$씩 감소하면 정반대의 결과가 되기 때문이다.
어제보다 0.1$\%$ 못한 삶을 사는 것을 지속하면
\vspace{5mm}

\textbf{1/1.001 : 365일 - 0.7배}
\vspace{5mm}

\textbf{1/1.440 : 005년 - 0.16배}
\vspace{5mm}

\textbf{1/1.440 : 010년 - 0.02배}
\vspace{5mm}

\textbf{1/1.440 : 020년 - 0.0006배}
\vspace{5mm}

물론 10년도 되지 않아 강제로 노예생활하거나 자살해버리는 것이다.
\vspace{5mm}

그래서 재수, 삼수하면서 오히려 '망해가는' 것도 이로써 설명된다.
예전에 말했던 허력도 저걸로 설명할 수 있다.
\vspace{5mm}

\textbf{그렇다면 전공과 관계없이 0.1$\%$ 성장만 한다면 - 물론 소득이나 공부량으로 계량해야겠지만 - 되는 것 아닌가?}
\vspace{5mm}

맞는 말이다.
\vspace{5mm}

전공선택도 중요하지만 가장 중요한 건 저 0.1$\%$ 성장을 계속할 수 있냐는 것이다.
그리고 0.1$\%$를 0.5$\%$로 늘릴 수 있다면 더욱 좋을 것이다.
\vspace{5mm}

여기서 상담하는 친구들에게 쓴소리를 던지는 이유다.
\textbf{미리 미래를 재단할 것 없이 자기가 길을 잡았으면 어제보다 나은 삶을 살면 된다.}
\textbf{그리고 어떤 교재를 보느냐 하기보다 어제보다 1문제 더 풀면 된다.}
\vspace{5mm}

하지만 그러지 않는다. 상담을 해서 그냥 자기 만족을 얻으려고 하는 종자들이 대부분이다.
\textbf{그래서 그들은 마이너스 복리효과로 막장코스를 밟는 것이다.}
\vspace{5mm}

학습에 관해 말하면 어떤 인강이냐 학원이냐보다도 더 중요한 건 저 복리효과를 제대로 누리는 것이다.
황금의 3개월에 대해서 비웃는 사람들도 있는 데 당연히 나는 '저런 병신들'하면서 비웃는다.
이런 친구들은 \textbf{복리효과가 뭔지 모르기 때문이다(게다가 실제로 공부할 수 있는 건 5월까지라는 건 다 알건데?)}
\vspace{5mm}

\textbf{누구보다 빨리 시작하고 더 많이 읽고 많이 푸는 것이 전제된 하에서} 명품 강의나 교재를 고르는 것이다\textbf{.}
\vspace{5mm}

이런 진리를 스스로 깨달을 수는 있다. 그런데 본인이 납득했을 때는 보통 5년 이상은 흘러간 후다.
남들은 벌써 \textbf{자기보다 6배는 성장해버린 것}이다.
더군다나 고3 때만도 못 한 삶을 산다면, 즉 0.1$\%$ 후퇴를 꾸준히 했다면 본인은 1/6으로 쇠퇴한 것이나
결과적으로 \textbf{36배 격차가 벌어진 셈}이다.
\vspace{5mm}

이 뻘글 시리즈 중에서 이번 편만큼 공포스럽고 소름끼치는 글은 없을 것이라고 본다.
관념적으로 아 나는 의대 갈거야 그렇게 하지말고
전국에서 50등안에 들 각오로 공부해야지 안 그러면 승산은 없다.
\vspace{5mm}

더욱이 현역이나 재수로 SKY 의치한에 간 경우는 저런 복리효과를 톡톡히 누린 경우가 많다.
첫째, 가정환경이 무난하다, 둘째, 본인이 공부하는 습관이 들어서 10년 이상 했다, 셋째, 좋은 사교육을 일찍 이용했다.
본인이 저 중 하나라도 아니라면 더 많이 노오력할 수 밖에 없다.






\section{[학습공학 028] 여우와 곰}
\href{https://www.kockoc.com/Apoc/622613}{2016.02.03}

\vspace{5mm}

선생들은 한번 가르쳐주면 알아듣는 애들을 좋아한다.
한번만 가르쳐주면 열을 안다고 하니까 이 아이는 머리가 좋군요, 잠재성이 있어요 그런다.
\vspace{5mm}

\textbf{그런데 정말 그런지는 실제로 의문이다.}
\vspace{5mm}

비열이 낮은 녀석은 한번만 학습해도 되는 \textbf{여우}같은 녀석이다. 한번 설명하면 바로 '주입당하기' 때문이다.
그런데 이런 친구들은 검증이나 검토에 약하다, 그리고 좋지않은 것을 배우면 그것도 바로 배워버린다(...)
수학설명 해주니까 바로 학습하는 친구들이 사이비 종교 교리도 바로 학습해버릴 가능성이 높은 것이다.
\vspace{5mm}

비열이 높은 녀석은 한번 가르친다고 해도 안 된다. 3, 4번, 아니 심지어 10번을 가르쳐야할 때도 있다.
그런데 이게 학습능력이 부진해서? 아니다. 이들은 \textbf{곰}같은 애들이라서 상당히 의심이 많고 보수적이라 함부로 주입당하지 않아서이다.
자기가 납득할 수 있는 방식이 아니면 학습을 안 하려고 한다. 이런 친구들이므로 배우지 말아야 할 것을 걱정할 필요가 없다.
\vspace{5mm}

당장 선생 입장에서는 여우 같은 녀석이 좋다, 조금만 일했는데도 학생 성과가 잘 나오니까.
그리고 그런 학생이 있으면 돈벌이가 되는 것도 부인할 수 없다.
그러나 대학 진학 이후로 이런 여우들이 성공한 케이스를 보지는 못 했다.
뒤늦게 두각 드러내면서 확실한 성과를 보여주는 건 여우가 아니라 곰이다, 혹은 여우같은 곰일지도 모르겠다.
\vspace{5mm}

실제로는 여우들이 문제다.
\vspace{5mm}

왜냐면 여우들은 나쁜 것도 함부로 배워버리니까. 그럼 그 나쁜 것을 지워버리라고 할 수도 없다.
한 때 공부 잘하다가 추락한 인간들이 대부분 이런 여우과들이다. 나쁜 데 빠지거나 잘못된 방법론을 학습해버려서 돌이킬 수가 없는 것이다.
반면 곰의 경우는 '반복'만 확보되면 안정적인 성과를 보여준다. 이들에게 필요한 건 '시간'과 '관리'다.
자기가 특정 내용을 학습하기 위해 5번은 봐야한다라고 본다면, 그냥 5번을 반복해버리면 안정적인 점수가 나오는 것이다.
배우지 말아야 할 것을 걱정할 필요가 줄어든다.
\vspace{5mm}

다만 문제는 곰이 \textbf{'여우'를 흉내내는 것}이다.
시간이 걸리더라도 곰은 곰 테크트리를 밟아야하지 여우를 흉내내지 말아야한다.
내가 보기에는 그냥 진득하게 하면 좋은 성과 거둘 애들이 3$\sim$4년 날리는 걸 보면 대부분 자기 스타일을 망각한 결과가 있다.
예컨대 100일 정도만 영어에 올인하면 재활이 성공할 수 있는 녀석이 그것도 못 견디고 단군신화의 호랑이가 되는 케이스가 많다.
남들은 한달만 인강듣고 끝냈는데요라는 데 현혹되는 것이다.
그냥 진득한 스타일로 100일 가면 될 것을 남들 방식을 무비판적으로 수용하다가 시행착오 5번 저지르고 시간만 날리고 성과도 보지 못 한다.
\vspace{5mm}

콕콕에서 일지 참여하시는 분들은 대부분 곰이니 괜히 여우 흉내내지 마시기들 바란다.
\vspace{5mm}






\section{[학습공학 028] 일주일에 며칠 공부?}
\href{https://www.kockoc.com/Apoc/626695}{2016.02.07}

\vspace{5mm}

의대에 가실 겁니까라고 하면 다들 이렇게 얘기한다. "그럼 7일 공부하면 되겠습니까"
그럼 가볍게 말하지. "어, 올해는 힘드시겠네요"
어안이 벙벙해서 이렇게들 이야기한다. "뭐 어쩌라고"
\vspace{5mm}

답변은 간단하다.
\vspace{5mm}

\textbf{"일주일에 10일 공부하시면 됩니다"}
\vspace{5mm}

그리고 다들 알 것이다. 저게 단순한 수사가 아니라는 것을
시간은 활용하기에 따라서 중층활용이 가능하다.
어떤 사람은 8시간 공부하고 나머지 4시간을 그냥 쉰다,
그런데 특별한 사람은 8시간 공부하고 나머지 4시간을 '자기가 더 좋아하는 과목'을 공부하면서 쉰다.
이러면서 벌써 두 사람분의 공부는 하는 것이다. 이렇게 하면 1일에 1.5일을 공부할 수 있다.
\vspace{5mm}

노예들은 하루를 어떻게 편하게 보낼지나 연구한다.
직장인들이나 자영업자들은 몇시간 일하고 얼마를 버나 계산한다.
\textbf{그러나 성공한 경영자라면 노는 시간까지도 투자해서 시장을 늘릴 수 없을까 한다.}
\vspace{5mm}

이건 공부에 있어서도 맞는 말이다.
공부가 아니라 입시에 미친 놈들은 합격하느냐 마느냐 그것보다도, \textbf{자기가 전국에서 몇등인가 그걸 따질 것이다.}
어떤 문제라도 내면 다 맞출 수 있고 남보다 잘 하는 걸 생각하지, 어디 갈 수 있느냐 하는 건 부차적인 문제다.
사실 xx대에 갈 수 있을까요라고 말하는 순간 그 사람은 튼 것이다.
당연히 xx대에 갈 수 있다라는 전제 하에서 소위 천재나 엄친아라는 아무개보다 얼마나 앞셨느냐 따지는 것이다.
\vspace{5mm}

왜 기술보다 마음이 중요한지 여기서 드러난다.
\vspace{5mm}

그냥 수험생이면 하루에 몇시간 공부할까를 따지겠지만
초수험생이면 어떤 공부로 휴식을 취할까 그런 걸 고민하고 있다.
마인드 자체가 다르니까 휴식과 공부관도 다르다, 스트레스 받는 구조도 다른 것이다.
\vspace{5mm}

그래서 엄마들 극성이 그렇다. 자기 자식들을 평범한 아이나 실패한 아이 옆에 안 두려고 하는 것은 이기적이긴 하지만 일리없는 게 아니다.
스타일 하나하나의 차이가 사실 엄청난 결과를 유발하기 때문이다.
아마 이 글을 읽는 상당수는 그게 뭔 소리야 미신일 거야, 머리좋은 게 장땡이야라는 진짜 미신 같은 소리를 할 것이다.
하지만 내가 본 바는 머리는 별 차이가 없다, 오히려 두뇌가 뛰어난 경우 그게 잘못된 스타일을 학습하면 데미지를 더 입는다.
하지만 '마음가짐'이 다르다면, 즉 어렸을 때부터 삶과 교육의 철학을 달리 주입받는 경우는 정말 무시무시한 격차를 낳는다.
\vspace{5mm}

서른이 가까워져도 에라 살아있다라는 것으로 행복해 하며 잉여질하는 사람도 있고(물론 나도 그렇지 않나)
방면 사춘기에 가까워졌는데도 자기 혼자 세사람 몫을 왜 할 수 없나 고민하는 예의바르고 능력좋아 짜증나는 녀석들도 있다.
\vspace{5mm}

그럼 결국 노오력한다라는 건 수식어의 문제일까.
평범하게 노오력해서는 쓸모없는 것이다, 본인이 지배자가 될 각오로 비범하게 노오력하지 않으면 사실 답이 없는 세상이 된 것이다.
그런데 단지 돈만 벌고싶다고 하면서 그런 고통을 감수할 생각이 없으면 \textbf{답은 없다.}
상담을 해줄 때에는 일주일에 하루는 쉬라고 한다, 왜냐면 그 경우 상대는 아직은 평범하기 때문이다.
평범한 사람이 평범하게 공부하면 그래도 가능성은 없는 건 아니다. 사실 SKY 공대급이면 가능할 수는 있다.
그러나 요즘 의대라면 비범하게 공부해야할 것은 너무 자명하지 않나?
\vspace{5mm}

이런 것은 강요도 아니라 그냥 내가 쿨하게 바라본 진리다.
더군다나 IMF 이후에 수정된(...) 10대들의 경우, 부모들부터 맞춤형 결혼, 거기다가 전략적인 임신빨이라도 받았는지
연령이 내려갈 수록 더욱 더 무서운 괴물들이 올라오고 있다.
(구체적으로 올해 고2 올라가는 세대들 이하부터는 뭔가 상당히 무섭다)
\vspace{5mm}

저렇게 하지 않고 잘 먹고 잘 산다라고 하는 경우는 그냥 무소유라도 읽고 욕심이라도 줄이는 게 좋을 거다.
노력과 고통이 수반되지 않는 꿈만큼 절망적인 건 없다.
꿈은 혼자 이루는게 아니다, 같이 이루는 것이다. \textbf{내 밑에서 추락해줘야하는 경쟁자들과 함께}.
경쟁율이 100대 1이면 거기서 99명의 꿈을 짓밟아야 내 꿈이 이뤄진다는 것만큼이나 간단한 진리를 다들 망각하고 있다.
\vspace{5mm}




\section{[학습공학 029] 정신적 거세}
\href{https://www.kockoc.com/Apoc/627852}{2016.02.08}

\vspace{5mm}

아내가 사랑하는 건 남편이 아니라 아들(...)
그게 교육에서는 결론적으로 정신적인 거세로 나타난다.
딸시집보내기 싫어하다가 정말 혼기 놓치게 만드는 아버지처럼
아들을 품안의 자식으로만 키우려고 하는 어머니들 때문에 마마보이가 되는 것을 넘어서 아예 욕망이 사라진 경우가 있다.
\vspace{5mm}

이런 친구들의 특징은 학습과정에서 일종의 '강박'으로 나타난다.
수학은 무조건 암기하는 것이야
영어도 무조건 정해진대로만 읽어야하는 거야... 이렇게
\vspace{5mm}

그런데 아시다시피 생각은 \textbf{이미지 $\rightarrow$ 문자, 문자 $\rightarrow$ 이미지}. 이 끊임없는 교대수열.
이미지에 대한 감각이 떨어지면 아무리 공부해보았자 한계가 온다, 물감이 없는데 그림이 선명할 리가 있겠나.
다양한 방식으로 오감도 충족시킬 뿐만 아니라 일정 부분의 성욕이 있어야 저런 이미지 능력도 원활해지는 것이다.
\vspace{5mm}

하지만 이걸 모른 채 그냥 자식을 무조건 잡아서 강요만 하면 되는 줄 아는 부모님들 때문에 무채색 학생들이 생기곤 한다.
아무리 노력해도 이미지 능력이 없으므로 머리가 안 돌아간다, 그러니 좌절해버린다.
장애가 있는 상태에서 노오력해버리면 소용없다, 그 장애를 일단 걷어내야한다.
\vspace{5mm}

해외 영화, 해외 만화, 해외 드라마를 적어도 사흘에 한편을 보는 걸 권장하는 이유다.
이국적인 건 무조건 감수성을 자극하게 되어있다. 거기다 해외 작품은 한국의 심의, 검열을 받지 않아 감각적인 게 선명한 경우가 많다.
\vspace{5mm}

아울러 노래방 같은 곳을 가도 좋지만 때때로 마트, 백화점에 가는 것도 권장한다.
미친 것 같지만 식료품 코너에 가서 개처럼 코로 킁킁거리면서 냄새를 맡아보는 것도 좋은 경험이고
구입할 생각이 없더라도 화장품, 의류, 가구, 전자제품, 서적 등을 하나하나 보고 디자인, 성능, 가격, 향기, 소리 등을 접하는 것도 좋다.
\vspace{5mm}

탁월한 지능은 선명한 감수성을 필연적으로 수반할 수 밖에 없다.
수능문제도 그것의 맛과 향이 있고 촉감이라는 게 있다.
하다 못해 2차원이라는 고상한 취향의 소유자가 각 문제들을 캐릭터화하는 것도 대단히 좋은 지적경험이다.
\vspace{5mm}

감각을 무시하면 자아는 불완전한 명제로 쪼그라져버린다.
그리고 그렇게 명제화된 사람이 거짓이라고 증명되어버리면?
\vspace{5mm}




\section{[학습공학 030] 후행학습}
\href{https://www.kockoc.com/Apoc/653080}{2016.02.26}

\vspace{5mm}

분명 똑같은 시간을 기울였는데도 강의가 이해가 가지 않고 문제도 안 풀리고
자기가 머리가 안 좋은가보다... 라는 친구들이 많다.
\vspace{5mm}

길게 쓸 필요가 없이 이유는 간단하다. \textbf{"기초"가 없으니까}
건물로 친다면 지반공사가 안 되어있는 상태에서 사상누각을 지어버린 것이다.
바닥이 흐물거리는데 층수를 높일수록 어떻게 되겠나. 더 붕괴되기 좋아지는 것이지.
\vspace{5mm}

그래서 수학에는 오히려 \textbf{'후행학습'}이 필요하다.
수학을 못 하는 이유는 머리가 나빠서도 사교육을 못 받아서도 아니라
엉터리로 잘못 배우거나 연습을 게을리한 채로 스킵한 부분이 계속 '발목'을 잡기 때문이다.
계산 실수를 자주 한다면 초딩 계산 연습을 다시 해보아야 한다.
기하학 문제가 나오면 얼어버린다면 초딩 도형과 중딩 기하도 다시 공부해야 한다.
경우의 수가 약하다? 당연히 수2 집합, 그리고 초딩 수학도 다시 공부하면 된다.
\vspace{5mm}

그런데 이런 후행학습(아마 이 용어도 내가 처음 쓰지 않을까 싶다만)이란 개념 자체가 그리 쓰이고 있지 않다.
거기다가 이상하게도 아랫 학년을 공부하면 자존심이 상한다라는 생각이 있어서인지 이런 후행학습을 하지 않는다.
그러니 끝까지 발목을 잡기 시작하는 것이다.
\vspace{5mm}

기초가 안 되어있어서 이해가 안 가고 문제가 안 풀리는 데도
하류들은 이걸 모르고 \textbf{"인기 강사"의 "족집게 특강"만 들으면 해결될 것이야라고만 믿는다.}
만약 해당 강사가 그런 문제를 알고 강의 중에 후행학습을 시켜준다면 구원받을지도 모른다.
하지만 그런 가능성이 매우 낮다.
게다가 하류들일수록 공부는 안 하고 일류가 본다는 교재만 잔뜩 사들인 채 풀지를 않는다.
후행학습이 필요한데도 이걸 무시하고 명문대 간 친구들이 하는 커리만 따라하면 될 거야... 라고 믿는데 될 리가 있나.
\vspace{5mm}

급할 수록 천천히 돌아가라는 말은 격언에 불과한 게 아니다, 이건 바로 경영이다.
상위권을 달리는 친구들의 교집합은 "기초"가 정말 튼튼하다는 것이다.
계산이 신속 정확하고 문제를 꼼꼼히 읽는 데다가 여태껏 해왔던 교육과정 중 빵꾸가 없으니 실수도 거의 없고 이해도 빠르다.
\vspace{5mm}

기초가 안 되어있는 친구들에게 권하고 싶은 말은 더 걸리더라도 정말 국영수 기초를 제대로 하고 가란 것이다.
하다 못 해 올해 시험에 늦더라도 내년 시험에 대비한다는 생각으로 기초만큼은 정말 단단히 하라는 것.
이걸 안 하면 입시가 문제가 아니라 \textbf{인생 전체가 저당잡히는 상황이 온다}.
기초가 제대로 다 잡혀있으면 그 다음부터 공부하는 속도는 지수함수 형태로 증가한다.
\vspace{5mm}






\section{[학습공학 031] 라이벌을 구입하라}
\href{https://www.kockoc.com/Apoc/655573}{2016.02.28}

\vspace{5mm}

누구나 하는 건 의미가 없는 것이다.
\vspace{5mm}

계획은 누구나 짠다. 그러나 \textbf{계획을 제대로 실천하는 건 힘들다.}
강의와 교재 구입은 원칙적으로 누구나 한다.
그러나 \textbf{강의와 교재를 제대로 활용하는 건 아무나 못 한다.}
\vspace{5mm}

그렇다면 현명한 수험생이라면
계획을 실천하게끔 강제하는 것,
강의. 교재를 적시에 소비할 수 있도록 강제하는 것
이걸 신경쓸 것이다.
\vspace{5mm}

물론 당연히도 이것까지 신경쓴 사람은 없다.
\textbf{계획을 짜면 그걸로 만족해버리고,}
\textbf{강의와 교재를 구입하면 그 다음 방치해놓는 것 역시 누구나 하는 짓이다.}
\vspace{5mm}

그럼 대안은 무엇인가.
수험에 있어서 가장 중요한 건 뭐니 해도 \textbf{"라이벌"}을 잡고 경쟁하는 것이다.
그냥 냅두면 공부 안 하는 인간들도 자기 라이벌이 책을 피면 피로를 잊고 공부하게 된다.
\vspace{5mm}

사람이 신기한 게 평소에는 돈이든 아파트든 사소하게 여기는 사람도
'자기 친구'가 거액을 벌었다거나 아파트를 구입했다라고 하면 의욕이 생긴다.
선의이든 악의이든 라이벌이 있어야만 나태해지지 않고 공부하게 된다.
스트레스 받는다? 공부하기 싫다?  이걸 해결해주는 게 라이벌이다.
라이벌이 있어도 공부하기 싫다... 이러면 어차피 그냥 당사자가 이미 공부를 포기한 것이다.
\vspace{5mm}

실강이 인강보다 나은 게 그거다.
실강을 들으면 라이벌이 확보된다.
돈을 주고 강의 뿐만 아니라 \textbf{'라이벌'들을 구입하는 것}이 실강이다.
그런데 인강은 그게 없다, 수강생들의 진도 현황이 실시간으로 표시되면 좋겠지만 이게 지원되지도 않는다.
그래서 인강이 좋겠네... 하는 친구들이 실강생을 못 따라가는 일이 벌어진다.
그걸 강사들은 자기 강의에 집중한 결과라고 자화자찬하겠지만
실제로는 그것보단 "강제성"과 '라이벌" 차이가 크다고 봐야 타당할 것이다.
\vspace{5mm}

다만 온라인에서도 이제 라이벌 시스템은 구현되지 않을까 싶다.
왜냐면 온라인 게임부터가 라이벌을 이용한 것이 아닌가
(그리고 왜 사람들이 온라인 게임에 중독되는지도 설명된다.
인강은 이제 개인 방송 아니면 온라인 게임으로 진화될 것이다)
\vspace{5mm}

\begin{itemize}
    \item  왜 부촌 집값이 높은지
    \item  엄마들이 기쓰고 아이들을 좋은 학교에 보내려하는지
\end{itemize}
...
등이 이 라이벌론으로 설명된다.
\vspace{5mm}

비싼 교재 여러 권이나 강의를 살 돈이 있으면
차라리 '공부 잘 하는 학생'에게 밥을 사주면서 같이 공부하는 게 낫다.




\section{[학습공학 032] 엉터리 계획술}
\href{https://www.kockoc.com/Apoc/667250}{2016.03.07}

\vspace{5mm}

인강 내용을 기억 못 하는 사람들이 막장 드라마 내용은 잘만 기억합니다.
왜 그러냐면 우리가 배우는 내용에는 \textbf{스토리}가 없지만, 막장 드라마에는 \textbf{스토리}가 있기 때문이죠.
더 들어가면 스토리는 기호와 신화로 환원되어서 우리가 본능적으로 끌리거나 추구하는 근원적인 게 있다... 가 되겠는데
\vspace{5mm}

탁월한 강사라면 딱딱한 전화번호부만 가지고도 스토리를 잘 만들겠죠.
\vspace{5mm}

이해가 되려면 일단 해당 내용이 스토리로 와닿아야 합니다.
그리고 그 스토리는 학습자 본인에게는 \textbf{'이미지'}로 새겨져야합니다.
저 스토리와 이미지가 수험지식의 양대 축이라고 해도 틀린 이야기가 아닌데 이걸 모르시더군요.
공부해 본 사람이면 다 고개를 끄덕거리겠지만, 안 했거나 대충 한 사람들은 이걸 모릅니다.
\vspace{5mm}

수학에서 꿀교재만 고르면 1등급 나온다... 그런 게 어딨습니까. 거짓말이지
물론 한권만 보더라도 올라가는 천재들이 없지는 않은데, 이 친구들도 좋은 가정에서 조기교육으로 두뇌가 완성된 케이스입니다.
평범한 친구들이 올라가려면 기본교재를 3회독하거나 아니면 3권 정도는 돌려서 천천히 기초부터 쌓아야 합니다.
기초적인 것을 반복해 숙달하고 그것들의 의미를 재미있게 파악하고(스토리), 훈련 과정에서 형상화하면(이미지)
그 다음에는 굳이 인강을 듣지 않아도 어려운 문제를 스스로 해결할 힘이 키워집니다.
이렇게 하려면 상당히 많이 시간을 할애해야하며, 중간에 지치지 않도록 단위시간당 학습량은 줄여야합니다.
(그래서 일찍 시작하라는 겁니다. 3개월이 확보되고 안 확보되고는 차이가 상당히 크거든요)
\vspace{5mm}

실패하는 흔한 유형 -
\begin{itemize}
    \item ⓐ xx만 보면 되지 않나요
    \item ⓑ 저 하루 12시간 할 건데
\end{itemize}
\vspace{5mm}

일단 ⓐ 하는 인간들은 걍 망하기 딱 좋죠. 조금만 생각해도 말이 안 된다는 걸 알 건데
정말 ⓐ가 가능했다면 그 전에 대치동 소수정예 학원에서 더 엄선된 교재로 가는 친구들 다 명문대 갔겠네요.
현실적으로 그런 일은 없습니다. 실적이 좋은 케이스는 선행과 후행이 정말 잘 되어있어서 거의 10년치 사교육을 한 친구들이죠.
분명 전 올해 초에 풍산자부터 차근차근 보라고 했는데도 이걸 안 지키고 지금 와서 후회하는 사람들도 있으신데
'단기간에 빨리 끝내려는 망상' 버리시고, 기본부터 하시기들 바랍니다.
\vspace{5mm}

특히 ⓑ가 문제입니다. 아니 평소에 공부도 안 하는 친구들이 12시간을 참 잘도 하겠죠.
자기가 그렇게 대단한다고들 착각들 하시는데 실제로 하루에 순공부 6시간만 지키는 것도 어렵다는 걸 나중에 알게될 것입니다.
문제는 지속력입니다. 본인이 계획을 잘 짜는 사람이라면 하루에 6시간 할 공부도 하루에 4시간 하는 걸로 줄일 수 있어야 합니다.
계획술의 핵심은 그 실행자의 피로도를 줄이고, 100$\%$ 이행가능하도록 분할, 분배하는 것입니다.
\vspace{5mm}

흔히 이런 말들 하죠. 공부 잘 하는 애들은 널럴하게 공부하더라. 그거 머리좋은 것 아냐?
이 광경에서 100$\%$ 확실한 건 그런 말 하는 친구들은 정말 머리가 나쁘다는 것입니다.
공부 잘 하는 애들이 '널럴하게 공부할 수 있는 스케줄'로 공부하니까 공부를 잘 한다는 생각을 전혀 못 하는 것이죠.
왜냐? 공부는 빡세게 어렵게'만' 해야한다는 고정관념에 사로잡혀있고, 본인들이 공부를 안 해보았으니까요.
\vspace{5mm}

공부 잘 하는 친구들이 하루에 널럴하게 공부해도 될만큼 스케줄링이 되어있다면, 이 친구들은 100$\%$ 계획을 이행할 수 있단 이야기가 되는 거죠.
남들보다 일찍 시작했고 클라이막스에 해당하는 구간마다 사교육 지원 잘 받고 그런 위기를 잘 극복하면서
공부해야하는 총량을 나눠주는 시간을 늘렸기 때문에, 단위기간동안 공부해야 하는 양은 남보다 적으면서 학습총량이 늘어나는 것입니다.
그래서 널럴하게 공부하면서 스트레스 덜 받고 계획은 100$\%$에 가깝게 이행하면서 실패없이 자기 학습량을 늘려나가 복리효과를 누리는 것이죠.
\vspace{5mm}

반면 풋내기들은 무리하게 계획을 잡습니다. 당연히 이행률은 결국 0$\%$에 수렴하고, 그동안에 들인 노력과 시간은 매몰비용이 되어버립니다.
계획이 번번히 실패하니까 실패와 불행에 중독되어버리고, 하루 순공부 12시간은 해야지 라는 \textbf{'말'}만 합니다.
그리고 교재와 인강타령만 하고 있죠(공부 잘 하는 애들은 선택고민 없이 진작 다 끝냈을 건데)
무엇보다도 여전히 공부는 빡세게 해야한다, 그런데 자기는 빡세게 못 하네, 공부할 팔자가 아닌가봐... 하면서 스스로 망해가기 시작하는 거죠.
공부 잘 하는 애들이 '널럴하게 공부하는 게 비결'이라는 쪽으로 생각만 했더라도 인생이 꼬이지 않았을 건데 말이죠.
\vspace{5mm}

지금 시점으로 본다면 3월이니까 최대한 시간을 많이 확보하고 수능에 필요한 공부량은 줄이는 것(가령 선택과목 선별)을 해서
스트레스 관리 잘하면서 계획을 100$\%$ 이행하는 그걸로 잡아야지, 계속 인강 교재 타령하다간 2017 바라보면서 피눈물이나 흘리겠죠.
오히려 현명한 사람은 2018을 목표로 삼으면서 그에 필요한 공부량을 다 계산한 뒤 역산해서 하루에 얼마나 공부할 수 있을까 추린 뒤
이걸 더 줄이는 방향을 강구해서 '널럴'하게 갈 수 있게 계획을 짤 텐데 말입니다.
\vspace{5mm}

"아니, 그럼 널럴하게 공부해야만 한단 말인가"
\vspace{5mm}

그게 아니죠. 널럴하게 짜놓는다는 건 다시 말해서 공부가 정말 안 되는 컨디션임에도 계획을 100$\%$ 달성할 수 있게 세팅해놓는다는 이야기입니다.
본인이 하다가 더 할 수 있다는 자신감이 확보되면 공부할 것을 더 늘릴 수 있겠죠. 그건 본인 자유가 아니겠습니까.
물론 무턱대고 추가하는게 아니라 2$\sim$3개월 하면서 확실히 안정되었다라고 하면 늘리겠죠.
\vspace{5mm}

전쟁에서는 결국 적보다 많은 병력을 동원하는 게 이긴다고 하죠(란체스터 법칙)
이건 수험도 마찬가지입니다. 더 많은 시간을 확보하고 더 편리한 환경을 확보해놓는 게 우선입니다.
그러려면 누구보다 먼저 시작해야 하고 그 다음 조금이라도 방해가 될 수 있는 요소는 제거해야합니다.
그런 다음에야 정신으로 승부한다고 할 수 있는 것인데
\vspace{5mm}

대부분은 자기를 초인으로 가정하고 단기간 내에만 끝낸다라고만 마음먹으면서 자기를 학대하려 하죠.
다들 매일 컨디션이 최상인 것으로 가정하고 비현실적인 계획을 세워놓고 그걸 실천 못 하고 좌절하는 바보짓을 합니다.
일본인은 초식동물이니 풀만 먹어도 된다라는 식의 발상으로 가는 게 아니겠습니까.
\vspace{5mm}







\section{[학습공학 033] 수험 RPG}
\href{https://www.kockoc.com/Apoc/675926}{2016.03.14}

\vspace{5mm}

가끔 워3 카오스를 한다. 물론
\vspace{5mm}

'너 왜 디 안 하냐'
'네가 사람새기냐. 이 발암종자야'
'이 녀석 때문에 게임 못 하겠다 판뜨자'
\vspace{5mm}

라는 온갖 악플(?)에 시달리지만 이 게임이 재밌는 이유는 간단,
승부 룰이 정말 긴박감이 넘치기 때문이다(개인적으로는 롤보다도 재밌다능)
그런데 카오스 입문을 할 때 자주 맛본 악플이 있다.
\vspace{5mm}

\textbf{'이건 RPG가 아냐 이 xx야'}
\vspace{5mm}

카오스는 승률 계산이 정말 중요하다.
5:5 할 때의 캐릭터 밸런스(스턴, 테러 캐릭터 확보. 힘캐의 적절한 뒷받침)도 중요하지만
초반에 협조플레이를 잘 해서 킬수를 올려 상대방이 렙업을 못 하도록 방해하거나
혹은 상대방의 그런 시도에 당하지 않아야 한다.
그래서 라인에서 렙업하다가 사냥 도는 적캐릭터들이 급습할 것 같은 예감이 되면 잘 튀어야 한다.
그리고 아군이 한판할 때 잘 합류해줘야하며, 적들이 테러 들어올 때 역테 들어가는 건 당연하다.
\vspace{5mm}

이렇게 계속 실시간으로 머리를 굴리며 학대하는 게 재미인데
\vspace{5mm}

가끔 보면 싸움을 하기보다 \textbf{한가롭게 사냥질을 하면서 아이템 쇼핑을 하는 캐릭이 있다.}
이걸 보고 "RPG하지마"라고 하는 것이다. 이런 아군이 있으면 그 판은 물건너간다,
이런 애들이 적군에 200원을 제공하는 고문관이니까.
\vspace{5mm}

공부와 수험의 차이를 여전히 모르는 사람들이 있다.
그리고 이 사람들은 "공부만 하면 좋은 대학에 가기 힘들다"라는 것이 뭔 소리인지 이해를 못 한다.
이 공부가 바로 CHAOS의 RPG 행태와 똑같은 짓인 걸 모르는 것이다.
여기서 말하는 공부는 경쟁을 망각한 학습행위고
수험은 \textbf{경쟁자들을 의식하고 그들을 쓰러뜨리는 학습행위다}.
\vspace{5mm}

왜 학원에 가는 게 승률이 좋은지 아나? 거기는 경쟁자들이 보이니까. 경쟁자들을 의식한 학습행위를 하게 된다.
반면 독학이나 인강은 경쟁자들이 보이지 않으므로 '공부 RPG'를 하기 딱 좋다.
마찬가지로 어떤 교재가 좋아요 어떤 강의가 좋아요 극혐질문하는 것도 마찬가지다.
그런 질문할 바에는 실제 경쟁자들이 뭘 공부하고 있는지 탑 시크릿을 캐내면 되는 것이 아닌가.
\vspace{5mm}

다른 이야기해보면 주식과 부동산도 마찬가지다.
주식도 가치투자니 기술투자니 하지만 교집합들을 구하다보면 '작전세력'을 읽는 게 답이라는 걸 알게된다.
(심지어 작전에 해당하는 자들이 가치투자라고 얘기하고 있는 게 현실이다)
가치투자가 유일한 길이라고 믿는 사람들은 재무제표 보고 큰 돈 묻어두었다가 예측과 다른 하락에 개당황해  기술투자로 갔다가,
처음에는 돈을 벌지만 그 다음에는 또 하락을 맛보면서 거액을 털린 뒤에야 아 역시 투자도 '경쟁'을 전제하는구나라는 진실을 늦게야 안다.
부동산 역시 그렇다, 눈치에다 후각, 거기다가 여성 특유의 육감을 갖춘 복부인들이 거의 다 해먹는다.
당연히 이들도 자기들이 마크 대상인 걸 알기 때문에 아주 은밀히 움직이면서 정부정책과 법령을 연구해 잽싸게 움직인다.
그렇게 프리미엄까지 따놓은 다음 그걸 호구들(=장래 하우스푸어)에게 팔아먹어 벌어대는 것이다.
\vspace{5mm}

수험 RPG를 하는 사람들은 정말 좋은 코스가 공개되는 걸로 착각한다.
말하지만 그런 것은 웬만하면 드러나지도 않는다, 누가 비급을 대놓고 싼 값에 팔겠나.
설령 비급이 있다고 해도 그건 본인들이 열심히 공부해서 스스로 발견하는 게 더 빠르고 효율적이다.
이 경우 현실적인 건, 공부 잘 하는 친구들이 하는 코스는 다 완료하고 그 다음 프리미엄을 추가하는 것이다.
물론 '다 풀면 된다'에 덧붙일 수식어는 \textbf{"남보다 빨리, 더 신속히, 더 많이"}
초반에 렙업 빨리 하고 남이 먹을 사냥감을 자기가 먹어버린 진영이 승리할 가능성이 높다.
\vspace{5mm}

최고의 스케줄? 남들이 고3 때 하는 것을 미리 고1이나 고2 때 끝내놓는 것이지.
아무리 머리 써보았자 1$\sim$2년 일찍 빨리 끝내고 회독수 높인 것을 이길 방도는 없는 것이다.
그런데 이런 단순한 진리를 깨닫지 못 한 "RPG 플레이어"들이 정말로 많다.
마찬가지로 어느 교재를 봐야하나요라고 하기보단, 그 교재들을 가능하면 다 보도록 '시간확보'하는 것이 낫다.
하지만 이 친구들은 이런 단순한 해답을 모른다, 뭔가 한권만 보면 될 거라고 착각을 한다.
\vspace{5mm}

자신의 공부가 정말 바른 길인지 아는 방법은 상위 1$\%$가 어느 정도 할 것인가 상상해보는 것이다.
신기하지만 이건 직접 보지 않아도 공부 안 한 자신들도 안다. 상위권들이 어떻게 공부할지.
아주 지독하게, 집요하게. 그리고 문풀이 남들의 수십배를 호가하며 항상 뭔가 읽고 있고 암기하고 있을 것이다....
이건 너무 당연하지 않나, 그럼 이대로 실천하면 된다, 하지만 진짜 실천도를 보면 그렇지 않다.
\vspace{5mm}

물론 고렙도 가끔 킬당하는 경우가 있다. 다시 말해 뛰어난 수험생도 실전에서는 망가질 수 있는 것이다.
그러나 n수의 원인은 n수생들 스스로가 고백한다. "저 공부 열심히 했는데요 $T_T$"
당연히 공부는 열심히 했겠지, 다시 말해 공부 RPG는 했지. 하지만 \textbf{경쟁 게임은 하지 않았지.}
공부만 열심히 하면 뭐하나, 결국 고렙 캐릭터들을 킬하지 못 하는데.
\vspace{5mm}

공부와 경쟁 중 그럼 뭐가 우선시되느냐
공부를 한다고 경쟁을 하는 건 아니다. 다시 말해 공부의 마스터베이션으로 끝날 수도 있다.
반면 \textbf{경쟁으로 들어가면 공부는 하게 된다.}
학원강사들은 절반의 거짓말을 한다. \textbf{독학으로는 불가능하다, 강의를 들어야만 이해할 수 있다.}
독학이 어려운 건 사실이다. 하지만 그건 강의가 좋아서가 아니라, 강의를 듣는 \textbf{라이벌의 존재가 경쟁 상황을 조성해주기 때문이다.}
\vspace{5mm}

남들보다 머리가 좋다는 건 이미지에 숙달된 것이지만
한없이 공부해도 즐겁다라는 것은 두가지이다.
하나는 정말로 그 학문의 마니아나 오덕이 된 경우다.
\textbf{다른 하나는 수험경쟁게임에서 남들을 킬하고 자기가 노데스인 상황을 즐기는 것, 즉 수험경쟁에서의 승리에 도취한 경우디ㅏ.}
\vspace{5mm}

저걸 깨닫는 건 나이와 상관이 없다.
20대 중반을 넘어서도 경쟁임을 여전히 모른 채 공부 RPG에만 빠져있는 케이스야 셀 수 없이 많고
이제 여드름이 막 나려고 하는데에도 시험으로 남보다 우위에 서는 것 자체를 즐기는 케이스도 널렸으니까.
\vspace{5mm}





\section{[학습공학 034] 교재 집착증}
\href{https://www.kockoc.com/Apoc/678660}{2016.03.16}

\vspace{5mm}

회사의 경영상태는 재무제표를 보면 알 수 있고
학생의 공부상태는 서재와 가방을 보면 대략 짐작할 수 있다.
우선 특정과목의 교재가 5권 이상 넘어가는 데 '깨끗'하다면 십중팔구 망해가는 코스다.
반면 특정과목의 교재가 기본서-기출-연습서 정도만 구비되어있고 '필기'로 더럽다면 망하지는 않는다고 할 수 있다.
\vspace{5mm}

회계에서 순이익 \textbf{조작}은 재고자산을 어떻게 인식하고 계산하느냐에 따라 달라질 수 있는데
수험에서도 교재들을 많이 구비하는 것을 '공부'한 것으로 인식하여 본인의 수험 실적을 조작할 수 있다.
뭐 윗 내용이야 상당히 보편적인 것이니 새로울 건 없는데 하나 추가하자면 '팬심'이다.
\vspace{5mm}

그런데 재밌는 건 n수생 중에서 특정 저자의 특정 교재에 지나치게 집착하는 경우가 있다.
그리고 그런 곳을 구경해보면 \textbf{수년 전부터 특정 교재를 보았다}라고 말하고 있는데 이거 공포스러운 것이 아닌가.
\vspace{5mm}

수험생들을 관찰해볼 때 이런 부류가 있다.
\vspace{5mm}
\begin{itemize}
    \item 특정 강사의 A 강의가 좋다 소리치면서 지나치게 집착
    \item 특정 저자의 B 교재를 무조건 완비해서 그걸로 수능을 완벽히 대비한다.
\end{itemize}

\vspace{5mm}

이 케이스도 반드시 망한다.
저런 친구들은 일단 수험이 뭔지도 잘 모르는 것이고
아울러 특정 교재나 강의에 집착하면 스케줄을 지키지 못 하기 때문이다.
\vspace{5mm}

정말로 공부 잘 하는 친구들은 교재에 집착하지 않는다. 말 그대로 교재는 도구이고, 강의는 그저 안내일 뿐이기 때문이다.
책도 굳이 많을 필요는 없다. 중요한 건 자기 머릿 속에 얼마나 많이 들어가 있느냐이지, 서재가 얼마나 풍부하냐하는 것이 아니기 때문이다.
머리에 들어가지 않은 교재는 그냥 재고덩어리나 뱃살에 불과할 뿐이다.
그렇다고 굳이 단권에 집착하지 말라는 건 아니다.
교재 20권이 있어도 그걸 발췌독만 하면서 필요한 것만 소화시킨 경우라면 이야기가 달라지니까.
중요한 건 기본적인 교재의 개념과 패턴이 일단 당사자의 머릿 속에 들어가 있느냐 하는 것이다.
\vspace{5mm}

이게 된 사람들은 교재나 강의에 그리 집착 안 한다. 걍 좋은 게 있다하면 '보충용'으로나 듣고 필요한 것만 발췌해서 가져간다.
하지만 머릿 속에 공부가 되어있지 않은 사람은 특정 교재를 완수하고 특정 강의를 완강해서 수능만점에 간다는 판타지를 계속 연상한다.
개인적으로 교재나 강의 질문하는 사람들에게 매몰차게 구는 이유가 그런 판타지에 빠진 환자들이 널렸기 때문이다.
\vspace{5mm}

\textbf{공부가 어느 정도 되고나면 두뇌가 자동학습 상태로 들어가고 그 뒤부터는 탐욕스럽게 해당 지식을 흡수한다.}
\textbf{그러나 두뇌가 알파고(...)와 비슷한 상태가 될 때까지 계속 연습하고 반복한다.}
\vspace{5mm}

라는 건 공부를 열심히 한 사람이면 경험적으로 알고 있지만, 공부를 해보지 않은 애들은 죽을 때까지 모를 수도 있다.
그럼 왜 이런 당연한 게 전파가 안 되냐하면, 잘 하는 친구들은 '그거 너무 당연하다' 생각해서 말을 안 하니까.
뇌를 학습지속상태로 만들어야한다는 걸 모르는 친구들은
업자들의 광고에 의존하면서 그 업자들이 말하는 풀셋이라는 데 미친 듯이 집착하는 것이다.
그리고 그걸 이용해서 돈을 버는 업자들은 어떻게 해서 수험생들을 속일까하는 것에 골몰한다.
\vspace{5mm}




\section{[학습공학 033] 성공부등식과 실패부등식}
\href{https://www.kockoc.com/Apoc/683846}{2016.03.19}

\vspace{5mm}

박승동이 제시한 회독은 n회독때도 전체 책을 다 보는것 모 정신과 의사가 제시한 방법은 1회독때 겉훑기 2회독때 발췌독 3회독때 정독 모 고시 3관왕은 1회독때부터 단권화해서 2회독부터 전체를 다 읽는것 사람마다 다 회독공부법이 다른데, 갑자기  "이런거는 다 아는줄 암" 이렇게 말씀하시면 수험생들 입장에선 어리둥절해 할 수밖에없죠
\vspace{5mm}

고시 3관왕 시절과 지금은 참고서부터가 다릅니다요.
지금 원칙적 폐지 상태인 사법시험 1차의 민법기본서는 2500쪽에 달하는 것으로 알고 있습니다.
그런데 과거의 민법은 많아보앗자 300쪽이었다고 알고 있습니다.
\vspace{5mm}

과거에 공부했던 사람들이야 그 기준대로만 생각하시 쉬워서리.
수학의 정석도 과거 것은 양이 적은 편입니다. 그래서 현재보다 부담이 덜하죠.
그런데 지금은 어느 시험이건 참고서가 두껍습니다. 이걸 1회독 때 전부 다 본다는 건 미친 짓입니다.
처음에 전부 다 훑어도 되는 것은 '수능특강' 수준입니다. 물론 전 이것조차도 분할해서 회독수 높이라고 권하겠습니다.
\vspace{5mm}

공부를 못 하는 이유는 간단합니다.
단위시간당 학습량과 학습수준 > 자신이 소화시킬 수 있는 단위시간 당 학습량과 학습수준
이라는 실패부등식에 따라 행동하기 때문입니다.
실패부등식대로 하면 그 날 학습은 성과가 안 나옵니다 스트레스가 쌓이지요.
\vspace{5mm}

공부를 잘 하기 위해서는
\textbf{단위시간당 학습량과 학습수준 <= 자신이 소화시킬 수 있는 단위시간 당 학습량과 학습수준}
이라는 성공부등식을 지켜야합니다.
성공부등식대로 가면 공부했단 생각이 안 들지도 모릅니다만, 스트레스도 덜 쌓이면서도 정말 안전하게 자기 계획을 이행하는 것입니다.
\vspace{5mm}

그런데 다들 의욕이 가득차서 실패부등식대로 가니 실패할 수 밖에 없는 것이지요.
공부 잘 하는 애들은 성공부등식을 충족시킬 수 있는 상태니까요.
자신이 실력이 없고 초보일수록 더 시간을 많이 잡아서 단위학습량 및 수준을 줄이고 천천히 가야합니다.
원래 회독수 학습법도 일본에서 나온 것으로 알고 있고 일본 쪽에서도 이렇게 얘기했는데 국내에서는 뭔가 왜곡된 것 같습니다(...)
\vspace{5mm}

여담이지만 국내 강사들이 말하는 학습법을 따라가는 애들이 왜 소수만 성공하는가, 즉 다수는 실패하는지 답이 나오지요.
의지와 열정을 강조하는 건 좋은데 그 나머지 실패부등식대로 공부하는 결과를 가져오기 때문입니다.
조국을 위한 뜨거운 열정만 있으면 일당백할 수 있단 소리인데 무슨 드라마 찍는 것도 아니고.
\vspace{5mm}

원래 회독 학습법은 에빙하우스의 망각곡선에 기초한 겁니다.
처음부터 완벽하게 다 읽어보았자 시간지나면 까먹습니다. 복습을 해야 망각을 줄입니다.
그 복습을 세는 횟수가 바로 회독수입니다.
이렇게 기억을 되살리는 동시에 처음에 무리하지 않게 나선형으로 성과를 내기 위해서는 난이도를 나선형으로 천천히 높여가는 것입니다.
이런 의미로 회독수를 늘리는 것인데 다수가 처음부터 완벽하게 다 읽는다라고 착각했으니(...) 그러니 공부 실패하는 것도 무리가 아닌 겁니다.
\vspace{5mm}

강의가 좋은 건 '핵심'만 짚어주기 위해서입니다. 핵심을 짚어준다는 건 양을 줄여주는 것이고
양을 줄인다는 건 즉, 자신의 능력 내로 학습량을 줄여주므로 성공부등식을 충족시켜준다는 이야기입니다.
혼자서 책을 읽을 경우 자기 능력을 고려하지 않고 학습량을 늘려서 진도는 진도대로 못 나가고 스트레스만 쌓일 수 있습니다.
즉 명강의가 소화가 잘 되는 가장 원칙적인 비결은 "양을 줄여주기" 때문입니다.
하지만 결국 나중에는 양을 다시 늘려야한다는 과제는 남아있는 것이지요.
\vspace{5mm}

수학의 정석이 문제인 게 이것일 겁니다.
학습자가 계획을 잘 짜서 분할을 하면 괜찮겟지만 보통의 학생에게 그런 능력을 기대하기 힘들 겁니다.
온갖 내용들이 압축적으로 들어가 있습니다. 그래서 초심자는 쉬운 내용으로 나선형 상승을 하는 게 아니라
바로 에베레스트와 맞닥뜨리고 좌절해버리죠.
\vspace{5mm}

학습계획은 결국 학습량/시간 이란 분수에서 시간을 적절하게 늘려 단위학습량을 줄이는 것에 생명력이 있습니다.
하지만 다수의 친구들은 단위학습량 ↑ 이란 망상에 빠지죠. 이게 현실적으로 가능하지는 않습니다
공부 잘 하는 친구들이 한시간에 100페이지를 본다면, 초심자들은 한 시간에 5페이지에 해당하는 분량만 봐야합니다.
그리고 쉬운 내용으로만 해서 전반적인 진도를 훑어나가야 개략적인 것을 알 수 있습니다.
\vspace{5mm}

여담이지만 미국의 학습법에 나온 SQ3R이 있습니다만 이건 국내 실정과 안 맞습니다(대학교 과정에서는 좀 맞을지 모르지만)
\vspace{5mm}











\section{[학습공학 034] 수평적 사고(lateral thinking)}
\href{https://www.kockoc.com/Apoc/702963}{2016.03.30}

\vspace{5mm}

차근차근 계단식 추론을 통해 엄격하게 결론을 도출해나가는 게 수직적 사고이다.
대표적인 것이 대전제-소전제-결론의 삼단논법이며 초창기 수능이 아닌 현재 수능 수학에서 요구하는 사고법이다.
이 수직적 사고와 다른 것이 수평적 사고 lateral thinking 이다.
\vspace{5mm}

\url{http://terms.naver.com/entry.nhn?docId=2178257&cid=51072&categoryId=51072}
\vspace{5mm}

소위 창의적인 사고는 저걸로 나오는데 문제는 현행 교육과정이든 사교육이든 그 어디서든 가르치지 않는다.
사실 이건 가르칠 수 있는 것이 아니다.
\vspace{5mm}

ex 1) 오렌지 10개를 샀다. 집에 가서 질투심들이 많은 아이 셋에게 나눠주려고 한다.
오렌지를 하나라도 버리거나 내가 먹어서도 안 되고, 한명에게 몰아줘도 안 된다. 어떻게 할까?
\vspace{5mm}

ex 2) 학원 입시설명회가 너무 각광을 받은 나머지 원생들이 우르르 몰려들어 자칫하면 사고가 나기 좋다.
입구에서 강의실까지만 천천히 걷게 하고 싶은데 어떻게 하면 될까?
\vspace{5mm}

ex 3) 아래가 절벽인 커브길이 있다. 많은 예산을 들여 온갖 장치를 해도 과속을 하는 차들 때문에 사고가 난다.
사고가 나지 않게 하는 방법이 무엇일까?
\vspace{5mm}

이런 질문에 대한 대답을 수능수학에서처럼 접근하면 꽉 막힌 답안이 나온다.
래터럴 씽킹의 핵심은 우리가 사로잡혀있는 기존의 가치관이나 프레임의 맹점을 공략해나가는 것이기 때문이다.
실제로 우리가 신봉하고 있는 이론은 현실에서는 정반대의 결과가 나온다. 고려하지 못 한 변수들도 많아서이지만
이론 자체가 상아탑에서나 먹히는 '그럴 듯한 착각'인 경우가 많아서이기 때문이다.
이 점을 인정하고 그 매트릭스에서 벗어나면 래터럴 씽킹적인 결론이 나오게 된다.
\vspace{5mm}

입시로 친다면 초기 90년대, 2000년대 초반 수학까지는 이른바 '야매' 전술이 먹힌 것만 봐도 된다(특히 극한)
말이 야매지만 실제로는 현실에서는 야매라고 보기 어렵다. 이 야매가 바로 래터럴 씽킹이기 때문이다.
다만 고교수학은 그것들이 틀리지 않았으며 교과서적 방식이 무조건 옳다라는 신념 하에 푸는 것이 윤리이고
평가원은 이러한 윤리를 준수하기 위한 온갖 노력을 아끼지 않기 때문에 근 5년동안은 래터럴 씽킹이 먹히지 않게 해놓았다.
하지만 래터럴 씽킹이 안 먹힐 리는 없지 않나. 다만 래터럴 씽킹을 하려면 교과서의 3배를 넘어서는 다른 과정까지 공부해둬야한다.
그럴 바에야 래터럴 씽킹을 포기하고 그냥 교과서대로 가는 게 '나은' 방법인 것이다
\vspace{5mm}

하지만 국어, 영어, 그리고 탐구는 이 래터럴 씽킹이 먹힐 수 있다.
'닫힌 집합'인 수학과 달리 국어, 영어, 탐구는 '열린 집합'이기 때문이다.
출제자의 의도라는 것이 수험외적인 내용에 한수 굽히고 들어갈 수 있는 것이 저 과목들이다.
이것들은 때로는 출제 오류, 복수 정답으로 나타나기도 한다.
\vspace{5mm}

수학 공부에 있어서 교과서적인 것의 맹신에 거부감을 느끼고 다른 것을 찾으려는 사람들이 많다.
물론 부질업는 일이다. 래터럴 씽킹은 우리의 기본적인 인식과 사고 자체를 뒤엎고 더 고차원적인 것으로 재구성하는 것이지,
이상한 꼼수를 암기하거나 야매 풀이의 집착 그 자체가 아니기 때문이다.
그런데 학원가나 해괴한 짜깁기 교재는 래터럴 씽킹을 어떻게 하느냐 가르치기보다는
그런 래터럴 씽킹에서 나오는 부산물을 비싼 값에 팔아먹는다.
물론 아무 것도 모르는 학생들은 그 부산물을 암기하기만 하면 점수가 잘 나올 거라고 단단히들 착각한다.
\vspace{5mm}

아래는 뭔가 타당해보이는 답
\vspace{5mm}

answer 1) 오렌지들을 갈아서 쥬스로 만들어 나눠주거나, 오렌지들의 씨앗을 심어서 열매를 맺게 한 뒤 나눠주는 것을 꾀한다
answer 2) 입구에서 강의실 사이에 서서 무료로 '자료'를 나눠준다. 자료를 받기 위해 천천히 걸어가게 될 테니까
answer 3) 모든 차선을 없애버리고 펜스도 최소화한다. 운전자들은 그걸 보고 신중해져서 감속하게 될 것이다(실제 사례 있음)
\vspace{5mm}

+ 고교수학에서 배울 수 있는 건 '수직적 사고' 그 이상 그 이하도 아니다.
이걸로 장사하는 사람들은 입시수학이 이런 것도 저런 것도 가능하다 하지만 개뻥이다. 그거 잘 해보았자 참고서 팔아먹기 그것 빼고는 안 됨.
수학이 뭔 소용이 있나요라고 질문하는 게 정상이다. 그럼 답변은 "수직적 사고" - 라고 해주면 된다.
\vspace{5mm}

++  수평적 사고는 가르치는 교육기관도 커리큘럼도 학위도 없다.
그러나 수평적 사고를 잘 하는 사람이 현실에서 성공한다. 현실에서 부딪치는 문제들은 수직적 사고로 안 풀리거나 무한 재귀인 경우가 많다.
소위 이걸 '눈썰미가 좋다', '수완이 대단하다'라고 표현한다.
\vspace{5mm}

+++ 써놓고 보니까 이게 꽤 중요한 떡밥인데.
왜 이공계가 심오한 것을 배우는데 현실에서는 쩌리가 되어버리는가... 에 대한 단서일 수도 있다.
실제로 변태적인 수준으로 수학을 공부한 사람들이 상아탑을 벗어나면 현실에서는 너프당하는 중요한 이유일 수도.
\vspace{5mm}

++++ 에디슨이 조수에게 전구의 부피를 구하라고 했던 일화 검색해서 찾아보시길.
\vspace{5mm}






\section{[학습공학 035] 스피드}
\href{https://www.kockoc.com/Apoc/702978}{2016.03.30}

\vspace{5mm}

이 스피드야말로 정말 오답의 원흉이다.
물론 다 빨리 푼다고 틀리는 것도 아니고, 천천히 푼다고 다 맞는 것도 아니다.
그러나 "틀보스"라는 말 그대로. \textbf{틀리고 보니 스피드}인 것이다.
\vspace{5mm}

그런데 이러면서 나는 빨리 풀어야한다고 말한다. 글을 읽는 사람은 '그래서 어쩌라고'라고 한마디할 것이다.
\vspace{5mm}

우선 빨리 풀게되는 이유가 무엇일까?
콕콕에 들어오는 사람들은 일단 우리 뇌의 도그마적 명제 2가지를 정리해 볼 필요가 있다.
\vspace{5mm}
\begin{itemize}
    \item 명제 1 - 우리 뇌는 자아정체성 유지를 위해 학습을 싫어할 수 밖에 없다.
    \item 명제 2 - 우리 뇌는 지루한 것을 피하기 위한 온갖 시도를 한다.
\end{itemize}
\vspace{5mm}

스피드도 착한 스피드와 나쁜 스피드가 있다. 소위 착한 뇌물, 착한 독재... 이런 이야기는 아님.
\vspace{5mm}

성실히 일해서 번 일당 50,000원은 착한 돈이지만, 불법 성매매를 하거나 사기쳐 번 500,000원은 나쁜 돈이다.
돈의 액수만 보는 사람들은 500,000원에 눈이 멀어서 그걸 벌려고 한다. 당연히 인생이 잘 풀릴 리가 없다, 그건 나쁜 돈이기 때문이다.
돈의 본질은 재화와 서비스의 흐름이다. 그 흐름은 인간의 행위고 따라서 윤리가 없을 수 없다.
나쁜 돈을 버는 사람은 나쁜 흐름에 휘말리는 것이다.
소위 수험가에서 돈많이 번다 어쩌구 그거 부러워할 게 아니라 동정해야한다고 하면 '너 정신승리지'라고 하는 한심한 사람들도 있지만
그 사람들은 '시간' 축으로 사고할 줄 모르는 듯. 나쁜 돈을 버는 사람이고 거기다가 무능한 사람이 어떻게 작살날지 모르지.
\vspace{5mm}

스피드도 그렇다.
\vspace{5mm}

나쁜 스피드로 공부하는 사람들은 당연히 오답이 나온다.
왜냐면 그건 '필수적으로 밟아야 하는 과정'을 생략해서 속도를 높인 결과이기 때문이다.
수학 문제를 '대충 읽는다'거나 써야할 풀이과정을 안  쓰고 조건도 정리 안 하며 계산도 분수도 모르고 암산으로 한다거나
아울러 검산까지 안 하면 오답이 나와야 정상인 게 아닌가?
그런데 몰지각한 사람들이 그렇게 필수적인 것을 빼고 '빨리 풀어야' 좋다는 식으로 잘못 가르쳐 나쁜 스피드를 전파시킨다.
이 나쁜 스피드가 배이는 시점이 바로 중딩 때이다. 문제는 중딩 수학에서는 이 나쁜 스피드가 먹힌다는 것이다.
그리고 이 친구들은 자기들이 잘한다고 깝죽대다가 고등학교 때부터 확 발려버리기 시작한다.
\vspace{5mm}

그럼 좋은 스피드는?
\vspace{5mm}

그건 처음부터 천천히 꼼꼼히 풀어나가면서 회독수를 늘려나가면 저절로 생긴다.
근거는? 그건 위에서 말한 제2명제를 떠올리면 된다.
비슷한 문제를 계속 반복해 풀고 회독수를 늘리다보면 '숙달' 과정을 거치기도 하지만, 일단 뇌에서 이걸 지루해한다.
뇌는 지루한 업무를 무의식 부서로 옮겨버린다. 그리고 뇌에서 실수하면 이 지루한 것을 또 천천히 해야하는 것을 알기 때문에
이런 경우는 절대 실수하지 않으려고 하는 것이다.
\vspace{5mm}

다시 말해서 처음부터 차곡차곡 하면서 회독수를 늘리다보면 스피드는 저절로 늘어나게 되어있다.
이거 안 되는 데요... 라고 하면 구라다.
\vspace{5mm}

문제는 나쁜 스피드의 오르가즘에 빠진 친구들이 그 쾌감을 못 버린다는 것이다.
다 포기하고 처음부터 바른 길로 차분히 풀어가면서 좋은 스피드의 세계로 갱생하면 되지 않느냐하는데
마약중독자들 치료만큼이나 매우 어렵다.
게다가 이건 회독수를 늘려야하기 때문에 시간도 많이 잡아먹는다.
\vspace{5mm}

지금 이 시점에 올해 수험 할 수 있어요, 저 아재 말 듣지 말아요하는 사람들도 있겠지만 그 친구들은 잘 모르는 것이다.
우리가 문제를 푼다... 라는 것에는 물경 10년에 걸친 교육과정이 압축되어있다.
똑같은 문제를 푼라고 하더라도 A 머리는 알파고인데 B 머리는 빗살무늬 토기다.
이 현격한 차이를 사람들은 '머리' 차이라고 하지만 엄격히 말하면 어떤 식으로 훈련되어왔느냐에서 비롯되는 것이다.
이걸 머리 차이로만 두는 사람들은 '환경'과 '교육'이 얼마나 중요한 건지 모르는 것이다.
머리가 좋은 친구라도 나쁜 환경에서 잘못 배우면 이상하게 가버린다(재밌는 건 이런 친구들은 롤충들에서 많이 볼 수 있다)
\vspace{5mm}

요약하면 착한 스피드는 하다보면 저절로 키워지는 것, 즉 우리 뇌에서 지루한 걸 피하기 위해서 알아서 확보해주는 것이니
빨리 풀려고 하지말고 천천히 풀어서 다 맞도록 하자.
\vspace{5mm}







\section{[학습공학 036] 시간 감도}
\href{https://www.kockoc.com/Apoc/714448}{2016.04.06}

\vspace{5mm}

책을 읽거나 문제를 풀 때에 가장 중요한 성격 항목을 더 구체화하면
이건 CPU의 클록킹에 비견될 수 있는 '시간 감도'와 관계가 있습니다.
\vspace{5mm}

준비물 : 초침이 있는 시계 ; 나 자신
\vspace{5mm}

절차 :
\vspace{5mm}
\begin{enumerate}
    \item 초침이 있는 시계를 준비한다.
    \item 초침이 정확하게 12를 가리킬 때 눈을 감았다가 꼭 1분이 되었다고 생각했을 때 눈을 뜬다.
\end{enumerate}
\vspace{5mm}

먼저 해보시고. 그 다음 아래 해석을 읽으시길 바랍니다.
\vspace{5mm}

\textbf{$\#$ 마이너스 16초 이상(즉, 초침이 44초가 되기 전에 눈을 떴다)}
\vspace{5mm}

불같이 급한 성격, 이런 사람들은 지나치게 무리하게 일정을 세운다.
그래서 짧은 시간에 이것저것 다하겠다고 과욕을 부리다가 소화시키지 못 하고 말아먹는다.
\vspace{5mm}

$\rightarrow$ 계획을 무리하지 않게 세울 것, 그리고 본인이 계획한 것의 절반 정도로만 실천할 것.
\vspace{5mm}

\textbf{$\#$ 마이너스 15$\sim$6초}
\vspace{5mm}

무슨 일이든 빨리빨리 처리하는 사람이다. 타인이 약속시간에 늦는 걸 용납하지 못한다.
기다릴 줄 모른다, 좀 더 느긋하게 빈 시간을 즐기는 여유를 가질 필요가 있다.
\vspace{5mm}

$\rightarrow$ 계획을 잡을 때 여백을 두실 것. 그리고 쉰다는 것의 효용을 느낄 것.
\vspace{5mm}

\textbf{$\#$ 마이너스 5초$\sim$플러스 5초}
\vspace{5mm}

시간에 대단히 민감한 사람이다. 시간에 대한 의식이 높고 스케줄 관리 능력도 좋다.
그런데도 스케줄 관리가 안 된다면 게으름 때문.
\vspace{5mm}

$\rightarrow$ 게으름만 피우지 마셈
\vspace{5mm}

\textbf{$\#$ 플러스 6초$\sim$15초}
\vspace{5mm}

다소 느긋한 성격이다. 조금 신속한 행동이 필요하다. 스케줄이 느슨하지 않은지 점검하자
\vspace{5mm}

$\rightarrow$ 게으름만 피우지 마셈 2
\vspace{5mm}

\textbf{$\#$ 플러스 16초 이상}
\vspace{5mm}

남들이 15분 밖에 안 남았다고 안절부절할 떄 15분이 남았다고 긍정할 수 있는 사람이다.
어떻게 보면 걸출한 인물이지만, 자칫 잘못하면 멍청한 취급을 받을 것이다.
시간을 의식할 필요가 있다.
\vspace{5mm}

$\rightarrow$ 이런 경우야말로 스탑워치를 써야할 듯.
\vspace{5mm}

학습방법의 문제는 당사자가 어떤 성격인지 구별하지도 않고 플래너 쓰니 잘 되더라 스탑워치 써야한다 그런 건데.
이건 시간감도에 따라 달라집니다.
성급한 사람이 스탑워치 쓰고 공부시간에 환장하면 말아먹죠.
반수해서 올해 수험 쇼부보겠다는 사람들 보고 2년 보라는 근거가 여기에 있습니다.
척 보아도 성격이 급한 사람인데 뭘 해먹을 수 있습니까. 바로 활활 타올라서 미이라가 되어버리지.
\vspace{5mm}

뭐라고? 그럼 아적아 네 놈은 어디에 해당하냐고.
\vspace{5mm}

제가 \textbf{마이너스 20초}입니다. 저도 왕급한 성격이죠. 그래서 \textbf{급한 성격이 어떻게 말아먹나 똑똑히 알고 있는 것}입니다.
\vspace{5mm}

마이너스 파는 그냥 느긋하게 일정 길게 잡고 하루에 최소공부량을 지키다보면 급상승합니다.
반면 플러스 파는 스탑워치 등으로 타이트하게 잴 필요가 있습니다.
\vspace{5mm}

학습방법은 사람의 케이스에 따라 달리할 필요가 있다는 걸 보여주는 좋은 예가 되겠습니다.
\vspace{5mm}




\section{[학습공학 037] 수학 문제를 풀고 정리하는 법}
\href{https://www.kockoc.com/Apoc/723102}{2016.04.11}

\vspace{5mm}
\begin{enumerate}

    \item 채점
    \vspace{5mm}

    채점을 빨간색으로 하는 경우가 있는데 그럴 필요가 없음, 빨강은 "가장 중요한 것" - 즉 틀린 문제나 모르는 문제에만 해야함
    일본 도쿄대 공부법에 보면 파랑색으로 한다는 이야기가 있는데, 그럴 필요도 없습니다.
    그냥 \textbf{검은 색}으로 하면 됩니다.
    펜 색깔 고려해서 이것저것 쓰는 시간도 아까우니 말이지요.
    \vspace{5mm}

    빨강은 정말 다시 확인해야 할 중요한 문제에만 쓰시길 바랍니다.
    \vspace{5mm}

    \item 문제 아래 풀이는?
    \vspace{5mm}

    정식풀이는 자잘한 계산이나 A4에 하고, 문제 아래 여백에는 풀이의 핵심만 한줄 두줄로 \textbf{'키워드'만} 적으시길 바랍니다.
    나중에 다시 훑어보면서 그 키워드만 보더라도 자기가 어떻게 풀었나 기억이 나면 됩니당.
    기억나지 않을 때에야 해설지를 보거나 다시 풀어보기 위해서입니다.
    문제 아래에다가 모든 풀이를 다 적는 경우는 비효율적입니다. 그러면 훑어보기도 어렵습니다.
    \vspace{5mm}

    \item  문제를 다시 풀어야하는가?
    \vspace{5mm}

    처음에 맞춘 문제 - 즉 풀이가 해설과 일치하거나 동등한 - 는 다시 볼 필요는 없습니다.
    자기가 틀리거나 중요하다고 생각해서 빨강표시하거나 형광펜으로 박스칠한 문제만 다시 풀어보시면 되는 것입니다.
    모든 문제를 다 풀 시간에, 틀리거나 중요한 문제를 3$\sim$4번 더 돌리는 게 효율적입니다.
    처음에 맞춘 문제는 다음에도 맞출 확률이 높지만, 틀린 문제나 몰랐던 문제는 또 그럴 가능성이 높습니다.
    \vspace{5mm}

    \item 해설지를 안 보고 답만 보고 채점하는데요?
    \vspace{5mm}

    삼계탕에서 닭은 안 먹고 국물만, 생선회에서 회는 안 먹고 쯔끼다시만 먹는 케이스라 하겠습니다.
    자기 풀이와 해설은 꼭 비교해보시길. 수학 실력이 이 때 \textbf{늘어나니까요}.
    사실상 인강, 학원, 과외로 수업듣는 거나, 해설지를 보면서 풀이를 비교해보는 것은 비슷한 과정입니다.
    장기적 효율성을 보면 후자가 더 중요하다고 하겠습니다.
    \vspace{5mm}

    \item  모르는 문제는 답을 보아야하는가?
    \vspace{5mm}

    4점 이상의 문제는 문제 측면에 자기의 도전방법을 요약해 적어두고 다음에 다시 도전하는 것도 나쁘지 않습니다.
    그 도전하면서 전략을 짜는 것 자체가 실력입니다. 인강과 학원의 문제는 그런 도전 기회를 차단해버린다는 것이지요.
    스킬북 보고 쩐다 하느니, 아무개 강의 듣고 좋다 하는 사람들 중에 정말 고수는 그리 많지 않습니다.
    당연하죠, 소프트한 스포 가지고 공부한 친구들이 늘겠습니까. 결국 맨땅에 헤딩해서라도 자기가 전략을 세우는 친구들이 올라갑니다.
    이걸 소위 문제를 "발효시킨다"라고 하죠. 모르는 문제는 풀지 말고 곱씹어보면서 한 일주일간은 계속 생각해보는 것입니다.
    일주일이 지나서라도 안 되면 해설지를 손으로 가리면서 위에서 한줄, 두줄, 세줄 식으로 보면서 최소한의 실마리를 갖고 재도전하는 것입니다.
    그래야 \textbf{자기가 어디서 꽉 막혔는가} 알 수 있습니다.
    \vspace{5mm}

    \item 기출문제집을 2권 이상 가도 괜찮은가?
    \vspace{5mm}

    기출문제를 한번 풀었다고 그걸 아는 것은 아닙니다. 기출도 결국 10회독은 해야 그 깊은 맛을 알 수 있습니다.
    시간이 허용된다면 기출문제집을 또 한번 푸는 것도 나쁘진 않습니다. 다시 풀어서 다 100점 맞는다면 안 보아도 되겠습니다만.
    \vspace{5mm}

    \item 한권 가지고 여러번? 아니면 여러권 가지고 한번?
    \vspace{5mm}

    한권의 경우는 위 3번입니다. 틀린 문제, 어려운 문제만 여러번 보라는 것이죠.
    그런데 결국 여러권 보아야합니다. 책들이 커버하는 범위가 다 차이가 있고, 문제 경향에도 유의미한 차이가 발견됩니다.
    신사고 라인의 경우는 대수적인 변형, 풍산자 시리즈는 발상의 변화나 역함수적 접근, 기출의 경우는 아시다시피.
    그리고 정석의 경우는 본고사 시절의 발상... 등이 있습니다. 이런 것을 다양하게 맛보면서 체험해야 머리가 좋아집니다.
    \vspace{5mm}
\end{enumerate}
이런 식으로 시중교재들을 착실하게 정리하시길 바랍니다. 그럼 좋은 결과는 안 나올 수가 없으니까요.
아, 그리고 위에서 깜빡했는데
\vspace{5mm}

틀린 문제는 반드시 번호 아래에 자기가 틀린 이유
"계산실수", "암산해서", "그래프 못 그림", "기호 실수", "딴 생각"... 이런 것들을 적어두시길 바랍니다.
\vspace{5mm}






\section{[학습공학 038] 문제의 인식과 해결}
\href{https://www.kockoc.com/Apoc/729341}{2016.04.14}

\vspace{5mm}

감각이 떨어지거나, 인지능력이 낮거나, 그리고 자기만 바라보는 사람들은 "문제"를 인식하지 못 한다.
그런데 이 중에서 가장 변명하기 힘든 건 바로 자기 중심적인 사고이다.
\vspace{5mm}

수학능력시험이 평가하는 건 그 학생이 문제를 풀 수 있는 사람이냐는 것이다.
그런 의미에서 문제를 인식하지 못 하는 것은 별 문제가 없을 수도 있다고 생각할 수 있지만 그건 오산이다.
어려운 문제가 어려운 이유는 그 문제의 정체를 우리가 제대로 인식하지 못하기 때문이다.
\textbf{출제자가 그 문제를 낸 의도가 무엇인가, 그리고 그 문제를 풀기 위해 어떤 도구를 이용하길 원하는가,}
\textbf{아울러 그 도구들을 어떤 절차에 따라서 사용하길 바라는가.}
\vspace{5mm}

저것을 할 줄 아는 사람이면 별 다른 사교육이 필요없을 것이다.
저게 생활습관화되어있는 사람, 즉 문제 인식 능력이 매우 뛰어나 남들이 보지 못 하는 문제를 인식하고,
그 문제를 제대로 정의할 줄 알며, 그리고 풀이과정을 세우는 사람은 수능시험 따위로는 재단하기 어려운 참인재다.
\vspace{5mm}

능력자라는 게 별 것이 아니다.
능력의 척도는 스펙이 아니라, 문제해결이다.
주어진 문제를 푸는 것이 아니라  남들이 알지 못 했던 문제를 인식하고 정의하는 능력이 참된 의미의 문제해결능력이다.
강의를 많이 듣고 어떻게 푼다... 라는 기술을 익혀보았자 한계가 있는 이유가 그것이다.
그 친구들은 문제를 비문제화하는 데 익숙해졌기 때문에 정말로 참다운 문제가 나오면 아예 읽지를 못 한다.
반면 참고서를 안 보더라도 평소에 사소한 것 가지고도 심각하게 고민하는(ex : 어떻게 하면 방귀로 감미로운 소리를 낼 수 있을까) 친구는
최소한의 도구를 가지고도 문제를 해결할 수 있는 준비가 되어있다고 할 수 있다.
\vspace{5mm}

문제해결의 3가지 방향은 다음과 같다
\begin{itemize}
    \item \textbf{첫째, 문제의 난이도를 조절한다, 즉 자기가 원하는 결과 수준을 조정한다}
    \item \textbf{둘째, 문제의 주어진 조건을 완화한다, 다시 말해 복잡한 문제라면 그걸 단순화해본다}
    \item \textbf{셋째, 자신의 능력을 높인다.}
\end{itemize}
\vspace{5mm}

그렇기 때문에 쉬운 문제부터 차근차근 양치기하는 것이 무난한 해법이 된다.
난이도가 쉬운 것부터, 조건이 단순한 것부터 연습해나가면서 자기 능력을 높여가는 과정이다.
이것 이외에 특효약이나 꿀코스라는 것이 정말 존재할 수 있을까.
꿀강의를 들었다 한들 그것이 문제를 인식, 정의, 해결, 정리하는 데 도움이 되지 않으면 '환상'에 불과하다.
\vspace{5mm}

그래서 어려운 문제를 많이 풀었다라는 건 정말 조심해야한다
정말 스스로 머리를 굴리면서 최소한의 도구로 그 문제를 푸는 사고를 복합적으로 해서 해결해오는 A가 있다면
반면 문제의 풀이과정을 암기하고 특정 문제는 어떤 스킬을 써서 풀어야하는지만 배운 B가 있다.
겉보기에는 둘 다 문풀량은 비슷해보이지만 실제 능력차이는 넘사벽이 아닐 수 없다.
A는 불의타 문제가 나오더라도 해결할 가능성이 있지만, B는 바로 포기해버리고 다음 해를 기약해야할 것이다.
\vspace{5mm}

학생들이 머리가 좋다 나쁘다 계산이 빠르다 느리다... 그런 건 그냥 무시해도 된다.
말하지만 그딴 건 나중에 아무 소용이 없다.
머리가 좋으면 뭐하나, 정말 문제가 나오면 해결을 못 하고 시스템을 만들지도 못 하고 수동적 스펙만 쌓으려하는데
반면 머리가 나쁘다고 하면서도 사소한 문제부터 차근차근 해결하고 근본적인 질문을 던지고 실천해나가는 친구들이 있다.
이런 친구들은 당장 수능에는 잼병일지 모르지만, 길게 보자면 '지배하는 자'가 될 수 있는 사람들이다.
\vspace{5mm}



\section{[학습공학 039] 눈 먼 자의 숫자}
\href{https://www.kockoc.com/Apoc/737509}{2016.04.20}

\vspace{5mm}

가치를 쫓는 것이 투자고, 차익을 쫓는 것이 투기다.
\vspace{5mm}

그런데 돈을 쫓아 투자를 하면 돈은 오히려 도망을 간다.
\vspace{5mm}

돈만 생각하니까 시세만 보게 되고, 가치가 아닌 가격만 눈이 들어온다.
\vspace{5mm}

마찬가지로
\vspace{5mm}

실력을 쫓는 것이 공부고, 점수를 쫓는 것이 수험이다.
\vspace{5mm}

그런데 점수만 쫓아 수험에만 치우치면 자기도 모르는 사이에 실력이 떨어진다.
\vspace{5mm}

점수만 집착하면 맞추는 데에만 급급하게 되고, 그래서 실력이 아닌 꼼수에 집착해버린다.
\vspace{5mm}

점수는 운의 영향을 받는다. 반면 실력은 악운조차도 이겨내는 것이다.
\vspace{5mm}

수험생인 이상 점수에 신경쓰지 않을 수는 없다. 하지만 그 점수는 어디까지나 실력의 지표로서만 의미가 있다.
\vspace{5mm}

각종 모의고사에서 고득점을 거두다가 실전에서 나가리나는 경우는 \textbf{진정한 실력이 없다는 얘기}다.
\vspace{5mm}

이걸 그 나이 또래들은 알기 어려울 것이다. 왜냐면 실력은 그 사람의 본성과 성격과도 관계있어서이다.
\vspace{5mm}

쓸데없이 나이를 먹다보면 인간이라는 게 더 눈에 들어온다. 그렇게 되면 우연이 아니라는 걸 알 수 있다.
\vspace{5mm}

수학문제를 스스로 풀 때에도 혹은 가르칠 때에도 이것은 사람의 성격과 마음 문제로 귀착되는 걸 알 수 있다.
\vspace{5mm}

스킬이나 테크닉을 백번 가르쳐줘보았자 가짜 점수만 늘어난다.
\vspace{5mm}

반면 문제를 어떻게 읽어야 할지, 그 급한 성격을 어떻게 고쳐야하는지, 초조한 마음을 다 잡고 트라우마를 극복하라하면
\vspace{5mm}

신기하게도 그 다음부터는 그 학생의 실력이 차츰 늘어나는 걸 알 수 있고, 무엇보다 수학공포증이라는 게 사라지는 것도 확인한다.
\vspace{5mm}

적어도 인간을 보라 하는 건 탁상공론만은 아니란 얘기다.
\vspace{5mm}

숫자는 양면적이다. 객관적으로는 명쾌하지만 주관적으로는 명쾌하지 않다.
\vspace{5mm}

숫자는 그대로이되 그 숫자를 바라보는 우리의 마음은 \textbf{욕망} 혹은 \textbf{불안}에 사로잡힌다.
\vspace{5mm}

그 욕망과 불안을 통제할 수 있는 사람에게 숫자는 더할나위없는 유용한 도구지만, 그렇지 못 한 사람에게는 악마다.
\vspace{5mm}

이걸 모르는 사람들은 숫자로 모든 걸 통제할 수 있다고 하지만, 마음이 강한 사람이 아니면 오히려 숫자를 보지 않는 게 나을 수도 있다.
\vspace{5mm}

한 예로 특정직업을 가지면 월 2천을 번다더라는 얘기는 실제로 본인에게는 허구에 불과하더라도 그 욕망에 불을 붙이고 있다.
\vspace{5mm}

그렇게 되면 정말 중요한 \textbf{본질과 가치}를 알 수 없게 된다.
\vspace{5mm}

극단적으로 말해 성적표를 보지 않고 그냥 공부를 하면 어떻게 될까.
\vspace{5mm}

성적을 모르니 실력을 모른다고 할지도 모른다.
\vspace{5mm}

그러나 그 친구가 어려운 문제와 난해한 텍스트로 집중훈련을 받는다면 그래도 실력이 안 오른다고 할 수 있을까.
\vspace{5mm}

오히려 성적에 좌우되지 않기 때문에 소신있게 한길만 쭉 파므로 휘둘리는 일은 없을지도 모른다.
\vspace{5mm}

현재의 사교육 시스템은 "총각 놀다가", "오빠, 잠시 쉬다 가세용"하는 홍등가처럼 정신없이 번잡하다.
\vspace{5mm}

거기에 휘둘리는 수험생들은 점수를 올릴 수 있다는 유혹에 빠져 정력만 낭비해버린다.
\vspace{5mm}

그럼 입시를 떠나서 정말로 미래에 이공계 과목, 특히 수학이 올킬을 하나?
\vspace{5mm}

수학 어디에도 마음, 본질, 가치 따위란 없다. 굳이 한가지 건질 게 있으면 "아름다움". 그 외에는 정말 찾을 수가 없다.
\vspace{5mm}

수학을 잘 한다고 윤리적이 되는 것도 아니고 사람을 통찰할 수 있는 것도 아니며 무엇보다 숨겨진 가치를 볼 수 있는 것도 아니다.
\vspace{5mm}

이런 이야기를 하지 않고 수학이 최고야하는 사람은 스스로가 마음, 본질, 가치를 읽지 못한다고 얘기하는 셈이다.
\vspace{5mm}

우리가 숫자를 보는 건 그걸로 본질을 간접적으로 알기 위해서일 뿐이다.
\vspace{5mm}






\section{[학습공학 040] 눈 먼 자의 숫자 2편}
\href{https://www.kockoc.com/Apoc/739290}{2016.04.21}

\vspace{5mm}

가령 입결이라는 것에 수능 이전에 신경쓰는 사람들은 '허영심'이라는 카테고리에 넣어도 되지 않을까 싶다.
\vspace{5mm}

어디 가야하는데 몇점 나와야하는가 보는 건 사실 별 의미가 없다. 그것만으로는 \textbf{내 점수를 높여주지 않기 때문}이다.
\vspace{5mm}

개처럼 벌어서 정승 같이 쓰라는 말이 정답이다.
\vspace{5mm}

그런데 이 속담이 먹히는 이유는 무엇인가. 정승처럼 벌어서 개처럼 쓰려고 하는 사람들이 다수이기 때문이다.
\vspace{5mm}

점수만 해도 그렇다. 그냥 개처럼 공부해서 점수를 높인 뒤에 정승처럼 지원하면 된다.
\vspace{5mm}

하지만 사람들은 \textbf{정승처럼 공부하고 개같은 인생}이 되어버린다.
\vspace{5mm}

가장 할 짓 없는 것이 과거의 입결 정보에 집착하는 것이다.
\vspace{5mm}

첫째로 중요한 건 미래이지 과거가 아니다. 그런데 올라오는 정보라는 건 그저 과거의 재탕이 아닌가?
\vspace{5mm}

둘째로 좋은 대학은 점수를 지불하고 사는 것이다. 그럼 점수를 높이면 되지 왜 저런 쓰잘데기없는 데 신경을 쓰는가?
\vspace{5mm}

모의고사 등급이 1등급 나오느니 만점 나오느니 하는 것도 마찬가지다.
\vspace{5mm}

실제로 모의고사에서 잘 나오던 사람들이 수능에서 망가지는 경우도 적지 않다.
\vspace{5mm}

왜냐면 모의고사는 잘 해보았자 시험용 문제이거나 과거의 문제 재탕이어서이다. 이 사람은 과거에만 강한 것이다.
\vspace{5mm}

하지만 수능은 꼭 예측하지 못 했던 문제를 낸다. 매년마다 출제 경향은 그렇게 통수를 먹였다. 그런데도 사람들은 과거의 정리에 집착한다.
\vspace{5mm}

그렇다면 그런 통수에 예방할 수 있는 실력을 키우면 되는 것이 아닌가.
\vspace{5mm}

내가 수험생이라면 수험사이트에서 미주알고주알 떠드는 건 신경쓰지도 않을 것이다.
\vspace{5mm}

그냐 시중에서 많이 푸는 문제집들을 모두 구입하고 어렵다라고 하는 교재들을 여러번 돌린 다음 확신이 들 때 모의시험을 쳐볼 것이다.
\vspace{5mm}

그리고 오답이 나오면 감사해할 것이다. 그래야 내 단점들을 파악하고 조기에 고칠 수 있기 때문이다.
\vspace{5mm}

모의고사 문제가 터무니없다고 해도 욕하지 않을 것이다. 터무니없는 문제가 미래에 저지를 수 있는 실수나 난관을 막아주기도 하기 때문이다.
\vspace{5mm}

그리고 수험사이트에서 잘 나간다하는 사람들에게 고개숙이면서 이것저것 캐낸 뒤 쓸모있는 정보만 얻으면 서슴없이 다시 공부하러 갈 것이다.
\vspace{5mm}

그리고 공부의 목표수준은 "출제자의 심리"를 읽고 "출제자의 함정"을 파악하는 수준까지 잡을 것이다.
\vspace{5mm}

사실 이 정도면 다른 것은 다 \textbf{필요없다}.
\vspace{5mm}

그리고 점수도 참조만 하지 크게 신경쓰지 않는다. 점수보다 더 중요한 건 반복되는 오답의 패턴이다.
\vspace{5mm}

그리고 내가 지금 할 수 있는 것과 할 수 없는 것을 분명히 파악하려하면서 내 성격과 능력에 어떤 문제가 있나 다시 고찰해본다.
\vspace{5mm}

모의 잘 나온다 하거나 수험 어쩌구 하는 건 걍 무시할 것이다. 어차피 그들은 '과거'에만 빠삭한 인간들이기 때문이다.
\vspace{5mm}

목숨을 좌우하는 건 평가원이 준비하고 있는 불측의 문제다. 당황하지 않고 그걸 풀 수 있느냐가 중요한 것이다.
\vspace{5mm}

보자, 이만큼 심플한 진단이 있나?
\vspace{5mm}

그런데 다수는 이렇게 하지 못 한다. 왜냐면 여기 콕콕러들 다수도 '숫자의 노예'인데 다른 사람들은 오죽하겠나.
\vspace{5mm}

숫자는 명쾌하고 합리적이라고만 하지만 앞편에서 말했듯이 그건만이 아니다.
\vspace{5mm}

숫자는 우리의 \textbf{욕망과 불안}을 자극한다.
\vspace{5mm}

까놓고 말하면 소개팅할 상대의 마음이 착하다 혹은 괜찮은 사람이야... 는 별로 자극을 주지 못 한다.
\vspace{5mm}

그러나 키가 얼마라던가 신체사이즈가 어느 정도라거나 한다면 거기에는 현혹되는 게 인간들이다.
\vspace{5mm}

본질을 보지 못하는 다수는 시험 점수가 몇점인가 거기에 현혹되지, 실제로 그 점수 이면에 있는 본질적 실력은 관심조차 없다.
\vspace{5mm}

그 본질을 모르기 때문에 점수가 높게 나온다고 생각하다 낮게 나오면 그 책임을 평가원이나 신에게 전가하려 한다.
\vspace{5mm}

이 수험시장에서는 돈을 버는 사기꾼들이 있지만 더 흥미로운 건 그 사기꾼들에게 낚이는 학생들이다.
\vspace{5mm}

똑같은 시간과 자본이 주어져있는데도 바보짓을 한다. 그러지 말라고 해도 바보짓을 결국 하고야 마는 것이다.
\vspace{5mm}

공부만 하더라도 실력을 키우는 공부를 하지도 않거니와 본인들의 습관에도 문제가 있다.
\vspace{5mm}

그리고 흥미로운 사실은 이들은 숫자 중독자들이다. 본질을 보지 못 하고 항상 점수나 등급이란 데에 매여사는 것이다.
\vspace{5mm}

물론 점수나 등급에 매여살더라도 본인이 매우 좋은 환경에서 공부해서 유치원 때부터 쌓인 실력이 있으면 결과는 나쁘지 않다.
\vspace{5mm}

그러나 개뿔 아무 것도 없는 사람들이 별 필요없는 숫자에 현혹되는 건 야동에 빠져사는 것과 별 차이가 없다.
\vspace{5mm}

극단적인 예를 들어보자
\vspace{5mm}

어느 친구가 수학 5등급이 나온다. 다수가 아, 답이 없군요라고 얘기할 것이다.
\vspace{5mm}

그런데 그 친구가 빡쳐서 \textbf{3만 문제를 제대로 풀었다}고 해도 답이 없다 이야기할 것인가?
\vspace{5mm}

이 때 숫자의 가치는 5등급 << 3만 문제이다. 그리고 누구나 다 그런 수긍을 할 수 밖에 없다.
\vspace{5mm}

3만문제를 푼다는 것 자체가 고부가 가치를 지니고 있기 때문이다.
\vspace{5mm}

이 세계에서 보면 참 본인 결과는 별 것 없는 사람들이 어느 학원이 좋니 어떤 교재가 괜찮으니 말은 많다.
\vspace{5mm}

그러나 그건 말 뿐이다. 정말 중요한 건 별 볼 일 없는 학원이나 쓰레기 교재도 좋으니 직접 학습을 해서 가치를 높이는 일이다.
\vspace{5mm}

다들 계획은 그럴싸하게 세운다. 그리고 일지를 쓰는 사람이면 돌이켜보면 알 것이다. \textbf{초심을 지키기 정말로 힘들다는 것}
\vspace{5mm}

예컨대 문제집 A, B, C를 제안하면 그것도 모자라니 더 해야한다고 하지만 실제로는 A도 겨우 1회독하는 수준으로 끝낸다.
\vspace{5mm}

계획을 세울 때 자기가 더 높은 자리에 올라가기 위해서라는 욕망과 불안감이 '숫자'의 조작을 부추긴 결과 비극이 되어버린 것이다.
\vspace{5mm}

실력을 높이고 싶으면 어떤 공부를 하면 되나
\vspace{5mm}

그거야 힘들게 공부하면 되지. 양을 슬쩍 늘리거나, 난이도를 높이거나, 공부시간을 더 많이 늘리거나.
\vspace{5mm}

다만 고통이 크지만 점차 줄어들면서 이전보다 더 나아졌다는 것을 스스로 확인할 수 있어야 한다.
\vspace{5mm}






\section{[학습공학 041] 눈 먼 자의 적응과 습관}
\href{https://www.kockoc.com/Apoc/742683}{2016.04.23}

\vspace{5mm}

모든 다짐, 모든 소망, 모든 결심은 거짓말입니다.
그런데 일단 거짓말을 해놓았어도 거기에 "실천"이 따르면 \textbf{참말이 되는 것이지요.}
그래서 우리는 거짓말쟁이라는 원죄를 안고 출발합니다
\vspace{5mm}

거짓말쟁이에서 탈출하기 위해 실천을 하는 것이지요.
\vspace{5mm}

습관이 무서운 건 한번 중단하면 그걸로 거의 사망에 도달한다는 것입니다.
한번 공부를 놓은 사람은 다시 돌아가기가 정말로 \textbf{어렵습니다.}
나이먹는다고 해도 뇌가 특별히 기능이 떨어진다거나 멍청해지는 건 아닙니다. 그런데 나이먹으면 공부하기 힘들어집니다.
그건 정확히 말해서 나이를 먹어서가 아니라, 공부를 중단했기 때문입니다.
극단적으로 말해서 공부를 한번 중단하면 10년치가 그대로 무산되어버리는 건 일도 아닙니다.
\vspace{5mm}

이건 세뇌론에서 얘기할 수 있는 이야기이지만 인격파괴하는 게 별 게 아닙니다.
시계도 없고 햇빛도 들이지 않는 암실에서 일주일간 재우지 않으면 됩니다. 정신력이 아주 강한 사람이 아니면 바로 폐인이 된다죠.
수면리듬만 망가져도 컨디션이 망가지는 게 사람입니다.
\vspace{5mm}

거기다가 적응의 문제도 있죠. 청소를 하는 건 자기 마음을 깨끗이 하는 거라는 말이 있죠.
무슨 뻔한 공자님 말씀이냐... 그게 아닙니다. 방을 안 치우는 사람은 끝까지 안 치우게 되는 건 게을러서가 아닙니다.
평소에 청소를 하지 않고 조금만 어질러진 것이나 먼지가 쌓이는 것도 관대하게 처리하면, 그 돼지우리를 더럽다고 못 느끼게 되어서입니다.
다시 말해 타인의 시선에서는 더러워보이는 방이 자기 눈에는 더러워보이지 않게 된다는 것입니다. 그래서 안 치우게 되는 것이죠.
\vspace{5mm}

공부를 꾸준히 하는 사람은 조금만 공부를 안 하더라도 자기가 공부를 덜 했구나 자책하게 되죠.
그러나 공부를 손에 놓게 되면, 나중에는 1시간만 공부한 것도 많이 공부했다고 자뻑해버리는 게 일도 아닙니다.
예, 인간이 적응의 동물이라는 건 상향평준화 뿐만 아니라 하향평준화도 가능하다는 이야기입니다.
부모님들이 괜히 너는 수준높은 애들과 놀아라... 하는 게 차별적인 발언만은 아닌 것입니다.
수준높은 사람들과 놀고 수준높은 경험을 하면 그만큼 성장을 하지만, 정반대로 가면 그만큼 쇠퇴해버리는 것입니다.
\vspace{5mm}

지금도 공부가 힘들다 하는 분들이 계실 건데 냉정히 말하면 "그건 수준이 낮았다"는 인증이 되는 것입니다.
물론 저는 이 시기에 지치는 건 당연하고 놀 때는 놀아야하며 힐링할 때는 힐링해야한다고 봅니다. 그건 다수가 평범해서이지요.
그러나 잘 하는 사람들은 이 시간도 아깝다고 공부하고 있으며 그게 냉정한 현실입니다.
어떻게 되어서든 평범한 사람들은 이 잘 하는 사람들을 따라잡기 위해 조금씩 나아가야합니다.
현재의 자기에 만족해버리면 안 됩니다. 인간은 간사해도 느슨해지기 좋고, 조금만 느슨해져도 그걸 용인하는 순간 내리막길을 걷게 됩니다.
\vspace{5mm}

그래서 공부를 한번 놓으면 그걸로 끝납니다.
혹자는 이렇게 얘기하겠죠. 다시 시작하면 되지 않느냐.
이론상 그럴 것 같지만 실제로 그렇지 않습니다. 공부를 한번 놓은 다음에 다시 복귀하는 경우는 정말 드뭅니다.
1등하던 친구가 공부를 손에 놓는 순간 그 자리는 다른 경쟁자가 차지합니다. 그리고 이 친구의 등수는 하락하겠죠.
한번 내리막길에 적응하고 나면 그 다음부터는 등수가 주욱 내려가더라도 아무런 생각이 들지 않게 됩니다.
가장 나은 건 잠시 쉬고 다시 죽어라해서 다시 원상태로 돌아갈 수 있도록 고통스러운 과정을 밟는 것입니다
그러나 멘탈이 그리 강한 사람도 없고 그렇게 다독여주는 어른도 별로 없습니다.
이렇게 해서 "한 때는 공부 잘 했는데 지금은 절망적입니다"라는 친구들이 양산되는 것입니다.
\vspace{5mm}

여기서도 공부를 하겠다... 라는 무수한 거짓말을 듣지만 당연히 믿지 않습니다. 제가 저 자신도 못 믿는데 말이죠.
공부한다는 게 쉽지 않다는 건 저 스스로도 느끼고 있습니다. 물론 이런 느낌조차도 소중한 지식이니까 여기 기록합니다만
그렇게 꾸준히 공부하는 상태 자체는 참 억만금을 지불하더라도 갖추기 힘든 것입니다.
지금 후회한다면 그런 것이겠죠. 왜 진작에 그 좋은 습관을 버렸을까, 왜 하필 그 나쁜 습관에 물들여졌을까.
\vspace{5mm}

그렇게 가면 공부에 휴머니즘 따위는 허용되지 않을지도 모릅니다.
\vspace{5mm}






\section{[학습공학 042] 후행학습의 필요성}
\href{https://www.kockoc.com/Apoc/756847}{2016.05.02}

\vspace{5mm}

중학교 수학에서는 두가지는 챙기고 나와야 한다.
하나는 연산능력, 다른 하나는 기하학적 사고.
이 두가지만 제대로 챙기고 고등학교에 진학하면 애로사항이 줄어든다.
연산 능력이 뛰어나면 피로도가 줄어들고 문풀속도가 남보다 2$\sim$3배는 빨라진다.
\vspace{5mm}

본인이 연산능력이 부족하다면 패배감이 느껴지더라도 중학교 수학 교재를 빨리 풀고 돌려보는 후행학습을 하는 게 좋다.
계산을 신속 정확하게 할 수 있다면 그만큼 확보한 시간과 두뇌 메모리로 더 창의적인 데 집중할 수 있기 때문이다.
수학은 창의력이니 논리력이니 이론상 그럴싸한데, 실전에서는 그게 안 먹힌다.
왜냐면 창의력이고 논리력이고 일단 기본적으로 독해력과 연산력이 뒷받침되어야 발휘할 수 있는 것이다.
테크닉이 아무리 출중해도 체력이 없으면 좋은 경기를 할 수 없다.
아무리 스킬을 많이 알아도 연산능력이 젬병이면 답이 없다.
\vspace{5mm}

롤 같은 게임을 해본 사람은 속성치(공격력, 방어력, 민첩, 힘, 공격속도 등)의  사소한 차이가 중대한 결과 차이를 가져온다는 걸 알 것이다.
평소에 문제를 잘 풀지만 시험에서만은 저득점이 나오는 사람들은 여러가지 유형이 있지만
그 중 하나가 쉽게 지친다는 것이다.
평소에 잘 뛰던 사람들도 수중에서 뛰라고 하면 느려질 수 밖에 없다. 시험보는 상황이 그렇다
평소에야 한두문제씩 풀고 여유부리니까 문제를 잘 풀 수 있을지 몰라도 시험시간에는 중압감 속에서 20 문제 이상을 풀어내야한다.
처음 1$\sim$5번까지 잘 풀다가 5$\sim$10번, 11$\sim$20번에서 집중력이 현저히 떨어지면서 평소라면 풀 수 있던 것도 못 푸는 경우가 생긴다.
결국 체력부족이다. 이런 경우라면 당분을 적절히 보급해서 두뇌 피로를 줄이고 문풀을 더 많이 해서 연산능력을 키워야 한다.
하지만 이런 걸 진단하지 않는 사람들은 역시 엉뚱한 방향으로 노력한다.
\vspace{5mm}

연산능력이 안 되는 사람들이라면 입시기간을 늘려서라도 초등, 중등 교재를 풀면서 다시 계산력을 잡아놓는 게 정답이다.
고등학교부터 성적 차이가 고착화되는 이유가 이것이다. 고등학교 과정은 '실력'에 크리티컬한 영향을 미치는 기초력을 배양하기 힘들다.
초중학교 때 기본이 잘 다져진 사람은 그걸 평생 가져가지만, 그렇지 못 한 친구는 고등학교 과정만 하기 때문에 엉성한 기본으로 버텨내야 한다.
\vspace{5mm}

이걸 모르는 사람들은 머리 차이라고 진단할 것이다.
두뇌 차이라는 건 물론 있다. 그러나 두뇌가 뛰어난 사람도 기초가 안 되어있고 그 부실한 기본실력으로는 절대 이겨낼 수가 없다.
정말이지 공부를 잘 하지 못 할 것 같은 사람이 두뇌가 훨씬 뛰어나다고 느끼는 경우도 많다. 그래서 이런 경우 더 답이 없다.
머리가 좋으면 뭐하나. 기본이 안 되어있고 자존심이 강해서 그 기본을 다시 다지자고 하면 거부해버리는데.
한편 칠뜨기에다가 그냥 어리버리한 학생도 기본이 잘 잡혀있으면 순조롭게 간다. 그래서 자기가 머리가 좋다고 착각하는 것이다.
좋다고 생각하니까 더 자신있게 공부하는 피그말리온 효과
그리고 이게 더 심해지면 자기가 선택받은 사람이라고 주모를 부를 것이다.
\vspace{5mm}











\section{[학습공학 043] 강사 대처법}
\href{https://www.kockoc.com/Apoc/763312}{2016.05.06}

\vspace{5mm}

어떤 강사가 좋다 나쁘다 순위만 나오지 성격별 대처법에 대해선 나오지 않았다.
강사는 상사라고 볼 수도 있겠지만 사람을 다루는 일반적인 테크닉을 학습에 적용하면 아래와 같이 정리할 수 있다.
자기가 듣는 인강이라거나 학원강사라거나 과외교사에 있어서 어떤 스타일인지 확인해보고 아래와 같이 대처하자.
\vspace{5mm}
\begin{itemize}
    

    \item[$\#$] 울트라리스크형 강사
\vspace{5mm}

독단적으로 밀어붙이기 좋아하는 스타일, 결과를 중시하되 사람들은 신경쓰지 않는다.
인강이라면 모를까 만약 학원이나 과외라면 학생들을 노예 부리듯이 할 것이라서 답답할지도 모른다.
다만 이런 형은 책임에 강하고 거짓말을 하지 못 한다. 그러니 학생 개인의 요구사항이나 불만을 거리낌없이 말하라
\vspace{5mm}

\item[$\#$]  원숭이형 강사
\vspace{5mm}

인간관계에 원만하고 $\sim$ 하자라고 추임새에 능하지만 대신 하는 게 없다. 강의가 치밀하지 못 하며 계획을 잘 못 지킨다.
인강 쪽에서는 찾기 힘들다. 물론 OT는 그럴싸하게 하는데 기한을 못 지키거나 하는 예외도 없지는 않다.
이런 강사는 관리 잘 해주는 원장을 만난다면 괜찮긴 하지만 아무튼 학생 입장에서는 적극적으로 요구해야겠으나
인강에서는 사실 규제할 수 없으므로 그냥 피하는 게 나을 수도 있다. 아니면 완강된 강의를 듣거나
\vspace{5mm}

\item[$\#$] 고양이형 강사
\vspace{5mm}

마음이 여리고 조용하다. 나긋나긋하고 여성적이며 상처를 잘 입는다.
이 역시 살아남기 힘든 유형이다. 바꿔 말해 이런 고양이형 강사가 살아남았는 건 실력이 있다는 이야기다.
인강이라면 역시 완강된 걸 듣는 게 좋겠지만 실강이라면 학생들이 이 강사를 격려해나가는 스타일로 가면 시너지가 좋을 것이다.
\vspace{5mm}

\item[$\#$] 알파고형 강사
\vspace{5mm}

모든 게 완벽에 가까운데 인간미가 없고 거의 로봇이다(이걸 보자마자 대부분 모 강사를 떠올리겠지)
인간미가 없고 살벌하기 때문에 스트레스를 줄 수도 있다. 그런데 역설적으로 이런 알파고형+독설파들이 인기는 좋다(주입은 시켜주니까)
이런 강사들에겐 인간성을 기대하지 말자. 이 강사들의 가르치는 방식은 꽤 논리적이므로 이걸 모방해서 적극적으로 베끼면 된다.
\vspace{5mm}

\item[$\#$]  목동형 강사
\vspace{5mm}

학생들보고 알아서 하라는 케이스이다. 썰을 그럴싸하게 풀며 과제를 별로 내주지 않아 인기는 좋지만 실속이 없을 수도 있다.
인강 쪽으로 들을 때는 초보자들에겐 좋지만 그 이상의 것을 기대하면 안 될 것이고, 실강일 때에는 역시 학생이 적극적으로 요구해야 한다.
학생들을 방임하는 스타일이라도 학생들이 구체적으로 요구하면 그만큼 해주려고는 한다.
\vspace{5mm}

\item[$\#$]  선도부형 강사
\vspace{5mm}

융통성이 없이 무조건 "안 돼$\sim$"라고 하는 스타일이다. 사실 가르치는 점에서는 안 좋다. 가르친다는 건 어떤 이치를 설득력있게 설명하는 것이라서리.
이런 스타일은 기숙, 독재학원의 관리자에 더 맞다.
만약 과외라면 피하는 게 좋다. 왜냐면 그런 선도부형에 적응하다보면 바보가 되어버린다.
\vspace{5mm}

\item[$\#$]  떠먹여주는 강사
\vspace{5mm}

하나부터 끝까지 다 자기가 설명하고 만족하려는 강사들로서 자기가 다 가르치면 학생들도 이해할 거라고 생각한다.
그보다는 시수를 늘리는 게 목적일 수도 있는데 이건 바람직하지 않다. 이 경우는 학생들이 스스로 목표를 세워서 자기 스스로 과제를 해보아야한다.
\vspace{5mm}

\item[$\#$] 성희롱/욕설 강사
\vspace{5mm}

최악의 강사 중 하나. 물론 인강에서야 나오지 않지만 실강에서는.
강의실을 작은 사회로 구현하여 온갖 욕설과 성희롱도 한다. 그런데 학생들도 거기에 물들이면서 점차 관념이 무뎌진다는 문제가 있다.
빠져나오는 게 답이다.
\vspace{5mm}

\item[$\#$] 카리스마형 강사
\vspace{5mm}

최악의 강사 중 하나. 첫째로 많은 학생들이 선호한다. 둘째로 많은 학생들이 효과를 보지 못 한다.
카리스마는 논리 없이도 상대를 납득시킨다. 그렇기 때문에 학생들은 강사의 카리스마로 해당 지식을 주입받지만 자기 걸로 만들진 못 한다.
즉, 사고능력이 정지된다는 것이고 수험상담을 해보면 '이 친구는 사고가 왜 굳었을까'하는 걸 보면 카리스마형 강사를 거친 케이스가 많다.
\vspace{5mm}

\item[$\#$] 반사회적 강사
\vspace{5mm}

최악의 강사 중 넘버원 수업시간에 하라는 수업은 안 하고 이상한 정치이념을 주입한다거나 비도덕적인 것을 가르치는 유형이 있다.
물론 인강에서는 찾기는 힘들다. 일전에 이런 이야기를 한 강사들이 마녀사냥에 가까운 정도로 얻어맞은 적이 있어서리
그러나 인강을 제외한 실강 쪽에서 찾기 어려운 건 아니다.
\vspace{5mm}
\end{itemize}
행동심리학에서 나온 DiSC 이론.
\begin{itemize}
    \item D = 주도 경향
    \item i = 감화 경향
    \item S = 안정 경향
    \item C = 신중 경향
\end{itemize}
\vspace{5mm}

자기가 저 중 어느 점에서 부족한가 판단한 다음, 그 부족한 점이 강한 강사를 찾아 강의를 듣는 편을 권장하고 싶다.
\vspace{5mm}







\section{[학습공학 044] 논리의 해악}
\href{https://www.kockoc.com/Apoc/766355}{2016.05.09}

\vspace{5mm}

수학에 있어서 논리의 중요성이 강조되지만
실제로 논리적으로 공부한 친구들이 시험은 영 형편없는 경우가 많다.
\vspace{5mm}

이게 공부를 안 해서인가.
그것은 아니다. 이건 논리적 사고가 \textbf{집중과 몰입을 파괴하기 때문}이다.
공부가 생각대로 안 되는 여러가지 이유 중 핵심적인 것이다.
\vspace{5mm}

논리적 사고를 키우는 가장 좋은 방법은 논리야 놀자를 읽는 게 아니라, \textbf{"아니오"라고 부정하는 습관부터 키우는 것이다}.
어떤 것이든 일단 먼저 부정해보고 왜 부정할 수 밖에 없는지 밀어붙이다가 더 이상 부정할 수 없으면 그제서야 긍정하면 된다.
우리나라 사람들은 일단 긍정하다가 부조리하다고 뒤늦게 깨달고 피해를 한참 본 다음에 아니오라고 ... 하지도 못 하고 걍 찌그러지는 경우가 많다.
덜컥 예스라고 했다가 피해를 본 뒤에는 나중에 자존심과 수치심 때문에 그걸 부정 못 하고 속으로 끙끙 참는 것이다.
한국인이 비논리적인 이유 중 하나다.
\vspace{5mm}

이 '아니오'가 바로 집중과 몰입을 방해한다.
평상시에 수학을 잘 푼다고 정말 논리적이라고 하는 사람들이 시험 성적이 엿같이 나오는 주된 이유라고 본다.
그 친구들은 시험 시간에 시간 부족을 느낄 정도로 정말 느리다. 왜 느리고 하니 문제 푸는 흐름을 타지 못 하며 사소한 데 집착을 한다.
그래서 시험의 목적을 잃어버리는 것이다. 그래서 그 분야에서는 갓이라고 불릴지 모르지만 실적은 사실 형편없는 게 당연하다.
(당연히 그 추종자들도 그럴 수 밖에 없다. 그 잘못된 패턴을 당사자도 모르는데 이게 전파되기 때문이다)
\vspace{5mm}

오히려 시험을 잘 보는 친구들은 논리적으로 설명하라고 하면 못 하는 경우가 많다. 그런데 정말로 답은 잘 구한다.
탐문해보면 그렇게 이르는 과정이 논리가 아니라 그냥 '이미지'이다. 이미지는 부정확하지만 그 속도나 범위는 좌뇌적 사고 저리가라다.
특정 사실을 떠올리기 위해 10가지 이상의 명제를 조합하는 친구와 이미지 전체상을 동영상으로 떠올리는 친구가 비교가 되겠는가.
물론 그런 친구들은 정말 논리적 사고가 필요한 킬러문제에서는 힘든 경우가 많다.
\vspace{5mm}

- 이 이미지 구사력이 좋은 것을 머리가 좋다라고 표현하는 것이다. 물론 그렇게 말하는 사람들은 좌뇌적 사고, 우뇌적 사고가 뭔지 모른다 -
\vspace{5mm}

비논리적이지만 집중과 몰입이 되는 우뇌적 사고와, 그 집중과 몰입을 깨지만 논리적인 좌뇌적 사고를 잘 융합하는 것이 중요하다고 하겠지만
도대체 이걸 어느 단계에서부터 그리고 어떻게 조절해야하느냐는 건 참 힘든 문제가 아닐 수 없다.
아예 우뇌적 사고로 밀어붙이면서 그 이미지를 정교화한 사람도 있겠고, 반면 좌뇌적 사고로 밀어붙이되 스피드로 담보하는 케이스도 있을 것이다.
하지만 일반 양민의 경우는 이걸 적절히 조합해야 한다는 것이 결코 쉬운 문제만은 아니다.
\vspace{5mm}

양치기 문풀을 하면서 대략 일정 단계에서부터는 그 패턴을 '논리'와 '이미지' 양쪽으로 스스로 정리해야한다가 그나마 현실적인(?) 대안이지만
사실 이것도 붕뜬 이야기다.
\vspace{5mm}

분명한 건 논리적으로 따진다라는 게 결코 좋은 것만은 아니란 것이다.
논리적으로 따지는 건 정말 그렇게 접근해야하는 사안이나 문제에서만 그러는 게 좋다.
\vspace{5mm}

시험 합격자들이 회독수를 늘리라는 건 정확히 말해 자기가 보는 텍스트들을 '이미지화'하라는 것과 같다.
실제로 많이 보다보면 페이지 전체가 포토 메모리화되어가는 것을 노리는 것이다.
\vspace{5mm}






\section{[학습공학 045] TOC 이론 - 개괄}
\href{https://www.kockoc.com/Apoc/766549}{2016.05.09}

\vspace{5mm}

TOC 이론 : 회사의 목표 - "현재에서 미래에 걸쳐 많은 돈을 버는 것"
이를 수험에 적용하면 "고득점을 받는 것"으로 바꿀 수 있다.
\vspace{5mm}

TOC 이론의 창립자 골드렛은 회사의 목표로 "현재에서 미래에 걸쳐 많은 돈을 버는 것"임을 주창했다.
그럼 TOC가 기존의 경영이론과 다른 것이 무엇인가
그건 시스템의 성과에 영향을 미치는 제약조건을 집중적으로 개선하는 것이다.
\vspace{5mm}

제약조건은 비유하면 사슬의 가장 약한 고리이다.
사슬을 세게 잡아당기면 어디서 끊어질까. 당연히 가장 약한 고리이다.
그 사슬의 운명은 약한 고리에 달려있다.
시험이 어렵게 나온다면 수험생은 어디서 무너질까, 당연히 가장 약한 부분이다.
\vspace{5mm}

골드렛이 더 골에서 예시한 건 '자녀'의 하이킹이다.
주인공 알렉스 로고가 보이스카웃 아이드을 데리고 하이킹을 나간다.
대열 중간쯤에 유난히 걸음이 느린 허비라는 아이가 있는데 허비보다 앞쪽에 있는 아이들은 점점 앞으로 나아가지만
허비 뒤쪽의 아이들을 그 느린 허비를 앞설 수 없어서 속도가 처진다.
이 하이킹의 목적은 아이들 전원이 목적지에 도착하는 것이다. 선두 애들이 아무리 빨리 목적자에 도착한들 의미가 없다.
대열 끝의 아이들까지 무사히 목적지에 도착해야 비로소 목표를 달성하는 것이다.
이 하이킹 대열이 시스템이라면, 가장 걸음이 느린 허비가 제약조건이다.
즉, 대열 전체는 이 허비의 걸음속도에 지배당한다.
그래서 알렉스 로고는 허비를 맨 앞에 세우고 허비의 걸음속도를 높이는 식으로 간다.
\vspace{5mm}

이걸 수험에 응용해보자. 많은 수험생들이 평균화의 환상에 빠져있다.
만약 국수탐이 1등급이 나오는데 영어가 3등급이 나온다 친다면 국수가 괜찮기 때문에 영어는 상관없다고 생각한다.
그래서 자기가 국수탐이 1등급이므로 국영수 모두 1등급인 친구와 맞먹는다고 착각한다.
하지만 실제로는 그 영어 3등급이 \textbf{그 수험생의 모든 것을 지배한다.}
이건 TOC 이론을 모르더라도 누구나 얘기할 수 있다. 하지만 본인은 그걸 긍정하려하지 않는다.
그 문제는 사소할 거야라고 일부러 외면해버리니 결국 '끌려다니는' 것이다
\vspace{5mm}

제약조건, 즉 보틀넥은 성적만이 아닐 수도 있다.
A라는 친구는 글씨를 못 쓰는 것
B라는 친구는 시험장에만 가면 쫄아버리는 것
C라는 친구는 계산실수를 자주하는 것.
.. 이런 식으로 각자의 제약조건은 다르다.
\vspace{5mm}

성적이 향상되지 않는 건 이 제약조건을 개선하지 않기 때문이다.
제약조건을 알고 개선하는데도 성적이 오르지 않는다라고 말할 수도 있다. 그 제약조건에 신경쓰느라 다른 데 신경을 쓰지 못 하니까.
하지만 이 제약조건을 제대로 개선하고 나면 그동안의 마이너스 분을 상회할 수 있기 때문이다.
\vspace{5mm}










\section{[학습공학 045-1] TOC 이론 - 실천}
\href{https://www.kockoc.com/Apoc/767346}{2016.05.10}

\vspace{5mm}

참조 : \url{http://newsplus.chosun.com/site/data/html_dir/2012/04/09/2012040900662.html}
\vspace{5mm}

TOC 이론에서는 부분최적화는 무의미하다고 얘기하고 있다.
반드시 최적화는 전체로 이뤄져야 한다.
예컨대 자기가 수학을 잘 하니까 수학으로 만회하겠다... 라는 건 평균적 사고로는 그럴 듯 하다.
그러나 평균적 사고로는 절대로 가치를 높일 수 없다(평균은 최고치보다 높을 수 없기 때문이다)
자기가 수학을 잘 한다면 수학에 매진할 게 아니라 가장 못 하는 과목에 자원을 할당하는 게 맞다.
\vspace{5mm}

제약요인은 특정 과목일 수도 있고, 특정 행위일 수도 있다.
국어와 수학을 둘 다 못 한다라고 해서 잘 보니 비문학 독해가 꽝이고 수학 문제를 대충 읽었기 때문이라면
그 사람의 제약 요얀, 즉 보틀넥은 바로 '독해 실력'으로 잡아둬야 한다.
이런 걸 잡아낸다면 자신의 독해 실력을 키우고 잘못된 습관을 바로 잡는 데 생각한 것의 10배 이상의 자원을 투자해야한다.
그걸 잡지 못 하면 다른 걸 공부해보았자 실력이 늘어나기 힘들기 때문이다.
\vspace{5mm}

그리고 이 경우 자신의 시험점수와 독해 실력 사이의 연결지점을 보아야 한다.
독해 방법을 달리하거나 습관을 고쳤을 때 점수 상승폭이라거나 시험 스트레스 등을 스스로 느끼고 체크해본다.
그로써 점수가 조정된다는 걸 확인한다면, 자신의 독해 실력을 키우기 위한 투자를 아끼지 말아야하겠지만
공부시간 할당이나 교재 배졍에 있어서도 상당한 여유분을 둬서 어떤 식으로든 장애가 생기지 않도록 해야한다(이하 '버퍼링')
에컨대 비문학 독해에 대해선 어느 교재가 좋냐 따지지 말고 구입할 건 다 구입해두고
보통 1시간이 소요되는 것이라면 3시간을 할당하면서 타 과목은 문제만 푼다면 그 독해는 오답노트까지 만들어 꼼꼼히 정리한다.
그럼으로써 자신의 점수가 올라가는 걸 확인하고 숙달시키는 것이다.
\vspace{5mm}

이렇게 하면 제약요인이 독해실력에서 "계산능력'으로 바뀌는 일이 일어날 수 있다.
독해실력을 극복하면 그 다음에는 계산이 느리거나 부정확하다라는 것이 그 다음 극복할 약점이 되기 때문이다.
\vspace{5mm}

제약요인은 비단 과목이나 행위만이 아니다.
그건 스마트폰이나 인터넷일 수도 있고, 펌프나 최유정일 수도 있고, 심지어(...) 콕콕과 콕챗일 수도 있다.
인정하기 힘든 요인이 알고보니 제약요인인 경우가 많다. 그렇다고 이걸 제거하라는 것이 아니다.
꼼꼼히 수치화시켜서 그것들을 조절해보면 된다. 가령 컥챗이 문제라면 로그인 횟수와 채팅 시간을 기록해서 그걸 통제해보는 것이다.
\vspace{5mm}

한달 뒤에 치르는 6평은 점수가 중요한 게 아니다. 오히려 오답분석을 꼼꼼히 해서 무엇이 '장애'였나 그걸 확인하는 게 중요하다.
어떤 문제가 본 수능에 나올지 그건 아무도 모른다. 하지만 자기의 제약요인이 끝까지 발목을 잡는다는 건 100$\%$ 트루다.
\vspace{5mm}






\section{[학습공학 046] 원리중심적 사고와 목적지향적 사고}
\href{https://www.kockoc.com/Apoc/767400}{2016.05.10}

\vspace{5mm}

수학문제 풀이에 있어서는 기하냐 대수냐 하는 것은.
더욱 근본적으로 가면 원리중심적 사고냐(일명 데카르트 사고), 아니면 목적지향적 사고냐(일명 드러커식 사고)냐 하는 것도 매우 중요한 떡밥이다.
이 이야기는 아마 내가 처음 하지 않을까 싶다. 왜냐면 원래 수학 참고서에 이런 이야기는 하지 않으니까.
\vspace{5mm}

원리중심적 사고는 교과서를 중시하는 것이다.
즉 기본 정의와 성질을 철저히 익힌 상태에서, 주어진 문제를 잘 해부하는 식으로 접근하는 정통파적 방법이다.
개인적으로도 이걸 매우 강조하는 건, 고교수학에서 배울 수 있는 유일한 장점이라고 하면 이게 전부이기 때문이다.
문제에서 미적분이라고 나오면 미적분의 정의를 떠올려가면서 문제를 해부해나가며 바로 전진, 다만 꽤 고지식하다.
\vspace{5mm}

목적지향적 사고는 출제자의 통수를 치는 방법이다. 이른바 시중에서는 사파로 알려져있다.
이건 꼼수로 보일 수도 있다. 그런데 엄밀히 말하면 일본어의 '야리코미'라고 하는 게 더 낫지 않을까 싶다.
유튜브에 올라온 게임 공략을 보면 기가 막히고 어이없는 방법으로 클리어하는 경우가 있다. 이건 개발자도 생각 못 한 경우다.
개발자가 생각 못 한 상상을 초월하는 발상으로 개발자가 게임 시나리오에서 내세운 과제를 실현해버리는 것이다.
\vspace{5mm}

이것이 기존의 꼼수와는 다른 건, 꼼수는 도구에 집착하지만, 이런 목적지향적 사고 - 일명 게임의 야리코미는 도구에 집착하지 않는다.
오직 목적만 바라본다. 다시 말해 문제가 원하는 '답'만 구하면 된다. 그러므로 교과서에 집착할 필요가 없다.
그렇다면 문풀 실마리는 어디서 발견하나? 그거야 당연히 문제의 목적을 읽고 역으로 그 목적에 도달할 수 있는 수단이 뭔가 생각해낸다.
고교수학(뿐만 아니라 대학수학에서도)에서 쓸 수 있는 툴은 의외로 한정되어있다. 또한 출제자 마인드라는 것도 생각 이상으로 단순한 경우가 많다.
원리중심적 사고처럼 기본 정의와 성질로 접근하는 게 아니라, 그 문제가 원하는 것으로 가려면 뭐가 필요하나 거꾸로 진행하는 것이다.
\vspace{5mm}

엄밀히 따지면 목적지향적 사고는 정말로 사파로 보일 수도 있다. 그 이유는 그런데 별로 탐탁치는 않다.
왜냐면 고교수학은 그냥 데카르트 수학 자체니까 당연히 데카르트적 접근법으로 가는 것,
즉 구식 과목에는 구식 스타일로 접근하는 게 정론이다. 구식에 구식으로 대응한다... 뭔가 불편하다.
다만 출제자는 21세기를 살아가는 사람이고 수험생들의 백분위를 가른다는 목적에 사로잡혀있다고 본다면
데카르트적인 사고에만 집착할 이유는 없다는 반론이 가능하다. 그게 바로 목적지향적 사고일 것이다.
\vspace{5mm}

내가 아는 한 이 분야 책은 다음과 같은 게 있었다.
\vspace{5mm}

모 인강강사가 수업시간에 신랄히 비난한 책인데, 사실 저자 스펙으로만 보면 절대 까일 사람은 아니다.
오히려 그 강사는 배움이 얕아서 책의 취지를 모르는 것 같다. 이 책은 고교수학을 신랄하게 풍자하면서 고정관념을 벗어나라는 주제다.
데카르트 사고의 문제가 다른 게 아니다. 고교수학을 잘 한다는 애들이 정말로 머리가 좋은가?
\vspace{5mm}

절대로 아니다.
\vspace{5mm}

머리가 좋은 애들은 모범생이 아니라 악동들이다. 이들은 출제자의 통수를 쳐대면서 더 높은 차원을 지향하지 하라는 말은 듣지 않는다.
역설적인 얘기지만 머리 좋은 애들일수록 공부를 못 하는 게 오히려 당연하다. 왜냐면 이들 입장에서는 공부가 \textbf{지루하기 때문}이다.
게임하라고 하면 12시간도 풀파워로 하는 애들이 공부 2시간을 못 하는 게 머리가 나빠서일 것 같나? 두뇌력이 없어서 그런 것 같나?
그냥 지루하고 재미가 없기 때문이다.
\vspace{5mm}

아무튼 내가 수집하고 연구한 책 중에서 발상을 뒤집은 책은 여러 가지가 있었지만 위 책도 좋은 예였다(물론 지금 구하기는 매우 어려울 것이다)
\vspace{5mm}

적어도 수험을 대비하는 입장에서는 데카르트적 사고 70$\%$ + 야리코미 30$\%$ 로 가는 게 낫지 않나 싶은데.
현실적으로는 목적지향적 사고는 공교육에서도 가르치지 않는 것이고, 이건 본인들이 그렇게 훈련해봐야 하기 때문에 힘들지도 모른다.
집안이 좋고 다양한 경험을 한 친구들이 머리가 좋다고 하는 게 다름이 아니라 간접적으로 저런 야리코미를 해본 경험이 많기 때문이다.
\vspace{5mm}

만약 본인이 온갖 문제를 다 풀어본다면, 역으로 목적지향적으로 변태적인 풀이, 변칙적 접근이 가능한가 시도해봄직도 나쁘진 않다.
물론 기본이 철저히 되어있다는 전제하에서는
\vspace{5mm}













\section{[학습공학 047] 란체스터 법칙}
\href{https://www.kockoc.com/Apoc/788196}{2016.05.22}

\vspace{5mm}

기온이 1도 상승 $\rightarrow$ 지구온난화 가속 $\rightarrow$ 바다에 침식되는 토지들 증가, 심지어 국가 붕괴 $\rightarrow$ 난민발생
기온이 1도 상승 $\rightarrow$ 식생의 변화 $\rightarrow$ 식량 생산에 치명적인 불연속점 발생 $\rightarrow$ 이하 동문
기온이 1도 상승 $\rightarrow$ 미생물계의 변화 $\rightarrow$ 전염병의 변화 $\rightarrow$ ..
\vspace{5mm}

이에 버금가는 것으로는 환율, 금리, 출산율 등이 있다.
\vspace{5mm}

작은 변화로만 알려져있지만 실제로 '합'을 구하면 엄청난 변화인 경우다.
그러나 사람들은 보편성이 없지만 거대한 변화면 주목하고, 보편성이 있지만 작은 변화면 주목하지 않는다.
\vspace{5mm}

가령 공부시간이 10분 차이가 나는 게 사소해보이지만 이건 공부시간 뿐만 아니라 삶 전체와 관계된 것이다.
그런데 학생들은 그 공부시간 조절을 정말 사소하게 생각한다. 매일 12시간 공부할 수 있다고 머릿 속으로만 믿는다.
하지만 이걸 실사해보면 매일 꾸준히 12시간 공부하는 경우는 찾기 힘들다. 사실 그 정도 되는 사람이라면 나중에 장관급도 해먹을 수 있다.
(실제로 장관들이 그렇게 공부해온 사람들이 아닌가)
\vspace{5mm}

자기가 하루에 몇시간 공부하나, 몇 문제 풀 수 있나, 그리고 오답률은 얼마인가.
이것들을 실사해보는 건 매우 귀찮은 일이다. 그리고 그 숫자들이 무의미해보일 수도 있다.
그러나 실제로는 이것들은 그 학생의 실적에 매우 강력한 연계를 갖고 있다.
만약 노력에 비해서 결과가 나오지 않느다면 그 때부터는 정성적인 것, 즉 학생의 성격, 환경, 트라우마 등을 분석해보아야겠지만.
대부분은 저런 숫자적인 것으로 결정난다.
\vspace{5mm}

$\#$ 나무위키의 란체스터 법칙
\vspace{5mm}

\url{https://namu.wiki/w/%EB%9E%80%EC%B2%B4%EC%8A%A4%ED%84%B0%20%EB%B2%95%EC%B9%99?from=%EB%9E%80%EC%B2%B4%EC%8A%A4%ED%84%B0%EC%9D%98%20%EB%B2%95%EC%B9%99}
\vspace{5mm}

학습에 있어서는 란체스터 법칙이 정말 절대적이다.
우리편 6명과 상대편 6명이 싸우면 서로 3명 3명 쓰러지고 비길지도 모른다.
그럼 우리편이 12명이고 상대편이 6명이면 우리편 6명은 살아남고 이기는가 하면 그게 아니다.
개략적으로 적으면 $12^2: 6^2 = 4 : 1$ 의 차이가 난다. 상대편 6명이 쓰러지면 우리는 1.5명이 쓰러진다.
\vspace{5mm}

이걸 학습에 적용시키면 공부시간, 문제 숫자 같은 것은 절대 일차함수가 아니란 이야기다.
문제집을 2권 제대로 푼 녀석은 1권 푼 녀석보다  4배 더 잘 한다고 계산할 수 있다.
선행을 2년 전에 한 녀석은 사실 8년 앞선다는 얘기다(시간은 고차원적인 것이니까)
\vspace{5mm}

그런데 보통 이 시점에 상담을 청하는 사람들의 문제는 저 란체스터 법칙을 생각하지 못 한다는 것이다.
노력과 결과가 '정비례'한다고 믿고 있다. 그래서 1년 내에 원하는 목표를 이룰 수 있다고 믿는다.
하지만 실제 수험을 저런 란체스터 법칙으로 설명하자면 222 등급을 달리는 사람이 의대에 가려면 생각한 물량을 세, 네제곱 시켜야한다는 얘기다.
실제로 그렇게 최상위권을 달리는 친구들은 어렸을 때부터 선행은 기본인데다가 엄청난 학습량이 쌓인 상태이다.
이 친구들을 노력으로 이기려면 1차원적인 노력으로도 불가능하다.
시간을 더 많이 잡고 어느 정도 성과로 이어지는 노력과 시행착오를 하면서 정말 장기간 '인내'해야한다.
\vspace{5mm}

상술하자면 고등학교 때부터 성적을 올리기 힘든 이유가 이걸로 설명된다.
\vspace{5mm}

중학교까지의 내신은 정말 1차 함수적으로 결정된다. 대부분이 그 때 수험을 시작하기도 하지만
특목고 가는 라인이 아닌 경우는 보통 문제집 양치기로 가볍게 결정되기도 하고 그 이상으로 평가하지 않는다.
중학교 성적은 나중에 서울대 의대 갈 녀석이건 아니면 콕콕대 갈 녀석이건 똑같이 100점이 나올 수 있다.
\vspace{5mm}

그러나 고등학교부터는 절대로 아니다. 이 때부터는 적나라하게 평가된다.
대체로 어느 학교건 최상위권을 달리는 애들의 학습수준은 중위권과 비교되지 않는다. 이미 일차함수적 패러다임으로는 접근되지 않는다.
로그적 분석을 해야 차이가 보일 정도다. $\log$(학습량)이라고 했을 때 최상위권이 6이고 중위권이 3이라고 상위권이 2배 했다 이야기가 아니자.
만약 밑이 10이라면 1000배 차이가 난다는 이야기.
\vspace{5mm}

그런데 중학교 수험의 패러다임에서 못 벗어난 학생이나 학부모들은 이걸 모른다, 아니 사실 평생동안 모를 수도 있다.
가장 먼저 선빵으로 공부하면서 학습량을 늘리면 그것이 새끼쳐서 계속 불어나서 저런 란체스터 법칙이 작용한다는 것을 모르는 것이다.
그래서 실제로 내신 따서 갈 수 있지 않느냐 내신 망하면 정시 가면 되지 않느냐... 이게 부질없는 게, 그거 따질 때가 대략 고2이다.
그런데 고2 부터는 이미 학습 우위가 양극화 현상을 보여준다.
고2 내신 잘 따는 녀석들은 그걸 이미 중학교 때부터 다 준비해온 놈들이 많으며(물론 학교 수준에 따라서는 그럴 필요가 없을 수도 있다)
수능 대비하면 된다는 사람들은 1년 정도 걸릴 거라고 안이하게 예상하다가 6월 무더위를 맛보고 나서야 고2 때 끝내야한다는 것을 비로소 깨닫는다.
\vspace{5mm}

그럼 루저들은 죽어야한단 말인가.
\vspace{5mm}

그냥 꾸준히 공부하면서 공부가 공부를 새끼치는 현상을 기다리는 수 밖에는 없다.
이래서 필요한 게 바로 '인내심'이라는 것이다. 그래서 자연스레 삼수, 사수(...)하는 것도 보편적이 되어버린다.
왜냐면 현역으로 가는 녀석들이야 3, 4년 미리 앞서 공부한 녀석들이 있으니까 이래야 공평(?)한 것이다.
자기가 비록 현시창이지만 좋은 대학에 가고싶단 사람들은 자기보다 10년 앞선 사람들을 어떻게 따라잡아야하나 고민해야한다.
당연히 늦게 가야하는 페널티는 붙는다. 하지만 늦게 가더라도 '달성'만 하면 된다. 그러나 달성률도 낮다.
왜냐면 자기 나이에 자괴감을 느끼고 공부기간이 훨씬 길어야한다는 것에 포기해버리기 때문이다.
\vspace{5mm}

하지만 인내심을 갖고 꾸준히 버티다보면 자기도 모르는 재능이 개화하는 경우가 생긴다.
그리고 그동안 공부했던 것들이 마법진처럼 발동하기 시작하면 실력이 a>1인 지수함수처럼 올라가기 시작한다.
바로 그 때를 노리고 공부하는 것이다. 그런데 이건 사람마다 달리 온다. 당연히 실패를 거듭한 사람이면 더 늦게 온다.
그 온다는 것을 확신하면서 고생을 하면 되는 것이지만 유감스럽게도 대부분은 단군신화의 모에 타이거가 되어버리고 만다.
\vspace{5mm}










\section{[학습공학 048] 쪽지 퀴즈}
\href{https://www.kockoc.com/Apoc/795077}{2016.05.27}

\vspace{5mm}

텍스트 :
\vspace{5mm}

쎈수학이나 마플의 개념 부분 등
\vspace{5mm}

방식 :
\vspace{5mm}

쎈과 마플의 "목차"를 백지에 적음. 그리고 퀴즈라고 소리치고 그 목차 아래에 기억나는 내용을 정확히 적기
\vspace{5mm}

Ex)
\vspace{5mm}
\begin{enumerate}
    \item 공간에서 평면의 결정조건 4가지를 적으시오
    \vspace{5mm}
    
    \item 미분계수의 두가지 표현방식을 적으시오
    \vspace{5mm}
    
    \item  $y=|g(x)-t|$의 미분문제에 관한 모든 논점을 적으시오(참고로 이건 마플에 있다)
    \vspace{5mm}
\end{enumerate}

평하자면 가장 가성비가 좋은 학습방법이다(공개하기 싫을 정도)
책을 읽을 때 그냥 읽지 말고 A4에 손글씨, 아니면 워드로 목차만 기록해놓는다
그 다음 하루가 지난 다음 그 목차 아래에 자기가 기억하는 핵심 내용을 적는다. 그리고 채점해보는 것이다.
\vspace{5mm}

이게 무슨 소용이냐하시는 분이 있겠는데 수학문제를 푸는 것 자체가 자기가 알고 있는 개념, 성질, 공식을 '복기'해가는 과정이다.
그걸 정확히 복기하지 못하기 때문에 문제를 못 푸는 것이다.
그렇다면 정확한 복기를 반복하면 된다. 한데 다들 이상한 교재에 낚여서 자꾸만 쓸데없는 문제를 풀고 있다.
\vspace{5mm}

쎈수학이나 마플 개념 - 그리고 해설에 나온 심화개념이나 문제에 숨어있는 수학적 팩트(신승범식 표현)을 정확히 모르고서
이상한 실모 문제 풀어보았자 수능장가면 아주 정확히 발린다. 그것도 기본 개념조차도 기억 못 하기 때문에
\vspace{5mm}

보통은 저런 퀴즈를 내면 처음에 어리둥절한다. 그런데 이거 10문제를 본인이 내고도 본인이 100점도 못 맞는다(나도 그렇다)
책을 읽고 문제를 풀 때 자기가 정확하게 알고 있다는 건 '착각'이라는 걸 보여준다.
가능하면 워드로 치는 게 좋다. 이런 퀴즈는 최소 10번은 치고 답을 적어야만 아주 정확해지기 때문이다.
1번만 봐서 안 다는 건 그런 머리를 갖고 있을 때나 먹히는 것이다.
우리가 평범하다라는 걸 인정하고 10번은 퀴즈본다, 그 자세로 가야한다.
\vspace{5mm}





