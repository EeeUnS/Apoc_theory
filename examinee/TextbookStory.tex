




\section{교재 이야기 ; 숨마쿰썰}
\href{https://www.kockoc.com/Apoc/464449}{2015.11.05}

\vspace{5mm}

올드 숨마쿰의 장점

\begin{enumerate}
    \item \textbf{상위권}
    \item \textbf{ 허세}
    \item \textbf{ 어머니 안 계신 난이도}
\end{enumerate}
\vspace{5mm}

올드 숨마쿰의 단점
\begin{enumerate}
    \item 불연속
    \item 기초 부족
    \item 구하기 어렵다.
\end{enumerate}
\vspace{5mm}

제가 아는 숨마쿰은 3가지임
\begin{itemize}
    \item 7차 교육 이전 숨마쿰 : 올드 숨마쿰
    \item 7차 교육 과정 숨마쿰(현 고3까지) : 7차 숨마쿰
    \item 개정 숨마쿰(현 고2부터) : 개정 숨마쿰
\end{itemize}
\vspace{5mm}

개정과정은 약간 안타까운 게 있음.
숨마쿰의 단점으로 지적되던 게 하위권에게 너무 불친절하다, 그리고 기초 문제가 부족하다여서인가
이런 방향으로 개정했다고 보는데 그 결과 숨마쿰만의 장점이 희석되어 버렸음.
\vspace{5mm}

원래 숨마쿰은 안경 낀 수학남녀가 허세 부리는 용으로 딱 좋음.
이게 비꼬는 게 아니라, 실제로 대단히 중요함. 수학 시험의 불안감을 날려주는 게 '나 정말 어려운 문제집 풀었다'라는 허세임.
솔까 요즘 나오는 것 중에선 블랙라벨과 실력 정석 빼고 허세부릴 수 있는 건 별로 없음(아니면 사진 속 일본 교재 '붉은색' $-$ 아카차트로 가든가)
\vspace{5mm}

저기 숨마쿰 중에서 좌측이 올드, 우측이 7차임
사실 현재 수험생들을 대상으로 한다면 7차로도 상위권용 개념은 괜찮다고 생각함.
정석을 보는 것보다는 차라리 이 편이 낫다고 보는 게 있음.
올드 숨마쿰 자체가 과거에 정말 수학을 잘 했던 사람들이 자기가 공부한 내용을 제대로 짜깁기한 경우였음.
문제 난이도는 알보칠 수준임.
7차 숨마쿰은 약간 희석시킨 알보칠 정도인데 서술이 더 쉬워짐. 톡쏘는 맛이 약하긴 하지만 요즘 교재에 비하면 그래도 난이도 있는 편임.
개정은 보다가 필요없다라고 보아서 구매하진 않았음. 왜 그렇게 개정했을지 이해는 가는데 $-$ 고객들은 중하위권이 많으니까 $-$
이건 뭔가 좀 아깝단 생각도 들고 있음. 과거 교재의 어려운 것만 다 요약해서 별권 발행해주면 좋지 않았을까 싶음.
\vspace{5mm}

개정과정 생까버리고 올드 숨마쿰대로만 팔았어도 잘 팔리지 않았을까 싶음.
n수생이 많이 늘었거니와 수험생들 수준이 전반적으로 높아져서 고난이도 문제 수요가 높아졌기 때문.
사실 실모는 붕어빵 아니면 질소과자스러워서 $-$ 4점 3문제 풀려고 30문제 전부 구입한다는 딜레마 $-$ 상당히 비싸게 구입하는 것임.
\vspace{5mm}

저 당시 저자진들이 과거 모 사이트 레전드인 걸로 알고 있음. 이 경우는 인정할만하다고 생각하고 있음.
초기에는 서술이 너무 불연속적이고 어렵다라고 보는데 여러번 읽고 느낀 것은 어, 그래도 정말 상위권 허세부릴만하다였음.
교과서 따로 보란 말도 없음, 오히려 교과서상 개념을 자기들이 더 상세히 설명해주고 있음.
\vspace{5mm}



\section{교재 이야기 : 풍산자썰}
\href{https://www.kockoc.com/Apoc/465841}{2015.11.06}

\vspace{5mm}

\href{http://news.naver.com/main/read.nhn?mode=LSD&mid=sec&sid1=001&oid=036&aid=0000004714}{링크}
\vspace{5mm}

사지 멀쩡한 사람들도 교재 대충 쓰는데 이 저자 분은 불편한 처지에서도 교재를 꽤 잘 쓰는 편이다.
혹자는 이런 비판을 할 수도 있다. 저자 처지를 내세워서 마치 공정무역 커피 맛있다고 뻥치듯 광고하는 것 아니냐 그러는 것.
\vspace{5mm}

우선 이 교재는 '수포자용'이라고만 오해를 사고 있는데 절대로 아니올시다.
\vspace{5mm}

수포자용이라는 건 이 교재가 쉬워서가 아니다. 사실 이 교재의 유제나 연습문제는 어려운 건 대단히 어렵다.
다만 그 어려운 것이 조잡한 교재의 고난이도 문제나 스킬 요하는 그런 걸 필요로 하지 않는다. 즉, 수능에만 적합하다는 이야기다.
허세만을 좋아하는 수험생들은 "\textbf{어라, 수포자용이라고? 정말 쉽겠네. 그럼 내가 보면 안 되지"}라는 프로세스로 움직여서 그런 것이지
\vspace{5mm}

이 교재는 중하위권을 배려해서 그렇지,
수학 개념의 독창적인 해석이나 수록 문제는 상위권용이다.
사실 풍산자의 직관적이고 감각적인 개념 설명은 오히려 킬러 문제를 풀 때에 매우 요긴한 것이다.
이 교재로 공부해보신 분은 웬만한 수험수학의 팁 $-$ 다시 말해 수능에 필요한 정도 $-$ 은 다 들어가 있다는 데 동의할 것이다.
\vspace{5mm}

상위권을 배려(?)하는 교재는 사실 누구라도 쓸 수는 있다. 적당히 짜깁기하면서 문제를 대충 어렵게만 내놓고 해설 무성의하게 쓰고
이거 이해 못 하는 건 네가 공부 안 해서임 ㅋㅋ 이라고 구라까면 되기 때문이다.
진정한 교재집필의 고수라면 '중하위권'용 책을 쓴다.
게다가 이 교재의 개념 설명은 볼 때마다 감탄이 나올 정도이다.
다른 논란을 불러일으킬지 모르지만 쎈의 개념 설명이 갤xx이라면 풍산자의 개념 설명은 아xx가 아닐까.
그 이야기는 거꾸로 말해서 개넘 설명이 너무 직관적이고 감각적이므로, 논리적으로 엄밀한 부분은 취약할 수 있단 이야기인데
이건 다른 교재로 보충하면 된다.
\vspace{5mm}

이 글을 읽는 고1, 고2는 뭔 교재로 재기해야 하나... 하면 이 교재로 틀잡는 걸 권한다.
교재 권장에서 가장 중시하는 건 '부작용'이 있거나 혹은 잘못된 것을 엉터리로 배우는 게 아니냐는 것인데
물론 이 교재도 그런 게 없을 수는 없겠지만 적어도 내가 아는 한도에서는 발견된 건 없다.
유일한 단점이라면 양치기를 하기 힘들단 것인데 이건 쎈이나 RPM을 병행하면 된다.
\vspace{5mm}

개념의 감각적 해석 $-$ 문학적 비유가 마음에 안 든다는 사람들이 있을지도 모르겠지만 그건 몰라서하는 이야기다.
\vspace{5mm}

여기서 다시 교재뒷담화를 하면 나는 기본적으로 교재 저자들이 국어 실력이 없고 문과적 소양이 부족한 경우는 피해야 한다고 보는 입장이다.
수학 공부를 열심히 했고 계산이 빠르고 수식이나 그래프 구사가 뛰어난데도 수학 점수가 안 나오는 애들이 왜 안 나오는지 아나?
이런 애들, \textbf{국어 실력이 꽝이다. 문과적 소양이 없다}.
\vspace{5mm}

수학 문제는 수식이나 그래프로만 쓰여진 게 아니다. '한국어'로 쓴 것이다.
문제를 풀 때 가장 중요한 건 문제를 '해석'하는 것이다. 그리고 저자가 뭔 의도인가 그걸 알아내는 것이 중요하다.
그런데 이런 독해는 유감스럽지만 수학 교재들에서 누락시킨다.
수학교재들은 대체로 수리적인 풀이만 선사하지, 문리적인 해석에 대해선 언급을 피한다.
그래서 이른바 문제집은 많이 풀고 인강은 들었는데 결정적인 데에서 막히는 '괴혈병'에 걸리는 것이다.
\vspace{5mm}

이건 잘 나간다는 수학고수들 관찰해보아도 그렇다, 그 친구들의 계산, 발상력은 보통 학생의 3$\sim$5배는 된다.
그러나 그것 때문에 공부가 편중되어서 국어나 문과적 소양이 나가리나버린다.
그래서 문제를 잘못 읽은 채로 자기 확신에 빠져 엉뚱한 데 헤매기도 하고,
개념의 논리로 풀면 간단할 것을 자꾸만 화려한 기교와 수식을 연마하다가 맛가는 경우가 많다.
\vspace{5mm}

이런 이유 때문에 교과서를 읽으라고 하는 것이고, 가능하면 수학을 국어적으로 읽으라고 하는 것이다.
풍산자의 개념 설명이 이 점에서는 (다소 부족할지 모르나) 종합비타민제 역할은 해주고 있다.
수학을 잘 한다는 친구들일수록 이런 점을 경시한다. 그리고 실전에서는 터무니없는 점수를 맞는다.
이런 일이 벌어지는 이유는 간단하다.
\vspace{5mm}

평소의 모의나 사설모의, 그리고 시중교재는 패턴화되어있다. 그래서 문제를 대충 읽어도 뭔지 알 수 있다.
그러나 수능은 교수들이 새롭게 '내는' 것이다. 그래서 출제 방향을 다소 비틀거나 꼰다.
그렇기 때문에 문제의 해석을 잘못 하면 나가리나기 딱 좋다. 국어를 무시하는 친구들이 여기서 덫에 걸려드는 것이다.
\vspace{5mm}





\section{교재 이야기 : 말이 필요없는 증명 썰}


\href{https://www.kockoc.com/Apoc/472719}{2015.11.10}


알만한 사람은 다 아는 유명하면서도 유명하지 않은 책이다.
제목 그대로 '그림'으로만 증명을 한다.
단, 난이도는 고교 과정을 훨씬 벗어나는 수준까지 이르며 볼 필요가 없는 것들도 있으며
쉬워보이지만 사실은 대단히 어렵다.
이 책을 처음 접한 건 과거의 해적판(...) - 과학고에서 P.S.S.와 더불어 수입해서 돌려보던 시절이 있었다라고 보면 된다.

\vspace{5mm}


수학에 있어서 소위 직관이라 알려진 'Visual Thinking'의 정수를 담은 책.
\vspace{5mm}


다만, 수학을 좋아하는 사람들만 보면 되고 수능용으로는 추천하지는 않을지도(올해 수능이 나와보아야 알겠지만)
그건 이 책이 안 좋아서가 아니라, 최근 3년간 수능은 Visual Thinking을 그다지 강조하지 않기 때문이다.
평가원이 싫어하는 건 케일리 해밀턴 정리나 로피탈만은 아닌 듯.
상위권이면 순식간에 풀 수 있는 이른바 '직관풀이'라는 걸 더 싫어하는 경향이 있다.
도형이나 그래프로만은 풀 수 없게, 혹은 풀린다고 착각하면서 오답유도를 하게 하면서 반드시 '식 풀이'를 하도록 하고 있다
개정교과도 논리와 식과 언어를 강조한다고 친다면, Visaul Thinking이 탄압받는(?) 시대인 듯.
\vspace{5mm}


하지만 그와 별개로 책 자체는 전지구적 명저이므로 수학에 관심있는 사람은 '보길' 바란다.
지능지수 높이기에는 딱.
다만 수능 준비라면 구입하지 말길(이라고 하지만 공부 안 하고 쇼핑 좋아하는 친구는 또 구입했다가 광광 우렀다가 되겠지)






\section{교재 이야기 : 교과서썰 1}
\href{https://www.kockoc.com/Apoc/476332}{2015.11.11}

\vspace{5mm}

교과서를 강조하지만 실제로는 교과서스럽지(?) 않은 사례들이 많다.
\vspace{5mm}

사실 이건 정말 하고 싶은 이야기다.
몇몇 저자나 강사들은 교과서가 중요하다, 교과서스럽게 공부해야 한다라고 한다.
이거 그럴싸한 말이긴 한데.
\vspace{5mm}

문제는 그 사람들의 강의든 교재든 그래서 교과서스럽냐... 하면 그건 아니었단 말이다.
교과서를 예로 들려면 출판사, 저자 라인, 페이지 등을 명기해야 하지 않나?
수능이 아닌 공무원이나 고시 참고서의 경우는 '참고문헌' 인용을 한다. 그거 집필자가 교과서를 읽고 짜깁기햇다는 근거다.
\vspace{5mm}

그런데 내가 보았던 어떤 수학교재들은 교과서가 중요하다고 말을 하는데
\begin{itemize}
    \item[$-$] 교과서 출처를 표시한 것도 아니고
    \item[$-$] 그렇다고 교과서 내용 인용하지도 않았으며(그게 어려운 것도 아니고)
    \item[$-$] 내용은 내가 아는 한 전혀 교과서스럽지 않다는 것이다.
\end{itemize}
\vspace{5mm}

그냥 까고 말하지. 저거, \textbf{교과서를 안 보았단 이야기 아닌가.}
그리고 교과서를 강조한다면 나처럼 "두산, 미래엔, 성지 교과서 같은 것 보세요. 아, 물론 이건 7차 교육과정입니다"라고 하면사 컥챗이나 하면 그만이지
강의나 책을 팔아먹을 이유가 없지 않나.
\vspace{5mm}

이런 일이 왜 벌어지는 걸까  하기 전에
"도대체 교과서스럽게 공부한다는 게 무얼 이야기하는 걸까"라는 질문부터 답이 있어야하지 않을까 싶은데
\vspace{5mm}

이 경우도 냉정히 이걸 추리해야 한다.
\vspace{5mm}
\begin{enumerate}
    \item \textbf{ 교과서에 수록되어 있는 건가}
    \item \textbf{ 교과서 정신(?)을 따라가는 것인가.}
\end{enumerate}
\vspace{5mm}

즉, 형식적 구분인가, 아니면 의도적 구분인가 이것도 명기해줘야지.
하지만 겉으로 교과서 강조하면서 실제로는 교과서도 잘 인용 안 하는 책들에서 이런 섬세한(?) 구분을 기대할 리는 만무하고
그냥 내가 정리해보면 당연히 2번이다.
교과서에 수록되어있느냐 안 되어있느냐 그게 중요한 게 아니라, 교육 당국아 강조하는 '실질적인 교과서 수준의 발상'이 중요하단 거지.
\vspace{5mm}

만약 수록이 문제라면 전체 출판사 다 검증해보아야하잖아.
\vspace{5mm}

하도 수험가에서 저런 식으로 앞뒤가 안 맞는 경우가 있어서 거금 들여서 교과서 익힘책 7차 과정 $-$ 무슨 고려출판사나 박영사까지도 다 구매해보았다
그리고 작년부터 올해까지 쭉 읽어보고나서 1번을 폐기할 근거를 찾았다.
\vspace{5mm}

수학익힘책 기준으로 친다면(설마 익힘책도 교과서가 아니라고 하면 할 말이 없다)
\textbf{로피탈도, 케일리 해밀턴도, 그리고 적분에 별 이상한 정리 같은 것들까지도 다 교과서 내용으로 보아야한다는} 결론이 나온다.
출판사들마다 저자 라인이 다르고, 저자 라인들도 조금씩 강조하는 게 차이가 있기 때문이다.
가령 적분과 통계에서 중앙교육에서 낸 책은 몇몇 내용은 시중교재도 초월하는 수준이었다는 것.
\vspace{5mm}

이론 뿐만 아니다. 단원 마무리할 때 나오는 퀴즈나 일화 같은 것들도 내용을 보면 준수리논술급도 있고
소위 교과서스럽다라고 알려진 걸 능가하는 사례들이 있다.
\vspace{5mm}

이렇게 실증해보면서 한가지는 알았다. 아, 교과서 강조해댄 사람들, 정작 교과서 보지도 않았구나(...)
\vspace{5mm}

그렇다면 여기서 발상의 전환이 바뀐다.
교과서는 최소한의 내용을 담고있다 (X)
모든 교과서 익힘책들을 다 들여다보면 시중교재를 능가하는 케이스도 있다 (O)
\vspace{5mm}

그렇다면 수록은 이제 무의미하다. 교과서에 실렸으니 그게 최고다라는 건 이제 폐기해도 되는 이야기인 것이다.
참고로 덧붙이자면 교사용 지도서 같은 것만 하더라도 대학교 수학의 내용까지 수록한 걸 본다면
가령 미래엔 지도서같은 경우만 하더라도 안드로메다는 저리 갔다는 걸 보면 형식설은 버려도 되는 것이다.
다시 한마디 : 교과서 강조해대는 사람들 그럼 교과서 제대로 본 것 맞냐? 아니, 뭐 보기라도 했다면 제대로 짜깁기 인용이라도 했겟지.
\vspace{5mm}

그럼  그럼 교과서적 정신은 무엇?
\vspace{5mm}

교과서에서 다들 안 읽고 넘어가는 : 단원의 취지, 학습목표, 공부하는 방향 같은 guide나 advice를 얘기하는 것이다.
가령 성지의 경우 단원 구성은
(제목) $-$ \textbf{(학습 목표) $-$ (생각 열기) $-$ (탐구활동)} $-$ (이론) $-$ (문제) $-$ \textbf{(생활 속에서 만나는 수학)}
이렇게 되어있다.
그런데 여기서 학생들이 공부하는 건 제목, 이론, 문제 $-$ 이걸 참고서로 강화해 푼다.
하지만 \textbf{학습목표, 생각열기, 탐구활동, 생활 수학} 같은 건 참고서에서 거의 언급하지 않거나 대충 넘어가지 않나.
그런데 이거야말로 사실 교과서의 정신이 아니겠나.
\vspace{5mm}

\begin{itemize}
    \item[] "저게 수능에 도움이 됩니까"
    \item[] "수능이 무얼 측정하죠"
    \item[] "사고력이죠"
    \item[] "이론만 암기하고 문제만 푼다고 사고력이 늡니까. 자기가 공부하는 학습 목표를 알고 공부한 것이 현실에서 어떻게 응용되는지 알아야죠"
\end{itemize}
\vspace{5mm}

교과서에 들어간 이론은 사실 그리 새로운 건 없다.
그러나 저런 \textbf{'목표', '생각 열기', '탐구활동', '생활수학', '쉬어가는 이야기'} 등은 수험과 관계없어보이지만 실제로는 본질적인 것이다.
사실 저 가이드라인들이야말로 교과서 정신의 정수가 아니겠나.
\vspace{5mm}

그리고 현실에서 교과서스럽게 공부하라는 건 사실 '여집합'의 의미가 있다.
이건 즉, 사교육에서 강조해대는 어떤 꼼수라거나 스킬을 쓰지 말고, \textbf{순수한 정의나 이론만 가지고 문제를 풀라 그 이야기다}.
다시 말해서 2차 곡선 문제가 나오면 무조건 공식부터 쓰지 말고,
2차 곡선 $-$ 포물선, 쌍곡선, 타원의 정의를 떠올리고 어떤 성질이 있었나 연상한 다음
주어진 문제들의 조건을 저 2차 곡선의 성질, 정의 순으로 대응시켜보면서 실마리를 파악하라는 것이지
A 선생이 B한 문제는 C로 풀라고 했지 헤헷 ... 어엉 ... 안 풀리잖아... 라는 짓은 하지 말라는 이야기다.
\vspace{5mm}

그런데 앞에서 언급한 강의나 책들은 입으로는 교과서를 강조하는데 내용은 전혀 교과서스럽지 않다.
더 놀라운. 아니 놀랄 것도 없는 사실은 이런 모순을 수험생들이 지적하는 것을 못 보았다는 것이다.
하기야 붕어빵에 진짜 붕어가 안 들어갔냐 하는 게 더 어리석은 짓일지도 모른다.. 붕어싸만코야 설탕과 쵸콜릿맛으로 먹는 것이지 뭘.
그러나 사소한(?) 모순을 저지르거나 그런 걸 지적하지 못 하면서 수학을 공부한다는 건 어불성설이 아닌가.
\vspace{5mm}

이런 것도 지적 못 하면서 무슨 수학문제 하나 어려운 것 풀었다고 좋아하나.
\vspace{5mm}

그와 별개로 개인적으로는 교과서를 다 구입하면서 피눈물(?)을 흘렸지만 지금은 컬렉터로서의 실속없는 자부심(?)이라는 걸 갖게 되었는데
교과서에 실린 이론이나 문제보다는, 앞에서 말한 교과서 정신에 해당하는 내용들이야말로 저자들의 정성이 들어간 진국임을 맛보게 되어서이다.
특히 나 같은 독서광으로서는 강의보다야 그런 쓰잘데기없는(?) 내용들에서 더 얻는 것들이 많기도 하지만
고교수학이 일상에서 어떤 식으로 구현되는지를 보는 실마리로서 얻은 게 많아서이다.
\vspace{5mm}

다 늙은 머리로 킬러문제를 풀 때 도움되는 것은 사실 저런 것이지, 연구해본다고 들었던 사설인강 그런 건 아닌 것 같다.
인강은 지금 느끼지만 들을 때는 그럴싸했지만 실제 지금 생각해보면 가장 얄팍했다(이채형 강의만은 예외)
오히려 깊은 사고에 도움이 되는 건 저런 교과서 정신, 일본 책이었다.
\vspace{5mm}

시간나는대로 글을 쓰겠지만 $-$ 몇몇은 이런 것도 부풀려서 교재 내서 돈번다고 할지 모르지만 내가 보기엔 참 정신나간 사람들이고 $-$
수학은 어떤 특정한 스킬이나 패턴을 암기해야만 하는 과목이 아니다.
특정 스킬이나 패턴을 암기하는 과목일수록 오히려 웃돈을 주더라도 강의를 들을 필요가 있다. 시간과 노력을 단축시켜주기 때문이다.
\vspace{5mm}

하지만 수학은 '요리'와 비슷한 과목이다.
\vspace{5mm}

무턱대고 설탕을 부어대고 조미료 뿌리거나, 인스턴트 요리 가져와서 뜨거운 물 붓는 걸 제대로 된 요리라고 하나?
물론 급하면 그렇게 먹을 수도 있다. 그리고 그게 다수 수험생들의 현실이다.
하지만 수리논술이든 4점짜리 문제든 그건 "자, 여기 신선한 광어 한마리가 있으니 손님을 만족시켜봐"라는 수준으로 내는 것이다.
즉, 단서 몇개만 줘놓고 그걸로 문제의 목적을 달성하라고 하는 '해결과정'을 묻는 것이다.
그럼 여기서 스킬, 꼼수가 먹히나? 뜨거운 물만 부으면 조리되는 인스턴트 식품만 먹은 애들이 저걸 다룰 수 있겠어?
\vspace{5mm}

그런데 그게 공교육과정에 없는 게 아니엇단 거지. 바로 '날재료부터 요리하는 것'을 교과서에서는 분명히 제시해주었으니까.
\vspace{5mm}

