

\section{[일지 가이드 160108] 가이드 제시}
\href{https://www.kockoc.com/Apoc/580103}{2016.01.08}

\vspace{5mm}

2월말까지
\vspace{5mm}
\begin{itemize}

\item 국어
    \begin{itemize}
        \item 최소 : 최근 3년치 기출 문제를 풀거나 강의만은 들어놓으실 것
        \item 평균 : 화작문 문제집 한권은 제대로 돌려서 오답정리할 것
    \end{itemize}
\vspace{5mm}

    \item 수학
    \begin{itemize}
        \item 최소 : 풍산자 등을 모두 다 풀 것
        \item 평균 : 쎈 또는 마플까지 다 풀 것
    \end{itemize}
    \vspace{5mm}

    \item 영어
    \begin{itemize}
        \item 최소 : 국어에 준함
        \item 평균 : 문법서, 구문서 한권을 제대로 떼거나 그에 준하는 강의 하나 완강, 영어어휘집 돌리고 있을 것
    \end{itemize}
    
    \vspace{5mm}

    \item 탐구
    \begin{itemize}
        \item 최소 : 수능개념 강의 과목마다 돌릴 것
        \item 평균 : 기출 다 분석해볼 것, 그리고 개념서 한권 돌릴 것
    \end{itemize}
\vspace{5mm}
\end{itemize}
이것만 2월말까지 다 한다면 그 다음 일주일간 놀러갔다 오셔도 됩니다.
11월에 시험치고 나서 공부할 땐 저 정도야 하겠지... 라고 망상을 품는데
사실은 4월되어서도 못 하는 케이스도 많습니다.
\vspace{5mm}

일단 평균치는 해야 승산이 있고, 최소는 해야 '인간'대접은 받습니다.
막말하자면 남학생들은 최소 못 하면 그냥 군대 가시고 (군필자라면 빡센 삶의 현장 찍으시거나)
여학생들은 그냥 연애$-$결혼 코스 가는 게 어떻냐고 권해드릴 수도 있습니다.
\vspace{5mm}

국영수탐 중에 2개 정도는 평균, 2개 정도는 최소 유지 정도가 낫겠죠.
그리고 돈이 많은데(!) 공부가 안 된다하면 '오프라인' 사설학원, 학생수준 높은 데 가십시오.
\vspace{5mm}

아마 여기 와서 일지 쓰는 분들이면 어느 수험사이트건 다 돌아다니면서 글 읽겠지만
결론은 결국 "양치기"라는 건 확인했을 겁니다. 양치기 한다고 다 성공하는 건 아니지만, 성공한 사람 중에 양치기 안 한 사람은 없다는 것.
일지들을 쭉 보면서 늘 확인하지만 본인이 '교재나 강의 고를 때 목표한 학습량'과 실제 학습량은 다르다는 것.
그러니까 선택장애 겪지 말고 2월말까지 '최소'는 다 끝내주고 인간대접받으시길 바랍니다.
\vspace{5mm}






\section{[일지 가이드 160110] 강의}
\href{https://www.kockoc.com/Apoc/582726}{2016.01.10}

\vspace{5mm}

사설 들으실 분은 가도 좋은데 효과야 딱히.
\vspace{5mm}

\begin{itemize}
    

    \item 국어
    \vspace{5mm}

    기출강의만 빠른 배속으로 돌려들어보는 것 권하겠음.
    선생들이야 뭐 다 한가닥하는 사람들인데 중요한 건 어느 선생을 듣느냐가 아니라
    A 선생은 $\sim$ 하게 보고, B 선생은 $\%$ 하게 보는 구나... 라는 \textbf{차이}를 본인이 발견하는 것임.
    국어는 정답을 찾으려는 태도 때문에 말아먹음.
    국어 과목의 특징상 정답이란 게 존재할 수 없습니다, 그런데도 불구하고 시험에서는 하나의 정답을 강요하죠.
    왜 정답이 존재할 수 없어서 C도 답이 되고 D도 답이 되는가, 그런데 왜 시험에서는 C만 인정해주느냐
    라는 논리프로세스를 익혀야 합니다. 이럴려면 '각각의 입장'에 따라서 결론이 달라진다는 차이를 봐야하는 것이죠.
    \vspace{5mm}

    정답을 찾는 태도보다도, 왜 오답이 정답보다 더 타당한가라고 소피스트적으로 억지주장을 할 수 있는 능력이 더 중요합니다.
    아니 뭔 소리야, 오답이 타당하다고 하면 국어점수가 나올리 없잖아.
    자기가 억지주장을 하기 때문에 왜 그게 억지인지 스스로 알게 되거든요.
    \vspace{5mm}

    \item 수학
    \vspace{5mm}

    나는 풍산자도 뭐도 모르겠다하면 그냥 EBS 수능개념강의만 따라가는 것도 권하겠음.
    그런데 이거 90강에 육박하니 만만치 않습니다. 하루 3강은 들어야 한달만에 따라잡을 삘,
    그리고 수능기출강의까지 들어주기만 하더라도 으음.
    사실 이 정도도 안 해요. 그래놓고 나중에 자기 인생이 운이 안 좋아서 꼬였다 헛소리나 하고 있지.
    아랫 글에서 적었지만 수학점수 잘 나오니 어쩌니 하면서 결국 상술로 연결시키는 그런데 속지말고
    남들이 뭐라 비웃든 개무시하고 풍산자(쎈)$-$마플로 가거나, 아니면 수능개념$-$기출강의라도 꼬박 따라가세요.
    실제로 EBS 수학강의 다 따라간 사람도 별로 없습니다(따라간 사람이 깔 리도 없고)
    게다고 올해는 3,4점 코스까지 마련할 모양이니
    \vspace{5mm}

    \item 영어
    \vspace{5mm}

    윤연주, 윤장환은 믿고 들어보는 강의라고 생각.
    개념강의 목록 보아도 넘치기만 하지 부족하진 않음. 저게 마음에 안 들면 작년강의 찾아 들어도 좋고.
    그런데 영어는 결국 보편지문 같은 걸 많이 읽고 양놈들 사고방식을 체화시키는 게 더 중요하죠.
    빈칸추론 안 되는 건 간단, 한국인의 정서와 양놈 논리가 불일치하는데 우리 정서대로 풀면 오답 나오기 딱 좋죠.
    역시 이것도 제대로 안 하는 사람도 있을 것임.
    \vspace{5mm}

    \item 탐구
    \vspace{5mm}

    말이 필요없음. 그냥 개념$-$기출강의 따라가면 그만.
    다만 시중교재로도 답이 없는 것들은 어쩔 수 없이 사설 들어야할 건디
    그런 건 대충 3, 4월 이후로 다른 사람들 평가 들어보면서 좋은 것만 골라듣길 바라고
    그 이전까지는 듄 강의 및 교재과 양치기 문풀로.
    물론 화생방 훈련이나 화투질, 무두질로 가시는 분은 없을 거라 믿습니다.
    \vspace{5mm}
\end{itemize}

일단 위의 것들은 자기가 정말 xx 과목에 뭐할지 모르겠다는 사람만 따라가면 됩니다.
듄 수능개념, 기출강의 우습게 보지말고 쭉 따라간 다음에도 개판이면 저 욕해도 상관없습니당.
괜히 xxx 강의 들어야한답시고 웹질이나 하면서 3월까지 놀다가 그 때부터 코스프레해서 n+1이나 하지 마시고요
\vspace{5mm}

그럼 저것들을 다 들으면 뭐하냐.
3월부터 EBS의 논술강의 들으시면 됩니당.
수능과 관계없잖아?
아뇨. 논술강의는 반드시 도움이 됩니당.
시중에 쓸데없는 내용만 담고 비싼 야매교재 같은 거 찾지말고 EBS 강의나 충실히 들으세요.
그래도 모자라면 그 때 가서 사설강의 들으시면 되는 것입니다. 괜히 프리패스 혹해서 인생 저당잡히지 말고요.
\vspace{5mm}









\section{[일지 가이드 160111] 수학 개념 증명해보기}
\href{https://www.kockoc.com/Apoc/584086}{2016.01.11}

\vspace{5mm}

풍산자, 쎈, 마플, 일품, 라벨에는 문제에 쓰여야하는 정의, 성질, 공식, 팁 등이 나와있다.
\vspace{5mm}

\begin{itemize}
    \item 풍산자 $-$ 감각적이고 직관적으로 설명
    \item 쎈 $-$ 정말 필요한 것만 알뜰하게 적어둠(처음에는 왜 이렇게 빠진 게 많아 하겠지만 나중에 내공늘면 알 것)
    \item 마플 $-$ 수능에 쓰일 수 있는 수준으로 응용, 심화시켜놓음
    \item 일품, 라벨 $-$ 고난도 풀이를 위한 수준으로 정제해놓음
\end{itemize}
\vspace{5mm}

그런데 저기 빠진 것들이 바로 '증명'임.
쎈 기준으로 가자면 개념 설명의 공식들은 증명되어있지 않은 것들이 많음.
\vspace{5mm}

2월말까지 최소공부량 달성하시면
본인들이 보신 책의 개념 설명에 빠진 증명들을 스스로 하거나 아니면 교과서, 다른 참고서, 인강을 통해 얻은 걸 가필해보시길 바람.
특정 단원의 특정 공식이 어떤 맥락에서 어떻게 나온 것인지 그걸 복기해보고 설명할 수 있어야 문풀실력이 늘어남.
예컨대 확률과 통계에서 중복조합 공식, 자연수 분할 공식, 집합 분할 공식의 경우는
$-$ 수식적 설명, $-$ 국어적 설명이 양쪽 모두 가능한데 이것들을 찾아보고 채워넣는 게 중요함.
(특히 확률과 통계는 해당 공식이 왜 나왔는지 생각 안 하고 풀면 '산수'가 되어버림. 그 공식이 나오게 되는 과정을 분명 복기할 줄 알아야 함)
중복조합은 작대기를 이용한 증명이라거나 아니면 a+b+c=k 와 같은 증명으로도 족하고 자연수 분할 등도 이에 준함.
\vspace{5mm}

수학이 양이 많은데 왜 이리 내공을 들여야하냐, 걍 빨리 풀면 안 되냐 뭘 과정 쓰고 그러냐 할지 모르지만
이게 국어, 영어, 탐구 문풀에 막대한 영향을 끼치기 때문에 함부로 할 수 없는 것입니다
얼마나 빨리 푸느냐도 중요하지만, 그 이전에 얼마나 '정확한 논리를 정교하게 구사하느냐'가 더 중요하다는 점을 일러드리고 싶음.
자기가 보았던 참고서에 나온 개념은 강도에게 포박당한 상태에서도 줄줄 암송할 줄 알아야 하며
모든 공식이 다 어떻게 유도되는지 그 전제, 조건, 과정, 맥락도 술술 풀어댈 줄 알아야함, 수리적 사고력이 여기서 출발하는 것임.
\vspace{5mm}

그럼 이걸 진작하지 왜 양치기를 하고 하느냐....
저런 증명은 어느 정도 패턴화가 되어있지 않으면 받아쓰기만 하더라도 당장 이해가 되지 않기 때문임.
선문풀하고 후증명해보고 나서야 자기가 어떻게 공부했나 반성해볼 수도 있으며 잽싸게 교정이 가능함.
그러면서 논리라는 걸 세울 수 있음.
\vspace{5mm}

본인이 논리가 잡히면 그 다음부터 고난도 문제에 스스로 부딪쳐서 '자기 논리로 문제를 해부해보는' 것을 경험해보시면 됩니다.
이걸 직접 해보아야 실력이 늘지, 다른 것 해보았자 절대 안 늘어요.
이걸 늦어도 5월부터는 해야하는 것입니다. 당연히 웬만한 양치기는 5월 이전에 끝내놓아야 함.
6평부터는 소위 킬러문제에 대해서 본인들도 풀고 토론해보고 그래야만 안 무서워하지 안 그러면
시험 끝날 때까지 꿀교재니(그런 게 어딨어), 특정 강사 강의 들어야하니 그러는 겁니다.
\vspace{5mm}







\section{[일지 가이드] 30번 쓰신 분들 이상}
\href{https://www.kockoc.com/Apoc/586339}{2016.01.12}

\vspace{5mm}

일지 게시판에 쓴 게 지워져서리 여기 올립니다.
어차피 들어와서 공부할 회원들은 다 모인 것 같아서.
\vspace{5mm}

가칭 '하원' 게시판은
\vspace{5mm}

\begin{itemize}
    \item[$-$] 칼럼 작성한 콕창
    \item[$-$] 일지를 쓰는 학생(30번 이상)
\end{itemize}
\vspace{5mm}

에게 권한을 드립니다.
\vspace{5mm}

그리고 거기서 일지를 쓰셔도 좋고 아니면 그냥 현재 공개 일지를 쓰셔도 좋은데
핵심은 올해 시험 대비하는 분들끼리 $-$ 상위권 허세 신경쓰지 말고 $-$ 서로 조언해주면서 친목질하며 수험을 영위해가는 것이 되겠습니다.
일지 피드백은 '일주일에 한번씩 합산'해서 올린 것을 보고 드림.
작년은 매일 했는데 시간도 많이 걸리고 비효율적이더군요.
\vspace{5mm}

수험사이트들이 지나치게 상위권 명문대 의대 간다에 치우쳐있는데
그런 건 별로 관심없다고 말씀드립니다. 개인적으로는 그런 걸로 허세 피우는 건 정말 싫어해서리.
어차피 한번 살다가 가는 인생 각자가 얼마나 열심히 공부해서 성과 거두고 자기 인생 펴나갈 수 있나 그런 것이 취지죠.
작년과 달리 수험고수(...)들도 많아졌고 조언 줄 사람들도 늘어났으니 사이트에 기여하고 자기도 고수가 되겠다(라지만 붙는 게 좋겠죠)
고 마음 먹으시면 일지 꾸준히 쓰시고 들어오시면 되겠습니당.
\vspace{5mm}

사실 수험생들에게 필요한 건 정보보다는 '안심하고 소통할 수 있는 공간'인데
그런 게 부족했으니.
\vspace{5mm}

조건은 어찌되었든 일지에 하루에 1번꼴로 30회씩 올렸냐는 것이고
내용상 허위거나 형식적인 게 아니라 진짜 공부했냐하는 것입니다.
가칭 하원은 대신 어그로를 끌거나 공부에 방해주거나 하면 자격박탈도 꽤 쉽다는 것이 제약조건으로 붙겠네요.
\vspace{5mm}

+
\vspace{5mm}

덧붙이면 수험정보는 어느 정도 비공개성이 필요하다일 건데 뭐 그런 것도 감안하지 않을 수 없겠죠.
\vspace{5mm}










\section{[일지 가이드 160116] 지금 공부가 되고있는 증거}
\href{https://www.kockoc.com/Apoc/590446}{2016.01.16}

\vspace{5mm}

\textbf{우울하다}
\textbf{힘들다}
\textbf{공부하기 싫다}
\vspace{5mm}

라고 하는데 꾸준히 교재는 풀고 있으면 지금 이건 정말 공부를 하고 있단 증거임(가을에 겪으실 걸 지금 미리 겪고 있음)
반면 공부가 너무 잘 된다거나 아무 생각 없다면 좀 의심을 해보아야할 듯.
\vspace{5mm}

공부를 하다가 우울해지는 건 아래 얘기한 '역금단증상' 때문입니다.
\textbf{원래 뇌는 공부하기 싫어한다 $\rightarrow$ 공부 상태가 지속된다 $\rightarrow$ 뇌에서 공부를 안 하기 위해 우울한 상태에 빠진다}
\vspace{5mm}

그럼 언제까지 하느냐. 2월 첫째주까지만 달리셨으면 합니다.
그 다음 주는 어차피 \textbf{'설날'}이 있어서 공부 하라고 해도 못 할 각이죠.
\vspace{5mm}

아무리 공부해도 이해가 안 된다... 싶으면 질문을 때리거나 인강, 과외를 이용하겠습니다만
대부분의 의문은 "탐구"나 "반복"으로 해결됩니다.
생소한 지식체계에 자신의 뇌를 적응시켜나가는 게 중요하다는 점에서는 힘들더라도 규칙적으로 반복해주는 게 가장 좋습니다.
반복하다가 뇌에서 적응하는 순간 저절로 이해가 되는 유레카 순간이 오면 이게 하나로 끝나는 게 아니라 쌓여왔던 의문이 거의 다 풀리기도 하는지라.
\vspace{5mm}

아 그리고 총회자격부여되신 분들은 상단 안전구역에서 총회에 나오는 '안식처' 게시판
그리고 기록실에 나오는 총회일지를 이용하실 수 있으실 것입니다.
안식처 게시판은 칼럼이나 일지를 일정량 이상 쓰신 분들께서 자유게시판처럼 이용하면서 부담없이 글을 쓰시면 되겠고(수질관리합니다)
현재 일지가 노출되는 게 거북하다 하실 분들도 많으니 그런 분들은 총회일지를 이용해주시면 되겠습니다.
\vspace{5mm}



