%\usepackage{amssymb}

\section{일격 레이드 들어갑니다.}
\href{https://www.kockoc.com/Apoc/246737}{2015.08.12}

\vspace{5mm}

\href{http://kockoc.com/BookRaid/245328}{링크}
\vspace{5mm}

제 팔은 절대 안으로 굽지 않습니다.
콕콕에서는 영리활동을 하지 않기 때문에 저야 거침없이 할 말을 합니다.
저런다고 일격이 나쁘다라고는 안 하겠지만, '아쉬운' 것은 당연히 지적하는 거죠. 그래야 내년에 더 좋아질테니까요.
제작자 분들이 단기적인 이익에만 눈이 멀었다고 보진 않아서요(불행하지만 그런 업자들이 너무나도 많습니다)
앞으로 좋은 교재를 만들고 싶으시면 하나하나 다 개선해나가셔야합니다.
\vspace{5mm}

불모지 탭에 보면 교재 사냥터와 교재 레이드가 만들어졌지요.
교재 사냥터에서는 마치 저평가 기업 고르듯 각자가 보는 소박한 교재 올리고 평가해보고 다른 사람들 반응 보시거나
아니면 지금 시작하고 있는 일격 레이드 등을 그대로 연동해서 따라가주시면 되겠습니다.
\vspace{5mm}

수험사이트들을 가보면 지나친 과장이나 혹은 까내리기가 상업적인 목적 하에 자행되고 있던데 그딴 건 당연히 할 리는 없죠.
그냥 차분하게 좋은 건 좋고 나쁜 건 나쁘고, 광고가 안 되었는데 괜찮은 건 평가해주고 하면 되는 것입니다.
\vspace{5mm}

현재 0회차 $-$ 즉 포장지 뜯고 해설지 훑어보면서 \textbf{개선점}을 제시했습니다.
사실 이건 만든 분들에게 꽤 엄하게 지적드리는 데, 저건 제작과정에서 조금만 신경쓰면 개선가능한 문제였다고 봅니다.
12회차까지 레이드 가면서 이제 개인 가격을 매기겠습니다만, 저것 때문에 우선 $-$10000원 계산을 할 것입니다.
\vspace{5mm}

그래도 적어도 손발이 사라진 건 다행(...)입니다. 뭐 그게 좋다고 하는 분들도 계시겠지만(...)
내일은 1회차 A, B형 간략 리뷰 올라가고 레이드 참여자들(현재 상원 국한)에 대한 지시가 있을 것입니다.
레이드에 참여 안 하시는 분들도 비슷하게 연동해서 가시거나 아니면 뜻맞는 분들끼리 진도 빼서 가시는 것도 괜찮겠죠.
\vspace{5mm}

그리고 다른 분들도 일격에 대해서는 $-$ 제가 제작자는 아닙니다만 $-$ 쓴소리를 아끼지 마시길 바랍니다.
\vspace{5mm}

+ 死만원으로 살 수 있었던 것
\vspace{5mm}

아, 물건너갔다.
\vspace{5mm}




\section{공부법 책은 사실 거의.}
\href{https://www.kockoc.com/Apoc/251344}{2015.08.14}

\vspace{5mm}

소위 공부하는 법 가르쳐줄께... 하는 케이스는 99$\%$ 장사치입니다. 그런 건 돈받고 파는 건 아니죠.
그러기 전에 본인들이 먼저 검증해야하지 않나 합니다만.
우리나라에서 공부 가장 많이 한 것으로 보이는 고시합격자 집단의 수기 읽어보면 새로운 것 그런 것 없어요.
결국 얼마나 시간낭비 줄이고 많이 (반복해서) 보느냐입니다.
\vspace{5mm}

일본의 공부법은 그나마 낫습니다만 이것도 별 것 없습니다.
일본의 한 미모(?) 변호사가 7번 읽기법 낸 거. 그걸로 끝입니다. 이것말고 다른 공부법 책들도 많습니다만 사실 무쓸모이죠.
뇌의 생리를 고려한다든가 더 효율적인 자료 정리한다든가.. 뭐 쓸모 없지 않은데 효율은 낮습니다.
그런데 이건 \textbf{"옆에서 잔소리해주고 빠따로 갈겨주는 사람"} 미만잡입니다.
\vspace{5mm}

그보다는 오히려
\vspace{5mm}

$-$ 전자파를 차단할 것, 컴으로부터 격리된 시간 가질 것
$-$ 자기가 약속을 어기면 고통을 입도록 시스템 짜놓을 것
$-$ 책상과 의자에 투자할 것
\vspace{5mm}

이런 게 중요하겠죠. 공부법 책 사둘 돈이 있으면 독서실을 끊든가, 컴퓨터 없는 방에서 쓸 수 있는 넓은 목재 table 사는 게 낫습니다.
사실 컴 얘기 나와서 그럽니다만 다 필요없고 \textbf{컴과 맛폰만 멀리해도 성적 올라간다는 게 불편한 진리죠.}
\vspace{5mm}

돌아다니다보면 별 것도 아닌 것 가지고 장사하려는 사람들이 꽤 많다라는 걸 봅니다. 수년 전부터 주욱 보였지만요.
\vspace{5mm}






\section{실모에 대해서 또 다른 생각 다시 적습니다만.}
\href{https://www.kockoc.com/Apoc/251556}{2015.08.14}

\vspace{5mm}

일격 1회 A,B형을 풀고 최대한 틀려보려고 하면서(어디서 실수할 것인가) 리뷰하고
해설 비교하면서 느낀 건데.
\vspace{5mm}

그 저자 분들의 생각과 달리 기존의 제 입장 $-$ \textbf{"야매교재 보지 말고 검증된 것 가지고 철저히 보아라"}하는 것
이 주장 고수해도 되겠네요.
\vspace{5mm}

기초가 탄탄하지 않은 상태에서 실모 보는 건 그냥 \textbf{자살행위}입니다.
문제가 좋네 안 좋네 그런 게 중요한 게 아니라
문제 하나를 틀리더라도 그걸 정리하면서 깨닫고 실력을 키우고 이런 게 중요하고
그런 차원에서 실모를 보아야하는 건데
\vspace{5mm}

뭔가 본말이 전도되어도 한참 전도되어버렸죠.
\vspace{5mm}

일전에 키배(?) 비슷하게 뜨면서 야매교재 옹호파(?)에서 그럼 일타, 일격은 왜 말이 없냐라고 하기에
그렇게 \textbf{입으로만 나대는 케이스} 경멸하는 차원도 있어서 지금 일격 하나씩 시간 들여가면서 리뷰합니다만.
1회차만 봐도 느껴지는데 전혀 제 생각은 바뀔 게 없을 것 같네요.
리뷰다는 것도 그렇자면 철저히 야매스러움을 벗어나고 품질강화하라는 조언입니다.
만약 판매량에만 신경쓰고 (이 시장은 정상적인 시장이 아닌 것을 아시겠죠) 자기가 스타라고 착각한다면
(본인들은 아니라고 하겠지만 제가 보기엔 딱인 걸 무슨) 그거 불행한 일이지요.
\vspace{5mm}

실모는 정 보려면 하나만 제대로 보시길요. 전 오히려 A, B형 1회 리뷰하는 것도 시간이 엄청 걸렸습니다.
\vspace{5mm}

+ 판매량 가지고 자랑하는 케이스 있던데 그건 사실 한심한 겁니다. 독자들이 평가해야지 업자가 그러는 건 웃긴 것임.
돌아다니다보면 그런 케이스도 있고 아마 여기도 예외는 아닐 것인데. 광고 차원이라면 모르겠고 출판사가 그러는 것까진 이해는 가는데
저자가 그러는 건 정말 쪽팔린 겁니다. 그럴 시간이 있으면 해설에 공들이세요.
\vspace{5mm}

+ 작년 말부터 시비 거시던 분들은 정작 하나라도 뭔가 기여한 건 없다는 것 재차 확인. 실천 안 하는 사람은 경멸하지 말입니다.
익게건도 그렇고 야매교재 논쟁도 그렇지만, 어설프게 시비는 걸지 자기가 상처입을 걸 두려워하는 그런 사람은 대놓고 경멸하지 말입니다.
\vspace{5mm}

+ 일격까로 읽힐지도 모르겠습니만, 더 정확히 말하면 까의 입장에서 지금 검토하고 문제삼을 건 다 삼아 지적하자 그 입장입니다.
1회차만 보면 문제는 킬러에 한해서 컨셉 괜찮은 것들이 있습니다.
물론 컨셉이 괜찮은 게 다는 아니고, 이게 정말 도움이 되는지는 더 신중한 입장을 취해야지요.
그런데 해설은 더 손봐야할 건 있어요. 노력한 흔적은 보이는데 그것만 가지고는 무리입니다.
최종 마무리가 약간 부족하다... 라는 아쉬움이 있습니다. 그 점에서는 제가 쓰는 보고서가 약간은 도움이 되었으면 합니다.
문제차원에서 보자면 보면 볼수록 괜찮다... 하는 것들은 있습니다. 그런 건 아예 가격을 주관적으로 산정해 넣었고 근거를 대답할 수는 있습니다.
\vspace{5mm}

이거 점수 안 나왔다고 좌절하시지 마시고, 약재 고아먹듯 문제 자체를 계속 여러번 풀고 생각해보세요.
\vspace{5mm}

+ 이것도 얘기해야겠는데 야매교재 논쟁 중 하나가
쎈수학으로 부족하다, 실모가 최고다. 그런 이야기인데. 이거 도무지 근거가(...)
그 분들 쎈수학이라도 제대로 공부하지 않았다 쪽으로 확신이 듭니다.
실모도 실모 장점이 있겠지만 내용이나 회독수 상승시 얻을 수 있는 것으로선 쎈수학, 정석 등을 따라잡을 게 없어요.
\vspace{5mm}

+ 더 공포스러운 게 실모 판매량이 증가한다.... 뭐 다 좋은데
판매량이 많다는 건 '기본적인 공부도 안 한 친구들이 실모를 본다' 그 이야기인데.
이거 업자들이야 그렇다 치고 소비자들은 자기들이 어떤 상황에 가는지 알지는 모르겠습니다.
그저 웹에서 문제 좋다 하는 것에 혹해서 자기 상황 모르고 구입하면 어떻게 되는 걸까.
\vspace{5mm}

+ 한편 쎈 등의 시중교재, EBS 본 사람에게는 일격 실모는 도움이 된다는 건 분명합니다.  겹치지 않는 부분들이 있어보여요
다만 그 겹치지 않는 부분은 시중교재와 EBS 등을 충실히 했을 때 그 \textbf{진가를 알 수 있다}는 게 함정이지만요.
\vspace{5mm}







\section{국어는 걍 답이 없습니다.}
\href{https://www.kockoc.com/Apoc/261711}{2015.08.19}

\vspace{5mm}

\href{http://kockoc.com/column/261153}{링크}
\vspace{5mm}

간만에 칼럼에 좋은 글이 올라와서 토론하다 느낀 것이죠.
\vspace{5mm}

반발하는 사람들도 많겠지만 일단 적으면
수학 실모는 만들기 쉬운 편입니다. 왜냐면 참조할만한 소스가 꽤 많고, 더군다나 조금만 꼬아내도 오류가 생길 일이 별로 없어요.
(당사자가 실력이 거품이 아니라면 말입니다)
\vspace{5mm}

그런데 국어는 그게 아니죠. 잘 내더라도 오답 시비의 가능성이 잠재하고 있습니다.
왜냐면 국어는 화작문을 제외한다면 절대 정답이 하나만 나올 수 있는 과목이 아니거든요.
그게 국어란 과목의 본질이기 때문.
\vspace{5mm}

그래서 국어는
\begin{itemize}
    \item[$-$] 사설인강, EBS인강
    \item[$-$] EBS 문풀이나 실모
    \item[$-$] 기출
\end{itemize}

\vspace{5mm}

그 어느 것도 답이 되기 어렵습니다.
애매한 걸 굳이 억지로 얘기한다면 "상식적인 수준에서의 국어적 사고"라는 게 있긴 하겠고
그게 기출을 통해서 키워지기는 하는데 한계가 많죠.
\vspace{5mm}

간혹 수험사이트들 돌면 국어 실모나 문제가 얼마나 개판이냐... 라는 지적을 봅니다만
그거야말로 오만한다고 봅니다. 그거 제작자나 그런 사람들 제가 알 리는 없으니 옹호해주는 건 아니고(그렇다고 구입하란 얘기도 아니지만)
수학 실모가 난립하는 반면 국어나 영어는 별로 없거나 '변형'이라는 이름으로 짜깁기 가는 건 다 그럴만한 이유가 있는 거죠.
만들기 어려우니까 말입니다(다만 만들기 어렵다면 만들지 말든가, 제대로 만들면 되지 않나)
\vspace{5mm}

그래서 그동안 국어는 평가원에서도 꽤 난이도를 자제한 편이긴 한데
작년 시험 기점으로 수학 뿐만 아니라 국어도 $\sim$ 하게 출제하면 된다라는 것을 눈치깠다고 확인할 수 있는 정황이 포착되죠.
수학은 즉 쉽게 내면서도 나름 변별력을 줄 수 있다라는 것 $-$ 쉽게 내는 게 추세긴 하지만 $-$ 을 알아챈 것 같고
국어도 역시 어떻게 하면 오답시비 안 내면서 수준있게 낼 수 있는지도 파악한 것 같더군요.
여기 쓰긴 그렇습니다만 그런 정황은 역시 일본 쪽 입시문제에서 발견된 것이긴 한지라.
\vspace{5mm}

지금 고3이면 모르겠고 고2라면
요즘 독학용 교재도 많이 나왔으니까 이과라고 하더라도 인문논술의 사고법 정도는 익혀두시는 게 좋지 않나가
지금으로서 드릴 수 있는 최선의 조언일 겁니다.
비문학, 문학 독해는 문제를 보자마자 머릿 속으로 미니논술을 작성하지 않으면 안 되는 식의 문제풀이로 갈 가능성이 높습니다.
영어의 빈칸추론이 이미 그런 수준이지만요.
\vspace{5mm}

+ 상관없어보이는(?) 인문학 질문이 나와서 그러는데 제 대답은 그렇습니다.
"그 인문학자들이 헤게모니를 잡고 있던 국가나 사회는 부유해졌나"
\vspace{5mm}

소크라테스야 왕따당해서 사실상 살해당했고
플라톤은 군주 한명 과외제자로 잘 가르쳐보려다 배신당했고(그러기보다도 라톤이 형은 덕후였잖아)
아리스토텔레스는 제자가 무려 알렉산드로스여서 요새로 치면 마케도니아 스터디 원장으로 잘 나갔다가
물수능, 아니 외지인 추방정책으로 물 더럽게 먹었죠
\vspace{5mm}

애시당초 인문학만으로 모든 게 해결되었다면 옛날 국가들이 삽질을 겪지는 않았겠고
경제학이 나타날 이유도 없었겠죠.
조선왕조 500년이 왜 궁핍하고 가난하게 살았는지만 봐도 좋습니다.
\vspace{5mm}

+ 수학 교육의 의의는 저런 인문학의 '광기'를 막아주기 위한 브레이크라고 해도 사실 지나친 말은 아닐 겁니다.
아마 수학을 가르치지 않으면 '답이 없는 국어' 공부만 하면서 또 이 지옥불 반도는 21세기 예송논쟁이나 하겠고... 가 아니라
문돌이 어르신들이 쓸데없는 것 가지고 답없는 논쟁 하는 것 보면 끝이죠.
\vspace{5mm}








\section{공부 못 하는 애와 잘 하는 애의 결정적인 차이.}
\href{https://www.kockoc.com/Apoc/284872}{2015.09.02}

\vspace{5mm}

하위권이 중위권으로 올라서거나 중위권이 상위권이 되기 위해서
상위권이 최상위권이 되는 과정은 태권도의 다리찢기와 비슷한 단계를 반드시 거쳐야 한다.
\vspace{5mm}

가령 중위권이 하루에 푸는 문제량이 `100개라고 하자
그런데 하위권 애들은 공부하겠어요... 하지만 80개 정도를 일주일 정도 지속하면 짜증을 내거나 그만두려고 한다.
마찬가지로 중위권 애들을 억지로 상위권 코스에 맞춰 150문제에다가 킬러 10문항을 풀게 하자
머릿 속에 안 들어간다고 하면서 공부를 거부하려 한다
\vspace{5mm}

'말은 잘 하는 사람'은 신용하기 힘들 것이다.
그런데 그렇게 따지면 학생들의 신용수준은 개차반이다.
공부를 잘 하고 싶다라고 하지만 실제로는 조금만 공부하는 상태를 만들어놓으면 그 유지를 하지 못한다.
\vspace{5mm}

공부를 잘 한다라는 건 단지 좋은 인강과 교재만 있다고 되는 것이 아닌 것이다.
그걸 소화할 수 있도록 학습량이 늘어야 하고, 그 학습량을 소화시킬 수 있도록 그릇을 바꿔야 한다.
그 점에서 보자면 가장 중요한 건 '성격'이고, 그렇다면 \textbf{공부를 잘 한다라는 건 '성격'을 바꾸는 것이다.}
아마 학생 개개인들은 그걸 알기 힘들 것이고 사실 관심도 없겠지만,
여러 사람들을 $-$ 최소 5명 이상 경험하다보면 교재도 스킬도 강의도 아닌 '성격'(그리고 그걸 좌우하는 집안환경과 유전)의 중요성을 알게 된다
\vspace{5mm}

이상과 달리 현실은 "그 자식은 그렇게 망할 수 밖에 없어"라는 탄식이 나오는 것도 그런 것이다.
그 점에서 보자면 인강이나 대형학원 강의는 준비할 게 많다 해도 편한 것이 있다. 강의를 듣던말던 학생들의 성격에 신경쓸 필요는 없기 때문이다.
\vspace{5mm}

그럼 공부를 못 하는 애들이 머리가 나쁘고 철학이 없나
짤방대로이다. 머리가 좋은 케이스가 많고(오히려 좋으니까 공부를 안 한다는 게 문제)
개똥철학은 정말 많아서 부모님이 해주는 밥을 먹고 용돈을 쓰면서 개똥철학을 논하는 케이스가 많다.
그렇기 때문에 "의미없고 자유가 없는 공부"는 할 필요가 없다고 자기 정당화를 한다.
유감스럽지만 수험은 '자유를 포기하고 의미 같은 걸 추구하지 말아야' 성공한다는 진리를 모르거나, 알면서도 인정하지 않으려는 것이다.
\vspace{5mm}

그 점에서 보자면 중2병적인 것을 촉진하는 일종의 인문학적인 개똥철학이나 소설작품 등의 해악을 부인하기는 어려울 것이다.
각자 자아를 찾아라, 의미를 갈구하라, 그리고 자유를 추구하라... 감동적이긴 한데 사실 '생산$-$소비'의 틀로 본다면 허무맹랑한 이야기다.
정말 자아를 알고 싶으면 우리가 뭘 자유롭게 '생산'해서 그걸 '소비'와 '투자'로 연계시킬 수 있느냐 따져야지
그런 게 되어있지 않은 이상은 순전 말뿐인 개똥철학은 뇌내망상에 불과한 것이다.
그래서 일찍 철이 든다는 것도 좋은 건 아닐 수도 있다. 수험판에서는 고득점만 인정될 수 있기 때문이다.
\vspace{5mm}

결국 자기가 생각하는 것 갑절 이상의 학습량을 유지하다보면 필연적으로 찾아오는 그 '장애'를 넘어서는 수 밖에 없다.
한마디로 자기의 한계를 넘어서야하는 것이다.
50문제 풀면 공부하기 싫어지고 다 때려치우고 싶어하는 근성, 조금만 공부하다가 바로 인터넷 접속하거나 딴짓하는 습관.
중하위권들은 이런 습관을 고치지 못 하면서 좋은 교재와 강의만 찾는다.
\vspace{5mm}

그럼 사설강의는 어딴가. 결론적으로 어떤 것이든 '달콤'하다.
실제로 강의를 듣고 공부가 재밌어졌다하는 것은 인정할 측면이 있짐나 반면 위험한 것도 있다.
인기있는 강의란, 실력을 높여주기보다는 오히려 '들을 때 피로감이 덜 하고 재미있는 강의'일 가능성이 크다.
그런데 그런 강의가 재밌는 이유는, 특정 단원의 특정 내용을 공부할 때 반드시 거쳐야하는 '어려운 과정'을 생략하거나 꼼수로 넘어가기 때문이다.
또한 특정한 내용은 단지 이해만 할 게 아니라 정말 지루해도 암기해야 하며, 부단한 연습이 필요한 경우가 많다.
하지만 인강이 이런 것까지 책임져주기 기대할 수는 없는 것이다.
\vspace{5mm}

중하위권이 인강을 듣건 학원을 다니더라도 '성적'이 안 오르는 것이 이렇게 설명되는 것이다.
괴롭고 힘들고 짜증나는 대목, 즉 자기의 한계를 넘어서는 대목, 헌독을 깨고 새독을 빚는 대목
이런 것들을 3$\sim$4개는 거쳐야만 비로소 공부할 수 있는 사람이 되는 것이지 그렇지 않으면 '개똥철학 읊는 잉여'로 전락한다.
아마 이 글을 읽는 상당수는 자기는 안 그럴 거라고 생각하겠지만 정말?
\vspace{5mm}

분명한 건 수험은 자기 의미를 찾는 과정도 아니고 자유는 일찌감치 포기해야하는 비인간적인 경쟁이란 것이다.
이 점을 수긍하지 않는다면 그냥 수험을 포기하고 다른 길로 가는 게 낫지 않나... 하지만
자본주의 사회에서 $-$ 아니 인류사회에서 그런 동물의 왕국급 경쟁이 없는 곳이 어디있을까.
\vspace{5mm}






\section{2016년도 9평 A형, B형 수학 분석}
\href{https://www.kockoc.com/Apoc/285090}{2015.09.02}

\vspace{5mm}

B형 총평 : 30+0$-$5
팩트 : 평가원은 쉽게 내면서 실점 나오게 하는 방침 터득
\vspace{5mm}

B형 문항별 분석
\vspace{5mm}

\begin{itemize}
    \item 1$\sim$6 : 생략. 그런데 이걸 내면서 출제자는 어떤 기분이 들까.
    \item 7 : 이거 식 말고 그림으로 푸는 게 좋죠. 1차변환 고수들은 행렬을 안 쓸 수 있으면 안 쓴다능.
    \item 아마 7번 틀렸으면 보나마나 3과 3root(2)를 헷갈렸을 겁니다.
    \item 8 : 이거 x 범위 제한 둬서 난이도 높이지
    \item 9 : 생략
    \item 10 : 이런 거 틀리는 애들도 있죠
    \item 11 : 익힘책 수준
    \item 12 : 좌표접선과 기울기접선 공식은 잘 활용. 그런데 이런 건 주관식으로 내면 더 좋았을 듯
    \item 13 : 생략
    \item 14 : 부피구하는 문제 :  출제자가 너무 착해서리 ($e^2-1)\pi$ 를 선지에 안 넣었죠.
    \item 15 : $\bigstar$ 쉬운 확률 문제 : 단, 식으로 접근하려던 친구들 시간 많이 잡아먹었을 겁니다. 오답률 생각보다 높지 않을까
    \item 16 : 점화식 문제 : 그나마 중간 정도 되는 문제였는데 이것도 힌트가 많아서리.
    \item 17 : 9평에 마지막으로 등장하는 행렬 합답형. 어째 실전에서는 어렵게 내지 않을까, 역시 쉬웠습니다.
    \item 18 : $\bigstar$ 오답률 높다고 보는 통계문제 P(Y>=26)>=0.5라는 조건 안 쓰면 엉뚱한 답이 나옵니다. 선지도 착하지가(?) 않습니다.
    \item 19 : $\bigstar$$\bigstar$ 업그레이드해서 30번으로 내도 좋았을 2차곡선 문제 P의 위치를 하나로만 생각하기 쉽죠.
    \item 20 : 뭔가 무색해지는 무등비 쉽습니다.
    \item 21 : 19금 장면이 나올 찰나에 스탭롤이 올라가는 적분그래프 문제입니다(...) 이거 틀린 사람들 이불은 오늘 무사하실까?
    \item 22 : 생략
    \item 23 : 생략
    \item 24 : 생략
    \item 25 : 생략
    \item 26 : 삼각형 넓이 구하는 방법을 묻는 문제. 정사영은 그냥 장식품
    \item 27 : $\bigstar$$\bigstar$ 그래도 꽤 흡족한 중복조합 문제였습니다. 요즘 중복조합은 케이스나누기 필수
    \item 28 : 불멸의 사인법칙
    \item 29 : $\bigstar$$\bigstar$ 푸는 방식이 다양합니다. 법선벡터 풀이도 있고 아니면 평방 풀이도 있음. 평방 풀이시 '내분' 잘 이용하세요
    \item 30 : '$\sim$의 정리'만 잘 쓰면 끝나는 싱거운 문제
\end{itemize}
\vspace{5mm}

A형 총평 : $28+2-3$
팩트 : B형이 어렵다고 A형으로 도망간 학생들 자살각
\vspace{5mm}
\begin{itemize}
    \item 1$\sim$6 : 생략
    \item 7 : 계차수열 정의알면 한줄 풀이
    \item 8$\sim$11 : 생략
    \item 12 : 좌표 계산할 수 있느냐 하는 기본문제
    \item 13$\sim$14 : 생략
    \item 15$\sim$16 : 생략
    \item 17 : 문과 수열치고는 대략 중간 정도
    \item 18 : 합답형 쉽게 나왔네요
    \item 19 : 케이스 구분은 중복조합 문제에선 이제 필수
    \item 20 : $\bigstar$ 대소관계 잘 파악해서 무등비
    \item 21 : $\bigstar$$\bigstar$ t의 범위에 따라 함수, 도함수 정의한 뒤 문제에 제시된 부등식의 의미 파악하고 풀면 됨
    \item 22$\sim$26 : 생략
    \item 27 : $\bigstar$ 계산 충실히
    \item 28 : 생략
    \item 29 : $\bigstar$$\bigstar$ 정규분포의 정의를 아느냐 물어보는 문제
    \item 30 : $\bigstar$$\bigstar$$\bigstar$$\bigstar$ 통수갑. B형에 냈으면 변별력 좋았을 문제임. 숨겨진 m의 범위를 찾아라
    \item 30번 문제는 저도 헤맸는데 '가수'는 자릿수와 독립해있지 본다면우리의 상식을 넘어서는 자릿수를 만들 수 있음.
    B형 풀고 싱겁네 하는 분들도 A형 30번은 풀어보시길.
\end{itemize}
\vspace{5mm}

\textbf{사견}
\vspace{5mm}

\textbf{​EBS 수특, 수완만 꾸준히 풀었어도 200점은 나오는 출제였습니다.}
(반론하고 싶으신 9평 문제와 직접 비교해보면 끝) 난이도로 비교하면 절반도 되지도 않았기 때문.
'기본'적인 걸 소홀히 하면 어이없이 틀릴 수도 있었던 문제들이 많았죠.
\vspace{5mm}

그런데 9평의 목적은 결국 막판 난이도 조절을 위해서 고3들 수준이 대략 어떤가 시험해보는 것이고
그런 차원에서 쉽게 내지 않았나 싶기도 한데, 제가 아는 한 올해 고3들은 수준이 낮지 않은 편이라서리.
그러면 본 시험은 쉽게 나오리라고 장담하기는 어려울 것 같습니다.
\vspace{5mm}

조언
\vspace{5mm}

\begin{enumerate}
    \item \textbf{실점한 문제에 관해서는 까다로운 것 100문제 정도 풀어보시길}
    예컨대 A형 30번에서 나가리났다하면 지표와 가수 관련된 4점짜리 문제만 골라서 풀어보라 그런 이야기
    선택과 집중이 필요합니다.
    \vspace{5mm}
    
    \item \textbf{A 문제는 $\sim$ 하게 푼다는 스킬보다는 문제 리딩을 얼마나 잘 하느냐가 성패를 좌우하고 있습니다.}
    이건 제 실모비판론과도 맥락이 닿습니다만, 현재 자알 팔리는 교재들 중에서 '리딩'을 도와주는 교재는 별로 없어요.
    틀린 문제에 한해선 교과서와 개념서 정독을 해보시고 어떻게 리딩을 할까 궁리해보시길 바랍니다.
    B형 21번과 30번은 리딩만 잘 하면 그냥 풀었으니까요.
    \vspace{5mm}
    
    \item \textbf{EBS 무시하지 마세요. 아마 저보고 EBS를 맹목적으로 추종하느냐 어쩌냐하겠지만}
    제 경우도 매일 수학문제집을 풀고 검토하고 연구하고 있습니다. 그나마 평균적으로 나은 게 EBS라서 그렇습니다.
    이번 9평이 본 시험과 일치한다고 보기는 어렵더라도, 출제 경향은 현재 EBS가 그나마 잘 반영하고 있다는 데 유의하시길요.
    \vspace{5mm}
    
    \item \textbf{점수가 낮게 나왔다고 좌절하지 마시길요.}
    다시 풀어서 맞는 문제라면 그건 가망이 있습니다. 다만 컨디션이나 멘탈, 무엇보다도 실전연습이 덜 되어있어서 그래요.
    꼭 실모 풀 필요 없이 시간 재서 수학문제 푸는 훈련을 계속 하십시오.
    \vspace{5mm}
    
\end{enumerate}
    
    


\section{수재의 조건}
\href{https://www.kockoc.com/Apoc/287758}{2015.09.03}

\vspace{5mm}
\begin{enumerate}
    \item \textbf{매일 예습복습을 꾸준히 한다.}
    \item \textbf{거창한 계획을 세우지 않는다. 실천할 수 있는 단 한가지 약속만 지킨다.}
    \item \textbf{자기 주장이 약하다, 그러나 실천은 강하다.}
    \item \textbf{참고서 10권을 한번 보기보단, 1권을 10번씩 본다.}
    \item \textbf{결과가 나올 때까지 최소 6개월, 길면 1년 반까지 기다릴 수 있다.}
    \item \textbf{자기가 상위권에 속했다는 자부심 하나로 스트레스를 푼다.}
    \item \textbf{ 남들이 10문제를 풀면 50문제를 풀 준비를 한다, 결국 남들보다 많이 하려 한다}
    \item \textbf{ 필기, 메모광인 경우가 많다.}
    \item \textbf{ 자기의 문제점부터 인식하고 타인에게 그런 지적을 가감히 듣는다.}
\end{enumerate}
\vspace{5mm}

재수삼수해도 안 되면 저 체크리스트 중 과반은 날라간 거라고 보면 되지 않을까.
어른들이 사위나 며느리를 맞아들일 때에 상대 집안을 본다는 꼰대스러운 건 시대착오적으로 보이기도 하지만
사실 일리가 없는 건 아닌 게, '환경'이 실제로 그 사람의 진짜 '성격'을 암시하는 경우가 유감스럽게도 대부분이기 때문이다.
이걸 가지고 유전이라고 하는 경우도 있지만 그건 유전을 잘못 이해한 결과가 아닌가 싶고,
오히려 더 중요한 건 "성격"이라는 게 이 글을 쓰는 자가 현재까지 종합해 본 결론임.
머리가 나쁘다 하더라도 수험에 맞는 성격이면 어찌되든 잘 되지만, 반면 머리가 좋은데 수험에 안 맞는다면 소용없다는 생각.
\vspace{5mm}

이게 현재 중요한 이유는
과거에는 가부장 체제에다가 뭔가 군사독재 분위기였는데, 역설적으로 이게 학생들을 '수험생'으로 최적화시켜주었음.
의심하지 말고 시키는대로 받아적고 암기하고 시험쳐라, 못 하면 몇 대 맞고 숙제하고 다시 보라.
수험의 정답은 저것임. "단순하고 무식하게 공부"하는 것.
\vspace{5mm}

그런데 지금은 그런 구속적인 분위기가 사라짐.
적어도 과거에 비해선 정말 자유로워졌음, 그런데 이게 수험에는 좋지 않음.
그런데 문제는 '상류' 집안이거나 상위 '중산층' 집안은 그 부모들이 자녀들을 보수적으로 키움.
이게 마치 해저심층수처럼 장기간 그 학생의 수험에 영향을 주는 것임.
인내에 익숙하고 하라는대로 일단 할 줄 암.
반면 자유롭게 키워진 수험생들은 유감스럽지만 인내심이 정말 약하고,
자기는 안 그런다고 하지만 '상술'에 정말 잘 낚임.
\vspace{5mm}

똑같은 대한민국이니 뭐니하지만 아파트 단지별로도 소위 수준이라는 게 차이가 '극명히 나는' 게 슬픈 현실임.
대한민국 아줌마 극혐 치맛바람 왜 휘둘러 네이버에서 개념 댓글 달리지만 사실은
대한민국 아줌마의 최상위 종족인 복부인은 세계 최강이며, 또한 교육열 역시 세계에서 날리는 수준임.
이게 근거가 절대 없는 게 아님,
\vspace{5mm}

논지와 상관없는 썰 보충하면 복부인들은 정말 돈냄새 기막히게 잘 맡고, 기획재정부도 모르는 돈의 흐름과 향방도 직감적으로 예측함.
그래서 정부가 시키든말든 자기들이 느낀대로 투자함. 실제로 그들의 수익률이야말로 어떤 금융회사보다도 나을 것임.
\vspace{5mm}

이건 교육열도 마찬가지임. 자녀를 키우는 본능이라는 게 있어서 그런지 모르지만
남자들은 아파트 단지나 환경이 뭔 소용이냐 하지만 여자들은 그런 데 민감해서 그런지
돈만 있으면 조금이라도 더 좋은 환경으로 옮겨가서 자녀들도 업그레이드시키려고 하고 실제로 그런 성과를 거두고 있음.
물론 황새 따라하는 뱁새도 없는 건 아니겠지만 평균적으로 본다면 절대 무시할 수 있는 수준은 아님.
\vspace{5mm}

아마 본인들이 공부를 잘 한다라고 착각하겠지만
어떤 친일시인의 말대로 7할은 가정환경이 빚어낸 것임.
\vspace{5mm}

그럼 n수생으로 가봐서 관찰하면 실패하는 사례 예를 들면 그럼
\vspace{5mm}

첫쨰, 자기가 하고싶은 공부만 하려한다
둘째, 자기가 점수가 잘 나오는 과목을 공부하려는 경향이 있다.
\vspace{5mm}

독학이 위험한 이유이기도 함. 자기가 하고싶은대로 공부하기 때문에 실제로 점수를 높여주는 공부를 안 한다는 역설이 발생함.
국어가 약하면 국어, 탐구가 약하면 탐구에 바로 투입해서 남들의 10배는 해야하는데 그렇지 않고
그럼 수학은 어쩌죠? 영어는 어쪄죠? 하면서 또 우왕좌왕하다가 귀중한 시간을 날려먹음.
\vspace{5mm}

셋째, 자기의 스타일을 고집하기 때문에 개선을 하지 못 한다.
넷째, 문제풀이 자체가 결국 '자신의 성격'이 칠할을 좌우하는 것을 모른다.
\vspace{5mm}

가장 중요한 건 자신이 문제푸는 스타일을 객관적으로 관찰하고 교정받아야하는데 이게 쉬운 일은 아닐 거임.
그런데 공부를 해도 점수가 올라가지 않는다면 그건 '사고'와 '실행'의 문제고, 이걸 좌우하는 건 자신의 성격임.
예컨대 성급한 사람이 실수를 안 할 리가 없고, 처음 보는 문제에 겁부터 먹고 바로 공황장애 빠지는 경우면 어쩌겠나.
\vspace{5mm}

이것만 봐도 처음에 뭘 해야할지는 답이 간단한데...
문제는 이걸 고치라고 조언을 해도, 사실 고치는 경우는 그리 많지 않다는 것임.
"나는 너무 소중하니까".
그냥 공부만 많이 하면 다 해결될 거라고 착각.
\vspace{5mm}

물론 수재의 조건을 일률화시킬 수 없음. 그러나 최소한 지켜야 할 것들이 존재함.
특히 자신의 성격이 저런 것과 위배된다면, 당장은 호전될 수 없을지라도 바로 고치거나 보완할 필요가 있음.
\vspace{5mm}

어제 수학 B형의 경우만 하더라도 조금만 생각해도 풀릴 것을 어려워보인다고 손도 못 댄 케이스가 생각보다 많았는데
이게 지식의 부족 탓이기만 할까.
\vspace{5mm}








\section{수학 B$\rightarrow$A형 돌릴 때 참조하실 것.}
\href{https://www.kockoc.com/Apoc/296976}{2015.09.06}

\vspace{5mm}

B형 수학은 온갖 무기가 동원되는 서바이벌 게임입니다.
그래서 교과외적인 무기도 $-$ 쓸모가 있다면 써도 되는 겁니다.
총 대신 탱크를 몰고가도 되고 무인기를 써도 좋습니다. 안 되면 주먹으로 가격해도 되서 꼼수나 스킬이 일정 정도 먹힙니다.
\vspace{5mm}

다만 요즘 와서는 화력보다는 '스나이핑'을 더 강조하는 분위기입니다.
꼼수나 스킬의 효용도 떨어지고 있어요. 문제가 쉽고 적고 떠나서 정말 교과서 개념 충실한 친구에게 유리해졌습니다.
\vspace{5mm}

반면 A형 수학은 맨주먹 빼고는 아무 것도 쓸 수 없는 격투기입니다.
격투기이면서 형식을 정말 잘 지켜야하기 때문에 멱살 잡는다거나 할퀸다거나 그런 거 안 먹힙니다.
흔한 \textbf{지수로그 지표 가수, 격자점, 그리고 문과 미적분에는} 꼼수는 더더욱 안 먹힙니다
A형 쉽지쉽지 그러는데 요새 킬러 수준으로 치면 A형 30번이 B형보다 더 어렵다고 느껴지는 경우가 많습니다.
B형 잘 푸는 친구들이 A형 격자점에서 헤헥대는 경우는 널렸습니다. 특히 성격 급하고 계산 실수 잘 하고 문제리딩 못 하면 답이 없죠.
\vspace{5mm}

본인이 자잘한 잡스킬 모르겠다, 기벡은 정말 공간감각 없어서 안 되겠다.
하지만 논리력은 명쾌하게 식을 통한 접근 정말 잘 하며 정리갑이다라고 하면 A형 가신 다음에
시중 교재에서 지수로그와 격자점만 확실히 정복하시면 100점은 어렵지 않을 것이지만
그게 아니라 잔실수가 많고 멘탈 개판이고 그렇다라고 하면 먼저 그 습관부터 고치시길 바랍니다.
\vspace{5mm}

최소한 제가 검증해본 바로는 수학은 지식보다는 이 역시 '성격'의 문제입니다.
중학교 때 숫자와 기하 감각을 졸라게 익하고, 고교 진학 전에 그리스 소피스트처럼 묻고 따지고 생각하는 싸가지없는 습관 들이면서
모든 것의 답을 궁구하려는 "왜?"라는 질문을 던진다면 고교수학을 잘 하는 것이고
저 중 하나라도 안 되면 운이 좋으면 모를까 그렇지 않으면 매우 힘들어지는 것이지요.
\vspace{5mm}






\section{이 시점에 감성파든 열심히 한다 하지 마셈.}
\href{https://www.kockoc.com/Apoc/418857}{2015.10.16}

\vspace{5mm}

\textbf{"아, 열심히 해서 n수 하지 않을 거야"}
냉정히 말하겠음. 부질없는 이야기입니다.
\vspace{5mm}

\textbf{"노력과 운과 상관없어"}
이거 수험이든 뭐든 겪어보지 않아서 하는 이야기입니다요.
\vspace{5mm}

저런 식의 메시지는 마지막 소중한 시간조차도 부질없는 감성에 젖게 만들며
본인의 실책을 극복 못 하고 결국 '정신승리'에 빠지게 하는 길입니다.
\vspace{5mm}

공부 초에는 만점받을 수 있어 얼마든지 공부할 수 있어라고 자신을 과대평가하다가
공부 말에는 이제 나 어떡하지라고 발 동동구를 거라고 작년 말에 \textbf{경고}했습니다.
\vspace{5mm}

그러니까 뽕맞지 말고 일찍 공부하라 나중에 시간없어서 힘들 것이다
그냥 생각없이 양치기 하고 빨리 오답정리하는 게 낫다... 라고 얘기했습니다.
\vspace{5mm}

시험은 아직 안 끝났으므로 단언은 안 합니다. 그러나 저 예측은 거의 다 들어맞고 있습니다요.
다만 '운'이라는 게 있기 때문에 아직까지 확언할 수 없지만
쪽지 보내시는 분 중에서 충고대로 하신 분들은 그래도 나름 모의에서 고득점 나오면서도 뭘 풀까라는 고민을 하시는 반면,
뒤늦게 공부시작해서 우왕좌왕하시는 분들은 자기 인생을 넘어 남들까지 탓하고 있습니다.
\vspace{5mm}

대한민국은 소말리아도 시리아도 아니지요. 님들이 공부를 못 한 게 아니라 \textbf{'안' 한 것입니다.}
공부를 안 해서 성적이 안 나온 것이라면 그건 정의로운 결과이니 거기에 슬퍼할 필요가 없습니다.
올해 목표를 못 이루었다면 다시 시작하거나 다른 길로 가면 되지, 별 이상한 감성에 빠질 이유가 없죠.
하지만 운이 나쁜 게 분명하다면 그건 올해 운이 안 좋은 것인데 불운이 내년에 다시 반복될 가능성은 높지 않으니
운이 나쁜 건 어쩔 수 없다, 다만 그 나쁜 운을 밀어낼 만큼의 노력을 하지 않았다라고 인정하고 다시 공부하면 되는 것입니다.
\vspace{5mm}

자기가 시험을 잘 치를지 못 치를지는 '무의식'이 정말 잘 알고 있죠.
그 무의식이 '야, 넌 올해 힘들 거야'라고 하면 그걸 인정하기 싫어서 다들 현 시점에 감성주의에 빠집니다.
마치 술을 마셔서 현실을 잃어버리듯 그렇게 '현실부정'을 하면서 난 잘 될 거야...
\vspace{5mm}

\textbf{'잘 될 거야'가 실제로 현실극복에 도움울 준 사례는 단 한번도 없습니다.}
\vspace{5mm}

오히려 '이래도 힘들어, 난 안 될 거야'라는 비관주의가 잘못된 자아를 바꿔 도움준 사례는 있을지 모르지만요.
마음을 비우고 수험의 본질을 돌아가십쇼.
\vspace{5mm}

\textbf{국어 $-$ 45, 수학 $-$ 30, 영어 $-$ 45, 탐구 $-$ 40을 8시간에 걸쳐 풀어내는 작업입니다.}
본인들의 인생을 좌우하는 건 어렵다는 킬러문제, 그리고 본인이 헷갈리거나 공부하지 않은 문제들입니다.
기본적인 문제를 다 풀어내고 저런 문제를 풀어내기 위해 공부하는 겁니다. 남은 시간이 촉박하면 가장 중요한 것에 할애하시는 것이지요.
물론 올해 공부를 열심히 했다고 자신이 생각하더라도, 전략이 틀렸으면 저 작업에서 실패할지도 모릅니다.
결과는 노력한대로 나옵니다. 다만 그 노력은 '타인'이 봐도 대단하다라고 느껴져야하는 거지, 본인'만' 인정하는 건 소용이 없지요.
\vspace{5mm}

\textbf{노력을 정말 한 사람이 노력을 까는 경우는 없습니다.}
\textbf{노력을 하지도 않은 인간들이 노력도 필요없다고 하죠.}
\textbf{노력 까는 사람들은 과거에도 현재에도 미래에도 노력과는 인연이 없죠.}
\vspace{5mm}

그럼 20일이 남았는데 어쩌냐 하는데
이 20일은 과거 100일의 효과를 누릴 수 있는 시기입니다.
일단 기온이 쌀쌀해서 공부하기 좋고 시험이 코 앞이라 긴장되며 그동안 공부한 게 쌓여서 가속효과를 발휘할 수 있습니다.
다시 말해서 본인들이 가속효과를 누릴 수 없으면 올해 시험은 그냥 마음편히 보는 게 낫다는 이야기일 겁니다.
\vspace{5mm}

아울러 이제 정말 웬만한 사이트 출입하면서 감정적으로 흔들리는 건 피하십시오.
하라는 공부는 안 하고 들락나락거리면서 잡글의 감성주의에 빠져서 올해 정말 낭비 안 했다라고 양심에 얘기할 수 있습니까?
전 격려글조차도 쓰지 않을 겁니다. 내용도 뻔하지만 이건 아무런 도움도 안 되기 때문입니다.
적어도 타인이 보기에 공부를 열심히 했다라고  평가받는 좋은 결과가 나와야하는 것이고
본인은 했다고 하지만 타인이 보기에는 저 놈은 왜 이렇게 쓸데없는 짓을 많이 하냐 하는 사람은 망해야 '정의로운' 것이 아닌가요?
시험공부도 대충 해놓고 나중에 내 인생 흐흐흑하면서 시험 잘 보길 바라는 사람은 도둑놈입니다.
그런 사람이 좋은 대학에 가면 우리나라는 그만큼 힘들어질 거라고 하는 건 과장은 아니라고 생각합니다만?
\vspace{5mm}

마음 비우시고 20일동안 후회 없는 공부하시길 바랍니다.
그리고 시험 당일날은 잘 보든 못 보든 '냉정'하세요. 절대 냉정하시길 바랍니다.
킬러 어려운 것 나오면 어쩌지 10평 잘 보았니 하는 것 다 개소리고
어차피 대부분은 시험 당일날 냉정을 못 지켜서 '틀리지 말아야 할' 문제도 틀리고
조금만 찬찬히 읽어봐도 맞는 문제 다 틀리지요.
\vspace{5mm}

그래도 위안을 원한다면 그냥 시험상황을 얘기해드리죠.
올해 시험은 응시자 풀이 최악인 상황입니다. 그리고 7차 교육과정 마지막이지요.
9평과 10모 컷만 보아도 문제가 아주 쉽다고 할 수 없는데도 그런 컷들이 나왔습니다. 현역조차도 평균적으로는 잘 하는 상황입니다.
물론 올해 시험 같은 경우는 기존 시험의 온갖 노하우와 시행착오까지 다 파악할 수 있으므로 특히 불리하다 할 수 없겠지만
본인이 정말 잘 치렀다고 하면 그보다 더 잘 치른 괴수들을 목격하게 될 것입니다.
\vspace{5mm}

따라서 올해 열심히 했더라도 원하는 결과가 안 나올 가능성은 높습니다.
노력을 안 하면서 요행이나 바라는 사람에게는 올해 같이 변명하기 좋은 해는 없겠지만요.
아울러서 이제는 취업이 되는 게 이상한 시대가 와서 대학졸업해놓고 또 수능치는 사람들도 늘어나고 있습니다.
즉 n에서 n+1이 된다고 해도 과거에 비하면 덜 부끄러운 시대이다라는 데 위안을 가져도 좋겠지요.
\vspace{5mm}

아무튼 냉정히 공부하시길요. 그리고 제가 보아도 '아 이 친구는 열심히 했어'라는 케이스는 대박날 거라고 봅니다.
그리고 목표를 달성하지 못 하는 사람들이 더 많겠죠.
중요한 건 이 대목인데, 바로 그 결과를 인정하고 다시 도전하면서 '공부 그만 하세요'라는 소리 들을 정도로 하는 사람은 내년에 대박나겠고,
귀중한 시간에 감성에 젖어서 하라는 공부 안 하고 그걸 핑계로 놀아제끼는 사람은 평생 힘들겠죠.
\vspace{5mm}










\section{콕콕에서 노력한다고 보이는 수험생}
\href{https://www.kockoc.com/Apoc/420994}{2015.10.17}

\vspace{5mm}

14명 정도입니다.
이 경우는 세가지로 확인하는데
\vspace{5mm}
\begin{enumerate}
    \item 일지 $-$ 일지 공부량을 보면 제가 생각한 것보다 1.5배 정도 공부한 케이스
    \item 학습란에 올라오는 문풀 글 $-$ 일지가 없어 모르지만 문제 보는 안목이나 풀이에서 내공이 확인되는 케이스
    \item 쪽지 $-$ 쪽지로 주고받는 경우에 확인되죠.
\end{enumerate}
\vspace{5mm}

그런데 노력했다하는 건 3개월 이상 정말 다른 사람들과 차이가 날 정도로 공부한 경우만 말함.
적어도 제 기준으로 보아도 그 정도 역시 '서울대'급을 노린다면 평범한 정도입니다.
실상은 1$\sim$2개월 정도 하고 힘들어죽겠다하면서 가을의 수필가로 갑니다. 그래서 여간 안타까운 경우가 아니죠.
\vspace{5mm}

특히 그나마 공부한다고 하던 게 8월 이후면 이게 답이 없습니다. 이제 그나마 공부가 되려고 하는데 수능 코앞이면 좌절해서
공부를 싫어해버릴 수 있는 거죠. 1$\sim$3월달에 저랬으면 순상승기 탔을 건데 말입니다.
보통 슬럼프 주기는 2주, 1달, 3달, 6달에는 오게 되어있고, 이 시기를 넘기면 실력은 비약합니다.
그래서 무조건 일찍 공부를 시작하는 게 답입니다.
\vspace{5mm}

가끔 오는 한심한 질문이 6시간만 해도 되어요... 라는 건데 먼저 6시간 일주일을 해보고 질문하지도 않은 케이스여서입니다.
그런데 12월부터 6시간 꾸준히 하더라도 지금 모평이면 올 1등급은 맞고 있을 가능성은 매우 높습니다(특히 이번 교육청 이과라면요)
\vspace{5mm}

이건 논쟁이 붙긴 하지만 1년간 일지$-$상원 루트 나름 보면서 소프트하게 조언해주는 결과를 보면
이게 수능합불까지 어떻게 될지는 모르나, 결국 3개월 이상 꾸준히 해본 사람들은 모평에서 좋은 성적도 나오지만
수능 걱정을 하더라도 정말 건설적으로 합니다.
그 이야기는 결국 다른 여건 탓할 필요 없습니다. 100일 정도 웅녀 인간되기 프로젝트 따라하는 게 정답이란 것이죠.
\vspace{5mm}

다만 문제는 그거. 올해 수능은 2년 3년 바짝한 괴수들도 응시한다는 것. 원래 응시 안 하냐하겠는데 경쟁 과열이 높아진 듯.
\vspace{5mm}

그리고 앞으로 콕콕이 여학생 위주로 갈 거라고 보는 합리적인 이유는
\textbf{여학생들이 더 공부를 열심히 합니다}. 남학생들은 그에 비하면 너무 많이 흔들리네요.
\vspace{5mm}

커리가 정해지고 스트레스와 감정문제만 해소되면 정말 하루 6시간은 우습게 하는 게 있습니다.
필요한 건 '고급정보', 그리고 일종의 감정적 문제를 해결해줄 수 있는 멘토 아니면 게시판의 존재만 있으면 되지요.
\vspace{5mm}

노오력을 이야기할 때는 여학생들은 언급될 이유가 거의 없을 겁니다. 연애만 아니면 거의 다 공부를 시작하면 열심히 하니까요.
\vspace{5mm}

문제는 남학생들인데
\vspace{5mm}

이건 뭐. 칼럼을 써도 그 내용 너 마음에 안 들어 어쩌구 '공격'하는 한심한 종자들도 대부분 남학생들이지만
오래 관찰해보면 왜 선진국도 여풍이고 우리나라도 여풍이 강한지 그걸 알 수 있는 것 같습니다.
실속은 생각하지 않고 '자존심'에다가 '감정' 중심으로 행동하는 게 남자들입니다. 이 글을 쓰는 저도 예외가 아니려나
\vspace{5mm}














\section{10일 남았는데 입시에만 신경쓰시길 바랍니다.}
\href{https://www.kockoc.com/Apoc/456618}{2015.10.31}

\vspace{5mm}

나이 처먹어서 늙어서 그런지 모르지만
요즘 입시생들은 수험을 하기보단 수험코스프레질을 한다는 느낌이 강합니다.
\vspace{5mm}

시험 한달 앞두면 해야될 건 본인 틀린 것 철저히 점검하고, 기본서 다시 회독수 높이고,
아울리 시험 시각 맞춰서 시험 치는 연습하고 그 뿐입니다. 사실 이것만 제대로 하는 것도 어렵습니다.
그런데 상당수가 계획만 짜면서 수험업자들이 기획한 상술대로 놀아납니다.
제가 관여할 바는 아닙니다. 어차피 이건 수능치고 나서 결판날 문제라서요.
\vspace{5mm}

아니라고 하는 사람들도 있지만 \textbf{결국 공부한만큼 나옵니다}.
그런데 여기서 공부한다는 건 '집중'하는 걸 말합니다. 집중한다는 건 세가지 의미입니다.
첫째, 공부 외 다른 건 차단한다
둘째, 자기에게 필요한 것을 $-$ 본인이 공부하기 싫더라도 $-$ 공부한다.
셋째, 나를 잊어버린 상태 $-$ 정신들고보니 10시간 이상이 흘러가 있다.
\vspace{5mm}

이게 되는 학생과 안 되는 학생이 갈립니다. 그런데 문제는 안 되는 학생은 그 되는 상태가 뭔지 끝까지 모릅니다.
그게 되는 사람들이 쓰는 교재는 \textbf{매우 간결하고 핵심을 잘 짚죠}, 그런데 그게 안 되는 사람들이 쓰는 교재는 정말이지 장황합니다.
현실은 후자의 교재가 많이 팔리고 있고, 그만큼 많은 학생들이 입시에서 물을 먹죠.
그래서 공부해도 안 된다고 착각을 합니다. 자기가 집중을 정말 하고 공부하지도 않았을텐데 말이지요.
\vspace{5mm}

10일 남았으면 학습란에 제가 제시한 글이나 즐미님이 쓰신 글대로 가주세요. 그것대로만 하더라도 대단한 것입니다.
10일동안은 가능하면 다른 데 휩쓸리지 말고, 교재도 더 이상 따로 추가하지 말고, 갖고있는 것만 제대로 다 '실제 수능'에 맞게 푸시고
틀린 문제는 3$\sim$5회 이상 다시 풀고 점검하시길 바랍니다.
\vspace{5mm}

이런 것도 안 하고 자기가 공부했다고 착각하면서 부모탓, 환경탓만 하는 스랙이 되지 않기만을 바랍니다.
\vspace{5mm}

가능하면 콕콕 접속, 아니 웹접속을 아예 안 하는 게 좋을 수도 있습니다.
이 글을 보고 난 다음 수능 보기 까지 한번이라도 접속하면 떨어진다... 라고 주문을 거는 게 더 나을 수도 있습니다.
그렇지 않으면 10일이라는 황금같은 시간을 날려먹고 수험사이트나 오가는 스랙질이나 하고 있는 자기 자신을 보게 되겠죠.
\vspace{5mm}

작년말부터 학습일지부터 시작해 올해 상담까지 다 추려보았습니다.
콕콕에서 작년부터 이유없이 공부법 가지고 시비걸면서 스랙교래 추천하는 양반들이 있었는데요
그래도 혹시나 제가 틀렸나 다시 점검하고 실증적으로 갔지만 결론은 결국 '양치기' 최고이고 '집중하는 게' 답입니다.
이 글 읽는 몇몇은 작년에 제가 충고했음에도 그 이행 안 지키고 지금도 스랙질하거나
뒤늦게야 실천하고 아 진작 할 걸 하는 사람도 있을 것입니다(그래도 후자는 낫죠?)
상담글에서도 확인되고 제가 관여 안 했지만 최근 탈콕하시고 공부하러 간 분의 글에도 나오죠.
그냥 님들이 하던데로 양치기하고 집중하면 됩니다.
10일동안은 님들이 틀린 문제 약점 점검하고 극복하는 건 충분한 시간입니다.
물론 안 할 사람은 끝까지 안 하면서 머리탓, 세상탓, 부모탓, 우주인탓하겠죠. 어차피 이런 사람들은 대학을 안 가는 게 세상을 위해 좋습니다.
\vspace{5mm}

올해도 자기가 못 나올 거라고 하지만 의외로 대박치는 사람들은 셋 정도는 있을 겁니다.
신비주의는 아니고 일지 보거나 상담할 때 아 이 친구는 내가 본받고 싶을 정도로 공부했구나 하는 케이스가 있죠.
그런 경우는 시험에서 예상외로 잘 나옵니다.
\vspace{5mm}

하지만 평소에 모평이나 실모뽕맞거나 하면서 산만하게 공부하고 집중 안 하는 경우는 '역시나'가 될 수 있으니
10일동안 똑같은 물을 뱀이 마시게 해서 독이 될 것인지, 소가 마셔서 맛있는 우유로 만들 건지 본인이 잘 극복하시기들 바랍니다.
시험에서 만약 감점나온다고 하면 그건 시험 난이도가 아니라, 본인의 공부가 어디까지 완성되어 있느냐 그 차이일 뿐입니다.
\vspace{5mm}

시험날도 하루는 24시간이고 태양은 동쪽에서 떠서 서쪽에서 집니다. 잘 마무리하고 보시길 바랍니다.
\vspace{5mm}






\section{자기 머리를 믿으세요.}
\href{https://www.kockoc.com/Apoc/458132}{2015.10.31}

\vspace{5mm}

두가지 케이스입니다.
\vspace{5mm}

양치기를 한 사람은 시험날 신기하게도 문제를 보면 문제가 쑥쑥 풀립니다.
즉, 수험뇌가 완성되어서 자기는 아무 생각이 없는데 문제가 해석되고 풀리는 것인데요
이 경우는 본인이 할 건 자기 풀이를 의심하고 계속 검토하는 것입니다요.
\vspace{5mm}

공부를 열심히 해서 잘 돌아가는데 너무 빠른 나머지 브레이크가 안 걸려 문제를 잘못 읽는 불상사가  생깁니다.
이런 건 제가 경고드린 분이 몇몇 계십니다. 반드시 문제를 종이로 가려서 한줄씩 읽고, 조건 하나당 번호 붙이길.
국어를 풀 때는 특정선지가 답인 이유, 답이 아닌 이유 명기하고
수학은 반드시 조건들을 찾아내 번호 다 붙여서 그것들 다 썼는지 확인하시는 것.
\vspace{5mm}

양치기가 안 된 분들은 다소 의식적이지만 작위적인 생각을 하셔야할 것입니다. 사실 이게 맞긴 한데
문제는 이게 엄청난 피로도를 선사하며 속도면에서 불리할 수 있다는 것입니다.
양치기가 안 되기 때문에 문제 봐도 하나하나 자기가 써나가면서 풀어야하는 분들은 브레이크가 아닌 엑셀모드로 가시길 바랍니다.
즉, 이 경우는 스피드업을 하면서 문제를 보고 '연상하고 이미지'를 떠올리는 쪽으로 가야만 제 시간에 맞출 수 있습니다.
\vspace{5mm}

요약하면
양치기가 선행된 분은 자기 뇌의 엔진을 믿되, 브레이크를 잘 걸어주시면서 실수를 방지하시라는 것 $-$ 브레이크 모드
양치기가 안 되어있지만 의식적인 생각으로 풀겠다하는 분들은 연상을 많이 해서 문제 실마리를 찾는 엑셀 모드로 가시길 바랍니다.
\vspace{5mm}

브레이크 모드는 본인이 문제푸는 실마리가 너무 많이 떠오르기 때문에 본인이 논리적으로 그걸 검토해서 OX질을 잘 해야합니다.
반면 액셀 모드는 떠오르는 실마리는 터무니없는 것들도 다 적어야 합니다. 그게 없으면 문제를 아예 못 풀 수도 있습니다.
\vspace{5mm}

시험 망할까 어쩔까 그런 반응은 당연한데, 자기 뇌를 믿으시길 바랍니다
님들이 직접 뛰는 게 아닙니다. 님들의 경주마는 두개골 안의 그것입니다.
경주마를 잘 먹이고 다독이고 하는 쪽으로 신경쓰는 편이 나을 것입니다. 달리는 건 기수가 아니라 말이지요.
\vspace{5mm}






\section{학벌의 장점.}
\href{https://www.kockoc.com/Apoc/460316}{2015.11.02}

\vspace{5mm}

고등학교 졸업 이전에 부딪치는 악은 대부분 \textbf{폭력}임.
고등학교 졸업 이후에 부딪치는 악은 대부분 \textbf{사기}임.
\vspace{5mm}

폭력은 어떤 것이든 법망에 걸릴 수 있습니다. CCTV가 도처에 깔려있고 개인도 스마트폰으로 증거자료 삼을 수 있으니까.
그러나 사기죄의 경우는 잡아내기가 어렵죠. 상대가 속이면 $-$ 전문용어로 기망행위를 하면, 본인이 거기에 낚여서 행동함으로써 범행이 완성되니까.
\vspace{5mm}

사기의 원동력은 인간의 \textbf{탐욕과 무지}입니다.
그런데 탐욕이야 욕심을 줄이면 된다고 보면 되긴 하는데(라지만 말처럼 쉽지는 않음)
무지는 이건 정말 답이 없음. 우리가 아는 것보단 모르는 것이 훨씬 많기 때문에 $-$
사기에 걸리지 않으려면 정치, 경제, 법률, 자연과학, 기술 전반에 빠삭해야할 뿐만 아니라
상대방의 보조개만 보고도 심리를 파악할 수는 있어야한다고 보는데 이게 쉬운 일은 아닐 뿐더러
사기치는 사람들은 정말 똑똑한 데다가 냉혈한입니다.
\vspace{5mm}

그럼 사기꾼을 잡아내면 되지 않느냐.... 라고 하는 사람조차도 자기가 은근히 속고 있다는 걸 모르고 있고
이 글을 적는 저 자신도 제가 모르는 사기에 이중삼중으로 걸려있다라는 게 정확한 진술일 거예요.
공부하고 나서야 아 내가 속았구나를 뒤늦게 알았을 때는 이미 다 털린 뒤지요.
\vspace{5mm}

사기에 걸리지 않으려면 상대가 뭔가 제시할 때 "아니오"라고 거부해야하고
특히 다수결이 좌우하게 되는 소통에서는 백분토론할 때처처럼 설득력있는 자기 주장을 발휘해야하는데
문제는 여기서 먹히는 건 국어영역에서 말하는 논리가 아니란 겁니다. 첫째로는 감성, 둘째로는 외모, 셋째는 권위란 겁니다.
\vspace{5mm}

그런데 여기서 학벌이 유효하게 먹힐 수는 있음.
사실 사람들은 그 주장의 내용이 뭔지 이해하지도 못 하고 이해하려고 하지도 않음. 그냥 좋다 유리하다하는 걸 결정하는데
터무니없는 주장을 하는 사람일지라도 xx대 졸업이라고 하면 \textbf{보이지 않는 무언가 있겠군...} 이라는 게 사람입니다.
메시지 이전에 메신저의 스펙부터 보는 게 인지상정이기 때문에, 이 점에서는 학벌이 유효하게 작용할 수 있을 뿐더러
사기꾼일지라도 상대가 일단 xx대 출신이라고 하면 섣불리 먹으려고 하지 않습니다(물론 같은 xx대 출신이라거나 그 이상이라면 다르겠지만)
물론 xx대 출신이라고 밝혀졌지만 하는 짓이 영락없는 호구라면 얄짤없지만.
\vspace{5mm}

공부에 있어서는 어떤 주장이 오갈 때 \textbf{"나는 그럼 xx대인데 당신은 어디 다니십니까"}라는 걸로 데우스 엑스 마키나로 가는 경우가 가능.
매우 재수없는 상황이라고 할지도 모르겠지만 우리나라 사람들은 다들 학벌주의 저주에 걸려있다라는 건 똑같음.
당연히 이건 금전과는 거리가 멉니다요. 가능하다면 저런 식의 쓸데없는 싸움은 안 하는 게 현명하지만요.
\vspace{5mm}

그런데 개인적으로는 온오프에서는 그 덕분에 사기꾼들을 잡아내고 헛소리한 걸 잡아내는 건 가능했다고는 생각함.
그 이야기는 거꾸로 말해서 이런 케이스가 아닌 경우는 지금도 사기꾼들에게 놀아나는 케이스가 널려있을 거란 이야기죠.
\vspace{5mm}

아, 물론 수능 이후에 만화 검은사기는 보시길 바랍니다. 패턴이 늘 반복되긴 하는데 한번 볼 가치는 있음.
재밌는 건 이 만화에 실린 사기수법이 수년 뒤에 우리나라에서 반복되었다는 것이죠. 즉, 사기수법이 만화책을 통해 먼저 들어옴(...)
\vspace{5mm}






\section{수능출제는 통수}
\href{https://www.kockoc.com/Apoc/461252}{2015.11.02}

\vspace{5mm}

난이도가 높다 낮다... 그런 건 무의미.
\textbf{출제가 예상가능한 영역인가, 예상불가능한 영역인가. 이런 게 중요함.}
올해 수능이 불수능이냐 물수능이냐하는 건 중요치 않습니다.
\textbf{출제 경향이 2014년도와 2015년도 수능과 어떤 점에서 똑같고, 어떤 점에서 다르냐. 이게 가장 중요하죠.}
\vspace{5mm}

과거 복기해봅시다. 이과수학만 보자면 2014년도 역시 '통수'였고 2015년도 어떤 의미에서는 통수였습니다.
둘 다 당시 수험생들이 준비하던 방향과 다른 각도에서 출제했음.
전설의 2012년도 스타일에 맞게만 공부한 2014년도 수험생들은 29번에서 제대로 통수를 먹었고
그래서 기하와 벡터에 치중했던 수험생들은 2015에서 또 한번 통수를 먹었음(난이도가 하향되었다고 하지만 어려운 문제는 어려웠죠)
\vspace{5mm}

그리고 지금 수험생들은 \textbf{2014와 2015 반반무많이로 공부하고 있는 현실}임.
\vspace{5mm}

평균적 방향
\vspace{5mm}

\begin{itemize}
    \item 국어 $-$ 어렵게 나올 것이다, 화작문 대비 철저히 해야징$\sim$
    \item 수학 $-$ 2014와 2015의 중간정도 보자. 으음, 이번에는 확통이려나?
    \item 영어 $-$ 쉽게 나오겠징. 그래도 불안하니까 듣기가 불안
    \item 탐구 $-$ 극헬이겠지.
\end{itemize}
\vspace{5mm}

그런데 문제는 다들 자기가 대비하는 방향으로 출제되고 있을 거라고 '착각'하는 것인데
과거 5년동안 수험생들의 의도대로 100$\%$ 출제된 경우는 없음. 수험생들이 대비하지 않은 방향으로 엿먹이는 건 지속됨.
당연히 사설학원이든 인강이든 실모든 그런 거 적중시킨 적도 없음.
\vspace{5mm}

이 얘기를 왜 하냐면 지금 시험이 불안한 사람들은 2014, 2015 기준으로 자기가 잘 볼 수 있느냐 없느냐 판단할 건데
실제 2016년도 시험 출제가 어떻게 될지는 아무도 모릅니다.
2014, 2015에 최적화된 사람이 열흘 후 수능에서 죽쑬 가능성도 높고,
2014, 2015에 맞게 공부되어있지 않았는데 조금 공부한 게 2016 스타일과 너무 잘 맞아 대박날 수도 있고.
정말 이건 아무도 모르는 것임.
\vspace{5mm}

예컨대 저래놓고 나서
갑자기 국어에서 문학을 어렵게 내고,
수학은 느닷없이 행렬이나 지수로그 격자점을 헬로 내버리고
영어 빈칸에서 고난이도 2문제, 순서잡기에서 헬난이도 내버리고. 탐구는 정작 쉽게 내면?
\vspace{5mm}

뭐 그대로 맛가는 거죠. 다만 이런 데도 안 흔들리는 사람들은 기본이 충실히 되어있는 케이스겠죠.
그리고 지금도 문제 점수 잘 안 나온다는 사람이 좌절하는 심경으로 치렀는데 대박날 가능성도 있음.
작년 이맘 때도 일지 체크해주면서 확인했는데 시험 전에 절망적인 심정으로 달린 사람이 대박 나기도 하고,
공부 거의 다 완료했단 사람이 문제 실수해서 날라가기도 하고.
엄밀히는 시험 전에 분명히 공부에 흠이 많은데 점수가 잘 나오니까 열심히 공부한 학생이 되고
반면 열심히 했는데도 점수가 안 나오니까 공부 안 한 학생으로 취급받고.
\vspace{5mm}

그러니 제가 드리는 말씀은 본인 뇌를 믿으라고 하는 수 밖에 없는 거임. 시험 칠 때 절감하실 것입니다요.
공부 안 되어있다라는 기준도 의식적으로 떠오르는 지식이 기준일 건데,
실제 수능의 문제풀이력은 무의식적으로 나오지요.
\vspace{5mm}

10일동안 기상시각 잘 조절하고 뇌 관리 잘 하면서 평정심 유지하는 게 나을 겁니다요.
예기치 않은 데에서 킬러문제가 나오면 어찌 풀까하는 걸 이미지 트레이닝하는 게 좋긴 할 것입니다만.
\vspace{5mm}

+ 생각해보니까 만약 탐구를 전부 다 물로 내버리고 영어를 불로 내버리면
그 때는 "영어 공부했어야하는데", "탐구 필요없네용$\sim$ 국영수나 할 걸"... 뭐 이럴 삘이고.
작년만 해도 수학은 저도 불로 나올 것이라고 생각했는데 다 실전미만잡이 아닌가 싶기도 하고.
어차피 정부는 퍼센테이지만 잘 조정하고 복수정답만 안 내면 욕 안 먹죠.
\vspace{5mm}

+ 실전주의적인 입장에서 말하면 탐구를 제외하고 국어영어수학은 각 과목당 2$\sim$3문제씩이 결국 발목잡고
그 문제 하나당 10$\sim$15분씩 소요되는 일이 벌어지죠. 출제자라면 이런 식으로 내지 않을까 싶은데 말입니다.
이 경우 수험생에게 중요한 건 차분히 문제, 지문을 읽고 '논리적'으로 정답을 유추하거나 확률 높은 걸 찍거나 하는 것일텐데.
\vspace{5mm}

+ 지금 수험생들에게 가장 큰 문제는 2015년도 출제 경향의 부채. 즉, \textbf{다 만점을 받지 않으면 안 된다는 강박이 가장 큰 문}제인 듯
하지만 예기치 않게 난이도가 헬이어서 만점의 저주라는 게 올해 깨질짇 모르는 일입니다. 그러니 절대 과거 경험으로만 접근하면 안 됨.
\vspace{5mm}

+ 최악의 시나리오는 국어나 수학에서 예기치 못하게 통수 맞고 그 뒤로 만점강박증에 자포자기해버리는 경우.
혹은 생각보다 쉽게 나온 국수인데 영어가 어렵게 나와 통수맞는 케이스
\vspace{5mm}






\section{교재퀴즈}
\href{https://www.kockoc.com/Apoc/462586}{2015.11.04}

\vspace{5mm}

Q. 가성비가 최악인 동시에 최고,  다수의 수포자 양성, 소수의 수학고수 양성한 책이 뭔지 기술하시오.
\vspace{5mm}

속내가 뻔히 보이고 가입일이 최신인 '선플'을 가장한 '악플'이 있어서 글지웠습니다.
이 사이트나 제가 마음에 안 드는 분들도 계시겠지만 활동을 하려면 \textbf{지능적으로 하셔}야죠.
아무튼 그 분들 덕분에 수능 이후 콕콕 사이트 운영에 대한 좋은 참조례가 마련된 건 감사.
\vspace{5mm}

그리고 전 올해 초부터 특정교재를 대놓고 '실명'으로 까지 않습니다. 그런 걸 가지고 고소미 먹이는 정신나간 놈들이 있기도 하지만
언급할 필요도 없는 교재는 대놓고 말하는 것만으로도 \textbf{타락하는 기분}이어서 말입니다.
특히 교재에 대해서 추궁하면 답 못 하고 도망가는 케이스라면 말입니다. 그런 사람이 없을 것 같죠?
\vspace{5mm}

A, B 교재가 궁금하신 분들 있는 것 같은데 힌트 드립니다.
그 저자들 \textbf{남자가  아닙니다}. 그 범주에서 찾아보시길
\vspace{5mm}

뭐 그건 그렇고 저기 Q에 해당하는 교재가 뭔지 한번 답들 해보시길.
사실 이 교재는 가성비가 대한민국 최악인 동시에 최고이기도 합니다. 제대로 정복하면 수학 최고수가 되죠.
너무 쉬운 문제인가?
\vspace{5mm}

그리고 이건 콕콕 사이트에서 활동하고 있고 하려는 사람들을 위해 본격하는 이야기입니다.
그리고 콕콕과 여기서 교재내는 사람들을 온갖 방법으로 공격하려는 사람들에 대한 경고성 메시지이기도 한데요
\vspace{5mm}

교재비판에 대해서 적죠
\vspace{5mm}

우선 교재명 언급만 하면 경찰서 출두 가능? 그렇습니다.
그럼 그런 거 가지고 먹이는 사람들 있나? 그렇지요.
벌금 싸게 나오는 것까지 고려해서 민사까지 감안해서 일부러 자기가 피해입었다고 썰까지 푸는 경우도 있죠.
그럼 이런 질문 하겠죠. \textbf{"일일히 자기 교재가 어떻게 언급되나 그거 찾는 사람도 있나요?"}
\vspace{5mm}

\textbf{예, 있습니다. 심지어 동료나 부하(?)들을 통해 알아보거나 제보받기도 할 걸요.}
\vspace{5mm}

그 사람들 하는 짓이 비겁하든 안 하든 어찌되었든 법의 논리는 다릅니다.
그리고 그 사람들은 일부러 '제3자'인 척 리플을 달아 질문을 유도해서 \textbf{법에 저촉되는 발언을 유도하기도 합니다.}
그 사람들은 품질이 아니라 명성만으로 먹고 살기 때문에, '진실된' 이야기일지라도 자기 교재에 관한 부정적인 평이면 개입하는 겁니다.
그리고 그런 노력 끝에 시장점유율을 유지하죠. 그래서 사람들은 '어라, 별로 좋은 교재가 아닌데도 뭔가 있나보다'라고 생각하고 구입하는 거죠.
\vspace{5mm}

응징하는 방법은 없습니다. 다만 그런 사람들이 알아서 시장에서 퇴출당하는 걸 기다릴 수 밖에 없지 않나 싶은데
제가 권하는 건 간단합니다. "언급하지 않는다고 고소미 먹이진" 못하니,
\textbf{첫째, 그런 교재들은 아예 언급하지 말고 좋은 교재만 언급한다.}
\textbf{둘째, 언급하고 싶으면 특정하지 말고 '단점'을 추상적으로 적는다(추상적 단점만 가지고는 특정 못 합니다)}
이렇게 가시길 바랍니다.
\vspace{5mm}

저저번에 올린 글도 참 흥미롭습니다만 저는 A, B가 어떤 교재인지 언급도 안 했고 그런 댓글이 안 달리길 바란다고 분명 적었습니다.
그런데 신기하게도 '단점'만 적었는데도 몇몇 분들이 교재 특정을 하기 시작했는데요
사람 의심하는 건 죄송하나, \textbf{평소에 활동 안 하던 분들이 댓글을 다는 건 좀 그랬습니다.}
게다가 특정하지 말라고 했는데, 일부러 특정하는 분위기로 몰아가는 것도요
\vspace{5mm}

그리고 오늘 아침에 가입일이 바로 11월 4일인 모 분께서 아예 B가 뭐냐고 멋대로 특정을 하기 시작하시던데
\vspace{5mm}

오늘 가입해서 '선플'을 가장해 B를 특정교재로 특정화시키려고 해서 엿먹이려한 사람, 아이디가 무려 \textbf{huntapoc}이더군요(조어관념 참 촌스럽다)
정보공유가 안 되는 것 같죠? 합법적이고 도덕적인 것에서는 유기적으로 다 돌아가고 있습니다.
여기 허혁재님이 콕콕의 비영리사이트적 특성을 강조하기 때문에 적극 안 나서고 있고, 영리적인 건 T 모 까페로 이전해서 그렇지
절대 망사이트는 아닙니다.
유감스럽게도 콕콕 사이트, 쉼터가 망해서(?) 그렇지 잘 돌아가고 있습니다.
수험에 관한 일종의 정의 관념이나 의리이지, 어디처럼 금전적 이해관계가 아니거든요.
\vspace{5mm}

앞으로도 여기 적지않은 공격이 들어올 것이라고 예상하고 있습니다. 저도 그래서 여러번 글로 낚시질을 해보면서 누차 확인한 바입니다.
이 사항은 허혁재님에게 제보되었으니 사이트 운영에 반영될 것입니다.
비겁한 질 해서 콕콕 사이트 흔들어보자라고 하는 사람은 그 시간에 공부나 좀 했으면 좋겠네요.
\vspace{5mm}

일부러 약자인 척하면 뒷공작하는 거 밝혀진 사람 꼼수는 과연 안 읽힐 것 같죠?
그리고 어떻게든 수험계에서 돈이나 벌어보자라는 사람은 이 사이트에 얼씬 안 하는 편이 낫습니다.
지금 검토하는 게 이제 어떤 뒷공작질하거나 선플을 가장해서 저렇게 나오거나 하는 경우들,
그걸 역으로 법에 얽게 만들어볼까 하는 것도 지금 생각하고 있습니다.
\vspace{5mm}

+ 그리고 예전에 제가 ㅇㅂ 대학게시판에서 활동했을 때, 거짓말아니고 거기 수험생들 흙수저들 불쌍해서(...) 이런저런 정보글 달았는데요.
아무튼 수험계가 참 웃기더군요(흙수저 얘기하지만 저야 금수저들과 놀아난 적이 있던 흙수저죠 $-$$-$ )
\vspace{5mm}

하도 웃겨서 거기서 졸업장 인증 때리고(그런 짓 쪽팔려서 이제 안 합니다만) 이런저런 정보글만 쓰는데
정보글만 써도 딴지거는 사람들이 있었습니다. 그리고 교묘하게 xxx 강사는 어떻게 생각해라고 하면서 간접광고하는 사람도 있었죠.
현실은? 그 때 거기 알바들 활약했다는 거 나중에 밝혀졌죠?
뭐 심지어 제가 거기 활동하는 것까지 파악해대는 모 학원도 있더구만요.
학원들이 머지않아 CIA에 진출할 모양입니다.
\vspace{5mm}

EBS 인강 칭찬하고 시중교재 부당하게 까이는 것을 변호한다고 까임.
심지어 EBS 알바란 소리까지 들었습니다(...)
그들이 뭔 생각인지 모르겠습니다.
\vspace{5mm}

더러운 세계죠. 만약 콕콕 사이트도 비슷하게 간다면 저도 그냥 떠나버릴 것입니다.
\vspace{5mm}






\section{개념서 함부로 쓰면 안 되는 이유}
\href{https://www.kockoc.com/Apoc/462860}{2015.11.04}

\vspace{5mm}

책 쓴다고 하는 사람들을 보면 '그냥 써볼까'일 건데요, 냉정히 말하면 책 쓴다는 건 의사의 수술행위에 비견된다고 봅니다.
책 한권이 사람의 인생을 바꾸기 때문입니다. 더 정확히 말하면 우리가 '사고'하는 건 강의나 책의 영향을 상당히 많이 받습니다.
\vspace{5mm}

우선 정석을 예로 들어봅시다. 전 정석을 매우 괜찮은 교재로 보고 있음.
학창시절에 썼던 교과서가 실력정석이기도 했지만, 지금 다시 봐도 책 5권 분량을 1권에 정말 빠짐없이 잘 넣은 책이라고 생각함.
게다가 예제$-$정석$-$유제라는 편집도 뭔가 논리적이거니와 이대로만 가면 일단 지식은 갖춰지죠.
\vspace{5mm}

그런데 정석의 저런 체계적 구조가 동시에 단점이 되어버립니다.
예제 아래 '정석'이 있는 구조로 학습해버리면, 학생들은 특정 문제는 특정'패턴'으로만 풀어야한다라고 학습해버립니다.
이건 일종의 강제이기 때문에 수학을 '싫어하게' 만드는 결과를 낳고, 실제로 부모님이 사준 정석 보고 수학이 싫어진 친구들도 많습니다.
책 자체는 괜찮은데 이 편집이 지닌 단점 $-$ 고수 입장에서는 정리가 잘 된 책이지만 고수가 아니면 '암기'로만 비쳐지는' 것
그래서 정석은 전국에서 가장 깨끗한 책 혹은 가장 더러운 책이 됩니다.
아예 안 보거나, 아니면 수십번 보거나 하기 때문에.
\vspace{5mm}

정석을 뛰어넘지 못 하고 정석에 익숙한 사람은 새로 보는 문제를 못 풉니다. 거기에 일대일대응하는 정석이 안 보이니까요(...)
\vspace{5mm}

수능 수학 기출 풀이를 보면 알겠지만 정작 쓰인 개념, 스킬은 교과서를 벗어나는 경우가 거의 없습니다.
(벗어난다고 보이는 경우도 출제 실수로 교과외적으로 풀리는 것이지, 교과서를 벗어나는 걸 의도하진 않았죠)
그래서 교과서를 잘 본 친구들이 수능은 잘 나오는데, 정석을 자기 것으로 만들지 못 하고 그 늪에 빠진 친구들은 안 나오는 경우가 생겨요.
\vspace{5mm}

실력정석은 시중 문제집을 다 풀고 1등급 나오고 만점 이상을 노릴 때 '보충서'로 보는 것이 좋다라는 게 제 소견입니다.
책 자체는 정말 나무랄 것도 없고, 그게 일본 것을 베낀 것일지라 하더라도 사실 상관이 없는 게, 이거 '좋은 것'을 제대로 베낀 경우여서입니다.
시작은 베끼는 것일지라고 하더라도 반세기동안 버텨온 역사라는 걸 가볍게 무시할 수는 없죠.
\vspace{5mm}

그런데 이건 다른 개념서들도 비슷하지 않을까 하긴 합니다.
내용이 문제가 아니라 개념이 '어떤 순서'로 학습되고 '어떤 관점'으로 조명되어야 하는 '독법'이 중요한데
그걸 책으로 제시하는 경우는 별로 없네요. 이래서 인강으로 가거나 하는 것이 아닌가 싶은데 말이죠.
\vspace{5mm}

문제를 풀 때에는 한 문장, 한 구절식 끊어 읽으면서 거기 주어진 조건에 어떤 논점들이 연결될 수 있는가 떠올리고 정리한 다음,
문제가 원하는 답에 도달하기 위한 과정을 '논술'할 줄만 알면 되는 건데
간단해 보이는 이런 과정을 제대로 설명해주는 국내 책은 찾기 어렵기도 하지만,
지금 생각해보면 그런 게 이런 건 '개념서'로 포섭하는 건 어렵다는 생각도 들고 있습니다.
4점짜리 문제를 풀기 위해서는 오히려 수리논술을 공부하는 편이 낫겠죠.
그냥 기출문제집의 해설이 기막히게 잘 쓰여있다면 그게 그냥 개념서보다 낫지 않나 하는 생각이 듭니다.
\vspace{5mm}

\begin{itemize}
    \item[$-$] 수학적 도그마로서의 개념서 : 점
    \item[$-$] 실전문제의 논리적 풀이를 설명하는 개념서 : 선
    \item[$-$] 해당 수학 개념을 다양한 관점에서 실사례와 연결시켜 보는 개념서 ; 면.
\end{itemize}
이렇게 개념서의 용도를 구분해야지, 개념서 한권으로 다 해결하는 건 어렵다라고 보는 것이죠.
\vspace{5mm}

\begin{itemize}
    \item 점 : 교과서도 좋습니다. 시중교재로 가면 형식면에서는 쎈, 감각면에서는 풍산자, 증명면에서는 셀파가 있죠.
    \item 선 : 기출문제집들을 보면서 본인이 스스로 개념서의 해당 페이지를 표시하는 식으로 책을 만들어나가야합니다.
    \item 면 : 수리 논술 교재(예컨대 남호영) 같은 것부터 시작해 각종 기출을 보면 됩니다.
\end{itemize}
\vspace{5mm}

당연히 이 글은 간혹 책을 쓰면 되지 않겠냐하는 허대장에 대한 디스인데요,
이건 허대장이 책을 쓸 능력이 없다... 가 아니라 너무 만만하게 보고 있다는 생각입니다.
일전에 제가 차트식 수학 구입을 권장해서 사보셨을 건데, 거기서 눈여겨볼 게 바로 '해설'입니다.
차트식 수학은 한문제 해설도 정말 일본인들의 장점이 묻어날 정도로 꼼꼼히 적어둬서 그걸로 개념서가 필요없게 했습니다.
이제 일격 A형 다 팔렸다니까 말씀드리면, 일격 A형 해설도 그런 식으로 더 보강할 수 있었지 않았을까 하는데요?
\vspace{5mm}

내년에 다시 치는 분이 없길 바라겠지만 다시 시작하신다면
기출을 보면서 본인이 참조하는 교과서나 검증된 기본서를 유기적으로 참조해보시는 습관을 들이시길 바랍니다.
\vspace{5mm}

+ 제시해보고 싶은 아이디어는 $-$ 뭐 여긴 훔쳐보는 사람도 많겠지만 제가 싫어하는 교재 저자라도 베끼라는 차원에서 제시합니다
학생들이 많이 보는 기본서 $-$ 정석도 좋고 교과서도 좋고 아무튼 그 중에 하나를 허대장님이 정하십시오.
그 다음 일타삼피를 내건 일격을 낼 때, 해설에 \textbf{책 참조 페이지를 표시하는 것}입니다.
가령 30번 기하와 벡터 문제,  쎈수학 기하와 벡터 p.xxx : 이렇게 말이지요.
아마 이렇게 참조만 하는 건 해당 출판사들이 확실히 반대하지 않는 한 오히려 고마워할 것입니다.
교과서에만 들어있는 내용이라면 뭐 그건 인용해도 되지 않을까 싶긴 한데, 그 외의 경우라면
학생들이 많이 보는 기본서를 가지고 표시해주면 이건 상호 윈윈이 아닐까 합니다.
\vspace{5mm}

굳이 억지로 기본개념서를 집필할 이유는 없습니다.
사실 가장 좋은 건 당해년도 EBS 수특, 수완과 연계시키는 것이 아닐까 합니다만.
\vspace{5mm}

+ 일단 이 글을 읽는 사람 중에서 난 수학이 싫다하는 경우는 머리보다 책을 의심하시길 바랍니다(만약 공부를 하는 경우라면요)
1등이 보는 교재 따라본다고 좋아지는 게 아닙니다. 소화 못 시키면 꽝나죠.
어려운 교재 보지 말고 쉬운 교재 골라보는 게 장땡입니다요. 정석을 굳이 볼 필요가 있느냐... 하면 전 없다고 말씀드리겠습니다.
\vspace{5mm}








\section{그냥 평상시대로 하십시오.}
\href{https://www.kockoc.com/Apoc/465514}{2015.11.05}

\vspace{5mm}

제가 수험사이트 쪽에 꽤 비판적인 이유를 적어보겠습니다.
\vspace{5mm}
\begin{itemize}
    \item 첫째, 수험이 뭔지 잘 모르면서 허세를 떠는 경향이 있다,
    \item 둘째, 별로 실전적이지 못 한, 비현실적인 낭만주의를 강조한다.
    \item 셋째, 실전에 도움이 되는 것보다는 장사에 혈안이 되어있다.
\end{itemize}
\vspace{5mm}

작년 말에 얘기했을지 모르지만 보통 이 기간은 공부기간에 가산하지 말라는 거, 이거 실감하시는 분들이 계실 것입니다.
왜냐고요? 멘탈이 알아서 부서지거든요.
이거 매년마다 보이는 현상이라서 별로 새로울 건 없어요.
\vspace{5mm}

그런데 이건 사실 마음의 문제입니다. 만약 님들이 수능날이 12월이라고 착각했다면 지금 긴장되거나 공부가 안 되거나 포기할 마음이 들까요.
갑자기 수능이 15일 뒤로 연장되면 우왕 기회다 하면서 달릴 사람은 달릴 것입니다.
즉, 이건 마음의 문제란 겁니다.
\vspace{5mm}

그런데 왜 마음이 압박이 되느냐.
\vspace{5mm}

전 시험 한달 전에 오프 모의든 실모든 그런 건 더 이상 추가하지 말라는 입장입니다. 콕콕도 올해 친 모양인데 그거 잘못된 거예요.
이게 실제 도움이 되느냐. 전혀 되지 않습니다. 이거 잘 나온다고 수능 잘 나오는 게 아니라, 원래 수능을 잘 칠 사람이니까 그것도 잘 나온 것이죠.
오히려 이런 것이 수험생들 마음에 엄청난 부담을 가져다주고 쓸데없는 낭만주의를 부추깁니다.
\vspace{5mm}

사실 시험은 그냥 '내일' 본다고 생각하면 됩니다. 이 얘기는 즉, 지금 일주일이 남았건 3일이 남았건 달라질 건 없다는 것입니다.
수능은 지식의 암기량으로 치르는 게 아닙니다. 얼마나 기본적인 걸 똑바로 알고 있는지, 그리고 시험날 얼마나 집중이 잘 되어있느냐 그것이죠.
그럼 눈 부릅뜨고 포스를 발휘해야 집중이 되는가. 그게 아닙니다, 그냥 평상시대로 보라는 겁니다.
그냥 일주일동안 시험 리허설만 줄창 하면서 시험날 최상의 컨디션을 발휘할 수 있게 조절하면 되지 지금은 뭔 교재 본다 그럴 때가 아니죠.
\vspace{5mm}

이렇게 긴장하고 하는 건 올초부터 여름까지 그랬어야하는 거지, 지금은 오히려 이완을 하면서 여유롭게 시험대비를 해야할 차입니다.
막판 정리한다 뭐다 하는데 지금 괜히 후까시 잡고 긴장해보았자 그거 일주일 못 갑니다. 오히려 소집일날 맥이 확 풀려서 더 맛이 간다니까요.
\vspace{5mm}

원양어선 냉동고 이야기 아시죠?
어떤 사람이 운나쁘게 냉동고에 갇혔는데 나오지 못 합니다.
자기가 죽어간다라는 기록을 남기고 정말 얼어(?) 죽습니다.
나중에 문을 열어본 동료들은 놀라죠.
자기 동료가 냉동육이 되어서가 아닙니다.
냉동고가 꺼져있어서 그 안은 '상온'이었거든요.
\vspace{5mm}

실제로 열심히 달려온 사람들은 지금 연초에 비할 데가 아닙니다.
실력은 객관적인 것입니다. 공부하신 분들은 뇌가 그만큼 단련되어 있지요.
왜 불안하나? 그거야 지금은 끊임없이 자기 비하를 해서입니다.
그럼 왜 비하를 하나? 시험이 부담되기 때문이죠. 그래서 자기가 공부 못 하는 못난 놈이라고 생각해야 안정이 옵니다.
사람은 불안할 때 자기 비하를 하면서 '나쁜 결과를 정당화시킬' 준비를 하죠.
하지만 이건 나쁜 방향으로의 진화이고 '노예'로 살아온 우리 조상들의 근성이 남아있는 결과죠.
\textbf{'나쁜 결과를 정당화시키'}는 건 \textbf{혼나기 싫어서}입니다. 즉 님들은 과거에 혼났던 망령에 사로잡혀서 핑계 댈 준비를 하고 있는 것이지요.
\vspace{5mm}

결과가 좋게 나오건 나쁘게 나오건 그렇다고 당장 죽거나 그런 건 아닙니다. 그리고 도대체 누가 혼낸다는 것이지요?
차라리 혼낸다면 내가 이렇게 열심히 공부했는데 도와주지 못 한 주변 사람을 혼내든가
시험에 나오는 내용을 빠뜨리거나 그걸 부각시키지 못 한 교재를 혼낼 준비나 하시지요.
정말 어이가 없는 게 돈도 자기들이 지불하고 공부도 해놓고, 나중에 인강 강사든 교재든 가족 눈치를 본다는 것입니다.
진정 갑(甲)은 공부하는 본인들이지, 공부 안 한 주변 사람들이 아닙니다.
공부해도 시험성적인 안 나올 수도 있는 것입니다. 그게 그렇게 놀라운 것도 아닌데 뭘 강박을 가지고 주변 눈치를 보십니까.
이렇게 되면 수능이 문제가 아닙니다. 죽을 때까지 남 눈치나 보고 살게 되어 있습니다.
\vspace{5mm}

그냥 평상시에 밥먹고 숨쉬고 화장실 가듯 그렇게 치세요.
어떻게든 돈이나 벌려는 업자들이나 이걸 이벤트화합니다. 그게 수험생에게 얼마나 부담을 주는지 모르면서요.
수능 끝나고 나올 때 매우 허탈하실 것입니다(n수생은 이걸 기억 하실 거예요). 그리고 아쉬우면서 아 그것만 더 공부했으면... 하는 생각이 들겠죠.
사실은 정확히 침착하게 문제 읽기만 해도 풀리는 것인데 괜히 긴장하거나 울거나 그래서 풀 수 있는 것도 놓치는 경우가 더 많습니다.
수능 시험 문제는 어려운 게 아닙니다. 단지 자기가 공부한 방향과 약간 '엇갈려' 있을 뿐입니다.
화제가 되는 문제는 늘 새로운 것이어서 그렇지요. 그렇다면 새로운 문제를 대비하려면? 침착해야 합니다.
하지만 가장 중요한 건 그 새로운 문제를 푸는 데 오히려 '방해가 되는 선입견이나 지식'이라면 버려야 합니다.
\vspace{5mm}

예컨대 수학의 경우만 하더라도 고정관념적인 패턴이 문제풀이에 방해가 되는 경우가 많습니다.
국어의 비문학 지문 문제든 영어의 빈칸도 수험생의 고정관념을 교묘하게 자극하는 케이스이지요.
이건 즉 지금 긴장하고 계신 분들이 강박적으로 암기하는 지식이나 풀이가, 오히려 수능에는 독이 될 수 있다는 걸 말하는 겁니다.
수능 시험을 치를 때에는 오히려 그동안 배웠던 것을 '잊어버리는' 것도 필요합니다.
뭔 뜬구름 잡는 소리냐 하겠지만 수능문제는 정말 뇌에서 알아서 푸는 것이고 우리 의식은 그걸 거들 뿐입니다.
\vspace{5mm}

자, 그렇다면 일주일 남긴 지금은 무엇이 필요할까요? 비장함? 단단한 각오? 아니면 감동적인 세러머니?
그딴 건 연초에 했어야하지요. 꼭 사람들이 연초부터 봄까지는 여유부리며 놀다가 지금 와서는 바짝 긴장하는데 이거야말로 웃긴 겁니다.
11월 11일에 예비소집이죠? 그 날은 친구들과 가볍게 대화하고 시험장에 다녀오고 그 다음 집에서 와서 가볍게 막판정리하고 푹 주무세요
그럼 11월 6일부터 11월 10일까지 가장 중요한 것?
\vspace{5mm}

건강관리입니다. 지금 마지막 수능이니 뭐니 그러면서 맹렬히 달리는 거 기특하긴 한데 그러다가 쓰러지면 아무 소용없습니다.
공부시간은 최소시간으로 잡고 수능시험 시각대에 맞춘 과목배정으로 문풀하면서 그냥 컨디션 관리를 하세요.
어떻게든 문제를 풀어서 점수 나오는 것 가지고 위안을 받고 싶어하겠지만 그건 시험 당일 컨디션에 그다지 도움이 되지 않습니다.
극단적으로 말하면 국영수탐 모두 약한 것만 골라서 아주 느리게 $-$ 수능시험 시각대에 맞춰서 훑어보고 여유있게 보내도 좋습니다.
주의하세요, 지금은 감기 시즌입니다. 게다가 수능날은 춥지요. 괜히 막판에 열심히 한다고 하다가 건강 날라가면 아무 소용 없습니다.
부담감은 문풀에 전혀 도움이 되지 않습니다. 강박관념에 시달리는 사람은 문제를 유연하게 읽지 못 하지요.
\vspace{5mm}

저 자신도 현역으로 대학에 들어갔고 나름 공부 잘 하는 인간들 틈에 있었으며
콕콕에서도 작년에 성공한 분들을 정리하면서 보면서 느낀 것 그대로 적는 겁니다.
합격하는 사람들은 뭔 실모 본다 어떤 교재 본다 이런 걸로 후까시 잡지도 않고 이상한 입시교주 숭상 같은 것도 하지 않습니다.
자기가 필요한 교재 정해서 그거 여러번 풀어서 자기 것 제대로 만들고, 정말 '핵심'적인 것만 제대로 스나이핑해버립니다.
이런 사람들은 온오프건 대화해보면 느끼지만, 절대 장황하지 않습니다.
오히려 일반인들에 비해 느슨하고 심지어 나사가 빠진 경우도 봅니다만, 집중할 때는 제대로 집중합니다.
평소에 느스한 상태로 있기 때문에 집중할 때에 제대로 집중할 수 있는 것이지요.
\vspace{5mm}

그에 비해 실패하는 사람들은 참 이것저것 많이 벌입니다. 그리고 막판에 감당을 못 해대죠
여러번 n수 하는 이유? 여러가지가 있겠지만 결국 이유는 '집중'하지 못 해서입니다.
본인은 머리가 나빠서... 라고 정당화하겠지만 틀린 이야기입니다. 정확히 말하면 '습관'이 잘못되어서 그런 것이지요.
습관이 잘못된 사람은 하나에 집중해도 모자랄 판에 여러군데 일 다 벌여놓고 수습을 못 해요. 대단히 산만하지요.
그리고 그 다음 해에 또 응시할 때에는 본전 찾겠다고 더 일을 크게 벌이다가 또 말아먹습니다.
\vspace{5mm}

이 정도면 메시지 전달은 충분히 되었다고 여기네요
자기가 계획안 것의 절반으로 줄이고, 11월 12일에 풀집중이 가능하도록 이제 심신을 이완시켜주시길 바랍니다.
시험 시각에 해당하는 시간동안만 공부하고 나머지 시간은 여유있게 보내시면서 '감기 걸리는 일'이 없도록 주의해주세요.
그리고 당황스러운 문제를 읽으면서도 침착하게 분석해서 그걸 풀어내는 자기 모습을 계속 상상하시길 바랍니다.
\vspace{5mm}








\section{운명}
\href{https://www.kockoc.com/Apoc/465783}{2015.11.06}

\vspace{5mm}

사주팔자가 다 맞나 안 맞나 그건 모르겠지만, 적어도 인터넷 덕분에 검증을 할 수는 있다는 것.
점집 50군데를 돌아본 사람이 정작 맞춘 곳은 2곳 밖에 없다라는 식으로 나오는데, 이 '확률'은 그냥 원숭이에게 OX를 맡기는 것보다 낮다.
\vspace{5mm}

그래도 사람의 운명이란 건 왜 정해져있다고 느끼게 되는 것일까에 대한 과학적(?) 접근을 시도해보면서 느낀 것.
그건 '익숙함'과 관계가 있어서라는 것이 요즘 드는 생각이다.
불행함에 익숙한 사람은 불행한 삶을 살지 않으면 불안해 한다.
공부 못 하는 것에 익숙해진 학생은 성적이 안 좋게 나오면 불안해 한다.
이게 뭔 개소리요 하는 사람들이 있을 수도 있겠다만 적어도 내가 관찰하고 경험한 바로는 이게 정말 만만치가 않다.
\vspace{5mm}

유전 이야기를 하는 경우도 많은데 이건 어설픈 유사과학적 접근이 아닐까.
과학적 접근이라는 건 관찰을 오랜 기간 해보면서 데이터를 축적해보고 가설을 검증해나가는 과정을 거쳐야하는 것이다.
그런데 유전적인 것이 실제로 검증된 적은 내가 아는 한 별로 없다. 부모가 유전자가 좋아서 자녀가 좋다면,
반대로 부모가 평범하거나 못난이인데 자녀가 좋은 경우는 어떻게 설명한단 말인가.
부모가 유전자가 안 좋아도 자녀가 좋을 수도 있다는 이야기라면, 부모 유전자가 좋아도 자녀가 안 좋을 수도 있단 얘기가 된다.
\vspace{5mm}

정말 제대로 검증해본다면 드라마처럼 의사가정과 서민가정의 자녀가 산부인과에서 바뀌었더라.... 는 걸로
'가정환경'이라는 통제변인 조절이 있어야하지 않나.
\vspace{5mm}

예전에 모 프로그램에서 쌍둥이 사주를 검증해본 적이 있다.
일란성 쌍둥이 자매였으므로 ⓐ 동일한 유전자 ⓑ 동일한 사주 $-$ 그런데 한명은 미국으로 입양되었다.
그 결과는?
\vspace{5mm}

한국에 남은 사람은 무속인이 되었고, 미국에 입양된 사람은 '대학교수'가 되었다.
이 당시 사주관련한 까페나 블로그에서는 대난리가 났다. 당연히 업자들의 그럴싸한 변명이.
이게 시사하는 바는 가장 중요한 건 '환경'이라는 이야기이다.
어느 지역에 사느냐, 어떤 가정에서 자랐느냐, 그리고 명문학교에 갔느냐 못 갔느냐 라는 것이 매우 중요하다는 이야기이다.
\vspace{5mm}

자기가 머리 탓을 하는 학생들은 우선 자기가 어떤 환경에서 살고있는가 그것부터 검증해보면 된다.
그냥 검증하지 말고 정말로 공부를 잘 하는 레전드들과 하나하나 비교해보면서 체크해보면 되는 것이다.
사소한 차이가 중대한 변화를 가져오는 것이다.
\vspace{5mm}

이렇게 보자면 내 입장도 수정해보긴 해야겠다. 내 경우는 어느 교재든 사실 그리 큰 차이는 없다고 본다.
비판한 A$\sim$F 교재도 사실 본인이 열심히 본다면 큰 차이는 없을 것이라고 보는 입장이긴 한데
생각해보니 이건 내가 어느 정도 많은 교재를 거쳐서 면역력(?)이 생겼기 때문이라고 할 수 있단 생각이 들어서 수정해야할 것 같다.
교재도 일종의 환경이라면 \textbf{'교재 잘못 선택해서 인생 말아먹은 케이스'가 많다}라는 게 논리적인 결론이다.
하지만 이걸 어떻게 얘기해줘야 할까.
\vspace{5mm}

공개적으로 교재 실명을 대면서 칭찬하거나 비판하기 힘든 이유는 현실의 민감한 이해관계 때문이다.
내가 양심과 노력으로 그거 검증해서 얘기해보았자, 당사자 입장에서는 $-$ 어차피 자기는 죄책감 없다, 돈만 벌면 장땡 $-$ 속이 타버릴 뿐이지.
또한 내가 좋다고 했는데 이걸 믿고 선택한 사람들이 정작 그 교재 때문에 망했다고 한다면(사실 교재보다는 다른 이유가 크다고 보지만)
이건 내가 할 말이 없어지는 것이기 때문에 교재 추천에 있어서 매우 신중한 태도를 취할 수 밖에 없다.
\vspace{5mm}

그런데 교재가 하향평준화되는 이유. 이건 시장의 축소도 축소지만 한가지 격세지감을 들어볼까.
합리적이 된다는 건 좋은 일인데 이게 과거보다 어떤 의미에선 이번 모 가수 제제 사건만큼 세기말이라고 느껴지는 게
가만히 보면 다들 눈에 $ $ 표시를 하고 다닌단 것이다.
이거 수험생들은 못 느낄 수도 있겠지만, 타락한 내 눈에야 훤히 보인다.
말로는 좋은 문제를 만들어서 공유하고 싶다... 라고 하는데 당연히 그건 공산당 독재 하기 전에 농민들 꼬실 때 하는 소리고,
현실은 나중에 자기 교재가 얼마나 많이 팔렸나 그런 걸로 자랑하고 위세떨고 다니는 참 거북한 광경을 보게 된다.
이런 것을 비판하면 "너 질투해서 그렇지"라는 반응이고 심하면 고소미를 먹기 일쑤인 것이다.
물론 그런 교재들이 훌륭한가.... 전혀 아니올시다이다. 벌거벗은 임금님이란 동화책이 괜히 고전인 게 아님.
그렇다고 그 사람들 실적이 훌륭한가, 내가 보기엔 그것도 아니야.
\vspace{5mm}

그런데 더 유감스러운 건 그런 사람들이 위세를 떨고 다니니까 지금 수험생들도 하라는 공부는 안 하고
어떻게 하면 자기들도 저 존경스러운(?) 선배님들처럼 되어서 교재 팔아서 수천수억을 벌어볼까 그 궁리를 하는 게 보인단 이야기다.
참교육이 따로 없다. 이런 것들에 영향을 받기 시작하면서 운명이 결정되는 것이다.
열심히 공부해서 수험에 매진해도 모자랄 판에 어떻게 교재 만들어서 그걸 팔 궁리해볼까라는 쪽으로 간다면 사실 심각하다.
그렇게 벌어댄 돈, 그 자신이 날려버리는 젊은 시절에 비하면 정말로 \textbf{푼돈}이다. 당연히 돈에 눈이 멀어댄 사람에겐 이런 충고가 안 들린다.
돈은 거짓말아니고 나중에 어떻게든 벌 수는 있지만, 공부와 경험에 투자했어야 하는 그 젊은 시절은 수천억을 줘도 돌아오지 않는다.
이 경우는 나중에 나이먹으면 정말 후회들 한다. 그 시절에만 할 수 있었던 중요한 배움이나 일이라는 게 있다.
\vspace{5mm}

환경이라는 게 이렇게 참 무서운 것이다.
인간이 운명을 바꾸기 어려운 것은 자기가 익숙한 환경이나 달콤한 금전적 수익을 버릴 수 없기 때문이다.
하지만 그게 정말 자기가 '패망하는' 길이라는 걸 안다면 과감히 벗어날 것이다.
2차 세계대전이 터지는 걸 알았다면 그 직전 프랑스나 이탈리아에 사는 유대인들은 과감히 미국으로 탈출했을 것이다.
하지만 모든 미래는 사실 $-$ 자기가 보고싶지 않은 미래일수록 $-$ 매우 터무니없어보여서 무시하기 좋다.
\vspace{5mm}

이번 시험을 치르고 결과가 시원치 않은 사람은 우선 '환경'부터 점검해보는 걸 추천한다.
\vspace{5mm}

운명을 바꾸는 법 $-$ 소위 개운이라고 한다면 그건 굿판이나 부적일 수도 있겠지만(사실 그리 효과는 크지 않다)
가장 좋은 건 사는 곳을 바꾸는 것이다.
사주팔자를 신봉하는 입장인 경우도 죽을 운일 때 외국으로 튀어서 겨우 악운을 피했다라는 믿거나말거나 글이 있다.
사는 곳을 바꾸기, 공부하는 곳을 바꾸기.... 이것만큼 영향이 가장 큰 것은 없을 것이다.
인강과 실강의 차이도 이런 데 있지 않을까.
\vspace{5mm}

인강은 혼자 듣는다. 따라서 자기가 공부를 안 하게 되더라도 그걸 파악할 수 없다.
하지만 실강에서는 공부하는 다른 친구들을 보게 된다. \textbf{자기가 조금이라도 안 하면 뒤처지는 것을 본다}(이게 가장 중요하다)
방에서 혼자 공부하는 건 인강에 준한다. 그러나 도서관에 가면 열심히 책읽는 사람을 보고 거기에 싱크로를 맞추게 된다.
\vspace{5mm}

교재가 감성적으로만 쓰여져 있다. 이런 교재를 본 사람은 감으로 문제를 풀 것이다.
교재가 스킬 위주로만 적혀있다. 스킬에만 의존할 것이다.
교재가 논리적이다. 시간이 걸리지만 그 수험생은 논리적으로 문제를 푼다.
\vspace{5mm}

하지만 공부를 열심히 하기는 쉬워도 자기 환경, 즉 관성을 바꾸기는 매우 어려워보인다.
공부하겠다고 다짐만 하면서도 1년 넘게 안 하는 사람들도 널렸다.
그 사람들은 다시 시작해야지 하면서 교재 주문 하고 계획짜는 걸 한 수십번은 반복했을 것이다.
\vspace{5mm}

운명을 바꾸기 힘든 이유가 이런 데에 있지 않을까.
\vspace{5mm}

사주팔자를 긍정한다고 하더라도 운명이 정해져있다면 이건 사실 필요가 없다.
운명을 바꾸고 싶어하니까 사주팔자를 따지는 것이다. 그런데 정작 사주팔자에는 운명을 바꾸는 방법은 없다.
단지 언제 좋을지 나쁠지 그것만 간략하게 제시되어 있다.
\vspace{5mm}

세계사는 사실 서양이 동양에 승리한 게 1부다. 지금은 그 서양을 배운 동양의 패자부활전 과정?
자본주의 산업혁명 여러가지 말이 많지만 그것들도 결과이지 근본적 원인은 아니다.
가장 근본적인 것은 바로 종교다. 그렇다고 우리나라 일부가 얘기하듯 하느님을 믿어서 어쩌구 그런 게 아니다.
기독교는 신과 인간이 계약 관계이다. 즉, 인간은 자기가 의무를 이행한만큼 신으로부터 권리를 얻어낼 수 있는 것이다.
이런 기독교의 사고방식이 적극적이고 개척적인 역사의 진보를 낳는데 기여한 것이다.
동양에는 이런 사상이 사실 없다(억지로 정신승리하려고 맹자 이야기를 끌어내거나 양명학을 얘기하지만 다들 한계가 뚜렷하다)
\vspace{5mm}














\section{콕콕에서 교재평할 때 룰을 정해드리겠음.}
\href{https://www.kockoc.com/Apoc/467585}{2015.11.07}

\vspace{5mm}

과거 게시물 보면서 댓글(?)을 보면 제가 왜 교재평을 구체적으로 안 했는지 그 이유 이제야 이해가시는 분 많을 겁니다.
\vspace{5mm}

저야 허심탄회하고 가볍게 글을 써도 두가지 면에서 부담이 되어요.
\vspace{5mm}
\begin{itemize}
    \item 첫째, 그냥 가볍게 쓴 것인데 조회수가 높아진다(... 이거 ㅇㅂ게시판에서도 그랬음 ... 그런데 그런 내용조차 조회수 높았으면 얼마나 막장이었냐)
    \item 둘째, 그거 해당 저자들이나 출판사가 보면 가만히 안 있음. 소통은 개뿔, 어찌되었든 비겁한 공격이 들어올 가능성이 있다.
\end{itemize}
\vspace{5mm}

그렇다고 교재 평을 안 할 수는 없겠고.
이건 안전한 룰이 없나 싶어서 생각해보는데 간단하더구만유.
다음과 같이 적으시길 바랍니다.
\vspace{5mm}
\begin{enumerate}

    \item   칭찬하는 교재는 구체적으로 \textbf{장점과 단점을 평해준다 : 단, 이건 권해줄 수 있는 교재에 한한다.}
    \vspace{5mm}

    호의적인  비평글은 구매자의 구매의욕을 불러일으킵니다.
    자기 교재  좋다고 하는데 뭐라하는 정신나간 업자는 없겠죠.
    다만 명예훼손이나 모욕적인 건 그래도 삼가시는 게 좋습니다. 단점은 구체적으로 사실지적만 하면 되는 것이겠죠
    \vspace{5mm}

    \item 칭찬하지 않는 싶은 교재는 그냥 \textbf{"언급을 하지 않습니다", "노 코멘트", "언급제외", 아니면 "잘 모릅니다"}라고 언급한다.
    \vspace{5mm}

    생각해보니 이게 그나마 무난하더구만유. 언급하지 않는다고 고소미 먹이면 그냥 이 대한민국이 정신나간 나라라서 말입니다.
    사실 제가 교재 저자라면 누가 깐다고 하면 적극적으로 소통하고 해명할 것입니다. 이게 정상이죠.
    그런데 이 나라에는 비정상들이 많아요. 똥이 더러우면 피해갈 수 밖에 없죠, 그렇다면 위와 같이 그냥 한줄로 정리하는 게 좋습니다.
    물론 교재비평글을 쓰라는 게 아니고, 콕콕 회원 분들이 교재 질문을 받을 때는 저렇게 한줄로 답하란 이야기.
    \vspace{5mm}

    \item 교재 비난을 하고 싶으면 \textbf{추상화}시켜라.
    \vspace{5mm}

    가령 "Z라는 양반은 탐욕스럽고 여자나 밝히고 한두번 쓰레기버린 게 아니고 인성 쓰레기다"라고 익명성으로 적는다면
    이거 자기가 Z니까 문제된다고 나서는 정신나간 사람은 없겠죠. 설령 엉뚱한 사람이 자기 욕한 거지라고 갑툭튀해보았자 이건 법과 관계없어요.
    다만 추상화라는 것은 정말 철저한 익명화를 얘기하는 겁니다.
    \textbf{가령 '그 실전모의고사 개쓰레기야, 필적확인문구가 현 정권 비난하는 시사적인 것이었어. 신문기사에 난 적도 있었지'}
    \textbf{라고 한다면 이건 안경 낀 이토 준지처럼 보이지만 실은 colorful bone으로 알려진 모 모의고사 저자로 특정될 가능성이 있어서 문제가 되죠.}
    하지만 반면 '거창하게 광고했는데 기출만 박아놓았더라', '분명 스킬 다 소개해준다고 했는데 스킬 별로 없더라'
    이렇게 기술하는 건 문제가 되지 않습니당. '교재값 비싸더라' 혹은 '팬들이 많다'도 그렇죠. 이건 엄연히 복수형인지라.
    \vspace{5mm}
        
        
\end{enumerate}
아무튼 최근에 있었던 모 사건을 보고나서 그렇다고 교재 비평 문화를 활성화 안 시킬 수는 없고 어떡하나 하다가
서로 낯붉힐 것 없는 룰을 정하면 된다고 해서 떠오르는 아이디어를 적어보았습니다.
\vspace{5mm}

가장 중요한 4번 추상화를 할 때에는 가령 교재 둘을 섞어서 가상의 교재를 가정해서 적어도 되겠고
혹자는 일부러 몇몇 사항들을 바꿔넣어서 아예 논란이 되지 않게 하는 방법도 있을 겁니다요.
하지만 가장 중요한 건 철저한 익명화죠. 그 경우 모 교재로 특정하는 사람 자체가 악플러 이거나 그런 걸 노리는 세력이라는 게 드러날 정도로 말이죠.
\vspace{5mm}

앞으로 콕콕 사이트는 계속 성장할 테고 들어오는 사람들도 많아지는 동시에 '악인'들도 적지않게 들어올 겁니다.
조만간 회원등급이나 권한에 대해서는 변경이 있을 거라고 예고 받은 바 있는데, 저런 룰을 관습화시키는 게 중요하다고 생각하네요.
특히 나쁜 교재야 특정할 필요없이 '나쁜 특징'을 가지고 추상화하면 그것만으로도 정보전달은 충분하기 때문입니다.
\vspace{5mm}

그리고 운영진 차원에서도 만약 그런 추상화된 비평글에 \textbf{특정교재나 강사 지목하는  댓글이 달리면}
\textbf{그건 삭제하고 그 회원은 권한 박탈하거나 차단시켜야겠죠.}
이렇게 하면 교재 비평글은 활성화될 수 있다고 생각하고
안 그래도 정보부족으로 시달리는 수험생들에게 시원한 사이다가 되겠군요.
\vspace{5mm}

아울러 문제가 많은 출판사나 법적인 문제가 있을 것 같은 교재는 '양해 구하고' 금지어로 지정하는 것도 사실 필요하다고 봅니다.
회원수가 늘면 어떤 게시물에다가 어떤 트러블일 날지는 모르겠는데 운영자 분 입장에서는 아주 번거로운 일이 되겠죠.
그렇다고 여기가 상업주의적으로 모 업체 교재만 아니면 광고 안 되는 그런 곳도 아니긴 하지만요.
\vspace{5mm}








\section{실패의 원인은 계량 실패}
\href{https://www.kockoc.com/Apoc/472231}{2015.11.09}

\vspace{5mm}

왜 성공하는 사람들은 자주 성공하고 실패하는 사람들이 자주 실패하는가... 는 한번쯤 고민할 문제가 있어요.
운이라는 게 있다고 봅니다만, 애당초 통제가능하지 않은 건 아예 고려하지 않는 게 나음
\vspace{5mm}

실패하는 사람들의 실패습관이야 여러가지 많습니다만.
가장 큰 문제는 밸런스가 깨져있단 것입니다.
\vspace{5mm}

A라는 프로젝트에 100이 필요, 보상은 1000, B라는 프로젝트에 50이 필요, 보상은 250, C라는 프로젝트에 300이 필요, 보상은 10,0000
그리고 자원이 1000이 주어져있다면 어떻게 하겠습니까.
\vspace{5mm}

여러가지 해답이야 나오겠죠.
그런데 오답은 지적할 수 있겠군요. 저기서 \textbf{2개 이상 선택하면 무조건 실패합니다}.
아니 1000이 있으면 100+50+300을 훨씬 초과하니까 상관없지 않냐 하는 질문이 따르겠죠
이런 질문을 하니까 망한단 것이죠.
\vspace{5mm}

현실은 덧셈이 아니기 때문입니다.
\vspace{5mm}

첫째로 프로젝트는 반드시 시행착오라는 걸 하게 됩니다.
명목상 A가 먹는 자원은 100입니다. 하지만 이건 '성공'했을 때의 기준이죠.
하지만 사람들은 평균적으로 3$\sim$5번은 실패합니다. 그래서 500 이상을 낭비해버리고, 이게 정상인 것입니다.
현실의 모든 작업은 반드시 실패하는 횟수를 전제해야하는 것인데 똑똑하고 욕심많은 사람일수록 이걸 고려하지 않죠.
실제로 이런데 A, B,C까지 다 했다면? 1000으로는 도저히 감당이 안 됩니다. 게다가 집중도 못 하고 실패는 더 늘어나니 자포자기하게 되죠
\vspace{5mm}

둘째로 자원소비는 산술평균적으로 이뤄지진 않습니다. 기하급수적으로 늘어난다는 것입니다.
수험 이야기를 한다면 2등급에서 1등급 올릴 때 10시간이 필요하면, 1등급에서 만점권까지 가려면 $10^2$ 시간이 필요합니다.
C는 명목상 300입니다만, 실제로 진행하다보면 300이 더 늘어나게 될 때 이게 1차 함수가 아니라 지수함수 꼴로 늘어납니다.
\vspace{5mm}

셋째로 새로운 프로젝트가 계속 생긴단 겁니다.
A를 했다가 실패했다고 칩시다. 그럼 실패한 A를 정리하는 프로젝트 A1, 다시 재도전하는 A2, 그리고 실패 원인을 뿌리뽑는 D 프로젝트가 생겨납니다.
다시 말해 프로젝트 하나는 절대 하나가 아닙니다. 이것도 진행하다보면 꼬리치고 새끼치면서 일감이 더 많이 늘어나게 됩니다.
이걸 만회하는 건 '학습효과'겠죠.
\vspace{5mm}

n수생들의 실패 원인은 다른 게 아니라, 목표 계산을 잘 못 한 것이 가장 큽니다.
멀리서 보면 에베레스트 산도 손바닥 안에 들어가죠. 그러나 실제로 등반을 하면 ㅎㄷㄷ
예쁜 여자도 사귈 때는 행복한 것 같죠. 그러나 그 여자가 흔한 동네 미용실의 욕지거리 아줌마로 진화하는 건 삽시간입니다(그러니 2D로)
문제집은 한권이죠. 그러나 대략 500문제가 있고, 여기서 오답이 150문제이며 모르는 문제가 30개면 프로젝트는 180개로 증가합니다.
\vspace{5mm}

자, 그렇다면 해답은 간단하죠.
\textbf{가장 자원이 적게 드는 것을 빨리 끝내는 수 밖에 없습니다.}
보상은 낮지만 그건 확실히 내 자본이 되기 때문이죠.
\vspace{5mm}

한데 수험생들은 올해 수능을 치고 또 1년이란 기간이 주어진다고 착각들 하겠죠(자기 청춘을 깎아만든 시간인데)
아마 올해 학습하신 분이 있어 몇몇은 정신들 차리겠지만, 그래도 또 허송세월하며 3월까지 날리는 사람 분명 있습니다.
손바닥 안에 들어가는 에베레스트산이니 꼭대기까지 금방 오른다고 착각한 게 문제건만
그런 건 고려 안 하고 공부는 자기 적성에 안 맞는다는 둥 머리가 나빠서 그런다는 둥 원인을 엉뚱한 데 찾죠.
\vspace{5mm}

열심히 공부하신 분들 콕콕에 계시죠. 그리고 정상적인 과정들 보입니다.
제가 봐도 미친 듯이 공부했는데, 지금도 덜 공부한 것 같다고 그럽니다. 예, 이게 정상입니다.
이 분들은 에베레스트산 꼭대기에서 산소부족으로 허덕이는 것이거든요.
반면 얼굴빛 좋으면서 아, 이번 시험 어쩔까하는 사람들 있습니다. 산 언저리에서 쉐르파와 노가리까며 기도하고 있어요.
마음은 가장 편할 겁니다. 이 사람들은 내년에 또 치고 내후년도 갈테니까요. 그 빚은 이자쳐서 갚게 되어있죠.
\vspace{5mm}

필승해법은
\textbf{'적의 수를 줄이고 다수로 포위해서 섬멸한다'}
즉, 각개격파죠.
일대일로 싸우면 멋져보이지만 죽을 위험도 높죠
그러나 비겁해(?) 보일지라도 30대 1로 승부하면? 평화적으로 마무리지을 수도 있죠.
그런데 실패하는 사람들은 일대일 승부, 혹은 일대백 승부를 선호합니다. 당연히 죽죠.
\vspace{5mm}

공부든 뭐든 다 마찬가지입니다.
\vspace{5mm}

일찍일찍 합격해 가는 사람들은 \textbf{꿈은 크지만 욕심은 적은} 사람입니다.
그 사람들은 높이 올라가려 합니다. 하지만 성급히 올라가지 않죠. 차분히, 늦게, 소극적으로 계단까지 만들고 올라갑니다.
처음에는 시간이 걸립니다, 그러나 안전하게 올라가므로 추락하는 일은 없죠.
그 반대로 \textbf{욕심이 큰 사람들}
내, 이 사람들은 바로 실패란 말을 이마에 써두고 있습니다.
마린 2명을 히드라 30마리에 보내는 미친 짓을 하고 있죠.
그래서 이들에게 수험은 '낭만'이 됩니다. 왜냐고요? 실패를 포장이라도 해야하니까요.
\vspace{5mm}

상담쪽지 보내는 분들에게 공통적으로 하고싶은 이야기가 이겁니다.
크게 욕심내지 말고 \textbf{"작은 것"부터 확실히 끝장내라}.
너무 당연한 교과서적인 이야기지만 실천하기는 가장 힘듭니다.
실패하는 사람은 사자 한무리를 사냥하다 망하지만, 성공하는 사람은 사자 12마리를 한마리씩 공략해서 성공합니다.
\vspace{5mm}







\section{기싸움}
\href{https://www.kockoc.com/Apoc/470657}{2015.11.09}

\vspace{5mm}

"우리가 사는 곳은 현실인가 가상인가"
\vspace{5mm}

여러가지 썰이 있고 이와 관련해서도 수만장의 논문이 나오겠지만
자연과학적인 입장에서는
우리가 사는 세계는 엄밀히 따지면 현실 그대로가 아니죠.
\vspace{5mm}

우리의 뇌를 통해 인지하고 재구성한 \textbf{가상현실}이죠.
\vspace{5mm}

수능시험을 앞두고 왜 멘붕하게 되느냐.
싸우고 싶지 않아서 핑계를 대고 싶기 때문입니다.
\vspace{5mm}

만약 수험생이 군주, 장수라면 싸우고 싶어서 웅장한 bgm이 깔리면서 의욕이 올라가겠죠.
그러나 유감스럽지만 우리들은 대부분 양민, 노비들의 자손입니다(족보 그거 구라인 것 다 알죠? 진짜 양반 별로 없어요)
얻어맞고 학대당하던 그런 하층민스러운 기질이라는게 유전되어서인가 힘든 상황이 오면 포기할 준비부터 하죠.
왜 핑계를 대느냐, 그래야 \textbf{덜 혼나기 때문}입니다.
\vspace{5mm}

지금부터 세팅해야하는 건 '냉정하게 킬러문제를 스나이핑'하는 전투 게임으로 가야하는 것입니다.
\vspace{5mm}

\begin{enumerate}
    \item 공간의 장악
    \vspace{5mm}

    수요일에 예비소집한 다음 장소 통보 받겠죠. 아주 멀지 않다면 시험장소에 다녀오시길 바랍니다.
    가서 내가 내일은 뼛속까지 다 짜내서 높이 올라가겠어라고 선포하고 오면 됩니다.
    간혹 곳곳에 부적을 붙이거나 콩팥을 뿌리거나 심지어 몰래 노상방뇨까지 하는 케이스도 있던데 뭐 그건 제가 권할 바는 아니지만
    이런 의식을 치르는 것부터가 호랑이가 영역권 표시하는 것과 비슷한 것이니 효과가 없다고 볼 수는 없겠죠.
    시험 당일에는 무조건 일찍 가신 다음 시험치르는 교실의 복도를 다 걷고 화장실에서 용변보고 거울보고 썩소 짓고 하시길요.
    주변 공간을 왜 스카우팅하느냐. 그래야 불안감이 사라집니다. 우리는 낯선 공간에 있으면 경계심을 발휘하므로 집중력이 떨어집니다.
    자기가 정확히 어떤 공간에 앉아있으며 주변에 어떤 지형지물이 있느냐를 알고 나면 더 이상 그 주변을 의식하지 않게 됩니다.
    \vspace{5mm}

    \item  시간의 장악
    \vspace{5mm}

    시험시각에 맞춰 해당 과목을 푸는 리허설도 하고 계실 것입니다.
    오늘도 6시에는 일어나셨을 거라 믿겠습니다만 화, 수요일도 새벽 5시 30분에 일어나는 세팅을 해두시길 바랍니다.
    그리고 시험시각에 맞춰 반드시 해당과목을 공부하거나, 설령 수업 때문에 못 한다고 하더라도 "지금은 국어시간이야"
    "지금은 수학 29번으로 내가 n+1을 하면서 가슴앓이할 시각이지"라고 중얼거리는 것도 나쁘진 않습니다.
    \vspace{5mm}

    시험 전날은 10시에 무조건 주무세요. 화, 수요일에 일찍 일어났다면 잠이 안 올 리가 없습니다.
    무조건 자야합니다. 그렇지 않고 11시 넘어간다... 새벽 1시, 2시까지도 못 잘 수도 있습니다.
    시험날 발휘해야 할 정신력과 체력을 이 때 낭비하는 불상사가 벌어질지도 모릅니다.
    이런 경우라면 차라리 더 일찍 자서 당일 새벽 4시, 심지어 3시에 일어나는 게 나을 수도 있습니다.
    일어나서 문풀하고 시험 리허설 치고 수능장에 가는 게 나을 수도 있어요. 올빼미들은 일어난 다음 한참이 지나야 컨디션이 좋아지니까요.
    수능시험이 끝난 뒤에 컥챗 와서 나 인생 어떡해 그러지 말고, 시험 종료가 되면 탈진 상태에 빠질 수 있도록 세팅해놓으시길 바랍니다.
    \vspace{5mm}

    \item 음악의 장악
    \vspace{5mm}

    벌써 암욜맨 링딩동이 떠돌아다니고 있던데. 그보다도 아침마당 bgm이 더 위력적이죠.
    \vspace{5mm}

    정 안 된다 싶으면 평소에 듣던 음악을 틀어놓으시길 바랍니다.
    그리고 수요일까지는 음악 중에서도 우울하고 슬픈 걸로 가시면 됩니다.
    활기차고 긍정적인 음악을 들으면 오히려 뇌에서는 이런 데 쓸데없이 호메오 스타시스를 발현해 우울해지려고 하는 경향이 있지만,
    슬프고 우울한(한편으로 차분한) 음악을 들으면 뇌에서는 긍정적으로 변하려고 하는 경향이 있습니다.
    음악이 대리 비관을 해주기 때문에 역설적으로 우울한 게 사라지는 것이죠.
    \vspace{5mm}

    단, 노래가 들어간 음악은 절대 안 됩니다.
    게임 bgm이나 military music 같은 것이 좋습니다.
    \vspace{5mm}

    물론 가장 좋은 건 음악 자체를 안 듣는 것입니다. 하지만 그게 힘들면 그나마 나은 걸로 덮어씌우란 것입니다.
    \vspace{5mm}

    \item 수험사이트
    \vspace{5mm}

    올해 망했다고 다른 친구들 발목잡는 사례들 벌써 보입니다.
    '개스랙'이라고 욕한사발 하고 접속하지 마시길 바랍니다.
    \vspace{5mm}

    \item 잠재력의 발휘
    \vspace{5mm}

    내 무의식이 문제를 풀 수 있다라는 신뢰를 보여주길 바랍니다.
    실제로 문풀에 있어서 더 중요한 건 무의식입니다. 문제를 푼다는 건 엄밀히 말해 뇌에서 알아서 풀도록 우리가 '잘 읽는' 것입니다.
    문제해석을 정확히 한다면, 그리고 문풀절차만 지킨다면 문제는 저절로 풀립니다.
    공부를 했음에도 문제를 못 푸는 건, 당사자의 뇌가 스스로 문제를 풀 수 있는 단계까지 공부가 되지 않아서이기도 하지만
    무엇보다도 본인이 문제를 잘 해독하지 않아서 그렇고, 그 다음으로는 절차를 안 지켜서 그렇습니다.
    해석과 절차 이게 안 되면 10년간 공부해도 수능은 절대 안 됩니다.
    \vspace{5mm}

    우리의 의식이 이렇게 100$\%$ 안전성을 보증하면 무의식은 편하게 발동함으로써 문풀 아이디어를 쏟아냅니다.
    거꾸로 말해서 우리가 부정확하게 문제를 읽거나 절차를 안 지키면, 무의식은 발동하긴 커녕 사려버립니다. 그래서 문제가 안 풀리는 겁니다.
    \vspace{5mm}

    어차피 수능시험은 10문제 중 9문제는 님들이 아는 문제, 1문제가 모르는 문제입니다.
    9문제 중 3문제는 그냥 푸는 문제, 3문제는 실수해서 망할 수 있는 문제, 나머지 3문제는 살짝 꼬아낸 문제입니다.
    현재 출제 경향으로는 누구든 저 1문제에서 당혹스러워하긴 마찬가지입니다. 여기선 누구나 출발선이 똑같습니다.
    아울러 실수할 수 있는 3문제, 살짝 꼬아낸 3문제는 시험당일날 맑은 정신으로 해석만 잘 하면 안전하게 대처할 수 있습니다.
    \vspace{5mm}

\end{enumerate}


이제는 참고서 볼 때가 아니라 위 1$\sim$5를 정확히 준수하실 때이니 잘 지켜주시기들 바랍니다.
믿건마말거나인데 구석기 시절 제가 수능을 칠 때, 전 그 추운날 시험장에서 반팔로 조깅을 했고
수돗가에서 일부러 머리를 감고 세수를 박박한 뒤 시험장에 들어갔습니다. 그렇게 해서 낯선 곳과의 기싸움은 선방했죠.
\vspace{5mm}

물론 저러란 이야기는 아닙니다만(....) 또 생각나서 적는다면
옷 두껍게 입지 말고 반팔+얇은 상의+얇은 외투. 이런 식으로 여러겹 입고 가시고 양말도 두겹 신고가시길요.
난방에 따라서 추워질 수도 있고 더울 수도 있으니 이거 가감해야할 것입니다.
그런데 가장 골치아픈 게 발이 추운 건데. 발은 따뜻할 수록 좋으니 양말 2겹 신고가는 건 매우 권장할만한 일이라고 하겠습니다.
머리는 차갑게, 발은 뜨겁게.
\vspace{5mm}




\section{검증되었네요.}
\href{https://www.kockoc.com/Apoc/479357}{2015.11.12}

\vspace{5mm}

어제와 오늘 새벽에 계속 대화하고 상담하고 답변하면서
무엇이 옳은 공부법인가에 대해서는 어느 정도 검증이 완료된 것 같습니다.
자뻑으로 들릴 수도 있는데 제가 예측한 것이나 충고한 건 거의 맞아떨어졌습니다(그래서 말을 삼가야겟습니다. 이 때가 위험하니까)
\vspace{5mm}

1년 정도 지나면서 그 공부법을 실천해서 성적을 올리거나
목표를 이루지 못 했더라도 전진한 케이스들이 있으니 이제 이 분들이 '영리목적과 관계없는 순수한 학습공동체'를 일궈나가시면 되겠지요.
\vspace{5mm}

몇몇 분들이 돈과 관계없이 왜 이런 걸 상담해주느냐에 대한 답변은
\textbf{루저가 패자부활전 할 수 있는 올바른 방법을 확인하고 정착시키는 것}이 제 인생 목표 중 하나이기 때문입니다.
그리고 이런 걸 악용해서 돈을 버는 사람들을 매우 한심하게 생각하고 있기 때문입니다. 수험생을 위하는 척 하면서 곤경에 빠뜨리는 자들.
\vspace{5mm}

상담 질문글이 있는데 다는 답변해드리기 어렵고 한꺼번에 글로써 정리해 올리겠습니다.
사실 그 이야기가 다 그 이야기이고,
어제 시험에 응시해서 썰을 푸는 분들의 글이 훨씬 도움이 될 것입니다.
\vspace{5mm}

검증된 것
\vspace{5mm}
\begin{enumerate}
    \item \textbf{인강 좆까, 양치기 만세}
    \item \textbf{일지$-$상원 시스템의 효용성}
    \item \textbf{시험 일주일 전 컨디션 관리의 효과}
    \item \textbf{실모 무용성}
    \item \textbf{뇌를 믿어라.}
\end{enumerate}
\vspace{5mm}

작년에도 비슷한 주장을 폈고 반론이 많았으나, 올해 콕콕 수험생들로 이건 확인되었네요.
특히 5번은 그걸 항의하는 사람조차도 본인이 철칙을 어긴 게 과학적(...)으로 밝혀져버렸습니다.
\vspace{5mm}






\section{장사철과 광고시즌 시작이군요.}
\href{https://www.kockoc.com/Apoc/489593}{2015.11.15}

\vspace{5mm}

속을 사람이야 속겠죠. 사실 그걸 뭐라고 할 수도 없겠죠.
4점짜리 어떻게 푸느냐 그러신 분들 많을 건데
\begin{itemize}
    \item[$-$] 마플
    \item[$-$] 쎈 등의 개념서
    \item[$-$] 일품, 라벨 등의 좀 수준높은 문제집
\end{itemize}

이걸 다 풀어보고, 그 다음 교과서 읽어보시면서
'문제해결의 전략'이란 것을 쭉 생각해보면서 그걸 풀어보는 훈련 하는 것 빼곤 답이 없습니다.
\vspace{5mm}

적중적중거리는데 정작 그런 식으로 100$\%$면 마플이겠죠(별 문제들이 다 들어가있으니)
자꾸만 무슨 적중거리는데 그 논리면 적중 안 하는 문제집들 없고, 설령 그런 것 산다고 해도 공부 안 하면 아무 소용없으며,
무엇보다 그런 식의 말도 안 되는 적중 기준으로 따져도 적중 안 된 실모들이 훨씬 많으니까 내년에도 상술에 좀 휘말리지 마십시오.
\vspace{5mm}

실모야 보충용으로 푸는 건데 지금은 정말 무슨 인디아나 존스의 성궤, 성배처럼 숭상시된다는 게 문제입니다.
그럼 너는 마플을 왜 권하느냐 할 건데 간단합니다. 가성비가 가장 좋은 기출문제집이고 단점은 '분량이 너무 많다' 정도여서입니다.
\vspace{5mm}

그리고 광고하는 사람들은 성공한 학생들만 보지 말고 실패한 학생들부터 좀 돌아보시죠.
뭘 자기 것 봐서 잘 나왔다 그러시는 분들도 계시는데 그거 일부인 것 아시죠? 5명이 성공하면 45명이 실패입니다.
교재가 많이 팔렸다는 건 그만큼 그 교재 고객들 실패한 사람도 더 많아진다라는 걸 이야기하죠.
\vspace{5mm}

교재추천 함부로 안 하는 이유가 이딴 식으로 '교묘한 상술전략' 같은 게 있어서입니다.
그 분들은 이런 데 홍보하지 말고 다른 데서 좀 홍보해도 되지 않습니까?
이 사이트는 비영리성 분명 강조한 곳일텐데요. 영리적인 까페는 탑라인 가시면 되지요.
\vspace{5mm}

사이트 일궈온 사람들은 영리적으로 교재 홍보하는 사람들이 아니라, 그것과 무관하게 자기 공부 고민 털어놓고
이걸 극복하고자 한 사람들입니다. 장사하시는 분들이나 특정 교재 좋다라고만 하는 분들은 좀 자중했으면 좋겠습니다.
\vspace{5mm}






\section{문제 평}
\href{https://www.kockoc.com/Apoc/489714}{2015.11.15}

\vspace{5mm}

어떤 교재가 적중했더라... 는 건 적어도 제 기준에서 보면 없습니다(그 기준대로라면 기출은 100$\%$ 적중이겠죠)
올해 A, B형 킬러문제라는 것을 다시 풀고 훑어보고 정리해보았는데
홍보하느라 떠들석한 곳과 달리 네이버 블로그들 검색만 해봐도 벌써들 괜찮은 풀이들 올려두고 알아서들 하네요
\vspace{5mm}
\begin{enumerate}
    \item A형 30번
    지수와 로그 같지만 실제로는 '부등식의 영역'(고1 수학) 문제입니다.
    \vspace{5mm}
    
    \item B형 21번
    역함수 미분, 그리고 그래프에 의존하지 않고 함수의 방정식화로 근을 구해 케이스 나눈다는 점에서 역시 고1 함수 문제입니다.
    \vspace{5mm}
    
    \item B형 29번
    14년도보다는 그래도 나아진 벡터 문제입니다. 이건 너무 오소독스해서리, 장점이 뭐냐면 14년도와 달리 단면화 찍기가 안 통했어요
    \vspace{5mm}
    
    \item B형 30번
    이 문제야말로 패턴, 업자들 풀이로 가면 힘들었을 거라고 생각합니다. 함수의 '설계'에 관한 문제이니까요.
    \vspace{5mm}
\end{enumerate}

간략히 말해서 새로운 건 없어보이긴 하지만 중요한 것이 없지 않네요.
고1 수학이 제대로 안 되어있으면 저 문제들 중에서 29번 제외하고는 풀기 어려울 수도 있습니다.
그런데 제가 아는 한 다수가 고1 수학을 대충 넘기고 이상한 스킬이나 특정한 풀이 같은 데 집착하다가 수능 뜨면 걍 죽어버리죠.
킬러 문제들은 패턴이고 스킬이고 안 먹힙니다. 원초적으로 문제에 쓰인 모든 조건들을 철저히 분석하고
연상할 수 있는 모든 기본 개념들을 동원하면서 모형을 설계해나가야합니다.
\vspace{5mm}

A형 30번의 경우도 범위 잘 잡아서 그래프들을 설계하고 그걸로 그림 잘 그린 다음
문제의 목적이 최단거리임을 알면 풀 수 있었습니다. 다만 숫자들이 기죽이는 게 있어서 격자점만 기대한 친구들은 나가리났겠죠.
B형 21번은 이거, 사실 \textbf{국어문제}입니다. 개인적으로 이 문제가 좋다고 보는 이유가 그것입니다.
수식적 표현을 의미적으로 파악하는 훈련이 되어있으면 바로 풀지만, 그게 안 되어있으면 패턴스킬질이나 하다가 안드로메다로 갔겠죠.
B형 29번은 사실 14년도에 나왔어야하는 문제라고 생각합니다. 이건 굳이 평할 건 없겠고
B형 30번도 매우 좋은 문제인 게 특정 논점에 치중한 스킬 풀이가 아니라, 논점들에 연연하지 않고
주어진 조건을 가지고 모형을 잘 만들고 검증해보는 접근 방법으로 가면 아주 쉽게 풀릴 수 있다는 것이었습니다.
\vspace{5mm}

다만 이런 것들은 학교든 학원이든 과외든 배우기 힘들 것입니다요.
특히 B형 30번은 보면 볼수록 좋다는 평이 나올 수 밖에 없는 게, 제대로 탈패턴화 아니면 정말 힘든 문제여서입니다.
본인이 문제에 현혹되지 않고 논리적으로 다양한 논점들을 자유롭게 논할 수 있어야 풉니다.
\vspace{5mm}

기출에서도 굳이 안 나온 거라면 30번 정도일 겁니다, 비슷한 문제는 올해 6평 30번 정도.
내년부터는 이 30번 비슷한 것들 $-$ 즉 단원에 얽매이지 않는데 학생들의 수리모형 만들기를 강조하는 문제가 나올 수도 있단 생각이 듭니다.
\vspace{5mm}

그럼 저걸 대비할 수 있는 교재가 있나.... 글쎄요.
함수를 직접 만들고 설계한다라는 일종의 함수모형화에 관한 직접적인 교재는 발견하기 힘듭니다. 간접적으로 제시한 건 있을지 몰라도.
(어째 그런 교재가 없지는 않다는 생각인데 좀 찾아보아야겠습니다)
고1 수학부터 바탕이 철저히 되어있고, 본인들이 직접 문제출제 같은 걸 해본 사람은 잘 풀었을지 모릅니다.
30번 자체가 국어로 치면 일종의 작문을 요구하는 즉, 논술적인 문제이기 때문입니다.
\vspace{5mm}

저런 문제가 나오면 '맨땅에서 어떻게 풀 수 있을까' 그걸 고민하는 게 중요합니다.
사실 맨땅 문제죠. 특별한 스킬이나 암기사항이 필요한 문제들은 전혀 아니었으니까요.
A형 30번이 요구한 건 그래프화 $-$> 케이스 나누기 $-$> 부등식의 영역이었고
B형 21번은 함수 그래프의 해석에다가 역함수 미분
B형 29번은 그냥 벡터 조작
B형 30번도 등장한 논점들로만 치면 깊은 건 없음.
\vspace{5mm}

어차피 이걸로 썰 풀기는 그렇고
공부들 하시려면 기출 신판 나오면 뭐 빨랑 정리하고
시중교재들 다 풀건 푸시는데
그 다음에 수리논술 문제들 한번 건드려들 보시기들.
\vspace{5mm}

현실적으로 인강 고르시겠지만 (추천해달라 가르쳐달라 댓글은 삼감)
패턴화, 유형화되는 건 피하시고 구태의연한 것도 거르시고 올해 30번 같은 걸 잘 대비해주는 것 고르시면 되겠죠.
\vspace{5mm}






\section{고1수학 투자하시기들 바랍니다.}
\href{https://www.kockoc.com/Apoc/490501}{2015.11.15}

\vspace{5mm}

일전에 고1수학이 중요하다고 한 글을 쓴 적이 있습니다만... 뭐 그건 예측이 맞다할 게 아니라 이거 너무 당연한 겁니다.
\vspace{5mm}

원래 고1 수학이 수능 범위에 안 들어가는 이유는
이게 '총론'이기 때문입니다.
7차 교육과정에서 행렬, 지수로그, 수열, 미적분, 확통, 기하는 "각론"이었죠.
\vspace{5mm}

총론은 고교수학 전체를 가로지르는 공통적인 내용,
각론이야 각 분야의 지엽적이면서도 개성있는 내용을 말하는 바입니다.
총론이 제외된 건 이건 '각론'의 내용으로 포함시켜 출제하겠다는 것이지요.
\vspace{5mm}

개정과정에서 수2는 문과수학 범위지만 이과수학 범위는 아닙니다.
그런데 착각하지 말아야 할 건, 수2가 이과수학에서 안 나온다는 게 아니라 '총론적'으로 출제된다는 것,
즉 다른 각론에 포함되어서 출제된다는 이야기이지 아예 안 나온다는 건 아니죠.
\vspace{5mm}

최근 3년간 수능출제에서 특기할 것.
$-$ 직관수학 멸종 : 직관수학으로 해결한다거나 하는 거 신기하게도 소리없이 사라졌습니다.
$-$ 스킬필요 급감 : 행렬이 사라지기도 했지만 케일리나 로피탈 논쟁도 없습니다.
$-$ 연속된 통수 : 국어나 영어도 그렇다 치고 수학도 킬러는 잘 보면 고1수학. 문과 수학 30번은 격자점 냈다가 \textbf{부등식의 영역}.
\vspace{5mm}

통수에 대비하려면? 그러니까 시중에서 말하는 출제경향대로'만' 공부하면 안 됩니다.
평가원이 이걸 정확히 알고 있기 때문에 저격질하는 것이지요.
\vspace{5mm}

이름난 실모들만 가지고 특정경향만 공부한 A
교과서나 시중교재로 전범위 다 기본을 공부한 B
물어볼 것도 없이 B가 유리합니다.
\vspace{5mm}

하도 문과 수학 30번이 어렵다라고 해서 오후에야 자세히 보았습니다만.
감상 3가지
\vspace{5mm}

\begin{enumerate}
    \item $-$ "고 1 수학 정말 공부 안 하셨구나"
    \item $-$ "A형 수학도 변별력 갖추긴 어렵지 않겠네"
    \item $-$ "갓가원과 EBS가 사설 능가했구만"
\end{enumerate}
\vspace{5mm}

지금 EBS 비난하는 사람들 없고, 올해 시험 이후로 평가원이 물수능냈다고 하던 입공부들 싹 들어가버렸죠.
\vspace{5mm}

바로 시작하시는 분들은 범위 상관하지말고 고1수학 철저히 하시고
특히 고1 수학 어려운 문제, 사정없이 풀어버리길 바랍니다.
\vspace{5mm}

공부하다보미면 위에서 말한 각론 $-$ 즉 지수로그함수미분적분확통 그런 것들은
결국 고1수학에 뒤집어씌우는 스킨에 불과하다라는 걸 알게 되실 겁니다.
\vspace{5mm}






\section{ebs 인강을 완강하고 사설 들으세요}
\href{https://www.kockoc.com/Apoc/491239}{2015.11.16}

\vspace{5mm}

지금 도저히 공부할 방법 모르겠다 하는 분들은요.
\vspace{5mm}

\textbf{그냥 ebs 들어가서 강의 따라가십시오.}
\begin{enumerate}
    \item 공짜이며 다운받을 수 있다.    
    \item 선별수강 가능하고 환승해도 된다   
    \item 강의력이 이미 일부 강의는 사설을 능가한다.   
\end{enumerate}


요즘은 안 듣습니다만 한참 인강 연구(?)할 때    처음에는 사설을 들으면서 오 거렸는데   나중에 EBS를 들으면서 오오오오오옷 거렸고 그 때 내린 결론은 '사설 함부로 듣는 게 아니다'라는 것이었습니다.   사설강의 뭐 들을가.   본인이 EBS 강의 완강이라도 해보고 고민하시길 바랍니다.   어떤 강의 듣느냐가 문제가 아니라, 자기가 수강한 강의 \textbf{'완강'을 할 수 있느냐가} 관건입니다.   그리고 인강의 문제는 딴짓입니다. 35분 강의 한 15분 들으면 살짝 쉰다고 웹서핑하다가 30분 날려먹고 그러고 있죠.
인강 듣지 말라는 이유 중 하나가 사실 이것 때문입니다.
이걸 막고 싶다?
\vspace{5mm}

파일 다운받은 뒤에 인터넷 끊고 보거나, 아니면 맛폰으로 옮겨서 보시길 바랍니다.
\vspace{5mm}

그리고 어느 강사가 좋냐.
그런 질문은 하지도 말고 사실 그런 것 따지지도 마세요.
그런 것 따지는 친구들이 성공하는 걸 본 적도 없습니다.
\vspace{5mm}






\section{수험판의 세뇌}
\href{https://www.kockoc.com/Apoc/491558}{2015.11.16}

\vspace{5mm}

교재와 인강 추천 글이 콕콕에 함부로 올라오지 않길 바라는 궁극적인 이유는.
제 눈에는 최소한 이 판은 제정신이 절대 아니란 겁니다.
\vspace{5mm}

인터넷 없이 동네서점에 나온 교재 꾸준히 공부하면 목표성취했을 친구들이
괜히 꿀교재 꿀인강 찾는다고 서핑질하다가 가랑비에 옷 젖듯이 특정 교재, 인강의 광신도로 전락해버리는 걸 많이 본다 그거죠.
\vspace{5mm}

작년부터 적지않게 상담은 했고, 콕콕 내에서 아니 저 대머리 늙은이는 왜 이렇게 사람을 단정지어... 그러실 건데
그런 이유는 간단합니다. 사람들이 서로 다를 것 같지만 '실패'하거나 '하류'로 전락하는 보편적인 패턴이라는 건 존재하고
그 중 하나가 저 \textbf{'세뇌'}입니다.
\textbf{용서할 수 없는 사실은 그 세뇌를 하는 자들은 많이 벌어들이고 있으며}
\textbf{그런 세뇌를 고발하는 것 자체를 일체 차단하려한다는 것이죠.}
\vspace{5mm}

만약 상담을 하려면 이런 질문이 와야죠
\textbf{"생1을 50 맞으려면 어떻게 해야해요?"    "올해 수능 B형 30번을 시간이 걸리지만 안정적으로 푸는 방법은 무엇일까요?}
\vspace{5mm}

그런데 정작 질문은
\begin{itemize}
    \item[] \textbf{"$\sim$ 교재 좋아요?"}
    \item[] \textbf{"$\sim$ 인강 듣고 싶은데 어떡해야하죠?"}
\end{itemize}
\vspace{5mm}

\textbf{....}
\vspace{5mm}

수험사이트에 들어가지 않은 사람들이라면 저런 질문을 하지 않겠죠.
본인들은 부인하지만 결국 나쁜 정보에 노출됨으로써 그 자본의 먹이가 되기 시작한 겁니다.
자본주의 나쁘다 싫다 노오력해보았자 뭐하냐 가서 시위하자? 다 좋은데 한마디로 웃기고 있네요입니다.
본인들이 이미 상품의 노예이고 광고에 계속 영향받고 있는 것조차도 해결 못 하면서 뭘 자본주의를 극복한다는 거예요?
\vspace{5mm}

제가 해주는 상담의 요체는 재미없습니다.
비만환자들에게는 "기름진 것 덜 먹고 운동을 하라"라는 재미없는 조언이 유일한 해결책이겠죠.
마찬가지입니다. 공부환자들에게는 "걍 검증된 교재나 보고 인강 줄이고 문풀 오답 정리 충실히 해라"가 되겠습니다.
물론 양치기를 해도 안 되는 특수한 케이스도 있습니다. 이런 케이스는 그런데 그럴 수 밖에 없는 이유란 게 있더군요.
\vspace{5mm}

세뇌를 느끼면서 답답한 건 이것이죠.
공부는 열심히 합니다, 그런데 아무리 보아도 제가 보기에 성적이 안 오른 건
그 사람들이 마케팅에 세뇌당해 구입하게 된 '교재 탓'이 크거든요.
비만 환자들에게 야, 네가 지금 우걱우걱먹고있는 햄버거부터 스랙통에 넣으라고 하는 게 속시원한 해결책이지요.
하지만 이렇게 하면 다들 화내거나, 뺏기기 싫어서 더 처먹기 시작하겠죠.
\vspace{5mm}

방금도 인터넷을 서핑해보니.
역시나 아니나다를까 또 호구들을 낚기 위한 \textbf{밑밥깔기 언플이 시작된} 모양이더군요.
내년에 망하기 싫은 분들은 콕콕에 올라온 성공기든 실패기든 진짜 수험을 치열하게 하신 분들께 직접 여쭤보길 바랍니다.
\vspace{5mm}

그럼 이렇겠죠. "그럼 너는 왜 여기서 찌질대냐? 공개적으로 얘기하지"
저야 이렇게 답하죠. "어차피 호구들이 그런데 낚여줘야, 즉 깔아주는 사람이 있어야 님들이 올라갈 수 있음"
\vspace{5mm}






\section{중하위권이 희망이 없을 리는 없고}
\href{https://www.kockoc.com/Apoc/492004}{2015.11.16}

\vspace{5mm}

뭐 그렇게 여겨질 수도 있사온데 적어도 제가 관찰한 바는 그럼.
중하위권이 노력을 해도 안 되는 경우는 머리가 나빠서가 아니죠. 여러가지 이유가 있습니다.
\vspace{5mm}
\begin{enumerate}
    \item ADHD, 공황, 인지 장애가 있는 경우 : 문제는 이걸 본인과 가족도 모른다는 겁니다.
    \item 공부한 걸 자기 것으로 만들지 못 한 경우 : 이것도 복합적 문제로 나뉩니다만, "생각하는 법"을 배우는 게 필요한 경우라고 해야겠죠.
    \item 학은 하는데 습(習)이 되지 않은 경우 : 보통은 업자들에게 낚여서 학은 하는데 습은 못 합니다.
\end{enumerate}
\vspace{5mm}

일지를 분석하든가 상담을 해보면
공부한 양과 질, 그리고 성적은 비교적 거의 정확히 일치합니다.
본인은 잘 모르겠지만 제 시각에서 보면 '잘 되는 이유'와 '못 되는 이유'란 건 존재한다는 것이죠.
\vspace{5mm}

다만 문제는 그거네요.
암세포도 생명인데... 라고 암세포에 정드는 일이 벌어진다는 겁니다.
시험지만 봐도 숨이 가빠오는 친구는 숨이 안 가쁘면 초조해하고
생각하는 법을 못 배운 친구는 생각하는 경험을 거부하는 경우가 많으며
한번 인강에 중독된 친구는 끝까지 인강만 따라가려 합니다. 그래서 문풀 경험이 부족하니 실전에서는 발리죠.
\vspace{5mm}

중하위권을 상위권으로 올리기 힘든 이유는
노오력이 부족해서도 아니고 머리가 나빠서도 아닙니다.
학습자 본인이 그 상태에 너무 친숙해져서, \textbf{변화라는 것을 거부}합니다.
즉, 기존의 자신을 바꿀 생각을 안 하는 것이죠. 고향 떠나기 싫다 그겁니다.
심지어는 스톡홀름 증후군도 아니고 그렇게 공부를 못 하거나 계속 실패하는 사람으로 걍 살다 죽겠다는 생각까지 하죠.
\vspace{5mm}

상담 청하는 분들은 부디 교재나 인강 그런 거나 질문하지 말고
자기가 바꿔야하는 것이나 청산히야 할 악습이 뭔지 그것부터 보시는 게 좋습니다.
그 악습을 고치지 못 하니까 \textbf{노력해도 안 됩니다.}
물론 노력을 해야만 그 악습의 정체가 비로소 드러납니다만요.
\vspace{5mm}






\section{+1수할 때 반드시 거쳐야 할 과정}
\href{https://www.kockoc.com/Apoc/493832}{2015.11.16}

\vspace{5mm}

입시 상담은 대부분 돈을 벌기 위한 수작인 경우가 많다는 거야 아시겠고
그런데 저는 그런 것보다는, 인간을 이해하고 그 인생에 개입해서 부정적인 걸 잡아내고 치유한다... 를 통해서
저 역시 인간을 탐구하고, 또한 그럼으로써 저 역시 제가 상실했다고 느낀 걸 치유하는 느낌을 받는다가 더 강합니다.
다만 제가 선택한 사람들과 대화하면서 깊이 들어가보면서 직설적으로 얘기할 때에는
이른바 감정이입이라는 걸 안 할 수가 없고, 뭔 중2병스러운 이야기냐 하겠지만 데미지를 입지요.
\vspace{5mm}

화두 중의 하나가 왜 장수생들이 피폐해지느냐입니다만.
\vspace{5mm}

원론적으로 말하면 장수생들은 잘 이끌어주는 구심도 필요하지만
1번 실패하면 사실 본인도 가늠할 수 없는 상처를 입고 그로써 온갖 성격적 변화를 거친다는 게 문제입니다.
콕콕 사이트가 이 점에서는 자부할 수 있는 게
아직까지는 남의 상처를 이용해서 장사를 하는 건 아직 없는 것 같고(홍보질은 강경하게 까고보는 것, 이거 유지되어야합니다)
그리고 정말로 공부한 사람들끼리 솔직히 사연을 털어놓기 때문에 힐링이 된다는 것이겠죠.
\vspace{5mm}

님들이 +1수를 고민한다면 이건 단지 다시 시작하는 게 문제가 아닙니다.
재수에서 삼수, 삼수에서 사수로 가면 반드시 상처를 입습니다. 실패는 실패이기 때문입니다.
그 실패한 이유를 분명히 직시하고 실패했다는 걸 인정하는 데도 사실 시간이 걸립니다. 하지만 이건 분명히 밟아둬야하는 과정입니다.
다시 시작하면 성공할 수 있는 가능성, 그리고 어떻게 하면 성공할 수 있느냐를 정말 제대로 계산하고 움직여야합니다.
즉, 실패의 인정과 고찰, 아울러 성공을 위한 과정의 치밀한 설계.
이게 없이 +1 수를 하면 좋은 결과를 기대할 수 없습니다. 왜냐면 치료하지 못 한 상처가 결국 곪아버리기 때문이지요.
\vspace{5mm}

교재를 추천해달라 인강이 뭐가 좋느냐.... 이런 질문을 하는 사람들 대부분은 \textbf{환자}들입니다.
반면 생각없이 무조건 교재, 인강 홍보해대는 사람들은 장삿꾼들이죠.
더 심각한 문제는 자기들이 정말 좋은 일을 하고 있다 착각하는 건데 절대 아닙니다.
나중에 얼마나 자기들이 나쁜 짓을 저질렀나 느끼면 죽고 싶어질 겁니다.
그들은 얼마나 많은 수입이 들어올까 즐거워하거나, 아니면 자기도 고소득자가 될 수 있다라는 것에 혈안이 되어있겠죠.
그냥 한마디로 제정신들이 아니죠.
\vspace{5mm}

상담하다보면 안 맞는 교재나 인강 구입하느라 아까운 시간 날리고 상처입은 케이스를 정말 많이 접합니다.
남들이 좋다고 하는 상품을 샀는데 성적이 그 따위로 나왔으니 이건 자기 탓이 아니냐고 자학하는 케이스들이 정말 많습니다.
이런 걸 접하다보면 정말 장삿꾼들에 대해선 좋은 감정을 품을 수가 없습니다.
+1수하는 사람들은 조금만 더 가면 '자살'까지 이를지도 모릅니다. 그래서 상담할 때에는(이것도 피곤해서 저도 어지간해선 피할 겁니다)
정말 조심스럽게 할 수 밖에 없고, 내가 그 사람이면 어디서 상처입었거나 실패했나 다 이입해보면서 스트레스를 안 받을 수가 없는 것이지요.
장삿꾼들은 한번이라도 자기들이 '자살까지 몇발자국 남은 학생들'을 가지고 장난질치나는 생각은 안 해보셨나보지요.
저는 얘기할 때마다 이 친구가 죽으면 어떡하나... 라는 걱정을 수백번은 해보았는데 말입니다.
\vspace{5mm}

+1수를 권유한 분들이 많습니다. 그런데 그걸 가지고 아마 오해한 분도 있겠지만요.
권유한 분들은 일지나 그간 행적으로서 제가 가능성을 보았기 때문에, 그리고 올해 실패했다면 \textbf{그 실패한 이유도 보였기 때문에} 그런 것입니다.
실패한 이유가 분명하다면 성공할 가능성은 높아집니다. 그 이유를 제거하면 되기 때문입니다.
그리고 사실 이 모든 상처를 극복하려면 일단 성공하는 수 밖에 없습니다.
제가 충고한 걸 듣고 바로 시작하는 분들은 \textbf{수능 보기 전 그 피말리는 기분}, 다시 살리시길 바랍니다.
1년이 늘어났다고(?) 해서 여유부리겟지만 그래선 안 됩니다. 11월 초에 느낀, 전신의 피가 메마르는 초조함을 다시 살려야합니다.
남들이 비아냥거리든가 가족이 눈치주는 거, 그거 아무 소용없습니다. 심지어 부모도 마찬가지입니다.
그 중 님 인생 책임져 주는 사람 아무도 없어요. 자기가 전신장애가 되더라도 끝까지 책임져 주는 사람이 아니면 무시해도 됩니다.
실패했을 때 비아냥대던 사람이 있으면 성공해서 다시 만나면 됩니다. 말 못 하고 억지로 칭찬하면서 속으로 부글부글대는 걸 즐기면 됩니다.
\vspace{5mm}

이 지겨운 수험을 더 해야하느냐. 이렇게 생각하면 안 되지요.
1년이란 시간을 더 가치있게 쓰느냐. 이렇게 생각하면 됩니다.
수험이 그럼 시간을 무가치하게 보내는 것일까요, 그건 아니지요.
자기가 진정 바라고자 하는 것을 노오력해서 현실적으로 도달할 수 있다면 그건 매우 가치있는 시간입니다.
더군다나 자기가 불가능하다고 생각하던 것을 이룬다면 그 뒤로 인생은 (당분간은) 매우 충만해집니다.
(왜 당분간이냐면 인간은 또 간사해서 언제 그랬다는 듯이 자만하고 다시 나태해지기 때문입니다)
\vspace{5mm}

+1수하시는 분들은 억지로 자기가 괜찮다라고 하지 마십시오.
올해 실패했다면 그 상처를 인정하십시오. 그래야 그 상처가 곪아서 자신이 잠식당하는 일을 막을 수 있습니다.
재수하는 사람들은 공부하는 종종 힘들 때마다 상당히 정신적으로 괴로워합니다. 자기가 실패자, 낙오자라는 생각 때문에요.
이걸 치유하는 방법은 "실패"를 긍정하는 것입니다. 실패를 제대로 했으니 그 반대로 나아가 성공도 할 수 있음을 차분히 바라보면 됩니다.
1년이란 시간을 가치있게 치열하게 보내면 시험 당일 \textbf{'뇌가 알아서 문제를 풀어주니'} 그걸 믿으시면 됩니다.
\vspace{5mm}

그러나 한편으로. 몇발자국 떨어진 곳에 있는 사신(死神)도 의식하길 바랍니다.
인간은 실패하면 모든 걸 리셋하기 위해 어리석은 선택, 즉 자살을 하는 충동을 받습니다.
자살하지 말기 위해 열심히 하라, 혹은 무한긍정하라 그런 이야기가 아닙니다.
\textbf{오히려 터무니없는 사건일수록 매우 익숙한 일상이 될 수} 있으니, 긴장하시라는 이야기입니다.
님들은 내년에 적어도 3$\sim$4번은 자살충동을 받을 것입니다. 다 포기하고 엉엉 울고싶다, 그냥 산에 들어가고 싶다 할지도 모르죠.
그런데 그건 냉정히 말하면, 공부하기 싫어하며 주인을 배반까지 하는 뇌의 '변덕'이기도 하고 '본능'이기도 합니다.
그럴 때가 오면 알아서 먼저 쉬거나 스트레스를 풀어주기도 해야겠지요. 시험볼 때까지 말 안 듣는 뇌와 계속 싸워야할 것입니다.
시간내서 자기가 내년 수능에 실패해서 안 좋은 충동을 받아 정말 자살했을 때... 를 한번 상상해보는 것도 나쁘지 않습니다.
그리고 왜 자기가 그런 안 좋은 충동에 휩싸였는지, 어떤 생활을 했는지 상상하면서 \textbf{미래를 과거형으로 복기해보는 것}도 권해볼만합니다.
\vspace{5mm}

1년 전과 달리 콕콕이 좋아진 것이 있죠. 이제 공부하는 사람들끼리 모여서 더 협조적으로 나아갈 수 있단 것입니다.
작년에는 일지를 일일히 체크해주고 그랬지만 지금은 그럴 필요가 많이 줄어들었습니다.
성공한 사람이든 실패한 사람이든 진지하게 자기 수험경험을 늘어놓고 조언해줄 수 있는 분들이 많이 늘어났습니다.
다시 시작하는 분들은 서로가 서로를 '관리'해주고 '따끔한 지적'을 해줄 수 있는 동료들을 찾으시는 것도 권해드리겠습니다.
사람은 자기에겐 한없이 관대하나 남에겐 엄격합니다.
본인은 시간관리를 못 하고 한없이 늘어져도 남의 공부에는 매우 엄격하고 현명한 관리를 해주지요.
이런 것을 이용하려면 타인과 손을 잡는 것도 매우 괜찮은 방법입니다.
아울러 상대가 괴로워하거나 스트레스 받는 걸 도와주다보면 자기의 문제도 간접적으로 해결할 수 있을 것입니다.
\vspace{5mm}

이제 내년에 어떻게 해야할지에 대해서 제가 눈여겨본 분들에 대해선 개입은 모두 한 것 같습니다.
나중에 나이를 많이 먹고 머리가 빠지다보면
합격보다도 이렇게 힘들게 노력한 순간이 더 인상깊고 가치있었다, 자기 인생이 바뀌는 중요한 시간대였음을 회고할 수 있을 것입니다.
저승사자도 공부를 열심히 하는 사람에게는 다가오지 못 합니다.
최소한의 학습량을 유지하면서 회독수와 문풀수를 늘리고 오답을 정리하면서 혼자 스스로 어려운 문제를 해결하는 '사고'를 하다보면
어느 순간 인간을 넘어 신과 하나가 된 듯한 몰입을 경험하실 것이고, 아 이게 바로 그것이구나를 느끼실 겁니다.
그 순간이 되어야 변덕스럽고 말 안 듣던 뇌가 비로소 제정신을 차리고 천상의 길로 나아가게 되는 것이지요.
\vspace{5mm}

제가 깊이 개입하면서 수험을 넘어 인생상담까지 간접적으로 하게 된 분들은 잘 될 것이라고  계산한 바입니다.
부정적으로만 흘러갔을지도 모르는 분들이 다시 바닥을 치고 올라가는 것을 보는 것은 매우 즐거운 경험입니다.
거액의 돈을 주고도 느낄 수 없는 희열감이지요.
\vspace{5mm}

그럼 모두 스타트하시길 바랍니다.
\vspace{5mm}






\section{교재 vs 놀이 하지 마세요}
\href{https://www.kockoc.com/Apoc/494964}{2015.11.17}

\vspace{5mm}

현대판 어부지리
\vspace{5mm}

\begin{itemize}
    \item[] 조개 : 야, 난 쎈을 풀겠어
    \item[] 학     : 야, 난 마플을 풀겠어
\end{itemize}
\vspace{5mm}

이러면서 vs 놀이로 3달 허송세월
\vspace{5mm}

\textbf{어부 : ㅄ들. 난 다 풀었지롱}
\vspace{5mm}

교재를 한정해서 골라야하는 건
공부량이 원체 많은 고시 공부에 한해서이고
수능은 사실 공부량이 많은 과목이 아닙니다요.
\vspace{5mm}

단권화는 \textbf{머리에 하는 것}입니다. 원래 단권화란 개념도 교재가 변변치 못 한 수십년 전에 먹히던 것이고
지금은 웬만한 교재들도 양이 풍부합니다.
\vspace{5mm}

그런데 다들 착각하는 건. 교재에 내용이 많다고, 그게 공부하는 님들 머리에 지식이 많이 들어간 게 아니란 겁니다.
양이 1000인 교재도 5번 보면 님들 머리에 300 정도 들어갈까 말까죠.
반면 양이 500인 교재를 30번 보았다면 님들 머리에 2000이 들어갑니다.
\vspace{5mm}

단권화는 머리에 하는 것입니다.
노트 정리? 그것도 노트정리를 하는 과정에서 머리에 들어가기 때문에 추천되는 것이지,
노트 잘 만들었다고 해도 시험 시간에 가지고 못 들어가는데요
\vspace{5mm}

교재 뭘 풀어야하나, 중복문항 피해야하지 않나.
\vspace{5mm}

까놓고 말해서 낭설입니다.
\vspace{5mm}

기출은 공부하려면 정말 개개 문항을 다 설명할 수 있을 정도로 반복해 풀어야합니다.
그 과정에서 평가원의 의도라는 것, 출제원리라는 걸 읽을 수 있습니다.
한번 풀었다. 그건 무의미하죠. 중요한 건 그 문제를 내가 설명할 수 있느냐, 심지어 변형출제까지 가능하냐는 것입니다.
변형출제 가능할 정도면 실모는 필요없습니다. 실모가 사실 기성문제 변형출제 수준이기 때문이죠.
\vspace{5mm}

그냥 닥치는대로 반복해서 풀고 암기하고 그게 공부의 시작입니다.
\vspace{5mm}

공부 잘 하는 애들이 교재 vs 놀이 하는 것 단 한번도 못 보았습니다.
물론 '업자'들은 그걸 조장하죠. 그게 마케팅이 되기 때문이죠.
심지어 어떤 업자들은 자기 교재만 보라고 하죠. 학생들이 다른 교재 보면 자기 교재가 형편없다는 게 드러나거든요.
\vspace{5mm}

교재 뭐가 좋냐.... 라고 하는 게 장수의 지름길입니다.
그러니까 쓸데없는 생각 말고 그냥 다 푸세요.
다만 개념서만큼은 그냥 현직교사들이 쓴 것이나 교과서 보는 걸 권합니다.
vs 놀이 굳이하려면 개념서일 건데 이상하게 이건 안 하덥니다만,
\vspace{5mm}








\section{죽음의 절벽}
\href{https://www.kockoc.com/Apoc/497307}{2015.11.18}

\vspace{5mm}

성공하기 위해 필요한 노력은 $10^10$  인데
지수 계산을 할 줄 몰라서 10 x 10 으로 계산하거나
1010으로 보는 것까지 그렇다 치고 그냥 10만 해놓고 난 노력했는데 안 된다는 게 보이죠.
\vspace{5mm}

우물 안 개구리라는 표현은 이런 데 쓰는 것이죠.
동네 뒷산만 본 사람 입장에서는 백두산, 한라산, 에베레스트산이 어떨지 감이 잡히지도 않을 테고
산골짝에서 개울만 본 사람이 바닷가에 간다면 순간 압도당하겠죠.
\vspace{5mm}

바다를 본 적이 없는 사람은 기껏 가본 호수가 넓은 줄 알고 난 노오력을 많이 했다 합니다.
그런 친구들은 \textbf{바다에 데려가보는 수 밖에 없겠죠.}
한번 정말 시간이 난다면 서울대 도서관이나 신림동 고시촌 같은 데에서 $-$ 가능한지 모르겠으나 $-$ 그 사람들 어떻게 공부하나 보세요.
그리고 그들이 몇년간 공부했는지도 확인해보시면 되겠습니다.
\vspace{5mm}

이래서 환경이 매우 중요하단 이야기입니다.
부모가 판검사교수의사변호사인 쪽이 평균적으로 공부를 잘 하는 건 유전이 좋아서가 아니라
그렇게 공부한 사람들 밑에서 자랐기 때문에 공부에 들이는 노력에 대한 눈높이가 매우 높기 때문입니다.
\vspace{5mm}

상담해보면서 느끼는 건 다들 들여야하는 노력의 양을 잘못 계산하고 있단 것입니다.
1010   이 필요한데 10X10 정도 해놓고 안 된다라고 하고 있죠.
\vspace{5mm}

수학만 보더라도
유치원 때부터 조기 교육 시작해서 고3때까지면 가히 10년 넘게 교육받고 스트레스 받습니다.
좋은 집안에 태어나든 아니든 그런 노오력은 기울여온 것입니다.
그래서 제가 \textbf{노오력 까는 병신들은 스랙이라고 외치는 겁니다.}
이 스랙들은 저 금수저들이 '노오력을 안 하고 부모에게 얻어먹기만 했을 것'이라고 소설쓰면서 자기는 노오력한다고 소설쓰거든요.
말씀드릴까요?
\textbf{금수저들일수록 더 노오력을 열심히 하고, 게다가 인성조차 더 좋습니다.}
그럼 흙수저는? 말로만 노오력한다고 해놓고 말초적인 유흥에 빠진 경우가 더 많고 인성은 더 개차반인 경우가 많았어요.
그런 주제에 나중에 이 사회가 개판 어쩌구 말로만 지껄이겠죠.
그런 사람들은 이 사회 싫으면 그럼 '괜찮은 나라'로 이민 걍 가지 왜 아직도 안 가나모르겠습니다.
\vspace{5mm}

격한 말투일수록 진실성을 담보하니 이게 선행이라고 생각하고 더 적지요.
\vspace{5mm}

부모들이 왜 자기 자식들 내신이 불리하더라도 좋은 학교에 보내려하는 줄 아십니까?
물론 허영심 없는 부모들이 없는 것도 아니죠. 그런 부모들은 실패합니다.
잔인한 진실은 그겁니다. \textbf{자기 아이들이 수준낮은 애들 만나서 타락하지 않길 바라는 겁니다}요.
그럼 이 이야기에 풀발기하겠죠. 네가 흙수저와 서민 까고 있니?
\vspace{5mm}

집 밖에 나가서 근처 유흥가에다가 법망을 아슬아슬하게 넘어서는 업종 장사하는 생산자나 소비자들이 어떤 계층인지 보시죠.
물론 금수저들도 비밀스러운 소비를 하고 있죠. 그러나 규모로 치면 전자에 비할 바는 아닙니다.
\vspace{5mm}

신림동 고시촌이나 노량진 가면 공부하는 사람들만 보지 말고 근처에 발달한 다른 사업을 보시죠.
학원과 서점과 독서실 뿐만 아니라 온갖 종류의 유흥업종부터 심지어 유전, 생식을 이용한 이상한 사업까지 가장 먼저 생기는 동네입니다.
자기들이 수험에 실패한 사람들은 절대 자기들이 실패한 이유, 솔직하게 고백하지 않습니다.
정말 절망적인 상황이 되어서야 지킬 자존심도 없어서 그 때서야 실토하죠. 하지만 그래도 '그런 사소한 걸로 왜 실패해'라고 부르짖죠.
그게 정말 '사소한' 걸까요?
\vspace{5mm}

공부하다가 조금이라도 딴길 가면 그거 수험 실패라고 생각하지 말아야 합니다.
수험 실패면 다시 극복할 수 있다라고 착각하거든요.
님들이 공부하다가 럴을 하거나 연애질을 하면 그 때는 \textbf{"저승사자"와 가까워지는 겁니다.}
나이 처먹으면서 느끼는 건 죽음은 늘 그림자처럼 우리 뒤를 따라다니고 기회를 엿본단 겁니다.
이 글을 쓰는 저도 언제 뒈질지 모르겠지만, 저도 살면서 느낀 게 정말 오래 살 것처럼 생각하던 녀석이 가버리는 경우도 많고
심지어 행복해보이던 사람도 갑자기 부고가 뜨는데 나중에 알고보면 그럴만한 숨겨진 사연이 있었단 건데
특히 수험은 자기를 대수술하는 과정인지라 이게 실패하면 인생포기=자살이라는 게 결코 농담이 아닙니다.
남들이 보기에는 저거 왜 죽냐... 그럴지 몰라도 본인들은 그렇게 못 느끼죠.
\textbf{조금이라도 일탈한다면 그건 저승사자와 키스하는} 것입니다.
\vspace{5mm}

수험은 그럼 재정의되죠\textbf{. 승천을 목표로 죽음의 절벽 앞에서 도움닫기.}
무슨 개뿔 중2병급 표현이냐... 할지 모르나 저건 제가 보기엔 진실입니다.
절벽에 떨어진다고 해도 죽지는 않겠죠. 다만 절뚝거리며 살 각오는 해야합니다.
\vspace{5mm}

공부하다가 다 때려치우고 걍 기술이나 배우고 일이나 하자... 라고 말로만 그러지말고
그럼 일주일간 알바 뛰어보고 해보세요. \textbf{엿같아서 공부하는 게 낫다}는 걸 하루만에 느낄테고
그런 걸로 일할 노력이면 걍 공부해서 좋은 데 갈 수 있다, 내가 게을렀다라는 걸 일주일만에 납득할 겁니다.
\vspace{5mm}

운명을 바꾸는 건 매우 힘들다라는 건 동의하실 겁니다.
그런데 왜 그럼 공부를 힘들게 하는 건 싫어하시죠?
힘든 공부일수록 운명을 바꿀 수 있단 이야기인데?
\vspace{5mm}

+
\vspace{5mm}

최근 심각하면서도 재밌게 보는 게 경제의 지리학인데.
\textbf{사는 데 따라서 인생이 달라진다}는 것.
강남 강북 집값 격차가 커진 게 자본가의 음모?
너무 구태의연한 이야기죠. 오히려 그건 사는 사람들의 교육, 생활습관, 그리고 외모까지 달라졌기 때문입니다.
\vspace{5mm}

+
\vspace{5mm}

쉬운 예 들어드릴까요?
\vspace{5mm}

남자들이 헤벌쭉하는 흔한 요조숙녀.
그런 요조숙녀를 양성하려면 부모가 사자돌림이거나 대기업 중역 이상이어야 하고
사는 곳은 서초강남송파목동이나 여의도는 되어야 하며
학교는 정말 때묻지 않은 명문교여야하며
부유한 교회, 성당, 사찰. 그리고 먹는 것이나 의료서비스도 최상의 것으로 받아야 '완성'되죠.
\vspace{5mm}

여자를 완성시키는 건 비단 성형수술만이 아닙니다. 결국 환경이죠.
그런데 그런 환경은 그 부모나 조부모들의 '노력'의 결과죠.
그런데 남자들은 이런 여자들을 너무 날로 차지하려고 하죠.
\vspace{5mm}

집값이 비싼 곳은 저런 요조숙녀가 많습니다.
반면 집값이 저렴한 곳은 해맑은 여자애들도 때가 되면 술집에서 일하거나 못난 남자 만나 폭행당하고 있죠.
\vspace{5mm}

+
\vspace{5mm}

그럼 언제까지 노력?
노력 자체가 즐거워지는 순간요. Runner's High가 올 때까지.
이번에 액상탄마님이 인증한 것에 오, 잘했구만 했지만 아직 부족하다 느낀 게 그거예요.
아직은 노오력 자체에 쾌감을 느끼지 못 했군. 즉, 열심히는 하지만 결국 미치진 못 했군 정도.
더군다나 대화해보면 분명 1년 전과는 달라졌습니다.
1년 전은 걍 바바리안이었는데 지금은 조선시대 말까지는 왔죠.
그 때야 아이구 내 인생 그랬지만 지금은 몸에 좋은 보약을 챙겨먹으려는 적극성까진 띠고 있죠.
\vspace{5mm}

그러나 아직 수학적인 사고 $-$ 즉 합리적인 사고방식까지 체화된 건 아닙니다.
예를 들죠. 팀의 준에이스는 갔지만, '감독'까지는 못 되었다는 것이죠.
\vspace{5mm}

1년 공부해서 저 정도면 상당한 것이죠. 남들은 10년 넘게 공부해서 겨우 가는 수준인데 말입니다.
그러나
\vspace{5mm}

\textbf{1년 공부해서 저만큼 올라갔다는 건, 그만큼 또 빨리 추락할 수 있단 이야기죠.}
\vspace{5mm}






\section{문과 이과 전향에 대한 썰}
\href{https://www.kockoc.com/Apoc/499454}{2015.11.18}

\vspace{5mm}

여러가지 이야기가 있습니다만.
성공/실패를 떠나서 제가 드리고 싶은 이야기와 더불어 현실적인 얘기를 드리죠.
\vspace{5mm}
\begin{enumerate}

\item \textbf{도전 자체는 새로운 것일수록 좋다.}
\vspace{5mm}

저야 현실적인 사람이고 다른 사람들이 커리 제시하면 안 될 것은 아니라고 보지만
전과에 대해서는 그다지 비관론을 펴진 않습니다. 수험은 말 그대로 도박이기 때문이지요.
문과<이과 라고 알려져있지만 이건 사실 제대로 추궁 들어가면 근거는 없습니다.
왜냐하면 여러가지 이유로 어린 시절에 수학을 잘 못 해서 문과 갔는데 지금은 머리가 트여 이과수학을 잘할 수도 있기도 하는 등
중요한 건 본인의 적성, 취향까지 감안해서 얼마나 잘 맞느냐인데.
사실 이건 아무도 검증해본 적이 없기 때문입니다. 직접 부딪쳐서 도전해보지 않는 이상은 정말로 알 수는 없습니다.
\vspace{5mm}

그보다도 '실패'를 생각하고 도전 자체를 기피한다는 것 자체가 제가 보기엔 가장 위험한 것 같습니다.
실패할 걸 각오하고서라도 성공할 수 있는 걸 염두에 두면서 도전해보는 사람이 살아남지,
"아, 나는 되지 않을 거야"하면서 칼을 뽑기도 전에 \textbf{포기해버리는 사람은 차라리 죽어버리는 편이 낫습니다}.
제가 그 시절로 돌아간다면 (물론 제가 다시 수험생이 될 이유는 없지만)
떨어질 것을 알면서도 그 낮은 가능성을 높이기 위해 별의별 수작을 다 부리면서 노력할 것이고
실패했을 경우에 충격을 줄이고 그 실패조차도 자산화하기 위한 장치를 마련해놓지
실패하면 어떻게 될까 하는 걸로 자포자기하지는 않을 거란 것이죠.
\vspace{5mm}

위기가 기회라는 말은, \textbf{자기가 직접 부딪쳐보는 위기가 기회가 될 수 있다}는 겁니다.
내가 위기에 도전하지 않으면, 위기가 나를 찾아옵니다.
저는 다소 운명론자이긴 합니다만, 그보다는 어떻게 하면 운명을 바꿀 수 있을까에 더 관심이 많습니다만
사실 운명이란 것도 일종의 기호적 매트릭스라는 점에서 그건 고정되있는 동시에 고정되어있지 않다는 것.
그리고 목숨을 걸고 도전하는 것 외에는 운명을 바꿀 방법은 아무 것도 없다는 겁니다.
이제야 뭔가 저주 같은 것을 깨기 시작한 콕콕의 한 사람이 도망친 곳에 낙원은 없고 색안경 껴도 낙원이 아니라 했는데 맞는 말이죠.
도망치면 운명은 못 바꿉니다. 도망칠수록 운명은 공고해집니다.
\vspace{5mm}

\item \textbf{2. 그럼 문과 $\rightarrow$ 이과는 무모하기만 한가?}
\vspace{5mm}


그건 아닙니다. 일단 분량의 차이는 열심히 하면 극복할 수 있기 때문입니다.     우선 이과수학도 편한 건 있습니다. 제대로 틀을 갖추면 그 다음에는 학문적 시스템을 이용할 수 있다는 것입니다.   문과수학이 경우는 이종격투기와 같아서 쓸 수 있는 무기가 별로 없습니다. 격자점 나올 걸로 생각한 사람들이 30번 고등수학에서 말아먹었듯.   그러나 이과수학의 경우는 온갖 화력전이라 개살벌하지만, 본인도 여러가지 다채로운 무기를 사용할 수 있습니다.   즉, 최소 공부량을 확보하고 제대로 체계적으로 접근해서 이과수학의 여러가지 tool들을 쓸 수 있으면 문과수학보단 낫다는 것이죠.  
\vspace{5mm}

수리적 마인드나 설계 측면에서는 사실 별 차이는 없습니다. 이번 문과 30번이든 이과 30번은 "설계"를 할  수 있어야하는 문제였죠.   다만 문과수학의 경우는 평면좌표축과 정수론에 치중해있다면      이과수학의 경우는 공간좌표에다가 공간논리적 감각에다가 실수까지 확장된다가 차이 정도인데   이게 넘사벽인가... 하면 그건 아니라고 보고 있습니다.    오히려 사람에 따라선 이과수학을 제대로 공부해서 수학의 컴플렉스를 엎어버릴 수도 있다는 생각입니다.   
\vspace{5mm}

수학의 경우는 결국 제대로 공부하느냐 그걸로 결판나는 것이죠.   예컨대 문제유형만 외운다거나 무슨 실모의 적중을 따진다... 가장 어리석은 생각입니다.   수학 공부는 결국 자기가 기본적인 것을 철저히 하고 tool을 능숙히 다루면서 논리기하대수 사고를 통해   새로운 유형이더라도 침착하게 대비할 수 있는 준비를 해놓는 거지, 예상유형을 암기하고 정리해놓는 것이 아니죠.   사실 이런 방식이면 이과생이더라도 문과 30번은 못 풀었을 겁니다.   


\item \textbf{과탐은 어떠한가?}   

사실 과탐이 가장 문제입니다. 이건 다들 어렵게 내는데 어떻게 대비할지는 뚜렷이 안 가르쳐주기 때문입니다.   그러나 이건 이과생들이 과탐에서 겪는 고통이 문과생들이 과탐을 공부했을 때보다 덜하다는 건 아닙니다.   오히려 언어능력적인 측면에 있어서는 문과생과 이과생이 똑같이 과학을 공부한다면 문과생이 더 유리할 가능성이 높습니다.   과탐에서 난감한 게 바로 말장난을 까는 것인데 이건 이과생들이 약하기 때문입니다.   언어외적인 것 $-$ 즉 기하나 수리적인 과탐킬러의 경우는 사실 이과생들도 툴을 스스로 개발하거나 강의참조해야한다는 점에선 딱히.   


\item \textbf{그렇다면 왜 문과 $\rightarrow$ 이과는 어렵다 하는가?}   

절반 정도는 과장된 허풍이 있다고 여깁니다만, 세뇌론에서 얘기하듯 이과>문과라는 추상적 고정관념이 다수의 무의식에서 실체화되어서입니다.   그런데 흥미로운 건 이게 직접 검증되었느냐... 그건 아니란 것입니다요.   흔히 드는 예가 이과에서 문과 가니까 등급이 올랐다는 것인데 이게 변인통제가 충실히 된 건 아니죠.   이과수학을 공부하다가 문과로 간 경우야 당연히 더 많은 내용이 선행되었거니와 당사자가 공부기간이 기니까 그런 것이지요.   이걸 제대로 비교하려면 정말 문과생과 이과생이 똑같이 둘 다 모르는 내용을 동일하게 학습했을 때의 학습성과로 봐야합니다만.   가령 집합과 명제를 똑같이 친다고 하더라도 이과생이 더 유리하다고 확언할 수 있을지는 의문입니다(공부량과 강의가 동일하다 가정 하에)   
\vspace{5mm}

이과생이 우월하다고 하는 것이 원래 이과교육이 뛰어나서인지, 공부량이 정말 많아서인지 그것도 구분해야겠습니다만   이 역시 딱히 검증된 건 없다는 측면에서는 수험자본이 만들어낸 일종의 세뇌라고 생각하고 있습니다.   
\vspace{5mm}

그것도 그렇거니와 저건 평균적인 수험생들을 염두한 것이고   수험생 개개인으로 치자면 또 역시 일반화시키기는 어려운 문제입니다.   수험생 개인이 공부할 의욕이 많으며 공부하는 방법을 깨달았으며 올바른 커리를 밟는다면 이과로 전과한 게 문제라고 할 수 있을까요?   사실 결과는 알 수 없는데 처음부터 단언하는 경향이라는 게 있는 것 같습니다.   

\item \textbf{그렇다면 이과 공부는 어떻게 해야하나?}   

수리 감각이나 머리가 좋아야한다기보다도, 오히려 기본적인 교과서 개념을 더 철저히 암기하고 따져야하지 않을까요?   저 개인적으로는 오히려 문과수학이 감각을 요구하고, 이과수학일수록 이론적인 걸 더 많이 요구한다는 점에서   문과수학이 이과적이고, 이과수학이 문과적이라는 느낌을 받은 적이 많습니다.   
\vspace{5mm}

다만 이과수학 과정의 텍스트는 언어적으로 풀이되지 않았습니다. 그래프와 수식으로 요약되어있죠.   만약 문과생들이 이걸 타당하고 합리적인 텍스트로 풀이하면서 이야기하듯 공부해나간다면 이거야말로 올바른 방법이 아닐까 싶더군요.   왜냐면 이과생들이야말로 문과적 소양이 없다보니 그런 텍스트 풀이를 하지 않고 생각없이 암기하다가    좀 심화된 출제로 나오면 대비하지 못 하는 경우도 많기 때문입니다.   
\vspace{5mm}

수학공부에 있어서 중요한 건 어떻게 푸느냐 보다도,   자기가 공부하는 내용이 어떤 맥락에서 어떤 의미를 갖느냐를 정확히 이해하고 암기하는 것인데   이런 측면에서는 문과생들이 더 잘할 수 있다고 보고 있습니다.   다만 문제는 문과에서 그런 올바른 '리터러시'를 교육한다고 보장하지는 못 하는 것입니다만.   


전과하신 분들이면 현역 고3 올라가는 이과생과 비교하면 됩니다. 사실 큰 차이는 없을 테고    황금의 3개월간 공부하시면 능가하면 능가했지 부족하지는 않을 것입니다. 
\vspace{5mm}

나는 할 수 없어라고 생각하면 사실 아무 것도 바꿀 수가 없죠.   윗 글은 위로가 아니라 그간 제가 생각하고 분석한 바인데 아마 납득가는 분들이 꽤 있을 것입니다만   이런 것들을 따져서 어떻게 승부할 것인가를 계산하고 노력해야지   아, 그래도 이과는 무리야.... 라는 태도라면 사실 \textbf{앞으로 살면서 할 수 있는 건 아무 것도 없습니다}.   그리고 사실 우리는 남들이 그래주기를 기대해야합니다. 그래야 우리가 위너가 되고 남들이 루저가 되기 때문이죠.   
\vspace{5mm}

세뇌론을 쓰게 된 경위이기도 하지만, 자기가 운명을 바꿀 수 없다면    그냥 죽을 때까지 노예로 살 것인지, 아니면 정말 적당히 살다가 삶을 정리할 건지 진지하게 고민해야합니다.   아무 도전도 안 하고 꿀빨다보면 편히 살 것이다... \textbf{우리가 가만히 있으면 불행과 사고가 직접 찾아옵니다}.   \textbf{우리가 능동적으로 찾아간 위기는 기회가 되지만, 수동적으로 맞이한 기회는 위기가 되죠.}      

\end{enumerate}
물론 말이야 쉽죠.의지대로 저걸 바꿀 수 있는 사람은 경험상 20명 중 1명 꼴입니다.    그런데 그런 사람들이 결국 상류층에 올라가는 거지, 그저 도망가면서 공부하고 적당히 학교 잘 가면 풀릴 거야... 글쎄요.   그런 경우가 있기는 하던지요?
\vspace{5mm}

마인드 차이라는 건 참 사소하지만 매우 중요한 차이를 가져옵니다.   이걸 언어로만 보면 못 느끼는데 사람들을 만나보면서 그들을 비교해보면 뼈저리게 느껴요.   남들이 고정관념에 갇혀있을 때 자기는 그걸 두려워하지 않고 뭔가 감행한 사람이 그게 부정적인 것일지라도 뭔가 해내기는 합니다.   
\vspace{5mm}
+   여담적으면 그럼 과거의 그 훌륭한 이과천재들이 지금은 뭘하고 계시나라는 질문을 던지면 됩니다.  사실 다들 평범해졌죠.  사회에서 인정받는 천재(?)들은 정치가, 고위공무원, CEO 정도인데 이게 이과 공부와 관련이 있나 의문.    현재는 검증되었다고 볼 수도 있는 터라서.  사회적으로 성공한 사람들의 공통점은 뭐 설명할 필요도 없습니다.  

\begin{itemize}
    
    \item  첫째, 미소 짓고 노오력만 하는 척 하지만 실제로는 경쟁자들을 어떻게 짓밟을까 사악한 고민을 신나게 한 사람들이죠.  
    \item 둘째, 위에서 말한대로 도전하는 것에서 브레이크가 없었단 겁니다. 뭐 무모하게 하면 망했겠지만.  
\end{itemize}



12월 이전은 그래도 시간이 많이 남았으니 나는 해도 안 된다라거나 부모님 눈치 본다... 이건 글쎄요.  내년 10월 정도에 얼마나 쓸데없는 고민이었는지 뼈저리게 깨달으실 듯.  

\vspace{5mm}
++  그러고보면 수학 잘 하는 사람의 소위 현학적, 허세적 태도가 '넘사벽' 신화를 공고히 한 콘크리트였던 것 같은데.  이런 걸 뭐라고 해야할까나.



\section{n수하지말라는 것에 대한 이야기}
\href{https://www.kockoc.com/Apoc/501678}{2015.11.19}

\vspace{5mm}

맨큐의 경제학 앞에 빌 게이츠나 농구선수가
대학을 포기하고 현업에 뛰는 걸 얘기하면서 기회비용을 언급하죠.
만약 본인이 대학에 안 가거나, 그리고 대학을 그냥 그런 데 나와도 더 가치있는 일을 할 수 있다면
당연히 삼수 이상은 할 이유가 없습니다.
\vspace{5mm}

그런데 문제는 삼수의 기회비용이 삼수를 하지 않는 기회비용보다 더 높은 케이스이죠.
즉 \textbf{삼수를 안 하면 더 문제인 케이스가 생각 외로 많다는 것}이겠죠.
\vspace{5mm}

저야 이런 걸 독려해보았자 딱히 수익 같은 것이 생기지 않기에 이해관계와 무관하게 말할 수 있는데
\vspace{5mm}

대학 안 가고 기술 배우면 된다..... 그럴 사람이면 \textbf{진작 중딩 때부터 그래왔겠죠}.
삼수보다 더 나은 도전이 있으면 공무원 시험 정도.
그런데 이것도 경쟁률은 수능 저리가할 수준이라는 문제가 있죠.
\vspace{5mm}

기술 얘기가 나와서 그러는데 기술도 '남이 돈을 지불해야' 의미가 있는 것이죠.
과거에 제자 한명이 걍 목공술을 배우면 대학 갈 필요가 없기에 한마디 했죠. 그럼 "누가 돈을 주는데?", "......."
기술도 어떤 분야 기술이냐, 수요가 많으냐, 그리고 진입해오는 라이벌이 있느냐 없는냐 다 따져보아야죠.
그리고 장사. 이거 정말 장사능력시험이라는 영역 새로 신설해야할 듯.
손님을 휘어잡고 기름칠 잘 하며 광고 신나게 해먹는 언변술 등은 국어영역,
공급수요에다가 재고납기 다 계산하고 수익율 계산하는 것은 수학영역,
거기다가 각종 행정적 규제, 법률, 세금 등을 탐구영역으로 하면 사실 이것도 공부할 것 많죠.
\vspace{5mm}

그것도 그렇거니와 가만히 분석해보면 재수해도 삼수해도 안 되는 게 아니라
재수, 삼수할 때 제대로 공부한 케이스도 그리 많지 않다는 게 문제임(...)
다시 말해서 올바른 공부 안 하면 n+1 해보았자 실패의 귀납법 완성이죠.
이거 어그로일지 모르는데 몇수 해도 안 되는데요... 이거 졸라 파고 들어가면 공부가 엉터리인 경우가 걍 대부분임.
그리고 이것도 알아야 함. n이 3 이상 넘어가면 실력이 늘어나는 게 아니라 허력(=실패를 조장하는 힘) 더 늘어나버립니다.
가끔 팔수 구수 해도 안 된다고 하는데 이 정도면 팔수 구수해도 안 되는 게 아니라, \textbf{실패하는 법을 5년 넘게 학습해버린 것임.}
\vspace{5mm}

걍 소주 노가리 까듯이 어이 살만하냐 하듯 분위기말하면 명문대에 꼭 가라할 건 아닌데
제가 명문대에 가라고 한다면 \textbf{일단은 합격을 해야 다년간 쌓인 체증이나 컴플렉스가 해소되기 때문}이라고 말씀드리겠음.
명문대 합격한다고 해서 효용이 그리 크냐 그건 아닌데, 사람이 달라지는 건 있음. 자신감이 생기고 나도 할 수 있다라고 느끼면서
내가 병신이 아니었구나, 단지 잘못된 방법이나 시스템을 타서 날 과소평가했구나, 날 손가락질하던 그 사람들이 문제였구나하는 것.
저거 명문대 대신 공무원이나 전문자격 시험으로 바꿔도 무방합니다만 아무튼 그렇다는 겁니다.
\vspace{5mm}

정말 눈여겨볼 사람들이 명퇴한 40대 이후임. 이 사람들은 님들의 부모님일 수도 있는데 잘 눈여겨보아야 함.
명퇴하는 사람들의 문제는 새로운 걸 학습할 능력이 없다는 거임. 회사형 인간은 쫓겨나면 답이 없음.
닭집 창업한다고 하지만 말이 좋아 창업이지 거액의 권리금 주고 프렌차이즈 가야죠. 아는 게 없으니까 $-$ 그 때부터 빚의 향연.
왜 개나 소나 닭집 창업하느냐 의심품겠지만 이건 당연합니다. 그나마 수요가 있고, 운좋으면 남의 치킨시장을 잠식해먹을 수 있으니까.
\vspace{5mm}

굳이 대학에 간다기보다도 "새로운 걸 학습하는 방법"을 알기 위한 측면에서 수험의 의의가 있는 겁니다.
10년 뒤, 아니 5년 뒤에 한국이 어떻게 변할지는 아무 것도 모름. 분명한 건 새로운 것을 배울 자세는 되어있어야 하고
시험이 없으면 우리가 시험을 만들어서라도 응시해야 할 판이죠.
\vspace{5mm}






\section{독학을 권유하는 이유}
\href{https://www.kockoc.com/Apoc/502091}{2015.11.20}

\vspace{5mm}

여기서 말하는 독학이란
\begin{itemize}
    \item $-$ 강의는 그냥 보충용, 가장 중시하는 건 "설명이 부족한 책도 자기가 설명을 하면서 읽기"
    \item $-$ 국영수탐 하루에 수능시험 문제의 1.5배 풀기를 말하는 것임.
\end{itemize}


\vspace{5mm}

일단 강의를 듣는 것은 혼자 책을 보다가 잘못 갈까 그래서 그런 건데
이 경우 간과하는 것이 있죠. 그럼 '강의'는 잘못된 게 없나. 유감스럽지만 많습니다.
\vspace{5mm}

독학의 단점은 시간이 처음에 많이 걸리고 불안하다는 것.
장점은 늦어도 6개월 버티고 나면 비로소 "나다운 것"을 알면서 자존심의 새싹이 돋아난다는 겁니다.
강사에 의존하면 끝까지 강의에만 빠져버리죠(이건 세뇌된 상태니까요)
아마 그 사람들은 첫키스를 할 때, 심지어 첫날밤을 보낼 때에도 인강을 틀지 않을까 싶을 정도입니다.
(그러다 강사가 정신교육하면 현자타임 오지 않을까)
\vspace{5mm}

학습의 목적은 \textbf{스스로 일어서는 것}입니다. 강의는 어디까지나 보충용이지요, 지금 잘못되어도 한참 잘못된 겁니다.
\vspace{5mm}

모르는 게 있으면 그걸 따로 메모하면서, 사이비 이론이라도 좋으니까 스스로 설명해보려고 하세요. 그래야 실력이 늘어요.
특히 수학은 한문제가지고 한달 정도까지 고민해 볼 수도 있습니다. 그런 고민과정에서 늘어나는 거지 강사가 찍어준다고 늘진 않아요.
강사가 찍어주면 처음에야 문제가 잘 풀리니까 오우 좋다 하겠죠. 세뇌론에서 말했듯이 쾌감의 원천은 \textbf{'논리적 사고의 중지'}거든요.
강의 듣고 기분이 좋은 건 모르는 것을 알아서가 아니라, '생각을 하지 않아도 되기 때문'입니다.
그게 독입니다, 나중에는 생각하는 방법을 몰라서 계속  강의만 의존합니다.
\vspace{5mm}

물론 생각을 하게 도와주는 강의도 없진 않죠. 하지만 그런 강의는 정말 찾기 어렵습니다.
\vspace{5mm}

지금 시작하는 분들은 도서관에 가서 역사책을 많이 읽어보세요.
어떤 책을 읽어야하느냐 물어보는 분들 많은데 저라면 서슴없이 역사라고 하겠습니다.
모든 책은 사실 모든 분야의 역사를 기록한 책입니다(미래도 가상의 역사죠)
역사는 기록자에 따라 조금씩 달라진다 해도 어찌되었든 FACT이기 때문에 다양한 함수관계가 응축되어있습니다.
그 역사를 읽고 해석하면사 생각하는 법을 배우는 것이죠.
\vspace{5mm}

책읽는 것도 내년 2월까지일 겁니다. 그 이후에는 읽고 싶어도 못 읽어요
\vspace{5mm}

물론 이건 평범한 케이스에 관해서이고
제가 학원가라고 하는 케이스는 무조건 학원 가세요.
혼자 공부할 수 있는 상태가 아니기 때문입니다.
\vspace{5mm}






\section{n수할 때 부모말을 들어야하나}
\href{https://www.kockoc.com/Apoc/503343}{2015.11.21}

\vspace{5mm}

n수가 문제가 아니라 전반적으로는 '참조'만 하지 들을 이유는 없음.
저 이야기는 부모님들이 자식의 인생을 다 책임져준다라는 걸 전제한 것인데
부모님들이 재산에게 '부양'의 대가로 재산을 물려줄 수 있을 지언정 \textbf{다른 거 할 수 있는 건 사실 없음}.
\vspace{5mm}

나이먹은 어른들이라고 하더라도 세상 어떻게 돌아가는지 아시느냐 그것도 아님.
그렇다고 철저하게 내 자식은 $\sim$ 하게 해야한다라고 조사하거나 고찰할 분도 없음.
보통은 다른 사람들 말 \textbf{: $\sim$ 카더라 수준을 보고 "얘야 $\sim$ 이게 좋다고 하던데"라고 하는 게 대부분임}
\vspace{5mm}

그리고 본인 인생은 본인이 사는 것임.
부모말 듣고 잘 되었다고 해도 자기가 잘했다고 미화할 것이고
부모말은 참조만 했을 뿐인데 나중에 망하면 이거 다 부모님 때문이예요라고 책임전가할 것임.
\vspace{5mm}

어떤 위로든 대화든 사실 \textbf{결과 외에는} 소용없습니다.
수험장삿꾼들이 정말 학생들 인생 신경쓴다고 생각함? 그거 개소리죠. 얼마나 벌 수 있을 것인가나 생각하지.
부모자식간도 비정함, 부모님들도 자녀가 말 안 들어처먹어도 \textbf{좋은 대학 가고 좋은 데 취업해서 두둑한 돈봉투 바치면 좋아하시지}
무능한 효자 효녀 좋아... 개뿔입니다. \textbf{입효도}를 누가 좋아하죠? 그건 누구라도 싫어할 것 같은데.
\vspace{5mm}






\section{여러가지.}
\href{https://www.kockoc.com/Apoc/504237}{2015.11.21}

\vspace{5mm}
\begin{enumerate}

    \item 좋아하는 교재일수록 독
    \vspace{5mm}

    참고서를 보는 건 궁극적으로는 '지적능력'을 키우기 위해서이다.
    그러므로 자기가 부족한 것을 채워줄 수 있는 교재나 강의를 선택해야하는데
    \vspace{5mm}

    대개는 자기가 부족한 게 아니라, 자기가 '좋아하는' 교재나 강의를 선택한다.
    국어를 잘 하는데 수학을 못 하는 친구는 수학을 공부해야겠지만 대개는 국어를 더 공부하려 하고,
    문풀량이 부족한 친구들은 문풀을 많이 해야하는데도 계속 인강을 들으려고 하며,
    반면 문풀이 충분한데 생각하는 법을 모르는 사람은 생각하는 방법에 관한 책을 읽거나 강의를 들어야하는데 관성대로 공부한다.
    \vspace{5mm}

    교재추천을 달가워하지 않는 건 여러가지 면이 있는데 이것도 한가지 이유다.
    어차피 추천해보았자 자기가 좋아하는 교재만 본다. 그런데 이걸 알아야지, 자기가 좋아하는 교재는 \textbf{쓸모없는 교재라는 것을}.
    봐야한다고 듣기는 했는데 보기 싫은 교재들이 실제로는 도움되는 교재들이다.
    \vspace{5mm}

    \item \textbf{자존심}
    \vspace{5mm}

    수험생의 가장 큰 적은 자존심이다.
    재수생까지는 자존심이 강하다. 그러나 삼수부터는 이게 무너지는데.
    \vspace{5mm}

    자존심이 적당히 무너지면 이것맘큼 좋은 게 없다. 그럼 선생이 하라는대로 자신을 개조하거나 적극적으로 변화하려하기 때문이다.
    하지만 자존심이 너무 무너지면 폐인이 되어버리는데 역설적이지만 자존심이 높을수록 더 많이 꺼져버린다.
    이 케이스가 서울대에서 참 많았던 것으로 안다(지금도 있겠지)
    \vspace{5mm}

    자기가 공부를 잘 한다라는 의식은 아무 소용이 없다. 뭘 하든 자기의 학습시스템이 얼마나 살상력이 좋으냐만 따져야지.
    영원한 1등도 꼴등도 없는데 사람들은 과거 경험만 가지고 그걸로 자신을 과대, 과소평가하는 경향이 있다.
    돌도끼 잘 던져 싸우는 족장님이 첨단 로봇이 전쟁하는 시대에도 위너가 될 리는 없지 않나.
    \vspace{5mm}

    \item 성격
    \vspace{5mm}

    정말 수험에서 성공한 케이스는 \textbf{성격도 변하는 케이스}다.
    물론 다 변한다는 건 아닌데 공부를 성공적으로 한다는 건 기존의 실패를 극복한다는 얘기고,
    기존의 실패를 극복한다는 건 어리석은 행위를 낳은 성격을 고치는 데 성공했다는 이야기이기도 하다.
    사람들은 수험에서 성공한 인간승리만 이야기하는데, 실제로 정말 중요한 건 성격이 바뀌는 것이다.
    다만 이건 본인이 알아차리기가 어렵기 때문에 잘 모르는 것일 뿐.
    \vspace{5mm}
\end{enumerate}
개인적으로 조언주거나 할 때에 어떤 교재를 보느냐보다 강조하는 게 사실 이 대목이다.
성격이 바뀌면 사고방식도 바뀌게 되고, 사고방식이 바뀌면 문제를 푸는 스타일에도 영향이 있게 된다.
특히 수학의 경우는 절반은 성격이 먹고 들어간다. 성급해하거나 자뻑성이 강하거나 하는 친구들이 수학을 잘하기는 어렵다.
차분하고 꼼꼼하면서도 생각이 깨어있고 마무리를 잘 하는 친구들은 뭔 교재를 보든 점수가 잘 나온다.
거꾸로 말해서 수학공부를 한다는 건 수학뇌를 만들기, 수학에 어울리는 성격으로 바꾼다... 라는 것이다.
\vspace{5mm}





\section{개정 과정}
\href{https://www.kockoc.com/Apoc/505123}{2015.11.22}

\vspace{5mm}

고3 올라가는 사람은 모르겠는데
올해 시험 친 고3 이상이 그거 물어보면 걍 답이 없음.
\vspace{5mm}

수학 기출이 뭔 적중으로 푸는 게 아니죠. "사고력" 단련하려고 푸는 거지.
제2코가 들어간다 안 들어간다 행렬이 들어간다 안 들어간다 이게 중요함?
\vspace{5mm}

그런 개념들이 어떻게 변형해서 어떤 논리로 쓰이는가 하면 개정과정 상관없이 공부하는 거지.
고3 이상이면 어지간히 공부 안 한 사람 아니면 개정 전, 후 따질 필요 없음.
\vspace{5mm}

단 전반적으로 개정전이 난이도가 높음. 자, 여기까지 말해줍니다. 그 이상까지 다 떠먹여줘야하는지도 의문이고
이런 걸로 질문하는 사람은 자기가 스스로 내용비교도 안 해보았단 얘기인데(네이버 검색만 해도 뜬다) 이건 답 없는 것 아님?
그게 답답합니다. 이런 질문은 하지 마시고 찾아보세요.
\vspace{5mm}

그리고 저라면 제가 기존교육과정 배운 사람이면 기출 가리지 않습니다.
\vspace{5mm}




\section{수학문제풀이에 있어서 국어의 중요성}
\href{https://www.kockoc.com/Apoc/505651}{2015.11.22}

\vspace{5mm}

수학문제에 쓰이는 언어는 3가지이다.
\vspace{5mm}

\begin{enumerate}
    \item 문자, 수식
    \item 기하, 그래프,
    \item 개념, 성질
\end{enumerate}
\vspace{5mm}

인강 강사들이 유행을 타는 게 1번으로 먹고 살던 사람은 2, 3번이 대세가 되면 날라가고
반면 2번으로 먹고 살던 사람은 1, 3이 대세가 되면 날라간다.
최근 추세는 2번을 죽이고 1번과 3번을 강화하는 쪽이다.
\vspace{5mm}

그런데 정말 중요한 건 3번이다.
왜냐면 1번과 2번은 사교육이 온갖 주입과 암기로 강화시킬 수 있기 때문이다.
그런데 3번은 그런 걸로도 먹히지 않는다.
\vspace{5mm}

최근 킬러 문제들은 반드시 저 (문자, 수식)이나 (기하, 그래프)를
국어적인 (개념, 성질)로 해석하는 작업을 반드시 거치는 걸 요구하고 있다.
물론 고수라고 하는 사람들은 그런 게 필요없다고 할지 모른다. 자기들은 그게 무의식적으로 체화되었기 때문이다.
하지만 평범한 사람들이 따라잡으려면 반드시 국어적 해석, 즉 개념, 성질로 포섭해보는 작업을 거쳐야 한다.
\vspace{5mm}

21, 29, 30을 아마 감각적으로 풀어댄 사람도 많겠지만
정말 똑바로 공부한 사람들은 머릿 속에서 교과서 목차$-$개념$-$성질을 검색하면서 어디에 해당하는지 분명히 짚었을 것이다.
그걸 똑바로 밟은 사람은 안정적으로 풀었을 것이다.
반면 아무리 수학문제를 풀었어도 저 3번 항목을 이용하지 못 한 사람은 힘들었을 것이다.
\vspace{5mm}

아마 저번부터 수학을 국어처럼 풀어라하는 것이 궁금한 사람이 있었을지도 모르는데 그게 이거다.
그리고 교과서를 보라는 이유도 그런데, 수학을 국어처러 공부할 수 있는 줄글을 그나마 명쾌히 쓴 경우여서이다.
가끔 수학고수라고 하는 사람들이 쓴 책을 보면 그 부분을 빼먹고 있는 경우가 많다.
\vspace{5mm}

국어적 해석 없이도 풀다보면 된다고 하는 사람도 있다. 그런데 그건 수화로 말이 통하는 것과 동급이 아닌가?
풀다보니까 된다라고 하는 건 그거 10명 중에 1명 정도만 먹히지 나머지 9명은 안 통한다.
다시 말해서 영어를 가르치지 않고 미국인과 눈빛과 손짓만 교환해도 의사소통할 수 있다라고 하면 누가 소통을 할 수 있나?
\vspace{5mm}

게다가 저 3번 항목을 제대로 짚어야
어째서 출제가가 그런 문제를 어떤 의미에서 냈고
풀이가 어떤 필연성을 갖게 되는지를 알 수 있다.
가끔 고수라고 하는 자들이 무조건 풀어준다...  하면서 풀이를 내고 환호한다 그러는데 그건 정말 바람직하지 않다.
왜 그런 풀이가 나오는지에 대한 '국어적 썰'이야말로 정말 중요한 것이다.
이런 과정없이 머리가 좋아서 푼다... 라고 하는데 과연 그런 방식이 한계가 없다고 보는가?
\vspace{5mm}

다른 이야기를 하면 수학문제를 풀 때에 꼼수나 스킬 이건 안 쓰는 게 좋다.
간단하다. 그런 꼼수나 스킬로는 풀 수 있는 게 '한정되어' 있어서이다.
교과서의 답답한 풀이는 소박해보이지만 사실 이거야말로 커버 범위가 넓다, 그리고 실수할 가능성도 거의 없다.
실수하는 케이스는 간단하다, '순서'를 안 지켰기 때문이다. 밟아아햐는 징검다리를 안 밟고 점프했다가 물에 퐁당 빠지는 것이다.
\vspace{5mm}

요새 학생들은 순수한 수리적 사고 $-$ 즉 식변환이나 그래프, 기하 패턴 파악능력은 좋다. 선행학습의 결과다.
그러나 처음 보는 문제를 어떻게 교과서상 개념으로 포섭할 것인가는 정말 약하다.
독서량이 부족해서 국어 실력이 좋지 않아서이다. 추상적 개념을 다룰 수 있는 지적능력이 부족하다보니 포섭력도 약한 것이다.
\vspace{5mm}

그래서 풀이 방향이 알려진 건 정말 빨리 푼다. 한데, 조금이라도 꼬아놓아서 풀이방향을 알 수 없는,
즉 개념과 성질로 차분히 생각해야만 접근 방법을 알 수 있는 건 건드리지 못 한다.
즉 문제의 독해력이 매우 부족한 것이다.
\vspace{5mm}

개념을 어느 정도까지 봐야하느냐라고 하시는데... 애국가 1절 암기하는 수준으로 줄줄 나와야한다고 본다.
교과서의 개념논리 흐름에 맞춰 풀이해야 낚시에도 안 걸리고 실수하지 않기도 하지만
그 개념들의 세부사항을 알아야만 낯선 문제를 수학적 개념에 포섭시켜 풀이방향을 잡을 수 있다.
\vspace{5mm}

수능수학이 과거보다 쉬워지는(?) 측면은 있다. 그런데 조심하자, 쉬워진다는 건 다루는 내용의 양적 측면이 줄어든다는 것이지
그 질적 측면도 쉬워진 건 아니다. 올해 시험도 과거에 비하면 쉽다고 하지만 풀이 방향은 다르다.
과거 2012년도까지의 문제는 발상 면에서 어렵다, 하지만 이것들은 아이디어를 잘 잡으면 논리가 없어도 맞았다.
논리가 없어도 맞다는 건 순서를 안 지켜도 일단 발상만 떠오르면 그럭저럭 답에 근접할 수 있었단 이야기다.
하지만 작년과 올해 시험은 '논리'를 강조하고 있다. 발상을 크게 요구하지는 않지만 순서를 못 지키면 풀 수 없거나
오답으로 끌려가도록 내고 있다
\vspace{5mm}



\section{그러니까 노력한 근거를 대야지}
\href{https://www.kockoc.com/Apoc/505846}{2015.11.23}

\vspace{5mm}

그런데 실증해보면 본인은 노력했다고 하는데 그게 '아닌 게' 보여서 말입니다.
\vspace{5mm}

추궁해보면 잠시 게임했다고 알고보니 하루에 2시간씩 게임한 게 6개월. 공부 시작한 건 고작 4개월인 경우.
살짝 연애했다고 하는데 더 추궁해보니까 그냥 1년의 절반 이상을 연애에 투자.
하여간 별의별 사례가 다 있음.
\vspace{5mm}

시간 검증은 그렇다 치고 그럼 풀어대 문제집 물어보면 횡설수설하다가 양치기 안 해도 되거든욧!
양치기 예외가 이번에 액상탄마님의 경우인데 수학은 그런 측면이 있었고 이 분은 사탐은 공전의 점수 나와서 오히려 검증된 경우임.
(수학은 공부하는 방법도 매우 중요하고, 특히 수리적 사고라는 건 1년만에 되지 않는다... 는 걸 보여준 좋은 경우인 것 같음)
\vspace{5mm}

작년 이맘때쯤인가 그래서 하도 짜증나서 제안한 게 일지 시스템.
그 이후로 교재 추천하는 질문은 몽땅 기각먹임.
일지 꾸준히 쓰는 사람들에게 조언 때려주고 어떻게 하나보았고
최근 10일동안 1명씩 다 읽어보고 확인해보았는데.
\vspace{5mm}

노력 했거든요 하는 거짓말까는 인간들은 걍 원양어선 타고 매일 참치나 먹고 살았으면 좋겠음.
일지 쓰는 사람이 다 합격하는 것은 아님.
그런데 잘 나온 과목과 못 나온 과목 비교해보면 \textbf{공부량에 확실히 비례함}.
수능시험이 그리 비합리적인 시험은 아님. 하긴 그럴 수 밖에 $-$ 인원수가 많기 때문에 우연성의 좌우도는 낮음.
교재 차이? 그런 건 없음. 어떤 교재를 보았냐보다도\textbf{, 기본교재'들'을 많이 보고 양치기했느냐}가 좌우함
강의 차이? 극단적으로 말해 강의 안 들어도 된다고 얘기해도 될 정도임.
그리고 작년말부터 공부한 사람들이 승률이 높음, 중간에 쉬는 기간이 있다고 하더라도 일찍 공부한 사람이 이김.
\vspace{5mm}

노력해보았자 소용없다는 사람들은 그럼 1년동안
일지 쓰면서 공부하면서 호출하고 질문하고 피드백해 본 뒤에도 그러나 보셈
수능시험은 그럴 수가 없음 시험임. 응시자가 도대체 몇명인데.
그리고 노력 까는 인간들의 문제는, 자기만 망하면 상관없는데
그런 개구라를 까서 다른 사람들까지 피해를 입힌다는 것임.
노력해도 안 되는구나 질려버린 후배들은
그래서 특별한 인강이나 교재부터 봐야하는구나 착각해서 해서 상술의 노예가 되어버리니까 문제임.
\vspace{5mm}

그런데 양치기만 해도 안 되는 경우는
\vspace{5mm}
\begin{itemize}
    \item 국어 $-$ 독서량이라는 게 정말 많이 좌우함. 독서를 생활화한 애들은 사실 테크닉도 필요없음
    \item 수학 $-$ 수리적 사고, 문제 읽는 법, 그리고 문제 읽고 알고리즘 짜는 법은 알아둘 필요가(그런데 이거 정작 가르치는 인강 없음)
    \item 영어 $-$ 영어적 사고, 직독직해 필요함. 특히 전치사에 대한 감각이 강조됨
\end{itemize}
\vspace{5mm}

이라고 얘기하면 됨.
\vspace{5mm}








\section{개정 헤매는 분들을 위한 조언}
\href{https://www.kockoc.com/Apoc/506939}{2015.11.23}

\vspace{5mm}

원래 교재추천이나 코스 같은 걸 일일히 물어보는 것 자체가 안 좋은 결과를 낳는 경우가 많아서 그러나
그래도 불안해하는 분들을 위해 적습니다.
\vspace{5mm}

올해 시험을 친 고3을 포함한 n수생
\vspace{5mm}
\begin{enumerate}
    \item 기존 교재 버리지 마실 것 : 기존 교재가 난이도가 높습니다.
    \item 빠진 내용의 기출이라도 푸실 것 : 수학적 아이디어나 발상이 어디가는 것은 아닙니다.
    \item 개정과정에서는 확통, 기벡만 추가로 살필 것 : 그러나 크게 바뀐 건 없습니다. 재배치되었을 뿐이고 확률은 정수분할 정도만 보면 됩니다.
    \item 기출은 그냥 보던 것 보실 것 : 내년에 기출문제집이 나온다고 하더라도 올해 기출만 추가된 정도입니다.
\end{enumerate}
\vspace{5mm}

올해 시험을 칠 고3
\vspace{5mm}

\begin{enumerate}
    \item 과거 교재 볼 필요는 없음, 님들 대상으로 한 교재를 보면 됨, 그 교육과정으로 7차 교육과정 따라갈 수 없음.
    \item 기출이 새로 나오면 보시길 바람 : 그런데 문제는 많이 푸세요.
\end{enumerate}
\vspace{5mm}

그런데 둘 다 신경쓰실 건
\vspace{5mm}
\begin{enumerate}
    \item 고1수학의 비중 높이실 것. 올해 수능 기출 킬러는 스킨만 고2 수학이었지 실제 발상은 고1 수학 쪽이었습니다.
    \item 전과목 포함해 마지노선을 4월말까지로 당기시는 게 좋음, 개편 혼란 때문에 아마 여러 시행착오가 있을 겁니다.
    \item 시중교재와 교과서에 충실할 것.
\end{enumerate}
\vspace{5mm}

1번의 경우는 이걸 왜 아무도 지적 안 하는지 모르겠습니다만 그 문제들이 변별력이 좋았던 건
다름 아니라 문제푸는 과저에 있어서 고1 수학을 물어보았기 때문입니다. 그 점에서라도 고1 수학 우습게 보지 마시길요.
이과의 경우도 범위에 직접적으로 안 나온다는 것이지, 어려운 문제를 만들 때 고 1 수학 원리 안 쓴단 말은 없었습니다.
2번의 경우는 그냥 겁주는 게 아닙니다. 올해도 5월 정도에 실력이 결정될 사람은 다 결정되었습니다.
안 믿다가 나중에야 네 말이 맞았다 그래 잘났다 어쩌구 반응이지만 이건 당연한 거예요.
방황하지말고 4월까지 졸라 달리시길 바랍니다. 노시려면 4월까지 마치고 노세요.
3번의 경우는 뻔한 소리가 아니라 지금 내년 출제가 어떻게 나올지 데이터가 없습니다. 예비평가 시행을 아직 안 했죠?
일차변환과 행렬이나 제2코사인이 빠졌다고 좋아할 게 아닙니다.
문과 수학은 오히려 늘어나버렸고(집합 명제 가지고 킬러내면 뭐 대박일 듯. 90년대 수능 한번 풀어보세요)
이과 수학의 경우도 외려 더 어렵게 낼 수 있는 여지가 커졌습니다.
\vspace{5mm}

그렇게 보자면 특정 출제 경향 따라서 공부한다... 아무 소용없습니다.
올해 실모 보면 된다 그거 경험해부신 분은 알지만 별로 의미없는 것 보았죠? 실모는 보충용으로나 먹혔죠.
또한 기출 역시 좋은 소스가 될지 모르나 이것만 가지고는 힘듭니다.
남들 보는 시중교재, 고1까지 포함해서 빨리 시작해서 더 많이 끝내세요. 그만한 대가는 분명히 보장됩니다.
\vspace{5mm}

쎈수학은 과거보단 쉬워졌습니다. 단 창의문제는 도전할만한 가치가 있고 경향이 다르니 풀어보시길
RPM은 바뀐 게 없죠(...)
블랙라벨도 크게 바뀐 거 없으나 개념서로서의 성격이 강해졌고 수험생들이 헷갈리는 부분을 잘 정리했습니다.
주목할 교재는 올림포스 평가문제집과 일등급 수학입니다. 새로운 변종을 경험해보실 수 있을 겁니다.
올림포스 평가문제집과 일등급 수학은 다 풀어보시는 게 좋을 것입니다.
\vspace{5mm}

그렇게 보자면 지금 시험 때까지 남은 기간은 '모자라'지요.
내년 수험생들은 시간부족에 더 시달리겠네요.
\vspace{5mm}

자, 이 정도로 말씀드리니 쓸데없는 질문이 없길 바랍니다.
\vspace{5mm}







\section{모의고사만 줄창 푸는 게 안 좋은 이유}
\href{https://www.kockoc.com/Apoc/509601}{2015.11.25}

\vspace{5mm}

모의고사를 내는 사람이건 푸는 사람이건
\textbf{해당 모의고사들이 특정 경향에만 치우쳐있다}는 것을 절대 모릅니다.
\vspace{5mm}

이른바 \textbf{평균의 함정}이죠. 그게 모의가 나빠서만은 아닙니다만
자기들이 생각하기에 \textbf{출제빈도가 높은 문항을 우선순위로 게재한다}는 게 문제임.
출제빈도가 높은 것을 풀면 그 다음부터 문제가 쓱싹 잘 풀리니까 이게 정말 효과가 좋은 걸로 생각하죠.
하지만 이건 결과적으로 우물 안 개굴짱을 만드는 것이죠.
\vspace{5mm}

그런데 수능은 \textbf{평균에서 꼭 벗어난 걸 내거든요.}
\vspace{5mm}

모의가 미래지향적이어야하는데 실제로는 과거지향적인 경우에 불과한 게 많다가 함정입니다.
물론 저자들도 적중시키고 싶어는 하죠. 그런데 그게 될 리가 있나.
그래서 수험생 입장에서는 모의고사만 믿는 건 현명한 전략은 아닙니다요.
\vspace{5mm}

작년에 모의고사에 비판적인 것을 보고 저 녀석 괜히 그런다 하실 분들도 있었을 건데
근거없이 저런 주장하는 건 아닙니다요.
\textbf{미래에 대비하고 싶으면 남들이 안 가는 방향, 커버하지 못 하는 것까지 다 준비해두는 수밖에 없습니다.}
더군다나 다수의 확신이라는 건 매우 높은 확률로 다수의 패배를 의미하기도 해서리
\vspace{5mm}

시중 기본서는 당장 수능과 거리가 멀어보이긴 하는데 이것만큼 미래를 준비하기 좋은 책은 없습니다.
반복하다보면 행간의 내용이 다 드러나죠. 그런데 모의고사는 그게 안 되는 것입니다.
\vspace{5mm}

나는 열심히 노력했는데.... 의 경우가 안타까운 것은 지나치게 다수의 확신을 따라가더라 그것입니다.
ABC에서만 나올 거라고 공부했는데 DEF에서 나왔다고 평가원을 욕하죠. 하지만 원래대로라면 A$\sim$Z 다 공부했어야하는 건데요.
올해 시험 치고 복기해보라고 한 기한 다 지나서 적는다면, 정작 자기가 집중적으로 공부한 것이 쓸모가 없다는 걸 느껴보라는 얘기였습니다.
\vspace{5mm}

내년도 4월 말까지 공부 해놓고 남들 어떻게 하나 보세요.
그리고 남들이 안 하는 것을 비밀병기 준비하듯 공부하면 됩니다.
\vspace{5mm}





\section{교재 : 한국사 강의없이 틀잡고 싶다면}
\href{https://www.kockoc.com/Apoc/511903}{2015.11.26}

\vspace{5mm}

\href{http://www.yes24.com/24/goods/2987183?scode=032&OzSrank=4}{역사신문}

\vspace{5mm}

\href{http://www.yes24.com/24/goods/17403469?scode=032&OzSrank=1}{남경태의 종횡무진 시리즈}
\vspace{5mm}

참고로 
\href{http://www.yes24.com/24/goods/17403476?scode=032&OzSrank=2}{동서양사도}
\vspace{5mm}

\href{http://www.yes24.com/24/goods/17403471?scode=032&OzSrank=3}{2}

\vspace{5mm}

를 보면 지엽은 모르지만 틀은 잡을 수 있음.
\vspace{5mm}

그런데 개인적으로는 역사신문 시리즈만 구매를 권하고 나머지는 도서관에서 빌려볼 수 있으면 빌려보는 걸 권하겠음.
\vspace{5mm}

역사신문 시리즈의 장점이 절대 지루하지가 않다는 것이죠. 서술 방향은 오히려 좌파 쪽이긴 한데 술술 읽힐 것입니다.
\vspace{5mm}

그러고보니까 이원복의 먼나라이웃나라도 볼만하지 않을까 싶긴 한데.
\vspace{5mm}

그냥 강의를 듣는다면 사설강의 들을 필요 없이 EBS만 가는 걸 권함.
\vspace{5mm}

그런데 그 강의조차 듣기 싫다 난 걍 책 읽을 거야라고 하면 동네서점에도 있을테니 참조해보시길.
\vspace{5mm}






\section{몰입한 상태}
\href{https://www.kockoc.com/Apoc/512266}{2015.11.26}

\vspace{5mm}

세뇌론에 적겠지만 수험장에서의 멘탈을 이야기하는 분들이 있어서 그냥 미리 간략히 적으면.
\vspace{5mm}
\begin{itemize}
    \item A $-$ 몰입을 해본 적이 없음, 그래서 신유형이 나오면 긴장하고 평소보다 능력치가 떨어짐
    \item B $-$ 평소에는 느슨하지만 몰입을 할 수 있음, 시험에 임할 때 자신을 잊고 몰입 상태에서 문풀을 시작함.
\end{itemize}
\vspace{5mm}

멘탈 붕괴된다라고 하는데 이거 본질에서 벗어난 거임.
낯선 상황에 긴장하는 건 누구나 마찬가지입니다.
그런데 A는 그 긴장에서 당황스러운 상태로 가는 반면, B는 바로 자기만의 몰입된 상태(일종의 세뇌된 상태)로 가죠.
\vspace{5mm}

우리가 호흡을 의식하면서 하는 건 아니죠. 무의식적으로 합니다.
우리가 말을 할 때에도 사실 생각없이 합니다. 일일히 생각하고 말하진 않죠.
아니 일상 영역에서의 행동은 의식하고 하는 게 없어요.
\vspace{5mm}

한 분야에서 공을 들인 고수들은 그 분야를 무의식적으로 해냅니다. 일일히 신경쓰지 않으니 피로감도 덜 하고 그래서 더 잘 해냅니다.
이 상태에 도달해본 적이 없거나 이걸 의식하지 못 하면 당연히 노오력에 회의감을 품게 될 것입니다.
그러나 이런 상태 $-$ 일종의 임장감이라고 하는 것에 도달해 기적적으로 뭔가 완수한 사람은 다른 분야도 마찬가지인 것을 알게 되죠.
\vspace{5mm}

이런 상태에 도달하려면 개인별로 차이가 있긴 하지만 무수히 많이 반복하는 수 밖에 없습니다.
비상 상태에서도 관련된 내용이 정확히 입에서 튀어나오도록요.
강도가 칼을 목에 대도 \textbf{그 내용이 다 정확히 나오도록 학습되어있어야 비로소 공부}입니다.
\vspace{5mm}

이게 멘탈과 관계있느냐. 글쎄올시다, 덜덜 떨고 파랗게 질리더라도 그 모든 지식을 복기할 수 있는 정도까지 가야 공부입니다.
그럼 이게 비현실적인 노력... 까진 아닐텐데 말입니다. 비상 상황에서도 정확히 임무를 수행하는 직업인들이 있을텐데요.
사실 제 입장에서는 그렇습니다. 멘탈 타령하는 사람들은 훈련을 덜 하고 싶어하는 사람으로 밖에 안 보이죠.
\vspace{5mm}

기출을 왜 여러번 푸느냐. 아는 것 반복할 필요 없지 않느냐.
이 때 피식 웃어주는 이유가 그겁니다. 이런 말을 하는 사람들은 공부를 그냥 데이터의 저장으로만 생각하는 거죠.
어떤 상황에서든 침착하게 학습한 바를  행하기 위해서 반복하는 것입니다.
\vspace{5mm}

시험 때 긴장 안 하고 싶다. 유일한 방법은 '실성'하는 거죠.
\vspace{5mm}

그리고 \textbf{약}
\vspace{5mm}

제가 싫어하는 사람이라면 권하겠습니다.
차라리 심호흡을 하고 자기최면을 하는 걸 익히세요.
\vspace{5mm}





\section{인터넷 강의 문제점}
\href{https://www.kockoc.com/Apoc/514702}{2015.11.28}

\vspace{5mm}

인강 없던 시절에도 그런 것 없이 서울대 가는 사람은 잘만 갔습니다.
개인적으로는 인강의 문제는 상당히 많다고 여기는데
근본적인 건 \textbf{"혼자 공부할 수 있는 힘"을 완전히 날려먹는다}는 것이고
그리고 거짓말이 상당히 많다는 것입니다.
\vspace{5mm}

일단 개소리 하나를 저격하죠 인강이 다 다뤄준다?
웬만하면 저격 안 하겠는데 저건 정말 개소리라서 한마디 합니다.
\vspace{5mm}

헛소리입니다. 저도 과거에 3년동안 듣고 정리해보았는데요.
일단 인강이 다 다뤄준다라고 느끼는 건, 본인이 \textbf{'책을 읽을 줄 몰라서'} 하는 이야기입니다.
수학에서의 문풀 사고법 빼고 나머지는 \textbf{모두 책에 있습니다.} 아니, 책에 더 많이 있다고 말씀드리지요.
궁금하면 님들이 신나게 필기해보신 다음 시중교재와 비교해보세요. 수능에 출제될 수 있는 건 거의 차이가 없습니다.
\vspace{5mm}

책을 읽는 게 힘들거나 아예 몰라서 처음에 입문용으로 듣는 건 추천할 만 합니다.
하지만 틀을 잡으면 바로 본인이 책을 읽고 스스로 정리하고 \textbf{추론}해야합니다. 그렇지 않으면 절대 늘어날 수 없습니다.
어디든 상담해보면 인강 때문에 흥한 경우보다 \textbf{망한 경우가 더 많습니다}.
그런데 흥한 경우는 혼자서도 독학이 가능하고 책을 읽을 수 있는 능력이 있습니다.
\textbf{무엇보다 조기교육을 받은 케이스가 많고 집안환경이 좋습니다(왜 이런 건 언급하지 않는 걸까)}
반면 집안환경이 좋지 않고 조기교육을 받지 않았는데 인강으로 가서 잘 되는 케이스는 없네요.
\vspace{5mm}

인강의 장점은 들을 때는 소화가 잘 된다는 것입니다. 단점은 소화는 잘 되는데 계속 배가 고프단 것이죠.
저런 거 그냥 필기노트로 해서 수학독본처럼 그냐 서술체로 가는 게 수험생의 시간을 단축시켜주지 않을까도 싶습니다.
예컨대 EBS에 보면 강의자막을 HWP 파일로 정리해주죠. 필기만 제공된다면 차라리 그것을 파는 게 훨씬 나을지도 모릅니다.
\vspace{5mm}

더군다나 인강의 문제점이 심하다고 보는 이유는 두가지가 있습니다.
\vspace{5mm}

\begin{itemize}
    \item 첫째, 매우 안 좋은 수학책이 있습니다.
    내용은 그럴싸한데 실제론 기본사고를 말아먹게 만드는 책이죠.
    그런데 이 책, 온갖 \textbf{인강을 짜깁기}했더군요. 짜깁기한 저자도 문제가 많지만 이건 인강 내용 그 자체도 문제가 많다는 겁니다.
    (그렇다고 잘 짜깁기한 것도 아닌 것 같지만)
    \vspace{5mm}

    \item 둘째, 강의를 바쁘게 한다고 하는데 그럼 연구나 개발은 언제 하느냐는 것입니다. 매년 강의가 똑같고 심지어 틀린 말 하는 사람도 있죠
    새롭다 신박하다 하는 풀이라는 것도 사실 학문적 체계가 의심되는 경우가 많으며 사실 수능에는  쓸모가 없습니다(...)
    최근 5년간 수능에 인강이 큰 도움이 되는 경우가 있기나 한가는 의문이 들더군요.
\end{itemize}
\vspace{5mm}

흔히 이런 이야기를 하죠. x등급에서 x등급으로 올린다... 이런 광고도 참 의문이 많습니다.
가정환경이 좋고 조기교육을 받았으며 체계가 잡힌 사람들을 올리는 건 솔직히 어렵지 않습니다. 그 경우는 강의 없어도 올라갈 사람은 올라가죠.
하지만 정말로 사고방식이 막장이고 안 좋은 환경에서 자란 경우를 올린 케이스는 사실 본 적이 거의 없습니다(...)
원래 이런 것을 연구해보는 게 취미(?)이기도 해서 조언해주고 캐리해주면서 느끼지만
이거 단순히 공부의 문제가 아니라 푸념도 들어주고 쓴소리도 거꾸로 적절히 해주고 \textbf{성격} 고쳐야 하는데다가
가정\textbf{환경} 전체를 다 뒤집으면서 '사람' 자체를 바꿔야한다고 느끼는데 과연?
\vspace{5mm}

저런 걸 구체적으로 안 들어가본 사람이 머리 타령하겠죠. 아닙니다, 뭘로 가든 성격이 문제입니다.
다운증후군 환자거나 정말 뇌를 다쳐서 맛간 경우가 아니라면, 온라인 게임이나 야동이 가능하면 머리의 문제가 아니라 성격의 문제입니다.
그 다음은 공부환경의 문제죠. 그리고 이 점에서 인강이 또 문제가 됩니다. 인터넷 접속을 유도하거든요.
\vspace{5mm}

그리고 한가지만 그냥 제가 발견한 사실 적죠.
\vspace{5mm}

상담이든 대화이든 해보면 참 유약하구나라고 느껴진 케이스들이 많고 다 추적해보면
독서는 빈약한데 정말 모든 것을 '사교육'으로만 의존했고 그렇게 키워진 케이스입니다.
겉으로는 똑똑한데 급소만 찌르면 그냥 무너질 수 있는 사람들이지요.
이런 것 보면 한심하고 짜증나서 '독서'를 하고 스스로 공부하라고 하는 것입니다.
그리고 컴 접속 줄이고 도서관에 가라는 이야기입죠
\vspace{5mm}

사교육이나 인강으로만 머리가 채워진 친구들이 솔직히 과연 정상일까, 실제로는 멀쩡해보이는 '환자'들이라는 게 제 개인적 평가입니다.
스스로 읽고 생각하고 자문자답해본 친구들은 느립니다. 정말 튼튼하게 자아가 성장하고 있고 쓰러져도 다시 일어날 수 있죠.
그러나 선생이 가르쳐주는 걸 따라하는 것만 시도한 친구들은 스스로 일어날 수 없습니다. 입시까지는 그게 운좋아 먹히더라도 그 다음이 문제겠죠.
이 친구들 대화하는 것 대사 분석하면 "트라우마"에 비견되는 명제의 반복을 계속 확인할 수 있다는 것도 적어보겠습니다.
대화하면서 이 친구들을 치유하는 건 그 명제를 다른 명제로 바꿔주거나, 아니면 그 근원 자체를 해체시키는 것이지만 그건 다른 문제겠죠.
\vspace{5mm}

인강을 듣고 싶으면 사설 신청하지말고 EBS 입문강의나 열심히 들으세요.
그 다음 시중교재 풀면서 잘 안 되는 단원, 이해 안 가는 내용, 문제들을 따로 메모해놓고
그걸 스스로 한번 해결해보는 걸 해보십시오. 시간이 걸리더라도 이걸 스스로 해결해야 다른 것도 해결됩니다.
오래 고민한 다음에 인강으로 해결(될 건지는 모르지만)하는 건 권합니다. 스스로 고심하고 아파보아야 실력이 올라가지
남이 하라는 대로만 해서는 올라가는 게 아니라 올려지는 것입니다.
\vspace{5mm}

콕콕 내에서도 +1수를 권하거나 적극 하라는 경우는 두가지입니다.
\begin{itemize}
    \item 첫째, 어떤 환경에서도 당사자가 공부할 수 있다고 생각하는 경우 : 사실 이 경우는 제가 더 이상 조언할 필요도 없을 정도입니다.
    \item 둘째, 여러모로 가능은 한데 본인이 뻘짓을 해서 말아먹은 경우 : 그럼 이건 뻘짓을 안 하면 되는 게 아닌가?
\end{itemize}
\vspace{5mm}

사실 그 외에는 알아서 하라고 싶을 정도입니다.
단, 모군은 \textbf{그 좋은 환경 버프업을 받았으면 하늘에 감사할 것이지 뻘소리는 적당히 했으면} 좋겠습니다.
가볍게 쓰는 글이라도 \textbf{그런 글에 현혹되어 정말 인생 날라가는 애들도 많다}는 사실을 알아두세요.
올해도 상담쪽지 받고 조언해주면서도 또 확인했지만 \textbf{대책없이 인강만 듣다가 망한 케이스가 90$\%$입니다요.}
혹자는 무슨 무료데이터... 개소리입니다. 일지쓰라고 하는 건 수험생들이 "저 열심히 공부했는데요"라고 징징대니까
그럼 얼마나 열심히 공부하나보자, 네가 공부하는 기록 써보고나 징징대라 하는 차원이 강하지, 사실 그 외는 도움될 것도 없습니다.
\vspace{5mm}

여기가 모처의 잘못된 사상이나 습관에 전염되는 건 강경하게 거부하는 바입니다.
돈은 돈대로 날리고 호구는 호구대로 되고. 이거 제 알 바는 아니라고 쳐도 너무 눈에 밟힙니다. 꼴불견이지요.
\vspace{5mm}

+ 학원가야하는 케이스 있죠.
\textbf{"혼자서 정말 공부가 안 되어서 인터넷 접속하는"} 그런 경우입니다. 컴 없는 데에서 정말 순수히 다른 친구들 따라 공부해야합니다.
그런데 학원조차도 \textbf{'집단적인 학습 분위기'} 그거 믿고 가는 겁니다.
이것도 독학가능하면 도서관으로 대체할 수 있습니다.
돈을 주고 '\textbf{환경}'을 구입하는 건 적극 권할 일입니다. \textbf{환경}은 상관관계가 뚜렷하니까요.
\vspace{5mm}

+ 다시 말씀드리지만 교재나 인강이 문제가 아닙니다.
먼저 환경, 그 다음 \textbf{습관}을 바꾸세요. 그리고 반드리 \textbf{라이벌}을 잡아야 합니다.
공부에 유리한 것만 늘리고 불리한 것은 제거하는 환경을 잡아야 공부가 됩니다.
그 다음 반드시 절박한 심정으로 습관도 바꿔야 합니다. 습관을 고치면 성격도 따라갑니다.
마지막으로 라이벌을 정해야합니다. 라이벌이 수험 스트레스의 진공 청소기입니다.
\vspace{5mm}

+ 그리고 다시 말하는데 남자 수험생들은 자기를 혐오하길 바랍니다.
이 색기는 망했다... 라고 하는 남자 1순위는 나르시스트입니다. 나르시스트가 극성맞은 엄마와 결합하면 마마보이가 되죠.
남자는 승부에서 지고 깨지고 다치고 심지어 기절까지하고 그러다 일어서고 싸우고 무기 바꾸고 흉터 생기고 그러면서 성장하는 거지,
나 잘 생겼다 존잘 이딴 드립치는 것이 아닙니다. 자기를 사랑하는 순간 더 이상 진화할 수 없고, 그 순간부터 내리막길인 것입니다.
왜 충고해줘도 안 먹히고 계속 그 지경인가 하는 케이스들 보면 공통점이 \textbf{못난 자기 자신을 극도로 사랑합니다}.
\vspace{5mm}

+ 현역으로 척척 붙어서 SKY 가는 친구들이 사회에서 잘 나갑니다.
이 친구들은 나르시시즘에 안 빠지거든요. 자기를 사랑하지 않으니까 과거의 자아를 버릴 각오가 되어있고 좋은 게 있으면 바로 갈아탑니다.
학습에서 중요한 건 '환경과 습관'이라는 걸 체득한 사람들입니다. 티는 안 나지만 성과는 무시무시하죠.
그러나 n수생들은 인간문화재도 아니고 자기만의 '전통'이라는 걸 고수하고 지키려고 합니다. 끝까지 그걸 안 버리려고 하죠.
자기 방식대로만 가겠다, 고집센 내 자존심을 지키겠다, 내가 했던 방식으로만 갈테야....
실패의 원인이 자기 자신이라는 사실을 인정하지 않습니다. 그래서 n이 커지면 더욱 그걸 인정하기 싫어서 포기조차도 정당화합니다.
\vspace{5mm}

+ 더 무서운 건 여학생들은 저런 나르시시즘은 없단 겁니다.
여자들은 남자들과 달리 출산 때 피를 보기도 하지만, 우선 '화장'을 하는 게 어색하지 않죠.
그건 자기를 사랑해서가 아니라 타인들을 현혹하고 유혹하기 때문입니다. 그래서 남자와 정말 마인드가 다릅니다.
필기시험에서 여풍이 강해지는 이유가 이 때문일 것입니다. 여자들의 단점이라는 정보와 체력 부족은 인터넷과 관리로 해결되니까요.
\vspace{5mm}




\section{영어에 관해서}
\href{https://www.kockoc.com/Apoc/518404}{2015.11.30}

\vspace{5mm}

형식이 다르다하지만 TOEIC과 TEPS가 도움이 됩니다.
국어는 PSAT/LEET가 정통코스가 되어간다고 생각해서 이 남자, 아니 이 영어는 어떨까 보았음.
\vspace{5mm}

최근에 영어가 어떻게 나오나 궁금해서 3일(...) 정도 공부했는데
\textbf{만만해보이는 토익조차도 오랜 세월동안 진화했다}는 것을 느꼈습니다.
3일 정도로 패턴화될 수 있을지는 모르겠지만
여러가지 차이를 순수하게 느낄 수 있었습니다.
\vspace{5mm}

우선 L/C조차도 간단한 추론, 순발력을 꽤 요구합니다. 호주발음이 듣기 개판이다 그게 문제가 아닙니다요,
문항과 답변이 직접 대응이 아닙니다.
\textbf{청해$\rightarrow$ 판독 $\rightarrow$ 해석 $\rightarrow$ 여러가지 명제들 추론 $\rightarrow$ 답 고르기}
이와 같은 과정을 짧은 시간 내에 거쳐야합니다.
L/C 파트 2부터 파트 4까지 쭉 이어지는 과정이더군요.
\vspace{5mm}

R/C 쪽도 만만치 않습니다. 지문은 매우 쉬운데 선지에서 꼬아내던데
이 역시 위와 같은 영어적 추론을 해야하기 때문입니다.
\textbf{읽기 $\rightarrow$ 판독 $\rightarrow$ 해석 $\rightarrow$ 여러가지 명제들 추론 $\rightarrow$ 답고르기}
거기다가 복수지문들까지 보니까 대충 190문제부터는 이미 머리가 지쳐있음.
\vspace{5mm}

그리고 느낀 바 $-$ 아, \textbf{빈칸추론의 핵심이 저기 있었군.}
영어가 힘든 건 간단합니다. 단순히 해석을 넘어서 \textbf{"영어 자체로 사고해야하기 때문"}입니다.
그런데 아이러니컬하게도 이에 근접한 게 필요는 없지만 호기심에서 다시 쳐보았던 토익 시험에서 엿보이더군요.
어제 그래서 시험을 마치고 서점에 가서 토익과 텝스 교재를 죽 훑어보았습니다.
성인 대상으로 하는 시험이다보니 교재 질은 매우 좋더군요.
책이야 유명한 건 다 좋아보였습니다. 패턴 정리가 꽤 잘 되어있었으니까요.
\vspace{5mm}

올해 시험에서 영어 망쳤는데 다시 시작하실 분들은
어차피 대학가서 치러야 하는 시험이니 저런 시험들 응시를 해보시길 바랍니다.
냉정히 말해서 그냥 수능 수준의 영어가 도움이 되나.... 최근 강사추천도 올라오곤 하지만 그런 행태는 제가 혐오하는 것이고
기본적으로 English 자체로 사고한다는 점에서는 어휘나 지문 수준이 다르다고 해도 저런 공인영어시험을 공부하는 게 낫다는 결론이 되겠습니다.
\vspace{5mm}

그런 다음에 괴서 성문종합영어에 나온 명문들을 주로 읽어보시면 되겠죠.
읽어보란 이유는 간단합니다. 영어권 지식인들이 생각하는 건 우리 조선인들이 생각하는 것과 달라도 한참 다르기 때문입니다요.
여러 학생들이 골치아파하는 빈칸추론은 말이 빈칸이지 실제로는 '주제추론'입니다.
그런데 English Writing의 경우 topic을 부각시키는 매우 객관적이고 체계적인 방법론이라는 게 있습니다.
사실 이걸 공부하면서 빈칸추론 문제를 풀 때 선지를 보지 말고 직접 주제를 추론한 다음 그 다음 선지를 봐야 안 낚이지
그냥 보면 거의 낚인단 이야기죠.
\vspace{5mm}

뭐... 내년에 +1수 하실 분들은 국영수탐 모두 수능을 넘는 상위과정 다 공부해야할 것이다란 이야기가 되겠습니다.
정말로 상위권이 되고 싶다는 분은 성문종합영어 잘 도전해보세요. 지금 날린다는 영어강사들도 젊은 시절에는 이거 공부 안 한 사람은 없음.
재밌는 건 다 성문종합영어 까는 사람들도 나중에 자기가 강의하는 내용이 성문스럽게 변하고 있다는 것.
\vspace{5mm}










\section{빚개념에 대해서}
\href{https://www.kockoc.com/Apoc/518747}{2015.11.30}

\vspace{5mm}

5수생이 있다칩시다.
\vspace{5mm}

보통 이런 경우 어떤 관념이 문제나면
자기가 날려먹은(?) 4년만큼의 본전을 챙겨야한다는 \textbf{보상심리} 라는 게 있습니다.
그래서 목표치를 더 높게 잡으면서 자기가 수험고수이니 더 많이 하겠다 그래서 꼭 성공해야한다는 강박관념이 있죠.
\vspace{5mm}

사실 생각해보면 별 의미없는 자기학대에 불과합니다.
목표치를 높게 잡는다고 해보았자 그 4년이 빛나는 것도 아니죠.
4년동안 공부했다면 당연히 구력은 있습니다, 그러나 '실패'도 학습되었을 뿐더러 '해결되지 않은 원인'이 있단 거죠.
\vspace{5mm}

그럼 어떻게 해야하느냐.
\vspace{5mm}

일단 4년은 잊어버려야합니다. 그냥 4년동안 병원생활, 식물인간, 징역살이, 외계인에게 납치... 당했다고 생각하는 편이 나아요.
그 4년은 경제학적으로는 매몰비용입니다. 뭘 하더라도 사실 복구는 못 해요. 심지어 성공한다 하더라도 4년이 의미있느냐 그건 아닙니다.
다들 이런 매몰비용을 복구하겠다고 목표를 무리하게 잡는 걸 넘어 학습방법도 터무니없는 걸 선택하기 때문에 실패하는 겁니다.
\vspace{5mm}

저 4년은 안 돌아옵니다.
내년에 시험치는 사람이면 겸손하게 자기가 고3과 똑같다고 여기세요.
\vspace{5mm}

만약 개인의 성찰과 반성, 그리고 기본 지식을 쌓는 과정에서라면 유의미하다고 반문할 수 있긴 하겠죠.
그러나 이 경우 손해는 더 큽니다. 4년동안 해서도 되지 않는 \textbf{실패도 따라오기 때문}입니다.
시험을 여러번 쳐도 안 되는 이유는 공부가 부족한 것도 있지만, \textbf{실패하는 패턴을 반복하는 게 가장 큽니다}.
학원에서는 공부하는 방법이나 지식을 전수해주겠지만, \textbf{학생 개개인의 실패 패턴을 지적해주거나 잡아주진 못 합니다.}
본인의 과제죠.
\vspace{5mm}

하지만 이걸 하는 건 자존심을 포기하는 것에 근사하기 때문에 혼자 하기 힘들 수도 있습니다.
달리기를 잘 하는 친구에게 너는 달릴수록 불행해지니까 달리지 마라고 하거나
아주 얼굴이 예쁜 여학생에게 자네는 얼굴이 불행의 근원이니 차도르를 쓰고 알라후 아크바르를 외치도록 하는 것과 동급입니다.
하지만 그런 자존심을 포기하고 여태껏 살아온 방식을 과감히 바꾸지 않으면 실패는 또 반복되죠.
\vspace{5mm}

빚을 못 갚으면 파산신청하고 갱생하는 게 낫습니다.
내년에 다시 시험 응시할 분은 과거는 싹 잊으세요. 과거에서 챙길 건 오직 교훈, 그리고 자기의 실패하는 패턴에 대한 반성입니다.
그래서 다시 시작해야 하는 겁니다.
\vspace{5mm}

과감하게 구식무기를 버리고 신식으로 갈아타면서 자기를 잊는 사람은 살아남겠지만
계속 한탄만 하면서 자기를 너무 사랑하는 사람은 또 실패합니다요.
\vspace{5mm}




\section{최근의 동태에 대한 비판입니다만}
\href{https://www.kockoc.com/Apoc/518822}{2015.11.30}

\vspace{5mm}

강사 개개인이 착하건 악하건 관계없이
누구 강의가 최고예요... 라는 식의 글이 올라오는데 그건 바람직하지 못 한 행태입니다.
최소한 그게 검증되었다라고 볼 근거도 없지 말입니다.
\vspace{5mm}

여기서 믿을 수 있는 건 일지 꾸준히 쓰고 자기 성적 공개한 케이스이지
나머지는 믿을 것도 없습니다.
\textbf{사이트가 가장 힘들 때에도 꿋꿋이 활동하고 자기 공부하면서 소신껏 밀어붙여 올라간 케이스면 몰라도}
나머지는 솔직히 아니라고 생각하는데요.
\vspace{5mm}

그럭저럭 사이트가 잘 버티고 하한선 찍고 상승세 찍을 때나 와서 모 강사가 최고예요라는 행태는
그 강사가 제가 좋아하는 사람이건 안 좋아하는 사람이건 바람직하지 않은 행태죠.
아무리 좋은 강사도 맞는 사람이 있고 안 맞는 사람이 있습니다.
사설강의가 뭐 한두푼도 아니고. 듣고 싶으면 여름 때 정말 부족한 것만 골라 들으세요.
\vspace{5mm}

이제 또 강의 교재 홍보철이 시작되죠.
이 사이트도 졸라 모니터링당할 거예요. 자기 영업방해되는 글 있나없나
설마... 돈에 눈먼 사람들은 별 짓 다 합니다만요.
유감스럽지만 가장 한적해야 할 이 사이트조차도 제 눈에는 이미 \textbf{자본의 마수가 뻗쳐있습니다}.
\vspace{5mm}

약 1년 전인가. 일지조차도 시비 걸었던 사람들이 누구인지 전 기억하고 있죠(시비걸 이유가 없을텐데)
심지어 그 때 학습론도 하도 시비걸어대고 해서 짜증나서 지웠습니다.
그 사람들 누군지 몰랐다는 건 오산. 걍 싸우기 싫어서 냅둔 겁니다. 어떤 성향이고 어떻게 활동했는지 똑똑히 기억합니다.
어차피 인생 말아먹을 인간들이라서 말도 안 한 거예요.
그런데 그 사람들이 왜 '제가 영리사업도 안 취하고 그냥 애들 불쌍해서 이렇게 해라 지시하는 것'을 시비걸었을지는 다 알고있지 말입니다.
\vspace{5mm}

몇몇 영리활동 취하는 사람들은 일지조차도 무슨 목적이다 생각하실 건데 그딴 것 없습니다.
공부하는 패턴이든 방법은 님들이 상상하는 이상으로 다 알고 있으니까요. 해서는 안 되는 공부법까지도 다 정리했구만 무슨.
다만 질문하고 상담하려면 일지는 쓰는 성실한 사람이 아니면 안 받아준다는 건 당연합니다.
올해 실패한 사람들도 있지만(아마 조만간 연락받고 또 소통하겠죠) 대부분은 생각한대로라서.
이런 걸로 돈벌려는 사람이야 딱 한 사람만 집어서 성공사례라고 하면서 바로 영업질 들어가는 추한 짓이나 벌이겠죠.
그런 건 영 관심이 없지 말입니다. 핏덩이들 어떻게 성장하나 그거 보는 게 재미지 뭔 저런 걸로 수험재벌해서 나르시시즘 빠지게?
\vspace{5mm}

그리고 제발이지 사교육에 너무 매몰된 짓 좀 하지 맙시다. 나이가 몇살인데 강사 강의 교재찬양만 하고 있는지.
가정환경이 좋아서 경제적 어려움이 없이 사교육받은 걸로 강사추천하고 다니는 건 제 입장에서는 뭐 저 병맛 하기 전에
과연 그 사람이 경제적으로 정말 궁핍해지는 상황이 오면 극복할 수 있을지 의문이지 말입니다.
제가 저성장 가치주로 보는 사람들은 대부분 '가난'을 겪고 막장환경에서도 공부하는 사람들입니다.
이런 사람들은 각성해서 제대로만 캐리되면 남들 10년치를 1년에 달성할 수도 있어서 과거의 실패는 별 의미가 없죠.
벗어나기만 한다면.
\vspace{5mm}

일전엔가 어그로 끄는 모 분이 대화방에서 일지 결과 가지고 시비를 거시던데 수준 참 인증하시더군요.
뭔 일지 가지고 결과 공개해서 .... 다 자기들 수준이죠. 그 딴 건 관심도 없습니다.
저도 학교 스펙은 꿀릴 건 없는 사람이라서 걍 말하지만 웬 잡졸들이 별 것도 아닌 거 가지고 수험생들 공포심 이용해 장사하는 건 별 관심 없어요.
그보다 더 관심이 있는 건 \textbf{절망적 상황에 처했다고 생각하는 사람이 그 고난을 극복하고 올라가는 케이스}입니다.
그럼 여기서 노하우를 얻으려고하느냐? 아뇨. 노하우는 의미가 없습니다. 그리고 그런 노하우는 너무 넘쳐서 탈이죠.
중요한 건 본인들이 실제로 넘어서느냐이겠죠. 어디로 올라가느냐보다도, 본인이 '상승하긴 하느냐' 이게 관건입니다.
\vspace{5mm}

그리고 인생의 가치라는 건 저걸 넘어서는 게 없죠.
돈 많이 번다 그걸로 과시한다... 뭔 진주 물고 있는 돼지도 아니고.
어차피 누구나 다 죽기는 마찬가지입니다만 분명한 건 살아있을 때 얼마나 많은 도전하고 벽을 넘어서느냐하는 것이죠.
\vspace{5mm}

물욕이나 세속적인 것에만 사로잡히는 돼지는 인간들이 아닙니다. 아직 이걸 구분 못 하는 사람들도 널렸을 것입니다만.
살다보면 인간의 형체를 했지만 악마인 경우도 있지만 걍 짐승이나 가축에 불과한 경우도 많다는 걸 알게 됩니다.
가끔 보면 수험을 이런 식의 극복이 아니라, \textbf{'자본'의 먹이}로 타락시키는 분들이 있는데요.
왜 그렇게 사나 모르겠죠.
물질을 지배해야지 물질에 지배당하면 안 됩니다.
상품을 이용해야지 상품에 끌려다니면 안 되지요.
\vspace{5mm}

적당히들 해처드시길 바랍니다.
\vspace{5mm}






\section{고1수학 풍산자 풀고있는 팀}
\href{https://www.kockoc.com/Apoc/518860}{2015.11.30}

\vspace{5mm}

풍산자 다 풀거나 혹은 풀고 있을 때고 고난이 있으면
올림포스 문제집 사서
\vspace{5mm}

\href{http://ebsi.co.kr/ebs/lms/lmsx/retrieveSbjtDtl.ebs?sbjtId=S20140000145&Clickz=C101}{링크}
J 모 강사 강의 듣고 따라가시는 것 권하겠습니다.
다 따라가지 말고 해당되는 단원만 가지고 올림포스 사서 거기서 문제되는 것만 발췌해 들으세요
\vspace{5mm}

사실 수학 답 없으면 그냥 EBS 올림포스 따라가는 게 답이라고 생각함. 강사들도 그리 나쁘지 않고 무엇보다 공짜.
올림포스 문제도 어려운 건 상당히 어렵습니다.
그리고 적통과 기벡은 제가 유일하게 들을만하다라고 하는 모 선생님 강의는 아직도 EBS에 남아있습니다.
\vspace{5mm}

이렇게 하면 사설에 돈 쓸 필요 없습니다.
그 돈 쟁여두었다가 내년 여름방학 이후에 평 좋은 선생 것만 몰빵해 들으세요.
그 점에서 소위 프리패스라는 건 별로 권하지 않습니다. 이게 나중에 선택권을 엄청나게 제약합니다요.
\vspace{5mm}




\section{EBS에 꿀강의 많으니 그거나 들으셈.}
\href{https://www.kockoc.com/Apoc/520024}{2015.11.30}

\vspace{5mm}

소위 시중에 파는 비싼 것들도 별로 크게 나을 건 없음.
\vspace{5mm}

전에 이것 가지고 언쟁붙을 때 재밌는 게,
EBS 욕하던 사람들에게 그럼 누구 강의가 어떤 문제가 있느냐하면 거기서 \textbf{다들 어버버대더라} 그겁니다.
도대체 EBS란 말이 나오자마자 무조건 EBS 까기만 하던 사람들이 강사가 누군지도 모른다라 :)
\vspace{5mm}

특히 수학은 과거에 비해서 달라진 게 없고 $-$ 내용이 빠지면 더 빠졌기 때문에 $-$
적당히 강사조합만 잘 짜고 인강 들으면서 딴짓만 안 하면 저렴하게 양질의 수업을 받을 수 있습니다.
교재는 수특 쓴다고 하지만 강사는 정작 자기 프린트로 보충하는 경우가 많아서 별 문제는 되지 않아요.
인강 듣다가 딴짓 안 하는 것만 지키면 뭐 딱히.
\vspace{5mm}

이런 조언이야 '재벌'되고 싶어하는 사람이야 싫어하겠죠.
그래서인가 가만히 공개 게시판이나 사람들 많은 커뮤니티에 보면 "돈 안 들이고 가는 저런 방법"조차도 비난하는 병맛들이 있고
몇몇 지능파는 슬그머나 xxx도 좋지 않느냐 혹은 ○○○는 어때요라고 해서 홍보를 하는 경우도 있덥니다.
아마 여기 게시판도 슬그머니 모니터링당하고 있을 거예요. 거기 생리가 그런 곳이라서
\vspace{5mm}

장담하는데 EBS 강의만 충실히 들어도 딱히 사설보다 모자라다고는 못 느낄 것이고
오히려 몇몇 강의는 훨씬 좋다는 판단도 들 겁니다. 저기도 강의 대충하는 곳은 절대로 아니기 때문에.
\vspace{5mm}

저기서 이채형, 손광균, 고동국, 서정원, 정승제도 뭐 들을만한 강의고
교재 집필진으로 유명한 김원중 선생님도 수리 가형 찍고 있네요.
들을만한 강의 없다고 핑계댈 그런 시대는 아닙니다.
\vspace{5mm}

이제 또 장사치들 난리치고 홍보하고 겁주고 그럴 시기가 왔는데
마음에 안 드는 친구들이나 그런 거 소비하라고 하고
님들은 소박하게 시중교재나 열심히 풀고 강의는 EBS로만 해결해보시길 바랍니다.
그래도 충분히 넘치니까요.
\vspace{5mm}

설마 EBS 추천했다고 비추먹는 그런 어이없는 일이 벌어지진 않겠지?
하기야 내년에도 어떻게 팔아먹고 부자될까 하는 돼지들 입장에서는 꾸에엑 소리가 나오는 대목입니다만.
\vspace{5mm}






\section{선택과목 고를 때는}
\href{https://www.kockoc.com/Apoc/521445}{2015.12.01}

\vspace{5mm}

백분위나 표점이 아니라
'상위권이 얼마나 적으냐', '난이도는 얼마나 되느냐'
이거 기준으로 가야합니다요.
\vspace{5mm}

아까 챗방에서 지2 백분위 96이라고 하면서 갓폭이 아니라 조폭(...)이라고 하는 아우성이
전체 2과목의 백분위가 공개되면서 간사하게 바뀌는 드라마가.
\vspace{5mm}

일단 비교가 되지 않죠.
올해의 경우 생2는 100일을 돌려도 2등급 따기도 힘든데
지2는 EBS 강의 하나 빨리 들어주고 한달 돌리면 정말 공부 안 한 게 아니면 만점 받기는 어렵지 않았습니다.
여기서 절약되는 시간만 보더라도 지2가 압도적으로 나았죠.
혹자 생1이나 화2 백분위 가지고 어 좋은 과목 아냐.... 그것들은 이미 수학보다도 어려워진 과목들입니다.
왜 생2 하는 것 안 말렸냐고 저에게 그러면 할 말 없습니당(...)
지2 해서 망했다고 하시는 분은 생2 가면 100$\%$ 망했죠.
사문과 생윤 선택은 제가 올해 극도로 말렸죠. 결과야 뭐.
(그러니까 아말듣 인생망 이래보았자 소용없음. 난 분명 충고했음)
\vspace{5mm}

제가 조언드리는 건 중하위권 대상이지 상위권은 아닙니다. 전 수험에 있어서는 공산주의자(?)라서리.
우선 생1, 생2, 화1, 화2를 기피하는 이유는 간다. 노력으로'만' 되지 않기 때문입니다.
시험 킬러문제를 공략할 수 있는 교재나 강의가 거의 없습니다.
사문과 생윤도 마찬가지요. 변별력 가를려고 온갖 장난질 다 쳐놓기 때문에 노력과 결과가 비례하지 않습니다.
내년에 선택하실 분들은 인기가 덜하면서도 상위권이 덜 포진한 것 고르세요.
다시 말해서 사교육 시장이 덜 형성되거나 특목고나 자사고에서 좋아하지 않는 과목을 고르는 게 낫습니다.
\vspace{5mm}

수능이 쉬워졌다... 그거야 수학만 전부인 줄 아는 구시대의 잔재나 하는 말이죠.
실제로는 더 어려워진 것입니다.
과거에 서울대 합격한 사람이 요즘 수능 치면 과연? 솔직히 저도 장담 못 해요.
\vspace{5mm}

무조건 제 말이 다 맞다는 건 아닌데
노인네가 충고하는 건 그만한 이유가 있다는 것 정도는 알아두시길.
\vspace{5mm}









\section{강사 오개념 발생하는 이유}
\href{https://www.kockoc.com/Apoc/524777}{2015.12.03}

\vspace{5mm}

강의 2시간을 준비하려면 20시간을 쏟아부어야하는데
현실적으로 강의 많이 하는 사람이 그럴만한 시간이 있을 건가.
오히려 수학이 가르치긴 더 수월할 겁니다. 교과서적 개념을 넘어서 원시적인 것까지 뚫고나면
그 다음이야 어떤 문제가 나오든 고교과정 수준에서는 해설할 수 있기 때문.
\vspace{5mm}

하지만 국어, 영어, 탐구는 아니죠.
특히 탐구는 그 내용들이 논리적으로 다 추론되는 게 아니라 일일히 자료 찾아보고 가야하는 것이라서리.
강사가 박사급이거나 강의를 적게 하는 대신 엄청 공부하는 스타일이라면 믿을만하겠지만 그게 아니면 저라면 안 믿을 것입니다.
\vspace{5mm}

수능 시장이 이상한 것이죠.
공무원 사법 행시 CPA 쪽은 강사들이 자기 자랑할 시간도 없고 사실 홍보질도 필요없습니다.
저긴 정말 실력대로 검증되고 있어서리. 교재 오탈자에도 더럽게 욕먹기 때문에 강사들이 짜깁기 책이라도 열심히 만듭니다.
만약 문제 하나라도 빗나갔다... 그대로 퇴출당하거나 복귀하는 데도 상당히 많은 시간이 걸리죠.
\vspace{5mm}

강사든 교재든 공부하는지 그거보고 고르세요. 대부분 강사, 교재평은 3/4가 걍 홍보에다 알바질이라고 보는 게 정확합니다.
그러니까 강의도 솔까 사설들을 필요 없고 EBS 들으면 된다고 얘기하는 겁니다.
EBS는 거기 담당 PD도 졸라 까다로운 걸로 알고 있어서 오히려 검증 면에서는 신뢰도가 높은 편입니다.
\vspace{5mm}

그리고 교재는.
저라면 EBS만 보겠고 사실 신사고, 천재, 두산동아 등에서만 내는 교재 위주로 가겠습니다.
저자보다도 출판사를 믿겠음. 여기 교재들도 오류가 없는 건 아니지만 그나마 수용할 수 있는 수준임.
그냥 수험생을 돈으로 보고 대충 책부터 내자하는 경우라면 볼 필요가 없다고 봅니다.
이거 신사고 알바란 얘기 나올 것 같은데 사실 신사고에서 낸 책들이면 웬만한 것 다 커버가 됩니다.
거기다가 EBS까지... 이거 다 볼 시간이나 있을까?
\vspace{5mm}

탐구에서 물지와 삼사 조합 얘기하는 이유는 별개 아닌 게
물리는 애당초 지엽 문제로 오답시비 걸릴 게 별로 없고,
지구과학도 사실은 과학이라기보단 매우 정밀한 시나리오라서 여기 나온 지엽도 실제 지엽은 아닙니다요.
그에 비해 화학과 생물은 고교과정의 것들이 전부가 아니라는 문제가 있죠.
삼사조합도 마찬가지입니다. 역사가 해석에 따라 달라지긴 하더라도 '기술'로서의 역사가 바뀌는 건 아니거든요.
누가 타임머신 타고 가서 조작질 안 하는 이상 오답시비 걸릴 일이 별로 없음.
\vspace{5mm}

재작년 이맘 때쯤인가 모 게시판에서 탐구가 수능을 좌우한다라고 했을 때 비웃던 사람들이 지금은 뭐할지 모르겠음.
그 당시에 EBS가 좋다라고 얘기해주고 앞으로 탐구가 정말 중요하다 생명과 화학 분산시키려고 어렵게 낸다라는 예측은 맞았기도 하지만
아마 야매교재도 점유율이 떨어질 겁니다. 그것들이 엉터리라는 게 참 많이 검증들 되어서리.
\vspace{5mm}







\section{자존심을 버려야 자신감이 생긴다.}
\href{https://www.kockoc.com/Apoc/524793}{2015.12.03}

\vspace{5mm}

둘 다 성을 自로 써서 그런가 같다고 여깁니다만 실제로는 그렇지 않죠.
\vspace{5mm}

자존심이 강하면 자신감을 잃기 쉽죠.
반면 자존심을 버리면 참된 자신감을 얻을 수 있죠.
\vspace{5mm}

고민상담해보면 대부분 문제가 자존심입니다.
여전히 자기가 잘났다, 내가 사랑스럽다... 라는 것을 말하죠.
그리고 실제로는 조언보다는 그런 자존심을 네가 지켜주었으면 좋겠다라는 메시지를 읽습니다만.
\vspace{5mm}

해결책은 두가지이죠.
하나는 자존심을 완전히 버리는 것 $-$ 쉽지가 않음
다른 하나는 자존심을 높이고 실제로 그렇게 자기 스펙을 높이는 것 $-$ 한계가 있음.
\vspace{5mm}

사람 보면서 아 이 인간은 망하겠군이라고 평가하는 첫째 기준은 자존심이 크냐 아니냐 입니다.
자존심이 큰 사람은 반드시 망합니다요. 자존심이 조금이라도 상하는 일이 있으면 이성을 잃고 어리석은 선택을 하거든요.
이 사람들은 자기 실력을 키우기보다는 다른 사람의 평가, 자기의 외모, 아울러 어떤 부나 권력의 과시를 합니다.
그렇게 안 하면 \textbf{못 견디기 때문입니다}. 그리고 그 맛에 빠지다보면 나중에는 겉은 화려한데 속은 파삭 쪼그라드는 거죠.
\vspace{5mm}

남을 칭찬할 줄 알고 때로는 고개를 숙일 줄도 알아야 행복한 게 별 게 아닙니다.
자존심 자체를 최소화하면서 자기가 얼마나 윤리에 부합하게 사는가 그거 하나만 보면 되지
나머지는 걍 신경 꺼버리면 애당초 걱정할 것 자체가 사라져버리기 때문이죠.
\vspace{5mm}

자살시도 자체도 자존심 때문이죠.
사실 객관적으로는 별 것도 아니지만 본인 입장에서는 프라이드에 상처를 입으면 죽고싶어하는 거죠. 마치 세상이 다 끝난 듯
\vspace{5mm}

\textbf{만신창이가 되더라도 나중에 이기면 된다..}. 라는 마인드로 가야지, 내가 잘 나갔었는데 생각하는 건 아무 소용도 없습니다.
\vspace{5mm}






\section{메모 : 시스템}
\href{https://www.kockoc.com/Apoc/528655}{2015.12.05}

\vspace{5mm}

순수히 경험과 관찰에 의한 짤막한 기록이니 걍 믿거나 말거나
\vspace{5mm}

\begin{enumerate}
    \item 나이먹는다고 사람이 성장하는 것은 아님, 오히려 나이에 맞게 성장 못 하면 좌절함.
    성장은 연속함수가 아니라 불연속함수, 경험이 쌓이면서 내적모순이 심화되면
    어느 순간에 이를 정리하기 위해 각성하는 순간이 있는데 이 때야 비로소 성장하기 시작한다고 할 수 있음.
    각성하지 못 하면 성장은 없다.... 보아도 좋음. 남자들이 평생 철이 들지 않는단 이야기는, 평생 성장할 수 있단 이야기.
    \vspace{5mm}

    \item 다수에게 강의하는 것보다도 한 사람을 제대로 가르치는 게 더 어려움.
    다수는 50명이 있다고 하면 그 중 최소 20$\%$는 알아서 올라가기 때무네 그걸 자기 실적으로 가산하면 됨.
    그러나 한 사람은 정말 제대로 공부한다는 건 '인생'이 바뀌는 문제임. 상당한 난이도가 있음.
    가르치거나 상담할 때는 반드시 충격을 주지 않으면 안 됨. 그렇지 않으면 각성할 수 없고 각성하지 못 하면 성장은 안 함.
    \vspace{5mm}

    \item 대기만성은 맞는 말이긴 한데...
    조기에 성숙한 사람은 크게 성장할 수 없음. 초기에 성공만 한 사람은 한번 실패하면 계속 곤두박질함.
    반면 오래 실패하면서도 스케일을 넓힌 사람이 그 벽을 넘어서기 시작하면 그 때부터는 승승장구.
    다만 이건 사람마다 시기 차이도 있거니와 각성하지 못 하면 글쎄, 그리고 큰 그릇이 만들어지기 전에 작업이 중단되면?
    \vspace{5mm}

    \item 성공과 실패는 2가지 유형이 있음.
    좋은 시스템에 편승해서 성공하는 A, 다른 하나는 좋은 시스템을 자작하는 B.
    나쁜 시스템에 올라타서 실패하는 C, 다른 하나는 나쁜 시스템을 자초하는 D.
    전반적으로 A, D 비율이 높음. B와 C는 적음.
    수험 뿐만 아니라 전반적으로 성공하는 사람들은 B임. 당연히 희소할 수 밖에 없음.
    수험 상담을 해보면 D가 많음.
    잘못된 시스템은 본인의 자존심과 직결되어있어서.
    그래서 자존심이 완전히 날라갈 때까지 실패를 자초함.
    \vspace{5mm}

    \item 나쁜 시스템을 자초하는 사람은 그냥 군대에 가거나 현강 학원 가는 게 나음.
    군대에 가면 기존의 잘못된 시스템을 리셋할 수 있는 장점이 있고
    현강 학원에 가면 공부잘하는 애들을 모방해서 좋은 시스템을 주입당할 수 있기 때문임.
    독학재수는 스스로 좋은 시스템을 만들 수 있거나 그런 시스템에 길들여진 사람이 해야 효율이 쩔지,
    공부 시스템이 안 잡혀있고 본인이 실패를 반복하는 사람이면 자살행위임.
    \vspace{5mm}

    \item 자기가 B라고 착각하는 경우가 많음. 그런데 B 정도면 이미 '창업주' 역량이 있음. 공부할 필요가 있나?
    SKY만 보아도 보통 \textbf{A$\rightarrow$D 케이스}가 많음.
    엄마의 skirt wind로 용케 좋은 대학은 갔는데 그 다음은 엄마도 커버 못 쳐주고 자기도 자존심 하나로 잘못된 시스템 고집하다...
    그 다음으로 \textbf{C$\rightarrow$A}는 많이 관찰\textbf{, C$\rightarrow$B 케이스는 딱 두 건}만 확인\textbf{.}
    \vspace{5mm}

    \item \textbf{7.} 과거에 사람들이 텍스트를 반복해서 암송, 필사하게 한 것은 성공적인 시스템의 틀을 주입한 것이라 보면 됨.
    지금 생각해보면 과거시험치는 사람들이 사서삼경 암송한 게 뭔 바보같은 일이냐 하겟지만
    사실 그 텍스트들이 대단한 것은 시스템의 모듈에 해당.
    그런데 요즘 학생들이 특정 텍스트들을 반복학습하면서 시스템 구축할 수 있긴 있나.
    보통 입시 상담은 어떤 좋은 시스템에 올라타야하느냐 묻는 정도니까.
    \vspace{5mm}

    \item 8. 언제까지 콕콕에서 상담질을 할지는 모르지만 상담가질을 하고 싶은 사람은 위에 준해서 생각해보시면 됨.
    \vspace{5mm}

\end{enumerate}


\section{올라가는 게 중요한 것 아닌가}
\href{https://www.kockoc.com/Apoc/529739}{2015.12.06}

\vspace{5mm}
\begin{itemize}
    \item[] A : 금수저 환경으로서 어린 시절 조기교육, 그 추이로 연고대 합격 가능 : 서울대 합격
    \item[] B : 막장 환경에서 뒤늦게야 공부시작하고 5수 넘김, 대학도 못 갈 경우였는데 혼자 힘으로 공부해서 서성한 합격
\end{itemize}
\vspace{5mm}

이 경우 누굴 선택할지는 뻔하다. 나라도 B를 고른다.
\vspace{5mm}

A는 저 상태에서는 이제 더 이상 상승할 수는 없다. 만약 그가 B의 역량이 있었다면 이미 유학가서 미국에서 한가닥했을 것이다.
그러나 그는 투입된 것치고는 효율은 낮다.
반면 B는 저 결과만으로 초라할지 모르지만 투입된 게 마이너스인데도 끝내 그걸 플러스로 돌리고 올라갔다.
대학수험만이 전부는 아니다. 이런 친구는 그 이후에도 꾸준히 도전해서 올라가고 실제로 내가 관찰해보는 사람들이 이런 케이스다.
\vspace{5mm}

물론 학벌로 치자면 어그로 끄는 이야기지만 국내대학에서는  '서울대'에 들어가면 그 이하는 대학으로 안 보인다.
그러나 인생 전체로 치자면 서울대가 중요할까, \textbf{장기적으로} \textbf{꾸준히 성장할 수 있느냐}가 더 중요한 것 아닌가?
20대에 서울대 현역으로 합격했는데 거기서 성장이 멈춘다면 이건 보통 심각한 게 아니다.
반면 그냥 똥통대학에 들어갔다쳐도 그 사람이 엄청난 포텐셜을 지니고 있으면 이 사람은 가히 새로운 창업주가 될 것이다.
\vspace{5mm}

개인적으로 소위 "머리가 결정한다"거나 "양극화론" 같은 걸 까는 이유는
이것들은 엄밀히 말해서 객관적 기술보다는, "그러니까 하류층은 공부하지 말고 거지같이 살아라"하는 걸 조장하기 위한
카스트 제도에 가깝기 때문이다.
한국사 공부하다보면 왜 왕족과 귀족이 불교를 환영했는가 하는 설명으로 "현 신분제도를 정당화"하는 것임을 알게 되는데.
진짜 기득권층은 절대 하류층들이 노력 따위는 못 하도록 한다. 모든 것을 \textbf{다 '유전자'로 돌려 바꿀 수 없는} 것이라고 세뇌한다.
\vspace{5mm}

여기서 상담하는 사람들 $-$ 이제 일지쓰는 사람 빼고는 거의 다 하지 않겠지만 $-$
얘기하다보면 문제가 되는 건 못 배우고 몰라서가 아니라, 배우지 말아야 할 것을 배우거나 잘못된 저주에 세뇌당한 경우.
말도 안 되는 이야기에 집착한다거나 본인에게 별 도움도 안 되는 강박적 메시지를 광신하는 경우가 정말로 많다.
\vspace{5mm}

한번에 서울대에 갈 수는 없다(굳이 서울대에 갈 필요도 없지만)
하지만 여러번에 걸쳐 올라가면 갈 수 있다고 확신되면 시간이 걸리더라도 체계적으로 공부하고 올라가면 되는 것이다.
\textbf{요컨대 과거보다 더 높이 올라가면 그걸로 일단 스타트하는 것이고, 그 과정에서 자기를 묶은 정신적, 물리적 족쇄를 벗으면 된다.}
공부해서 과거보다 올라간다, 나아진다라고 확신만 들면 그대로 공부하면 되는 것이다.
그렇게 하나하나 성공사례를 스스로 만들다보면 가속이 붙어서 나중에 급상승하는 것이지
처음부터 다 해먹을 수 있을리는 없지 않은가.
\vspace{5mm}

실패하는 이유는 두가지이다.
\begin{itemize}
    \item 첫째는 서두르기 때문이다.
    \item 둘째는 어려운 것부터 하려고 하기 때문이다.
\end{itemize}
\vspace{5mm}

마음이 초조할수록 오히려 더 오래 걸리는 쉬운 길을 택해야 한다.
빨리 가야한다고 암벽등반하다가 추락해버린 사람들이 얼마나 많은가.
\vspace{5mm}

물론 막장환경에서 공부가 안 되는 사람들도 계속 안 되는 굴레라는 게 있다.
그런데 엄연히 말하면 이건 유전자가 아니다. \textbf{환경 문제가 정말로 크다.}
환경 개선을 하지 않거나, 아니면 그런 환경에 순응해버린 노예의 마인드를 청산하지 않으면 이게 계속 발목을 잡는다.
어느 쪽이든 마음만 먹어서는 곤란하다. 조그마한 과제라도 성공시켜나아가야만 변화가 있다.
\vspace{5mm}





\section{정말 돈이 없다면 모르겠지만.}
\href{https://www.kockoc.com/Apoc/530358}{2015.12.06}

\vspace{5mm}

정말 돈이 없으면 모르겠지만 그게 아니면 이 시기는 알바 뛸 게 아니라 다시 공부 달리고 있어야 할 시즌이죠.
사실 조금만 생각해보아도 알 수 있음.
\vspace{5mm}

11월 $\sim$ 2월 : 대학생 인력공급 넘침, 업자들의 착취가 가능하고 추워서 일하기도 고달픔.
이런 시기에 공부 안 하고 일하면 올해 5$\sim$11월까지 달렸던 \textbf{공부 감각이 모두 소실되어버리죠}.
그래서 3월달에 예열한다고 해보았자 감각 찾는데 3개월 날림.
그래서 실제로 6월달부터 시작. 그대로 반년이 증발되어버림, 이래놓고 또 입시 실패했다고 엉엉댐.
\vspace{5mm}

차라리 지금 공부해서 내년 3월까지 달리고 그 때 알바 뛰는 게 낫지 않나 생각이 들죠.
\vspace{5mm}

그런데 이제 알바 뛰고 공부하겠다... 그 알바비 많아보았자 3개월 300만원.
그런데 11월부터 2월까지 공부해서 얻는 게 300만원의 10배는 넘어서지 않을까 싶은데.
이거 조금만 생각해도 도달하는 결론임. 본인이 정말 돈이 필요하다면 모를까 그게 아니면 '뻘짓'하는 거죠.
\vspace{5mm}





\section{인강을 들을 때 처음부터 배속 높이진 마실 것.}
\href{https://www.kockoc.com/Apoc/530936}{2015.12.06}

\vspace{5mm}

빠른 배속으로 돌려 듣는 것이 뇌활성화에 좋다는 것은 과학적으로 검증되긴 했으나
그건 어디까지나 "로직을 제대로 이해했을 때"라는 걸 전제합니다.
\vspace{5mm}

인강을 듣는 이유는 "생각하는 방법을 보고 따라하기" 위한 겁니다.
단순히 듣고 보는 게 아님, 탁월한 강사들은 어떻게 문제를 읽고 분석하고 생각하고 풀어나갈지 그걸 복기해주죠.
일부 머리가 빠른 사람들은 그걸 지겨워하면서 2배속 이상 돌리는데
\vspace{5mm}

이렇게 하면 내용이 일단 해마에 저장은 됩니다만 아마 곧 잊어버릴 겁니다. 임팩트가 없기 때문이죠.
실력있는 선생일수록 중요한 부분은 매우 '슬로우'하게 저음 깔면서 강조합니다.
이걸 그대로 따라가줘야 제대로 인지되면서 내 것이 됩니다. \textbf{정확히 말해 그 강사의 사고 프로세스를 내 머리에 복사하는 것이죠}.
그런데 이걸 재생속도를 높인다는 건 야동을 10배속으로 돌리는 것과 똑같은 참사(?)를 불러일으킬 수 있습니다.
\vspace{5mm}

일단 강의를 들으려면 제대로 들어야합니다. 정말 중요한 부분은 1배속으로 여러번 반복해 들어야하죠.
대략 $1.2\sim1.4$배속으로 가야지 처음부터 2배속으로 가면 현명할 것 같지만 그만큼 흡수율이 떨어지는 것입니다.
강의가 세뇌라고 해서 무작정 부정적인 것만은 아닙니다. 강의를 듣는 목적이 '제대로 세뇌당하기' 위해서입니다.
세뇌당하고 나서 그 다음 배속을 높여 반복해 들어야 내 것으로 만드는 것이죠.
시간 단축하고 이 사람 저 사람 다 듣겠다고 빨리 듣고 음미는 안 합니다. 그러니까 발전이 없는 것이죠.
\vspace{5mm}

처음에는 배속을 적당히 하면서 들어주고, 나중에 복습용으로 빨리 들어주는 게 낫습니다.
처음에 제대로 들었다면 그 다음에야 1.8배속 이상 가더라도 뇌에서 인지하면서 숙달이 되는 것이죠.
이렇게 활용한다면 사설강의도 별 필요도 없습니다. EBS 강의로 쏠쏠한 효과를 누릴 수 있죠.
사설강의는 여름 이후에 필요한 것만 골라서 알짜만 챙기면 됩니다.
\vspace{5mm}

그리고 강의를 들을 때는 처음에 비판적인 태도는 갖지 마십시오. 그냥 내 머리에 강사가 말한 것을 덮어쓰세요.
어느 정도 세뇌당한 다음에 비판해도 늦진 않습니다.
강의를 들을 때는 강사가 말하는 방식, 속도, 어조부터 시작해 제스처까지 그냥 흉내내는 것도 좋습니다.
\vspace{5mm}






\section{하나마나한 이야기}
\href{https://www.kockoc.com/Apoc/531614}{2015.12.07}

\vspace{5mm}

자꾸만 쓸데없는 걸로 고민하는 사람들이 많죠.
\vspace{5mm}

계획이 실패하는 이유는 간단합니다.
꿈에 부풀어서 한시간 동안에 다 자란 돼지 한마리를 다 먹을 거야... 문학적으론 가능하겠지만 현실적으로 가능하나요?
아무리 배가 고프더라도 저걸 먹을 수 없죠.
계획이 실패하는 사람들은 사실 실천도 안 해본 사람들입니다.
\textbf{계획량  < 실제 업무량 이라는 부등식을 지켜야하는데 보통 계획을 자기 능력치의 5배 이상 잡아놓고 공부가 안 된다고 그러죠}
지금 알바 뛴다 공부 3월에 시작한다는 사람들도 이런 케이스예요.
자기가 공부를 많이 할 수 있다고 착각하는 것입니다.
\vspace{5mm}

그런데 실패하는 인간들은 특징이 있죠. 계획만 참 거창하게 짜기 좋아한다는 것.
왜냐? 계획 짤 때는 즐겁거든요. 그리고 계획 못 이루면 다시 우울증. 그러다가 계획 짜면 또 즐거움.
공부 안 하고 쾌감을 누릴 수 있음, 이것도 어떤 면에서 마약입니다.
\vspace{5mm}

그것도 그렇거니와 쓸데없는 걱정 $-$ 즉 기우도 마찬가지입니다.
자기가 공부를 못 한다거나 못 생겼다거나 하는 생각은 백날 해보았자 아무런 이득 자체가 없어요.
그거 한다고 나아지는 것도 없습니다. 그럴 시간이 있으면 차라리 강의 듣거나 책을 읽거나 피부관리, 표정연습이라도 하는 게 낫습니다.
푸념 늘어놓는 게 정말 자기들이 걱정해서... 헛소리입니다. 그것도 역시 쾌감을 얻기 위한 거죠.
\textbf{자기가 못 났다라고 해서 다운되어있다가 남에게 격려, 칭찬을 듣고 다시 쾌감 누리고 또 우울해하고}.
자기가 정말 진지하게 못 났다고 생각하면 그런 말조차 안 해요. 학원가거나 바로 병원가는 거지.
이것도 그냥 어찌보면 중독적 행태입니다.
\vspace{5mm}

가끔 보면 남자 녀석이 무슨 기생오래비도 아니고 자기면상 올리면서 존잘 이러는 병신같은 케이스도 그렇죠.
그거 보면 "이 녀석 정신적으로 맛이 갔구나"라는 걸 느끼는 경우가 많죠. 뭔 남자가 못 났으면 얼굴 팔아먹고 있냐란 생각도 들지만,
\textbf{남들이 어떻게 반응해주나 그거에 쾌감 느낀다면 그건 다른 데에서는 스트레스를 엄청 받고 있으며 자아도 흔들리고 있단 이야기거든요}.
\vspace{5mm}

위 세가지 유형. 병적인 쾌감에 중독된 케이스입니다.
자기 공부에 바쁜 사람은 저럴 여념도 없어요.
그래서 전 저런 사람에게는 그냥 진실된 평가 내려주고 알아서 하라고 합니다.
한번 받아주면 그 다음부터 '쾌감' 얻으려고 병적인 행태를 계속 반복하거든요.
그건 근절하고 끊어내야한다는 게 제 생각입니다.
이거 받아주면 끝도 없어요 정말이지.
\vspace{5mm}

저 공부 계획 어때요... 라고 할 시간에 문제하나라도 더 풀고
나 못 생겼어... 할 시간이 있으면 어떻게 꾸미고 코디할 것인지 연구하고
자기 얼굴 사진 올릴 시간이 있으면 가서 이웃이라도 도우면 됩니다.
\vspace{5mm}

자기들이 정말 중요하다고 생각하는 게 4$\sim$50대 이후에도 중요한지 생각해보세요.
얼굴 타령 참 지겹게들 하는데 그거 어차피 30대 이후에 노화 안 되는 사람 없습니다.
나이먹을수록 빛이 발하는 건 얼마나 많이 공부했느냐, 현명한 판단을 하느냐 하는 겁니다.
자기가 못 생겼다 어쩌구... 매력이라는 것은 외모도 외모지만
그 전에 "스트레스를 덜어주고 마음을 편안히 해주는" 것에서 비롯되는 겁니다.
무엇보다 계획충. 자기가 실패한 기간동안 그냥 소박하게 공부했어도 합격했을 거라고 계산하면 답 나옵니다.
\vspace{5mm}

정신승리 얘기가 아니고 남자든 여자든 공부하면 매력이 늘어납니다.
눈빛이 정말 달라지거든요.
아무리 잘 생기고 예쁘면 뭐합니다. 눈빛이 흐리멍텅하면 기둥서방 술집여자지.
\vspace{5mm}








\section{공부시간 산정법}
\href{https://www.kockoc.com/Apoc/539674}{2015.12.11}

\vspace{5mm}

가장 현실적이고 탁월한 견해.
\textbf{"암기하는 시간}만 공부시간으로 정한다"
가령 12시간 책상에 앉아있다, 그 중 암기한 시간이 2시간이면 공부시간은 \textbf{2시간으로만 잡아야 한다.}
\vspace{5mm}

그런데 수능은 암기와 거리가 멀잖아요.
\vspace{5mm}

\begin{itemize}
    \item[] \textbf{ⓐ 문제풀이를 하고 해설을 읽고 정리하는 시간}
    \item[] \textbf{ⓑ 필기한 것을 '읽고 암기하는' 시간}
\end{itemize}
\vspace{5mm}

사실 이걸 제외한 나머지 시간은 공부시간에 빼야하지 않을까.
\vspace{5mm}

강의를 들어도 기억이 되는 경우가 있긴 한데 이건 좀 애매하다.
A 강사 강의는 한번 듣기만 해도 기억될 수도 있고, B 강사 강의는 안 그럴 수 있는데 이걸 정형화시킬 수 있을까.
\vspace{5mm}

그리고 국어, 수학, 영어에서 킬러문제를 해결해나갈 때의 "논리적 접근"이라는 건 단순 암기가 아니라
체험$-$체화를 거쳐야하는데 이건 어떻게 잡을 것인가(사실 이걸 강사들이 해결해줘야하는데 이런 강의는 정말 찾기 어렵다)
\vspace{5mm}

아무튼 공부시간이 많다고 해도 소용없는 게
\vspace{5mm}

\textbf{공부시간이 12시간이라고 해도 '암기 등에 쓰는 시간'이 0시간이면 그건 공부한 게 아니기 때문이다.}
열심히 장사를 했다. 하루에 1000만원 어치를 팔았다. 그런데 비용이 1000만원이다, 이런 무슨 소용이 있을까.
인강을 열심히 들었다. 그래서 정리를 안 하고 거의 다 망각해버린다.... 이게 공부인가?
\vspace{5mm}

남는 건 결국 '기억' 밖에 없다.
단지 그 기억이 시각적 기억이냐 청각적 기억이냐,
그리고 심상 기억이냐 경험 기억이냐 추상 기억이냐 그 차이일 뿐이다.
역으로 말해서 인간이 기억을 못 한다면 학습이 필요가 있을까.
\vspace{5mm}

다들 공부했다고 하소연하지만 이렇게 구체적으로 학습'회계' 관점에서 들어가보면 인과관계는 매우 잔혹해진다.
강사들은 '이해'만 하면 된다고 이야기하지만 실제로 자기들이 그 지식을 암기하고 있단 사실을 망각해버린다.
\vspace{5mm}

그렇다면 암기는?
\textbf{반복}이지.
\vspace{5mm}

그렇다면 공부의 실패는?
\vspace{5mm}
\begin{itemize}
    \item[] ⓐ 암기할 대상을 잘못 고르거나 누락
    \item[] ⓑ 암기를 제대로 하지 않았다.
\end{itemize}
\vspace{5mm}

이걸로 정리되지 않나?
\vspace{5mm}

입시기간이 길어지고 똑똑한 게 많은 사람들이 실패하는 경우.
\begin{itemize}
    \item[] A 관점 : 우와 수능 어렵나봐, 저 사람들 저렇게 실력좋은데
    \item[] B 관점 : 암기시간 재보면 얼마나 나올까.
\end{itemize}
\vspace{5mm}

유감스럽지만 B 관점이 옳을 가능성이 높다.
\textbf{공부할 대상을 잘 선정하고 많이 반복하고 암기한 것 가지고}
\textbf{자기가 남들보다 머리가 좋고 우월하다라고 과시하는 허세들이 많은 것도 현실이다.}
\vspace{5mm}

1번 읽고 기억하면 머리좋은 것이 아닌가요?
10번 읽고 기억하면 그건 머리 나쁜 게 아닌가요?
\vspace{5mm}

그럴 수도 있다. 그런데 10번 읽는 게 나쁘단 말인가.
9번 더 읽는데 100년도 걸리는 것도 아니고.
만약 10번 읽어야 기억하는 사람이라면 그냥 쿨하게 10번 읽으면 된다.
1번 읽어서 기억하는 사람이라면 그만큼 예민하고 날카로우며 학습하지 말아야할 것도 학습하는 문제가 생길 테니까.
그런데 문제는 10번 읽어야하는 사람들은 2번 읽고 포기한다는 것이다.
\vspace{5mm}

학습 비결이라는 게 별 게 없다. 사실 그 노하우라는 건 대형서점에 전시된 책들에 나와있다.
물론 '비결'이 아닌 가짜 비결들도 많다. 그리고 수험시장은 그런 가짜비결조차도 비싼 값에 파는 경우도 많다.
실제로 문제가 되는 건 비결을 몰라서 아니라, 비결에 집착하다가 \textbf{가짜 비결에 낚여서 허송세월한 케이스가 레알 많다는 것이다}.
그 친구들이 "암기시간만 공부시간"이라는 극단적이지만 매우 간결한 것을 받아들일까, 그럴 리야 없지 않나.
\vspace{5mm}

+ 사례
\vspace{5mm}

내가 괜찮다고 보는 모 콕콕러는 특정 과목 인강을 들으면서 이해가 안 간다고 할 것이다.
그런데 저건 매우 정상적. 이과 처음 시작해서 이제야 인강보면서 바로 이해간다면 그게 거짓말이나 사기이지 정상이겠는가.
그 인강도 반복해서 3$\sim$5번 듣고 책도 10번 읽어야 한다. 그래야 이해되는 게 정상이다.
내가 흡족한 건 "모른다"고 분명히 고백한 것이다. 허세들의 문제는 모르는 걸 아는 척 하다가 발리는 것이다.
제대로 안다는 건, \textbf{자기가 무엇을 모르는가 그걸 분명히 아는 것}이다.
자기가 모르는 것, 못 하는 것을 제대로 알아야 올라간다.
\vspace{5mm}

금수저들은 자기가 머리가 좋다고 생각하지만 그건 개소리.
자기들이 처한 환경에서 돈걱정없이 공부하는 것도 그렇지만, \textbf{환경 자체에서 그런 학습지식이 수도없이 반복주입된 걸 본인들이 모른다.}
환경은 에스컬레이터나 달리는 기차와 같다. 가만히 있어도 움직이긴 움직인다.
그러나 어느 순간에는 스스로 걸어가야 할 순간이 온다.
모 의원 아들 청탁 사건 같은 게 벌어지는 게 그 때문이다.
사교육 빨이 먹힐 때야 환경이 좋은 걸 모르고 지 머리가 좋은 걸로 안다.
\vspace{5mm}

그러나 스스로 헤쳐나아가야 할 때는? 그 때는 정말 흙수저 미만잡이다.
\vspace{5mm}

+
\vspace{5mm}

사실 또 한편으로 흥미로운 것은 요즘 x스쿨 사태에서 보다시피
자기들이 금수저라고 대놓고 말하는 금수저는 없단 것이다. 다들 흙수저인 척 하지.
그리고 이건 공공연한 사실이다. 금수저들도 \textbf{자기들처럼 공부하면 안 된다는 것을} 안다.
x스쿨이 정말 제대로 공부시켰어도 x시 존치 반대했을까. 자기들이 공부를 안 하고 실력이 없으니까 무서워하는 거지.
\vspace{5mm}

간혹 보면 건물주 미만잡 금수저 미만잡 거리는 애들. 그럴 시간에 공부나 하지 뭐하나 모르겠다.
\vspace{5mm}






\section{개인적 검증.}
\href{https://www.kockoc.com/Apoc/540628}{2015.12.11}

\vspace{5mm}

A. 인강 30시간짜리 들어봄
\vspace{5mm}

들을 때는 개꿀.
복습 안 했음(...)
3일 지나니까 다 까먹음.
\vspace{5mm}

B. 인강은 살짝 듣고 기본문제풀이해본 것
\vspace{5mm}

반복횟수 보니 대략 5번은 넘어감
어렴풋이 기억남
\vspace{5mm}

C. 지금도 백지 복기 가능한 과목(뭔지 물어볼 필요는 없을 듯)
\vspace{5mm}

가끔 뻘짓하긴 하는데 이건 백지 주면 처음부터 끝까지 설명가능.
연구용으로 인강을 들은 적은 있지만 그건 별로 도움 안 되었음. 이미 그 이전에 독학으로 완성
다만 상세한 이론 연구를 위해 온갖 책을 찾아 읽고 스스로 연구해보았음.
\vspace{5mm}

황금의 3개월에서 1/9가 지나감.
이 시점에서 경고하고 싶은 건, 시간 많이 남았다고 여러 과목 동시에 진행하지 말고
2$\sim$3과목으로 좁히고 목표량을 100이 아니라 10+10+... +10으로 쪼개서 그 10 하나를 끝낼 때마다 스스로에게 상 주도록 쪼개고.
처음부터 큰 것을 하려고 하지 말고 작은 것을 수도없이 반복하라는 것.
\vspace{5mm}

10page 분량이 있다고 칩시다. 그럼 이게 정말 10page일까요?
실제로는 100page여야합니다. 10회독은 해야하니까요. 그런데 실제로도 10page에서 100page 분량 내용이 나옵니다.
우리가 배우는 교과서, 참고서 내용이란 것은 '요약, 압축'된 것이죠. 공부하다보면 요약, 압축된 것이 풀리기 시작합니다.
\vspace{5mm}

2월까지 공부할 때 만약 1회독만 하는 거라면 그 계획은 폐기하고 다시 시작하시길.
예를 들어서 탐구과목을 2개 해서 2회독하는 것과 탐구 과목 1개를 해서 5회독한다면 후자가 낫습니다.
분량을 줄이고 회독수를 늘리는 것이 공부의 비결입니다.
왜 실패하는지 아십니까? 시간 많다고 전과목을 다 건드려보면서 인강까지 다 듣고 학익진(...)으로 가기 때문입니다.
이거 양도 많지 성과도 단기간에 안 나오지, 일주일 지나면 다 때려치우고 싶어집니다.
타겟을 철저히 좁히세요. 황금의 3개월동안 한과목만 봐도 좋습니다. 범위를 좁혀서 회독, 반복을 높이시길 바랍니다.
\vspace{5mm}

이것만 하면 됩니다. 무슨 아무개 강의 듣는다거나 특별한 교재 봐야한다거나 그럴 필요 단 하나도 없습니다.
공부가 안 되는 건 회독수가 적고 연습이 부족해서입니다. 그 다음으로 본인의 사고 프로세스가 문제있거나 잘못 알고 있는 것도 있지만요.
하지만 대부분은 회독수가 적어서 안 올라가는 것입니다.
이런저런 방황하다가 모 선생 강의 듣고 깨우쳤다하는 케이스.
물론 모 선생이 잘 가르쳐서 그럴 수 있지만, 그 모선생 강의를 듣는 시점에 '회독수와 연습'이 임계점에 도달했기 때문인 경우가 많습니다.
\vspace{5mm}

그럼 몇번을 반복해야하나?
하루$\sim$사흘에 1번이라고 하면 10번은 최소한 봐야한다고 봅니다.
이제는 정말로 제대로 알지 못 하면 시험에서 날라가기 때문이죠.
\vspace{5mm}







\section{첨단장비의 노동환원}
\href{https://www.kockoc.com/Apoc/540961}{2015.12.12}

\vspace{5mm}

공부 노력은 안 중요해, 그냥 xx 강사 강의만 들으면 된다.
그냥 가볍게 반문하겠음
\vspace{5mm}

제 친구 중에서는 당시 특목고$-$서울대 라인 밟은 사람들 많습니다(아시는 사람은 챗에서 들었을 테니 말 안 하겠음)
그런데 그 친구들 중에서 누구 하나라도 공부방법을 깨달았다, $\sim$ 안 해도 된다라고 하는 단 한명도 없었고
저 역시 '올바른 공부방법이 무엇인가'라고 하면 목숨 걸고 말하라고 하면 말을 못 하겠음.
\vspace{5mm}

그런데 \textbf{"지금은 놀고 그냥 나중에 하면 되지 않겠느냐"}라고 깨달았다고 하는데 그게 참.
거기 댓글 단 사람 중에서 몰x군은 작년 이맘 때에 제가 늦으니까 빨리 했는데 안 했고
래x은 플래너 인증 요청하면 알겠지만 죽어라 달린 분이 그런 댓글 다는 건 좀 그렇지 않나.
\vspace{5mm}

자기 약점이나 방향이라는 건 '죽어라 공부하고서도 한계에 부닥쳤을 때'에 비로소 느끼는 것이고
그렇다면 그 전제라는 건 죽어라 했을 때를 전제하는 건데
웃긴 건 공부에 방향이 중요하다라고 하는 사람들을 잘 보면 정작 그렇게 \textbf{죽어라 공부하는 경우}가 아닙니다.
\vspace{5mm}

자기가 약점이나 방향이라고 느끼는 것이 정말 \textbf{'진짜 약점이나 방향'}이라고 생각하는 건지?
가장 위험한 생각이 이거죠.
\textbf{1000시간 공부할 바에는 걍 약점과 방향 잡고 10시간 공부하는 게 더 똑똑해.}
아마 다들 이런 생각하고 있을 것임. 래서 +1수를 늘리는 겁니다.
왜냐? 약점과 방향을 잡는다는 건 훌륭함. 그런데, \textbf{자기가 아는 그 약점이 진짜 약점이고 그게 전부라는 보장이 어디있죠}?
빨리 양치기하라는 이유가 별 게 아닙니다. 그래야만 그나마 \textbf{진짜 약점과 방향을 알 수 있어서}입니다.
그건 본인이 알지 남들은 아무도 몰라요.
\vspace{5mm}

오히려 불편한 진실은 "아, 빨리 시작하면 뭐해. 아, 양치기하면 뭐해. 걍 xx 인강 듣고 방향 잘 잡으면 되지"
이 메시지조차도 \textbf{공부량을 줄이거나 공부를 안 하려는 아주 교묘한 핑계}라는 거죠.
정작 실적이 좋은 사람들은 자기 확신은 안 합니다. 공부하면 할 수록 자기가 더 모자르다라는 걸 깨닫거든요.
졸라 달릴 수 밖에 없지 O, X 확 떨어지는 것 아니라는 걸 알기 때문입니다.
\vspace{5mm}

혹은 이런 이야기를 하겠죠. 첨단장비 걍 훔쳐쓰는 게 더 효과적이지 않나.
\vspace{5mm}

그런데 훔쳐쓰는 건 한계가 있죠. 고장나면 어떡할 거임? 그리고 자가 생산은 언제할 것임?
님들이 가볍게 쓰는 맛폰조차도 하드웨어$-$모듈$-$부품, 소프트웨어$-$모듈$-$코드.
이거 하나하나 분해해보면 \textbf{수십년, 아니 수백년전부터 진행되어온 지식, 육체 노동자의 노동이 집약된 결과입니다}.
그냥 하늘에서 뚝 떨어진 게 아니란 것이죠. 어디든 사람의 '손길', 즉 노동이 압축되어 있습니다.
우리야 그걸 천박한 자본주의적 거래로 너무 가볍게 소비하고 있습니다(그래서 살기좋은 세상인 것이죠)
\vspace{5mm}

그런데 공부가 돈을 주고 상품만 딱 구입하면 능력치가 올라갑니까.
그렇다면 누가 공부를 합니까, 걍 돈주고 능력치를 사지.
하지만 현실적으로 그런 경우는 없다는 데 유의하시길 바랍니다.
비싼 돈 주고 컨설팅을 한다... 아니 그게 효과가 있으면 다 그걸 구입하겠죠.
하지만 실제로 수율을 보자면 그것들도 형편없는 걸로 압니다.
\vspace{5mm}






\section{잡담}
\href{https://www.kockoc.com/Apoc/543390}{2015.12.13}

\vspace{5mm}


\begin{itemize}
    \item \textbf{정석}
    \vspace{5mm}

    정석부터 잡으려는 분들이 계시는데
    정석이야말로 1등급 넘어서면서 수리적 사고가 숙달되어서
    눈에 보이는 것들을 집합$-$명제로 함수관계로 환원시키거나
    모든 운동을 벡터와 변환으로 보거나
    심지어 소리까지도 삼각함수로 근사시켜서 주기를 논하거나.
    \vspace{5mm}

    ... 그 정도가 아니면 처음 보는 건 그냥 자살일 수 있으니 유의하시기들 바랍니다.
    \vspace{5mm}

    수능시험 수학이 쉬워졌다 어쩐다하지만 그건 피상적인 이야기입니다.
    적어도 과거보다 나아졌다고 생각하는 것 $-$ 특히 최근 경향을 보면 소박한 "수리적 사고"를 검증하는 건 성공했다고 보이는데요
    이런 시험이면 정석을 주교재로 삼는 건 실익이 적습니다.
    과거 수학시험이야 DB 승부다보니까 우리보다 수학 선진국인 일본의 온갖 패턴이 들어가있던 정석을 그냥 공부해도 동
    \vspace{5mm}

    최근 수학은 심지어 내신조차도 선생들이 업그레이드되었기 때문에 '암기만 해서' 푸는 것은 안 냅니다.
    중요한 건 \textbf{문제를 독해하는 능력}입죠.
    거기에 정석이 직접 도움을 주진 못 한다라고 얘기하겠습니다. 정석은 독해능력이 있는 사람에게만 도움을 주는 책입니다.
    \vspace{5mm}

    \item \textbf{돈을 주면 특효약을 얻을 수 있다?}
    \vspace{5mm}

    공부에서 참 흔한 이야기입니다.
    콕콕에서도 그렇게 일종의 야매교재를 칭찬하거나(그래서 올 수능은 적중하셨나)
    특정 강의나 학원만 지독하게 신성시하는 경우가 있었사온데.
    \vspace{5mm}

    그러니까 그 특효약이 뭔지나 가르쳐주었으면 합니다.
    자기만 따라오면 공중부양이 가능하다 기경팔맥이 뚫려서 200살까지 산다 그렇게 드립치면 뭐해요.
    실제로 그게 뭔가 \textbf{보여줘}야지.
    \vspace{5mm}

    다 필요없고 올 수능만 봅시다. 그래서 \textbf{적중한 교재가 있었습니까}.
    그렇게 시험전까지 실모 안 보면 망한다 누구 실모 좋다고 했는데 실제 적중도는?
    수학에서 고1수학 발상 나온다라고 한 사람 단 한명이라도 있던가.
    영어는 쉽게 나오니까 걍 상관없다고 하더니만 어렵게 나오니까 평가원이 그렇게 출제하면 안 된다 드립치질 않나.
    이거 다들 보면 걍 제정신이 아닌 것 같음.
    \vspace{5mm}

    \item \textbf{멍청한 짓을 반복하면 바보다.}
    \vspace{5mm}

    상담하면서 늘 느끼지만 실패하는 가장 큰 원인은 "멍청한 짓을 반복해서"입니다.
    그런데 본인은 그걸 잘 못 느낍니다.
    저야 상담해줄 때야 온화하게(?) \textbf{"$\sim$ 하는 게 낫지 않느냐"}라는 권유형을 취하지요.
    그러나 그게 정말 내용상 권유형은 아닙니다. 제가 상담해주는 방식은 간단합니다.
    \textbf{"피상담자가 어떻게 하면 제대로 말아먹는가"하는 걸 먼저 가정하고 시나리오를 써보고 이야기드리는 것이죠.}
    피상담자가 A라는 코스로 가면 망할 게 분명하면 $\sim$A에 해당하는 B, C, D를 권유하는 것입니다
    과거에는 대놓고 A로 가면 망한다... 했는데 정작 A로 가면 망한 사람들이 "너 나 비난했지"라고 해서 권유형 취한 거지 내용 바뀐 건 아닙니다.
    \vspace{5mm}

\end{itemize}

상담이야 책임감있게 해줄지는 몰라도 상대방이 "에이 안 그래도 되잖아요"라고 해서 잘못 가도 그건 \textbf{제가 알 바도 아니죠.}
지금 황금의 3개월 1/6 지났죠. 자, 지금이야 마음 편하죠? 2월 말 되면 또 사람들 마음 바뀝니다.
상담이야 해주겠고 정말 상대가 그렇게 한 경우에는 나름 응답을 하지만 또 자기 멋대로 하는 경우는 제가 생각할 경우는 아니죠.
더군다나 대놓고 말 안 하지만 익명성을 이리저리 바꾸거나 시험해보거나 하는 한심한 케이스도 많던데 그건 당사자들이 알겠죠?
\vspace{5mm}

덤으로 상원이 아닌 경우 상담은 늦게 대답해줍니다.
신중히 답할 것도 있지만 '먹튀'만 하는 케이스는 제가 좋게 보지 않아서요.
\vspace{5mm}






\section{다수의 선택}
\href{https://www.kockoc.com/Apoc/545560}{2015.12.15}

\vspace{5mm}

출제경향은 다수의 생각을 배신하는 경향이 있다.
작년 수학과 국어 출제 수준도 그랬고 올해 영어 역시 마찬가지였다.
\vspace{5mm}

수험사이트에 올라오는 코메디성 글이 많지만 가장 웃긴 것은
일개 학생이나 대학생이 \textbf{"평가원 너희가 그렇게 출제하면 안 된다'라고} 훈장질을 하고 있다는 것이다.
변별력을 갖추려는 평가원으로서는 다수 학생들을 통수먹이는 건 당연하지도 않나
\vspace{5mm}

저렇게 말하는 사람들은 실제 입시 결과도 기대 이하지만 입시를 떠나서 앞으로 사회 생존도 염려된다.
그건 수능시험의 본질조차 생각하지 않고 있기 때문이다.
대학수학능력시험은 다수를 위한 시험이라면 다수가 하는대로 가도 된다.
예컨대 운전면허시험이라거나 토익 700점 넘기는 수준이면 다수론대로 따라가도 된다.
그 정도의 결과는 "다수"를 위해 열려있는 것이기 때문이다.
\vspace{5mm}

하지만 수학능력시험은 \textbf{소수를 위한 시험}이다.
개나소나 만점을 외치니까 우와 나도 만점을 받을 수 있어 하지만, 실제로 그런 \textbf{만점을 받는 건 소수다.}
그것도 그냥 \textbf{소수}가 아니다. \textbf{"\textbf{다수}"의 생각과 행동을 읽고 있는 \textbf{소수}}여야한다.
\vspace{5mm}

그렇기 때문에 최소한 다음 조건들을 충족해야 한다.
ⓐ 다수가 보는 시중 교재나 강의는 빠삭해야 한다.
ⓑ 출제자 수준을 넘어선 소수여야 한다.
\vspace{5mm}

둘 다 충족해야 실패하지 않는 것이다. 하나라도 X 가 되어버리면 거기서 날라가버린다.
본인은 소수에 속한다고 하지만 시험만 치면 어이없이 실수하거나 날라가는 경우는 ⓐ가 안 된 것이다.
본인은 시중교재를 충실히 풀었지만 어려운 문제에서 막히는 건 ⓑ가 되지 않은 것이다.
\vspace{5mm}

그럼 다수와 소수를 둘 다 선택하라... 그래서 힘든 것이다.
분명 \textbf{다수의 행렬에 속해서 영화를 관람하고 있어야} 한다,
하지만 결정적일 때 \textbf{나 혼자 비상구로 도망갈 준비를 해놓아}야 한다.
다수의 행렬에 묶여있으면 극장 화재 시탈출하지 못 하고 압사당할지도 모른다.
그렇다고 처음부터 비상구에 있으면 영화를 보지 못 하고 허송세월해야한다.








\section{과거의 문제와 미래의 문제}
\href{https://www.kockoc.com/Apoc/551104}{2015.12.18}

\vspace{5mm}

선행을 미리 했던 학생들이나 특목/자사고 간 친구들이 공부를 '잘' 하는 건 맞다.
그러나 신기한 건 그들 모두가 수능을 잘 치르진 않는단 것이다. 수시로 잘 갈지는 몰라도 정시는 생각보다 수율이 낮은 경향이있다.
이것도 왜 그런가 생각해보면 간단하다.
우리가 하는 공부 $-$ 특히 사교육+기출 중심의 공부는 "과거의 문제"를 푸는 공부이다.
물론 과거의 문제를 학습하고 익히는 건 기본이다. 하지만 수능 문제는 "미래의 문제"다.
과거의 문제에만 집착하면 \textbf{미래의 문제를 풀기 어렵다}.
\vspace{5mm}

그래서 입시판은 늘 이변이 발생한다.
그냥 기대 안 했던 A가 고득점을 받는다. 그 때까지 A를 천대하던 사람들이 "너는 잘 할 줄 알았어"라고 입장을 바꾼다.
유망주이던 B가 터무니없는 점수를 받는다. 사람들은 싹 침묵해버린다.
그렇다고 A가 잘하고 B가 못 한다라고 단언할 수 없다. 단지 출제'경향'에 안 맞았을 뿐이다.
이런 경향을 무시하는 사람들은 모든 것을 "운"이라고만 치부하기 좋다.
물론 운이라는 건 없지 않다. 그러나 운은 '인과관계'를 정확히 모를 때에 그 현상을 설명하기 위한 도피적 개념에 불과하다.
\vspace{5mm}

잘 할 줄 알았는데 못 나온 케이스들의 특징은 다음과 같다.
\vspace{5mm}

\begin{itemize}
    \item[$-$] 모평 등급에 연연한다 (모평은 어디까지나 과거 문제의 짜깁기일 뿐이다, 물론 6,9평은 다르지만)
    \item[$-$] 기출과 인강을 돌리려고만 하지 생각을 하지 않는다(막연히 푸는 것과 생각하는 건 다르다)
    \item[$-$] 자신의 단점을 개선하려하지  않는다(수험은 자기를 바꾸는 과정이지 수험상품 소비가 아니다)
    \item[$-$] 환경을 바꾸지 않는다(공부는 의지대로 하는 게 아니다. 환경이 도와주는 것이다. 단지 우리가 의지로 환경은 바꿀 수 있다)
    \item[$-$] 절박하지 않다(사실 이게 가장 크다. 내가 느낀 n수 실패의 가장 큰 원인은 당사자들이 지나치게 편하다는 것이다)
\end{itemize}
\vspace{5mm}

분명한 건 2017 수능은 2016 수능이 아닌데도 사람들은 올해와 마찬가지로 \textbf{그렇게 바라볼 것}이란 사실이다.
\vspace{5mm}

대충 이런 체크리스트로 나눠진다.
\vspace{5mm}
\begin{itemize}
    \item[] 1$-$1 수능을 수능으로 보는가 
    \item[] 1$-$2 수능을 학력고사처럼 보는가
    \item[] 2$-$1 어떤 문제도 원점으로 돌아가 풀 수 있도록 하는가  
    \item[] 2$-$2 과거 기출문제를 암기하는데 그치는가
    \item[] 3$-$1 기본적인 것을 설명할 수 있는가 
    \item[] 3$-$2 잡기에 맹목적인가.
\end{itemize}
\vspace{5mm}

공부를 열심히 해왔다 안 해왔다를 떠나서 저것들로 한번 점검해보면 대답하기 싫은 방향으로 답이 나온다.
분명 공부를 열심히 하면 좋은 성적을 거두는 건 맞는데.... 문제는 n수생 이상부터는 \textbf{실패하는 방법도 학습해버린다라는 게} 문제.
실력 1000을 높이는데 동시에 허력을 900을 높이면 1000$-$900=100이 되는 셈이다.
그런데 사람들은 \textbf{자기의 허력을 공개하는 건 꺼린다}.
\vspace{5mm}

여담이지만 벌써 한달이 지났다. 2016 수능까지면 1/12가 지난 셈이다.
다들 시험 전까지는 아 일주일만 있으면 공부했을텐데... 라고 간절히 호소하지만 지금은 그딴 건 없다.
인간이 다 그렇게 간사한 동물인 것이다.
\vspace{5mm}

전 분명 일찍 시작하라고 했으니 나중에 원망하지 마셈.
\vspace{5mm}







\section{[콕콕운영제언] 콕콕사이트의 보수적 운영}
\href{https://www.kockoc.com/Apoc/552129}{2015.12.18}

\vspace{5mm}

일단 1년간 주욱 관찰해보면서 확인한 것은
'비공개'로 "상호신뢰가능한 회원들끼리 소통하는" 것이
공개된 공간에서 서로 믿지 못 하는 것보다 나았다는 것을 확인했습니다.
\vspace{5mm}

\begin{itemize}
    \item[] \textbf{ⓐ 심리적 위안} : 사실 이게 가장 크죠. 불안하거나 스트레스 받을 때 호소하는 것
    \item[] \textbf{ⓑ 질서있는 갈등} : 사람들이 모인 곳이야 으레 다툼과 갈등도 있지만 상호 신뢰에다가 중재절차가 반영된 터라
    \vspace{5mm}
    
    \item[] \textbf{ⓒ 안전한 소통}
    \vspace{5mm}
\end{itemize}

아까도 챗방에 보니까 '아무 말도' 없는 회원이 호출을 해보니 바로 나가버리던데.
안 그럴 거라고 믿고 싶지만 부적절한 의도로 대화 내용을 캡처해서 악용해버리는 경우가 불가능하지 않습니다.
\textbf{겉으로는 정정당당한 척 하면서 뒤로는 이중 아이디나 부계정을 써서 타인 행세하는 비겁한 인간들이} 늘 있기 때문입니다.
요즘은 이게 진화해서 일부러 명예훼손이나 모욕죄로 엿먹이기 위해 발언을 유도하는 질떨어지는 자들도 있습니다.
\vspace{5mm}

콕콕이 너무 친목화되어가느냐 하는 경우도 있습니다만 일단 알려둘 것은
여기는 영리 사이트가 아니란 것이죠. 핵심멤버가 자기 책을 팔아먹는 경우도 있겠지만 그건 사이트 목적과는 다소 거리가 있습니다.
이 사이트의 목적은 n수의 안식처 \textbf{수험생들끼리 소통하고 정보교환을 해서 자기들의 벽을 넘어서는 극복의 축적}이지요.
저도 '상업주의'와 무관하게 그런 게 가능할까 하는 방법의 검증과 시스템의 확립을 위해서 여기 있는 거지
천박하게 자기 광고해서 컨설팅 비용이나 야매교재로 재벌이 되는 게 아닙니다. 오히려 그런 것을 '배격'하는 걸 목적으로 하죠.
\vspace{5mm}

그러나 돈이 걸린 곳은 어디나 더러운 인간들이 있습니다.
그런 사람들에게까지 무단으로 개방해버리는 경우 문제가 터집니다.
올해 있었던 몇가지 사건들만 보더라도 사실 그 진실을 알고 있는 사람으로 말하건대 이 사회에서 승리하는 건 '정의로운' 사람이 아닙니다.
윤리 따위는 개나 주더라도 돈으로써 법을 이용할 줄 아는 사기꾼이 일단 우세합니다. 현실이 그런 건 어쩔 수가 없죠.
그렇기 때문에 선량한 사람들을 보호하기 위해서라도 '성벽'을 쌓는 건 불가피합니다.
\vspace{5mm}

일단의 비공개 게시판이 있는 건 양해하셨으면 합니다. 꾸준히 활동하고 글을 쓰며 공부하려는 분들은 환영하고 받아주니까요.
그리고 실제로 그런 비공개 게시판이 '스트레스'와 '불안감'에 시달리는 수험생들에게 도움이 된다는 게 개인적인 판단입니다.
공개 게시판의 경우는 수험생들이 \textbf{자기를 방어하기 위해 일부러 성적을 과대발표하거나 무리한 공부를 하는 식으로 파멸해가는 경향이 있습니다.}
즉, 우리가 알고있는 소위 수험생들의 허세라는 건 자기 방어적 성격이 강하단 것입니다.
만약 수험생이 남 눈치를 보지 않고 비슷한 처지에 있는 사람들끼리 소통하고 대화할 수 있다면 저런 허세를 부리지 않아도 되었겠죠.
콕콕에서 중시하는 건 누가 좋은 대학에 갔느냐가 아닙니다.
\textbf{수험에 실패한 사람도 다시 일어설 수 있는 시스템의 완성이죠.}
우리가 모든 게임에서 이길 수는 없습니다. 승승장구하는 사람도 한번은 지게 되어있습니다.
중요한 건 졌을 때 그대로 쓰러져있지 않고 어떻게 일어나느햐는 법을 배우고 실천하는 것입니다.
\vspace{5mm}

이런 점 때문에 비공개 게시판이 늘어나는 것이고, 적어도 외부인이 보기에 친목성이 강화되는 것으로 비치는 건 양해해주셨으면 합니다.
챗방조차도 사실 '안심하고 대화할 수 있도록' 하려면 자격요건을 제한해야하는구나를 확실히 느낀 새벽이었습니다.
미리 말씀드리지만 관리자 허님을 포함해 몇몇 사람들은 대화 참여자들의 ip를 확인할 수 있습니다.
오늘 새벽에도 제가 호출했을 때 말없이 나간 분도 예외가 아닙니다.
이런 불편한 일을 막기 위해서는 대화방도 재편해서 lv 1 이상이 아니면 $-$ 즉 게시글을 쓰는 등 소정의 절차를 취하지 않으면
대화방의 출입을 제한하는 조치가 필요합니다.
\vspace{5mm}

분명 이 사이트는 작년보다 나아졌습니다.
슬프지만 이건 자본으로부텨 견제를 당할 위험이 크다는 것이죠.
무슨 무협지를 쓰느냐.... 하겠지만 슬프게도 그렇지 않습니다. 제가 이 사이트 와서 경험한 것들도 내막 알아보니 장난아니었어요.
교재에 대한 솔직가감한 평가조차도 업자들에게는 치워버릴 방해물로 밖에 보이지 않거든요.
실제로 수험시장은 '여론'이 중요합니다. A 교재나 B 강의가 좋다고 하면 다들 우르르 소비하죠. 액수도 장난이 아닙니다.
콕콕러들이 선량한 의도로 수험시장을 평가한들 그건 타인들 눈에는 "자기 장사를 방해하는" 것으로 밖에 안 비칠 겁니다.
\vspace{5mm}

돈에 눈이 먼 사람들은 이 사이트는 좀 피해주셨으면 좋겠습니다.
그리고 이거 나름대로의 사이트 문화로 정착시키겠지만
피해야 할 강의나 교재를 직접 언급하는 경우는 없을 겁니다. 무조건 그 경우는 \textbf{"언급제외",} 즉 볼드모트급으로 분류할 것이니까요.
바보가 아닌 이상 자기 상품이 언급제외되었다고 나서는 업자 분은 없을 것이라고 믿겠습니다.
추천해야 할 강의나 교재도 거의 공짜거나 가성비가 좋은 것으로 한정하는 쪽으로 조정할 것입니다.
\vspace{5mm}

+
\vspace{5mm}

챗방 이용자 분은 처음보는 닉이거나 말없는 경우는 말을 걸어보시길 바라고.
왕관이용자 분들이나 lv이 좀 높으신 분은 수상한 닉이 있으면 제보해주시길 바랍니다.
씁쓸하지만 이런 장치가 아니면 안전한 챗도 불가능해진 상황이 오는 것 같군요.
대화방에서 제재하는 방향은 "벙어리" 아니면 "ip밴"이온데
벙어리는 사실상 대화 내용을 열람할 수 있어 ip 밴 밖에 없습니다. 퇴장기능이 따로 있는지는 모르겠습니다만.
결국 번거로운 걸 막으려면 가장 많이 쓰는 콕방을 일정 레벨 이상으로 자격제한하고 그런 식으로 가는 게 불가피해보입니다.
\vspace{5mm}

+
\vspace{5mm}

오늘 새벽에 대화방에서 제가 말을 걸었던 "x토x" 회원은 바로 탈퇴했군요.
그냥 말을 걸고 자기 소개를 하라고 했을 뿐인데 바로 탈퇴라.
물론 그 전에 ip로 대략 어느 지역인지는 확인해보았긴 했습니다만.
\vspace{5mm}






\section{여러가지}
\href{https://www.kockoc.com/Apoc/556008}{2015.12.21}

\vspace{5mm}

\begin{enumerate}
        

    \item 실모, 그리고 교재
    \vspace{5mm}

    그 저자들조차도 시중에 있는 문제집을 넘어서 과거 본고사 것이나 일본 것까지 다 연구했을지는 의문입니다(저야 보유만 했습니다만)
    쎈이나 EBS 문제가 안 좋다라고 하는 분들에게는 블라인드 테스트 던지고 싶음.
    \vspace{5mm}

    일단 실모는 "수리적 사고가 잘 잡힌 친구"들에게는 좋을 수도 있습니다. 하지만 그게 안 되어있으면 보지 말아야 합니다.
    이 사이트와 관련있는 일타삼피 $-$ 고득점 N제 $-$ 조차도 교과서나 시중 기본서 양치기가 안 된 친구들은 \textbf{보면 안 됩니다}.
    \vspace{5mm}

    일타삼피를 포함한 실모는 실제로 참신해보일지 몰라도 본질적으로는 "학원가" 냄새를 못 벗어나죠.
    저는 이런 걸 흔히 '사파'라고 부릅니다. 물론 사파 내공을 쌓아도 시험에 합격하기만 하면 나쁠 건 없지요.
    그런데 정파 $-$ 즉 교과서나 기본 틀이 잡혀있는 친구들이라면 실모의 독을 마셔도 소화시키기나 하지
    그게 안 된 친구들은 상담해보면 안 본 실모가 없는데 기본적인 개념을 물어봐도 답을 못 하는 경우가 있습니다.
    \vspace{5mm}

    문과 이야기가 있는데 사실 적중도는 안 좋죠. 30번 제외한 나머지는 그냥 쎈으로 넘치고, 30번 자체는 고1 수학 꽝이면 못 풉니다.
    상당히 많은 친구들이 격자점만 나올 거라고 생각하고 그것만 대비하다 털린 케이스도 있었을 것이고
    이과 수학은 더욱 그렇습니다. 30번 문제는 아예 사파적인 것은 대놓고 꺼져라고 외치지 않던가요?
    문이과 불문하고 고득점 킬러문제는 '패턴'이 먹히지 않고, '생각'을 여러번 해야하며, 매 과정마다 정확한 식과 그래프 전개 요구한다.... 였습니다.
    무릇 어떤 교재가 좋다고 하려면 근거는 분명히 대야한다고 생각합니다.
    물론 무조건 "좋다"라는 썰만 듣고 잘못된 수험 전략을 짜는 건 본인 책임입니다만 치러야 할 대가가 상당히 크네요.
    \vspace{5mm}

    지학사에서 나온 풍산자 약점공략 시리즈 주목.엄지손가락 듭니다.
    구성이 색달라서 저자진들을 보니 으음, 에이스들이고 교재 접근이 매우 훌륭하네요.
    그리고 신사고에서는 특작이 부활했습니다.
    그리고 마플은 잽싸게 팔리는 모양이더군요.
    여기다 EBS까지 추가되고 하면 사실 교재가 없어서 문제가 아니라 시간이 없어서 골치아플 것입니다.
    \vspace{5mm}

    \item 인강
    \vspace{5mm}

    인강의 문제는 지나치게 길다는 것입니다. 한 코스를 완료하려면 하루에 3시간이어도 기본 2주일까지 가는 경우가 많습니다.
    하나하나 필기를 다 해야하고 여러번 들어보아야하죠.
    그런데 문제는 이렇게 정작 하고 나면 그 지식들이 다시 '증발된다'는 것입니다.
    귀로 듣는 것보다는 읽는 것이 20배는 더 빠릅니다. 반복해서 읽고 암기하는 것이 학습의 정도죠.
    사실 이 때문에 인강을 들을 때나 아하 하는 사람이 점수는 잘 나오지 않는 것입니다. 인강듣다보니 복습하고 문풀할 시간과 체력이 날라가죠.
    \vspace{5mm}

    인강은
    \begin{itemize}
        \item[$-$] 과목의 감이 없어서 큰 흐름을 잡는다거나
        \item[$-$] 이해가 안 가는 대목만 설명을 듣고 이해한다거나
        \item[$-$] 문제푸는 큰 틀을 익힌다거나
    \end{itemize}
    이 정도에 국한해서 보는 게 낫다는 생각입니다
    굳이 알차게 활용하고 싶으면 mp3만 따서 이어폰 끼고 듣고다니면서 시간 절약하는 방법이 있을 것입니다만.
    EBS 수능개념 강의 정도만 듣고 문풀 하시다가 5,6 월 되어서 자기의 약점이나 취약 과목 및 단원에 관한 것만 골라서 듣는 게 낫다고 충고드립니다.
    \vspace{5mm}

    \item 왕따
    \vspace{5mm}

    여기서까지 이런 이야기가 들리는데 참 \textbf{미개하고 비겁한 짓}입니다.
    물론 왕따를 시켜야만 하는 불가피한 이유가 있다면 모르나, 그 이유는 당당히 공개할 수 있어야죠.
    해도 되는 왕따라는 건 그 대상이 주변인들에게 악행을 저질렀는데도 대화가 안 통하는 케이스 정도인데 10대에 이런 케이스가 있나요?
    \vspace{5mm}

    \item 환경
    \vspace{5mm}

    서울 강남 한복판에 산다면야 걸어가기만 해도 온갖 문화를 누릴 수가 있죠.
    그러나 본인이 독도나 마라도에 산다면?
    수험생들은 자기 환경이 얼마나 축복(저주)받았는가를 모릅니다.
    공부를 자기만 한 줄 알죠. 70$\%$는 부모님이 해주신 것일텐데 말이지요(특별한 예외를 제외하곤)
    부모가 어린 시절부터 투자한 녀석들이야 기출만 풀고 학원만 다니는데 왜 고득점이 안 나오냐 하겠죠.
    \vspace{5mm}

    공부는 사실 환경이 전부입니다. 환경에 적응한다는 것은 \textbf{사고방식도 달라진다}는 걸 의미하죠.
    지적환경 구축을 위해서는 \textbf{독서도 많이 해야하고 공부 잘 하거나 머리를 많이 쓰는 사람들 근처에 있어야 합니다}.
    공부 열심히 할거야라고만 부르짖지 말고 환경을 바꾸는 게 좋습니다.
    환경을 바꾸라는 건 방해받지 말고 공부할 수 있는 장소와 시간을 확보하는 것도 중요하지만
    꾸준히 지적자극을 받을 수 있는 '독서'와 '강의', 그리고 '경쟁'까지 포함하는 개념입니다.
    \vspace{5mm}

    \item 고레카와 긴조
    \vspace{5mm}

    헌책방에서 다시 겟해서 읽는 자서전입니다. 일본의 전설적인 투자자 $-$
    이 양반은 부모가 부자도 아니고 머리 좋은 사람도 아닙니다니다.
    20세기 초 일본의 젊은이들이 그랬듯이 초등학교만 졸업한 뒤 옷가게 사환으로 일하다가
    '책'을 읽고 중국$-$유럽으로 건너가 장사를 하기로 마음먹은 게 14살이더군요(...)
    정말 홀홀단신으로 건너가 일본군도 따라다니고 들개에게 죽을 뻔하고 굶어죽기 전까지도 갔다가
    부기(회계) 실력으로 기회 잡아서 일본군에 납품하다가 나중에는 중국인들의 동전을 녹여판 주괴를 수출해
    수억씩 벌어들였는데 쑨원의 혁명군에 투자했다가 지는 바람에 쫄딱 망해서 자살을 생각했던 게 19살 $-$ 미성년.
    (주괴 수출이 허가되지 않을까봐 세관장을 권총으로 협박한 장면도 인상적인데 생각해보니 고2가 그랬다는 게 흠좀무)
    그 이후로도 대박과 쪽박을 반복하다가 3\textbf{년간 도서관에서 온갖 경제 서적과 자료를 탐독하여 고수}가 된 뒤에
    31살에 경제연구소를 신설해 교수까지 제자로 삼고(초졸이면서) 사업하면서 고위층과 연줄 맺고... 그 이후는 알아서 책구해서 보시길.
    (이 사람이 우리나라에서 최초로 용광로 고로를 설치했을 것입니다. 그게 포스코 박물관에 있다던데)
    \vspace{5mm}

    자꾸만 유전거립니다만 그건 별 의미가 없습니다. 본인이 환경을 적극적으로 바꾸는 게 더 중요하다는 일례가 되겠습니다.
    나는 한 때 성적이 잘 나왔는데(중학교 때겠지만) 지금은 왜 그럴까 하기 전에 책 한권이라도 더 읽고
    최대한 보수적으로 실력테스트해보면서 조금이라도 문제가 있는 건 철저하게 바꾸고 개선하는 편이 나을 것입니다.
    \vspace{5mm}

    \item 공부를 한다 $-$
    \vspace{5mm}

    는 것으로는 안 되고 \textbf{미쳐야합니다}.
    수학에서 미적분을 공부하면 사회의 모든 현상을 미적분으로 생각해보아야합니다(실제로 그렇게 쓰이고 있지요)
    국어에서 이해가 안 가는 지문을 읽는다면 강의에만 의존하지 말고 네이버 검색이라도 해서 관련된 화제들이 어떤가 다 찾고 생각해보아야하고
    영어는 아예 외국인에게 내가 구사할 수 있는 대사 목록으로 암기해버려야겠죠.
    \vspace{5mm}

    머리좋은 사람도 \textbf{"미치도록 좋아하는 사람"}을 못 이깁니다.
    과탐의 화학과 생명과학은 "공부를 잘 하는 사람"이 아니라 정말 그 문제에 환장한 '마니아'들 아니면 꺼져라고 소리치고 있죠.
    관점을 바꿔 보자면 생1의 경우 열심히 한 사람에게는 억울하겠지만, 본인이 유전성애자(...)였다면 아니 이런 천국이라는 소리가 나왔을지도 모르죠.
    \vspace{5mm}

    소위 지능지수 천재 $-$ 에 대한 열광은 1990년대에나 유명했던 걸로 압니다만
    지금은 nerd의 시대죠.
    머리좋냐 안 좋냐보다도 본인이 얼마나 거기에 \textbf{미쳐있느냐}가 더 중요합니다.
    단지 고득점을 맞는다... 로는 분명 실패합니다. 중요한 건 내가 그 과목에 얼마나 미쳐있느냐는 겁니다.
    \vspace{5mm}

    \item 사교육의 미래
    \vspace{5mm}

    뭐긴요 인류의 기원인 \textbf{아프리카}로 돌아가는 거겠지.
    가 아니라 실제로 아프리카 방송 스타일이 '학습효율' 면에서도 훨씬 낫죠.
    지금 가장 앞서나간 게 EBS 인강이죠.
    사설 인강은 처음부터 끝까지 다 들어라, 강사님에게 세뇌당하여라, 교재에 애정을 품고 5000원짜리 50000원에 사라 하는 거면
    EBS  인강은 발췌해 들을 수도 있고 콜라보 강의도 있지만
    \vspace{5mm}


\end{enumerate}

이 어느 쪽도 피드백은 약하죠.
\vspace{5mm}

전국에 입시고수들은 늘어나니 머지 않아 그 사람들이 출판사와 모종의 협약을 맺거나(사실 맺을 필요가 있나, 부수 늘려주는데)
직접 아프리카로 문제풀이를 하면서 모르는 것 설명해주고 피드백받고 별풍(...) 받는 식으로 가겠죠.
피드백도 피드백이지만 채팅창에 여럿이 들어온다는 것부터가 이미 학원분위기 연출인지라.
그러고보니 콕도 어차피 대화방 있으니까 나중에 일격, 일타 저자 분들이 공지하면서
채팅창 띄우고 같이 문제푸는 타임 갖는 것도 좋을 것 같긴 한데. 생각해보니까 이거 가능하잖아?
\vspace{5mm}

