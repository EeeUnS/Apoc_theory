
%%%%%%%%%%%%%%%%%%%%%%%%%%%%%%%%%%%%%%%%%%%%%%%%%%%%%%%%%%%%%%%%%%%%%%%%%%%%%%%%%


\section{괴담}
\href{https://www.kockoc.com/Apoc/556127}{2015.12.21}

\vspace{5mm}

흡혈귀나 좀비 화의 문제 $-$
\vspace{5mm}

흡혈귀나 좀비가 하루에 2배씩 개체를 늘린다고 한다, 한달 뒤는?
2의 30제곱은 약 1000,000,000 = 10억
사실 흡혈귀나 좀비 자체가 과학적으로도 말이 되지 않지만.
\vspace{5mm}
\begin{enumerate}

    \item 수학괴담
    \vspace{5mm}

    하$\sim$상위권 통틀어 한문제 푸는 데 평균 5분이라고 치자. 어려운 문제와 쉬운 문제를 합쳐서 대략 가정
    그럼 2000문제라고 하면 10,000분 = 대략 166시간이 나온다.
    그럼 하루에 수험생들이 공부할 수 있는 수학시간은 2시간이라고 가정하면 이건 83일이 된다.
    일주일에 6일 공부한다고 하면 14주이고, 그렇다면 대략 3개월이다.
    자, 그렇다면 쎈만 하더라도 대략 7$\sim$8000문제가 되는데 그럼 1년.
    그런데 내가 듣는 괴담은 xx고 애들은 쎈, RPM, 라벨 다 푼다... 인데
    그렇다면 2000문제가 아니라 15,000문제를 넘어가는 꼴인데 그럼 몇년이 걸린단 이야기인가.
    \vspace{5mm}

    이게 시사하는 논점은 꽤 많다.
    첫째로 한문제에 5분당 걸릴 일은 없다. 많이 풀다보면 시간이 상당히 단축. 그러나 고난이도를 대비하면 한계가 있으리라
    둘째로 실제로 시중교재까지 저렇게 제대로 다 풀어대는 경우는 별로 없다. 만약 있다면 그건 문제가 중복되어서 시간이 별로 안 걸려서이다.
    셋째로 저런 괴담은 공부하지 않는 학부모들이 자녀 압박용으로 퍼뜨린다/
    \vspace{5mm}

    이렇게 수치화해서 접근하다보면 말도 안 되는 괴담들이 있다는 걸 알게 된다.
    그렇다면 그렇게 괴물적으로 처리한다는 xx고 분들이 다 입시 결과가 좋으신가?
    \vspace{5mm}

    \item 실모
    \vspace{5mm}

    의아스럽운 것 : 실모가 좋다, 많이 팔린다라는 이야기는 듣지만
    정작 검증해보면 정작 제대로 적중한 적은 없는데다가, 그렇게 많이 팔리는 데 왜 '많이 성공은 못 하시느냐'이다.
    가령 10,000부가 팔린다면 그럼 그 중 몇명이 드라마틱하게 올랐다거나 그 검증이 있어야 하지 않나. 그런데 그런 적은 단 한번도 없다.
    \vspace{5mm}

    그것도 그렇지만 실모 양치기 이야기가 나와서 그렇다면 $-$ 실모 1만원당 4회분 치면 여기서 의미있는 문제는 1회당 3문제 정도
    그럼 1만원에 12문제 정도가 유의미하단 얘기다, 나머지 문제야 시중교재들이나 기출에 널려있다.
    한 문제당 1,000원인데 이것들이 적중한다면 그리 비싸지는 않지만 적어도 실모 양치기란 말은 뭔가 이상하다,
    풀 대신 고기를 뜯어먹는, 웬지 개처럼 날렵한 양들을 몰다가 알퐁스 도데의 별 같은 분위기에서 아가씨 대신 아저씨를 만나는 기분?
    \vspace{5mm}

    수학문제 풀 때에만 논리적일 게 아니라(아니 그것도 논리적이지 않은 것도 문제지만)
    그냥 실모 좋다 나쁘다 떠나서 이런 것도 논리적으로 따져보면 좋을 것 같은데 수년 째 이런 논의가 안 된 것 자체가 신기하다.
    모처에서 모 강사 교재 비싸다... 라는 논의보는 기분임. 메시지는 좋다 그래, 그런데 그 메시지와 메신저가 모순이면 이상하지 않나?
    \vspace{5mm}

    \item 모두가 의사가 된다면
    \vspace{5mm}

    그나마 의료계통은 공급통제가 이뤄지고 있다고 하지만 이것도 의아스러운 건 많다.
    흔히 하는 이야기가 고령화 덕분에 노인들 시장으로 한 의료산업이 발달할 것이다...
    그런데 이건 뭐 흙파서 장사하는 것도 아니고 가장 중요한 전제가 있지. 그 노인들이 '돈'을 지불할 의사와 의향이 있나.
    막연하게 고령화라고 표현하면 그럴 듯 하지만, 실제로 2$\sim$30년 후 호구가 되어주실 수 있는 부유한 노인분들이 얼마나 되나가 중요하지 않나.
    \vspace{5mm}

    6$\sim$70년대 사람들은 이렇게 생각했을 것이다. 나라가 부유해지면 결혼을 많이 하고 출산률이 늘어나니 이 분야로 투자하자.
    지금 현실은 어떤가? 혹자는 가난해서라고 하지만 사실 195$\sim$60년대와 비교해본다면 아무리 헬조선이니 뭐니 해도 비교될 수가 없다.
    지금 저출산인 이유는 간단히 말해서 '민주주의', '남녀평등', '자아실현', '개인주의'다.
    이게 정당하고 정당하지 않고를 떠나서, 사람들 가치관이 더 이상 가족중심이 아니며 출산과 육아 자체를 '행복에 반하는 것'으로 보고 있어서디ㅏ.
    \vspace{5mm}

    그럼 앞으로 의사가 무조건 잘 나갈 거라고 하는 것도 이런 식의 섬세한 검증이 필요하지 않나?
    내일의 주식시장도 모르는데 먼 미래의 흐름이라는 것을 정확히 예측한다는 건 어렵다.
    그나마 선형적이고 기술적인 예측 하나만으로 보는 건 정확성이 있다 해도 '개인의 장래'와 관계된 건 세부적이고 미시적인 건 예측불가이다.
    \vspace{5mm}

    \item 인생 재단
    \vspace{5mm}

    자식이 재수하거나 삼수하는 등 20대 때 실패하는 경우에 비아냥을 많이 받는다.
    그런데 그 비아냥거리는 사람들은 왜 위인전에 으레 등장하는 실패 에피소드에 대해선 말을 하지 않을까.
    성공한 사람들을 보면 정말 죽음 직전까지 실패한 경우도 많다.
    오히려 큰 성공 직전에는 무시무시한 실패가 있는 패턴이 많지 않나.
    \vspace{5mm}

\end{enumerate}

중요한 건 실패를 했다기보다도 그 실패한 시점에서 가만히 무릎꿇고 노느냐, 아니면 그걸 \textbf{극복하려 하느냐 그게 아닌가}.
좋은 기회는 아무 것도 아니면 위기가 되지만, 위기는 잘 대응하면 기회가 되는 것이라는 건 표어가 아니라 실증 사례이다.
다들 정주영 이병철 돈 많다 어쩌구저쩌구만 하지 그 사람들이 정작 겪었던 실패나 불행, 그리고 뭔가 새로 시작할 때 받은 비아냥은 신경쓰지 않는 듯.
\vspace{5mm}

사실 성공하는 사람들을 찾는 방법은 다음과 같지 않을까.
\vspace{5mm}

\begin{itemize}
    \item 첫째, 남들과 다르고 터무니없는 것을 공부하고 준비한다.
    \item 둘째, 적극적으로 일을 벌이면서 실패를 자주 한다. 그런데 실패해도 일어나려 한다.
    \item 셋째, 그를 비웃는 사람들이 대단히 평범하고 멍청하다.
\end{itemize}
\vspace{5mm}

성공을 미래형으로 보느냐 과거형으로 보느냐 그 차이다.
성공'한' 사람에게 박수가 의미가 있을까, 성공'할' 사람에게나 의미가 있지.
하지만 대중들은 성공'한' 사람만 쳐다본다.
\vspace{5mm}






\section{학생 모의고사는 단 한번도 검증된 적이 없죠.}
\href{https://www.kockoc.com/Apoc/558076}{2015.12.22}

\vspace{5mm}

\href{http://kockoc.com/column/556875}{링크}
\vspace{5mm}

\begin{itemize}
    \item 첫째, 저 짤방의 학생은 실모를 푼 게 아닙니다. 사설모의고사들을 많이 풀었던 것이죠(...)
    현재 언급되는 실모들은 학생 모의고사들이 주류입니다.
    글쓴이가 제목대로 검증되었다고 하려면 저 여학생이 학생모의고사들을 300회 풀었느냐를 보여줘야죠.
    \item 둘째, 글쓴이께서는 정작 이과 수학을 모른다고 하시더군요.
    정작 본인께서 이과수학을 공부해보시고 갖가지 교재들을 풀어보시고 저런 말씀을 하시면 모르겠습니다만
    그렇지도 않은 데 그렇게 말씀하시는 건 상당히 문제가 많습니다.
\end{itemize}
\vspace{5mm}

그래서 저 글은 제목부터 고쳐야합니다.
일단 글쓴이께서는 수험도 수험이지만 '검증'이 무엇인가, '근거'가 어떤 건가 그것부터 확실히 하셔야하지
이과 수학도 모르시고 거기다가 모의고사를 많이 풀었단 방송 인터뷰를 "학생모의고사를 양치기하면 된다 검증되었다"하는 글을
실모양치기 효용검증사례(이과편)이라고 제목을 붙이는 건 문제가 있다고 지적드립니다.
\vspace{5mm}

문과수학에서 글쓴이 성적 이상을  '실모 양치기' 안 하고 거둔 사례도 있습니다.
어떤 암이 있다 칩시다. 허강탕을 먹어서 나은 케이스도 있고, 허강탕을 먹지 않고 나은 케이스도 있습니다.
그런데 허강탕을 먹어서 나은 케이스가 허강탕이 가격창렬이지만 최고다라고 하다가
허강탕은 먹되 그 이전에 항암수술은 받아야지... 라고 하면 그건 매우 우스운 논의가 될 것입니다.
\vspace{5mm}

그리고 이 사이트가 왜 실모에 비판적이냐... 그냥 이 사이트 혼자 '정상'일 거라는 생각은 안 하시나 모르겠습니다.
실모찬양론이 올라오는 건 어른의 사정이 있죠.
\begin{itemize}
    \item 첫째, 학생들이야 다 보지 않았으니 실모가 좋다고만 말하겠죠.
    \item 둘째, 실모들을 판매하는 사람들이 그럼 실모가 나쁘다라고 말할 리는 없죠.
\end{itemize}
그런데 이번 칼럼란에 재밌는 일이 벌어졌죠.
\textbf{미래 허강탕 판매업자이자 현재 실모 저자인 분이 실모에 매우 비판적인 입장을 취하시고}
반면 글쓴이 같은 분께서는 실모를 많이 풀면 된다라는 이야기를 하십니다.
사실 이 정도면 교통정리는 된 것 같습니다만.
\vspace{5mm}

실모든 사설인강이든 적당히만 이용하면 나쁠 거야 없죠.
그런데 문제는 별로 검증되지도 않았는데 무작정 좋다라는 이야기 때문에
1등급을 받지 못 하는 다수의 중하위권 학생들이 그 호구가 되어서 자신의 공부까지 망치는 경우가 많다는 것입니다.
흔히 실모찬양자들은 이렇게 얘기하죠. "뭐 1등급 안 나오는 애들은 기본교재부터 보라고"
\vspace{5mm}

그런데 말입니다. 실제로 '1등급' 대상으로만 판매할 리는 없잖습니까.
1등급 대상으로만 한다면 돈을 벌 수 있을 리가 없지 않습니까. 1등급이 안 나오는 친구들에게도 팔아야 재벌이 되는 것 아닌가요?
그럼 학생저자가 아닌 실제 교사, 학원강사 및 원장, 박사급 이상은 능력이 없어서 어려운 문제집을 안 내셨을 것 같나요.
그 분들은 현재 학생저자들 저리가라할 실력의 소유자들입니다.
어렵게 못 내는 게 아니라 더 많은 독자들을 위해서 중간 정도의 난이도로 내신 것입니다.
무엇보다 가장 중요한 건 그 분들이나 그 출판사들은 허위과장광고 따위는 안 한다는 것이죠(이게 뭘 시사하는지는 아실 것입니다)
만약 그 분들도 "이 모의고사만 보면 1등급이 나온다."라는 식의 광고를 하는 게 안 부끄러웠다면
훨씬 더 어려운 문제를 실은 모의고사를 내서 버셨겠죠.
\vspace{5mm}

학생모의고사라면 그냥 풋풋한 아마추어리즘이 생명력이어야하는 것이 아닌가 싶은데
마치 지금은 \textbf{"공인된 필수과정"}처럼 인식되었다는 게 문제입니다.
그렇다고 적중이 되느냐 물어보면 올라오는 답변은 "그러니까 걍 실전경험을 누리기 위해서"이라는데
실전경험이면 그냥 다른 저렴한 파이널 풀거나 시간재서 풀거나 복사집에서 학원사설모의고사(이것도 복붙성이 많지만)들을
제본받아서 풀어보아도 충분한 겁니다.
\vspace{5mm}

그리고  검증.
여라가지 차원이 있지만 적중도로만 보자면 그냥 역대 기출 '분석'만 해보면 됩니다.
시중에 어려운 문제가 없어서 학생모의고사를 풀어야한다는 논거의 맹점이 여기서 드러나는데
사실 수능 기출에서 문제가 되는 것들은 단순히 어려운 게 아니라, "패러다임이 다른 문제"를 냈다는 것입니다.
가령 적군이 화살을 쏠 거라고 생각하고 갔는데 레이저총을 들고 온다거나
이번 미술전에서는 정물화로 승부보아야지 하고 갔는데 갑자기 추상화를 그리라고 한다거나 하는 게 문제였습니다.
학원가 모의고사나 문제집은 보통 창의성이 없습니다.
A라는 기출이 있으면 그 A만 가지고 A', A", A"', A"'' 이렇게 꼬아내죠. 당연히 이렇게 가면 어렵습니다.
그러나 실제 수능기출은 A 시리즈를 기대하고 간 학생들에게 B라는 문제를 냅니다. 그러니까 못 푸는 것이죠.
\vspace{5mm}

학생모의고사에 대해서는 저는 이 글 보는 저자들이 혹시 상처(?)라도 받을까봐 해서 말 아낍니다만
사실 일필까지를 포함해서 과연 기출의 그 참신한 패러다임 쉬프팅까지 간 경우는 없다고 생각하고 있습니다.
추앙받는 실모들을 몇달 전에 서재정리하다가 과감히 버렸는데 도대체 그것들이 왜 추앙받는지는 3일동안 보아도 모르겠더군요.
그냥 아마추어리즘 학생모의고사라고 하면 어, 해설이 이렇게 부실하고 문제도 꼬아낸 정도는 인정할 수 있다.. 정도였지만
정말 이게 수능에 도움이 되는 걸까라고 하면 고개를 갸웃거릴 수 밖에 없겠더군요.
\vspace{5mm}

...
\vspace{5mm}

그리고 수학문제가 과거보다 쉬워졌다....
"어렵다"와 "쉽다"의 구분 자체가 참 애매하고 부정확한 것 같습니다.
쉬워졌다고 하면 그 어려운 방식으로 공부한 사람들이 무조건 다 96, 100이 나와야겠지요.
하지만 실증 사례를 보면 평소 모평에서 잘 나오거나 실력자라고 하던 친구들의 결과가 기대 이하인 경우가 많습니다.
혹자 이걸 실수라고 할 수도 있겠지만 우연도 반복되면 필연으로 수렴하는 것입니다.
\vspace{5mm}

사실 수학문제가 쉬워졌다... 라는 것은 어느 한가지 주장에 불과한 것이지요.
중간과정을 생략하고 말하면 요즘 수학은 쉬워진 게 아닙니다.
단지 문제를 꼬아서 내는 게 아니라, "근본적이고 기초적인 과정으로 풀어야만 패러다임 쉬프팅에 대처할 수 있게 낸다"는 것입니다.
예를 들자면 그 사람들은 멍게새우쵸콜릿맛이 나는 짜장면을 요구하지 않습니다.
그냥 짜장 아이스크림을 요구합니다. 다만 이 짜장 아이스크림은 짜장면의 기본을 잘 지키면 만들 수 있습니다.
\vspace{5mm}

점수가 깎이는 이유는 두가지입니다.
\vspace{5mm}

하나는 21, 29, 30에서 내는 생각하는 문제가 수리적 사고가 잘 박혀있지 않으면 풀 수 없게 낸다는 것입니다.
수리적 사고가 체화된 친구들에게는 그리 어렵지 않은데, 그냥 어렵다는 문제를 컬렉팅하는 친구들에게는 매우 어렵습니다.
풀이과정을 보면 생각보다 간단하죠. 하지만 시험장에서 떨리지 않고 이 신문제를 풀어낸다는 건 컬렉터들에게는 난감합니다.
\vspace{5mm}

다른 하나는 터무니없는 실수입니다.
21, 29, 30을 풀어댄다는 친구들이 쎈수학 B등급도 안 되는 것에서 터무니없는 실수를 해서 감점당합니다.
기본적인 개념과 연산의 문제인데, 이런 것은 쎈이나 RPM 같은 것만 충실히 했으면 그냥 충분히 대비할 수 있었지요.
그런데 이 어느 쪽이든 학생 모의고사를 꼭 풀어야만 대비된다... 라는 건 상관없습니다.
오히려 학생 모의고사에 집착하거나 그것들에 의존해버리면 악화시킬 수 있다라고 할 수 있죠.
\vspace{5mm}

앞의 것을 대비하려면 어려운 문제를 잡다하게 풀 게 아니라
시간 제한 걸어놓고 4점짜리를 스킬없이 순수히 교과서상 개념으로 본인이 '과정'을 분설해보는 훈련을 하면 됩니다.
자기가 생각 못 했던 새로운 유형을 "교과서"에서 배운 것으로만 도전해보는 것이 중요한 것이지
"많이 풀다보면 그래도 유형의 교집합이 있어서 대처가능할까"라는 식은 먹히기 힘들죠.
\vspace{5mm}

뒤의 것을 대비하려면 역시 시중교재로 양치기를 하면 됩니다.
학생 모의고사들은 생각보다 빠진 게 많아요. 출제 경향 좆는다고 하다가 어 F 개념은 나오지 않아라고 하다가
정말로 F 개념이 나와버리면 다 침묵해버리는 게 현실입니다. 이건 유형이 다 망라된 교재로 양치기하는 게 낫습니다(가격도 착하죠)
\vspace{5mm}

+
\vspace{5mm}

더불어 노골적으로 말하면 모의고사라는 것들도
창작한 문제를 공유하고 평가받고 싶어서 쓴다... 라가 아니라 그냥 $\_$ 이거 아닌가요?
돈을 버는 건 나쁘지 않습니다. 아니 오히려 제대로 문제를 만들어서 효용을 준다면 재벌이 되든 뭐하든 그건 욕먹을 게 아니죠.
문제는 터무니없이 비싸다는 것, 그리고 '아마추어리즘'으로 도피입니다.
가령 학생모의고사라서 이윤추구 동기가 적다면 대단히 싸게 팔 것입니다.
돈욕심이 없다면 문제가 다소 표절끼가 있다거나 해설이 엉터리여도 양해할 수 있죠.
그건 정말 아마추어리즘이니까요.
\vspace{5mm}

반면 값이 비싸지만 프로의식을 추구한다면 품질이 좋아야합니다.
정말 문제가 순수 창작이고 하나하나 개발하는데 엄청난 시간이 걸리며 해설도 백종원 요리방송만큼 이해할 수 있어야 하죠.
돈값을 한다면야 비싸게 팔든 누가 뭐라고 하겠습니까.
그런데 문제는 돈버는 건 프로인데 품질이 아마추어리즘이라는 것이죠.
다시 말해서 \textbf{돈은 프로처럼 벌겠다는 건데, 품질 문제가 지적되면 아마추어로 도피한다는 것}입니다.
\vspace{5mm}






\section{해설은 읽는 것이죠}
\href{https://www.kockoc.com/Apoc/559722}{2015.12.23}

\vspace{5mm}

수학의 해설을 보는 게 문제느냐 하는데 사실 그건 "읽는 방법"의 문제가 있습니다.
이런 질문들을 해보셨을 것입니다.
사실 유명강사의 강의 내용이나 해설은 그리 큰 차이는 없는데(접근방법 차이나 디테일은 있을지 몰라도)
\textbf{강의를 듣는 것은 괜찮고, 해설을 읽는 건 안 되느냐 말인가.}
\vspace{5mm}

읽는 것은 \textbf{대상을 입체적으로 조명한 뒤 논리적으로 타당한 순서로 밟아가는 과정}입니다.
\vspace{5mm}

수학 강의는 똑같은 내용을 더욱 상세히 쪼개서 '순서'대로 납득이 가도록 이야기해줍니다.
그 문제의 취지가 무엇인지 설명해주면서 기본 정의, 성질, 공식에서 어떻게 실마리를 잡아 풀어가는지 \textbf{순서대로} 이야기해줍니다.
그렇기 때문에 학생들은 '읽으려는 시도'를 하지 않아도 그 '절차와 과정'을 주입받을 수 있습니다.
그래서 강의가 도움이 된다는 것인데 이것도 일정 시점에서는 한계에 부딪치겠죠. 남이 생각해준 것이지 자기가 생각한 게 아니라서 그렇습니다.
\vspace{5mm}

반면 교재 해설은 지면과 분량의 한계상 풀이에 도움이 되는 내용을 '압축'해서 썼기 때문에 그걸 바로 알 수가 없습니다.
문제의 의도라든가 요건 해석 $\rightarrow$ 관련된 교과서상 개념 찾기 $\rightarrow$ 이용할 수 있는 조건과 단서 $\rightarrow$ 알고리즘 구성 같은 게 간략히 나와잇죠.
그래서 본인들이 읽을 실력이 없거나 읽으려는 노력을 하지 않고 해설을 보면 "푸는 패턴"으로 전락해버리는 게 해설입니다.
하지만 본인들이 읽으려고 하면서 그 문제나 해설을 쓰는 사람의 의도가 무엇인지 생각하면서 행간까지 쪼개서 순서를 잡으려고 하면
강의 이상의 무엇인가를 선사해줍니다.
\vspace{5mm}

요컨대 해설을 \textbf{본다}와 \textbf{읽는다는} 건 다릅니다.
단순히 보려는 자에게는 해설은 쓰레기일 수도 있습니다.
그러나 읽으려는 자에게는 해설은 무궁무진한 소스가 될 수 있죠.
단순히 보려거나 들으려는 자는 사실 아무 것도 알 수가 없습니다.
그러나 독해하거나 청해하려는 자는 쓰레기 교재를 금은으로 바꿀 수 있습니다.
\vspace{5mm}

해설을 읽지 않고 무작정 풀어대니까 되었다... 그리고 칭찬하는 댓글이 달렸는데 글쎄요 모르겠습니다.
만약 교재 해설이 정말 문제가 많다면 그래도 됩니다. 최소한 제가 자비로 구입해 명성대로인가 확인해보았던 학생 실모들 몇몇은
해설을 차라리 안 보는 게 낫다고 생각하는 경우가 있었으니까 그래도 되었을지 모르니까요.
또한 온갖 문제를 다 모아내서 보충용으로는 좋은 RPM의 경우도 도대체 이건 해설이 맞냐하는 경우도 있습니다.
하지만 그게 아닌 쎈이나 다른 교재들 $-$ 즉 대중적으로 많이 팔리고 저자진들도 검증된 경우의 해설은 그리 나쁘지 않습니다.
지금 서점에 등장하는 유명한 기출 문제집들도 해설을 잘 꼽아 씹으면 인강 못지 않을 것입니다.
\vspace{5mm}

그럼 수학문제를 풀기만 하면 되는가. 아닙니다, 푸는 데 성공한 경우라도 반드시 해설과 비교해서 읽어보아야합니다.
그래야 자기가 어디서 부족한가 또는 논리적 문제점이 있는가 없는가를 확인해볼 수 있습니다.
풀지 않고 해설을 보는 건 어리석은 짓입니다.
5번 정도 풀어보는 시도를 하고 그 실패의 과정을 남긴 다음 해설과 비교하며 읽어야 합니다.
해설을 수천번 보아도 깨달을 수 없습니다. 그러나 해설을 1번 제대로 읽으면 깨달을 수 있습니다.
본다는 건 현상을 그냥 긍정하고 아무 생각도 안 하는 것입니다.
반면 읽는다는 것은 현상을 의심(부정)하고 그래서 "왜?"라는 물어보는 것입니다.
회의하고 부정하고 가정하고 하면서 현상을 해체하다보면 '납득할 수 밖에 없는' 명제 단위까지 도달합니다.
참과 거짓이 분명히 드러나는 단계까지 가면서 성장하는 것입니다.
\vspace{5mm}

그렇기 때문에 역설적으로 수학실력은 수학공부가 아니라 국어공부를 통해서 늘어난다고 할 수도 있을 것입니다.
중학교 때 성적이 좋다가 고등학교 때 추락하는 현상을 설명하는 공식으로 "수학 = 국어 + 산수"를 들기도 합니다.
연산하고 답을 낼 줄만 안다면 그건 산수이겠지요. 사실 다수가 수학을 산수로 착각하고 있습니다.
반면 수학은 왜 그런 답이 나오느냐 하는 과정, 그리고 그 과정은 왜 나왔느냐하는 발상, 또한 발상은 어디서 나왔나 ... 묻는 과정입니다.
\vspace{5mm}

한국이 안전사고에 둔감하다라는 것을 가지고 기성세대만 욕할 것도 없습니다.
안전사고라고 하면 고속성장이니 불감증이니 말이 많습니다만 간단합니다. "논리적 지침을 안 지켜서"입니다.
우리나라 사람들이 문제인 것은 빨리 빨리가 아니지요. 빠른 것은 좋은 것입니다.
빨리 빨리가 문제가 아니라 '지켜야 할 것을' 안 지키고 skip해버리기 때문입니다. 그러니 매주매주 버라이어티한 사건사고가 터지는 것이죠.
작년에 선박 침몰 사건이 있었습니다. 그런데 이 사건과 관련된 것 어느 것 중에도 "체계적인 논리"란 것은 없습니다.
문명사회의 모든 사건은 사람이 만든 것입니다. 그럼 그 사람들은 어떻게 공부했을까요?
수학이란 과목에서 "풀면 된다"에만 집착하는 사람들은 그냥 평범하게 살아야합니다.
이런 사람들이 요직에 앉거나 큰일을 벌이면 분명 많은 사람들을 다치게 할 게 분명하기 때문입니다.
\vspace{5mm}

+
\vspace{5mm}

수학은 본디 유럽들 것이었지 우리 것이 아니었습니다.
중국인들은 순환론적인 특정이데올로기(ex 음양론)로 모든 것을 정당화하려 했습니다. 그래서 동양은 2000여년 발전이 지체됩니다.
하지만 그리스인들은 항상 부정하고 의심하고 회의했습니다. 기독교도 신이란 어떤 존재인가 논쟁을 하고 싸우고 그랬습니다.
그래서 16세기 경부터는 이미 비교할 수 없는 격차가 벌어집니다.
\vspace{5mm}

자기비하도 필요할 때는 해야합니다. 우리 전통에 수학은 없다고 단언할 수 있습니다.
회의하고 성찰하고 검증하는 것이 없기 때문에 수학을 배워서 '근대인'이 되는 것입니다.
근대인 = 즉 시민이 되기 위해서는 자본과 교양이 필요합니다.
많은 학생들이 이야기하는 흙수저 금수저는 '자본'이죠. 하지만 자본만 갖추면 뭐합니까. 생각할 줄 모르는데
다들 인터넷 글, 댓글만 보고 베끼고 짜깁기하면서 그게 '교양'이라고 착각하겠지요.
그런 교양의 기본으로서 수학을 배우는 것입니다.
\vspace{5mm}

문제를 많이 풀어댔다 어떤 것이든 풀 수 있다... 라고 말하는 친구들은 나중에 분명 큰 낭패를 볼 것입니다.
이건 입시 수준이 문제가 아닙니다. 사고, 행동 방식의 문제입니다.
자기가 밟고 있는 과정이 정말 문제가 없는가, 합당한가 따지지 않고 답만 구하면 된다..
이런 방식으로는 분명 초기 성공을 거둡니다. 실속 위주로 과감하게 승부하는 사람이 아무래도 승률은 높으니까요.
그러나 중요한 필수과정들을 스킵해버리면 챌린저호 꼴이 나버립니다.
분명 입시를 위해서는 점수가 가장 중요한 기준이겟지요.
하지만 명문대에 가는 것이 "더 크게 망하기 위해서"라면?
가능하면 "근대인"이 되는 방향으로 공부하는 게 바람직한 것이고
원래 우리 한국의 것에는 수학이 없으니 수학을 제대로 공부해서 그런 결함을 보완해야겠다고 해야할 것입니다.
\vspace{5mm}

입시에서 목적을 달성하지 못 한 경우도 아 나는 모자라단 말인가... 라는 감상은 그냥 한달 전에 끝냈어야하는 문제고
냉정하게 자기가 왜 실패했나 수학문제를 차분히 풀 듯이 그 해를 발견해나가면 되는 것입니다.
지금도 공부가닥을 못 잡았다면 역설적으로 그건 본인들이 입시수학을 풀 줄만 알지, 그걸 체화시키지 못 했다는 이야기이겠죠.
\vspace{5mm}

++
\vspace{5mm}

사실 이게 가장 중요한 것인데 뒤늦게 언급하자면
\vspace{5mm}

가애든 동네학원이든 인강이든
"그냥 이렇게 풀면 된다"라는 인스턴프 풀이야말로 잘못 가르치는 것입니다.
\vspace{5mm}

패턴 문제풀이라는 건 인스턴트 풀이가 가능합니다. A라는 문제가 있으면 ⓐ로 푼다, 이런 문제이죠.
그리고 사교육에서는 대개 이렇게 가르칩니다. 간편하고 쉬워서 다들 선호합니다.
그러나 수능 경험해보신 분은 아시죠. 어려운 3점이나 4점에서는 인스턴트 풀이가 안 먹힌다는 것을요.
\vspace{5mm}

ㄱ에서 ㅎ까지 가려면 ㄱㄴㄷㄹㅁㅂㅅㅇ ... ㅈㅊㅋㅌㅍㅎ 까지 \textbf{순서}대로 밟는 게 맞습니다.
그런데 이게 매우 번거롭기 때문에 "자음"이라는 개념으로 퉁치고 ㄱ$\sim$ㅎ로 요약하는 것이지요.
하지만 이런 요약을 하려면 먼저 자음체계를 제대로 배우고 납득하여 머릿 속에 체계를 단련시킨 다음에 가야합니다.
문제는 이렇게 교육시키는 경우가 적단 것이죠. 학교에서 엉터리로 대충 가르치는 경우도 많고 학원도 가애도 예외는 아닙니다.
그나마 인강은 다수가 보고 평가하기 때문에 덜할지 모르나 사실 심사하면 문풀에만 치중해 대충 넘어가는 경우가 많기도 하지만
저렇게 체계적으로 가르치면 '재미'가 없고, '재미'가 없으면 수강생이 줄어든다는 점이 가장 큽니다.
\vspace{5mm}

그래서 수학을 잘 한다 = 인스턴트 풀이가 가능하다... 라고 착각하는 수험생들이 헤매는 것입니다.
무엇이든 그래서 "하나"로 다 끝낼 수 있다는 컵라면스러운 것을 요구합니다.
수학은 컵라면처럼 3분 내에 물끓여 넣으면 완성되어야하는데 왜 연애하는 내 친구는 잘 되고 나는 안 될까
이걸 설명하기 위해 "쟤는 머리가 좋고 나는 머리가 나쁘다"라는 카스트제도 가설을 세우고 믿고 그렇게 망해갑니다.
\vspace{5mm}

수학을 못 하는 이유는 상당수 그렇습니다. '푼다' = '한큐에 해결해야한다'로 착각합니다.
그런데 어떤 수학문제든 한큐에 해결되는 건 사실 없습니다. 2점짜리 계산도 실제로 냉정히 들어가면
그 숫자가 어떤 집합에 속하는가, 그 연산은 xx 법칙이 성립하는가, 주의해야 할 조건은 없나 다 면밀히 따져야합니다.
이런 게 귀찮기 때문에 대충 넘어가는 식으로 가르치는 게 우리들의 잘못된 교육이죠.
수학은 컵라면이 아닙니다.
순서대로 지켜야 할 논리 과정들을 모두 떠올리고 그걸 순서대로 순열, 조합시키는 개념들의 이항정리라고 보면 됩니다.
\vspace{5mm}

공부를 못 한다는 친구들이 노력을 해도 안 되는 건 '논리적 체계'가 안 잡혔기 때문입니다.
체계가 안 잡혔다라는 건 "사고나 행동, 즉 일의 순서"를 못 지킨다는 것이지요.
논리는 결국 '순서'입니다. OX, 즉 참과 거짓도 근본적인 것들을 순서대로 나열하면서 가리는 것이지요.
\vspace{5mm}

제가 가르칠 때애는 절대 빨리 풀지 말라고 합니다. 그럼 제가 스피드를 혐오해서?
왕년에는 세자리세자리 곱도 암산으로 해댔습니다. 지금은 두자리두자리곱은 가능한 수준이지만 저도 왕년에 암산마니아였고 급했습니다.
그렇기 때문에 스피드의 폐해를 알고 있습니다.
스피드를 높이는 손쉬운 꼼수는 "과정"을 생략하는 것입니다. 즉, 다 순서대로 밟지 않고 중간에 뛰어넘는 것, 즉 스킵해버리는 것이죠.
열심히 숙련하고 올바른 기법을 개발하면(올바르다라는 건 그 기법이 모든 순서를 지키면서 근거를 갖추었다는 겁니다) 속도는 높아집니다.
그러나 이거에 많은 노력이 들기 때문에 많은 학생들은 스킵을 해서 스피드를 높이려는 유혹에 빠집니다.
\vspace{5mm}

결과는 본인들이 아실 것입니다. 사실 이건 글을 읽거나 컥챗에서 대화해보아도 알 수 있습니다.
뭔 궁예질이냐 하겠습니다만. 스킵하거나 물타기 해서 점수 억지로 올리는 친구와, 그렇지 않은 친구들은 말과 글에도 차이가 납니다.
실제로 N수까지 가는 경우는 운빨도 없지 않겠습니다만, 알고보니 기존에 공부 잘 한다고 했던 것이 스킵으로 이뤄낸 거품 실력이기 때문입니다.
반면 찬찬히 기초부터 한 친구들은 올해 시험도 그렇지만 천대받다가 정작 시험점수는 잘 나오는 경우가 있습니다.
\vspace{5mm}

남들보다 빨리 시작하자, 이게 황금의 3개월 모토입니다.
그런데 공부는? \textbf{남들보다 천천히 하자}. 다들 이걸 모르시더군요.
N수가 실패하는 이유는 3$\sim$4월에야 공부를 서둘러서 하기 때문입니다. 서둘러서 하니 부실공사가 되어버리고 스킵으로 완성된 실력입니다.
이런 친구들일수록 더 천천히 개념을 읽고 더 천천히 문풀을 해야합니다. 대신 스킵하는 것 없이 논리적으로, 답답하게 해야하죠.
처음에는 이래도 되나 할지 모르지만, 모든 순서를 다 밟으면서 훈려하다보면 '참속도'가 올라갑니다.
이렇게 천천히 공부해야하기 때문에 빨리 시작하라는 것입니다.
느리게 하는 공부를 빨리 시작하라는 것인데, 대부분은 늦게 시작해서 성급하게 공부하려합니다. 이래서 실력이 오르겠습니까.
\vspace{5mm}








\section{과잉언급되는 천재들}
\href{https://www.kockoc.com/Apoc/571900}{2016.01.01}

\vspace{5mm}

\textbf{$\#$ 중수와 고수를 나누는 구분은 다음과 같다}.
\vspace{5mm}
\begin{enumerate}
    \item \textbf{고1 수학 최고난이도까지 숙달되어있느냐 $-$ 블랙라벨, 실력정석 문제를 껌으로 풀 수 있느냐.}
    \item \textbf{국어실력이 탄탄한가}
    \item \textbf{ 성격이 급하지 않고 차분한가}
\end{enumerate}
\vspace{5mm}

고1 수학이 매우 잘 잡혀있으면 미적분도 사실 한달 내면 수능 수준으로 거의 다 정복할 수 있다.
(여기서 머리타령할 사람이면 고1 수학을 반복하는 걸 권하겠음. 고1수학은 정말 총론 중 총론이다).
그러나 인강을 줄창 들고 진도를 마친 친구라고 할지라도 고1 수학이 잘 안 잡혀 있으면 계속 문제가 터진다.
국어실력이 꽝이면 문제를 읽을 줄 모르거나 조건을 누락한다는 것도 그렇고
가장 중요한 건 성격인데 $-$ 수학 공부에 있어서 노력이란 "차분한 집중이 가능한 성격"의 형성까지 의미한다.
성격이 불안하거나 매우 급하거나 해서 문제를 해부하기보다는 답만 구하고 넘어가려고 하면 절대 발전할 수 없다.
\vspace{5mm}

그럼 위 3가지가 잘 된다고 천재인가. 그렇다고 할 수는 없다.
환경만 잘 갖추고 트레이닝 코스를 잘 만들면 1$\sim$3은 불가능하지는 않기 때문이다.
\vspace{5mm}

천재의 요건은 \textbf{대량생산은 불가능해야하지 않나.}
그런데 우리나라 사람들은 조금만 뛰어나도 천재라는 딱지를 붙이는 게 문제인 것 같다.
보통 천재라고 하려면 그 혼자만의 두뇌를 가지고 나머지 세계인구를 상대하는 수준,
즉, 집단이 이뤄놓은 문명을 개인이 바꿀 수 있을 정도여야하는데 우리나라는 이런 기준에 관대한 듯?
\vspace{5mm}

그것도 그렇지만 주제파악 못 하는 경우도 있다.
만약 본인이 개막장 환경에서 공부에 전혀 도움이 되지 않는 환경에서도 공부하고 싶어서 올라간 경우,
이런 경우라면야 '수재(秀材)' 정도로 칭할 수 있다. 아직까지는 인간의 영역이라는 이야기다.
천재(天材)에서 천(天)이라는 게 뭘 의미할지 생각해보아도 야 저런 표현은 함부로 쓰면 안 되겠구나라고 알 수 있다.
그 뛰어나다는 것도 선천적인데 생각해보면 수험지식은 선천적인 것과는 거리가 멀며, 오히려 선천적인 게 방해가 되는 경우도 많다.
\vspace{5mm}

\textbf{$\#$ 노력은 [log(a)X]}
\vspace{5mm}

노오력을 가장 잘 설명해주는 것은 로그함수일 것이고, 정확히 말하면 여기다가 가우스 기호까지 처리는 해야한다는 것.
다만 밑인 a는 사람에 따라 달라진다. 환경, 성향, 성격, 취향, 절박함 등이 어우러진 것.
될 놈은 된다 안 될 놈은 안 된다라고 하거나
노오력해보았자 소용없다는 케이스는 메시지보다는 '메신저'를 우선 보아야 하는데
전자의 경우는 가르치는 사람인 경우가 많은데 이건 결국 "성적은 유전자가 좌우하니 난 모르겠다"라는 책임방기와 똑같고
후자는 나가서 노가다도 안 뛰면서 엄마가 주는 밥이나 챙겨먹는 댓글충인 경우가 많다.
어차피 이건 간단히 반박되는데 남의 자식보고는 어차피 유전자가 좌우한다고 하는 인간도 자기 자식은 노력시키려고 하겠고
댓글 달면서 헬조선 싫어 노오력해보았자 소용없어하는 놈은 집안 다 망하고 굶어죽을 지경가면 그 때에는 또 먹고살려고 노오력하고 있다.
\vspace{5mm}

핵심은 노오력의 성과는 아주 천천히 나타난다는 것이다.
만약 a가 10이라고 하면 10, 100, 1000, 10000.... 이런 식으로 가야먄 겨우겨우 성과가 나타난다는 것이다.
그래서 여기서 참을성이 없는 사람은 '물타기'를 한다. 바로 [log(a)X] + F : 즉 F라는 상수를 일시적으로 더해 결과를 높이는 것인데
그럼으로써 사실상 거품으로 성적을 올리고 자기가 공부를 잘 한다고 착각하다가 입시에는 죽쑤는 사람들이 생겨나게 된다.
저기서 F라는 것은 사교육의 족보나 꼼수 혹은 야매교재를 보면서 일시적으로 점수를 올리는 경우.
이 친구들은 수험경향이라거나 무슨 평가원 코드라거나 어려운 이야기는 잘 하다가도 정작 시험성적은 개차반이거나 쉬운 것도 대답 못 한다.
\vspace{5mm}

노오력은 정말 정직하게 해야한다. \textbf{자기가 스스로 한 노력의 성과는 안 사라진다.}
그게 혹자 재수삼수라고 할지라도 그렇다. 자기 스스로 노력해서 실패의 늪에서 일어난 인간은 정말 빨리 성공의 낙원으로 가기 때문이다.
순수한 노력 X를 기울여서 만든 $[log(a)X]=S$는 $X^S$로써 작용한다.
정직하게 노오력한 사람들은 일확천금의 유혹이나 아주 멍청한 사기에 당하지 않는 이상은 잘 나간다고 보면 되겠지만,
반면 자식사랑한다고 부모가 사교육시켜주는대로 거품실력을 올린 친구들은 사실 그 이후로는 잘 나가기 어렵다.
\vspace{5mm}

다만 현 입시가 순수히 노력해서 '현역'으로 갈 수 있는 수준인가에 대해서는 약간 이견이 있다.
대략 짐작하면 평범한 애가 혼자 공부해서 하려면 +2년은 더 추가되어야하지 않나라는 생각.
왜냐면 학교는 도움이 되긴 커녕 방해가 되는 케이스도 많고, 지금 입시가 참 구조가 엿같아서 공정하지 않은 측면도 있기 때문이다.
그러나 이걸 본인 노력으로 이겨내야한다는 것만큼은 올바른 정론이라고 얘기하고 싶다.
\vspace{5mm}

\textbf{$\#$ 그럼 천재는?}
\vspace{5mm}

입시는 천재를 원하지 않는다. 실제로 천재가 있더라도 현행 교육제도에서 매장당한 케이스가 많을 것이다.
그리고 대부분의 천재썰은 부풀려진 케이스가 많다.
실제 입시에서 원하는 건 두뇌가 아니라 '엉덩이'라는 게 중요.
\vspace{5mm}

단도직입적으로 말하면 가장 중요한 건
\begin{enumerate}
    \item \textbf{ 환경}
    \item \textbf{ 습관}
    \item \textbf{ 기초}

    이 세가지이다. 이 중 하나라도 되어있지 않으면 이게 심각한 애로사항을 초래한다.

    \item 성격

    이 역시 만만치 않지만도 들 수 있겠지만 사실 1$\sim$3만 제대로 되어있으면 덩달아 치유되는 것이다.
\vspace{5mm}
\end{enumerate}

부모가 우리 아이 천재예요하는 경우나 자기가 천재라고 착각하는 아이들은 실제로 머리가 좋냐... 하면 그건 아니다.
수험과목이 신체 스펙을 요구하는 스포츠도 아닌데.
그러나 \textbf{기본기가 정말 잘 되어있는 경우. 남들이 한 패턴 떠올리려면 30초 걸릴 걸, 본인들은 1초 내에 3개 정도 떠올린다는 것.}
한마디로 지식로딩 속도가 좋다는 것인데 이게 천재라고 할 수 있는 건가... 충분히 훈련과 숙달로 가능한 것이다.
그리고 그건 본인들이 정말 학습에 도움이 되는 환경에서 자라서 그렇다.
하지만 그렇다고 할지라도 이 글을 읽는, 자신이 불행하다고 믿는 n수생들이 그런 환경을 자기가 스스로 조성 못할 것은 아니다.
\vspace{5mm}

습관은 두고볼 것도 없다. 적어도 내가 보았던 공부 잘 하는 친구들도 그렇고 내가 그에 속했던(...) 때도 그랬지만
정말 엄격한 청교도적 생활 습관 유지하면서 학습량을 꾸준히 달성해서 목표량 혹은 목표량의 30$\%$ 초과달성을 하루도 빠짐없이 3달 내내,
기상시각은 일정히 유지하고 공부에 방해되는 것은 멀리하면서 그렇게 살 때 성적이 잘 나오는 것.
기초. 모두가 망각하지만 실제로 수능 어떤 과목이건 어려운 문제가 안 풀리는 건 그와 관련된 기초가 잘 안 잡혀있어서 그렇다.
수험에서의 창의력이건 순발력이건 그건 가장 원론적이고 총론적인 기초가 숙달되어있느냐 뿐인데
그런 기초적인 것일수록 '쉽다고' 무시하는 케이스들이 많다만. 물론 그 무시하는 사람들이 공부를 잘 하느냐 하면 그건 아니다.
\vspace{5mm}

천재 따위는 수험에 필요하지도 않다. 그러나 '수재'가 될 필요는 있다고 보고
그 수재가 되기 위해서는 환경, 습관, 기초는 분명히 갖춰야 한다.
\vspace{5mm}

\textbf{$\#$ 부모}
\vspace{5mm}

서글픈 이야기지만 성적이 나쁘거나 공부하기 싫은 케이스는 가정환경과 무관하지 않더라는 것
뛰어난 학생이군요라고 할 수 있는 경우는 정말 부모들이 이런 '깨인 분들이 계시다니'하는 경우가 많고,
반면 내가 봐도 머리는 좋은데 왜 공부는 못 할까하는 경우는 정반대.
그리고 여기 추가하자면 조기교육을 시켰느냐 안 시켰느냐하는 게 참 오래 가는 것 같다.
5$\sim$6살에 어떤 조기교육을 성공적으로 시켰다면 그게 10년 이상 복리로 늘어나는 셈이니 그 차이는 무시하기 어려운 게 아닐까.
\vspace{5mm}

이래저래 상담하는 경우 내용 절반 이상이 사실 부모님들 문제다.
그리고 내 조언은 간단하다. 성년자가 되었다면 부모님에게 정신적 의존을 하지 말라는 것.
아니 그리고 나도 나이처먹으면서도 느끼는 건데 어른들 하는 말이 다 옳은 건 아니다(내 말도 다 옳은 건 아니지 않나)
잔인한 진실을 적으면 자식들보고 n수 그만두고 대학가라하는 경우는
진지하게 자녀 인생을 고민하기보다는, \textbf{"댁 아이는 어디 갔나요"}라는 질문을 회피하기 위한 게 많다.
자기 자식의 인생보다는, \textbf{자기 체면을 신경쓰는} 그런 부모들이 많다.
\vspace{5mm}

한데 나중에 "엄마가 N수하지 말라고 했잖아요. 나 그래서 그렇게 살았는데 이게 뭐야 책임져"라고 하면 반응은?
\textbf{"그럼 공부하지 왜 내 말 들었냐. 네 인생은 네가 알아서 할 것이지"}
따지고 본다면 스무살 넘어서도 부모님 시키는대로 사는 것도 웃긴 일이다.
그 때부터 하지 말아야할 것을 고르는 건 본인의 '윤리'로 해야하는 것이다.
\vspace{5mm}

\textbf{$\#$ 대학가도 별 거 없다는데 왜 공부해야 하나}
\vspace{5mm}

\textbf{엄밀히 말하면 공부하는 사람이 되기 위해서이다.}
실제로 이십대 중반을 넘어서 공부하는 사람 그리 많지 않다.
대학은 어떻게 보면 본연(?)의 기능에 돌아가고 있는 것이다. 취업기관이 아닌 학문기관으로(...)
하지만 대학이 취업을 시켜주든 말든 자기가 노오력해서 좋은 대학에 들어갈 수 있었다라는 경험은 다른 성공들의 가능성을 높여준다.
\vspace{5mm}

10년 전까지야 일단 좋은 대학에 들어가서 취업해서 조직에 뼈를 묻고... 아니 걍 조직 기생충으로 들러붙어 살아간다는 게 먹히긴 했다.
그러나 지금은 모든 연령이 다 먹고사는 것, 그리고 내가 뭘 공부해야 살아남을 수 있을까를 고민한다.
공부할 수 있는 걸 고민할 수 있으면 그나마 행복이다. \textbf{공부하고 싶어도 공부 못 하는 사람들이 더욱 많다}.
수능을 포기하고 다른 시험을 공부한다거나 바로 돈버는 노선에 뛰어들면서 한달만 지나면 느낄 것이다.
그나마 국가와 사회의 전폭적 지원이 이뤄지고 참고서 가격도 저렴한 편인데다가 정보얻기 좋은 게 수능이었음을.
아마 이런 걸 가지고 "저 늙은이는 팔자좋은 소리하고 있네"라고 할 사람도 있지만 내 반응은 간단
"자기들이야말로 아직까지 배가 불러서 팔자좋은 지 모르지. 왜 진작 내 말 안 들었을까 후회할테니 구경이나 해야지 ㅎㅎ"
\vspace{5mm}

대기업 취업이라는 개념도 아무리 늦어도 10년 내에는 사라지지 않을까. 인류문명의 경제 시스템 자체가 바뀌고 있는데 무슨.
아프리카 BJ 들이 잘 나간다는 것을 당연하게 여기면서, 왜 그들이 잘 나가게 되었는가하는 근본적인 시스템의 변화,
그만큼 그들이 돈을 벌었으면 누가 돈을 잃어쓸까하는 생각을 해봐아야 하지 않을까.
조직들이 무너지고 초개인들 $-$ 즉 부지런히 노오력하고 공부해서 개인 능력을 키워 영업하는 개인들이 잘 나가는 세상은 현재완료.
어떻게 보면 공부할 건 더 늘어나버린 것이다.
예컨대 자기를 연예인으로 내세워서 이미지 팔아먹으면서 돈벌어보았자
본인이 책을 안 읽고 공부한 게 없어서 무개념 발언을 하거나 상식도 없어서 엉뚱한 답변하면 한순간에 날라가는 거지.
\vspace{5mm}

$\#$ 황금의 3개월 중 1/3
\vspace{5mm}

콕콕 내에서도 공부할 사람은 하고 안 할 사람은 안 한다.
초가을 정도 되면 반응은 달라질 듯.
여기에 대해선 이견이 많지만 참 답답한 듯. 작년에 안 겪어보셨나.
시동 걸고나서 본격 공부가 되려면 최소 3개월은 지나야 한다.
(물론 하루 12시간 내내 공부만 하는 독종 케이스는 제외. 그런데 독종이라면 황금의 3개월 말하지 않아도 공부하지 않았나)
\vspace{5mm}

수험은 남을 이기는 것이고, 남을 이기려면 \textbf{더 많이, 그리고 더 일찍 공부하는 수 밖에 없다}.
찔릴 놈들이 많겠지만 적어볼까
\vspace{5mm}
\begin{itemize}
    \item[] \textbf{$-$ 남들보다 늦게 시작한다}
    \item[] \textbf{$-$ 앞서서 공부한 사람들을 압살할 수 있는 꿈의 교재나 강의가 있을 거라 착각한다}
    \item[] \textbf{$-$ 수험사이트 검색하면서 그런 거 없나하는데 시간 허비한다.}
\end{itemize}
\vspace{5mm}

그런 게 있을 턱이 있나.
머리 좋은 놈은 노오력하는 놈 못 따라간다. 물론 노오력하는 놈은 좋아하는 놈 못 따라가지.
그런데 노오력하는 인간들끼리 비교하면 일찍 시작하고 반복을 많이 한 놈을 못 이긴다.
\vspace{5mm}

내가 EBS 강의 빠는 가장 좋은 이유는 그건데
\begin{itemize}
    \item 첫째, 발췌해서 들을 수 있다.
    \item 둘째, 정말 기본적인 것만 설명하고 기교가 적다(기교는 남는 게 없다. 기초적인 것만 남지)
    \item 셋째, 다운받은 다음에 반복청취할 수 있다.
\end{itemize}
\vspace{5mm}

황금의 1개월 날린 사람들도 딴 생각하지 말고 자기가 취약한 과목 EBS 다운 받은 다음 2월말까지 돌리길
수능개념 강의로 xx 과목이 있으면, 그 과목과 관련된 선생들 인강 올해판을 다 다운받고 이해 안 가더라도 끝까지 시청한다.
필기까지 끝냈으면 그 인강의 mp3 버전을 다운받은 뒤, 스마트폰에 넣고 dice player 같은 걸로 1.5 배속 재생하면서 계속 듣고 다니셔라.
그렇게 해서 최소 3회청을 달성해도 힘들면 그 때 사설들으면 되는데
보통 강의를 1번 다 돌리고 그 다음 또 돌린 다음 3회청까지 하면 그 과목 체계는 거의 다 잡힌다.
\vspace{5mm}

이것도 본인들이 '안' 해서 그렇지 뭘.
\vspace{5mm}






\section{인강 활용법}
\href{https://www.kockoc.com/Apoc/572680}{2016.01.02}

\vspace{5mm}

$-$ EBS 강좌 기준 $-$
\vspace{5mm}

\begin{itemize}
    
    \item[] \textbf{$\#$ 국어, 영어}
    \vspace{5mm}

    동영상강의 : 한번 듣고 필기만 할 것,
    필기할 때에는 \textbf{곰플레이어의 캡쳐 기능을 이용하는 게 편함}. 즉, 강의 들을 때는 캡처만 하고 듣고 나서 필기하시라는 이야기
    (이것이 EBS 강의 추천 이유, 일단 다운받은 다음 곰플 같은 것으로 배속수 조절하면서 필기는 캡처하면 되므로 흐름이 끊기지 않음)
    그 다음은 mp3 강의 다운받으신 다음 통학, 산책할 때 음악 대신 들으실 것. 최소 3회청 이상하면 강의 뽕을 뽑을 수 있음.
    언어과목의 경우는 순수한 음성만으로 알고리즘 강화가 가능
    \vspace{5mm}

    \item[] \textbf{$\#$ 수학}
    \vspace{5mm}

    천천히 들으실 것. 무엇보다 필기가 중요함.
    역시 곰플 캡쳐 기능을 이용하는 게 편리할 것임. 강의를 한번 들어준 뒤 필기 제대로 하실 것.
    그런데 필기를 어디할 것이냐가 문제일 건데 이 경우는 노트 아니면 A4에 따로 필기하는 것을 권함.
    A4에 필기하는 경우라면 나중에 바인더링 제본할 경우를 고려해 좌측 란은 비워두시길 바람.
    수학강의는 mp3 강의에서 얻는 것은 없을 것임, 필기한 것을 반복해서 보시거나, 아니면 그냥 동영상강의를 빨리 돌려보는 걸 권함.
    역시 한번만 들어보면 아무 소용없다는 걸 강조
    \vspace{5mm}

    \item[] \textbf{$\#$ 과탐/사탐}
    \vspace{5mm}

    수학에 준함, 단 음성강의를 들을 가치는 있음. 국어, 영어 강의가 지겹다면 탐구강의 음성만 듣고 다니는 것도 도움이 됨.
    문제는 필기일 것인데 이게 강사가 올려준 pdf만으로도 커버링이 되지 않는 경우가 많음.
    시간이 걸리겠지만 역시 노트나 A4에 따로 필기할 것을 권함.
    강사마다 케바케이긴 한데 그 필기를 개념노트나 시중기본서에 단권화할 수 있는 경우가 있고 아닌 경우도 있겠지만
    그냥 나중을 생각한다면 단권화하지 말고 따로 노트를 만들어서 제본할 것을 권하겠음.
    어차피 단권화는 머리에 하는 것이지 책에 하는 것이 아님.
    \vspace{5mm}

    \item[] \textbf{$\#$ 사설강의가 더 나은데요?}
    \vspace{5mm}

    처음에는 그렇게 느껴질 것임, \textbf{처음 듣는 맛이 다르니까}.
    그러나 최후에 남는 건 기교가 아니라 '기본'임을 강조하고 싶음, 수능에 준해서라면 기교가 먹히는 경우는 드뭄.
    기본 지식이 반복, 압축, 집적, 세밀, 융화되어가면서 실력이 되는 것임.
    재미없는 강의라도 여러번 들어서 그걸 거의 암송할 수 있을 수준으로 만드는 게 훨씬 나음.
    잘 고른 강의를 나중에 2배속으로 들으면 그 강의속도대로 뇌가 움직임.
    \vspace{5mm}

    \item[] \textbf{$\#$ 강의를 인상깊게 듣는 법?}
    \vspace{5mm}

    손과 발을 움직여주는 게 핵심임, 자세한 원리는 나도 모르겠지만 들으면서 손가락으로 강사 하는 말을 필사하는 흉내 내거나
    강사의 리듬에 맞춰 발가락을 꼼지락거리는 사소한 것도 집중도를 높여줌.
    실강은 강사들이 일종의 연극을 하는 것과 유사해서 몰입할 수 있음, 그런데 인강은 영화를 보는 것과 같아서 몰입도가 떨어짐,
    가능하면 1.3$\sim$1.4 배속으로 돌리고(말이 빨라지면 듣는 사람도 긴장하니까) 캡처는 곰플 기능을 이용해서 간편히 하길
    자막이 지원되는 경우 혹은 그게 아니라면 강사의 말을 그대로 돌림노래식으로 반복하는 것도 좋음.
    아니면 자기가 좋아하는 음악을 BGM으로 깔고 강의를 들어도 좋지 않을까 싶기도
    \vspace{5mm}

    \item[] \textbf{$\#$ 강의 자막 파일}
    \vspace{5mm}

    정 강의 듣기 싫으면 간혹 올라오는 강의자막 hwp를 다운받아 출력해보는 것도 답임.
    읽는 것을 좋아하는 사람이면 더 나은 선택일 수도 있음.
    물론 자막이 지원되는 강의라는 전제
    \vspace{5mm}

    \item[] \textbf{$\#$ 강의는 한번 들어도 되지 않나요?}
    \vspace{5mm}

    아무개 강의를 따라가서 고득점 나온 건 아무개 강의가 훌륭해서일 수도 있지만
    $-$ 사실 요즘은 다 공부하고 노력하기 때문에 강의 질은 큰 차이는 없음 $-$
    오히려 똑같은 이야기를 반복청취하고 필기한 것이 더 중요한 듯. 즉 반복이 중요
    강의는 스스로 생각하지 않고 그냥 세뇌당하는 과정이라는 한계가 있지만,
    '반복세뇌'당하는 과정에서 지식 주입당하는 효과는 무시 못 함.
    손이가요 손이가 새우깡, 어르신의 인사돌, 이가 탄탄 이가탄.... 등의 광고가 대중들을 휘어잡는 힘 : 반복
    \vspace{5mm}

\end{itemize}




\section{수학에서 꿀교재 찾으려는 망상은 버리시길.}
\href{https://www.kockoc.com/Apoc/575074}{2016.01.04}

\vspace{5mm}

상대적으로 수험에 좋은 책이 있을지 몰라도, \textbf{한권으로 다 대비되는 꿀교재는 없습니다.}
\vspace{5mm}

엄격히 말하면 실력정석도 본인이 아주 뛰어난 실력자가 아니라면 '안 보는 편'이 낫습니다.
정석은 풍부한 수학적 소스를 담고 있습니다.
문제는 그것을 '주입식'으로 적어놓았다는 것이고 사고하는 방법을 가르쳐주는 것과 거리가 멀다는 것입니다.
\vspace{5mm}

그래서 중급자까지가 실력정석을 보면 생각을 하기보단 \textbf{'암기'해버리}는 방향으로 가고,
암기해버리는 방향으로 공부하다보면 \textbf{시험에 나올 문제만을 담는다는 꿀교재}를 찾는 망상에 빠지게 됩니다.
\vspace{5mm}

교재로 치면 현 수능은 교과서'만'으로도도 대비가능합니다.
하지만 교과서만 봐서 무리잖아,
당연하죠. 교재도 중요하지만 더 중요한 건 \textbf{문제를 해부하는 '행동' 영}역이니까요.
\vspace{5mm}

아무 것도 모르는 사람이 인강을 들으면 좋은 건, 인강에서는 그 문제 해부 방법이 소개되어있기 때문입니다.
그러나 중요한 건 본인이 직접 문제 해부를 해봐야하는 것이지, 백날 해부방법만 알아보았자 소용없다는 것입니다.
아무 인강이나 듣고 (EBS 인강으로도 충분합니다) 문제 해부 절차를 양식화시킨 다음,
어려운 문제를 해설을 보지 않고 수일 걸리더라도 혼자서 끙끙 수리논술 풀 듯이 해부해보는 경험을 해보아야합니다.
이런 해부에 여러번 성공하면 수학 실력만 올라가는 게 아니라 '인간'이 바뀌게 됩니다.
\vspace{5mm}

수능에 나오는 역대 기출에 쓰이는 도구는 모두 교과서 수준에서 끝납니다.
더 좋은 방법이 있을 수도 있습니다만 그런 건 지양되어가는 분위기이며, 아울러 교과서 개념만으로 스스로 풀 수 있게 공부해야지,
자꾸만 잡스킬이나 잡개념에 의존하려하면 \textbf{실력은 절대 늘어나지 않습니다.}
궁금하시면 기출분석을 해보시면 되는 데 그 어떤 것도 야매 교재나 학생 모의고사가 반드시 필요하다... 그런 것은 없습니다.
개정과정으로만 치면 시중에 좋은 교재는 많이 나왔고(그걸 다 풀 수 있을지도 의문)
정말 문제해부와 직접적으로 관련이 높다면, 작년 기출 경향으로 본다면 수리논술 문제들을 그 절차에 맞게 풀어보는게 더 낫습니다.
\vspace{5mm}

위와 같은 것만 지키면 수학은 그리 무서울 것도 없습니다.
마치 수학을 잘 한다는 것이 유전이라거나 어떤 특정한 교재만 보아서 된다는 식으로 여론을 조성하는 장삿꾼들이 있죠.
당연히 그런 교재나 방식은 개인적으로 분석완료되었습니다만,
'도움이 되기는 커녕 학생들 사고방식까지 말아먹을 수 있다'는 게 제 결론입니다.
교과서상 개념들을 아주 정확히 익히고 그걸 순서있게 나열해나가는 '논리적 양식'을 습관화시킨다,
이게 중요합니다.
\vspace{5mm}

예를 들면
A $-$ 중학교 수학과 고1 수학이 철저히 완비되어있지만 고2 수학부터는 모른다
B $-$ 중학교 수학과 고1 수학은 그저 그렇지만 고2 수학 진도가 선행되어있다라고 하면
이 경우 A와 B 중 누가 잘 하느냐... 논할 필요가 없고 A가 그냥 무조건 잘한다고 보시면 됩니다.
왜냐면 A는 수학의 총론적인 부분이 매우 잘 잡혀있어서 수능 범위 내용들은 'skin' 정도로 인식되고 진도나 이해가 매우 빠릅니다.
이런 친구들은 고2  수학 진도를 $-$ 수능 기출 문제까지 병행하면서도 $-$ 현행 과정으로 6개월 정도로 끝냅니다.
반면 B의 경우는 선행은 되어있지만 제대로 이해할 힘도 없고 논리적으로 사고하는 게 되어있지 않아 지루한 암기로 수학을 받아들입니다.
유감스럽지만 많은 사람들이 B에 속하고 그래서 입시에 망할 수 없다고 해서 장삿꾼들 배나 불려주고 있는 게 현실이지요.
\vspace{5mm}

21, 29, 30이 도저히 안 된다는 분들은 오히려 속도를 늦추세요.
그냥 시중교재만 착실히 풀면서 하루에 2문제식 그냥 저런 문제들을 애써 답구하려하지말고, 문제를 끊어읽고 어떤 개념을 써야할까 생각만 하십쇼.
그렇게 가다보면 어느 순간에 에라 모르겠다 하면서 문제를 해부하다가 유레카 하면서 답까지 도달한 자기 자신에 희열감을 느낄 것입니다.
\vspace{5mm}

$\#$ 잔소리 $\#$
\vspace{5mm}

+ 정석은 거의 30년 넘게 발전이 없었다... 라고 해도 지나치지 않았다고 봅니다.
이건 장점인 동시에 단점이기도 합니다. 장점이라고 하면 그래도 정석은 부정할 수 없는 어떤 굳건한 위상을 지키고 있다는 것이죠.
그러므로 수학실력이 되는 사람은 정석을 읽으면 얻을 수 있는 게 많습니다.
하지만 초심자들이 정석을 본다면?
혼자 억측하는 건 무리이므로 거금 들여 일본 책과 교재들을 연구했지만 현재는 정석같은 방식을 고수하는 책은 드뭅니다.
차트식 수학조차도 해설은 매우 친절한 편이고, 어떻게 문제를 풀어야 할지 알고리즘을 부드럽게 적시한 편입니다.
\vspace{5mm}

+ 문제를 암기하지 말고 개념을 암기하시길 바랍니다.
개념 암기라고 하면 문제쎈에 나온 개념 + 각 공식과 정리의 증명 + 유의사항 정도면 충분합니다.
한데 학생들은 개념 암기는 안 하고 자꾸만 야매교재에 나온 이상한 스킬이나 문제패턴을 암기하고 있더군요.
수능에서 21, 29, 30을 제외한 나머지는 쎈 B형 수준에서 거의 다 풀립니다. 21, 29, 30은 거의 다 신패턴아니면 기출반복이죠.
\vspace{5mm}

+ 그리고 간혹 보다보면 수험을 하기보다는 xxx님의 xxx 강의를 듣는다, xxx 교재를 푼다라는 수험코스프레를 더 즐기는 사람이 많습니다.
고득점을 받아 원하는 대학 원하는 학과에 합격하기 위해서 xxx 강의나 xxx 교재를 듣는 것입니다. 그것들은 목적이 아니라 수단입니다.
그런데 어떤 언플이나 광고가 자행되었는지 몰라도, 수험의 목적을 잃고 \textbf{xxx 강의나 xxx 교재를 보는 것을 목표로 하는 학생들이 있습니다(...)}
\vspace{5mm}






\section{일지 뉴비들에게}
\href{https://www.kockoc.com/Apoc/575186}{2016.01.04}

\vspace{5mm}

일지에 대해서 무슨 장사를 한다거나 자랑질(...)을 한다거나 하여간 그 당사자의 인격을 보여주는 반응들이 있습니다만.
일지 제안한 것은 별 개 아닙니다.
칼럼을 쓰니까 별별 질문들이 다 날라오는데 \textbf{'실천적인 공부'와는 매우 거리가 먼 것}이어서였습니다.
공부를 정말 한다면 날라올 수 없는 질문들이 귀찮아서(...) 일지 쓰라고 한 것입니다.
\vspace{5mm}

수험생들에게 상담해준다고 글 쓰다가 며칠 지나지 않아
배너 광고에 올라오면서 장사하는 사람이 한두명이 아니라 그럴지 모르는데
개인적으로는 그런 건 별 관심도 없습니다.
상담해주는 것을 기화로 노골적으로 장사하는 인간들은 매우 천박하다고 여기고 있어서입니다.
그렇게 돈벌어보았자 곧 날려먹죠. 그런 사람들의 그릇이야 뻔하니까요.
제가 관심이 있는 건 "공부를 못 하는 사람들이 극기(剋己)하고 성공하는 과정"의 반복과 일반화일 뿐입니다.
입으로만 헬조선이니 흙수저이니 그러지 말고 본인들이 어떻게 해서 금수저들을 이길 수 있을까 하는 것을 실제로 모색하는 것이지요.
\vspace{5mm}

이번에 일지 쓰시는 분들은 대략 2주차 되셨을 것이고 본인들이 양심이 있다면 느끼셔야합니다.
공부를 안 할 때에는 뭐든지 다 공부할 줄 알앗지만, 실제로 해보니까 \textbf{하루 최소 공부량도 채우기 힘들구나라는 것을요}.
그리고 사실 이게 정상입니다. 막연히 하겠다라고 생각하는 것과, 실제로 해보는 것은 전혀 다른 문제이기 때문입니다.
학습량을 정말 늘려야하는 경우는 간혹 보다가 지적해드리겠지만,
지금 10일차 넘게 쓰신 분들은 일주일에 하루 정도는 정말 원없이 노는 시간 가지시고
나머지 6일은 정말 보수적인 학습량을 최소한으로 책정하고 그것이라도 100$\%$ 달성하려고 하시길 바랍니다.
\vspace{5mm}

이건 콕부심(...)이고 뭐고 그럴지 모르겠는데
현재 콕콕 사이트가 1년동안에 공부를 그래도 하려고 하는 알짜 회원들이 늘어나는 것은
사이트 주인장이든 저든 아무 상관이 없습니다. 오로지 정말 공부하려는 사람들만 우대하고 활동하게 하는 시스템이죠.
게다가 여긴 상업주의와는 거리가 멀고 일격도 졸라 까는 곳이기 때문에 특정교재를 보아야한다는 것에 구애받지 않고
정말 좋은 게 어떤 건지 판별하면서 공부 경영을 계속할 수 있습니다.
\vspace{5mm}

일지를 쓰실 때에는 하루에 공부한 것과 놀았던 것, 몰라서 질문해야할 것과 깨달은 것등을 중심으로 적고
일주일에 최소 한번 정도는 다 '합산'해 정리시간을 갖고.
중요한 고민이나 질문에 대해서 콕콕사이트 네임드들에게 호출해보시기들 바랍니다.
\vspace{5mm}

+++ 딴소리 +++
\vspace{5mm}

이 사이트가 실모나 사설인강에 대해서 매우 비판적이다 어쩐다 그러는데.
이게 정상인 겁니다. 솔직히 제가 돌아보면 타 사이트는 지나치게 '상업주의'에 몰두해있고
xx를 안 듣거나 xx를 안 풀면 망한다... 라는 공포분위기가 조성되어있습니다.
물론 상품이 좋다면야 그걸 권하는 건 안 말립니다만, 곰곰히 보면 장단점이 제대로 검토된 적은 없습니다.
\vspace{5mm}

학생인 척 해서 xx 좋다 하는 알바들은 꽤 많습니다.
그걸 방지하려면 거의 공짜로 누릴 수 있는 컨텐츠 빼고 나머지는 정말 엄격하게 비판하고 이야기해야 합니다.
그런 비판을 감수하지 않으려는 강사나 교재라면 그냥 '배제'해도 됩니다. 그건 정말 자기들이 자신이 없다는 이야기거든요.
인터넷 글 보고 우왕 좋겠다 하면서 구매해보니까 생각보다 별로라서 후회하는 사람 한둘이 아닐 건데요?
\vspace{5mm}








\section{표절}
\href{https://www.kockoc.com/Apoc/578553}{2016.01.07}

\vspace{5mm}

아마 언젠가 대한민국 교육 전체가 표절이라고 된통맞는 날이 오지 않을까 싶은데.
\vspace{5mm}

\begin{enumerate}
        
    \item  사교육 분야는 원체 고수가 널림,
    지금 수험사이트들을 보면 자기들이 고수라고 하면서 돈 많이 번다 어쩌구 하지만 그건 다 개소리고
    고수급으로 치면 상상을 초월하는 노인(...) 분들도 많습니다.
    그런데 이 분들은 인강도 잘 찍지 않고 그냥 오프라인 강의로 활약하거나 아니면 업계를 떠났거나 그럼
    왜 그런가 생각해보면 간단한데
    벌만큼 벌어서 그런 것도 있지만
    어차피 그런 걸 \textbf{인강으로 공유해보았자, 그리고 교재로 내보았자 100$\%$ 표절당합니다.}
    \vspace{5mm}

    \item  우리나라는 지나치게 표절에 관대.
    물론 합리적인 기준은 있겠죠.
    가령 모의고사 어려운 것 100문제를 1년 걸려서 만들었다 하면 그건 창작이라고 볼 여지는 있는데,
    수학 전분야 다룬 것을 500쪽 넘게 쓰는 게 1년 정도 밖에 안 걸렸다 하면 100$\%$는 아니어도 일단 표절 의심은 갑니다.
    뭘 이런 걸 표절이냐 아니냐 하기 전에 '국제적 기준'으로 따지면 되고, 게다가 '참고문헌'과 '출처표기' 제대로 해보면 답 나오죠.
    고교생이면 모르겠는데 대학에 들어간 사람이 '이런 게 뭔 표절이예요'라고 하면 이미 '공부할 권리'는 포기했다고 보면 되겠죠.
    대학에 들어가서 등록금을 낸다는 것은 본인도 이미 '지적재산권의 거래'에 관여하고 있다는 것이고
    더 높은 학위를 받는다는 것은 표절에 있어 매우 엄격한 논문까지 쓰겠다는 것인데 말입니다.
    \vspace{5mm}

    \item  그럼 표절하지 않은 사람이 어디있느냐.
    책수집하다보면서 느끼는 건데 '표절했다고 보기 어려운, 그리고 저자 본인이 정말 창작했다고 할 수 있는' 책들이야말로
    일찍 절판될 뿐더러 빨리 사라지더라는 것. 특히 국내의 수학 책들은 유전자풀로 말하면 자가복제 갈라파고스 열화복제되었다 보시면 됩니당.
    왜냐면 저작권이 인정되지 않을 뿐더러 노력해서 성과 공개하면 '이런 데 표절이 어딨냐 걍 하는 거지'라고 베껴버리니
    충분히 독창적인 수학책을 쓸 수 있는 사람들도 집필을 안 하는 거죠. 수년 걸려 제대로 써보았자 보상 못 받는데 뭐하러 공개함?
    대신에 그런 것들을 짜깁기 표절해서 적당히 구라까는 교재, 아니면 문제수로 승부보는 교재들만 살아남는 것이죠.
    \vspace{5mm}

    \item  표절이 아닌 책들은 분량이 컴팩트한 대신에 그 문장 하나하나마다 논리가 잘 잡혀있습니다.
    제가 수집하는 책들이 주로 이런 것들임, 이런 책들을 찾아 읽어야 생각하는 법이 바로 잡히기 때문입니다.
    이런 경우 저자 스펙을 보면 30대 중반 이상은 넘어서고, 박사 이상이거나 아니면 그만큼 필드에서 활동했거나 뭐 그렇죠.
    거기다가 참고문헌을 제대로 적시합니다. 이런 저자들은 자존심이 있기 때문에 남의 저작물도 존중할 수 있는 것이죠.
    반면 표절한 책들은... 설명 안 해도 되겠죠? 나중에 나이 먹으면 자기가 그런 책을 냈다는 것에 부끄러워하지 않을까 싶기도 함.
    \vspace{5mm}

    \item  아무튼 표절하는 현실 인정한다 치면, 적어도 남의 표절을 까려면 자기나 자신이 속한 쪽의 표절도 다 까야죠.
    Clean Hands의 법칙은 최소한 지키라는 것이죠. 그런데 그런 걸 까지는 않음,
    그렇다고 해서 자기들이 생각하기에 피해입었다는 쪽은 그럼 오리지널인가?
    절대 그렇지는 않을 건데 말입니다. 우리나라 사교육의 오랜 역사를 보면 현재 인기있는 사람들이든 교재든 찰나의 포말에 불과할 건데?
    게다가 사실 우리나라 교육이 태생부터가 말이 좋아서 일본에서 받아들인 것이지 일본 것들을 수도없에 베끼지 않았나?
    \vspace{5mm}

    \item  동대문 보따리 장수가 프랑스, 이태리 명품 짝퉁을 만들어 돈을 버는 것까지는 좋은데(.. 라고 하지만 이거 상표법 위반이죠)
    거기서 자부심을 갖고 "내가 명품을 만들려고 얼마나 고생했는 줄 알아"라고 착각해버리면 답이 없죠.
    처음에야 '생존'하려고 표절한다고 하겠지만 그 표절로 부를 쌓으면
    슬그머니 "나 원래 잘난 놈인데"라는 자아실현(?)으로 생각이 바뀌고 분수를 잃는 것입니다만.
    본인이 창작실력이 없는데 성과는 내야 하니 남들이 표절이라고 확인할 수 없는 쪽을 찾아 표절하게 되겠지요.
    \vspace{5mm}

    \item  그래서 사실 교재 고민하는 사람들도 "짜깁기 표절한 것들 중 어느게 좋아요"라고 고민하는 것이니 웃길 따름인 것임.
    어차피 강의나 교재나 다 베낀 것들 투성이면, 그냥 양많은 것 선택하면 되는 데 뭘 고민하는 건지.
    수능에서 강사들 오류 저지르는 것이 뭐겠음, '짜깁기의 오류'인 것이죠.
    강사들도 그냥 지식 정리해주는 업자 정도로 보면 되는데 슬그머니 '참선생'의 반열에 서려고 하는 것도 웃기지만
    강의라는 것도 방만한 지식을 정리해준다... 정도로 보면 그냥 그런 것 잘 해주는 강의 하나만 아무거나 들어야 함.
    표절문제 제기되면 이 바닥이 원래 그렇다 하는 곳인데 xxx 들어야한다하는 건 남대문, 동대문에서 뭘 사야할까 고민하는 것과 똑같음.
    어차피 최종정리는 자기가 해야하는 것이 아닌가?
    \vspace{5mm}

    \item  교재도 마찬가지죠. 여기서 사설이니 EBS니 따지는 건 걍 웃긴 짓이긴 함.
    그러나 EBS 권하는 건 그나마 이제 무난하게 정리되어있어서 그런 것이고 여러가지 지원받을 수 있는 게 많아서 그런 거죠.
    어차피 다 짜깁기면 저렴한 가격에 많은 지식을 체계적으로 정리할 수 있는 교재 고르면 되는 것임.
    이게 무슨 박사급 논문쓰는 것도 아니고 걍 다 짜깁기한 것 가지고 노는 것인데
    이건 뭐 패스트푸드 점에서 햄버거 브랜드 따지고 편의점에서 삼각김밥이나 컵라면에 까다롭게 굴려는 것과 별로 다를 바도 없죠.
    
\end{enumerate}





\section{확통에 추가된 분할}
\href{https://www.kockoc.com/Apoc/580053}{2016.01.08}

\vspace{5mm}

누가 도입했는지는 모르겠지만
수험의 정석 $-$ 스케줄링이 바로 \textbf{저 자연수의 분할}에 있다.
이게 나중에 경영과학 같은 데 쓰는 것이기도 하는데 내용은 잘 도입하신 듯.
\vspace{5mm}

자연수 분할의 수 : P(n,1)+P(n,2)+...+P(n,n)
성질 : P(n,k)=P(n$-$k,1)+P(n$-$k,2)+...+P(n$-$k,k) , P(n,k)=P(n$-$1,k$-$1)+P(n$-$k,k)
\vspace{5mm}

고교 확통의 흥미로운 점은 2가지.
\vspace{5mm}

\begin{itemize}
    \item 첫째, 제목은 명색이 확률과 통계인데 정작 제목에 없는 '경우의 수'가 진주인공이라는 것(경우의 수만 잘 마쳐도 사실 90$\%$는 끝남)
    \item 둘째, 미적이나 기벡과 달리 '현실의 영역'에 절반 정도 걸쳐있기 때문에 학생 본인이 스스로 풀이방법을 개발할 수 있고 해봐야하는 것.
\end{itemize}
그래서 확률과 통계는 스스로 경우의 수를 만들어보고 그걸 본인의 현실 문제에 대입해보면 되는데
자연수 분할을 일종의 '학습량' 분할이라고 생각하면서 접근해보기 $-$ 가령 n=100 문제라고 하고 k=스케줄 단위라고 하면
합리적인 공부 계획을 세우는 데 도움이 될 수도 있다(물론 발상이 그렇지 그렇게 다 고지식하게 접근하라는 이야기는 아님)
\vspace{5mm}

그러고보니 경제수학의 전조인가.
\vspace{5mm}











\section{카스트 제도}
\href{https://www.kockoc.com/Apoc/581753}{2016.01.09}

\vspace{5mm}

인터넷은 신(新) 중세시대를 도래시킨 것 같다.
\vspace{5mm}

말의 영향력이 커진 사회다.
조회수와 추천수가 많으면 그 말은 사실이 되어버린다.
열람자들이 그걸 일일히 검증해보고 믿는 것이 아니다. 그냥 조회수, 좋아요가 많으면 믿는다.
초기에 인터넷이 도입될 때만 하더라도 이것이 정보격차나 폐쇄사회를 해소할 수 있을 것이라는 낙관적 전망이 있었다.
현실은 거품정보 양산, 그리고 기존의 폐쇄사회를 붕괴시킨 것은 맞지만 새로운 폐쇄사회를 낳고 있다는 것.
\vspace{5mm}

수험의 경우는 웃긴 게 있다.
무슨 갓$\sim$$\sim$ 시리즈가 돈다. 10대 애들이야 철이 없으니까(?) 그런 컬트에 열광한다 치자.
어차피 수험이라는 게 결국 사교육에서 지적재산권 전수받고 그걸 정리해서 팔아먹거니와
수험 잘 한다고 천재도 아니고(수재라면 몰라도), 그리고 올바른 방향으로 노력하면 되는 것인데 거기서 왜 함부로 갓$\sim$$\sim$ 을 붙이지?
누구나 처음부터 공부를 잘 한 것은 아닐텐데?
\vspace{5mm}

수험이 의미있는 건 '노력'으로 가능한 승부여서이다.
천재들만 시험을 잘 보고 엉덩이로 공부하는 애는 못 본다면 이 게임이 재미가 있겠나.
그런데도 마치 수험에서 천재가 존재하는 양 그런 식으로 이야기를 하면서 갓$\sim$$\sim$ 어쩌구 하는 것,
소꿉장난이나 서바이벌 게임 치고는 이미 '업자들' 논리까지 개입한 것 같기도 하고 아무튼 꼴불견이다.
인터넷에서 이런 컬트에 빠지지 않고 그냥 착실히 공부하면 인생 폈을 친구들이 이런 데 휘말려 시간낭비하는 것을 보면 더욱 그렇다.
\vspace{5mm}

대체로 흐름이
\vspace{5mm}

\textbf{갓아무개 등장 $\rightarrow$ 모 과목 몇분만에 다 풀고 만점 $\rightarrow$ 갓$\sim$$\sim$은 $\sim$$\sim$$\sim$ 강의나 교재를 좋아해 $\rightarrow$ 수험생들을 위해 전격판매}
\vspace{5mm}

그런데 과거에 저런 것 없어도 공부할 사람은 했고 잘 나갈 사람은 잘 나감.
그래도 혹시나 해서 저런 게 정말 도움이 되나 다년간 보지만 내 기준에서 보자면 저게 썩 도움이 된다고 보지는 않았다.
일단 교재들을 보면 4$\sim$5등급 애들은 전혀 이해할 수 없는 것도 많지만
저런 것들을 보았다는 친구들이 이미 1$\sim$2등급이라면 그런 걸 보지 않아도 잘 나올 수 있었단 얘기가 되기 때문이다.
\vspace{5mm}

특히 수학과 과학을 좋아한다는 사람들이 왜 수험에 대해서만큼은 카스트제도식 미신을 전파시키는지 모르겠다.
될 놈은 되고 안 될 놈은 안 된다... 라는 것이야말로 하나마나한 이야기고(저런 말한 녀석이 자기가 망하고 있을 때도 저 말을 추종할까)
머리가 좋다라고 하는 건 수험에 있어서는 큰 상관관계는 없어보이고(환경이나 선행이라면 모르겠지만 글쎄.)
예컨대 아인슈타인이나 퀴리부인 같은 사람들이 한국의 수험에서 좋은 성과를 거두었을까? 에디슨이면 몇등급 떴을까.
\vspace{5mm}

갓이과 어쩌구하는 녀석들이 극혐인 이유는, 수험에 대해서만큼은 이상하게도 \textbf{이과적 분석을 하지 않는다}.
마치 수학과 과학을 잘 하는 게 천부적인 재능이고 신의 영역이라는 식으로 중세시대 마인드로 돌아가는데
이걸 보면 사이비 종교 집단 $-$ 가령 일본의 오움진리교에서 도쿄대 나온 이공계들이 있더라하는 게 이해가 안 가는 것도 아니다.
고교수학문제를 빨리 풀면 뭐하나, 이미 사고방식이 르네상스가 오기도 전 중세 봉건영주 수준인데
\vspace{5mm}

수험에 미신이 있을까.
작년 수능도 콕콕 내에서도 성공한 사람과 실패한 사람이 있다. 그렇다고 실패한 것이 범죄는 아니지.
물론 실패함으로써 당사자는 징역 1년 이상에 벌금 3000만원 이상에 처해지는 죄수가 된다.
사실 재수생들부터는 죄수생인 게 맞다. 그래서 수험은 프리즌브레이크 석호필의 탈옥인 것이다.
하지만 탈옥한다고 예수부처알라를 외치는 바보는 없을 것이다.
기도를 열심히 하니까 웜홀이 열려 나갈 수 있더라... 이걸 믿는 바보는 없지 않나.
\vspace{5mm}

노오력하면 보상이 오나요 그러는데 이건 OX 문제가 아니라 부등식 문제가 아닌가.
중학교 과정까지 마친 친구가 서울대급으로 가려면 사실 5년은 걸린다고 보는 게 내 생각이다(평균적이고 일반적인 기준)
그런데 이미 선행해서 중학교 때 고교과정까지 미리 보는 녀석은 그 5년이 3년으로 감해지는 것이고
고등학교 3학년 때까지도 정심없이 놀다가 아 이제 공부해야하겠다하면 5수는 당연한 것이 될 수도 있다.
그런데 노력해도 안 되는데요... 라는 친구들을 얘기하다보면 노력은 분명 했다. 하지만 그게 '기준치 미달'이라서 그렇지.
노력하다가 중도포기하는 친구들의 토테미즘이 곰이 아니라 타이거라는 것만큼은 확실한 것 같다.
쑥과 마늘을 먹고 버티라고 했는데 이런 것 해서 뭐해 하고 뛰쳐나간.... 뭐 종족번식에는 성공하시긴 한 것 같다.
\vspace{5mm}

길게 공부해야하는 게 당연한데도 실패하는 사람 다수는 "아, 이걸 왜 해야 하는데"라고 하다가 저기 갓$\sim$$\sim$$\sim$ 시리즈들을 본다.
그리고 그 갓$\sim$$\sim$$\sim$ 들의 노하우만 알거나 그들이 보았던 교재만 보더라도 금방 따라잡을 수 있을거라는 착각을 하는데
그런데 있으면 이 글을 쓰는 내가 먼저 낼름했을 것인데 \textbf{내가 아는 한 그런 건 '없다'}.
문제를 더 빨리 풀 수 있는 툴과 스킬이라는 것도 수능에서는 안 먹히는 경우가 태반이지만
소위 지식의 효율적 가공 같은 것 $-$ 2x2 매트릭스나 MCSE 같은 건 이미 공개된 것이고
무엇보다 저런 것들은 반드시 부작용을 수반하기 때문이다.
\vspace{5mm}

올해 수능을 응시해서 성공할 수 있을까요 없을까요... 이건 무의미한 이야기가 아닌가, 그걸 누가 아나?
목사, 스님, 무당들도 수능은 무서워한다. 합격에 대해선 다 침묵한다.
심지어 밤마다 잠을 못 이루게 하는 악귀들도 어려운 수학문제 냅다보여주면 된다, 수학문제도 못 푸는 악귀는 병신처럼 보이지 않겠나.
뻔한 이야기지만 중요한 건 본인이 '공부에 미쳐있는 상태'를 스스로 자초하는 것이고
우선 수능과 관계없이 숨쉴 때마다 본인이 공부한 텍스트나 문제가 연상되는 그런 충만한 상태까지 가면 되지 않나.
\vspace{5mm}

학벌이 무의미한데 공부해서 뭐하냐는 질문도 있다.
$\sim$벌이 의미하는 것은? 학벌, 재벌, 문벌, 군벌 등의 특징은 "경쟁에서 자유롭다"는 것이다.
과거에 서울대만 졸업해도 먹힌 이유는 간단하다, 서울대에 합격한 뒤에는 \textbf{공부를 안 하더라도} '갓서울대'니 뭐니 해도 칭송받아서이다.
그러나 지금은 서울대를 가던 노인대를 가든 \textbf{공부를 안 하면 살아남을 수가 없다.}
서울대에 들어가더라도 본인이 공부하지 않으면 쓰레기가 되기 때무네 학벌이 무의미해지고 있는 것이다.
명문대에 가면 '너 공부 쫌 했네' 정도만 본다, 사실 그 정도면 족한 것이다. 그 뒤에도 공부할 건 우글우글하다.
세상의 변화속도가 점점 빨라지고 활동무대가 넓어지고 있기 때문에 공부하지 않으면 따라잡을 수 없다.
\vspace{5mm}

수능을 치건 안 치건 어찌되었든 공부는 계속하고 있을 수 밖에 없는 것이다.
동네의 한적한 공인중개사도 지역 아파트 브랜드동호수견적 다 외우고 어떤 매물이 오가는지 금융시장이 어떤지 딱 꿰뚫고 있어야 살아남는다.
하다 못해 노점상조차도 어느 거리에서 몇시에 사람들이 많이 오가는지 단속은 어떻게 해야 피하는지 그런 건 다 연구한다.
\vspace{5mm}

그런데 오죽 수험판만 신기하게도 카스트제도적 인식이 남아있다는 게 신기할 뿐이다.
공부 못 하는 친구는 죽을 때까지 못 하나?
석달 붙잡아놓고 국영수 문풀 1000개 이상 시키고 인강 빨리 돌리고 하면 스트레스는 받지만 호전이 있을 건 당연하다.
문제는 이걸 \textbf{'안' 한다}는 것이다.
왜 안 하냐? 해도 실패한다는 것이다. 그럼 왜 실패하냐? 여기서 어물쩍댄다.
자기들은 했다고 하지만 실제 기록을 보면 공부를 안 한 경우가 많다.
그나마 이건 낫다, 그런데 더 심각한 건 그냥 수험사이트보니까 반드시 $\sim$ 강의 보고 $\sim$ 교재 안 풀면 안 될 것 같단 것이다.
그런데 더 심각한 건 $\sim$ 강의 보고 $\sim$ 교재 풀어도 이해가 안 간다는 것. 그냥 자기들이 바보라는 것을 인증했으니 공부해보았자 소용없다란 결론.
\vspace{5mm}

인생을 포기하는 매우 합리적인 결론이 아닐 수 없다.
\vspace{5mm}






\section{문과의 시대가 다시 오지요.}
\href{https://www.kockoc.com/Apoc/584304}{2016.01.11}

\vspace{5mm}

그게 누가 뭐라고 해도
권력, 돈, 그리고 이성을 휘어잡는 것은 문과를 졸업했건 이과를 졸업했건
\textbf{"말글"}을 다루는 사람입니다.
관념적인 이야기이긴 합니다만
인간은 '기술'을 지배하고, \textbf{마음}이 인간을 지배합니다.
그럼 그 마음을 이공계 학문으로 대체할 수 있느냐. 지금까지 그런 숱한 시도가 벌어지고 있고
경제학 같은 경우도 투자심리를 계량화하려는 시도를 하지만 아마 불가능할 것입니다.
아직까지 심리가 어떤지 마음이 무엇인지도 불분명하기 때문입니다.
\vspace{5mm}

수학계통이 다시 인기(?)를 얻은 이유가 기업에서 수리적 사고능력을 중시해서라고 하는데.
그렇다고 과거동안 '비즈니스'에서 써야하는 수학적 사고의 툴이 새로운 게 생긴 것도 아닙니다.
현재의 이공계 인기는 간단합니다.
원래 취업시장에서 이과는 그냥 능력있는 노예, 문과는 잘리기 쉬운 '동지'로서의 인식이 있었는데
문과 취업시장은 헬 중의 헬이 되어버렸고(루피, 동료는 필요없다), 이제는 기업에서 원하는 건 노예 아니면 아웃소싱이어서입니다.
문과가 망한 게 아니라, 문과 쪽이 매우 중요한데도 우리나라의 문과 교육이 진보도 발전도 없어서 그래요.
대졸하고 나서 4개 국어 기본인데다가 법률, 경제, 경영, 심리에 쌈박해서 개인창업이 가능하고 바로 해외진출 가능한 다음 욕먹었을지.
의치한이 열심히 공부한다고 하지만 선결 조건은 '인원수 제한'이죠.
만약 면허가 없었다면, 그리고 인원수가 과다배출이었어도 그랬을까 하냐면 그건 아닌 것입니다.
\vspace{5mm}

인터넷 덕분에 화이트칼라 일감이 사라졌다... 는 건 맞는 말입니다만
그렇다고 해서 우리가 쓰고 있는 언어가 사라진 건 아닙니다. 언어의 영향력은 훨씬 더 강해졌습니다.
기존문과교육이 관리자를 위한 것으로만 교육되어서 그렇지, 'CEO'를 위한 문과 교육의 수요라면 줄지 않았을 겁니다(공급이 없어서 그렇지)
\vspace{5mm}

다수가 좋다고 하는 건 너무 맹목적으로 좆지 마세요. 그런 것 없습니다 $-$ 다 돌고 돌게 되어있어요.
먹고살기 위해서 의치한이나 공무원에 간다는 건 타당한 이야기지만, 성공하기 위해서라면 이건 다른 이야기가 되어버립니다.
지금 10대 분들이 기성세대와 차이가 없는 게 '대학 간판'이 모든 걸 결정한다는 것에 너무 매몰되었다는 것인데
대학은 근본적으로는 '공부하러' 가는 곳이지, '취업하러' 가는 곳이 아닙니다.
부득이하게 간판을 보는 현실 때무네 대학 간판을 높일지 몰라도 그 간판 하나로 결정되는 것은 아닙니다.
본인이 치열하게 공부하고 연구해서 뭘로 하든 그걸로 독보적인 실력을 보여주겠다... 그런 각오로 가는 곳이라고 봅니다.
남들이 카더라하면서 답은 [   ]다 하든 말든 그건 5년 뒤에는 언제 그랬냐 얼마든지 말바꿀 수 있습니다.
\vspace{5mm}

컴퓨터나 인터넷이 등장해서 정보처리를 해주니 공부할 게 별로 없을 것이다... 라고 믿었던 시대가 있었지만 현실은?
그 컴퓨터나 인터넷도 믿지 못해서 모든 걸 다 꿰어차고 암기하고 공부해야해서 공부량은 기하급수적으로 늘어나고
죽을 때까지 학생이어야하는, 그리고 죽기 전에는 죽음이 뭔지도 공부해야하는 그런 세상입니다.
현재의 경기불황 경제위기는 더 이상 기존의 싸이클이 아닙니다.
'질적'으로 뭔가 변한 것입니다, 즉 세상이 또 바뀌어버리고 만 것입니다.
\vspace{5mm}

미래예측서에 공통적으로 강조하는 건 전문직이 아니라 '영업능력'입니다.
A라는 상품을 팔려고 하는데 한국에 수요가 없다... 그럼 아프리카 남미까지 뒤져서 팔아버리는 게 더 중요해진 것이지요.
꼭 국제적이지 않더라도 뭔가 '팔아대는' 것 자체가 사실 재화와 화폐의 흐름을 촉진시키기도 하지만 말입니다.
\vspace{5mm}

+
\vspace{5mm}

믿거나말거나 모르지만 진짜 무서운 부자들은 있는 척도 안 하지만 아니 스스로 검소하게 삽니다.
파리떼들을 막고 싶어서도 그렇겠고 돈이 가족들을 타락시킨다는 것도 그렇겠지만
\vspace{5mm}

근본적으로는 "배고파야만 볼 수 있는 게 있다"
\vspace{5mm}

부자도 오래가는 부자가 있고 그렇지 않은 부자가 있는데 후자가 더 많습니다.
이 경우 그가 갖고있던 돈은 정말 그의 돈이었을까, 아니면 돈이 의지를 갖고 당사자 품에 들어왔다 나가버렸을까.
명백히 소유권은 내가 갖고있지만 정말 내가 감당할 수 없는, 즉 부릴 수 없는 돈이라면 그냥 세상이 나에게 맡겨둔 것에 불과한 겁니다.
과분하게 들어온 돈은 내 것이 아니죠, 그냥 내가 돈의 노예가 되어버릴 뿐. 언제든 나간다고 해도 이상한 게 없는 겁니다.
남이 돈 갑자기 많이 벌었다 질투할 것 없습니다. 그 돈은 다시 나가거든요.
눈 앞에 잔칫상이 차려져있는데 그게 무려 500인분입니다. 내게 남겨진 시간은 3시간입니다, 어쩌겠어요?
아까워서 꾸역꾸역 처먹다가 배터져 죽는 게 해피엔딩일까요?
\vspace{5mm}

믿거나말거나이지만 사주분석해보면 '재운'이 있는 사람들은 정말 신기할 정도로 돈이 잘 들어오는데
다시금 신기할 정도로 돈이 무섭게 나가버린다는 예측이 있습니다. 이게 다 맞는다는 건 아닌데 그런 패턴은 있더구뇨.
재극인이라고 해서 재운이 강하면 인성(=배움, 학습, 인내)이 극당하죠. 사주가 맞고 아니고 떠나서 이건 시사하는 바가 많아요.
돈에 눈이 먼다... 라는 게 참 시사하는 바가 많습니다. 어떻게 보면 자기가 쓰지도 못 할 돈인데 집착하다보니 자기 오성마저 망가지는 거죠.
돈을 번다... 라기보다 돈을 의지있는 생명체로 여기고 "돈에게 사랑받는 존재가 되기", "돈을 컨트롤 할 수 있는 능력 갖추기"로 바꿔야겠죠.
일본과 미국의 거부들도 그렇지만 우리나라 재벌들만 보더라도 그 사람들의 도덕성 여부를 떠나보자면
자본주의 사회가 아니더라도 어디서든 한가닥 해먹었을 그런 사람들입니다. 미래를 읽고, 남들이 안 하는 걸 착수하며, 사람을 부렸으니까요.
\vspace{5mm}

물론 빵을 외면할 수는 없겠습니다만 반드시 돈을 좆기 위해 공부한다고 하면 이건 '먹고살기 위해 몸을 팔아도 된다'와 똑같은 얘기가 되죠.
공부할 수 있는 시간은 한정되어 있습니다.
\vspace{5mm}





\section{자위권 해결 실천편}
\href{https://www.kockoc.com/Apoc/587686}{2016.01.14}

\vspace{5mm}

\begin{itemize}
    \item 도움(?)이 되는 온갖 컨텐츠들을 외장하드에 넣은 다음 꽁꽁 싸맨 뒤 깊숙한 곳에 박아넣을 것
    \vspace{5mm}
    
    \item \textbf{욕망이 발동하면 무조건 외출}할 것 $-$ 이어폰 낀 다음 휴대폰에 저장한 EBS 인강을 플레이하면서 나가 15분 이상 걸을 것
    \vspace{5mm}
    
    \item 그게 힘들다 싶으면 그냥 수면 취할 것.
    \vspace{5mm}
    
    \item 그런 식으로 해서 달력에 표시해나갈 것. 마치 쿠폰카드처럼 그게 30개 모이면 놀러갈 것이다라고 약속할 것.
    \vspace{5mm}
    
    \item 충동을 이기고 일주일차 이상 가면 몸상태가 호전되는 것을 느낄 것임. 2주차를 넘기면 자위에 빠져있던 때와 달라짐.
    \vspace{5mm}
\end{itemize}

+ 산책할 때는 그냥 나가지 말고 인강 다운 받은 것이나 음성녹음 같은 걸 끼고 움직일 것.
국어, 영어, 탐구 같은 것은 음성녹음만 들으면서 산책하는 게 학습효과에 도움이 되는 경우가 많음.
\vspace{5mm}

+ 적절한 자위는 도움이 된다.... 라는 논리면 적절한 술담배마약도 도움이 된다고 할 수 있음.
조금이라도 도움이 안 되는 건 그냥 안 하는 게 나음. 그리고 그것도 못 할 바에는 그냥 공부를 안 하는 게 바람직
\vspace{5mm}

+ 공부가 안 되고 피곤하면 소설이나 만화책을 보아도 좋지만 이것도 자위에 도움(...)이 되는 일이 있어서리.
그냥 이어폰$-$인강 들으면서 나가거나 아니면 수면이나 보충하는 게 나음.
\vspace{5mm}

+ 보통 청소년 상담에서 "신중한 검토 끝에 합의보거나 조절하는 게 낫습니다"의 의미는 \textbf{"걍하지 마 병신들아"} 이 소리임.
\vspace{5mm}






\section{실패가 두려운 게 아니라 도전을 못 하는 것이 더 무서운 것입니다.}
\href{https://www.kockoc.com/Apoc/587914}{2016.01.14}

\vspace{5mm}

제가 n수 조장한다는 여론이 있는데 여기서 해명(?) 비슷하게  하지요.
\vspace{5mm}

일단 여러분들은 '실패'를 두려워하면서 도전 자체를 포기하려하겠죠.
사실 이거 아무 것도 안 해도 해결됩니다.
나이먹으면 \textbf{도전도 못 하거든요}.
\vspace{5mm}

그런데 나이처먹은 사람으로서 말씀드릴 수 있는 건.
실패조차도 장기적으로 보자면 플러스가 됩니다. 그게 다른 도전에 도움이 되기 때문이지요.
\vspace{5mm}

무인도에 살면 웬수조차도 마주치면 반갑다란 말이 있죠.
지금 여러분들은 성공 vs 실패 라는 틀로만 보려고 하겠지만 사실 이건 불완전한 구분입니다.
성공, 실패의 전제는 '도전이 가능하다'라는 것입니다.
그런데 도전이 항상 가능한가.... 그렇지 않다는 게 문제죠.
\vspace{5mm}

실제로는 "도전할 수 있나 도전할 수 없나"가 더 중요합니다.
우리가 윤리적이다 비윤리적이다 따지든, 가난하냐 부유하냐라고 하든, 혹은 금수저냐 흙수저냐 하는 것. 이건 거의 다 유치한 것이죠.
이건 어디까지나 우리가 \textbf{'살아있을' 때에나 의미있는 것}입니다.
배가 부르고 몸이 편하면 자기가 잘 생겼느니 돈이 많으니 권력이 높으니 하면서 참 헛된 자랑을 하면서 그걸로 만족을 느끼려하겠죠.
그러나 본인이 말기암이라거나 아니면 사고가 나서 죽기 직전이면 "근심없이 숨쉬는 것 자체가 행복임"을 느끼는 겁니다.
\vspace{5mm}

자꾸만 시험실패하면 어떡해요... 라고 하는데 올바른 공부방법으로 빡세게 하면 시험으로 승부볼 수 있는 것이면 3$\sim$4년 내면 붙습니다.
(다시 말해 저걸 초과하면 그건 본인의 방법이 문제가 많거니와 제대로 공부 안 했다는 이야기입니다. 그래서 제가 오수썰에 비판적입니다)
님들은 떨어지면 어떡해... 라고 하겠지만 사실 이건 매우 배부르고 한심한 고민입니다.
지금 지구상에 님들과 나이가 비슷한 사람들은 공부하고 싶어도 못 하는 사람들이 대부분입니다.
강제 노역당하지 않고 총들지 않고 수험에 몰두할 수 있는 것 자체가 행복이고,
시험에 도전해 볼 수 있는 것 자체가 축복인 걸 모르는 사람들이 많습니다.
헬조선 거리는 사람도 해외 여러국가 다녀오면 현재 우리나라는 살기 나쁜 곳이 아니라고 말하게 됩니다.
\vspace{5mm}

도전할 수 있을 때에 남 눈치보지 말고 도전하십시오.
이 말에 수긍을 하든 안 하든 그건 제가 알 바는 물론 아닙니다.
그건 본인들이 나이먹으면서 느끼는 것이니 제가 뭐라할 건 아니기 때문입니다.
하지만 본인이 가고싶은 길이 있다면 그 길을 못 가면 그건 평생 한이 되는 것만큼은 트루입니다.
\vspace{5mm}

다만 도전을 한다면 자기에게 쓴소리해줄 수 있는 사람들 2$\sim$3명은 확보해놓고 매일 잔소리를 들으면서 도전하십시오.
\vspace{5mm}






\section{[공지] 콕콕 총회}
\href{https://www.kockoc.com/Apoc/589984}{2016.01.15}

\vspace{5mm}

\begin{itemize}
    \item[] ⓐ 일지를 30일 이상 작성한사람
    \item[] ⓑ 칼럼을 3편 이상 작성한 올린 사람
    \item[] ⓒ 운영진 및 기존 상원멤버
    \item[] ⓓ 콕콕 사이트와 유관한 교재를 출판하거나 교육활동을 하는 사람
\end{itemize}
\vspace{5mm}

일단 이 중 하나에 속하면 저에게 알려드리면 총회 자격을 드립니다('그리고'가 아니라 '또는'입니다)
저 요건을 만족시키면 뭔가 특별한 사정이 없는 한 총회 게시판에다가 카타콕 일지(비공개 일지)로 쓸 수 있습니다.
다만 들어오는 게 쉬운 반면 나가기도 쉽습니다. 즉, 어그로를 끈다거나 비속어를 쓴다거나 하면 쉽게 방출됩니다.
공개적으로 일지 쓰는 것에 부담을 느끼시거나, 그냥 비슷한 처지에 있는 사람들끼리만 서로 얘기하고 싶다고 하거나
일종의 친목 도모하면서 오순도순(...) 수험에 매진하겟다는 분들은 저 요건 중 하나를 충족시키면 댓글로 신청해주시길 바랍니다.
가볍게 확인한 다음 바로 사이트 관리자에게 보고해서 출입권한을 드리겠습니다.
아무래도 공개 일지는 다 공개하기 그런 것도 있고 불안한 것도 있사온데 그런 건 줄어들 것입니다.
총회의 1차 기능은 수질관리(...)이겠죠. 그리고 제 경우는 당연히 상원, 총회 순으로 관리하고 피드백합니다.
\vspace{5mm}




\section{각자에게 맞는 배우는 방식은 다릅니다.}
\href{https://www.kockoc.com/Apoc/592462}{2016.01.17}

\vspace{5mm}

A는 독서가 최적이고 강의가 쥐약인 반면
B는 강의가 최적이고 독서가 안 맞을 수도 있습니다.
C는 강의$-$복습 주입방식을 좋아하는 반면 D는 토론하거나 자기가 가르치면서 깨닫는 걸 선호할 수 있죠.
(사실 흥미로운 건 D입니다. 분명 자기는 그 내용을 모르는데 남에게 가르쳐야한다는 과제가 주어지면 제대로 공부하고 가르치면서 깨달으니)
\vspace{5mm}

이게 사람마다 참 방법이 다 달라요.
제가 xx만 들으면 된다, xx만 보면 된다라는 것이 위험하다고 보는 이유인데
사람마다 이렇게 접근방식이 다른 만큼 학습의 보편타당한 원리는 정말 신중히 접근해서 추출해야하는 것일지언대
수험 컨설팅을 하면서 특정 상품만 좋다라고 하는 케이스가 많습니다.
\vspace{5mm}

이건 금단, 역금단증세와 다릅니다.
본인이 공부를 안 해서 성적이 안 나오는 것과, 현재 듣는 강의나 교재가 안 맞아서 성적이 안 나오는 건 정말 다른 문제입니다.
졸라 노력했는데도 성적이 안 나오는 경우는 특수한 경우를 제외하면(다년간 껌씹고 오토바이타고 다녔다더라하는 것)
방식에 문제가 있는 것이기 때문에 이건 신중히 얘기하면서 접근 방법을 고려해보아야합니다.
본인에게 맞는 방법을 찾기만 한다면 다년간 고생했던 것이 3개월 내에 해소될 수 있습니다.
\vspace{5mm}

게다가 공부머리라는 것은 절대 1차 함수가 아닙니다. 상당히 불규칙한 f(x)를 베이스로 깔고 있는 가우스 함수이지요.
그게 일찍 트인 학생도 있지만 늦게 트인 학생들도 많습니다.
수학을 문자로만 접근하는 학생도 있지만 반면 이미지 $-$
그것도 자신의 운동이미지로 연상해서 가는 학생도 있는 등 접근방식은 참 다양합니다.
\vspace{5mm}

수학에만 한정해 말하면 참 재밌습니다.
국내 수학 사교육은 마치 자기 강의만 들으면 된다, 모 교재만 보면 된다는 식으로 만병통치약을 강조한다 그건데
경문사에서 나오는 수학교육에 관한 책이든, 일본에서 나오는 양서들을 보면
수학을 왜 학생들이 싫어하게 되었나라는 걸 진지하게 고찰하면서 나름의 해결방식들을 내놓고 있다는 것입니다.
그럼 누가 거짓말을 하고 있는 것일까요?
\vspace{5mm}

분명한 사실은 1등급도 거품 1등급이 있단 것입니다. 잘 나가다가 어느 순간 2$\sim$3등급으로 떨어지는 케이스가 있죠.
당사자는 자기는 열심히 하는데 왜 그런가 부르짖습니다. 그런데 이걸 '기반공사가 약한 마천루 부실공사'라고 하면 싹 설명될 터인데 말입니다.
사실 수학공부는 출제 경향이 바뀌더라도 점수 변동이 적은 게 당연할 터인데 말이죠.
아마 이 점에서만큼은 수학의 정석도 할 말이 많을 겁니다. 수학의 정석 입장에서 국내수학 교육은 자기를 중심으로 공전하는 것들로나 보이겠죠.
물론 우리가 아무 기초도 없이 정석을 지향하면 타죽을지도 모릅니다.
\vspace{5mm}

이런 것 다 무시하고 빨리 가고 싶다고 소위 합격기에 나온 커리만 따라가는 건 위험합니다.
n이 3이상 되어버린 사람들이 사실 이런 부류라고 봅니다. 자기 문제점을 해결하고 가야하는데 그렇지 못 한 채 계속 부실공사만 한 것이죠.
\vspace{5mm}






\section{계획을 짜고 실천하는 법}
\href{https://www.kockoc.com/Apoc/596969}{2016.01.19}

\vspace{5mm}

여러 번 언급한 이야기이겠지만 고민하는 분들이 많아서 다시 한번 올리지요.
\vspace{5mm}
\begin{enumerate}
    \item 목표달성에 필요한 작업량을 산출한다
    \item 주어진 시간을 산출한다.
    \item 작업량을 시간단위로 나눈다. (하루 필요작업량)
    \item 하루에 할 수 있는 작업량을 냉정하게 계산한다 (하루 가능작업량)
    \item 3번과 4번을 비교한다, 만약 하루 필요작업량이 하루가능작업량보다 크다면 하루 가능작업량을 늘릴 수 있는 방안을 모색한다.
\end{enumerate}
\vspace{5mm}

그런데 보통은 위와 같은 단계를 거치지 않고 그냥 하죠.
\vspace{5mm}

사실 그러기 힘든 게 어떤 교재까지 공부해야하냐 그걸 몰라서 그런데 해답은 간단합니다.
2016년 5월까지 EBS 수특, 시중교재, 기출을 다 끝낸다고 잡으면 되기 때문입니다.
그리고 계산해보시면 아시겠지만 저거 다 끝내려면 11월 말에 시작했어도 힘들다는 결론이 나옵니다.
\vspace{5mm}

이것도 머리가 아프다면?
\vspace{5mm}

그럼 수능 시험 당일에 풀어야 하는 문제량 : 국어 40, 수학 30, 영어 40, 탐구 40을 보정해서
\textbf{국어 120, 수학 100, 영어 120, 탐구 120 문제 가량을 꾸준히 할 수 있느냐}만 측정해보시면 됩니당.
당연히 처음에는 그게 불가능합니다만 진정한 고수라면 저 정도는 하루 내에 풀어낼 수 있어야하겠죠(거의 다 아는 문제여야하고)
\vspace{5mm}

이 시기 되어서 업자들이 또 광고질해대고 장사하자... 하겠지만
그런 건 라이벌들이나 권해주시고 님들은 그냥 기출, 시중교재나 풀어서 저런 양적달성이 가능한지나 검증해보세요.
특히 수험교재들은 적어도 사견상으로는 신사고, 천재, 미래엔 등에서 나온 것으로도 충분하다고 여기며
수학의 경우 특작, 수능다큐, 풍산자의 그 테마별 시리즈, 블랙라벨 수능전략서에다가 교과서만 봐도 다른 야매교재는 볼 필요 없다 생각합니다.
이러고도 세뇌당해서 자꾸만 이상한 교재들 보겠다는 분들 계시는데 인생 그렇게 낭비해도 제 알 바는 아니지만
최근 3년동안 정말 적중한 사례나 있는지 한번 검증해보시면 됩니다.
\vspace{5mm}

제가 빨리 풍산자와 쎈, 마플 풀어야한다고 했는데 벌써 1월 중순입니다. 한달 반 지나면 3월이네요.
냉정히 말해서 그거 다 끝낸 사람 그리 많지 않을 겁니다.
\vspace{5mm}








\section{부채도사}
\href{https://www.kockoc.com/Apoc/603292}{2016.01.21}

\vspace{5mm}

재작년부터 이미 평가원의 출제를 '사설'이 못 따라잡기 시작했다는 것.
\vspace{5mm}

개인적으로 예측한 것 중에서 맞은 것은 화생하지말라, EBS가 다시 중심이 될 것이다.
결정적으로 틀린 것이 바로 2015 수능 수학 난이도.
\vspace{5mm}

전자는 대충 수험생들 현황이나 온라인에 올라오는 정보를 취합, 그리고 무엇보다 저 자신이 교재 연구(?)라는 걸 하기에 알 수 있지만
후자는 평가원에서 독자적으로 내리는 것이나 사실 뭐라고 할 수가 없었죠.
물론 소가 뒷걸음질치듯 작년 수능에서 영어가 다소 어렵게 나올 것이다라고 하는 건 맞았으나,
그건 평가원을 해킹해서가 아니라 다들 영어를 물로 본다면 통수작렬 가능성 있다고 보았기 때문입니다.
\vspace{5mm}

시험은 경쟁과 출제의 교향곡입니다.
대체로 경쟁이 어떤 상황인지 출생년도에 따른 수준은 짐작가능합니다.
가령 현재 고2$\sim$고3들은 꽤 현실적이고 선행 3년치는 기본이라서 정시로 따라잡으려면 수년 걸린다... 정도는 IMF라는 대사건으로 요약할 수 있죠.
그러나 이에 대해서 평가원이 어떻게 나올지는 아무도 모르는 것입니다.
변별력이 없다고 하면 아예 물로 내버려서 실수하는 걸 노릴 건가, 확 마그마로 내버릴 것인가. 이건 부채도사입니다.
\vspace{5mm}

그런데 간혹 보면 후자가 가능하다고 보는 케이스가 있는데 그런 분들은 그럼 일주일치 주식이나 맞춰보라고 말씀드리고 싶습니다.
수험생들의 대략적인 수준이나 사교육 공급$-$수요 같은 것들이야 일정한 경향성을 띠고 있기에 그런 경쟁 정보에 의한 예측은 개연성이 있지만
평가원이 정말 어떻게 낼 것이다라는 건 사실 제대로 맞춘 사람은 단 한번도 없거니와
그런 식의 이야기를 하면서 장사를 해대는 업자들은 수능 이후 몇달동안은 그냥 침묵해버립니다.
\vspace{5mm}

다만 지금 하나 머리털 걸고 자신있게 말할 수 있는 건
올해 실패하는 수험생들은 '학습량' 부족 때문에 아작나기 참 좋을 것이라는 사실입니다.
이건 그럴 수 밖에 없어요.
\begin{itemize}
    \item 첫째, 사교육 어느 쪽도 2017 수능이 $\sim$ 하게 나올 거다라고 자신하거나 준비하지 않고 있다는 것.
    \item 둘째, 이 역시 학생들도 몰라서 그냥 꾸역꾸역 공부하고 있는 수준이라는 것
\end{itemize}
\vspace{5mm}

특히 재밌는 건 이와 관련해서 기존 교육과정에서 썰 풀었던 사람들도 지금은 유의미한 이야기를 못 하고 있단 사실입니다만...
사실 이건 안이할 수 밖에 없는 게 고2들 모의고사도 그렇고 작년 수능에서도 '신호'라는 것을 분명히 던져주었다는 것이고
그건 정말 왕도를 통해 공부한 사람이면 앞으로 어떤 식으로 애들을 엿먹일 수 있겠구나라는 실마리를 대충 잡을 수 있다는 것입니다.
그리고 뻔한 이야기지만 N수생들에게는 더 가혹해질 수 있습니다.
아니, 정확히 말하면 교과서와 기본을 무시하고 모의고사나 강의에만 의존하는 N수생들에게는 더더욱요.
\vspace{5mm}

2015 수능에서 평가원은 "우린 쉽게내면서도 너희 엿먹일 수 있어"라고 경고했고
2016 수능에서 평가원은 "논리적이지 못 한 녀석, 모델링 못 하는 녀석은 꺼져. 탐구는 복불복"이라고 얘기했죠.
각자 어떻게 공부해야할지 이것만 봐도 방향은 잡히지 않나요? 국어 영역 공부해서 어디 쓰나. 이런 데에 써야지.
사실 2015와 2016 연속으로 '업자'들의 그건 털렸죠.
\vspace{5mm}

올해 다시 N수하겠다는 분들은 작년보다 보수적으로 기간 잡고 학습량은 3배 이상일 거다 각오하고 가세요.
그리고 이럴 때일수록 그냥 기본적인 교재나 푸세요. 적어도 제가 돌아본 바, 업자들도 감 못 잡기는 마찬가지입니다.
그에 비해서 어떤 과목이든 낼 수 있는 방향이나 소스는 참 무궁무진하거든요.
\vspace{5mm}






\section{수학 기본서 평가}
\href{https://www.kockoc.com/Apoc/604542}{2016.01.22}

\vspace{5mm}

평도 점수도 주관적
\vspace{5mm}
수학의 바이블 $\bigstar$$\bigstar$$\bigstar$$\bigstar$$\largestar$$\largestar$$\largestar$
\vspace{5mm}

장점
\begin{itemize}
    \item[$-$] 해설이 친절함
    \item[$-$] 문제의 등급별 구분이 과외에 좋음(숙제내주기 좋은 구조)
    \item[$-$] 어려운 문제가 괜찮은 것들이 있음.
\end{itemize}
\vspace{5mm}

단점
\begin{itemize}
    \item[$-$] 처음에는 좋은데 나중에 보면 깊이가 없음. 초기에 비해서 발전이 없음.
    \item[$-$] 문제 유형이 생각보다 망라적이지 않음
    \item[$-$] 독자적인 사고를 길러주는 책은 아님, 교과과정을 쉽게 설명해주기 위한 눈치
\end{itemize}
\vspace{5mm}

수학의 정석(실력정석) $\largestar$ or $\bigstar$$\bigstar$$\bigstar$$\bigstar$$\bigstar$$\bigstar$$\largestar$
\vspace{5mm}

장점
\begin{itemize}
    \item[$-$] 제대로만 본다면 \item[$-$] 가령 5회독 이상이라면 포스 작렬함
    \item[$-$] 행간에 숨겨져있는 내용들이 장난 아님, 개념과 연습문제 하나가 최종보스급인 게 있음.
    \item[$-$] 이 책을 제대로 정복하고 수학 못 한다는 친구는 못 보았음
\end{itemize}
\vspace{5mm}

단점
\begin{itemize}
    \item[$-$] 대부분 베개로 라면 받침대로 씀, 사놓으면 뭐하나 읽지를 못 하는데
    \item[$-$] 문제를 해설이 못 따라감, 그리고 초심자가 보고 이해할 수 있는 해설이 아님
    \item[$-$] 소수의 실력자를 낳은 동시에 다수의 수포자를 양산. 다들 자기가 소수가 될 거라고 해서 달려들지만
\end{itemize}
\vspace{5mm}

숨마쿰라우데(구판, 7차, 신판) 평균해서 $\bigstar$$\bigstar$$\bigstar$$\bigstar$$\largestar$$\largestar$$\largestar$
\vspace{5mm}

장점
\begin{itemize}
    \item[$-$] 실력정석과 온갖 학원가 수학을 적당히 비빔밥화해놓았으며 개념 설명이 납득가는 수준
    \item[$-$] 편집이 매우 괜찮은 수준이거니와 적어도 학생들이 쓴 교재 치고는 억지가 아니라고 느껴질 수 있음.
    \item[$-$] 적어도 몇몇 설명이나 문제는 최상위권 지향
\end{itemize}
\vspace{5mm}

단점
\begin{itemize}
    \item[$-$] 구판이 가장 좋다는 건 개정해나가면서 하향되었다는 이야기가 아닌가, 정체성을 잃어버린 듯
    \item[$-$] 실린 문제가 어디서 많이 본 느낌이 남, 수학교재가 안 그런 게 어딨냐만 사실 실린 문제나 연습문제는 아쉬움
    \item[$-$] 처음 풀 때는 뭔가 괜찮은데 나중에는 실력정석이 끌리기 시작하는 이 느낌은 뭘까.
\end{itemize}
\vspace{5mm}

풍산자 수학 $\bigstar$$\bigstar$$\bigstar$$\bigstar$$\bigstar$$\bigstar$$\largestar$
\vspace{5mm}

장점
\begin{itemize}
    \item[$-$] 개념 설명이 쉽지만 핵심을 잘 찌르고 있고, 실린 문제도 쉬워보이지만 사실 어려운 것들이. 저자가 신경쓴 티가 역력함
    \item[$-$] 시중 나온 책 중에서 돌리기 가장 좋은 책임. 일단 수학 기본서는 빨리 돌릴 수 있는 것이 우선이 아닐까
    \item[$-$] 독학이 가능한 몇 안 되는 책
\end{itemize}

\vspace{5mm}

단점
\begin{itemize}
    \item[$-$] 저자가 최신 수능 경향을 반영 못 하거나 생각 안 하는 걸로 보임. 이건 다른 교재로 보충하길 바람.
    \item[$-$] 이 시리즈만으로는 수능 대비할 수는 없음. 그냥 기본서 중 기본서로 보는 게 좋음
\end{itemize}
\vspace{5mm}

문제 쎈 $\bigstar$$\bigstar$$\bigstar$$\bigstar$$\bigstar$$\bigstar$$\largestar$
\vspace{5mm}

장점

\begin{itemize}
    \item[$-$] 가장 망라가 잘 되었으며 단권화하기도 편리함. 등급 구분도 잘 되었고 개념 '편집'도 괜찮음. 탄탄한 안정성
    \item[$-$] 수험 동향을 가장 잘 반영하고 있음. 다수가 보는 무난한 책이면서도 단점을 찾아볼 수가 없음.
    \item[$-$] 실린 문제수에서도 타의추종 불허함.
\end{itemize}

\vspace{5mm}

단점
\begin{itemize}
    \item[$-$] 개념 쎈을 보기도 어중간한 개념 설명. 개념 설명에서 증명은 타참고서보거나 인터넷 검색해서 알아서 채워넣을 것
    \item[$-$] 해설이 간혹 납득 안 가는 것들이 있음. 그리고 일부 해설은 너무 억지 티가 나는 것도 없지 않음
    \item[$-$] 가장 좋은 풀이 엑기스나 유형조합을 하필 해설에만 넣어서. 다수 학생들이 그걸 놓치고 있음(...) 이거 따로 편집해서 핸드북으로 팔 것이지
\end{itemize}
\vspace{5mm}

마플 : 기출 $\bigstar$$\bigstar$$\bigstar$$\bigstar$$\bigstar$$\bigstar$$\largestar$, 개념 $\bigstar$$\bigstar$$\bigstar$$\bigstar$$\largestar$$\largestar$$\largestar$
\vspace{5mm}

장점
\begin{itemize}
    \item[$-$] 기출 : 유형별 정리 잘 해놓음, 양이 꽤 많은 편, 문제 선정 괜찮음,
    \item[$-$] 개념 : 기본 개념에다가 기출을 잘 노임
\end{itemize}

\vspace{5mm}

단점
\begin{itemize}
    \item[$-$] 기출 : 해설이 다소 억지인 게 있음, 다른 단원에 있어야 할 문제가 잘못 섞인 경우도 있음.
    \item[$-$] 개념 : 보충 개념서일지는 몰라도 이걸로 처음부터 가면 독학가능할까?
\end{itemize}
\vspace{5mm}

유난히 설명이 잘 된 수학  $\bigstar$$\bigstar$$\bigstar$$\bigstar$$\bigstar$$\largestar$$\largestar$
\vspace{5mm}

장점

\begin{itemize}
    \item[$-$] 개념서면에서 특화. 설명이 매우 상세함, 그리고 신선한 접근이 돋보임
    \item[$-$] 도해적 설명 자체가 매우 괜찮음.
\end{itemize}
\vspace{5mm}

단점

\begin{itemize}
    \item[$-$] 업데이트 되지 않음, '개념보충서'로만 간주하는 게 좋음, 특정개념이 정말 이해가 안 갈 때에만 봐도 됨
    \item[$-$] 문제선정이 아쉬움(업데이트가 안 된 탓이 큼). 상세한 설명은 해법셀파나 교과서를 참조해보아도 된다는 점에서 대체성.
\end{itemize}
\vspace{5mm}

수학의 원리  $\bigstar$$\bigstar$$\bigstar$$\bigstar$$\bigstar$$\largestar$$\largestar$
장점  
\begin{itemize}

	\item[$-$] 개념이나 문제가 꽤 크리티컬한 것들을 잘 겨냥해놓았음. 핵심 사항의 핵심적인 설명을 알고 싶으면 보시길  
	\item[$-$] 저자의 내공이 느껴지는 책. 심화개념서를 보고싶은데 정석이 싫다면 이 책으로 가는 게 좋다고 권함  
\end{itemize}
단점 
\begin{itemize}
    \item[$-$] 중심이 되기는 뭔가 아쉬움. 상위권용임을 차라리 표방하는 게 좋지 않았을까. 서브로는 좋지만 메인으로는 그닥임  
    \item[$-$] 생각 외로 독학은 어려울 수 있음, 책이 애당초 중상위권 이상을 겨냥하고 있기 때문.
\end{itemize}


제가 보지 않았거나, 별로 수험과 상관없다고 느껴지거나, 비평할 필요가 없는 것 등이면 언급은 안 합니다.
참고로 위에서 보통은 2$\sim$3 종류는 보지 않을까 싶은데 최근 수능은 위의 것들로도 넘칩니다. 
수능에서 커버가 안 되는 문제들은 사실 어떤 강의를 들어도 힘들 것입니다.
 본인들이 부지런히 연습해서 수리적 마인드를 안 키우는 이상은.
 메이저 출판사가 낸 것들은 거의 다 괜찮으니 어떻게 보느냐가 관건이지 뭘 보느냐는 딱히 중요하지 않죠.
 어차피 국내수학은 거의 다 짜깁기이죠. 대부분 수학의 정석에서 파생된 것이고, 정석조차도 뿌리는 일본인지라.
 물론 주체적으로 그 이후에도 계속 공부해서 교재에 반영하는 새로운 흐름이 있어서 다음과 같은 걸 잘 구분해야합니다.
 \begin{itemize}
     \item[] A $-$ 일본 것을 열화복제한 것이 국내 사교육 시장에서 진화된 경우
     \item[] B $-$ 일본 본토에서 최근까지 발전해 온 것을 국내에 유입시킨 경우
 \end{itemize}
 그런데 짜깁기 스킬이 수능이나 수리논술에 먹히지는 않죠.
 교과서가 재부각된 게 그런 이유.
  왜냐면 교과서가 역설적으로 B 경향을 잘 반영해서리그 저자진 분들이 경문사 등에서 낸 책을 보면 연구를 엄청 열심히들 하셨죠.
  사실 한권으로 정리되는 게 있을지는 의문. 
  완성시키는 건 자기 머리이지 책이 아닐텐데.
\vspace{5mm}







\section{[교재글] 개정교육 과정인데 비싼 교재 살 필요가 있나요.}
\href{https://www.kockoc.com/Apoc/605445}{2016.01.22}

\vspace{5mm}

분명 이런 가상의 질문이 있을 것 같아서 적습니다만.
\textbf{자연수와 집합의 분할 빼고는} 사실 구입할 필요가 별로 없습니다.
\vspace{5mm}

그럼 자연수와 집합의 분할은?
쎈 기준으로만 치면 내용에 비해서 쎈 역시 문제 해설이 부족하다는 걸 느낍니다.
머리 쓰시다보면 자연수, 집합 분할의 추가정리는 스스로 발견하실 수 있겠죠.
\vspace{5mm}

지금 참고서 시장이 웃긴 게
메이저 출판사 교재조차도 그냥 기존 과정들 개정에 맞게 재배치하고 문제수준 '하향'한 다음 신유형과 기출 덧붙인 수준입니다.
그럼 실제로 올해 수능이 어떻게 나올 건지 대비되어있는가... 하면 이건 \textbf{아무도 모릅니다}.
그렇다면 학생들이 할 수 있는 것이라곤
시중 교재와 기출 문제집이나 열심히 풀다가 EBS 수특, 수완 나오면 그거나 풀면서
모의고사 나올 때마다의 경향을 매번 주시하면서 눈치작전 들어가는 것이죠.
\vspace{5mm}

제가 수험생이라면
풍산자, 쎈, RPM, 일품, 라벨, 일등급, 실력정석 중 택 3하고 교과서 꾸준히 풀다가
이해가 안 가는 건 EBS 강의 아니면 사설 메이저 강의 공신력있는 것 하나만 따라가면서
돈 안 쓰고 모아두고 있다가 6평 치고 나서 좋다는 게 확인된 문제집이나 강의를 구입할 것 같네요.
실제로 시험 경향에 맞게 보정한 교재들은 그 때 정도야 나오겠죠.
학생들도 어리둥절할 뿐만 아니라 교재 만드는 사람들도 다들 어리둥절해야 정상입니다.
제가 교재 만드는 사람이면 가만히 눈치보면서 모평 어떻게 나오나 본 다음에 7월달에야 낼 듯.
\vspace{5mm}

콕콕에서 공부하시는 분들이면 현재로서는 출제경향은 모르겠으니 그냥 기본교재나 다 푼다... 와 같은 농부모드로 가시길요.
EBS에서 출시하기로 한 교재 보니까 3, 4점도 있는 걸 보니 풀만한 교재들이 부족해서 망할 일은 없을 겁니다.
오히려 6월에 가서 학습량이 부족해서 교재들을 소화 못 해서 내년을 기약해야하는 불상사의 가능성이 높을 뿐이죠.
작년까지의 교재 보신 분들이라면 굳이 바꾸진 마세요. 달라지는 것 생각보다 없고, 오히려 하향된 측면도 없지 않으니.
정말 꼼꼼히 만드는 업자들이라면 아마 6월 정도는 지나야 낼 것입니다. 그래야 경향성이 거의 다 파악되니까요.
\vspace{5mm}

아마 올해는 교재 리뷰를 쓰지 않을까 싶은데 $-$ 다룰 가치가 있는 것들만 $-$
교재와 관련된 글은 댓글란을 닫기로 합니다.
본문은 별 문제도 없는데 생각없는 사람들이 교재를 특정하거나
일부러 대놓고 모 교재 아니냐라고 특정하려하던 수상한 아이디들도 있어서(부끄럽지도 않냐?)
그런 건 아예 차단해버릴 생각이어서입니다.
\vspace{5mm}

적어도 제가 언급하는 교재들은 '단점'에도 불구하고 공부할 가치는 있는 것들이라고만 아시면 됩니다.
\vspace{5mm}

+ 다만 기존 이과였는데도 삼각함수 같은 것이 힘들다거나  문과였다가 이과 갈아타시는 분들이라면 그냥 개정 교재로 가는 게 편합니다.
\vspace{5mm}

++ 현명한 학생이라면 6평 전까지 시중교재 풀 수 있느 것 다 풀고, 그 때 가서 출제 경향 파악되면 거기에 집중하겠죠.
\vspace{5mm}








\section{수험계의 착취}
\href{https://www.kockoc.com/Apoc/615234}{2016.01.29}

\vspace{5mm}

편의상
\vspace{5mm}
\begin{itemize}
    \item[] A $-$ 업자
    \item[] B $-$ 상위권
    \item[] C $-$ 하위권
\end{itemize}
\vspace{5mm}

으로 잡자.
\vspace{5mm}

A라는 업자는 B를 겨냥한 상위권 교재나 강의를 낸다.
그런데 여기서 주의할 것은 상위권을 가르치는 것이 하위권을 가르치는 것보다는 더욱 손이 덜 가고 쉽다는 것이다.
왜냐면 심화로 갈수록 낼 수 있는 건 한정되어 있거니와, 상위권 학생들은 핵심만 짚어주면 알아서 따라오거나 청출어람하기 때문이다.
당연히 합격률이 높게 나온다.
\vspace{5mm}

그러나 여기서 주의할 건 저건 그 상위권들은 A를 거치지 않았어도 좋은 결과를 내었을 가능성은 매우 높다는 것이다.
\vspace{5mm}

그런데 문제는 C의 선택이다.
C는 자신이 하위권인 걸 머리로는 알면서도 가슴으로는 인정하지 않는다.
결과에 더욱 절실하기 때문에 몸에 좋은 보약 아니 머리에 좋은 것은 빚을 내서라도 하려고 한다.
그래서 A 업자가 B로써 결과를 내보인 그런 비싼 상품을 구입한다.
물론 그 결과가 어떨지야 우리는 너무나도 잘 알고 있다.
\vspace{5mm}

하위권들이 상위권이 되는 코스들은 분석하기 쉽다. 왜냐면 없으니까.
그런데 사실 분석하고 말 것도 없다. 이 사람들은 일단 n이  3이상 넘어가는 경우가 거의 대부분이고
수험과정을 보면 1$\sim$2년 정도 시행착오를 하다가 아주 기본으로 돌아가서 '쉬운' 것부터 \textbf{반복을 엄청 많이 하다보면 드라마틱하게} 오른다.
소위 머리가 좋다는 친구들은 대부분 환경이 좋고, 환경이 좋다는 건 어린 시절부터 예습, 복습 등 반복이 습관화된 경우가 많다.
학교와 학원이 시간적 효율성이 떨어질 것 같지만 효과가 좋은 건?
수업이 아무리 엉터리여도 그리고 야자가 야만적인 것처럼 보여도 그 시스템은 반복을 보장해주거든.
\vspace{5mm}

그럼 A가 파는 상품이 도움이 되는가?
유감스럽지만 그렇지 않다. 하위권들에게 도움되는 건 없다.
그러나 A가 버는 돈은 \textbf{'하위권'들의 눈물이다.}
\vspace{5mm}

기성세대고 뭐고 다 욕할 것도 없다니까. 자기들이 속한 수험판에서의 착취도 스스로 극복 못 하면서 무슨.
하위권들이 상위권이 되는 방법은 \textbf{훨씬 더 쉬운 교재와 강의를 더 많이 반복하는 것} 뿐이다.
반복을 하다보면 이해와 암기도 최소량이 보장되고, 그래서 자신감이 생기면 뇌에서 공부의 쾌감을 느껴 계속 공부하게 된다.
반면 상위권들이 본다는 코스 가면 당연히 이해부터 될 리는 없고, 그래서 자기 머리가 나쁜가보다라고 연속좌절하는 것이다.
\vspace{5mm}

단원 내용이 이해가 안 가면 그 전 단원이나 기초 과정에 돌아가서 더 많이 연습한 다음 다시 돌아와야 한다.
그래도 이해가 안 가면 관련 내용에 관한 EBS 강의 여러개를 3번 이상 들어보면 된다.
그래도 이해가 안 간다? 거짓말할 것도 없다. 저렇게까지 반복했을리도 없거든.
뇌를 길들이는 방법은 반복 뿐이다. 이 사실만 알고 있으면 호구가 될 위험도는 낮아진다.
\vspace{5mm}

물론 자기들이 어려운 수험생일 때는 업자들을 욕하다가,
자기들이 그 입장이 되면 수험을 신비화시켜서 장사하려는 더욱 악랄한 인간들도 있지만.
뭐 인간세상 돌아가는 게 다 그렇고 그런 게 아니겠나.
운이 좋아서 $-$ 가령 찍은 게 맞아서 $-$ 수험에 성공한 사람도
자기가 머리가 좋아서 그리고 정말 실력이 좋은 선택받은 존재라고 과시하고 싶어지는 게 인지상정일 것이다.
물론 그딴 것은 더 나이먹어보면 알겠지만 없다.
수험만큼 평범한 것도 없으니까.
\vspace{5mm}




\section{여러가지 잡담}
\href{https://www.kockoc.com/Apoc/619855}{2016.02.01}

\vspace{5mm}
\begin{enumerate}
    \item \textbf{환경}
    \vspace{5mm}

    어제 총회챗에서 상담하면서 종합한 건
    다들 공부머리가 있고 노력할 여지는 있는데 환경이 문제입니다.
    \vspace{5mm}

    집독학은 가능하면 피하세요. 집은 '내가 지배할 수 있는 공간'입니다. 그래서 공부가 잘 되지 않습니다.
    인간은 원래 공부하기 싫어하는 동물입니다(공부를 함부로 해버리면 자기정체성을 상실하는 탓입니다)
    공부는 강제당해야합니다. 그러니 공부를 강제하는 환경을 우선 조성해야합니다.
    집에서 하면 시간낭비를 안 하고 편히 집중할 수 있다하겠지만 실제 그렇게 해서 성공한 사례 거의 없습니다.
    \vspace{5mm}

    학원을 가도 좋지만 도서관이 가능하면 도서관에 일찍 출근하세요.
    도서관에서 최소 4시간을 버티면 됩니다. 4시간 하고 공부가 질렸다면 책을 빌려 봐도 좋고 도십시오.
    심심하다하면 인강 mp3 끼고 여행(?)을 가거나 몰링을 하세요
    특히 집에 있다가 폐인되신 분들은 하루에 1시간 30분은 쏘다녀야합니다.
    \vspace{5mm}

    \item \textbf{진로가 뭔지 모르겠다.}
    \vspace{5mm}

    제가 가장 싫어하는 게 코 앞의 일도 처리 못 하면서 왜 지구반대편을 걱정하느냐입니다.
    수능이 코 앞이면 수능을 잘 치고 그 다음을 고민하시면 됩니다.
    무슨 입결상담이니 그런 것도 다 소용없습니다. 점수 잘 받으면 끝나는 문제 아닌가요?
    \vspace{5mm}

    다들 자기들이 합리적이라고 착각하겠습니다만 솔직히 한시간 뒤에 뭔 일이 일어나지도 모르는데 미래를 기정사실화하고 고민하고 있죠.
    수험생들이 이런 경우는 대부분 공부하기 싫은 뇌의 핑계일 뿐입니다.
    \vspace{5mm}

    ex) 한의사 안 망해요?
    \vspace{5mm}

    한의대 가서 졸업한 다음에 따지세요
    \vspace{5mm}

    \item \textbf{과고의대 떡밥}
    \vspace{5mm}

    가장 한심한 논쟁입니다.
    우선 세금을 이야기하는 경우를 보면 대한민국 사람들은 특정 주제가 되면 그 전문가가 된다는 게 떠오릅니다만.
    \vspace{5mm}

    일단 기본적으로 왜 과학자 꿈에서 의사 꿈으로 바꾸느냐 그것부터 고려해야죠.
    간단히 말해서 똑같이 착취당해도 과학자들은 대우를 못 받습니다. 그런데 의사들은 대우는 받습니다.
    그렇다면 이 문제는 과고 출신들이 특정 커리큘럼 밟고 자격 요건 통과하면 의사만큼의 경제적 대우 해주는 걸로 해결할 문제죠.
    그러나 10년 넘게 흘렀어도 바뀐 것이 없습니다. 그럼 뭐 어쩌란 건지?
    \vspace{5mm}

    만약 직장인이라면 "유학갔다오는 대신 우리 기업에 5년간 복무해 그 지식을 활용해야한다"라고 계약하는 건 가능합니다.
    그런데 그걸 왜 특목고생들에게 '강요'하는 건지 모르겠네요. 거기 직접적인 법률 상의 권리 관계가 존재하는지도 모르겠고 말입니다.
    경제적 보상이나 그런 걸로 유인할 생각 안 하고 "투자한 만큼 뱉어내라" 이런 한심한 이야기나 하고 있으니 문제죠.
    가령 지금 변호사가 과거만큼 인기가 좋나요? 떨어지고 있죠. 보상이 낮아진다는 걸 아니까 그런 겁니다.
    의사로 몰리는 걸 막고 싶다? 그럼 이공계 처우 높여주든가, 아니면 의대 정원 늘리든가 하면 됩니다.
    \vspace{5mm}

    왜 본질적인 해결은 간과하고 그냥 '만만한 학생'들만 두들겨대는지 모르겠음.
    이 나라가 헬조선인 이유는 간단해요. 법 지키고 노력하는 사람들만 두들겨대니까 헬조선이죠. 애당초 그 어원도 그런 데서 유래된 것이고
    그런데 이 나라는 가만히 보면 '공부 열심히 하려는 사람'부터 작살내려고 하지요.
    \vspace{5mm}

    과학고 죽이기가 아마 초반에 있었죠. 비교내신제 날려먹기가 어디서 찌른 결과더라하는 루머가 있었고
    그 당시에는 왜 카이스트가 아니라 서울대에만 가느냐(...)라고 겐세이먹인 걸로 기억하는데
    어떻게 보면 공익적인 메시지 같지만 지금 보면 그냥 어이없는 공격입죠.
    \vspace{5mm}

    이공계 대우가 좋다면 의대로 빠지는 경우 드물건데 말입니다. 그런데 세월 지나도 이거 나아진 게 있나요?
    장학금 지원해줄 테니까 노예나 되라고 하는데 바보가 아닌 이상 누가 여기에 고분고분 따름?
    이공계 살리자고 하는 사람들이 사실은 살릴 생각도 없어요. 이공계가 나라를 살린다하면서 드립치면서 자기 자식은 다 고시, 의대로 보내던데 뭘
    자기 자식들을 그런데 보낸다는 건 결국 자기 자식들에게 유리하게 움직이겠다는 뜻이죠.
    \vspace{5mm}

    그럼 거꾸로 과학고 등에도 세금지원을 안 하면 이제 누가 과학고에 가나요. 다 일반고 가서 내신 학살하고 돈 많이 버는 데 가겠지.
    그래놓고 나면 이 놈의 나라가 헬조선이니 이공계가 망한다 또 그딴  드립치고 있겠죠.
    과학고 가서 열심히 공부해서 의대가면 욕먹고, 일반고 가서 의대 가면 욕 안 먹고. 이건 문제없나보죠?
    \vspace{5mm}



\end{enumerate}

\section{국어에 정답이 있을까.}
\href{https://www.kockoc.com/Apoc/619599}{2016.02.01}

\vspace{5mm}

문법(文法)이야 확실히 O, X 를 가릴 수 있죠.
왜냐면 문법은 어떻게 써야하느냐 지시하는 것이니 이랬다저랬다하는 게 곤란하니까.
\vspace{5mm}

문학/비문학 독해 문제에서 100$\%$ 정답은 사실 존재할 수가 없음.
1번 선지가 정답이고 2번 선지가 오답이라고 하는 건 1번 선지가 더 타당한 것이고,
더 타당하다는 것은 여러가지 관점에서 1번 선지가 2번 선지보다 우세하기 때문에지
모든 관정메서 1번 선지가 O고 2번 선지가 X여서가 아닙니다.
\vspace{5mm}

다시 말해서 문제를 풀 때에는 다양한 패널들이 있다고 칩시다.
패널 유재석, 김구라, 강호동, 조혜련 등
그리고 그 패널들이 각 선지마다 O나 X 패널을 들면서 지지나 반대를 표시하겠지요.
그래서 가장 많은 지지를 얻는 게 답이 되는 것입니다.
\vspace{5mm}

그런데 이걸 모르고 국어에서 O, X가 분명히 갈린다라고 착각하는데 사실 이건 매우 위험한 겁니다.
거꾸로 이것도 답인 것 같고 저것도 답인 것 같은데 어쩌냐하는 경우는 점수는 안 나올지 몰라도 사실 이게 정상인 것이죠.
\vspace{5mm}

국어에서 100$\%$ 정답과 오답이 갈리는 경우는 문제를 아예 그렇게 명쾌하게 낸다면 모르지만
사실은 그 수험생 자신의 가치관과 관정미 \textbf{그런 수험국어에 맞게 '토르소'가 되어버린 것}입니다.
문제는 그런 수험국어의 관점이라는 건 정상적인 사고와 거리가 멀거니와, 나중에 사람을 정말 꽉 막힌 선비로 전락시켜버린다는 것입니다.
수학과는 다릅니다, 수학의 생명력은 창의력이 아니라, 참과 거짓을 분명히 가를 수 있다는 것입니다.
형이상학적인 학문이니만큼 무엇이 진리이고 무엇이 거짓인지 밝히지 못 하면 쓸모가 없지요.
\vspace{5mm}

그러나 국어의 독해 쪽은 참과 거짓이 명쾌히 밝혀진다는 게 사실 거짓말인 것입니다.
그렇기 때문에 초기에 국어 독해를 풀면서 왜 이게 답이고, 저게 답이 아닌지 모르겠다하는 거야 말로 실제로는 정상인 것입니다.
\textbf{이 단계에서 제대로 공부하려면 왜 특정 선지가 정답인지, 오답인지 검사와 변호사 입장에서 주장해보는 식의 "나홀로논쟁"을 해봐야합니다.}
\textbf{그리고 여러가지 관점에서의 O, X 합산으로 다수결을 해보는 것이 맞습니다.}
적어도 수능기출은 다수결은 가능하게 출제해놓으니까요.
\vspace{5mm}

한데 시중 참고서든 강의든 국어에 무조건 100$\%$ 정답이 있다라고 해버리니까 여기서 상식있는 학생들이 헷갈리는 것입니다.
A라고 생각할 수도 있고, B라고 생각할 수도 있지 뭔 소리야 하면서 말이죠.
\begin{itemize}
    \item[] 국어 = 답이 여러 개일 수도 있다 = 다양한 관점에서 O, X를 검토해보면서 나홀로논쟁으로 다수결을 해볼 것
    \item[] 수학 = 답은 하나다 = 다양한 관점에서 다양한 접근법으로 단 하나의 결론에 도달할 것.
\end{itemize}
\vspace{5mm}

그런데 지나치게 수학이 강조된 결과, 수학에서의 접근법 그대로 국어로 가져가는 경우가 많습니다.
그리고 글을 읽거나 이야기해보면 순수한 국어적 관점을 몰각해버린 케이스도 적지 않아요. 우선 독서량부터 절대적인 결핍상태가 많지요.
\vspace{5mm}





\section{실패하는 애들은 다 이유가 있음.}
\href{https://www.kockoc.com/Apoc/621373}{2016.02.02}

\vspace{5mm}

이 시기에
\vspace{5mm}
\begin{enumerate}
    \item 논쟁
    \item 게임
    \item 컥챗
\end{enumerate}
\vspace{5mm}

과거에는 왜 어른들이 "하라는 공부는 안 하고"라는 말을 했나 반항하기도 했는데
지금 나이처먹고보니 그냥 그게 다 \textbf{진리}다.
기성세대나 상류층이 다 해처먹어서 그렇지 않느냐고 했을 때에도
지금 생각해보면 개념파에 해당하는 어른들이 '다 자기 탓이다'라고 해서 화냈던 기억이 나는데
자기 탓이라는 건 비단 '노오력' 뿐만 아니라 '학습'까지 포함한다는 것을 뒤늦게야 알았다.
\vspace{5mm}

예컨대 열심히 노오력하는 노예가 있다고 치자, 노예가 해방되는 방법은 하나다.
세상을 전복시키든가, 아니면 도망쳐버리든가, 아니면 몰래 공부를 해서 권력을 얻어 노예문서를 불태우든가
저렇지 않고서 열심히 일한다고 해도 기약은 없는 것이다.
하다 못해 부패하고 썩어빠진 어른일지라도 \textbf{"공부하지 말라고 한 적"}은 없었다.
젊은이들을 어떻게 착취해먹을지 고심하는 사장이라도 공부 이야기에서는 1$\%$는 착해질 수 있는 것이 아닐까.
\vspace{5mm}

망하는 애들은 다 그만한 이유가 있다.
본인이 돈을 버는 것도 아닌데
2월이 되었는데 게임하고 있거나 이상한 논쟁이나 벌이고 게임을 한다면 공부를 안 하고 있으면
당연히 망하는 거지 그럼 흥할 일이 있나?
지금 공부 안 하고 막판에 몰아치기 하면 된다... 그래서 그 몰아치기가 성공한 예를 알고싶다.
\vspace{5mm}

실패한 이유 가지고 하늘 탓할 것 없다. 공부한 것 따지면 되는 것이지
상담해보면 결국 구구절절한 사정도 \textbf{"공부하기 싫어서 꾸며낸 구라"}가 대략 95$\%$는 된다.
공부할 놈은 정말이지 목에 칼이 들어와도 책 읽고 문제풀고 있다.
자기가 내일 죽을 것을 알면서도 사과나무 대신 기출문제를 암송하고 있으면 된다.
\vspace{5mm}

그래놓고 시험치고 나서는 자기들이 열심히 공부했다고 자기 스스로 말한다.
공부했는지 안 했는지는 주변 사람이 평가하는 것이지 본인이 스스로 평가하는 게 아닐텐데?
그리고 공부 열심히 한 사람은 \textbf{절대 자기가 열심히 했다고 말하지 않는다}.
겸손도 있지만 사실 저게 맞는 말이거든, 아무리 해도 해도 부족하다고 느껴지는 게 공부임.
\vspace{5mm}

분명 집 떠나서 도서관에서 꾸준히 공부하라고 얘기했고, 게임 손에 잡지 말라고 강력하게 경고했으며
황금의 3개월 날리면 힘들 거라고 했는데도 이걸 어기는 친구들을 내가 어떻게 봐야할까?
\vspace{5mm}














\section{우유부단한 게 가장 최악}
\href{https://www.kockoc.com/Apoc/624579}{2016.02.05}

\vspace{5mm}

주식에서도 돈잃는 가장 최악의 패턴은
\vspace{5mm}
\begin{itemize}
    \item \textbf{오르기 직전에 못 견디고 팔아버림,}
    \item \textbf{떨어지기 직전인데도 오르고 있다고 사들임.}
\end{itemize}
\vspace{5mm}

기술보다 마음이 중요하다는 전형적인 예다.
\vspace{5mm}

마찬가지로 $\sim$ 할까 하는 사람들의 패턴은 세 가지이다.
\vspace{5mm}
\begin{enumerate}
    \item \textbf{부모님 충고를 듣고 $\sim$ 하는 경우}
    \item \textbf{부모님 무시하고 망하든 말든 해보겠다라고 하고 소신것 나가는 경우}
    \item \textbf{어느 것도 못 하고 시간만 낭비하는 경우}
\end{enumerate}
\vspace{5mm}

저 중 최악은 3번이다.
\vspace{5mm}

인생은 턴제 RPG가 아니라서 선택을 보류하는 동안에도 시간이 흘러간다.
한여름에 아이스크림 2개 중 하나를 택일강요받는다면 어느 걸 먹을까 눈돌릴동안 다 녹아버린다.
\vspace{5mm}

그럼 꼭 이런 이야기를 하지. \textbf{"실패하면 남는 게 없잖아요"}
아 뭐 이런 병신들이 다 있나.
\vspace{5mm}

실패가 남는 게 없긴 뭐가 없어. \textbf{"지혜"와 "교훈"}을 얻는데.
책에 쓰여진 그런 것 말고 본인이 괴로워하면서 체득한 자기만의 지혜와 교훈인데
아니 무엇보다 실패할 것을 알면서도 도전하는 사람이 남과 다른 게 있지. 그게 \textbf{'용기'} 아냐?
가슴 근육 키우고 배에 군주제 실현한다고 용기 있는 게 아님, 실패할 것을 알면서도 그래 해보자 나서는 게 용기 아녀?
용기 키운다고 해병대 갈 필요가 없다. 손해보는 것을 알면서도 그리고 심리적으로 위축되어도 해보는 게 용기지.
남들이라면 무섭다 손해본다라고 할 때도 \textbf{'에잇, 경험이다'라고 하는 게 용기 아냐?}
\vspace{5mm}

지혜도 없고 교훈도 없고 무엇보다 용기도 없다면 $-$ 특히 그게 남자면 $-$
그 사람이 좋은 대학 가더라도 인생은 별로 기대할 건 없다.
\vspace{5mm}

물론 용기와 만용은 다르다.
용기있는 사람은 실패를 하더라도 적어도 '안전선'은 마련해놓는다, 하지만 만용을 부리는 사람은 자살도 서슴지 않는다는 것.
그런데 만용은 '용기없는 사람'들이 궁지에 몰렸을 때 빠지는 극단적인 상태다.
용기를 강조하는 이유는 간단하다, "만용"을 부리지 않으려면 용기가 필요하기 때문이다.
\vspace{5mm}

예컨대 똥통대학 걍 다닐까요 아니면 재수할까요.
사실 20대 전체로 보면 어느 것이든 큰 차이는 없다.
똥통대학을 다니더라도 본인이 영업력이 출중하고 인맥 잘 잡아서 그 분야에서 먹거리를 잡으면 되는 것이고
재수를 하면 죽기살기로 해서 학벌 높이면 되는 것이다.
그 어느 쪽이든 실패하더라도 본인이 복기하면 지혜, 교훈, 용기를 모두 얻을 수 있다.
\vspace{5mm}

하지만 상당수는 선택이 모든 걸 좌우한다라고 믿고 있는 것 같은데
나 역시 그렇게 믿었지만 요즘 생각은 달라졌다.
우리 스스로가 용기도 없고 신중하지도 않고 노력하지도 않은 것 가지고 괜히 '선택' 탓을 하는 게 아닌가.
그럼 당시 선택은 그 당시에는 나름 신중히 숙고한 결과 아니었던가.
선택을 어느 쪽을 하는 게 문제가 아니라, 한쪽을 선택했으면 그냥 그걸로 밀고 나가는 게 답이다.
\vspace{5mm}

선택을 잘한 사람이 돈 많이 벌고 떵떵거린다, 나도 저렇게 살고싶다"라는 망상.
제가 말씀드리겠음, 그런 사람들이 \textbf{정말 나중에 제대로 망합니다}.
미신론적인 추명학으로 말하면 별 노력도 안했는데 돈이 굴러오는 사람은 재운이 들어오는 건데요,
그거 절대 공짜 아닙니다. 재운이 나갈 때는 정말 신기하게 잘 망합니다
게다가 그런 사람들은 실제로는 지혜도 용기도 없고 무엇보다 따라온 사람들이 다 돈보고 따라온 사람들이라서
재운이 나가면 정말 비참하게 망하고 아무도 안 쳐다봅니다.
\vspace{5mm}

이야기한 김에 더 적으면 '비정상적으로 좋은 운'을 자기 능력으로 착각하는데
능력과 실력은 "나쁜 운"이라도 걷어내는 것을 말하는 것입니다.
진짜 실력자들은 운이 나쁜 사람들에게 있습니다, 그리고 이런 사람들이 나쁜 운이 걷히고 좋은 운이 오면 대성하죠.
\vspace{5mm}

괜히 어설프게 선택 잘 하면 인생 트인다 그딴 망상 갖지 말고. 소신있게 결정하세요.
결과는 어떤 선택을 했느냐보다도 어떤 노력을 했느냐로 좌우되니까요.
\vspace{5mm}






\section{콕콕에서 연구할만한 주제들}
\href{https://www.kockoc.com/Apoc/626685}{2016.02.07}

\vspace{5mm}
\begin{itemize}
    \item[$-$] 어떤 볼펜이 수학풀이에 더 적합한가?
    \item[$-$] 풀이과정에서 연습장 공간을 어떻게 분할하여 나눠볼 것인가
    \item[$-$] 시험장에서 떠올릴 수 있는 과목 '두문자'는 어떻게 개발할 수 있나?
    \item[$-$] 카페인을 이용한다면 어떤 타이밍에 먹는 게 좋은가?
    \item[$-$] 적정 수면 시간과 타이밍은 어떤 타임이 좋나
    \item[$-$] 과도한 공부 이후 반드시 오게 되는 스트레스 관리는 어떻게 할 것인가
    \item[$-$] 공부할 때 들어도 무해한 음악은?
    \item[$-$] 주의 환기를 위해 봐도 좋은 사진이나 그림은?
    \item[$-$] 문제를 읽을 때 어떤 순서로 어떤 개요를 그려나가야하는가?
\end{itemize}
\vspace{5mm}

...
\vspace{5mm}

찾아보면 주제는 참 다양하다.
\vspace{5mm}

....
\vspace{5mm}

수험칼럼 따위가 아니라 사실 저런 걸 공동으로 연구하고 발표해보는 게 과학의 단계가 아닌가 싶은데
자뻑은 아니고 스스로도 학습일지를 읽거나 상담해보면 어느 정도 패턴을 파악해볼 수 있지만 이건 아직 '미신'에 불과하다.
왜냐면 맞기도 하고 안 맞기도 하니까.
\vspace{5mm}

그런데 로켓트에 들어갈 엔진이라거나 인공 심장에 들어갈 판막에 들어갈 '부품'의 정밀성.
그런 정밀성을 갖춘 학습공학이나 학습시스템을 정리해서 한 사람의 인생을 바꿀 수 있다면 이건 대단한 일이 아니겠나.
사실 사소한 필기습관이라거나 볼펜 종류만 가지고도 성적이 바뀌어 그걸로 인생이 갈라지는 경우가 많다.
\vspace{5mm}

특정 강사 강의나 인강만 들으면 잘 할 수 있어라는 토테미즘 부족사회 제정일치 그런 시대가 아니라
어떤 하드웨어나 소프트웨어로 효율성있는 시스템을 밟아 바뀌어나가느냐 하는 과학의 시대가 이미 왔어야하지 않나 싶기도 하지만
요즘 드는 생각이 인간이 만들어낸 최고의 자본은 황금도 화폐도 아닌 결국 \textbf{'지식' 밖에 없다}는 것이다.
그리고 그 지식도 뇌로 숙달된 것이 아니면 죽은 것에 불과하다.
\vspace{5mm}

그리고 이것이 만약 수능에만 집중된 것이면 나 역시 업자 장사치에 지나지 않는다는 얘기를 들을지 모르지만
사실 저건 사람이 태어나서 죽을 때까지의 전과정에 적용될 수 있다는 걸 요즘 느끼고 있다.
한 개인이 실천할 수 있는 이상적인 자가교육시스템이란 어떤 것일까.
\vspace{5mm}

일종의 독설적 칼럼이라면 나도 신나게 쓸 수 있고 심지어 주작할 수도 있다. 그리고 이게 얼마나 미신투성이인지는 안다.
그렇다고 학습법이라는 게 정해져 있느냐하면 그런 건 아니다. 미연구된 분야가 많다.
가령 A4 용지만 가지고도 학습에 어떻게 활용할지에 대한 참 방법이 많은데 국내에서는 이것조차 정리된 것도 없다.
콕콕에서 꾸준히 활동하면서 총회까지 들어와 글을 쓰는 분들이 이런 주제를 가지고 서로 연구하고 결과를 대조해보았으면 좋겠다.
\vspace{5mm}






\section{모 강사 자서전(?)을 읽고}
\href{https://www.kockoc.com/Apoc/626687}{2016.02.07}

\vspace{5mm}

유명하다는 강사 자서전(?)을 읽었다. 물론 거기서 뭔가 기대한 건 아니고 어떤 식으로 사람들을 휘어잡았나 보기 위해서이다.
\vspace{5mm}

꽤 괜찮은 요식업 만화로 국내번역명 '라면요리왕'과 '라면서유기'라고 있는데 참 통념을 깨는 만화다.
장사해먹기 위해서는 대중들의 싸구려 입맛에 맞춰야한다라는 무시무시한 진실이 통념없이 드러나있다.
진주인공 대머리 세리자와부터가 은어로 맛을 낸 진짜 라면을 내도 안 팔리자
에라 모르겠다라고 기름기 듬뿍인 라면을 냈는데 그게 잘 팔려서 대박난 케이스다.
\vspace{5mm}

모든 이를 만족시키는 최고의 요리를 내면 돈을 번다.... 그건 거짓말인 것이다.
마찬가지로 오늘 들으면 내일 죽어도 좋은 그런 강의가 정말 인기가 좋은가? 사실 그렇지 않다.
수억대 연봉 강의는 들어보고나면 내 개인적으로는 실망한 경우가 많았다. 혹시 내가 부족해서 그런가 했는데 그건 아니다.
그런 강의들이 잘 팔리는 건, 생각하는 게 \textbf{싸구려인 학생들이 원하던 싸구려 내용이기 때문}이다.
사실 무턱대고 강의듣는다는 친구들은 혼자 공부할 줄 모르는 사람들이다. 아니, 그보다 공부하는 것 자체를 싫어한다.
공부하는 것 자체를 싫어하는 사람들을 만족시키는 방법은 두가지이다.
하나는 강의 내내 유머를 적절히 넣고 쉬운 내용으로서 공부했다라는 포만감을 주는 것이고
다른 하나는 욕설이 섞인 카리스마 강의로서 '마조히즘적'인 것을 일깨워주는 것인데
\vspace{5mm}

저기서 성공한 건 후자다.
내가 읽은 자서전도 후자를 참 적절히 써먹은 케이스다.
그 내용을 읽고난 것은, 컴플렉스 덩어리였던 사람이 참 이런저런 경험을 하고나서 사람 다스리는 법을 알고나서
학생들을 어떻게 갈궈야 속으로 좋아하는지에 대한 '마조히즘적'인 진실을 일찍 간파했었구나라는 것이다.
(물론 오해사기 싫어서 말하면 나는 이런 걸 대단히 혐오한다. 일단 성격도 이상하다고 느끼는 데다가 변태같아서 그렇다)
\vspace{5mm}

강의기법이나 내용이야 뭐 별의 별 것은 없고.
\vspace{5mm}

그럼 왜 폭력적이거나 독설적인 강의가 인기가 좋을까.
\vspace{5mm}

\begin{itemize}
    \item \textbf{첫째, 그런 폭력적인 것을 경험하고 나야 비로소 교육받았다고 착각하는 구슬픈 유전자 때문이다.}
    \item \textbf{둘째, 사람들에게 있어서 교육이란 뭔가 '학대당하는' 것이다. 조선인들에게 민주주의적인 교육은 필요없다.}
    \item \textbf{셋째, 힘있고 폭력적인 메시지야말로 역설적으로 주입이 잘 되고, 이게 실제로 고득점에 도움이 되기 때문이다}
    \item \textbf{넷째, 강사의 카리스마를 돈을 주고 소비하는 건 터무니없는 가격인 걸 알면서도 해외 명품을 소비하는 것과 비슷한 만족감을 준다.}
\end{itemize}
\vspace{5mm}

그래서 저런 강사를 한번 듣고나면 xxx 들어라 하는 자발적인 전도사가 된다.
물론 이런 카리스마가 먹히는 건 길어도 한 5년?
특히 이런 강의들은 이상하게 현강이 더 강조되는 것 같은데 하기야 '폭력'은 인터넷보단 직접 마주보는 게 더 실감나지 않겠나.
\vspace{5mm}

그런데 문제는 저런 달콤한 폭력에 중독된 사람은 \textbf{끊임없이 저런 강의만 찾아다닌다는 것이다.}
왜냐? 카리스마있게 야 이 xx야 욕하면서 주입하는 그런 것에 길들여지고 나면, 민주주의적으로 차분하게 '생각한다'는 것을 잊어버리거든.
게다가 혼자 밍밍한 느낌으로 사고하는 것보다 카리스마 강사가 이렇다 저렇다하는 메시지가 더 선명하니까 그걸 다 흡수해야겠단 절박감이 오지.
\vspace{5mm}

슬프지만 이런 수법은 앞으로도 꽤 먹힐 것이다. 가르치는 걸로 일단 호구지책하려는 사람은 위와 같은 걸 잘 감안하시길.
그 당사자들이야 숨기고 싶은 노하우일지 모르지만 솔직히 노하우치곤 뭐 이런 싸구려가 다 있나 싶을 정도다.
이런 걸 부러워해서 '앞으로 강의하고싶다'고 하는 케이스는 타락하기 전에 정신차리거나 다른 버전으로 가는 걸 권하고 싶다.
솔직히 말이 카리스마지, 실제로는 시험에 절박해서 공포에 질린 사람들을 걍 협박하는 것과 큰 차이가 있나?
그리고 내가 아는 한 그런 식의 강의하다가 그만 둔 사람들, 다른 분야에서 일 제대로 못 한다. 그만큼 그 사람들도 비정상이 되었단 얘기거든.
\vspace{5mm}

나도 한 때 강의라는 걸 꽤 많이 들어보았지만 그런 강의가 도움이 되는 건
그런 강의 내용을 내 스스로 반박할 때가 아닌가 싶다.
혼자 책 찾아 읽어보면서 내가 들었던 강의 내용이 틀렸구나를 하나하나 찾고 논증하는 것.
그럼 그게 돈이 되어서? 아니, 그런 걸 하나하나 잡아내는 것 자체가 그냥 \textbf{'재밌어서'} 그렇다.
돈이 안 되는 학문이든 개발의 동기는, 과거에 몰랐거나 불가능했던 걸 내가 눈충혈되고 밤새면서 '알게 되거나', '가능하게 만드는 그것'인데
소위 독재자처럼 군림하던 사람들의 그런 내용이 실제로는 출처가 어디서 나왔는지 그리고 어떤 게 구라인지 간파하는 게 그냥 재밌는 것이다.
\vspace{5mm}









\section{라이벌}
\href{https://www.kockoc.com/Apoc/626811}{2016.02.07}

\vspace{5mm}

혼자서 20km를 뛰라고 하면 못 뛴다
그러나 하프마라톤 대회라면 뛸 수 있다. 다른 사람이 뛰는데 내가 멈출 수는 없기 때문이다.
에어로빅 등의 GX도 그러하다.  혼자서는 하기 힘들지만 같이 하니까 할 수 있는 것이다.
독학재수, 특히 집독학을 비추하는 이유도 그렇다.
혼자서 하면 사실 공부를 하다가 자의적으로 중단하더라도 아무도 못 막고 본인이 수치심을 못 느낀다.
똑같이 공부하는 '라이벌'이 있어야만 공부할 수 있는 것이다.
\vspace{5mm}

일지를 쓰고 남의 일지를 보라고 하는 이유가 그 때문이다. 다른 라이벌들이 \textbf{어떻게 공부했나 보고 자극받고 따라가는 게 효과가 좋다}.
이 일지조차도 콕콕에서는 참 오해하는 사람들이 많은 것 같은데 내 입장에서는 그게 오히려 더 이상해보였다.
공부에서 '경쟁'은 정말 필요악 중 필요악이다, 특히 가르치는 사람이 서로 비교질하는 게 그게 즐거워서가 아니다.
비교를 시켜서 마음에 상처받게 하고 열등감을 느껴야만 본인이 공부한다,
그런 말을 하면 "네가 안 해도 하잖아"라고 하는데 그런 애들은 다른 점에서 문제가 많다.
경쟁을 좋아한다고 다 공부를 잘 하는 건 아니지만, 공부를 잘 하는 애들은 \textbf{경쟁 자체를 즐기고 당연시 한다}.
친구 아들 딸을 이야기하는 엄마가 옳다는 건 아니다. 그런데 그 엄마들도 사실 경쟁하라고 부추기는 건 나쁘지는 않지
다만 부추기되 도움을 실제로 안 주니까 문제일 뿐이다.
\vspace{5mm}

서로 공부를 자기가 몇시간 했으니 어떤 교재를 몇회독햇느니 하는 걸 자랑하는 것이 갈등을 낳는다고 하더라도 이건 권장될 얘기다.
\textbf{저런 얘기를 들어서 빡친다고 하더라도 그런 빡침 자체가 본인에게는 매우 도움이 되는 것이기 때문이다.}
남이 10시간 정말 공부한 것을 보고 슬프거나 우울한 감정이 들더라도 그것 자체는 본인의 기준치를 높여주므로 결국은 좋다.
\vspace{5mm}

\textbf{최근에 왜 비교질을 하고 싸움붙이느냐라는 지적을 받은 적이 있어서 내가 한소리해야겠다.}
그런 걸 문제 삼으니까 당신들이 발전이 없는 거라고.
애들끼리 경쟁하고 참고서 뭐 보나 서로 산업스파이질하고 공부시간 속이고 하던 것?
그거 내가 중1 때 하던 짓이다. 다시 말해서 옛날부터 공부하던 사람들은 저런 건 너무 당연시했단 것이다.
심지어 공부 잘 하던 사람들끼리 서로 속사정 잘 알면서도 8시간 공부한 걸 2시간 공부했다고 농담하던 게 겸손으로 통하던 시절도 있다.
그리고 이건 현재도 똑같은 보편적 진리다.
공부 잘 하는 애들이 스트레스를 푸는 건 라이벌을 어떻게 이겨먹느냐 연구하고 노력하는 것이다.
프린세스 메이커 게임에 스트레스가 0이 되는 이벤트가 라이벌 등장인데 이건 정말 현실적인 것이다.
\vspace{5mm}

자기는 저런 비인간적인 경쟁이 싫다고 하면 \textbf{그냥 '평범한 대학' 가서 '평범하게' 사시면 된다.}
그런데 저런 것도 비판하기나 하고 실천도 못 하면서 남들보다 잘 살거야 그딴 드립은 치지 않았으면 좋겠다. 세상에 그런 게 어딨냐?
겉으로는 성인군자숙녀인 척하는 우등생 남녀들이 속으로는 얼마나 많이 연구하고 노력하고 남들 하는 것 벤치마킹하는지 모르나?
\vspace{5mm}

소위 공부 잘 한다는 친구들의 칼럼도 읽을 때는 주의해야하는 게 그거다
얘들이 말하는 많이 푸는 게 그냥 많이 푸는 게 아니다. 다른 애들보다 최소 1년 이상, 그리고 2배 이상은 가는 게 평범하게 푸는 것이다.
\textbf{자동차라고 하면 롤스로이스가 기본이고 간식이라고 하면 거위 간, 그리고 술이라고 하면 발렌타인 두자리 년수}
이걸 평범이라고 한다. 그래서 여기서 빈익빈부익부 가중되는 것이다.
\vspace{5mm}

일지 쓰시는 분들은 자기의 복제 캐릭터 아무개를 가정해보자.
일지를 하루 단위로 쓰던 일주일로 쓰건 그 아무개가 어디까지 공부했을 것이다라고 소설을 써보길 바란다.
그렇게 비교해보면 자기가 얼마나 태만하게 공부하는지가 느껴질 것이다.
\textbf{그리고 총회에서 상담하는 분들 중 가능성있다는 분들은 의도적으로라도 경쟁하고 비교시킬 것이다.}
물론 그게 싫으면 빠져도 되지만, 저건 그냥 놀이가 아니다. 저렇게 해야 사실 오래 버틸 수 있고 올라갈 수 있다.
\vspace{5mm}










\section{계획을 짜는 알고리즘을 간략히 적어보자.}
\href{https://www.kockoc.com/Apoc/628832}{2016.02.08}

\vspace{5mm}

계획을 짜는 건 자기 컴플렉스를 푸는 게 아니다.
계획은 "100$\%$ 실천가능한 공부단위"를 작성하는 것이다.
\vspace{5mm}

가령 쎈을 다 풀겠습니다... 이건 계획이 아니다, 그냥 목표다. 이걸로는 절대 실천을 할 수 없다.
반면 \textbf{"하루에 쎈을 30문제 풀기로 하겠습니다. 그럼 3달이면 100$\%$ 완료합니다, 30문제이니 지치지 않고 풀 수 있습니다"}
라고 해야 이게 계획인 것이다.
\vspace{5mm}

비유하면 좀 그렇지만 '나라는 가축에게 사료를 언제 얼마나 배급할 것인가'
이게 계획이다. 그런데 대부분은 계획을 짠답시고 비싼 사료만 사놓고 이걸 한번에 먹으려고 하다 배터져 죽는 상황을 반복하고 있다.
\vspace{5mm}

다시 말해 계획은 \textbf{단위 공부량을 줄이는 것}이다.
대신 양을 줄이되 \textbf{현실성을} 담보하고, 그럼으로써 \textbf{짧은 기간에 많은 양을 끝내는 것}이다.
다들 착각하는 게 하루 공부량을 많이 잡기만 하면 할 수 있다고 하는 데 그건 틀린 이야기다.
자기가 할 수 있는 능력 범위 내에서 공부하는 게 우선 전제되어야 한다.
\vspace{5mm}

그리고 반드시 계획은 '자본'을 늘려나가야한다.
예컨대 쎈, 마플, 급품벨을 한꺼번에 푸는 것과, 쎈 먼저 그 다음 마플, 그 다음 일등급, 일품, 라벨 순으로 풀어나간다고 하자.
총량은 별로 다를 것 없어보이지만 실제로는 다르다.
전자의 경우 학생은 정말 아무 자본도 없는 상태에서 쎈, 마플, 급품벨을 모두 상대해야한다.
\vspace{5mm}

하지만 후자의 경우 학생이 쎈을 끝내면, 그 쎈이 자기 아군이 되어서 마플 공략을 도와주고, 마플을 마치면 쎈과 마플이 급품벨 공략을 도와준다.
즉, 하나하나 완성해내가면서 자기의 자본을 늘려나가는 것이 전제되어야 한다.
'순서'대로 하나하나 공략해나가는 것은 한단계 한단계 아군(자본)을 늘려서 그 다음 적을 처리해나가는 가장 현명한 방법이다.
\vspace{5mm}

그렇다면 대략 1주일이나 2주일 정도 잡고 2과목의 교재 한권씩을 끝내나가면 된다.
예컨대 수특을 잡는다면 매일 하루 30문제 수학, 하루 영어 10지문은 기본으로 보면서 월수금은 국어, 화목토는 탐구.
이런 식으로 진행해나가면서 3월말까지 끝내는 걸로 기한을 잡는 것도 훌륭한 전략이다.
공부량이 적다고 할지 모르지만 그건 틀린 생각이다. 3월말까지 이것만 완수하더라도 4월초에 수특이 이미 내 '아군'이 된 상태다.
그만큼 실력도 늘고 부담이 준 상태에서 다른 교재들을 공략해나가면 되는 것이다.
\vspace{5mm}

아마 이런 걸 잘 모르는 사람들은 머리 탓을 할지 모른다
그러나 본질적으로는 '순서'대로 일처리를 하지 않은 것이 문제다.
공부든 일이든 순서대로 처리해나가면서 자기 자본을 조용히 늘려가 나중에 규모를 키우는 것이 정석이다.
그렇기 때문에 아무 실력도 없는데 처음부터 실력정석이나 어려운 실모를 푸는 건 바보같은 짓이다.
절망적일수록 가장 쉬운 교재를 정해서 그걸 확실히 내 걸로 만들고 차근차근 나가는 게 공부다.
\vspace{5mm}

그럼 이렇게 쉬운 계획을 누가 방해하는가?
그건 자신의 \textbf{과욕}과 \textbf{불안감}이다.
욕심을 부리는 건 좋은데 그걸로 자기 능력을 과대평가하면서 짧은 기간에 너무 많은 걸 '순서없이' 끝내려 한다.
게다가 불안감 때문에 이것저것 풀지 않으면 안 될 것 같다.
이 글을 보는 다수 n수생들이 사실 이런 코스로 세월을 날렸을 것이다.
\vspace{5mm}

명심하시길, 스피드와 물량도 중요하지만, 그 \textbf{모든 건 '순서'가 지켜져야만 의미가 있다는} 것을.
순서를 지켜나가면서 조금씩 10원, 20원 쌓다가 나중에 10,000원, 20,000원까지 모으다보면 그것이 기하급수적으로 불어나는 것이다.
남들이 어떻게 한다더라 신경쓰지말고, 인내심으로 버티면서 순서대로 자기 자본을 늘려나가야 한다.
그렇기 때문에 공부에서 기술보다 '마음'이 중요한 것이다. 기술은 시간과 노력을 단축시켜준다. 이건 \textbf{'인내심'}의 적이다.
하지만 너무 늦으면 안 되기 때문에 일단은 '빨리' 시작해야하는 것 뿐이다.
\vspace{5mm}


\section{슈퍼 마리오로 설명하는 입시}
\href{https://www.kockoc.com/Apoc/633676}{2016.02.13}

\vspace{5mm}

급식충(?)들은 모를 수도 있는 수퍼마리오 초판입니다.
보통 2시간은 걸리는 게임인데 4분 57초만에 피치공주를 구하는 동영상입니다.
현재 수능난이도에서 만점이 나와야하는 과목은 대체로 이렇게 문제를 풀어나가면 된다고 보면 됩니다.
\vspace{5mm}

저런 플레이가 가능한 건
\vspace{5mm}

\begin{itemize}
    \item [$-$] 수도 없이 연습했기 때문이다.
    \item [$-$] 적과 함정이 어디서 나오는지 암기했기 때문이다.
\end{itemize}
\vspace{5mm}

그럼 실제 시험은 어떤 식으로 나오나
\vspace{5mm}

+ 초고수 플레이
\vspace{5mm}

잘 하는 친구들은 저 정도는 한다가 보면 되겠습니다.
그럼 저게 어떻게 가능할 것인가.... 당연히 연습과 암기죠.
물론 처음보는 게임이라고 할지라도 '보편적인 게임 진행'에 숙달된 사람이라면야.
\vspace{5mm}

강의에만 의존하는 건 프로게이머 방송만 보고 게임을 잘 할 수 있다라고 믿는 것과 같습니다.
\vspace{5mm}





\section{공부를 하면서 스트레스를 받아야 하는 이유}
\href{https://www.kockoc.com/Apoc/636254}{2016.02.15}

\vspace{5mm}

믿거나말거나
고교시절에 의무적으로 과학 보고서를 써서 내야하는 게 있어서 햄스터 실험을 한 적이 있습니다.
스트레스를 주었을 때 햄스터의 지능에 어떤 영향이 있을 것인가... 라는 것인데 참고서는 전파과학사에서 나온 스트레스 어쩌구.
일단 가설은 스트레스가 지능을 떨어뜨린다... 였는데
\vspace{5mm}

엉성한 실험이었다고 하지만 오히려 실험결과는 스트레스를 가한 햄스터가 더 똘똘했단 것입니다.
이거 실험을 엉터리로 했나 했지만 기한은 다가와서 그냥 그렇게 냈는데
이게 생물선생님에게 우수하다라는 평가를(... 아니 다른 녀석들은 어떻게 쓴 거야 도대체)
\vspace{5mm}

...
\vspace{5mm}

일반적으로 공부할 때 스트레스를 받아야하는 이유는 그래야 머리가 '좋아'지기 때문입니다.
한계를 넘어서는 자극을 받으면 기존의 인지구조가 변형되고 뇌에서는 그런 충격을 줄이기 위한 변화를 모색하는 경향이 있습니다.
하루에 30문제가 한계량인데 50문제를 풀게하면 스트레스를 받고 고통을 느낍니다.
이 경우 관찰되는 방향은 세 가지입니다.
첫째는 문풀을 줄이는 방향으로 가는 것 $-$ 즉 공부를 회피하거나 공부하지 않기 위해서 핑계거리를 만드는 경우
둘째는 문제를 푸는 인지구조가 바뀌어버리는 것 $-$ 즉 머리가 좋아지는 경우
셋째는 그 고통 자체를 즐기는 것 $-$ 즉 공부변태가 되는 경우
\vspace{5mm}

대체로 둘째와 셋째가 바람직(?)하다고 할 수 있습니다. 사실 셋째를 권장하는 이유는 '천재'를 이기는 건 변태 빼고는 없으니까요.
하지만 대부분은 첫째로 갑니다.
이유없이 짜증내거나 화를 내는 경우 $-$ 물론 당사자는 그럴 싸한 이유를 만듭니다 $-$ 이력을 분석해보면 '공부 스트레스'를 받은 케이스죠
\vspace{5mm}

어떻게 보면 여기서 범재와 수재 갈리는지도 모르지요.
\vspace{5mm}

망치로 사정없이 두들겨서 변성시켜야하는데 대상이 고정되어있지 않으면 멀리 튕겨나가버리겠죠.
그리고 거기서 학습한 효과로 망치를 피해나갈 것입니다.
하지만 제대로 묶여있다면 망치로 얻어맞으면서 아주 단단해지고 치밀해지겠죠.
감금, 수감되어있다면 지적자극에 얻어맞아야되고 그렇게 하면 뇌가 바뀔 수 밖에 없습니다.
\vspace{5mm}

독학으로 하면 공부가 잘 된다고 하는 경우는 이런 본질을 모르는 경우죠.
사실은 하기싫은 공부야말로 진짜 공부입니다. 그런 공부를 해서 엄청 스트레스를 받으면서 변성 단계에 이른 다음에,
적절한 시기에 다른 공부를 하거나 휴식을 취하면 그동안 뇌가 재성형되고 그 다음에 다시 공부하면 이렇게 쉬웠냐하는 느낌을 받죠.
\vspace{5mm}

.....
\vspace{5mm}

실패하는 이유는 별 게 아닙니다. 하기 싫은 공부를 안 해서입니다.
하고싶은 공부를 해놓고 공부량이 많다고 해보았자 소용이 없습니다. 그걸로는 뇌가 안 바뀝니다요.
\vspace{5mm}








\section{슈퍼 마리오로 설명하는 고수}
\href{https://www.kockoc.com/Apoc/636264}{2016.02.15}

\vspace{5mm}

포도님이 링크시킨 영상의 주인공이 더 막 나가는 플레이를 소개
\vspace{5mm}

$\#$ 개조(Kaizo) 마리오 3 스피드런
\vspace{5mm}

절대 지루하지 않습니다, 보는 도중에 여러번 감탄사를.
물론 여러번 죽기도 하셨지만, 컨트롤과 공략이 파이브 스타 스토리즈의 파티마급이 되시네용
대충 국어, 영어, 수학의 고수라고 하면 저런 느낌이라고 보시면 됩니다. .
\vspace{5mm}

$\#$ 원판 마리오 공략 스피드런
\vspace{5mm}

스테이지를 뛰어넘는 마술피리 안 쓰고 그냥 공략한 건데 이걸 1시간 내에.
그것도 그렇지만 마지막 5분 영상이 압권입니다. 저런 식으로 클리어가 가능하구나
아예 닌텐도 코드를 다 꿰뚫고 있었네.
\vspace{5mm}

이제 공부의 패러다임은 다름아닌 '게임'이죠.
남에게 자기가 문제푸는 걸 어떻게 뽐낼 수 있느냐 그 자체로 동기부여하는 것도 매우 중요하다는 것.
수학 양치기를 한계량까지 한 사람은
평범한 학생이 30분 끙끙거리는 걸 3$\sim$5분 내에 예술적으로 풀이하죠.
그게 처음보는 문제일지라도 $-$ 그리고 그 맛에 공부합니다.
\vspace{5mm}






\section{통과의례}
\href{https://www.kockoc.com/Apoc/636843}{2016.02.15}

\vspace{5mm}

외모지상주의를 조장하는 건 아니지만 그냥 비현실적인 예를 들겠습니다.
A란 사람은 못 생겼는데 열심히 노력해서 돈을 벌었습니다. 그리고 그 돈으로 성형을 합니다.
B란 사람도 못 생겼는데 돈이 없어서 혼자 운동하고 얼굴요가(...)를 해서 저절로 미남/미녀가 됩니다.
\vspace{5mm}

외모지수는 똑같습니다. 그런데 이런 경우 사람들이 선택할 건 B일 겁니다.
'돈'을 주고 타인의 행위로 완성시킨 걸 A의 것으로 보지 않기 때문입니다.
하지만 B의 경우도 어떻게 보면 자가 성형인데 저런 경우는 자연미로 인정받습니다.
\vspace{5mm}

그럼 공부도 마찬가지겠죠
만약 C와 D라는 공부 못 하는 사람이 있다고 칩시다.
C는 거액의 돈을 들여서 오버테크놀러지 기계로 뇌에 데이터를 전송받아 똑똑해집니다.
D는 거액의 돈으로 사교육을 받아 노가다 수험을 한 다음에 똑똑해집니다.
둘 다 거액의 돈을 들였다고 하더라도 누굴 선호하겠습니까. 대답할 필요는 없겠죠.
이게 참 신기한 것입니다. 우리가 사람을 평가하는 기준이라는 것은 무의식에 있을 것 같은데 나름대로 논리가 서있단 것이죠.
원래 배우자를 선호할 때 받는 유전자의 암묵적 명령이라는 게 여기에 있을지도 모릅니다.
\vspace{5mm}

수험에서 배우는 게 쓸데없다고 하더라도, 수험 자체가 유의미한 이유가 여기에 있을 것입니다.
주인공이 정말 진정한 주인공이 되기 위해서는 '고난'과 '역경'을 거쳐야하기 때문이라는 것은 '신화 코드'의 하나죠.
그런 통과의례를 거친 사람이야말로 종족번식에 유리하다는 유전자의 가르침일지도 모릅니다.
수험에서 배우는 건게 쓸데없더라도, 그것이 수험생 본인을 '단련'시켜준다는 건 바뀌는 게 없는 것입니다.
하다 못해 운이 나빠서 결과가 안 좋다고 하더라도, 그렇게 단련된 뇌가 어디 가는 게 아니죠.
\vspace{5mm}

모든 게 유전자 덕이다하는 사람들을 보면 세계사 공부도 안 했는가 싶죠.
황족, 왕족, 귀족, 하다 못해 천재의 가계도 같은 걸 추적해보면 그거 3대 이상 가는 경우는 별로 없습니다.
총회 게시판에서 언급된 합스부르크 왕가의 경우도 끼리끼리 혼인으로 금수저 극대화 전략 폈다가 유전병으로 말아먹었죠.
영국 왕실이나 일본 천황가도 상징적 존재가 아니면 이미 축출당했을 것입니다.
\vspace{5mm}

반면 흔히 언급되는 록펠러나 로스차일드, 발렌베리 일가의 경우는 '교육'을 \textbf{후덜덜하게 시킵니다.}
아니 무엇보다도 유대인이나 화교만 보아도 공통점이 있죠. 혈통은 사실 별 것 없는데 교육이 장난이 아니라는 것.
유대인들은 분파도 다양하지만 현재의 유대인들은 구약성서에 나오는 그 유대인과는 정말 거리가 멉니다만
그들은 유대교 전통에 따른 탈무드라는 교육자본을 갖고 철저히 교육에 매진해왔으며
화교들 역시 이런 점에서는 마찬가지였습니다.
\vspace{5mm}

요즘 수험가에서는 의치한 가면 월 얼마 번다 이런 걸로 무조건 거기 가야한다.
라고 하는데 개인적인 생각은 저런 바람은 이뤄지지 않을 거라고 여깁니다.
\vspace{5mm}

흔히 하는 이야기가 수요와 공급이라 하는데 이건 자본주의 경제를 절반만 언급한 것입니다.
만약 특정 직종이 공급이 통제되어서 그만큼 차익을 누린다고 하는 걸 시장경제는 가만히 냅두지 않습니다.
기술이 발달하건 여론으로 제도가 바뀌든 해서 그걸 반드시 날려버립니다. 그런 게 유지되면 사실 자본주의 사회가 아니죠.
게다가 정작 꿀빤다는 의료직종에 종사하는 분들도 그 분들이 대학교 갈 때 돈보고 간 것은 거리가 있을 것입니다.
(정작 그 때 돈보고 대학간 건 공대나 경영대가 아니던가요)
\vspace{5mm}

오히려 지금은 남들이 외면하지만 미래에 성장가치가 있는 전공에 가는 게 나을지도 모릅니다.
똑똑한 놈들이 몽땅 의치한에만 매진한다면, 타 전공 분야를 10년 이상 바라보고 자기가 거기서 일인자가 되는 전략이 나은 것이죠.
그럼 그런 일인자가 되기 위해 필요한 건?
\vspace{5mm}

수험이라는 고통스러운 의례를 자력으로 통과하는 것입니다.
\vspace{5mm}

제가 싫어하는 사람이라면 돈보고 가라고 얘기하겠지만
제가 생각하는 사람이라면 너무 돈을 보지 말라고 이야기하겠습니다.
요즘 느끼는 것이지만 거액의 재산이든 돈에 대한 탐욕이 사람의 눈을 멀게 한다는 것을 느끼고 있기 때문입니다.
검소하게 살아야하는 이유는 저축을 위해서 혹은 청빈이 아름다워서가 아닙니다.
가난하게 살아야만 더 많이 보이기 때문입니다.
\vspace{5mm}









\section{선택을 못 하는 이유}
\href{https://www.kockoc.com/Apoc/636857}{2016.02.15}

\vspace{5mm}

흔한 고민이 반수냐 아니면 그냥 재수냐 그건데
이럴 때에는 본인이 합리적이라는 생각을 버려야하며
아울러 선택안은 다시 정리하면
\vspace{5mm}
\begin{enumerate}
    \item \textbf{$-$ 대학교도 대충 다니면서 운좋게 반수}
    \item \textbf{$-$ 여전히 고졸, 하지만 남는 시간으로 딴짓하면서 재수하기}
    \item \textbf{$-$ 이것도 저것도 못 하고 그냥 끌려가기}
\end{enumerate}
\vspace{5mm}

이렇게 해야 정답일 겁니다.
\vspace{5mm}

체크리스는 ⓐ 할 수 있는 것, ⓑ 해야하는 것, ⓒ 하고싶은 것 인데
여기서 할 수 있다는 건 과정이 아니라 '결과'입니다.
가령 학교 다니면서도 수능 칠 수 있다고 하는 건 애매합니다.
전과목을 B를 맞을 것인가 A를 맞을 것인가 분명히 얘기하고 과반이 A가 나올 수 있다라고 하는 등의 기준을 짜고 결정해야죠.
그게 아니면 이것도 저것도 못 합니다.
그런데 재수를 하자니 본인이 롤을 한 경력이 있고 하루 공부시간이 5시간 넘은 경우가 없으며 이걸 하면 고졸이다라고 하면 골치
\vspace{5mm}

\begin{itemize}
    \item 반수 : 할 수 있는 것도 꽝, 해야하는 것 애매, 하고싶은 것 꽝이면 이 경우는 그냥 재수로 빨리 돌려야겠죠
    \item 재수 : 할 수 있는 것 애매모호, 해야하는 것 꽝, 하고싶은 것 오케이.
\end{itemize}
\vspace{5mm}

이렇게 나열해서 비교한 뒤 과감히 빨리 선택하는 게 나음
\vspace{5mm}

이건 거꾸로 말해서 뭘 '버릴지' 확실히 결정하라는 이야기임. \textbf{버리지 않으면 얻지도 못 합니다.}
다들 얻는 것에만 환장해서 버리는 것을 모릅니다, 버리지 못 하니 얻지도 못 하고 시간만 버리는 것이죠.
그런데 N수생들 보면 실패한 이유가 별 게 아닙니다.
\textbf{자존심이 강해서 버리지 못 하고, 버리지 못 하니까 얻지도 못 하고, 그렇게 시간은 날라가고.}
\vspace{5mm}

쓴소리하자면 벌써 2월 끝나가죠. 황금의 3개월 공부하는 사람은 이미 공부했죠.
아마 공부한 사람도 느낄 겁니다. 3개월도 금방 지나가는데 이거 공부할 것 더 널렸네.
하지만 공부 안 한 사람은 앞으로의 기간이 참 길다고 착각들 하죠.
\vspace{5mm}

다시 강조합니다만 버리세요. 그리고 그 놈의 자존심 제발.
이게 가장 큰 적입니다.
어그로끌자면 현역으로 유수 대학 못 갔으면 그 사람이 자존심 챙길 자격이라도 있습니까.
바로 부족한 것 인정하고 백의종군하고 노예생활해야 인간이 되는 거지.
그런데 그 놈의 자존심 때문에 그 자존심을 채우는 '낭비적인' 방향으로 움직이니까 시간만 더 갉아먹는 거죠
그리고 장사치들 배나 신나게 불려주고 말입니다.
\vspace{5mm}

일지쓰라는 게 일단 공부하는 사람들이나 질문하라는 것도 있지만
지금 고백하는 또 하나도 있죠.
\textbf{일지를 꾸준히 쓰면 작성자가 자존심이 저절로 무너집니다,}
\textbf{자기가 공부를 생각보다 게을리하는구나, 엉터리로 하는구나를 확인하니까요.}
\vspace{5mm}






\section{수험에 대해서 착각하는 것.}
\href{https://www.kockoc.com/Apoc/638785}{2016.02.17}

\vspace{5mm}

\textbf{공부 + 경쟁 = 수험} 입니다.
공부를 잘 하는 것만으로는 불충분합니다. 결국 경쟁자들을 내리찍고 자기가 올라가야 끝나는 거예요
내가 봐서 좋은 건 남이 봐도 좋습니다. 그렇다면 그 남을 물리치지 않고서는 성공할 수 없어요.
그래서 $\sim$ 해도 되나요... 라는 질문을 보면 이건 십중팔구 실패하겠구나 보는 겁니다.
다들 으르렁거리면서 앞서나가려고 미친 듯이 공부해도 실패해서 3$\sim$4년 공부하는 경우도 있는데
보통 공부를 안 하시던 분들이 이제 공부를 하면 자기가 드라마 주인공이라도 되는 줄 안다는 것이죠.
\vspace{5mm}

씁쓸한 이야기입니다만
그렇다고 요즘 서민이나 하류층이 헝그리 정신이라도 있는가하면 그것도 아닙니다.
한명씩 잡아 추궁해보면 충분히 공부할 수 있는데 안 하고 본인이 게으름 피우거나 게임이나 환락에 빠진 경우가 대부분입니다.
\textbf{헝그리 정신은 정작 부모 스펙도 괜찮은 부잣집 자제들이 갖고 있다는 게 더 절망적인 사실}이죠.
심심하시면 정말 명문대 진학 성공한 사람들 표본을 모아서 가정환경 확인들해보시길요.
집안이 하류이면서 자기가 서민인데도 불구하고 영화나 드라마처럼 뭔가 반전이 있을 거라는 근거없는 믿음에 빠진 케이스가 많습니다.
어느 수험사이트 가던 성공한 케이스들 가정환경 분석해보시죠.
가정환경이 안 좋은 경우는 정말 본인이 악바리 헝그리 정신을 발휘한 케이스입니다.
\vspace{5mm}

그리고 냉정히 말하면 올해 시험이라면 이제는 사실 전 일지 써온 사람들 빼고는 조언할 필요는 없다 여깁니다.
황금의 3개월은 이제 다 지나갔습니다. 그리고 여기서 이미 절반 이상은 \textbf{'결판'났다}고 보고 있습니다.
과장하는 게 아니라 실제로 이 3개월동안 공부한 것이 복리효과가 붙어서 정말 막판을 결정합니다.
11월부터 2월까지 금방 지나가죠? 이제 5월까지도 순식간이지요.
사실 11월부터 2월까지 안 한 사람이 3월$\sim$5월에도 할 가능성은 낮죠.
자기들은 할 수 있다 착각합니다만요.
\vspace{5mm}

특히 웹검색하면서 꿀교재 꿀강의 찾으면 된다..
헛짓하지 말고 문제집이라도 꾸준히 다 푸십시오. 그거 공부하기 싫어서 결국 책, 강의 수집한다하는 뇌의 발작 그 이상 그 이하도 아니니까요.
황금의 3개월 기간동안 공부 안 햇다, 그런데 게임을 잡았다 하는 남자분이면 그냥 병역 빨리 처리하라고 얘기하고 싶고
그게 아니라고 하면 반드시 집독학하지말고 도서관에 가든 하다 못해 스파르타 학원에 가서 개고생하는 걸 권하겠습니다.
그것 외에는 방법이 없을 겁니다.
\vspace{5mm}

그리고 올해 아니면 내년이다.... 아마 내후년 혹은 내내후년이 될지도 모릅니다.
\vspace{5mm}






\section{공부는 자기 좋으라 하는 겁니다.}
\href{https://www.kockoc.com/Apoc/639054}{2016.02.17}

\vspace{5mm}

"돈은 잃어도 머릿 속에 든 것은 잃지 않는다" 기억상실 크리는 어쩌고요.
도 그렇지만 기본적으로 \textbf{공부는 나 좋으라고 하는 것}이지 남 좋으라 하는 것이 아닙니다.
\vspace{5mm}

\textbf{공부 힘들어죽겠다 미치겠다 노오력은 무슨 노오력이냐 하는데}
솔직히 부모든 누구든 공부 강요할 이유가 없습니다. 부모자식도 남남이거든요.
\vspace{5mm}

최근에 와서 어떤 미친 놈들인지 몰라도 의대 돈 많이 번다라는 선동을 하고 있더군요.
그래서인가 개나소나 의대간다 어쩐다라는 식으로 병신들 같이 선동당하는 사람이 늘지 않았나 싶은데
그런데 말입니다, 공부 안 하는 사람이 의대 가고싶다고 하는 건 '걍 날로 돈벌고 싶다'라는 저속한 욕망 그 이상 그 이하도 아닌데
공대도 마찬가지이지만 공부 안 한 사람이 운좋게 의대간다 칩시다, 재앙이죠. 그 사람 손에 \textbf{한두명 목숨 잃을 게 아니니까요}.
돈만 바라보는 것은 미개한 '단식부기'적인 사고입니다.
현대사회의 틀은 복식부기죠. 차변$-$대변, 의무$-$권리, 임차$-$임대, 거액의 돈$-$무거운 책임
\vspace{5mm}

이렇게 보면 공짜라는 건 사실 없습니다.
\vspace{5mm}

그러나 공짜가 없는데도 보면 사람들 사이에 격차라는 게 생기죠.
타고난 선천적 능력 때문이기도 하지만, 결국 누가 시간을 덜 '낭비'했느냐입니다.
만약 게임을 했다고 칩시다, 순간적으로야 즐겁겠죠. 그런데 그 게임을 한 즐거움은 절대 '적분'되지 않습니다.
공부는 매일매일 해도 그것이 효과없는 것 같지만 조금씩조금씩 쌓여서 복리효과를 발생시킵니다.
그러나 게임이나 환락은 할 때에는 매우 즐겁지요. 그리고 조금씩 인생을 갉아먹기 시작합니다. 그것의 효용? 사실 없어요.
없는 걸 떠나서 그런 걸 즐기는 평범한 사람들을 '병신'으로 만들지요.
\vspace{5mm}

부모님이 무능하건 유능하건 이런 건 압니다. 왜냐면 자기들의 성공, 실패가 저렇게 좌우되었다는 것을 알기 때문이죠.
그래서 최소한 한번 이상은 자녀들에게 공부하라고 독려합니다.
그러나 어느 순간부터는 독려를 안 하는데 물론 부모가 소라넷이나 가고 등산불륜이나 하는 개쓰레기인 경우도 있지만,
'내 자식은 공부할 녀석이 아니다'라는 확신을 받았기 때문입니다.
그럼 이 확신을 바로 잡으려면? 그거야 자기가 미친 듯이 변태적으로 공부한다는 걸 '실천'해서 보여줘야죠.
그럼에도 불구하고 다수의 젊은이들은 '오랄 스터디'만 강합니다. 입으로만 $\sim$ 하겠다, 그러니 돈 내놔.... 이런다는 것이죠.
\vspace{5mm}

부모님이 자기 공부 도움 안 준다라고 하는 사람은 가슴에 손 얹고
정말 한번이라도 게임하지 않았나, 게으름피우는 모습을 안 보여주었나.. 등을 돌이켜보길 바랍니다.
정말 공부를 제대로 하는 사람은 눈빛이 살벌하거나, 변태적이거나 그렇기 때문에 꽤 강력한 기를 발산합니다.
하다 못해 강도조차도 '절도'로 끝내야겠구나라는 느낌을 줄 정도죠. 공부에 빠진 사람은 그냥 거기에 미쳐있거든요
\vspace{5mm}

그래서 상담 요구하는 사람에게는 제가 냉담한 겁니다. 공부에 미쳐있지도 않으면 그냥 그걸로도 과반은 실패한 건데 뭐 어쩌란 거야.
그리고 공부하기 싫어? \textbf{자기 좋으라고 하는 걸? 그럼 안 하면 될 것 아냐. 나이가 몇인데 어리광부려?}
\vspace{5mm}







\section{재종학원이 나은 이유}
\href{https://www.kockoc.com/Apoc/640034}{2016.02.17}

\vspace{5mm}

재종학원이 만능은 아닙니다만 적어도 인강보단 나은 이유는
바로 \textbf{호손 이펙트}로 설명됩니다.
\vspace{5mm}

\href{https://ko.wikipedia.org/wiki/%ED%98%B8%EC%86%90_%ED%9A%A8%EA%B3%BC}{링크}
\vspace{5mm}

간단히 말해서 \textbf{사람은 "타인의 관찰"을 의식하는 순간 행동이 달라진다}는 것입니다.
\vspace{5mm}
\begin{itemize}
    \item [$-$] 돼지우리에서 살던 여자가 남친이 라면먹고 가고싶다고 하니까 방을 치운다
    \item [$-$] 사단장님이 방문하신다니까 부대 전체가 깨끗해졌다
    \item [$-$] 혼자서 못 뛰던 장거리 코스를 마라톤 대회에 참여하니까 완주할 수 있었다.
\end{itemize}
\vspace{5mm}

만약 강의 내용을 받아적고 정리한다면 그건 인강을 못 따라갑니다.
그러나 성과는 '실강'이 더 좋은 이유는 저걸로 설명됩니다.
인간은 \textbf{타인과의 '관계' 속에서 긴장하고 집중}하니까요. 혼자 냅두면 대단히 태만해집니다.
\vspace{5mm}

생활습관 안 잡히는 사람들은 백날 계획 세울 필요 없이,
그냥 조직 속에 들어가거나 타인과 경쟁하는 모드로 가는 게 직빵입니다.
그리고 더불어 '집단 속'에 있다는 것 때문에 안심이 됩니다. 혼자 죽지는 않거든요
\vspace{5mm}

그러나 \textbf{반전}
\vspace{5mm}

다만 들어간 이상 거기서는 반드시 선두를 유지해야 합니다.
처음에야 좋다고 하지만 3, 4월 지나면 또 다시 거기서 태만해지죠. 파레토의 법칙이 어김없이 작용
결국 공부 안 하는 80에 속해서 집단으로 태만해지는 일이 벌어질 수 있습니다.
그 내부에서도 20 안에 들지 않으면 소용이 없는 거죠.
이른바 80의 함정에 빠지면 돈은 돈대로 내면서 작살날 수도 있습니다.
조직도 결국 평안함이 계속되면 "혼자 죽지는 않는다라는" 느낌이 오히려 독으로 작용하는 것이죠.
\vspace{5mm}

비싼 돈 들여 갔다면 라이벌 정한 다음에 그 라이벌을 성적으로 작살내야 영화보러간다거나 술마신다거나(...) 하는 식으로 잡으시길 바랍니다.
그냥 재종 갔으면 따라가면 되지.... 라고 하다가 관료주의 함정에 어김없이 빠져버리니까요.
\vspace{5mm}








\section{수험은 중국무술이 아닙니다.}
\href{https://www.kockoc.com/Apoc/640886}{2016.02.18}

\vspace{5mm}

무술 중에서 '기'를 발산하고 '내공'을 발휘한다.... 는 것은 뻥이죠
실제로 중국무술의 경우는 기껏 해보았자 그 기원이 명말청초 때입니다. 그것도 대부분 명맥 끊김.
기를 모은다거나 장풍 쏜다거나 하는 것은 무협지나 일본만화에서 나온 것이지 실제로 검증된 게 없어요.
\vspace{5mm}

그런데 특히 그런 무협지든 일본만화는 두 가지가 있죠.
첫째, 주인공이 알고보니 혈통이 금족보였다
둘째, 주인공이 개고생하다가 기연을 만나 내공과 필살기를 전수받는다.
여담이지만 이걸 깨부순 게 헌터헌터인데 연재가 참 불안정해서리(특히 그 핵무기 장면은 소름이 쫘악)
\vspace{5mm}

그런데 공부 안 하다가 꼭 어떻게 공부해요하는 사람들이 저런 중국무술적인 발상에 빠져 있는 것 같습니다.
공부 잘 하는 사람들이 보는 특수한 책을 자기가 보면 바로 깨달을 수 있다.... 빨리 갈 수 있을 것이다라는 것.
\vspace{5mm}

그런데 그딴 건 단언코 말해서 없어요.
일부 저자들이 자기들 책이 굉장하다 어쩐다 해서 비싼 값에 팔아치우는데 까고 말해 '무안단물' 수준입니다.
수학으로 치면 가장 좋은 책은 제가 보기에는 쎈수학입니다. 그리고 쎈수학을 권했습니다,
다들 아폭이 무슨 신사고 알바냐 어쩌구 하는데 더 놀라운 사실은, 쎈수학 C 스텝까지 다 끝낸 경우 생각보다 없어요.
국어요? 시중교재 다 고만고만합니다. 가장 좋은 건 본인이 깊이있고 어려운 책을 꼼꼼히 읽어왔냐는 것입니다.
영어요? 강의와는 무관해보입니다. 이것 역시 어린 시절부터 원서 읽고 미드 듣고 회화 꾸준히 해온 애들이 잘 합니다.
재종학원 찬양? 본인이 얼마나 좋은 환경에서 특혜받고 자랐는지 모르는 경우입니다.
\vspace{5mm}

저도 많은 표본을 보앗다고 단언은 못 하겠지만 정리되는 진리는
\vspace{5mm}

\begin{itemize}
    \item \textbf{$-$ 양치기를 꾸준히 한다고 성공한다고 할 수는 없다, 그러나 성공한 사람 중에 양치기 안 한 녀석은 없다.}
    \item \textbf{$-$ 머리가 좋은지 안 좋은지는 모른다, 그러나 두뇌회전이 빠른 경우는 다 교육환경이 대단히 좋은 케이스였다.}
    \item \textbf{$-$ 어려운 문제 잘 푼다고 자랑하거나 소위 수험경향 따지는 녀석들이 시험 당일 쉬운 문제에 털리고 색다른 문제에 통수맞는다}
    \item \textbf{$-$ 수험서나 강의 명품 따지는 놈들은 최소 3년 내내 그러고 있다.}
    \item \textbf{$-$ 성적은 회독수에 비례한다. 1번만 강의 듣거나, 책을 읽거나, 문제를 풀어보았자 소용이 없다, 10회독은 해야한다.}
    \item \textbf{$-$ 수능에서는 출제 난이도 따지는 건 무의미하다. 무조건 난이도 높다고 생각하고 대비해야한다.}
\end{itemize}
저기 어디 지름길이 있는지 제가 궁금합니다. 그런 걸 알면 \textbf{진짜 부자가 될 수 있거든요}.   종합격투기 나가서 목숨을 건 싸움을 해야하는데 무슨 $\sim$ 경을 외우고 기를 모으면 이길 수 있어...
이런 것도 아니고 말입니다요.
\vspace{5mm}

나중에 과학기술이 발달해서 뇌를 이식 혹은 복제할 수 있다거나
사람끼리 서로 USB로 연결해서 기억을 몽땅 전송해서 신경망도 복제할 수 있다면 모를까.
그게 아닌 이상 나머지는 다 \textbf{'거짓말'}입니다.
\vspace{5mm}

중국무술이 망한 이유 중 하나가 문화대혁명이라죠.
문화대혁명 때 홍위병들이 다구리치니까 고수도 별 것 없었다고(...)
아니, 그 전에 무술이 대단하다면 다구리도 맞서 싸울 수 있어야하잖아.
\vspace{5mm}

현역 때 잘 하는 애들이 있죠. 그런데 그 친구들은 어린 시절부터 공부를 꾸준히 해온 케이스입니다.
선행을 미리 한 애들은 딴 친구들이 3$\sim$4수할 것을 미리 3$\sim$4년 앞당겨 공부함으로써 현역으로 가는 것이고
선행을 하지 않은 친구들일지라도 초, 중학교 때 공부하는 습관이나 틀이 잘 잡혀있어 한번 보면 바로 익힙니다.
\vspace{5mm}

적어도 수험에 대해서는 분명 과학적으로 정리되는 것들이 이미 있는데
공부를 하지 않은 분들이 여전히 '무속적'으로 접근하는 것을 보면 참 안타깝습니다.
무속이 과학으로 바뀌기 힘드니 결국 '종교'로 진화하는 것이고 그러니까 특정교재만 보면 된다, 특정강의만 들으면 된다로 바뀌죠.
\vspace{5mm}

적어도 교재 질문은 최소한 국영수탐 남들이 많이 푸는 걸 3회독은 한 다음에 했으면 좋겠음요.
질문하는 것 답변해주기 싫은 이유는 다른 걸 떠나서 본인들이 공부를 안 하는 케이스여서입니다. 안 할 건데 왜 질문을 하지?
실천하고 질문해주셨으면 좋겠어요.
대답이야 뭘 해보았자 남들 많이 푸는 교재 그냥 달달달 외우고 반복하고 하는 걸로 끝납니다.
솔직히 말해서 쎈수학만 가지고도 10번 이상 돌리고 부족한 내용 검색질을 하든 타 수학교재로 보충하든지 질문하든지 채워도
그걸로도 수리영역 150$\%$는 대비할 수 있습니다.
\vspace{5mm}

+
복싱, 씨름, 스모가 차라리 현실적이죠.
특히 스모는 우스꽝스러운 뚱보들 때문에 겉이미지는 그래보입니다만
그게 살찌우는 게 오히려 현실적. 뭘로 가든 체중 키우는 것 못 따라간다....
사실 테크닉은 '물질'의 부족을 메우기 위한 것입니다. 물질이 풍부하다면 테크닉도 크게 필요가 없죠.
지금 이 세계를 지배하는 건 동양의 신비인가요? 아니면 미국의 물량주의인가요?
\vspace{5mm}

++
수험서 보아도 참 수요와 공급의 원리.
아마 쎈수학 같은 책이 소수만 갖고 있고 복제가 불가능했다면 한권당 가격은 꽤 어마어마했을 것입니다.
실제로도 1권당 1천만원의 가치는 있습니다. 그런데 이게 대량복제되어서 판매되니까 사소해보이는 것이죠.
하지만 무엇보다 수험서는 자기가 공부하지 않는 이상은 폐품입니다.
\vspace{5mm}






\section{아주 흔한 수험패망의 루트}
\href{https://www.kockoc.com/Apoc/640984}{2016.02.18}

\vspace{5mm}

A라는 책은 내용이 풍부하고 크게 손색이 없음,  그런데 이 책은 어디서든 쉽게 구할 수 있고 무엇보다 저렴합니다.
B라는 책은 사실 내용이 틀린 게 많고 지나치게 치우쳐져 있습니다, 하지만 신비주의적으로 광고되고 잘 알려져 있지 않습니다.
\vspace{5mm}

어찌되었든 붙을 학생이라면 A나 B나 모두 보겠죠. 그런 걸 가릴 시간이 있으면 걍 다 보는 게 맞으니까
하지만 B에 대한 안 좋은 이야기를 하겠고 그 결과 B가 도태되어야하겠죠.
\vspace{5mm}

한데 재밌는 건 학생들이 B를 꽤 신경쓴다는 것입니다.
신경쓰는 이유는 간단합니다. '좋아보인다'는 것이죠
그런데 왜 좋아보이느냐, 읽어보긴 했느냐 하면 그건 아니라고 합니다.
남들이 B가 좋다고 하니까 왠지 보지 않으면 큰일날 것 같다고 이야기하지요.
\vspace{5mm}

그래서 A와 B를 모두 구입합니다. 하지만 수능 직후 A나 B나 깨끗합니다(...)
\vspace{5mm}

자, 이런 일이 왜 벌어질까요?
\vspace{5mm}

수험이란(그리고 사실 삶이란) "뇌와의 끊임없는 싸움"입니다.
뇌는 그것의 자극을 추구하는 경향이 있습니다. 식욕, 수면욕, 성욕... 등 자극적인 것을 추구합니다.
그래서 어느 일본 과학자가 쓴 책에서 그러더군요. 뇌는 주인을 배신한다고(반면 대장은 주인에게 충실하다나)
그런데 여기서 '불안감'이라는 게 중요합니다.
\vspace{5mm}

수험생은 누구나 불안합니다. 당연히 이 불안감은 공부로 이겨내야하죠.
하지만 공부도 괴롭습니다, 그렇기 때문에 뇌에서는 공부가 아닌 다른 방식으로 불안감을 해소하려하죠.
그건 바로 "남들과 다른 것"을 소유하는 것입니다.
그렇기 때문에 불안한 수험생들일수록 남들이 보지 않거나 검증되지 않은 \textbf{새로운 교재를 사려고 하지요}.
마침 그런 교재일수록 1등급을 보장한다거나 명문대 합격생의 그럴싸한 추천사(사실 글 내용은 별 게 없어요)가 달려있습니다.
자위행위를 하고 순간적인 쾌감에 행복해지듯 수험생들은 이런 새로운 교재를 구입하고 포장을 뜯을 때까지는 행복합니다.
다만 책페이지를 펼치면서 '역시 공부해야하는 건 마찬가지구나'하는 순간 \textbf{현자타임}이 도래하죠.
"결국 꿀공부는 없었어"라는 걸 깨닫죠. 그리고 공부해야하는 참고서가 늘어납니다. 부담감도 곱절이 됩니다.
\vspace{5mm}

더 심해진 불안감에 다시 또 새로운 교재 없나 기웃거리면서 '공부 안 하는 핑계'를 만들고
그렇게 교재 모으기 하다가 시간이 촉박해지면 "강사님만 믿숩니다 T$\_$T"하면서 거금 들여 인강결제를 시작하죠.
그런데 그 인강도 듣다가 다시 회의감이 들죠. 그게 인강이 별 게 아니다라는 걸 알아서가 아니라, 이걸 듣고 공부하는 게 힘들어서입니다.
그래서 다시 기본교재로 돌아가 독학해야할까.... 이렇게 우왕좌왕하는 가운데 가을이 옵니다. 그리고 n+1이 달성되죠
\vspace{5mm}

이걸 읽으면서 본인 이야기라는 데 찔리는 사람이 한두분이 아닐 걸로 알고 있습니다.
이건 상상이 아니라 실제로 관찰, 상담에서 적지않게 확인된 사례이기 때문입니다.
\vspace{5mm}

그럼 비극이 어디서 시작된 걸까요?
그거야 \textbf{인내하지 못 해서였죠}.
\vspace{5mm}

저 친구는 그냥 교재 늘리지 말고 시중교재 하나만 가지고 여러번 돌렸으면 진작 그 n수의 고리를 끊어낼 수 있었습니다.
괴로운 것을 참아내면서도 참고서를 다 풀어내고 틀린 것 오답 체크하고 모르는 것 정리해서 질문하고 하다가
어느 순간에 일정한 경지에 이르는 순간 스트레스가 풀리면서 공부를 더 하고 싶다... 하면서 공부가 쾌감이 되는 순간,
즉 공부하기 싫어하는 자기 뇌를 굴복시키고, 불안감을 긴장감으로 바꾸는 경지에 이르렀을 것입니다.
\vspace{5mm}

그런데 그걸 공부의 고통을 못 참으니까 시간과 돈을 있는대로 날려먹는 것이죠.
학원 $-$ 실강을 그나마 권하는 이유? 적어도 거기선 공부가 강제되고 도피할 수는 없으니 말입니다.
비싼 돈을 지불했기 때문에 아까워서라도 다니게 되고 그래서 저런 충동구매는 적어도 안 합니다. 학습량이 최소한 쌓이는 건 있단 것이죠.
\vspace{5mm}

여기 와서 기웃거리는 분들은 일지를 보고 그냥 따라하세요.
꿀루트가 있다? 그런 쓰레기 같은 생각을 하니까 인생을 그렇게 살아오신 겁니다. 그딴 건 없습니다.
\vspace{5mm}

위의 수험서 소비 패턴.
전형적인 사치된장패턴이라고 할 수 있죠.
공부도 안 해놓고 EBS 강의를 깝니다, 그런데 사설강의 찬양하는 이유는 결국 "비싸서'라는 결론에 도달하죠.
물론 사설은 꽤 재밌게 즐겁게 합니다. 그러나 뭘로 가든 종착점은 마찬가지일터인데 말이지요.
공부에 지불하는 건 그런 수업료보다는, "시간"과 "인내"입니다.
\vspace{5mm}

혹자는 이런 질문을 하겠죠. 당신의 이야기는 알겟는데 그건 꽤 평범하지 않느냐.
\vspace{5mm}

거시적으로 볼까요. 그런 평범한 것조차 제대로 해내는 학생들도 소수입니다.
누구나 계획은 거창하게 세우지요. 그런데 그 계획을 50$\%$ 이상이라도 주어진 기간에 해내는 사람은 10명 중 1명 될까말까입니다.
여기서 등수가 갈리는 겁니다.
\vspace{5mm}

어떤 교재나 강의가 좋은지 또한 어떤 수험생이 잘하는가 하는 가쉽에 능통해서 계획을 그럴싸하게 세우면 뭐해요.
\textbf{실천을 안 하는데}
아무리 좋은 계획도 실천 안 하면 망상이죠.
그냥 공부하다보면 계획 세우기도 귀찮아집니다. 책 읽고 문제푸는 것으로도 정신없이 시간이 지나가거든요.
\vspace{5mm}

계획은 이러쿵저러쿵 세우는데 지금 풀어놓은 교재가 없다.... 그 순간 본인은 그냥 쓰레기인 것입니다.
계획 세울 때야 즐겁게 하죠. 그런데 정작 이런 사람들이 일주일을 넘기는 경우도 드뭅니다.
중간중간에 스트레스 받는다 하기싫다 이거 꼭 해야해 우리나라는 너무 공부에 미쳐있어... 이렇게 또 가기 시작하죠.
\vspace{5mm}






\section{과외 구할 때}
\href{https://www.kockoc.com/Apoc/641984}{2016.02.19}

\vspace{5mm}

적당히 알아서들 구하시겠지만 학생 입장에서 선생을 검증해보는 것은 필수죠.
\vspace{5mm}

\begin{enumerate}
    \item \textbf{기본 개념, 공식, 성질, 정리에 대해서 증명할 수 있는가 확인해보자.}
    \vspace{5mm}

    예컨대 수학이라면
    dy/dx의 정확힌 정의, '미적분학의 기본정리'라거나 '통계의 각종 정리'에 대해서 왜 그런 것이 나왔는지 증명가능한지는 확인해봅시다.
    저걸 모르고 문제를 푼다면 그런 과외는 받을 필요도 없습니다. 단지 문제푸는 기계일 뿐인데 그럼 EBS 강의로도 충분합니다.
    그런데 상당수 문제 잘 푼다는 친구들이 저런 것도 설명 안 하고 '걍 하면 된다'라고 하는데
    걍 하면 된다 수준이면 평범한 문제는 풀지 몰라도 새로운 문제나 통수 문제는 못 풉니다.
    어려운 문제를 푸는 힘 자체가 가장 기본적인 개념을 사유해보고 하나하나 증명하는 데에서 나오는 것입니다.
    적어도 그런 걸 해주지 않으면 아무 의미는 없습니다. 이런 걸 한 줄 안다는 건 적어도 그 선생이 꾸준히 공부한다는 이야기이죠.
    \vspace{5mm}

    \item \textbf{어떤 유형의 학생이고 어디가 문제이며 과거 몇학년 때 공부가 안 되었는지 정확히 짚어야한다.}
    \vspace{5mm}

    과외가 학원과 다른 건 개인에 대해서 바로 문제가 뭔지 지적해주고 해결책을 제시해줄 수 있어야한다는 겁니다.
    심각한 문제의 경우라면 해결하는 데 시간이 걸릴지 몰라도 방향은 바로 나와야 합니다. 거의 점술가 수준으로 나와야 합니다.
    사실 공부가 망한 패턴은 한정되어있기 때문에 이걸 정확하게 짚는 건 어렵지 않습니다.
    하다 못해 학생의 방 $-$ 특히 서재나 학부모 이야기 5분만 듣더라도 대충 과거 이력이 짐작이 가야합니다
    물론 요새는 학부모나 학생들도 이런 건 다 검증합니다. 하다 못해 과외 처음 한다고 구라까고 하나하나 검증해나가죠.
    \vspace{5mm}

    \item \textbf{특정과목 전문이라는 건 말도 안 되는 이야기다.}
    \vspace{5mm}

    국어와 영어는 못 하지만 수학은 잘 한다.... 개소리입니다. 수학만 잘해서 어떻게 명문대에 가나요?
    무슨 의사 전문의도 아니고 수험과목에서 그런 건 없습니다. 공부 잘 하는 친구라면 국영수탐구 골고루 다 잘합니다요.
    수학문제 풀이에 있어서 국어실력은 상당히 중요하고, 마찬가지로 국어 독해에 있어서 수학적 문제해결 능력이 필요해요.
    탐구는 말할 것도 없습니다, 영어가 다소 동떨어지긴 했지만요.
    그럼 석박이 잘 하는가 그것도 아닙니다. 수험은 주식이나 부동산 투자와 비슷합니다, 결국 누가 더 연구를 했느냐로 갈려요.
    기출만 보아도 평가원이 어떤 의도인지 대략 짚을 수 있고, 수험생들 대부분이 어떤 경향에 치우쳐져있는지,
    아울러 시중 교재, 강의가 어디서 부족한지 짚으면 그럼 답이 나옵니다.
    수능에 올림피아드가 필요하나? 필요 없죠. 그러나 이과 수능에 수리논술이 필요한가?
    기출 분석을 해보았다면 필요하다라고 말해야할 것입니다
    특히 학생 한명만을 본다면 수학을 가르치면서도 이 친구가 국어에 약하다는 걸 눈치까고 국어 공부하라고 하거나
    자투리 시간에 영어 질문도 바로 받아줄 수 있어야합니다.  그런데 그게 아니다, 별로 기대 안 하는 게 좋을지도 몰라요.
    국어 영어는 못 하지만 수학은 잘 하니까 국영수 골고루 잘 하는 것보다 더 뛰어나다?
    그럴싸해보이지만 사실 헛소리죠.  그런 논리면 국영수 골고루 잘 하는 친구가 수학만 했으면 더 잘 나왔을 것입니다.
    \vspace{5mm}

    \item \textbf{무료 보충 가능해야.}
    \vspace{5mm}

    시간제 뿐만 아니라 진도에다가 난이도까지도 보증해줘야합니다.
    제 시간에 못 끝내준다거나 진도가 밀린다고 하면 적정한 수준에서 보충해 줄 준비까지 되어있어야합니다.
    적어도 책임지려는 사람은 보충을 해줍니다, 하지만 책임질 준비가 안 되어있다면 이건 생깔 것입니다
    물론 교통비는 알아서 익스큐즈해야할 문제겠지요.
    과외선생 중에서도 돈만 바라보는 사람이 있고 반면 자기 자존심이 강한 사람이 있습니다.
    후자는 무료 보충을 기꺼이 해줄 것입니다. 물론 그만큼 학생이 실력이 올라가길 바라겠지만요.
    \vspace{5mm}

    \item \textbf{당연히 본인이 교재 없이도 즉석에서 문풀, 설명 가능해야}
    \vspace{5mm}

    설명 불요라고 봅니다. 쎈C스텝이나 기출 킬러를 제외하고는 그냥 즉석에서 풀고 설명해줄 수 있어야 합니다.
    그리고 쎈C스텝이나 기출 킬러라고 하더라도 답지를 그냥 읊는 게 아니라
    그 문제가 교과서상 어떤 개념을 어떻게 변형시켰으며 출제자가 어떤 의도였는데 어디서 삑사리났는가도 얘기해줘야합니다.
    \vspace{5mm}

    \item \textbf{상위권 전문은 피하시길 $\bigstar$$\bigstar$$\bigstar$}
    \vspace{5mm}

    사실 이게 가장 중요합니다. 얼핏 보면 상위권 전문이 잘 할 것 같지만 실제 그렇지는 않습니다.
    상위권 전문이라는 건 상위권 학생만 받겟다 내지 상위권 내용만 가르치겠다는 건데, 정작 이건 가르치는 입장에선 어렵지 않습니다.
    상위권 학생은 알아서 공부하는 습관이 잡혀있거니와 스스로 깨우쳐 따라옵니다.
    원래 좋은 대학에 갈 수 있는 게 탄탄하기 때문에 사실 냅둬도 알아서 갈 수 있을 가능성이 높아요.
    그래서 이 경우 선생의 역할은 별 게 없는데 실적이 상당히 부풀려집니다.
    \vspace{5mm}

    진짜 실력자들은 최하위권을 최상위권으로 만드는 경우이겠으나 이 경우는 단 한번도 본 적이 없습니다(이거 로또급 아녀?)
    보통은 하위권을 중위권으로 만들거나, 중위권을 상위권으로 만드는 경우가 그나마 현실적입니다. 이건 손이 꽤 많이 가는 작업입니다.
    문제를 풀어주는 것보다 더 중요한 건 공부의욕을 북돋아주고 학생이 겪는 슬럼프나 고민을 해결하는 걸 넘어 미리 예견해 줄 수 있어야죠.
    그런데 상위권들은 이런 애로사항이 별로 없어요. 그래서 과외 선생이 정작 하는 것도 없고 진화도 못 합니다.
    하지만 하위권을 중위권으로 올렸거나, 중위권을 상위권으로 올린 선생들은 실력도 있어야하지만 대화능력이나  사람통찰도 잘 합니다.
    안 그러면 사실 성적을 올릴 수 없으니까요. 그걸 알아서인지 모르나 학부모들은 '졸업생'을 요구하는 경우가 있습니다.
    아무튼 문제 잘 풀어준다... 이것만 하자면 과외 할 필요 없이 그냥 인강만 가면 되는 겁니다.
    \vspace{5mm}

    \item \textbf{7. 가격}
    \vspace{5mm}

    가성비로 보아야겠는데 일단 가격과 수업 수준이 비례하는 건 아닙니다.
    선생들도 보면 돈에 눈이 어두운 사람이 많고, 반면 가르치는 것 자체를 좋아하는 사람도 있죠.
    어찌되었든 돈만 벌면 된다는 케이스도 있지만, 반면 먹고살기 위해 이 짓 하지만 자존심은 지키자는 케이스도 있어요.
    \vspace{5mm}

    비싸게 부른다고 잘 가르친다... 그런 경우는 없어요. 오히려 학부모들의 허영심을 알고 비싸게 부르는 케이스가 있겠습니다만
    이 경우는 그냥 척 보아도 사기꾼이 아닌가요?  가격을 높이 받을 수도 있겠지만 그건 그만한 근거라는 게 필요합니다.
    저도 이런저런 이야기 많이 들어왔습니다만 정말 학벌 별 것 없는데 '부르는 게 값이다'라는 케이스 적잖게 있더군요.
    어떻게 보면 밑바닥이니까 그런 이야기 나온다라는 생각이 들덥니다.
    정작 잘 가르치는 고수들은 사람을 별로 안 받으려 해요. 오래 할 일이 아니라는 것도 알고 있고 대개 \textbf{자기 공부하는 게 있습니다}.
    그리고 보통 피곤한 일이 아닙니다. 적당히 시세대로 받으려고 합니다.
    물론 그런 사람들을 찾는 것 자체가 꽤 어렵습니다만 어머니들이 잘 찾죠.
    \vspace{5mm}

\end{enumerate}

이런 점들을 감안해 알아서 잘 구하시기들 바랍니당.
물론 과외 구한다는 건 본인이 '병자' 상태라는 걸 느낄 때나 도움 되는 거고
그게 아니면 그만한 돈으로 차라리 집이나 차를 사거나 해외여행 가자 하고 EBS 강의 졸라 듣고 매질하는 게 더 도움될 겁니당.













\section{계획이 안 맞는 사람도 있음}
\href{https://www.kockoc.com/Apoc/645403}{2016.02.22}

\vspace{5mm}

계획을 세워야한다고 하지만, 사실 그런 계획이 안 맞는 사람들도 있다.
\vspace{5mm}

\begin{itemize}
    \item[$-$] 즉흥성이 강하다
    \item[$-$] 성격이 격정적이거나 산만하다
    \item[$-$] 뭔가 얽매이는 걸 싫어한다.
\end{itemize}
\vspace{5mm}

하여간 이외에도 여러가지 특징들이 있겠지만, 결론적으로 계획을 짜도 실천도가 낮은 사람들이 있다.
이런 사람들은 계회을 아주 간소화시켜버리거나 차라지 짜지말고
\textbf{"과제"를 설정하고 " 3$\sim$5일 정도 데드라인" 잡은 뒤 그 기간동안 과제에만 몰두하는 방식}으로 가는 게 낫다
실천도가 낮은 사람에게는 계획이 실적으로 '둔갑'해버린다. 그래서 계획만 짰는데 그게 공부한 것처럼 느껴져버리면 문제다.
이런 사람은 차라리 계획을 포기하고, "과제" 하나를 잡은 다음 단기간 내에 스퍼트해서 그걸 끝내는 방식이 나을 수도 있는 것이다.
\vspace{5mm}

어제도 충고했지만 실천도가 낮은 사람은 전과목 골고루하기보다는
아무개 선생 인강 50강이라고 하면 그걸 20강 1부, 30강 2부라고 나눠서
1부는 4일동안 끝낸 뒤 하루 놀고, 2부는 5일동안 끝내기라고 가는 게 낫다. 물론 다른 과목은 건들지도 말고
그럼 열흘이 지나면 그 아무개 선생 인강은 다 청취해버린 것이기 때문에 아무튼 '성과'는 생기고, 이로써 선순환 루트를 탈 수 있다.
\vspace{5mm}

자신이 계획을 세웠을 때 일주일에 3일 이상 날라가버린다거나, 실제 이행률이 50$\%$ 미만이라면
계획을 포기하고 \textbf{단기과제 스퍼트형으로 가는 것}도 시도해볼만하다.
이거 문제가 많냐고 할지 모르지만 사실 그런 지적이 바로 아랫 글에서 말한 '대안없는 비판'의 전형이다.
한 과제만 잡아서 그것만 몰두하는 것도 문제가 없는 건 아니지만, 적어도 이건 실천과 성과는 담보해준다.
거창하지만 실제로 실행이 40$\%$ 미만인 것이야말로 최악인 것이다.
\vspace{5mm}

+ 그리고 이건 조심스럽게 말하면, 오히려 공부 잘 하는 애들은 계획형보다는 이런 단기스퍼트형이 더 많지 않나 생각되는 것도 있다.
실제로 계획을 본인이 꼼꼼히 세워서 그걸 실천에 옮긴 경우야말로 찾아보기는 어려웠기 때문에.
물론 다수는 학원이나 과외가 시키는대로 따라가므로 적극적 계획형이라기보다는 수동적 계획형이 많다라고 할 수도 있겠지만
벼락치기의 마수는 사실 피해갈 수가 없다.
\vspace{5mm}

++ 비유하자면 이런 것이다 10명을 모두 공략하면 1:10으로 싸워야 한다.
반면 한명한명 공략해서 승리해 아군을 만들어나가면 1:1, 2:1, 3:1, ... , 9:1로 싸울 수 있다.
하나씩 처리하는 경우의 장점이 그렇다. 일단 달성한 과제는 그 다음 과제 공략에 도움을 준다.
반면 한번에 여러개를 처리하는 경우 본인은 그 여러 과목과 상대하느라 조기에 탈진해버릴 수가 있다.
\vspace{5mm}

+++ 두달 잡고 간다... 사실 이건 힘들다. 두달 동안에 지쳐버릴 수도 있고 뭔일이 날지도 모르기 때문에
뭐든지 일단은 짧게는 이틀, 아주 길어도 9일 내로 끝내는 게 좋다. 즉, 공부할 대상을 잘 세분화시켜서 그렇게 하나하나 처리하란 말씀.
수학 전범위를 일주일동안 할 수는 없지만, 가령 "미분1"만 본다고 하면 일주일이면 가능하다.
두달 이상 가는 독종도 없지는 않지만 드문 편이다. 사실 이런 성격이면 그냥 알아서도 현역으로 좋은 대학 잘 간다.
그런데 안 그러니까 지금 고민이지 않겠나
\vspace{5mm}












\section{데드라인}
\href{https://www.kockoc.com/Apoc/650700}{2016.02.25}

\vspace{5mm}

이제 황금의 3개월은 끝났죠.
\vspace{5mm}

그 다음 분기는
\begin{itemize}
    \item 3월$\sim$5월 까지의 100일
    \item 6월$\sim$8월 까지의 100일
    \item 9월$\sim$10월 까지의 60일
\end{itemize}
\vspace{5mm}

그나마 제대로 공부하는 기간이 3월$\sim$5월까지인데 솔직히 이 때도 안 할 사람은 안 합니다.
시동이 걸리는 대략 3월 중반 정도. 3평 보고 나서 일부는 3뽕.
그리고 4$\sim$5월에 좀 공부를 하고 6평을 치르죠.
\vspace{5mm}

그리고 6평 보고나서부터는 무조건 흔들립니다. 그리고 이 때부터 새로운 교재들도 무진장 늘어나죠.
기초 실력을 쌓는 건 거의 포기, 이게 시험 당일까지 발목을 잡습니다.
\vspace{5mm}

데드라인을 제시하면
\vspace{5mm}

\textbf{EBS 수특 $-$ 늦어도 4월말까지}
\textbf{기출 정리 $-$ 늦어도 4월말까지}
\vspace{5mm}

이렇습니다. 사실 이 정도는 해놓아야 안심합니다. 이걸 넘어서면 올해도 물건너갔다고 보셔도 되겠죠.
그럼 내년은? 역시 힘들 겁니다. 지금 고2들이 여간 잘 하는 게 아니라서(IMF 베이비라서 그런지 몰라도)
솔직히 지금 와서 학원 고르고 $\sim$ 한다라고 했을 때 가능성은 꽤 낮아지죠.
\vspace{5mm}

여전히 인기강사 상품 쇼핑하는 분들도 있는데 일단 신청했으면 제발 빨리 끝내세요.
수험이란 결국 남들보다 '많이', '신속히' 정리하는 놈이 이깁니다. 강사 믿고 끝까지 간다고 해도 확률은 20$\%$ 내외입니다
지금 와서 쎈 봐도 되냐 풍산자 봐도 되냐 그딴 헛소리하지말고(하려면 황금의 3개월동안 했어야지)
기출이라도 빨리 정리하세요. 이거 할 수 있을 거라고 다들 착각하실 건데 망상과 실천은 다릅니다. 실제로 해보면 경악하실 것입니다.
\vspace{5mm}

강의만 들으면 된다... 황금의 3개월동안 기본교재와 기출 푼 경우라면 먹히겠지만 그게 아니면 강의 들을 때는 행복하겠지만
문풀 들어가면 비명을 지를 겁니다. 지나치게 강의환상에 빠져서 게을리한 결과 개판오분전이라는 걸 꼭 본인 인생 걸고 확인하는 사람들이 있죠.
지금 막 시작하는 사람들은 4월말까지 기출 빨리 정리하시길 바랍니다.
\vspace{5mm}

솔직히 말해서 입시는 지금 시점에서 50$\%$는 이미 결정.
공부 힘들어 죽겠다 어쩐다하는 사람도 11월부터 2월까지 꾸준히 공부했으면 여름에 우위를 확인하실테고
지금부터 학원강의만 따라가도 되지라는 사람들은 그냥 '형벌'받는다고 생각하고 죽어라하시길.
나머지는 걍 답 없어요.
\vspace{5mm}










\section{수학 커리}
\href{https://www.kockoc.com/Apoc/651201}{2016.02.25}

\vspace{5mm}
\begin{enumerate}
    \item 풍산자
    \item 쎈 (복습용 RPM은 선택적으로)
    \item 마플(마더텅이나 자이로 가도 좋음)
    \item EBS 시리즈(수능특강부터 시작해서 N제, 수능완성까지 포함)
    \item 급품벨(일등급수학, 일품 수학, 블랙라벨) 중 택2
    \item 실력정석(바이블이나 수학의 원리 도 좋음) + 교과서(구할 수 있으면 교사용 지도서도 좋음)
    \item 선택적 실모, 혹은 구할 수 있으면 과거 본고사 문제
\end{enumerate}
\vspace{5mm}

이렇게만 해도 11월까지 다 풀 수 있을지도 고민인데 교재 모자라서 혹은 강의 안 들어서 망하는 일은 없음.
사실 쎈과 마플까지만 다 풀어도 문제를 풀고 오답정리하는 과정에서 개념의 전반적 틀과 논리는 다 갖춰집니다.
마플까지 가면 안정적 2등급은 무조건 나옴(왜냐면 마플까지 끝내는 인간도 생각보다 적음)
\vspace{5mm}

EBS 시리즈 넣는 건 간단함, 시중교재를 살짝 비틀거나 색다른 문제를 군데군데 박아넣습니다.
급품벨은 당연히 \textbf{풀어야}합니다. 다 풀 필요는 없고 2개만 골라서 풀어도 좋죠.
\vspace{5mm}

1, 2, 3을 대략 4월까지. 4번은 나오자마자 순삭.
그리고 5번은 8월까지
\vspace{5mm}

그렇게 한 다음 실력정석을 보면서 필요한 것만 발췌해 읽고 풀면 됩니다.
정석은 초보자에게는 맹독이지만, 고수에게는 만독불침의 명약입니다. 그러니까 5까지 다 하고 보면 됨
그 다음 교과서들을 주문해서 군데군데 잡다한 것들을 봅니다, "이렇게 놀라운 출제소스들이 있다니"에 다시 놀라게 됩니다.
마지막으로 선택적 실모 보시면 됩니당. 출제자들의 마인드나 성향이 읽혀질 것이고 해설 보면서 실력의 한계도 읽어낼 수 있을 겁니다.
\vspace{5mm}

그런데 이제 곧 3월이니 1, 2부터 시작하면 시간이 주욱 밀리죠.
지금 기출도 중간 정도는 풀고 있어야 합니다.
콕콕 내에서도 빨리 하라는 잔소리 듣고 하라는 대로 해서 실력이 늘어난 걸 절감한 케이스가 있고
반면 3월이 되었는데 이제 와서 저거 할까요 하는 사람들도 있습니다(...)
\vspace{5mm}

저대로만 다해도 안정적 1등급은 뜹니다. 저걸 해내기 힘들어서 그렇지.
저걸 다 하고도 모자라면 그 때부터는 일본 문제집을 수입해보거나 경문사에서 나온 수학교양서를 읽으시면 됩니당.
기타 이상한 야매교재들은 볼 필요가 없습니다(그런 것 보다간 기본적인 사고가 망가집니다)
\vspace{5mm}

비밀이랄 것도 없어서 이건 그냥 걍 공개합니다. 사실 다 아는 내용 아닌가 싶은데
무엇보다 고2 분들은 저거 빨리 밟으시길. 몇몇 상위권 표본들이 벌써 5까지 병행하는 케이스를 들었습니다(...)
님들은 수학을 빨리 정리하고 국어로 몰두하셔야할 것이니다. 2018년도 수능의 핵심은 '국어'인 것이 명백관화해졌습니다.
\vspace{5mm}

수학을 일단 저렇게 가야하는 이유는 '컷'이 높기 때문입니다.
컷이 높은 과목은 "실수'를 안 하는 게 핵심입니다. 심화의 필요성이 줄어들죠, 특히 학생들이 쓴 출처불명 야매교재 심화인 척 하는 건 자살행위.
그럼 컷이 낮은 과목은? 일부 탐구과목들이 있겠고 혹은 국어가 해당될 수도 있습니다(아마 수학 컷을 낮출 일은 없어보입니다)
이런 과목들은 '심화'해서 공부해야합니다.
\vspace{5mm}









\section{공부를 해야하는 이유}
\href{https://www.kockoc.com/Apoc/658157}{2016.03.01}

\vspace{5mm}

친구 : 겉으로 위로하는 척 하지만 뒤로는 킬킬킬 웃으며 행복해한다.
가족 : 역시 위로하는 것 같지만 사실 별로 도움은 주지 않으며 나중에 왜 xx는 하는데 너는 못 하냐 나온다.
\vspace{5mm}

끝까지 배신 안 하는 건 자기가 공부한 것임.
돈은 본인이 능력이 없으면 사정없이 떠남.
낭비도 무식한 놈이 하는 것임.
\vspace{5mm}

어차피 이런 얘기를 해도 다수는 그럴 리가 없어
당신은 세상을 너무 비관적으로 보는 게 아니냐 하지만
결국 저렇게 되어있음, 진실의 맛은 씁쓸함,
거꾸로 말해서 씁쓸한 맛이 나는 것은 믿어도 좋고
달콤한 맛이 나면 의심해보아야 함.
\vspace{5mm}

아마 다수는 부유하고 화려하게 살고 싶어서 공부할 것임, 그리고 그건 첫동기로서 나쁘지 않음.
그러나 중간부터는 사실 그것도 만만치 않구나 깨달아야하지만
공부가 어려울수록 진입자가 적기 때문에 \textbf{이 놈 덕분에 앞으로 생존할 수는 있겠구나...} 그런 마인드로 가야함.
잘못된 걸 공부한 게 아니라면 이 놈은 어지간해선 배신하지 않음.
물론 잘못된 것을 학습하면 (얼치기 운x권 사상이라거나 사이비 종교 교리) 그런 건 정말 배신하지 않고 확실히 인생을 조져줌.




\section{반도의 수학교재}
\href{https://www.kockoc.com/Apoc/659705}{2016.03.02}

\vspace{5mm}

개정되었다는 교재들을 보면 그냥 개정이 되었는지 개정하는 흉내를 냈는지 의심이 가는 게 많음.
소감은 딱 : \textbf{'이런 꿀장사가 따로 없겠구나'.}
조금만 교과서만 봐도 교정할 수 있는 오류가 10년째 그대로인 경우도 많음.
\vspace{5mm}

그래서 교과서가 낫다는 말이 나오는 것임.
왜냐? 교과서 저자진들은 계속 연구하는 분이거든
그나마 제대로 공부하고 연구하며 해외수학교재도 번역 공부하는 분들이 보이니까.
일부 교과서 저자 분들은 '번역자'도 하고 계심, 그럼 번역 과정에서 좋은 책 내용을 흡수해서 교과서에 반영하는 건 당연하지 않겠음?
\vspace{5mm}

반면 학생이나 학원강사가 잘 쓸 수 있을까.
솔직히 학생교재는 볼 필요는 별로 없다고 생각함. 그 학생이 아무리 잘 써보았자 자기가 배운 사교육 내용을 그냥 짜깁기한 것임.
그리고 학원강사들도 강의하느라 연구하고 정리할 시간은 \textbf{별로 없음}.
그렇기 때문에 오히려 사설 쪽의 내용은 풍부한 것처럼 보여도 그것 자체가 저기 섬나라 것을 베껴온 경우가 많으며
실제로 진보하지 못 했다라는 문제가 있음.
\vspace{5mm}

이래서 역설적으로 교과서만 진보한다는 이야기가 나오는 것임
교과서를 보라고 하는 결정적 이유가 이것임. 정확히 말하면 \textbf{교과서 저자나 공교육 교사가 쓴 책을 보는 게 낫다}고 말해야 함.
쎈이나 EBS 보라는 것도 포괄해 말하면 '제대로 뭘 알고 있을 가능성이 높은 분들이 쓴 것'을 보면 되는 거라고 얘기함.
\vspace{5mm}

그리고 교과서의 장점은 잡스킬이나 지엽이 아님. 바로 "읽는 방법"을 선사해준다는 것임.
교과서에서 풍부한 내용을 기대할 필요는 없음, 어디까지 교과서에는 기본, 기본의 기본, 기본의 기본의 기본 .... 이 나와있음.
그런데 최근 3년치 수능에서 특정 문제를 못 풀었다면 그건 스킬을 몰라서가 아니라, \textbf{그 문제를 읽는 법을 몰라서 그럼}
그럼 왜 읽는 법을 모를까. 그거야 기본도 안 되어있어서 그런 것임,
미적분이 어디서 출발했는지 확통의 기본 프레임이 뭔지 그게 안 되어있으니까 문제를 조금만 꼬아내도 못 푸는 것임.
평소에야 맹물은 맛이 없어서 탄산음료나 커피를 마시겠지만 힘든 운동을 한 뒤에도 과연?
\vspace{5mm}

신기한 것은
일본인들은 책 한권 쓰는데도 수년걸리고 정말 제가 봐도 잘 쓰는데도 그걸로 재벌타령하거나 그러지 않는데
우리나라 사람들은 머리가 정말 뛰어난지 책도 정말 쉽게 쓰고 그걸로 참 잘 벌어들이지 말입니다.
그런데 왜 모 아이돌 경쟁 프로그램이 일본 프로그램과 유사한 그런 광경이 떠오르는지
\vspace{5mm}

그리고 그런 천재(?) 저자에게 출처 물어보면 답을 한 경우가 없었음. 그래, 장사하셔야지.
\vspace{5mm}

+
뭐 하기야 학창시절에 취미가 괴수님들이 쓰신 책이 뭘 표절했나 확인하는 거였는데 이것도 참.
제가 확인한 사람만 무려 5분이었음. 일본책이고 미국책이고 아니 베끼려면 제대로 베끼지 그것도 원생들 시켜서 대충 번역(...)
그러신 분이 제자들에게는 똑바로 살라고 그러던데.
\vspace{5mm}

++
위에서 말했지만 최근 3년치 기출보면 스킬이 문제가 아니라 문제를 '해석'하고 해결의 '프레임'을 짜는 게 더 중요한 문제입니다.
이미 평가원 문제는 진보했어요. 그런데 아직도 거액을 지불하고 잡다한 스킬이나 이상한 문제 푸는 것에 집착하는 사람들이 많죠.
문제 "독해력"는 인강이나 잡교재가 간접적으로 해결해 줄 수는 있으나, critical path는 결국은 혼자 끙끙대며 철학자질하는 것입니다.
\vspace{5mm}

+++
일본 것 베끼면서 대한독립만세라고 해보았자 아무 소용없죠. 입으로는 한민족 만세하면서 일본 걸 참 잘도 표절한다는 게 거참.
가장 미련한 게 자기들이 배우거나 애용하던 게 뒤늦게 일본 것임을 깨닫고 이걸 '순화'시킨다고 억지로 명칭 바꾸거나 하는 것인데
그냥 일본식 표현쓰는 게 낫습니다. 우리가 쓰는 한자어가 대부분 일본산인 걸 알면 어휘 자체를 다 갈아엎을 것도 아니고.
\vspace{5mm}






\section{승부는 곧 결정납니다.}
\href{https://www.kockoc.com/Apoc/662562}{2016.03.04}

\vspace{5mm}

현실의 논리가 사람의 직관을 배신하는 경우가 많은데 그건 수험 스케줄도 마찬가지.
이제 6월까지 가면 그 때부터는 공부를 더 하기 힘들죠. 뭐 한다고 \textbf{말은 잘 하는데} 그런 사람은 20명 중의 1명 정도.
\vspace{5mm}

거의 다 9월까지는 끝낼 거야... 라고 하지만 9월 되어서도 못 끝내서 10월까지 잡았다가 멘붕해서 양줄이려고 하다가 말아먹고.
거기다가 6월$\sim$8월은 무더위가 오죠. 지구온난화 덕분에 5월 중순부터 그럴지도.
그래서 많은 학생들이 이 때에 넉다운당합니다.
\vspace{5mm}

이렇게 계산하면 실제로 승패는 지금부터 100일 이내, 즉 5월말이면 95$\%$는 결정난다고 보아도 무리는 없죠.
그럴 리가 없어, 네가 소설쓰는 거야... 라는 분도 계시겠는데 저야 한마디. 당신은 그렇게 생각했으니까 여태껏 그렇게 살아오신 것임.
\vspace{5mm}

지금 이 시기에도 인강 뭐 들을까 고민하면 사실 답이 없다고 보는데... 왜냐면 공부량도 양이지만 그냥 마인드나 습관이 실패하는 유형 그대로라서.
그냥 아무 생각없이 하나 잡으면 '끝내는 결과' 모드로 가야지, 자꾸만 시작해야지... 하면 영원히 시작만 하게 됩니당.
시작에만 능한 사람은 '시작'만 하면 다 되는 줄로 착각하죠. 시작이 반이다라는 말은 시작만 하면 영원히 '반' 밖에 못 한다는 이야기이도 한데.
\vspace{5mm}

황금의 3개월동안 어영부영하다가 지금 공부하는 사람은 이제 \textbf{하루 순공부시간 8$\sim$9시간 잡고} 5월까지만 공부한다... 라는 마인드로 가는 게 현실적일 듯
그것도 안 되는 사람들이야 그 때부터 또 이상한 잡다한 교재에다가 작년에도 별 쓸모없던 실모 같은 것 보겠다고 돈쓰고나 있겠고.
물론 11월부터 공부하신 분은 하루 순공부 6시간만 유지하시면 됩니다. 일주일 하루 쉬는 건 당연하고.
\vspace{5mm}

이번 한달도 공부 그렇게 못 하면. 그냥 내년 기약하는 게 나아요. 자신감 떨어뜨리고 협박하는 게 아니라 현실적인 이야기를 하는 겁니다.
왜냐면 3월달도 공부 안 한 사람이 그 이후에도 공부 제대로 할 리는 만무해서리. 이런 사람들은 걍 알바 뛰고 개고생하고 정신차리는 게 먼저임.
\vspace{5mm}






\section{친구나 가족 다루기}
\href{https://www.kockoc.com/Apoc/663083}{2016.03.04}

\vspace{5mm}
\begin{enumerate}

    \item 자기보다 잘 나가는 친구와는 의외로 오래 유지할 수 있습니다.
    만약 여러분이 n수생이고 친구가 잘 나가는 대학생이거나 직장인(...)이면 자기가 폐를 끼친다고 생각할 수도 있겠습니다만.
    실제로는 본인이 친구에게 \textbf{더 많은 행복}을 선사합니다.
    다른 게 아니라 친구는 여러분을 보면서 '우월감'을 느끼기 때문입니다.
    그 친구가 다른 잘 나가는 사람들과 같이 있을 때 못 느끼는 행복한 감정을 여러분을 통해서 느낄 수 있죠.
    눈치없는 사람들이 이런 걸 가지고 "아니 제 친구들은 안 그러던데요"할지 모르겠지만 그거 참 순진한 이야기임.
    \vspace{5mm}

    \item 부모님들은 실제로는 여러분이 잘 되는 것보다는
    여러분을 통해 어떻게 자랑을 할 수 있을까, 아니면 얼마나 등골 덜 뽑힐까 하는 데 관심이
    그러다고 뭐라 할 수는 없습니다. 부모님이 금전적 지원을 한다면 이 분들이 님들 인생의 오너죠.
    그렇다고 님들이 배당을 한다거나 아니면 빚을 갚는 채권자 관계가 되는 것이 아니라면 사실 할 말은 없음.
    명문대에 갈 필요없다, 씩씩하게 자라다오... 라는 걸 믿는 순진한 사람은 아무도 없습니다.
    어차피 배당(...)은 불가능하고 오히려 '투자'해달라고 할 게 뻔하다면 체면이라도 세워주야하지 않겠습니까.
    \vspace{5mm}

    \item 자기가 망하고 있을 때에는 부모님보다는 친구의 '악담'이 더 진실에 가깝습니다.
    부모님은 주주이므로 내부인이지만 친구는 경쟁자이기도 하기 때문에 손님의 관점, 즉 객관적 입장에 서 있기 때문이죠.
    자기 문제점은 자기를 싫어하는 사람이 훨씬 잘 알고 있습니다.
    그렇다면 친구가 자기를 싫어하는 사람에 속하냐 하면 \textbf{그럴 수도 있고 그렇지 않을 수도 있습니다.}
    \textbf{적당히 거리를 두고 있는 친구가 필요한 이유입니다.}
    \vspace{5mm}

    \item 부모와의 관계 설정이 문제일 건데 주주의 의견은 존중만 하되 들을 필요가 없습니다.
    다시 말해 \textbf{투자받은 만큼 성과를 내서 체면을 세워준다...} 빼고 여러분이 할 수 있는 건 아무 것도 없습니다.
    흔히 말하는 진로나 직업 설정에 대해서는 부모님이 아주 잘 나가는 사람으로서 세상 돌아가는 걸 아는 분들이 아니라면 무시해도 됩니다.
    예컨대 부모 말 들어서 진로 선택했는데 그게 망하는 길이었더라.... 그거 아무도 책임지지 않습니다.
    반대로 님들이 부모의 반대 무릅쓰고 원하는 길로 가서 죽어라 노력해서 대박 터졌다, 그럼 부모님은 자기가 기여했다라고 생각합니다.
    원래 인간사 돌아가는 게 다 그렇게 되어있습니다.
    \vspace{5mm}

    \item 간혹 가다보면 부모님이 자기 의사를 존중해주고 .... 라는 식의 동화를 보는데 그딴 건 없습니다요.
    의사존중이라는 건 존재할 수 없죠. 의사를 존중한 척 하면서 교묘히 설득하는 것이지.
    자식 뜻대로 하게 해준다라는 건 그냥 '무관심' 아니면 '무관심을 가장한 관심' 양쪽인데
    무관심은 사실상 포기입니다. 이래서 잘 된 케이스는 없어요.
    반면 관심이라고 하는 건 부모가 자기가 희생해서라도 자식을 위해 투자하겠죠.
    \vspace{5mm}
\end{enumerate}

가장 좋은 건 자녀도 열심히 공부하고, 부모도 그렇게 투자해주는 경우입니다.
그럼 가장 안 좋은 건 둘 다 X냐 하겠지만 실제로는 그게 아니죠. 둘 다 X면 사실 서로 원망할 껀덕지가 없어서 그나마 낫습니다.
최악인 것은 자녀가 열심히 공부하지 않는데 부모가 투자한 경우죠.
\vspace{5mm}






\section{사설 인강 따라가고 있으면 그걸 그만둬야하느냐.}
\href{https://www.kockoc.com/Apoc/663118}{2016.03.04}

\vspace{5mm}

수험 끝났다고 훈수 두는 사람들이 이래라저래라가 먹히는 건 11$\sim$12월이고
지금은 하던 것만 끝까지 가야합니다.
\vspace{5mm}

가장 최악인 건 자기가 하던 걸 중단하고 \textbf{다른 게 좋지 않을까... 기웃거리는 것입니다}. 그럼 결국 어느 것도 하지 못 하기 때문입니다.
EBS갔다면 그냥 EBS 가시고, 사설 간다면 그냥 사설 가시면 됩니다. 그럼 둘 다 유의미한 차이는? 그다지 없습니다.
자기 통제 안 되어서 재종 간다면 재종 따라가는 게 답입니다. 다만, 재종에서 탑을 달려야합니다.
\vspace{5mm}

이 시기에 와서 저에게 쪽지보내서 강의 뭐 추천해달라 뭐하라하는 사람들 보면 이제는 짜증납니다만
그냥 자기가 끌리는 것 하면 그냥 그거 가시라는 것입니다. 그리고 이제 3월인 이상 제가 잘 충고해줘보았자 그건 별 소용이 없어요.
시작이 중요한 게 아니라, 자기가 하던 게 있으면 그걸 완료, 정리하고 계속 반복하라는 것입니다.
\vspace{5mm}

강의나 교재 같은 거 평론이야 저 같은 사람은 할 수 있죠. 왜냐면 과목 내용을 그나마 알고 있고
심각하게 학생이 해당 교재나 강의와 안 맞는다 하면(이건 노력해도 안 되는 정말 극악의 케이스입니다) 그 때야 충고할 수 있으니까요.
\textbf{그런데 대부분 안 맞는다는 건 뻥이고 그냥 공부하기 싫어서 핑계대는 것이죠.}
강의나 교재 때문이라고만 생각하지 자기가 쓰레기인 건 절대 인정하지 않으려고 하죠.
그냥 하는 것 꾸준히 하면 되는 거지 뭘 일일히 물어보고 하는지 모르겠네요.
\vspace{5mm}

뭐가 좋느냐 따지지 말고 하나라도 잡으면 꾸준히 끝까지 가시기 바랍니다.
이제는 뭘 따질 시점도 아니고 아울러 교재나 강의 추천해달라 그런 댓글 달면 상대가 아주 심각한 상황이거나 병자가 아닌 이상은 씹어버립니다.
자기들이 공부를 꾸준히 할 거라고 정말 진정 믿고 뭔 강의가 좋냐 따지는 것이면 착각도 이만저만이 아니죠.
\vspace{5mm}





\section{불행중독증}
\href{https://www.kockoc.com/Apoc/665182}{2016.03.06}

\vspace{5mm}

여기 콕콕에도 있고 어디든지 널렸습니다만.
\vspace{5mm}

자기가 불행하게 살았고 힘들다.... 라는 걸 자꾸만 강조해대면서 거기서 자존감을 얻으려는 케이스가 생각보다 많습니다만.
간단히 말해서 지금 행복한 사람도 언젠가는 불행해지고, 불행한 사람도 언젠가는 행복해집니다.
새옹지마라는 말에서 눈여겨볼 건 고대, 중세에도 인간살이는 마찬가지였을 거란 사실입니다.
\vspace{5mm}

우선 과거에 자기가 피해를 보았다... 라는 것에 대해서는 가족이나 친구나 지인을 원망하겠습니다만
과거에 집착하는 것은 그냥 현재 노력을 하기도 싫고 미래를 위해 준비하기 싫다는 "고급 핑계"에 불과합니다.
왜 고급 핑계냐고요? 본인 스스로도 이게 핑계인지 모르니가 고급 핑계지요.
\vspace{5mm}

과거에서 얻을 건 교훈 밖에 없습니다. 그런데 이 교훈은 결국 "자기 반성"으로 이어집니다.
부모가 내 인생 말아먹었다? $-$ 그럼 왜 그 부모 말을 그냥 듣거나 거부하지 않았을까 하는 자기 반성으로 결국 이어집니다.
아무리 부모라고 하더라도 기본은 남이고,
분명히 내가 그렇게 따라가지 않을 수 있거나 다른 길을 갈 수 있었다고 생각하면 내 문제가 되지요.
\vspace{5mm}

무엇보다도 잘 찾아보면 자기보다 불행한 사람은 훨씬 많습니다.
딴 이야기지만 입으로만 서민타령하면서 못 살겠다 하는 거... 저는 그건 일단 의심해요
정말로 먹고살기 힘든 사람들은 하루하루 입에 풀칠도 하느라 일하느라 정신없고 그렇게 목소리 낼 여유조차 없어요.
\vspace{5mm}

마찬가지로 정말 힘든 사람들은 두가지입니다.
아예 맛이 가서 말을 하기도 힘들어서 통각신경이 마비된 상태로 하루하루 살아간다거나
아니면 그걸 극복하려고 부지런히 일해서 시간이 모자라죠.
\vspace{5mm}

그럼 너는 왜 이렇게 단정적으로 글을 쓰냐하는데 제가 이런 쓰레기짓을 해보았기 때문에,
그리고 온오프에서 이런 걸로 적발한 사례가 많아서 압니다.
확신적으로 말하는데 '과거타령'하면서 남과 비교질하는 건
오히려 여유가 넘친다는 이야기라고 해도 틀린 이야기는 절대 아닙니다.
이런 경우 실패한 건 '타인' 때문이기도 하지만 결국은 '나 자신'의 스타일 문제입니다.
\vspace{5mm}

이런 케이스는 한두번은 상담을 들어줄 수 잇지만 그것도 자꾸 하다보면
'당사자는 계속 그것에 중독되어서' 또 쓰레기짓을 하죠.
과감히 혼낼 때는 혼내고 지적할 때는 지적해야 합니다.
이런 사람들은 기본적으로 "과거타령"하는 것에서 뇌가 쾌감을 얻고 있으니까 문제입니다.
아니, 이렇게 스트레스 푸는 것도 나쁘지 않냐....
불행한 것으로 쾌감을 느끼는 뇌라면 그 주인의 인생을 어떤 방향으로 이끌지야 명백하지 않습니까?
\vspace{5mm}

상상을 초월하는 n수생들 많아요. 요즘은 실패를 하면 기본 5년은 넘어갑니다.
그런데 이런 사람들 분석을 해보면 의식주 고민은 오히려 덜 하고, 인간의 기본 욕망은 다 채우고 있습니다.
생각보다는 공부할 여건이 다 갖춰져 있는데 계속 실패를 반복하려고 하죠.
왜 그럴까... 라고 하며 다방면으로 해답을 찾았는데
지금 내린 결론은 당사자의 뇌가 그렇게 무의식적으로 이끌고 있다는 것입니다.
\vspace{5mm}

마약중독에 빠진 사람, 즉 뇌가 마약으로 쾌감을 얻는 사람은 계속 마약을 할 수 밖에 없죠
이게 게임이든 알콜이든 성적인 것이든 마찬가지입니다.
심지어 남자에게 학대를 당하는 여자들조차도 그런 학대상황에 안 벗어나는 경우도 일부분은 이렇게 설명할 수 있을 것입니다.
마찬가지로 수년동안 진전이 없는 사람들을 보면 "과거에 xx 때문에 망했다"라는 걸 강조한다는 공통점이 있었습니다.
그리고 계속 그런 이야기를 수차례 반복해요. 본인도 괴로운 이야기일 건데 왜 그럴까.
간단하죠, 고통과 쾌감은 통하기 때문입니다.
\vspace{5mm}

이 글을 읽는 사람 중에 상당수 $-$ 특히 ㅇㅂ에서 오는 사람들은 $-$ 는 여기 해당할 거라고 확신하고 있습니다.
단언코 말하지만 이건 '치료'의 대상이지 위로의 대상이 절대 아닙니다. 말이 좋아 과거지, 그거 그냥 자기 자신도 모르는 고급 핑계죠.
비교해서 말하면 성공해나가는 사람들이나 부자들이 과거 타령은 안 합니다. 그 사람들은 늘 미래를 바라보고 있죠.
과거에 이래서 망했는데 ... 할 시간이 있으면 미래에 어떻게 하겠다라고 생각해보고 구체적인 준비를 하는 편이 낫습니다.
\vspace{5mm}

개인적인 얘기지만 정작 제가 타인에게 과거 불행이나 고민을 털어놓은 적은 별로 없습니다.
왜냐면 살아오면서 정말 두자리 수 되는 사람들에게 저런 식의 불행중독증에 걸린 두자리식 '인생한탄'
"나도 한 때에는 잘 나갔는데 $-$", "이게 다 ○○ 때문이다"... 라는 식의 이야기는 지겹게 들었기 때문입니다.
그리고 그 사람들은 절대 그거 한번만 이야기하지 않습니다. 수십번을 이야기하고 그리고 수년동안 안 변합니다(...)
그냥 정신차렷 소리 지르면서 공부하기도 싫고 일하기도 싫으니까 한탄하는 거지 윽박지르는 게 명쾌한 해결입니다.
\vspace{5mm}










\section{미래준비자}
\href{https://www.kockoc.com/Apoc/666823}{2016.03.07}

\vspace{5mm}

가난한 집안, 막장 가정 등의 문제는 그런 부모의 가치관이 자녀에게 그대로 전파될 가능성이 높다는 것이죠.
아래 불행중독증에 적었지만 그런 불행에 중독된 사람은 계속 불행해지려고 하는 성향이 있습니다.
하지만 미래준비자들은 실패조차도 미래의 성공을 향한 자산으로 삼으려고 하지요.
\vspace{5mm}

아울러 불행중독자들은 자꾸만 타인과 비교하면서 나는 왜 이렇게 못 났을까, 저 녀석은 그래도 ○○가 작을 거야라고 위안하지만
미래준비자들은 뛰어난 사람이 있으면 어떻게 그 재능을 배울 수 있을까, 아니 그 사람을 부하로 삼거나 동지로 삼을 수 있을까 생각합니다.
\vspace{5mm}

간단히 생각해봅시다.
불행에 중독된 사람이 지도자가 되면 그 사람 혼자 망하는 일로 끝나지 않습니다.
과거에만 집착하지 미래를 준비하지 않는 사람은 자기 책임이란 걸 지지 않고 어떻게든 타인에게 전가시키려고 하죠.
그리고 이건 실제로도 20세기 공산권 지도자들(이 사람들은 그 쪽 분야 대가였습니다)이 정말 잘 보여주었죠.
\vspace{5mm}

그리고 지도자들은 직접 일하는 게 아닙니다. 타이밍 잘 맞춰 적재적소에 인재들을 배치할 줄 알고 갈등도 조절할 수 있습니다.
인재들이 자기 전문분야를 공부하는 데 그친다면 지도자들은 사람을 공부하고 부릴 줄 알며
동시에 어떤 미래가 전개될지 생각해보고 가능성있는 시나리오를 짜서 그 방향으로 리스크를 감수하고 움직입니다.
\vspace{5mm}

불행에 중독되어있지 않고 항상 미래를 준비하니까 실패할 가능섣도 낮지만
실패했을 때도 손절매를 빨리 하며 그리고 그 손실을 넘어서는 교훈을 얻어냅니다.
\vspace{5mm}

....
\vspace{5mm}

이 정도는 책에 다 있는 내용이기도 하지 않을까 싶은데, 문자로 쓰여있는 것과 실천하는 것은 다른 것 같습니다.
저 역시 실천을 못 하는 건 마찬가지가 아닌가 싶은데
아무튼 상담하다보면 수험상담이 야매성 정신상담으로 이어지는 경우가 많고
그래서 귀납적으로 발견하는 건 성공하는 사람과 실패하는 사람의 \textbf{마인드가 정말 다르단 겁}니다.
\vspace{5mm}

입시에 성공한 사람들은 쿨한 성향이 있습니다. 구체적으로 말하면 과거사에 집착하지 않으며 실패를 바로 인정해 손실을 최소화한다는 것이죠.
거기다가 열심히 공부한다... 와는 다릅니다. 그것보다는 자기 인생을 열심히 경영한다... 가 더 맞는 이야기이겠죠.
대화하면서 느끼는 일종의 공감각적 이미지는 "사로잡혀있지 않다"라는 것입니다.
반면 N수생들은 대화하면서 이입을 하려고 하면 뭔가에 사로잡혀 있고 거기에 집착하고 있다는 걸 느낄 때가 많습니다.
저주에 걸린 건데 그 상태에서만 노력해보았자 저주의 약효가 커지지 않을까, 그런 생각도 들기도 합니다.
\vspace{5mm}

자기 과거가 어두우니까 집착을 하면 그만큼 보상도 클 것이다라는 미신이 작용하는게 아닌가도 싶은데 말 그대로 미신입니다.
과거가 힘드니까 앞으로 잘 될 것이다... 라는 건 어디까지나 서사문학의 틀일 뿐이고
실제로는 그런 힘든 과거에서 벗어나는 것, 즉 \textbf{역경을 극복해야 잘 되는 것이지} 역경에 집착하면 평생 노예가 되어버립니다.
\vspace{5mm}









\section{[수험교양 001] 장우석, "수학멘토" $\&$ "수학 철학에 미치다"}
\href{https://www.kockoc.com/Apoc/668096}{2016.03.08}

\vspace{5mm}

읽으면서 수학적 사고를 키울 수 있는 몇 없는 책 중 하나.
사실 이 시점에 수험생이 읽는 건 좀 늦지 않을까 싶지만 수학에 관심이 있다면 쉬는 동안 읽으면서 사고를 체화시키는 것도 좋다.
우선 이 책부터 소개하는 건 간단, 괜찮은 수학책들은 주로 일본, 미국, 독일 책이다.
(여담이지만 좋은 책을 고를 때에는 나중에는 저자고 출판사도 다 제끼고 국적부터 확인하는 습관부터 들게된다).
그런데 위 책들은 우리나라 사람이 쓴 것 치고 대단히 철학적이거니와 핵심적인 것만 건드리고 있어서이다.
\vspace{5mm}

내용 자체는 수험과 거리가 있을 수도 있다. 그러나 그런 건 수험서에 맡겨야 하는 일이고.
진정 수학적으로 사유하는 것이 무엇인가에서는 개인적으로 위 책들의 영향을 적지 않게 받았다고 생각한다.
수학멘토는 저자가 나름 수학과 철학을 접목시키면서 공부한 노트이고
수학 철학에 미치다는 것은 유명 수학자들의 연대기를 따라가면서 어떻게 수학이 발전했는가를 고딩이 이해할 수 있게 정리.
이 2권을 둘 다 읽고 고민하다보면 아주 간략하게 '근대 수학'의 가치관... 이라는 것을 형성할 수 있을 것이다.
\vspace{5mm}

수험 수학을 공부하다보면 처음에는 문제가 풀리지만, 고수 단계에서는 문제를 '풀어야' 한다는 것을 알게 되고,
나중에는 문제를 풀고 안 풀고가 중요한 게 아니라 그 문제가 일각에 불과한 '눈에 보이지 않는 빙산'을 읽어야 한다는 것을 알게된다.
그럼 그 빙산은 어디서 떨어져 나왔을까라는 데까지 이르게 되면 거대한 남극대륙을 떠올리게 되면서
과거 수학자들은 어떻게 사유했을까에 대해서 궁금해지게 된다(이 지점부터가 바로 29번, 30번의 시작이라고 할지도 모른다)
물론 위 책들을 읽는다고 풀 수 있는 건 아니지만 수험생에 있어서는 초고수의 영역에 해당하는 분야에 대한 개관서로는 적합하다고 할 수 있다.
인강이 맞지 않는 사람들은 이런 책들에서 의외의 깨달음을 얻을 수도 있지 않나 싶어 소개한다.
\vspace{5mm}







\section{입시 분석이 무의미할 수도 있는 이유}
\href{https://www.kockoc.com/Apoc/673033}{2016.03.11}

\vspace{5mm}


\begin{enumerate}
    \item 서울대에 들어가는 사람들은 그런데 큰 관심도 없음.
    특목고, 자사고, 일반고 최상위권들은 어차피 끼리끼리 다 알고들 있어서 자기들끼라 우열관계로 분석들 하고 있음.
    저런 걸 좋아하는 학생들은 자기가 공부를 잘 한다고 착각하지만 실제로는 그저 그런 케이스이고
    가장 좋아하는 게 부모들인데 알고보면 공부가 뭔지 잘 모르거나 이며 현역이 아니며 저런 이야기에서 심리적 위안을 얻는 부류
    \vspace{5mm}
    
    \item 속칭 스나이핑해서 들어간다고 하지만 이것도 정말 효용이 있는지는 의문.
    대학에 들어가는 이유를 두가지로 분류하면 하나는 실속이고 다른 하나는 체면치레인데
    실속있는 곳은 스나이핑한다고 해도 어차피 못 들어가고, "나 대학갔다"하는 식으로 체면치레하는 곳이나 가끔 빠지는데
    이런 데 들어가면 그 때야 좋고 지금 3$\sim$5월 봄날 캠퍼스 생활에 싱글벙글이지만 가을이 오면 현실을 체감하죠.
    \vspace{5mm}
\end{enumerate}

봄날의 햇살에 꾸벅꾸벅 졸다가 머리가 빠지지 않았던 과거의 꿈을 꾸었는데
정작 그 때에 저런 등급컷을 보았느냐... 하면 그런 것도 존재하지 않았지만 공부 정말 잘 하는 애들이 저런 걸 신경쓰던가 하는 생각이 불현듯.
냉수 마시고 생각해봄에 저런 걸 보는 친구들은 결국 공부량이 부족하거나 알고리즘이 잘 잡히지 않아서 고민하는 친구들 아닌가.
\vspace{5mm}

최상위권이 되지 않으면 이제는 정말 무의미할 수 있는데, 정작 최상위권이 저런 걸 신경쓰는 건 '초중딩' 때까지라는 게 함정.
왜냐면 정말 전국에서 놀겠다고 하는 친구들이야 선행으로 끝낼 걸 다 끝내든가, 아니면 어린 시절에 영재교육 비슷하게 받아서
선행을 안 하더라도 한번만 듣고 열을 깨우치는 괴물들인데 그들에게 저런 분석이 의미있는가 하는 생각이.
게다가 저런 분석은 어디까지 점수가 나오고 난 다음에야 검토하는 거지,
저런 분석 자체가 \textbf{점수 자체를 올려주지도 못 하는데} 필요한 것이긴 함?
점수 자체를 올리려면 본인들이 죽어라 공부하든가 그게 힘들면 스파르타 시스템 들어가서 본인을 개조하든가 해야지.
\vspace{5mm}

참 쓸데없는 짓들 하는 거죠.
\vspace{5mm}

상담하다가 저에게 질책 들으신 분도 있죠. 가령 실력정석 한권만 보면 된다... 정석, 좋은 책이죠.
그런데 공부 잘 하는 애들이 교재를 줄이려 하는 케이스는 제가 아는 한 단 한번도 없음.
정말 한권만 본다고 하는 경우라고 해도 수십번을 봐서 마스터하는 케이스인데 개인적으로는 저는 이건 비추.
왜냐면 여러권을 보면서 그런 복습효과도 생기지만, 무엇보다 수학교재는 각각 빠진 게 조금씩 있어서 이걸 보충해나가야함
그래서 "양을 줄인다"거나 "특정 강의만 들으면 된다" 이딴 건 없는데 참 신기하게도 이런 게들 있다고 착각하고 물어보고나 있음.
수학교재를 8권 풀어도 내신 1등급 장담 못 하고 특정 강의만 듣고도 인생 조진 애들이 많은데 그런 판타지를 쓴다는 것 자체가.
\vspace{5mm}

그냥 고3 중위권 수준이 쎈 열심히 풀면 격려는 해줍니다. 어차피 이런 친구는 상위권 끝자락이라도 들어가면 희망이 있는 거니까.
그러나 고2가 고3들 푸는 킬러까지 죄다 풀고 풍산자 쎈 마플 급품벨 실력정석 연습문제 껌으로 풀어도 그건 '모자라다'고 얘기합니다.
그거 푸는 친구들이라면 최상위권을 노릴 건데 전국구 최상위권은 경시대회 문제는 기본이고 상상을 초월하는 수준으로 다 끝내놓은데다가
두뇌 수준도 엄청 비상한 것이 아니기 때문에 저것만으로도 부족하죠.
\vspace{5mm}

그런데 이런 후자들에게 입시분석이 의미있을지?
\vspace{5mm}

간략히 말해 다들 참 쓸데없는 밈에 사로잡혀들 있죠.
\vspace{5mm}

기본적으로 저는 평범한 학생들을 위한 과목 테크트리 정리에 관심이 있지만
정말 최상위권이라면 그 때는 인격이 바뀌는데 $-$ 솔직히 지금 수험사이트에 올라온 것들이나
거기서 명사라고 하는 사람들이 정말 최상위권?
\vspace{5mm}

그건 아니라고 생각합니다.
정말 최상위권인인 친구들은 몇마디만 던져보아도 반응이 일반인들과 다릅니다.
말에 가시가 돋혀있다라는 건 기본인데다가 그들의 말 한마디한마디는 상대방의 뇌세포 집적도를 테스트해보려는 게 깔려있죠.
그러니 어느 교재 봐야하나요 무슨 인강이 좋아요하는 질문은 걍 단세포이죠.
최상위권이라면 안 끝내놓은 게 없는데 그런 질문 따위 던지겠습니까.
오히려 그런 친구들이 눈으로만 싹 훑고 문제 푼 걸 가지고 어떻게 풀었냐 제가 질문하면서 논리적 흠결이 없나 확인해보아야하는 구만.
\vspace{5mm}

여담으로 머리좋다는 친구들이 싹 훑고 문제 푸는 건 "초고화질 이미지"로 푸는 것이고
뭘 모르는 사람들이 유전타령하지, 사실 이건 다년간 사교육받고 훈련하면 가능합니다. 이렇게 하는 사람이 적어서 그렇지
(다시 말해서 여기서 유전타령하는 사람들을 전 ㅄ으로 봅니다. 정말 아는 게 없구나 싶어서)
이런 친구들에게 논리 문제 내면 일반인들보다도 더 버벅거리죠. 왜냐면 이미지 접근은 논리에는 안 먹히거든.
\vspace{5mm}






\section{3평 가형문제 평}
\href{https://www.kockoc.com/Apoc/673299}{2016.03.11}

\vspace{5mm}

\begin{enumerate}
    \item $-$ 새로운 출제경향이 안 보임, 기출의 반복 $-$ 그냥 교육청은 이에 대해선 방관하거나 손놓은 듯.
    \item $-$ 눈여겨볼 것이 "함수" 강화야 작년 수능 출제 경향이기도 한데, 이거 시중 사설이나 그걸로 대비되는 건 절대 아닐 건데?
    \item $-$ 돌아보면서 점수대들 보니 올해 고3에 대한 개인적 평가는 그리 틀리지는 않은 것 같습니다.
\end{enumerate}
\vspace{5mm}

교재가 문제가 아니라 학생이 문제야... 라고 하고싶은데 내년에는 그런 이야기는 안 나올 삘이고
(정확히 말하면 현재 고3들은 공부 안 하는 걸 고2들은 공부하고 있다라는 딜레마가)
올해판 마플이나 급품벨 잘 풀었으면 무조건 96 이상은 나왔을 문제들인데 점수 보고를 보면 정말 어지간히들 공부 안 한 듯.
물론 n수생이 점수가 안 나오면 더 심각하겠죠.
\vspace{5mm}

상세문항 분석 올릴까 보았는데 전혀 그럴 가치가 없다고 보여서 걍 넘어갑니다.
문항분석보다도 공부 안 하는 학생들 멘탈 분석을 하고 싶다고 하면 독설치고 지나칠지 모르겠지만.
\vspace{5mm}









\section{실력정석은 막판에 봐야 진가를 발휘}
\href{https://www.kockoc.com/Apoc/675627}{2016.03.13}

\vspace{5mm}

실력정석 1권으로 뭉개고 보자... 라는 케이스가 있는데
다시 말하지만 이 책은 국내 수학책에서는 탑일지 몰라도, 렙이 안 되는 사람이 손잡았다간 기빨려 미라되기 좋은 책이다.
\vspace{5mm}

개념 $-$ 예제 $-$ 유제 $-$ 연습문제
\vspace{5mm}

다소 해설이 구린 것만 빼고보면 문제 하나하나가 일독할 가치가 있다.
특히나 이제는 기출만 무작정 짜깁기하거나 변형하면서 '수학 최고'를 자부하는 사이비가 널리는 시대에 실력정석의 가치는 상대적으로 높아진다.
\vspace{5mm}

날 보고 정석혐오자라고 착각할지 모르는데
애당초 수학교과서를 실력정석으로, 영어교과서를 성문종합영어로 배웠는데 무슨(...)
그런데 그렇기 때문에 정석의 단점이 뭔지도 안다. 이 책은 수학이 어느 정도 반열에 들지 않으면 정말 수학을 극혐하게 만들기 좋다.
\vspace{5mm}

반면 교과서적인 것도 거의 다 마스터하고 시중 문제집에다가 기출까지 다 보고 나서 보면
정석 문제 하나하나가 거의 몇쪽에 달하는 설명을 할 수 있는 떡밥투성이라는 것을 알 수 있다.
그 이야기는 공부가 되어있지 않은 상태에서 정석을 봐보았자 별로 효용이 없다는 것이다.
거기다가 정석이 직접적으로 수능출제 경향에 도움이 된다... 그건 아니다.
정석을 통해서 '수학적 사고력에다가 계산력'을 극강의 엽기적인 수준으로 늘릴 수는 있는 것이다.
\vspace{5mm}

가만 보면 하수일수록 정석부터 잡는 경향이 있다. 그거야 한권으로 끝내겠다는 망상 때문(...)
\vspace{5mm}

실제로 일본의 차트식 수학은 오히려 난이도 안배를 잘 하고 해설을 풍부히 달았다. 이에 가까운 책은 오히려 해법 셀파이다.
교재가 어디서 기원했으며 또한 어떻게 변천했으며 그 원산지에서는 어떤 트렌드인지 모르니까 정석 한권만 보면 되는구나라고 착각하는 것이다.
특히 그 일본 트렌드를 잘 반영한 것이 역설적이지만 교과서이다.
교과서란 형식에 얽매이기 때문에 교과서가 좋은 것을 느끼면서도 그걸 말로 설명 못 하는 건데
당연하지, 그 중에서 일본 책까지 수입해 보면서 평가하는 사람이 몇이나 있겠냐.
\vspace{5mm}

지금 급부상하는 수학교재는 마플 출판사 정도. 기출도 원숙해졌지만 교과서와 시너지도 참신한 시도를 많이 하고 있다.
\vspace{5mm}

아무튼 정석은 맨 나중에 봐도 되는 책이다.
시중교재 검증된 것만 풀고 교과서 마스터하고 기출도 여러번 돌리고 정석을 보면
어떤 걸 패스해도 되는지, 그리고 어떤 문제가 심오한 의미가 있는지 스스로 알 수 있다.
그럼 보는 시간이 단축되는 것인데 왜 처음부터 실력도 안 되면서 정석을 보는 고집을 피우는지 알 수가 없다.
\vspace{5mm}









\section{내신따기 힘들어진 건 사실이죠.}
\href{https://www.kockoc.com/Apoc/675938}{2016.03.14}

\vspace{5mm}

원래대로라면 특목자사고 갔을 애들이 \textbf{내신 노리고 그냥 일반고에 잔류하는 케이스가 증가}했는지라.
그래서 원래대로라면 내신이 정말 잘 나왔어야 하는 친구들이 이상하게 나오지 않아서 왜 그런가 싶었는데.
\vspace{5mm}

특목자사고 간다고 해도 탑을 달리지 않으면 깔아줘야하기 때문에 노력은 노력대로 하고 스트레스만 줄창 받기 딱 좋고
그렇다고 일반고에서 수능에 관한 노하우 같은 것이 부족하지는 않기 때문 $-$ 어차피 다 학원에서 배운다는 분위기이고
\vspace{5mm}

그래서 과거와 같이 굳이 좋은 학교에 가야한다라는 분위기는 가라앉은 것 같습니다.
이 역시 시장의 자정작용(?)이라고 볼 수 있을지도 모르지요.  대신에 일반고 양민들이 내신 따기는 어려워졌다는 부작용이(...)
\vspace{5mm}

다만 교육당국에서 뻘짓을 한 것이 바로 중딩 자유학기제인데 당연히 그 기간에 선행을 하지 뭘 하겠습니까(...)
교육시장의 움직임이라는 걸 모르지 않으면서도 정부는 룰을 개정하는 뻘짓을 하죠.
\vspace{5mm}

아무튼 그래서 고2 올라가는 사람은 결단을 내려야합니다.
가장 좋은 시나리오는 내신도 좋고 수능 공부도 되는 경우입니다. 하지만 몇이나 이걸 걸머쥐려나
그 다음으로는 내신이라도 잘 따놓는 경우입니다. 그런데 이게 수능공부보다 더 힘듭니다.
차악은 내신은 망하더라도 수능 공부라도 끝내놓는 경우입니다. 정시는 매우 힘들다 쳐도 수능으로 만회할 수 없는 건 아니니까.
\vspace{5mm}

하지만 대부분은 내신도 망하고 수능도 망하죠(...)
그래서 1학기 여름방학 때 결단을 참 잘 해줘야합니다.
\vspace{5mm}






\section{올해도 똑같네}
\href{https://www.kockoc.com/Apoc/677337}{2016.03.15}

\vspace{5mm}

하라는 걸 11월에 안 하고 계속 딴짓하다가 3월이 되어서야 이제 어쩌면 좋냐 하는 패턴은 매년 다를 바 없는 듯.
그리고 그 상태로 6월이 된다. 그냥 올해 시험은 물건너간 겁니당.
\vspace{5mm}

마플을 지금부터 풀면 되냐고 하는데
저는 \textbf{고2들에게 풀라고} 합니다. 그래도 \textbf{늦기 때문}입니다요.
얘들 경쟁상대는 그냥 커리 다 밟은 애들이 아닙니다.
양민들이 10줄 풀이써야 푸는 걸 눈으로 정확히 계산해 푸는 괴물들이 진짜 라이벌이라는게 문제죠.
속칭 머리 좋은(정확히 말해 머리가 잘 만들어진) 애들을 상대하려면 빨리 가는 수 밖에 없습니다.
그걸 알기 때문에 콕콕에서도 11월부터 빨리 시작하라 한 것인데
\vspace{5mm}

일지를 보니 그나마 저걸 지킨 사람들은 허덕대면서 자기 목표달성을 30$\%$ 가능성 정도는 할지 모르겠지만
나머지는 글쎄, 모르겠습니다.  아무리 꿀교재, 꿀강의라고 할지라도 여러번 반복하는 것은 못 따라가며
이들에게 필요한 건 '시간'일 터인데 말입니다요.
\vspace{5mm}

아마 4$\sim$5월에 허황된 광고하면서 교재 내는 사람들이 있을 겁니다.
말하지만 본인이 초고수가 아닌 한 그걸 잡는 건 스케줄 망가뜨리며 망하기 딱 좋은 짓입니다(교재검증이야 차치하고서라도)
수학교재는 자기가 다 풀고 오답정리하고 중요한 걸 떠올릴 수 있지 않는 한 그냥 '악성재고', '비계덩어리'입니다.
\vspace{5mm}

초짜이거나 잘 모르는 사람일수록 무슨 꿀교재나 꿀강의 의존하는데 정작 그거 결과 물어보면 '기대보다 별로'라는 데 실망하고,
왜 그딴 걸 믿어서 양치기를 안 했을까라고 후회하는 경우가 태반입니다(그 반대는 사실 한번도 본 적이 없네요)
\vspace{5mm}

이렇게 보면 양민들의 문제는 머리가 안 좋다 그게 아니죠.
스케줄 자체를 잘 못 짭니다. 다시 말해 본인이 수험기업이라고 하면 \textbf{기업경영을 못 하는 셈}이죠.
냉정하게 상황판단을 하고 플랜을 잘 짠 다음에 어떤 과정을 밟은 것인가 잡고 시행해야하는데
남이 이거 좋다더라... 하면 거기에 휘둘리면서 낭비를 합니다.
\vspace{5mm}

이제 들어올 수 있을지 모르나 콕챗에서 콕콕수험고수들이 하는 말들은 서로 약간 다를지 몰라도 공통점은 있는데
최소한 시간 관리 측면에서만큼은 진짜 '보수적'으로 가야한다는 것, 즉 최악 상황을 감안해야한다는 것만큼은 진실이고
수험 최고의 거짓말은 '열심히 공부하겠다'라는 것입니다.
\vspace{5mm}

그런데 올해 시험 물건너가면 사실 내년 시험도 힘들 거라고 보는데
그나마 올해 고3들은 작년 고3에 비해 실력이 낮은 경향이라도 있는데 내년 고3들은 아주 잘할 거라고 보고 있어서 말입니다.
(거기다가 영어 절평, 아주 역대급이 되지 않을까 싶음)
\vspace{5mm}








\section{수험커뮤니티는 자칫하면 사이비 종교화되기 쉽죠.}
\href{https://www.kockoc.com/Apoc/680137}{2016.03.17}

\vspace{5mm}

오프라인에서 수험사이트 아무개나 수험 컨설팅 어쩌구 이야기를 들으면 이렇게 반론해줍니다.
"제가 같이 수업들었던 친구들 절반 이상이 서울대 갔고 어쩌구저쩌구해도 그렇게 교주짓하는 사람들은 없습니다"
"그 친구들 진짜 학벌이나 경력이 정말 객관적으로 대단한가요?"
\vspace{5mm}

이렇게 추궁하다보면 왜 그런 데를 언급하느냐 하면 '말빨이 좋아서'라고 하더군요.
학생이든 학부모든 다 입시공포증에 걸려있습니다. 죽음에 대한 공포가 종교의 기원이라지요.
\vspace{5mm}

냉정하게 말해서 인터넷에 올라온 수험썰들이 도움이 되는 경우는 별로 없습니다.
인터넷을 끊고 그냥 현강으로 학원수업들으면서 시중교재 열심히 파도 입시에 성공할 가능성은 높을지 몰라도
인터넷에서만 언급되는 것을 맹신하면 5년 이상 날리는 것도 일이 아닙니다.
\vspace{5mm}

현상을 예측할 때는 변증법을 주로 사용하면 됩니당.
그런데 이 변증법을 쓸 때의 정반합에서 반(反)은 물질적 불균형에서 찾을 수가 있죠.
과연 수험시장이 정상적으로는 거액을 벌 수 있는 구조인가.
냉정히 생각하면 그럴 수는 없습니다. 그런데 그런 일이 벌어진다는 것은 이 판도 정상이 아니란 얘기죠.
\vspace{5mm}

적지않은 수험생들인 결국 거액을 주고 '재고'를 떠안습니다.
재고라는 것은 소화시키지도 못 하는 참고서를 말하는 것이지요.
재고가 생겼다는 것은 이미 공부를 제대로 못 했다는 이야기입니다. 그럼 수험에서 실패할 가능성이 높지요
\vspace{5mm}

현재 일지를 써보시면서 총회까지 들어오신 분들도 열심히 하는 분들 많습니다.
그런데 느끼셨을 건데, 시중 교재조차도 단기간에 끝내기 힘들다라는 것을 깨달으셨을 것이고
그럼 어떤 교재 보느냐라고 물어보는 게 얼마나 어리석은지 아셨을 것입니다.
\vspace{5mm}

그럼 왜 사람들은 그런 꿀교재물어보는 것에 빠질까요. 그거야 그 때는 수험이 과학이 아니라 \textbf{종교여서 그렇지요}
이 때 그들이 원하는 수험서는 '부적'입니다. 그것만 갖고 있으면 온갖 액운을 피할 수 있고 복을 얻을 수 있다고 믿는 것이지요.
하지만 구매한 다음에 서재에 꽂은 참고서는 풀지 않는 이상 애물단지입니다.
\vspace{5mm}

하지만 재밌는 건 구경하다보면 이 책만 보면 수능 1등급이 나온다라는 식으로 '부적'을 파는 사람들도 많죠.
아울러 자기들 말만 들으면 자녀들이 좋은 대학에 간다고 약파는 '무속인'들도 있습니다.
현실의 무속인조차도 갓 신내림 받은 사람이면 모를까, 2$\sim$3년 지나면 약빨이 다 해서 그 다음부터는 콜드리딩으로 말맞추기하는 현실일텐데.
\vspace{5mm}

더 좋게 말해보았자 그들은 \textbf{수험떴다방} 혹은 \textbf{수험브로커} 정도겠죠.
저는 그 사람들이 수능을 쳐서 정식으로 관악에 들어가는 걸 보여주는 게 가장 확실하다고 생각합니다요.
그런데 그렇지 않고 인터넷 글이건 책에서든 핵심은 피한 채 온갖 미사여구로 인간극장을 찍고 있더군요.
\vspace{5mm}

어차피 저는 그런 걸로 사업하는 건 별 관심도 없기 때문에 그냥 과학적으로 이것만 얘기할 수 있습니다.
\vspace{5mm}

첫째, 남들보다 훨씬 더 정확하게 신속하게 돌아가는 두뇌를 후천적으로 만드는 커리가 필요하다
둘째, 시중의 상위권들조차도 알지 못 하는 특수한 교재는 분명히 존재한다.
\vspace{5mm}

첫째의 경우는 저도 그런 사람들을 만나보았고 같이 공부했으며 겪어보았으니까 말할 수는 있습니다.
다만 선천적인 것보다도 오히려 후천적인 게 더 중요하다고 보이는데 과연 그런 환경들의 엑기스만 뽑아내서 어떻게 구현할 수 있을까.
한가지 분명한 건 여기서 작은 것의 반복이라는 게 키라는 것입니다. 소위 '구몬' 시스템 같은 것
물론 이 이외에도 엄청나게 많습니다. 무엇보다 어린 시절에 어떤 자극을 받았느냐하는 게 상당히 큰 역할을 끼치는데
그걸 2, 30대 양민들에게도 어떻게 적용할 수 있을까 하는 게 제 관심사입니다.
\vspace{5mm}

둘째의 경우는 사실 입증된 적이 있죠. 과거에 과고, 외고에서 집중적으로 설대에 간 적이 있습니다.
지금이야 힘든 일이지만 그 당시 가능했던 이유 중 하나가 그 때는 인터넷 강의가 등장하지 않았고 주요소스를 독점했기 때문입니다.
이게 유의미한 이유는 현재 특목자사고에 가는 친구들도 머리가 상당히 좋은 편인데도 그런 성과를 못 거둔다는 것이지요.
2000년대부터 특목고에서 독점하던 것이 전국적으로 퍼졌습니다. 그래서 지금은 비급이라는 것이 보편화된 것입니다.
농담처럼 한 이야기가 아니라 쎈수학 같은 책이 소수만 보았다면 이건 정말 한권에 100만원 넘어가는 비급 취급받았겠죠.
\vspace{5mm}

비급이 존재하는지 안 하는지 모르겠지만 적어도 수학 29번과 30번, 그리고 과탐 킬러를 시중인강이나 교재로 커버 못 할 수 있다는 것.
그런데 출제자는 그걸 낸다는 점에서는 직접적이든 간접적이든 그런 비급을 만들어낼 수 있다고 할 수도 있을 것입니다.
그런데 재밌는 건 그런 비급은 '출판되면' 더 이상 비급이 아니란 것이죠, 대중적인 인강도 마찬가지입니다.
폐쇄적인 집단에서 소수만이 교육되고 타인에게 전파하지 않는다.... 과거 특목고는 그랬던 적이 있었습니다.
\vspace{5mm}

머리와 비급이라면 이건 수백, 아니 수천만원을 지불해도 아깝지 않을 것입니다. 인생을 바꾸는 거니까.
하지만 시중에 많이 팔리는 것들이 이런 것을 담보한다는 건 거짓말이죠.
마약을 못 구하면 마라톤을 뛰면서 러너스 하이로 천연마약인 도파민 분비를 촉진시키듯이 공부하는 수 밖에 없을지도 모릅니다.
\vspace{5mm}

그런데 정작 사이비 종교들은 저 두가지를 얘기하지 않죠.
그런 것을 알고 있었다면 굳이 호구들을 현혹해서 장사할 필요도 없었겠지만.
\vspace{5mm}







\section{공부에 있어서 회독수}
\href{https://www.kockoc.com/Apoc/682149}{2016.03.18}

\vspace{5mm}
\begin{itemize}
    \item 기본서 $-$ 10회독
    \item 객관식 $-$ 5번씩은 풀 것
    \item 기출 모의 $-$ 3번 정도
\end{itemize}
\vspace{5mm}

평균화시킨 레시피(?)가 저 정도는 됩니다.
학교나 학원에 다니는 열심히 공부하는 친구들도 저 정도는 자기도 모르는 사이에 달성되는 것이죠.
\vspace{5mm}

그런데 인강만 듣는 경우 한번 듣고 복습을 안 하고 끝납니다(피로도가 높아서, 거기다가 딴짓을 해서)
1회독만 하면 90$\%$를 망각하죠. 짧은 시일 내에 다시 복습을 해야 그 망각도를 절반 정도 줄일 수 있습니다.
\vspace{5mm}

강남 모 지역의 학원들은 실적이 좋습니다. 하지만 실질 실적은 갸우뚱할 수 밖에 없는 게, 애시당초  '시험을 쳐서' 우수한 애들을 뽑으니까요.
우수한 애들이란 어린 시절부터 관리가 되어서 머리가 만들어진 애들입니다. 이런 친구는 회독 습관이 들여있어 학습도가 높습니다.
그래서 학원 입장에서도 한번만 가르쳐도 열을 깨우치는 애들이니 가르치기도 편하거니와 성과도 좋은 것이죠.
\vspace{5mm}

반면 회독수가 안 된 친구들을 저런 머리로 만드는 것?
시간과 노력도 많이 걸리지만 성과도 기대만큼은 아닙니다. 그래서 학원에서는 이런 친구들은 걍 버리거나 아니면 은폐시켜버리죠.
그 점에서 인강만큼 편리한 것도 없습니다. 왜냐면 인강을 들어도 효과가 없는 경우는 책임을 질 필요가 없기 때문입니다.
\vspace{5mm}

그럼 과연 10회독이 쉬운 건가.
평균적으로도 그렇고 일지도 그렇지만 고3 기준으로 해도 5월까지 기본교재 2회독만 해도 다행입니다.
물론 초상위권들은 능력이 좋으니 더 많이 돌려 7회독은 해놓으며 시험 직전에 15회독까지 하기도 하죠.
수학문제 풀이 뿐만 아니라 출제가 어떻게 되는지도 그리고 출제자가 어떤 함정을 까는지도 달달 암송할 정도입니다.
하지만 하위권들은 이런 경지를 모르니까 모 선생만 잘 따라가면 된다... 라고 생각하고 계속 우물안을 맴도는 것이죠.
\vspace{5mm}

아무튼 10회독도 해보지 않은 사람들이 머리나 수험을 탓하는 건 아니라고 생각합니다.
그리고 진정 뛰어난 학원은 이런 회독수를 확보해주는 곳입니다.
\vspace{5mm}

제가 쎈이 좋은 교재라고 하는 이유는 A, B 스텝에서 내재적 회독수가 확보되는 편집구조여서입니다.
정석도 비슷할 수 있지만 정석은 난이도 격차나 편집에서 회독수를 달성하기 어렵습니다.
\vspace{5mm}

반면 야매교재들에 대해서 까는 이유는 간단합니다,
자기들의 노트만 짜깁기해놓은 수준이고 어떻게 공부해야할지 저자들도 모르는 게 보여서요.
그 저자들 이력보면 왜 그럴 수 밖에 없는지 이해가 안 가는 게 아니죠.
\vspace{5mm}

수학의 경우 제가 양민들에게 권하는 코스가 (이건 수도없이 반복해서 이제 다들 외우실 겁니다)
기본적으로 쎈(풍산자)를 보고 그 다음 복습용으로 RPM 선택 가능,
그 다음으로 마플이나 마더텅을 보면서 급품벨 중 2권 선택, 그리고 EBS 따라가라는 것입니다.
실모중독자들이나 광신도들은 이걸 가지고 EBS가 안 좋다 깔 것입니다(라고 하지만 참신한 문제는 정작 EBS에 있던데)
이렇게 권하는 것은 저런 과정으로 가면 회독수가 확보되어서 양민들이 고수가 될 수 있는 현재로서 가장 안전한 나선형 코스여서 그렇습니다.
물론 지금은 쎈과 풍산자는 1$\sim$2회독 완료하고 마플을 풀고 있어야하는 시점이지요.
\vspace{5mm}

지극히 평범하고 '실천가능한' 안을 분명 제시했습니다.
물론 이건 여흥이지만 이렇게 해도, 안 되는 친구들은 여전히 수험신비주의에 사로잡혀
아무개 강사를 반드시 들어야 한다, 이상한 야매교재 봐야한다 그러면서 또 올해 1년 공치는 걸 야동 보듯 몰래 훔쳐다보고 있죠.
악취미(?)라고 할지 모르겠지만 제가 쓰는 xx론의 근거는 "실패'입니다. 실패하는 과정의 반대로 가면 성공이 있기 때문.
\vspace{5mm}

어떻게 보면 공부를 하기 싫어하는 그 친구들의 뇌가 무의식적으로 '비현실적인 안'을 일부러 고르는 것일수도 있어요.
수험실패를 일부러 하기 위해서, 공부하기 싫다를 '넌 공부할 수 없어'라고 자포자기하게 미리 까놓는 것이죠.
\vspace{5mm}






\section{회독학습을 오해하시는 것 같은데 간략히 적죠.}
\href{https://www.kockoc.com/Apoc/683063}{2016.03.19}

\vspace{5mm}

아래 글에 댓글로 적었습니다만.
\vspace{5mm}

쎈을 기준으로 한다면
\vspace{5mm}
\begin{itemize}
    \item 1회독 : 개념 읽기 $-$ 개념에서 유도하고 싶은 건 네이버 검색이나 교과서 참조 무방 $-$ A형 풀고 오답정리
    \item 2회독 : B형 대표문제, 그리고 각 유형별로 난이도 中 문제 절반 풀고 오답정리
    \item 3회독 : B형에서 中 문제 나머지 풀 것.
    \item 4회독 : B형에서 上 문제만 풀어나갈 것
    \item 5회독 : C형 절반 풀 것 (짝홀 나눠서)
    \item 6회독 : C형 나머지 풀 것
\end{itemize}
\vspace{5mm}

이렇게 나눠 가시면 6회독 달성이실 텐데. 그리고 그 반복 과정에서 얻어질 것이 충분히 많을 텐데 말입니다.
저렇게 나눠가는게 처음부터 C스텝까지 다 푼다고 하는 것보다 훨씬 쉽고 기억하기에도 좋으며 책을 내 것으로 하는 데 도움이 됩니당.
Divide and Conquer 교범에 충실한 방법이고, 7번 읽기 공부법의 야마구치 마유도 비슷한 이야기를 했죠 아마.
\vspace{5mm}

쎈 한권 기준으로 한다면 1회독은 7일, 2회독은 14일, 3회독은 10일, 4회독은 10일, 5회독은 10일, 6회독 10일 정도로 잡으시면 됩니당.
\vspace{5mm}

이거 아는 줄 알았는데 예상 외로 모른다는 것을 파악해서 요령 알려드립니다. 아니 이런 건 안 배우셨나들 다 $-$$-$
\vspace{5mm}

그럼 마플의 경우는?
\vspace{5mm}

그거야
\vspace{5mm}
\begin{itemize}
    \item 1회독 : 행복한 1등급 제외하고 홀수번만 풀어나걸 것
    \item 2회독 : 행복한 1등급 제외하고 짝수번만 풀어나가기
    \item 3회독 : 행복한 1등급 절반 건드리기 + 경찰대/사관 기출 절반 건드리기
    \item 4회독 : 행복한 1등급과 경찰 사관 기출 아작내기
    \item 5회독 : 행복한 1등급 최고난도 풀어버리기(분량이 별로 없어서리)
\end{itemize}
\vspace{5mm}

이것도 비슷하게 스케줄화하면 됩니당.
그리고 교재 한권만 집중적으로 팔 필요 없음
\vspace{5mm}

가령 쏀과 마플을 본다면
위에 제시한 쎈 1, 2회독 $-$ 마플 1회독 $-$ 쎈 3회독 $-$ 마플 3회독 $-$ 쎈 4회독 $-$ 마플 4회독 $-$ 쎈 5회독 $-$ 마플 5회독 $-$ 쎈 6회독
이런 식으로 조합해나가도 됩니다. 실제로 저도 이런 식으로 하라고 지시하고 있는데 말이지요
즉 공부해나가는 난이도의 함수를 미분가능하게 설계하는 게 묘미임
\vspace{5mm}

그런데 적극적으로 회독수 공부가 뭔지 물어본 사람은 오늘에야 나왔죠(...)
자기들이 적극적으로 물어보거나 핵심적 질문을 안 하고 저보고 '말바꾼다'라고 하면 할 말이 없음.
저렇게 쪼개서 공부하는 건데 그걸 모르셨단 말인가(...)
\vspace{5mm}

그런데 말은 조금씩 바뀔 수 밖에 없는 것도 있음, 왜냐면 상황은 계속 변하니까.
가령 xx가 좋은 문제집이더라 하면 xx은 당연히 추가해야죠. 왜냐면 그래야 다른 수험생들과의 경쟁에서 이기니까.
예컨대 인수(는 보지 않았지만)가 좋은 교재라고 하면 이것도 추가하면 해야하는 것입니다.
그래서 처음부터 완벽한 계획 짜는 건 위험하죠. 시간은 한정되어 있는데 풀 문제집은 생기니까 말입니다.
\vspace{5mm}

그런다고 하더라도 쎈과 마플만 제대로 풀면 4점짜리 대비책 여름에 따로 세울 것 빼고는 망해도 2등급은 뜰 수 밖에 없음.
일지 쓰는 분들도 조사해보니 이 정도까지 한 사람도 거의 없고, 다른 데도 스텔스 잠행해보니 다들 입공부하고 있죠(그러고 망했다고 울겠지)
\vspace{5mm}

다만 이걸 알아두셔야 함
\vspace{5mm}

자기 등급이 올라갈수록 싸워야하는 적은 줄어들지만, 대신 그 적의 수준은 후덜덜하게 높다는 것.
예컨대 100위권에 든다면 100명만 승부하면 되겟죠. 그런데 그 100명이면 정말로 공부로는 일당백하는 괴수들입니다.
일당백이 수사적 의미가 아니라 국수영 문풀이나 사고 흐름으로는 30명 양민을 연결한 것보다 낫다는 겁니다.
\vspace{5mm}





\section{1인자와 2인자}
\href{https://www.kockoc.com/Apoc/684760}{2016.03.20}

\vspace{5mm}

성공은 반추할 수 없지만 도취하기 좋습니다.
실패는 기뻐할 수 없지만 반추할 수 있습니다요.
\vspace{5mm}

젊은 시절 많은 경험을 해보라고 하지만 그게 구체적으로 뭔지 말해주진 않습니다.
개인적으로 그게 뭘까 찾아보고 읽어보고 내린 결론은
"경험"을 통한 깨달음이란, "실패를 수습하고 그로써 교훈을 얻고 자신의 단점을 보강해나간다"는 것입니다.
\vspace{5mm}

실패는 '기출 문제'입니다.
나의 실패는 \textbf{나만이 풀 수 있는 기출문제}입니다.
\vspace{5mm}

첫째, 문제의 존재가 확실하다
둘째, 무엇이 오답인지, 그리고 어떤 것이 정답인지 알 수 있다.
\vspace{5mm}

상담하다보면 누구나 자기만의 기출문제가 있습니다.
상담해주는 의도가 뭐냐라고 하겠지만 다른 게 아닙니다. 타인의 기출문제들을 풀어볼 수 있기 때문입니다.
그런데 똑같이 실패를 겪고도 기출문제를 반복해 푸는 사람이 있는 반면, 외면해버린 채 다른 문제집을 풀려하는 사람이 있습니다.
정확히 말하면 전자는 소수고, 후자는 다수입니다. 시간이 지나면 전자가 성공하고 후자는 풀어야 할 기출문제집이 늘어납니다.
자신의 기출문제를 정복한 사람은 타인의 기출문제도 풀어보며, 아울러 자기와 타인을 위한 예상문제도 만들어냅니다.
그런 예상문제를 미리 만들어 풀어주는 사람이 지도자입니다.
\vspace{5mm}

극단적으로 말해서 문제를 푸는 건 누구라도 할 수 있습니다만, 어떤 문제가 나올지 정확히 예측하는 건 아무나 하는 게 아닙니다.
본인 스스로 가상의 문제를 내보아야할 뿐만 아니라, 운명이란 이름의 출제위원이 내는 의도까지 읽어내야 합니다.
본인이 실제로 매우 평온한 삶을 살아갈 수 있는데도 불구하고 문제들을 가정해보고 해결하려한다면
그 사람은 어떤 문제가 나오더라도 고득점을 거둘 수 있을 것입니다.
\vspace{5mm}

지식이 문제해결을 위한 재료라면, 지혜라는 건 그 문제를 해결해나가는 '실천'을 말하는 것입니다.
지혜는 책만 읽는다고 되는 것이 아니라, 적극적으로 지금 벌어진 문제를 해결해나가고 미래의 문제까지 미리 끌어다 해결해야 늘어납니다.
학교나 학원에서 가르치는 건 오직 정보입니다, 정보 자체로는 아무 것도 할 수가 없습니다.
그 정보를 체계적으로 정리하고 경중을 가리면서 의미를 부여해야 비로소 문제해결에 도움이 되는 지식이 됩니다.
그러나 지식 자체만으로는 문제해결을 할 수 없습니다, 지식이 없어도 문제를 해결하려는 실천적 지성인 '지혜'가 있어야죠.
이 지혜는 본인이 직접 겪어보지 않는 이상은 늘어나지 않습니다.
\vspace{5mm}

수능에서 N수한다 어쩐다고 하겠지만 냉정히 말해서 이 분들은 보고를 놓치고 있죠.
물론 저는 입시에 한해서는 현역으로 좋은 대학 간 것이 진짜라고 봅니다만, 이런 케이스는 고생과 실패를 별로 겪지 않기 때문에
역설적으로 매우 위험해집니다. 다시 말해서 수동적인 지식 축적은 가능해도 적극적인 지혜를 키우는 것까진 힘들다고 생각합니다.
(사실 이런 사람들 인생을 추적해보면 3인자까지는 몰라도, 1인자나 2인자는 찾기 어렵습니다. 우연만은 아니지요)
하지만 1인자나 2인자에 해당하는 사람들은 학벌이 생각 외로 보잘 것 없고(?), 거기다가 실패한 적이 꽤 많습니다.
즉 이건 그들이 시간이 걸리더라도 자신들의 기출문제를 정리했으며, 지식은 비천할지 모르나 지혜가 쌓였다는 걸 말하는 것입니다.
\vspace{5mm}

공부를 잘 해보았자 소용없다라는 건 물론 공부를 안 한 사람들의 공격일 수 있지만
선해해본다면 '공부를 잘 해보았자 우두머리, 즉 리더가 되지 못 하면 무의미하단' 이야기입니다.
(돈을 많이 버는 것도 우두머리입니다. 부하가 되어보았자 벌지 못 합니다)
그런데 우두머리의 조건은?
남들이 쉬쉬하고 기피하는 문제를 본인이 해결하고 주도하며 남의 문제를 해결해주려는 것입니다.
우리는 그런 사람들에게 고개를 숙이게 됩니다.
자기 목숨까지 걸고 리스크를 감수하면서 부딪쳐가는 사람이 1인자고,
자기 목숨까지는 걸지 않되 지혜를 제공하며 보좌하는 게 2인자입니다.
\vspace{5mm}





\section{몇가지 질문에 대한 답변}
\href{https://www.kockoc.com/Apoc/685888}{2016.03.20}

\vspace{5mm}
\begin{enumerate}
    \item \textbf{작년 수능이 기대보다 망했는데 또 도전해야겠느냐.}
    \vspace{5mm}

    다른 것을 떠나서 실패를 했다면 왜 실패했나 그 원인부터 파악해야 합니다.
    과학적인 분석이 없는 수험은 곧 종교행위가 됩니다. 그게 실패든 성공이든 종교가 되어버리면 그 다음부터는 답이 없습니다.
    운좋게 성공해놓고 왜 성공했는지 이유를 모르면 어차피 곧 망합니다.
    실패했지만 그 원인을 철저히 분석해서 개선하면 그 다음에는 실패할 수가 없습니다.
    \vspace{5mm}

    우선 이것부터 해놓으면 도전해야 할지 말아야 할지가 보입니다.
    실패를 좌우하는 건 상당히 '사소한 원인'에서 비롯됩니다.
    그런데 그 사소한 원인은 알고보면 '빙산의 일각'이죠. 그걸 분석해보면 자기가 인식 못 했던 거대한 문제가 발견됩니다.
    그 문제를 잡아내어야 합니다.
    \vspace{5mm}

    재수가 삼수 사수로 이어지는 건 그 문제가 해결되지 않아서입니다. 우리는 그 미지의 대상을 모두 '운'으로 돌리죠.
    운이라는 것도 우주적 차원에서는 \textbf{필연}이 됩니당.
    \vspace{5mm}

    \item \textbf{영어 처음부터 단어 외워야하느냐}
    \vspace{5mm}

    이것도 고정관념이 있는 것 같은데 그냥 말씀드립니다.
    마더텅이나 자이 같은 기출 가지고 그냥 어려운 지문부터 풀고, 답 낸 이유 적고, 틀린 다음 해설과 비교하고 해설 자세히 보세요.
    즉 영어는 어려운 지문부터 걍 공략하고 까이고 극복하길 바랍니다(이건 국어도 마찬가지입니다)
    \vspace{5mm}

    수학은 쉬운 것부터 회독수를 늘리는 게 좋습니다. 왜냐면 수학의 진정한 실력은 기초에서 나오기 때문입니다.
    수학의 기초 개념이야말로 사실은 '고차원 수준의 보물'들입니다. 그렇기 때문에 그 기초들을 익혀야하는 것입니다.
    어려운 문제를 $\sim$ 하게 풀면 되죠라는 식으로 인기 강사가 아무리 지껄여보았자 본인이 확실히 기억하고 있는 교과서 개념만도 못 합니다.
    \vspace{5mm}

    하지만 영어나 국어는 다릅니다. 영어나 국어의 사고법은 어려운 지문을 잘 해석하고 그 문제를 푸는 것이 오히려 기본일 수도 있습니다.
    수능에서는 자잘한 걸 요구하기보다도 실제로 학생이 어려운 지문을 잘 분석하고 분해해보느냐를 따져보길 때문입니다.
    이런 건 본인들이 어려운 지문에 도전해보는 걸로 늘어납니다.
    \vspace{5mm}

    다시 말해 수학이 찰흑을 빚어 뭔가 만드는 것이라면, 영어나 국어는 거대한 대리석을 깎아나가는 조각과 같습니다.
    엄선된 글들을 여러번 읽으면서 거기에 내재되어있는 사고법들을 익히고 이것들을 독해하는 걸 훈련하는 역삼각형 방법으로 가는 게 낫습니다.
    어떤 식으로 푸느냐보다도, 각자의 '독법'이라는 걸 만들어야 합니다.
    국어나 영어는 절대 객관적일 수가 없습니다. 원래 주관적인 논리에서 주관성을 최대한 배제해 객관성에 근사시킨 것일 뿐,
    그것들이 객관적일 수는 없는 것입니다. 그러므로 국어나 영어의 어려운 지문을 읽고 풀어가는 것은 표준화된 방법이 나오기 힘듭니다.
    글을 읽는 건 독자 자신의 철학과 성격에 종속적입니다.
    \vspace{5mm}

    덧붙여 말하면 문과 분야의 객관성이란, 주관적인 것들이 서로 충돌하고 갈등을 빚다가 나오는 타협적인 것에 불과하지
    애초에 객관적인 것이란 존재할 수 없습니다.
    형식논리학으로 참 거짓을 실제로 분별하는 건 어렵습니다. 애초에 세상이 연역논리로 설명되기는 힘들기 때문이죠.
    어떤 주장이 참이다 거짓이다보다는, '더 옳은 점이 많다'라는 개연성, 타당성으로 가는 게 현실입니다.
    실제 수능 국어나 영어의 독해는 그 개연성과 타당성을 전제로 한 조건부 확률적 문제를 내고 있죠.
    \vspace{5mm}

    \item \textbf{시험 망하면 어쩌냐}
    \vspace{5mm}

    평생 잘 먹고 잘 살 줄 알았던 4, 50대 아재들조차 모가지 잘리는 그런 세상입니다.
    젊었을 때 실패해본 적이 없기 때문에 우물쭈물하다가 거액의 빚을 지고 한순간에 망해버리는 케이스가 적지 않네요.
    승부를 했다고 이길 수는 없습니다. 그런데 본인이 탁월한 투자가라면, 이기는 방법보다는 '지더라도 안 망하는 방법'을 궁리하겠죠.
    다시 말해 공격력과 방어력 중 선택하라면 방어력이고, 딜러와 탱커 중 택해라면 탱커가 낫다는 것입니다.
    \vspace{5mm}
\end{enumerate}
만약 본인이 특정 시험에 모든 것을 걸었다면 그것 자체를 걍 바꾸시길 바랍니다.
수능을 쳐도 안 되면 원칙적으로는 다른 길로 갈 수 있도록 하는 것도 나쁘지 않습니다.
다만 그 다른 길에 대한 의식이 수능까지 말아먹는 거라면 이건 제3의 안을 택해야겠지만요.
\vspace{5mm}

최근 헌책방에서 정말 엄청난 책을 구했습니다. 시세 잘 쳐주면 30만원짜리일 건데 5000원에 구했죠.
(그 저자는 당시는 평범한 엔지니어였지만 지금은...)
그런데 거기 명문이 적혀있더군요. 아무 것도 모를 때는 세상이 모순 덩어리인 것 같지만
알고보면 자기가 잘못 알고 있어서 착각하는 경우가 많다고.
\vspace{5mm}

수능에 응시하는 것도 마찬가지입니다. 실제로는 수백만가지의 길이 있는데 우리는 의도적으로 하나만 보고 있는 거죠.
\vspace{5mm}






\section{진도}
\href{https://www.kockoc.com/Apoc/692514}{2016.03.24}

\vspace{5mm}

아부하기 싫어서 적는다면 지금 상당수가 '늦습'니다.
일단 이 시점에 뭘 봐야하느냐 물어본다면 이미 상당히 밀린 것입니다요.
이제 3월말인데 기출도 절반 이상은 다 돌렸어야합니다. 즉, 쎈이든 풍산자든 하나는 끝냈어야한다는 얘기죠.
\vspace{5mm}

상당수가 자기들이 교재를 잘못 선택해서 혹은 강의를 안 들어서 아니면 잘못 들어서라고 착각할 건데
이건 무의식적으로 교재나 강의 탓을 하는 '책임전가'입니다.
실패 요인은 \textbf{'진도'를 일찍 빼지 못 해}서입니다.
즉, 스케줄을 못 맞추었기 때문입니다.
\vspace{5mm}

거의 다 1, 2월까지는 시간이 꽤 많을 거라고 착각에 빠지면서 설렁설렁하죠.
그런데 남은 시간은 자기가 생각하는 것의 1/3도 되지 않고, 해야할 건 자기가 생각하는 것의 3배를 넘어갑니다요.
이게 맞는지 틀린지야 본인들이 경험해보시면 됩니다.
작년 일지 사례와 콕콕 입시 성과로 보건대 성적이 좋은 쪽은 정말 해야할 것을 남들보다 다 빨리 빼고 일찍일찍 돌린 경우입니다.
소위 수험사이트나 야매교재업자들이 강조하는 방법으로 성과가 좋은 건 거의 못 보았습니다요(좋으면 제가 그걸 추천했죠)
\vspace{5mm}

뭔 교재를 보아야하느냐가 아니라, 자신의 속도와 밀도를 높이는 데 신경쓰시길 바랍니다.
자기가 풀지 않은 교재는 그냥 재고입니다. 뭐든 일단 끝까지 풀면서 뇌를 단련시키는 과정입니다요.
\vspace{5mm}

진도가 밀리지 않은 사람과 그렇지 않은 사람들과 얘기해보거나 글을 읽어보면 '마인드'도 확실히 차이가 납니다.
전자는 자기 할 게 바빠서 사소한 질문은 하지 않습니다. 오직 힘들어죽겠다 그 한마디입니다.
후자는 여전히 뭔 교재가 좋아요라는 질문을 병적으로 던집니다. 그리고 힘들다라는 말을 하지 않아요,
그러다가 6월 넘어가면서 또 포기각 나오겠죠.
\vspace{5mm}









\section{뭔 교재 보았냐 물어보지 말고}
\href{https://www.kockoc.com/Apoc/694072}{2016.03.25}

\vspace{5mm}
\begin{enumerate}
    \item 어떻게 실패했냐
    \item 어떤 환경이었느냐
\end{enumerate}
\vspace{5mm}

이 2가지만 보시면 됩니당.
\vspace{5mm}

SKY 붙거나 의대에 간 사람들은 겸손을 가장해 자기가 머리가 좋아서... 라고 하겠지만 그건 개뻥이고
대부분은 천부적 환경이 좋거나, 아니면 본인이 노력해서 환경을 잘 만들어서 기존의 나쁜 환경을 극복한 케이스입니다.
당연히 '실패하는' 코스를 피해간 것이죠
성공하는 방법을 모르면 실패하는 방법을 제대로 알면 됩니다. 그리고 그 반대로 가면 일단 망하지는 않으니까요.
아울러 사람을 규정하는 건 환경입니다. 환경이 의식과 행동에 영향을 주기 때문이죠.
\vspace{5mm}

동양에서는 일을 도모할 때 따져보는 것으로 \textbf{천시지리인화}가 있습니다.
요즘 와서는 천시 = x수저로 바뀌는 것 같고  인화라는 것은 어떤 네트워크에 들어가느냐 하는 걸로 결정되는 걸 본다면
역설적으로 남은 것은 지리인데, 이 지리라는 것이 실제로는 우리가 의지대로 바꿀 수 있는 것일지도 모릅니다.
시간의 가치도 자기가 어디서 사느냐 혹은 어디서 일하느냐에 따라 달라집니다.
같은 시간이지만 번화가 마천루 로얄층에서 보내는 것과 어디 달동네 폐가에서 보내는 게 똑같을 수는 없죠.
그런 점에서는 천시조차도 지리에 지배되는 감이 없지 않으며 또한 사람들도 번화가에 몰리는 것을 보면 인화 역시 마찬가지입니당.
\vspace{5mm}

뒤늦은 감이 없지 않지만 올해 수험 시작하는 사람 중에서 현명한 사람이라면 어디서 공부하느냐 하는 걸 먼저 물어보았고 실천했겠죠.
집에서 공부하지말라하는 것도 마찬가지입니다. 공부가 잘 되는 사람이 질문할 리는 당연 없겠고
보통 안 되어서 질문할 건데, 그럼 그건 '집의 영향'도 만만치 않기 때문입니다.
\vspace{5mm}






\section{집착}
\href{https://www.kockoc.com/Apoc/694074}{2016.03.25}

\vspace{5mm}
\begin{itemize}
    \item 최상위권과 상위권 : 3년
    \item 상위권과  중위권 : 3년
    \item 중위권과  하위권 : 3년
\end{itemize}
\vspace{5mm}

너무 그럴싸하게 근사한 것 같은데 대충 이 정도의 격차가 발생합니다.
그럼 왜 저런 격차가 생기느냐 하는 건 "환경" 차이가 가장 크지만, 누가 먼저 일찍 공부했느냐도 중요하겠죠.
초딩 때 고1 수학까지 건드렸다면 남들보다 6년 앞서간 겁니다. 무리하지 않게 공부하면 당연히 최상위권이 될 수 밖에 없습니다.
\vspace{5mm}

그런데 비극은 '공부해야하는구나'라는 걸 너무나 뒤늦게 깨닫고 있는데다
그걸 깨달았을 때에는 공부에 방해가 되는 요소들이 많아졌다는 것이죠.
거기다가 마음은 급하지 그래서 올해 당장 합격하고 싶지.... 하지만 현실적으로 그게 가능하지가 않죠.
공부 못 하는 친구들의 특징인 집착하는 게 너무 많다는 겁니다.
과거에도 집착하고 현재 자기가 갖고 있는 교재나 질러버린 인강에도 집착하고.
\vspace{5mm}

집착한다는 것부터가 이미 행동양식이 비효율적이라는 얘기죠.
사고와 행동이 효율적인 사람은 미래에 방해가 될 것은 과감하게 정리해버립니다. 이것들의 효용이 적다는 걸 알고 있어서입니다.
자기 교재나 인강에 지른 돈이 많더라도 그게 도움이 안 된다고 합리적인 확신이 든다면 과감히 버립니다.
이 사람들이 신경쓰는 건 경쟁자와의 승부죠.
\vspace{5mm}

상위권들은 집착을 해도 미래의 순위에 집착합니다. 그래서 낭비 요인이 없으니 더 쭉쭉 나가는 것이죠.
계속 그 자리를 지켰으니 트라우마 될 것도 없고 더 좋은 방식이 있다고 보면 과감히 옮길 수도 있습니다.
전진해왔기 때문에 밑천 건져야한다 그런 생각도 덜 합니다.
\vspace{5mm}

그래서 제가 업자들이 문제라고 보는데. 다수의 중하위권들이 집착을 하는 요인 중 하나가 업자광고에 세뇌된 게 있어서입니다.
xxx 강의나 xxx 교재 보면 안 되나요... 공부 안 한 친구들이 저런 강의나 교재가 좋다고 말하는 '배경'이 무엇이겠어요?
그런 걸로 돈 번 인간들은 지옥에 가지 않을까 싶습니다.
\vspace{5mm}









\section{교과외적(?)인 증명과정이 필요없는 게 아닐텐데}
\href{https://www.kockoc.com/Apoc/706283}{2016.04.01}

\vspace{5mm}

수학을 열심히(?) 공부해도 성적이 안 나오는 이유에 대해서 나름대로 가설을 세워보고 찾아본 결과는
수학은 다른 과목과 달리 \textbf{공부한 문제가 그대로 시험에 출제되지는 않는다}는 것입니다.
다시 말해 A라는 문제를 연습했다고 치면 시험에 나오는 것은 A가 아니라 A'' 혹은 A''B'''라는 것입죠.
학생 중에서 난감한 경우가 "무조건 암기하라", "패턴을 외우라" 하면서 이 문제는 이렇게 푼다라고만 암기하도록 배운 케이스인데,
이렇게 푸는 방식을 외우는 케이스가 점수가 잘 나오는 경우는 매우 드뭅니다, 사실 정상적이라면 나올 수가 없어요.
\vspace{5mm}

암기할 대상은 '정의와 성질' 정도면 족합니다. 그리고 그 암기도 문제를 푸는 과정에서 저절로 이뤄지는 것이죠.
\vspace{5mm}

그럼 수학공부를 왜 하느냐라고 하면
생각하는 방식을 연습하기 위해서입니다, 수학공부의 잇점은 딱 하나. "예술적으로 사고한다" 그 정도입니다.
잠시 딴 이야기하자면 공대에 가서라도 그리고 대기업에서도 유감스러우나 수학을 쓸 일은 그리 많지 않습니다(공업수학이 수학?)
수학이 정말 중요하다라고 하는 건 어디까지나 그 전공자나 교육자들이나 하는 이야기죠.
\vspace{5mm}

일본의 전설적인 로켓공학자인 이토가와 히데오씨가 자신의 책에서 아래와 같이 얘기한 적이 있죠.
이토가와 히데오는 전투기 설계로 유명했고(하야부사), 그리고 펜슬로켓 아이디어도 대박친 분입니다.
그의 이름이 소행성에도 붙을 정도죠(https://namu.wiki/w/$\%$EC$\%$9D$\%$B4$\%$ED$\%$86$\%$A0$\%$EC$\%$B9$\%$B4$\%$EC$\%$99$\%$80)
공대 진학하셔서 뭔가 뚝딱 만드실 분이라면 이 분이 쓴 책은 읽어보시길 권하며.
\vspace{5mm}

나는 학교를 졸업하고 비해기 회사에서 10년 간 전투기와 폭격기의 설계도를 그렸지만 피타고라스의 정리를 쓴 적은 단 한번도 없다. 또 그 후에는 음향항과 의학기기를 10년 했는데, 이 사이에도 한번도 없다. 그리고 로켓 개발 10년 사이에도 없엇다. 또한 조직공학 연구소에서 10수년 있는 동안 전혀 한 번도 없다. 그것 뿐인가, 초등학교에서 배운 산수, 중학교의 대수, 기하 고등학교의 미분 적분에서부터 대학의 수학까지 현재까지의 일에서 실제로 사용한 경험이 없다.
한번도 없다는 것은 조금 극단적이고, 2번은 있다. 한 번은 공학 박사학위 논문이었는데 '수학을 넣지 않으면 학위를 받을 수 없다'고 해서이고, 또 한 번은 대학에서 강의를 할 때 '수학을 넣지 않으면 학생들을 바보로 만드니까'가 이유였던 2가지 기회 뿐이다.
선입견으로 비행기 설계에서는 수학을 사용할 것이라고 말하지만, 비행기 설계의 좋고 나쁨은 그 비행기에 타는 파일럿에게 좋은지 아닌지가 최대의 포인트이다. 조종사들이 어떤 비행기를 조종하고 싶어하는지 그걸 발견하는지 발견하는 것이 설계의 가장 큰 뜻이고, 수학 등을 억지로 사용해도 훌륭한 비행기는 생길 수 없는 것이다. 따라서 왜 그렇게 몇천 시간이라는 시간에 걸쳐 수학을 공부해야하는 것인가 이해하기 어렵다.
(중략)
수학을 하면 머리가 좋아진다는 것이다. 이것은 일을 절차대로 생각하게된다는 얘기인데 찬성할 수 없다. 경험상으로 대학교수회의 때 수학과 교수의 의논이 가장 사리에 맞는 이야기였다고 생각할 수 없기 때문이다. 만약 만에 하나 수학을 하여 머리가 좋아진다면 미국의 대통령도 수학을 전공하는 사람이 되어야하지 않을까.
(중략)
수학은 과학이 아니다. 기술도 아니다. 오히려 철학 분야에 들어가야 하는 것으로 뛰어난 수학자의 업적을 보면 수학이라는 무기질보다도 예술적인 향기조차 풍기는 느낌이 든다. 산수, 수학을 공부하는 것은 절대 불가결하게 필요한 단계를 하나하나 올라가는 과정을 가르키기 위한 것이다. 예를 들어 3+6은 얼마인가하면 양손의 손가락을 이용해 9를 답한다. 그런데 다음으로 56이면 양손의 손가락으로는 부족해진다. 아이는 !?라고 하다가 발가락이 있다는 것을 곧 알 수 있다. 이 \textbf{'오를 수 있다', '됐다'}는 것이 상당히 중요하다. 할 수 있으면 기쁘고 자신감도 생긴다. 또한 자신감이 생기면 의욕이 생긴다. 즉, 어떤 단계라도 하면 된다.
오른다는 적극적인 성격의 인간을 만들기 위해 수학이 있는 것이다. 말하자면 머리를 좋게 하기 위한 방편이다.
\vspace{5mm}

인용하다보니 수학공부의 의의까지 넣었는데 저는 저 생각에 동의합니다.
역설적인 이야기지만 수학공부를 많이 해야 들어갈 수 있는 학과에서는 정작 수학을 쓰는 일은 별로 없습니다.
그러면 수학을 왜 크게 반영하느냐. 그건 위에 인용한 대로입니다. 수학 자체가 좋은 머리를 보장하는 건 아니나
수학공부가 \textbf{좋은 머리를 만들기 위한, 일종의 계단식 상승의 방편}이 될 수 있기 때문입니다.
공부하면서 좋은 머리로 상승해 온 학생이 아니면 어려운 수학문제는 풀 수가 없습니다.
수능에서 수학을 제외한 나머지 과목은 미친 듯이 암기시켜서 고득점을 보장받을 수 있습니다.
그러나 수학만큼은 그게 먹히지 않습니다. 그래서 정말 '좋은 머리로 성장해 온 학생'인지 확인할 수 있고,
아울러 입시의 공정성도 일정 부분 보장할 수 있다고 할수 있습니다.
\vspace{5mm}

수학공부를 열심히 하는 건 그 문제가 그대로 출제되어서가 아니라,
그런 문제들을 풀면서 \textbf{'머리'를 만들어가기 위해서}입니다.
3+6을 손가락으로 세다가 5+6을 제시받은 아이는 발가락을 쓰겠죠.
그런데 그 다음 30+14라고 하면 엄마아빠형누나친구의 손가락발가락을 다 동원하다가
그것들을 숫자로 추상화할 수 있다는 걸 알게 되면 비약적인 발전을 하게 되는 것입니다.
\vspace{5mm}

물론 이런 식의 문제는 수학에만 있는 건 아닙니다. 국어에도 영어에도 과학에도 있습니다.
다만 수학이 그 점에서는 더 적합하다... 정도이죠. 그런데 바로 이게 학습자의 머리를 좋게 하는 것입니다.
\vspace{5mm}

그런데 입시에 최적화한다고 이런 '계단식 상승'을 제외하고 나올 것만 공부한다.... 당장 문제는 맞춥니다.
그러나 분명 벽에 부딪쳐서 못 뛰어넘습니다. 문제를 어떻게 푸는지 그 방법을 알지, 머리가 좋아진 게 아니기 때문입니다.
자기가 모르는 게 조금만 섞여도 거기서 생각을 하지 못 합니다.
수학문제를 잘 푸는 친구들은 자기가 모르거나 난해한 문제도 스텝바이스텝으로 추리하면서 실마리를 찾아갈 수 있는 사고를 할 줄 압니다.
\vspace{5mm}

입시에 나오지 않는다고 어려운(?) 증명을 할 필요가 없지 않느냐는 건 이걸 모르는 것이죠.
물론 과한 증명을 할 필요는 없다.. 가 아니라 사실 그걸 가르쳐줄 선생도 없을 것입니다만
교과서에 나온 공식이나 성질이 어디서 비롯되었느냐하는 증명을 해보는 건 필요합니다. 그 과정에서 머리가 좋아지기 때문입니다.
수능 수학은 그 정도가 아니기 때문에 그렇게 할 필요가 없다.... 그게 어디 정해지기라도 했답니까.
\vspace{5mm}






\section{잔인한 4월}
\href{https://www.kockoc.com/Apoc/706818}{2016.04.01}

\vspace{5mm}

11월부터 공부해 온 사람들은 이제 지쳐서 공부하기 싫은데도 관성 때문에 공부하고 있을 때임.
이런 분들은 최대한 버티다가 불꽃놀이 시즌 막판에 좀 쉬고, \textbf{수면}시간을 평소보다 늘려주시고(6시간이면 7시간 잔다거나)
그래서 5월 초까지 설렁설렁하셔도 됩니다. 사실 그래야합니다, 5월은 어차피 6평 때문에 긴장 바짝할 터인데 그 체력과 정신력 비축해야하는지라
\vspace{5mm}

보통은 수면시간을 줄이라하는데 왜 늘리냐 $-$ 거기 답변은 간단합니다.
공부 외 활동에서 우리 정신 건강/육체 건강에 도움이 되는 건 사실 수면 뿐입니다.
잠을 개운하게 잘 자고 나면 피로와 스트레스가 풀려서 공부를 기분좋게 할 수 있습니다.
그 수면시간을 줄여서 자기학대를 하는 건 멍청한 짓이죠. 멍한 상태에서 공부할 테니까.
물론 가을이 되면 수면시간을 저절로 줄이게 됩니다. 그러니까 봄날에 괜히 피로 쌓아두지마시라는 이야기.
\vspace{5mm}

그리고 공부를 2개월 이상 한 사람은 독학하는 사람은 학원 다니고 싶어할 테고, 학원 다니던 사람은 독학하고 싶어할 것인데
그게 정말 진지하게 공부방법을 반성하는 것인지, 아니면 지금 공부하는 게 힘들어서 뇌에서 핑계대는 것인지 확인해보시길 바랍니다.
물론 독학으로 공부를 했는데 공부시간이나 학습량이 나오지 않았다면 진지하게 학원으로 갈아타야합니다.
다만 "학습량"이 많이 나오는데도 그걸로 스트레스받아서 공부하기 싫다 힘들어 그만두고 싶어라는 메시지가 주인을 기만하는 경우가 있습니다.
\vspace{5mm}









\section{개정수학이 개정 전 수학과 다른 것.}
\href{https://www.kockoc.com/Apoc/708462}{2016.04.02}

\vspace{5mm}

궁극적으로는 경쟁을 더 가중시킬 듯.
\vspace{5mm}

개정 전 수학은 고 1 때부터 삼각함수와 순열, 조합을 박아넣어 그 때부터 다수의 수포자를 양산할 수 있던 데다가
행렬, 지수로그, 순열, 방정식과 부등식, 함수의 극한과 연속, 미분 곳곳이 초심자에게는 대단히 진입하기 어려운 구조였는데
\vspace{5mm}

현행 과정은 삼각함수와 지수로그함수를 미적분 2로 밀어넣어버렸고
초월함수나 로그함수 몰라도 일단 미적분1까지 끝낼 수 있게 해놓은 구조라서 수포자 양산이라는 비극을 초래하진 않을 듯.
다시 말해서 이전 과정은 원래 수학을 잘 할 수 있는데도 교과 과정에 치여서 중도포기하는 비극도 없지 않았으나
현재는 그렇지 않다는 것이죠.
\vspace{5mm}

그런데 이게 역설적으로는 경쟁을 더 빡세게 할 것입니다.
첫째로는 시험 출제는 어차피 경쟁이 좌우하는 거라서 교과 과정이 쉬워지는 것과 별 상관없다는 것.
둘째로는 수포자들이 줄어들기 때문에 경쟁이 심화되고 평가원으로서는 불수능 출제를 해도 무리없다는 것.
\vspace{5mm}

교과과정이 쉬워졌으니까 만만할 것이다... 라고 하면 없는 코를 빌려와서 다치게 생겼다는 것.
\vspace{5mm}

특히 이번 과정이 인상깊은 게 \textbf{'선행'이 무의미}합니다. 공부하거나 연구해보신 분은 알 것임.
수1, 수2를 제대로 하지 않고는 미적분 1을 못 하게 해놓았고, 미적분 1을 못 하면 미적분 2도 힘들고, 미적분 2를 못 하면 기벡도 못 하는 구조임.
선행한다고 진도 빨리 나아가보았자 별 실익이 없음. 고2 이거나 고2 올라갈 사랑이라면 학교 진도에 맞춰가면서 고난도 문제 푸는 게 바람직.
\vspace{5mm}

반면에 이전 과정은 선행을 하지 않으면 수포자되기 참 좋은 구조였죠. 교과과정 자체가 지나치게 눈높이가 높았음.
그래서 재능이 없어도 선행해서 감을 잡으면 이득을 보았고, 재능이 있어도 선행을 안 하면 교과과정에 치여 수포자되기 좋았는데
그런 부조리(?)한 일이 현행 과정에서 벌어지진 않을 것 같습니다.
\vspace{5mm}








\section{양민들을 위한 수학교재테크트리}
\href{https://www.kockoc.com/Apoc/717659}{2016.04.08}

\vspace{5mm}

초심단계
\vspace{5mm}

\begin{enumerate}
    \item 쎈수학 A형 + 수력충전이나 연개수문(선택사항)
    \item 쎈수학 B형 下, 中
    \item RPM (고난도 문제 제외)
    \item 마플(마더텅 자이도 무방) 중난이도 문제
    \item 쎈수학 B형 上
    \item RPM 전부 다 풀 것 $-$ \textbf{RPM 끝}

\vspace{5mm}

중간단계
\vspace{5mm}

    \item  일품수학 개념만 정리
    \item  풍산자 필수유형 고난도 빼고 풀 것
    \item  마플(마더텅 자이) 최고난이도 문제 빼고 4점짜리 절반 풀기
    \item  일품수학 1등급과 수능문제 절반 풀기
    \item  풍산자 필수유형 고난도 문제 절반 풀기
    \item  EBS 올림포스 고난이도 빼고 다 풀 것.
    \item  일품수학 1등급과 수능문제 다 풀기 $-$ \textbf{일품 완료}
    \item  풍산자 필수유형 고난도 문제 다 풀 것 $-$ \textbf{풍필유 완료}
    \item  EBS 올림포스 다 풀 것 $-$ \textbf{올림포스 완료}
\vspace{5mm}

고수단계

\vspace{5mm}

    \item 수학의 바이블 고난이도 문제 빼고 정리 $-$ 단, 이미 아는 문제는 스킵하고 읽어도 됨
    \item 실력 정석 : 예제와 유제는 그냥 읽고 기본 연습문제 다 풀 것
    \item 블랙라벨 스텝 1 풀기
    \item 마플 최고난도 문제 풀기 $-$ \textbf{마플 완료}
    \item \textbf{수학의 바이블} 고난이도 문제 다 풀 것 $-$ \textbf{수학의 바이블} \textbf{완료}
    \item 실력정석 실력문제 : 절반 정도 시도해볼 것. 단, 모르는 경우 별표치고 답지보고 정리
    \item 블랙라벨 스텝 2 절반 풀기
    \item 그동안 소박맞은 쎈수학 C 스텝 절반 풀기
    \item 실력정석 실력문제 : 나머지 절반도 시도, 역시 모르는 문제는 별표치고 답지보고 정리 $-$ \textbf{실력정석 형식적 완료}
    \item 블랙라벨 스텝 2 나머지 절반 풀이
    \item 쎈수학 C 스텝 나머지 절반 풀기 $-$ \textbf{쎈수학 완료}
    \item 블랙라벨 다 풀기 $-$ \textbf{블랙라벨 완료}
    
    고수단계까지 완료 후, 자기가 틀리거나 별표친 문제는 다시 풀이 읽고 정리해 볼 것.
    위와 같은 스텝은 1단원마다 해도 되고, 혹은 3$\sim$4단원별로 해도 됨.
    가령 미분을 다룬다면 미분계수만 저렇게 돌려도 되고, 아니면 미분 전체로 다 돌려도 좋음.
    \vspace{5mm}
    
    그 이후
    \vspace{5mm}
        
    중간고수 점검단계
    \vspace{5mm}

    \item 교과서 구해서 풀어볼 것(선택적 : 안 해도 무방)
    \item 기본 개념과 원리 증명해볼 것 : 미적분학의 기본정리, 지수의 확장 밑의 축소, 이항분포 공식 증명 등
    \item 일등급 수학 그냥 쫙 풀기 : 일등급 수학이 가장 어려워서가 아니라, 난이도 배분을 잘못해놓은 게 있어서 이 때 푸는 게 걍 유리함.
    \vspace{5mm}

    최강고수 단계
    \vspace{5mm}

    \item[-] 이스터에그 :  콕콕의 모쏠 아무개를 납치해 자료 내놓으라고 한 뒤 그걸 풀 것
    \item 고1 수학이 제대로 안 된 경우 위와 같은 과정으로 고1 수학 정리할 것 : 단, 4월부터는 권하고 싶지는 않음.
    \item 수리논술기출 구해서 풀어볼 것 ; 정답에 집착하지 말고 일단 푸는 훈련을 하는 게 좋음.
    \item 과거 본고사 문제 구해서 풀어볼 것 ; 구하기 어려우면 천일수학을 구매한 뒤 도전.
    \vspace{5mm}

\end{enumerate}
그런데 다수가 고수단계까지도 가지도 않고 그냐 실모 봐야하느니 강의 들어야하느니 그러고 있다는 게 함정.
어떻게 공부해야하지 말고 저기 적힌 단계별로 고수단계까지 다 완료해놓고 오셈.
고수단계까지 완료하고 나면 \textbf{질문을 하는 입장이 아니라 받는 입장이 된다는 게 함정}
\vspace{5mm}

저 단계로 해도 약 30회독임. 아무리 바보여도 공부가 안 될 수가 없음.
초고수 단계까지 간 다음에야 경문사 책에 각종 서적을 읽으면서 "중요한 것은 사고"라는 것을 알게 될 것임.
이 단계까지 가면 패턴화고 뭐고 필요없음, 본인이 패턴을 만들고 있을 것임.
\vspace{5mm}

그리고 저런 접근은 다른 과목도 마찬가지입니다.
제 입장에서는 저런 식으로 스케줄 짜서 공부하는 건 '상식'이었는데, 요새 친구들은 그게 상식이 아니라는 걸 뒤늦게 알고 놀랐음(...)
그게 인강의 폐해에다가 야매교재 문제가 아닌가 싶음.
\vspace{5mm}

스킬? 심화개념?
저것까지 하다보면 본인이 알아서 깨닫고 기억합니다.
저게 너무 많다고 해도 10회독 단계는 되도록 줄이실 것. 뭘 놓고 빼야할지 그딴 질문하는 인간은 걍 떨어지라고 저주할 것임.
그런 건 알아서들 하시길. 다만 올해 치는 사람들은 30회독은 무리일수 있으니 좀 빼야할 것임.
\vspace{5mm}

저렇게 공부하고 나면 이 판에서 사기치는 인간들 때문에 짜증날 것이고, 왜 진작 이렇게 안 했느냐에 지나간 세월이 한스러울 겁니다.
\vspace{5mm}

그리고 이런 질문 : 그럼 실력정석의 실질적 완료는?
\vspace{5mm}

그거야 사람에 따라 다릅니다. 실력정석이 좋은 점은 문제를 잘 선별해놓았단 겁니다 $-$ 물론 고수들을 위해서.
실력정석은 실력이 안 되는 친구가 처음에 보면 기 빨아먹혀 망합니다. 그래서 10회독을 초과하지 않은 단계가 아니면 안 보는 게 낫죠.
그러나 해당 단원 문제를 정말 많이 풀어서 귀신에게도 그 내용을 설명할 정도가 되면 매우 좋은 책이 됩니다.
나중에 되면 앞에서 공부한 내용 대부분이 실력 정석으로 압축정리가 저절로 될 것입니다. 그걸 실질적 완료라고 평하겠음.
\vspace{5mm}

저 정도 공부하면 당연히 점수가 나올 수 밖에 없지 않느냐 할 것입니다. 예, 맞아요. 다들 저 정도는 공부해야합니다.
당연히 저런 학생들을 시험쳐서 뽑는 학교가 실적이 좋겠고
저런 학생들을 이용해서 자기 교재가 실적이 좋다고 하는 야매들도 많은 겁니다.
\vspace{5mm}

저 과정의 의의는 일반 양민들이 할 수 있는 것이란 겁니다. 저거도 힘들면?
그럼 초심 단계의 문제집들을 늘려서 회독수를 늘리거나, 저 단계에서 인강을 선별적으로 들어주면 됩니당.
\vspace{5mm}






\section{올해 시험치는 분들을 위한 테크트리}
\href{https://www.kockoc.com/Apoc/719082}{2016.04.08}

\vspace{5mm}

과정은 아래 준해서 하되 정말 양 줄여주면
\vspace{5mm}
\begin{enumerate}
    \item 마플(마더텅, 자이도 괜찮음) $-$ 필수
    \item 쎈 $-$ 필수
    \item 급품벨 $-$ 하나만 보면 좋음(권하자면 일품)
    \item EBS 수능특강 $-$ LV 2, 3만 발췌해서 풀 것
    \item EBS 수능완성 $-$ 기출 빼고 풀 것
    \item EBS N제 $-$ 작년에 준해서 보자면 가성비 좋을 것이므로
    \item EBS 올림포스
    \item 실력정석 $-$ 연습문제만 발췌독할 것, 실력정석이 싫으면 숨마쿰을 보아도 좋음.
\end{enumerate}
\vspace{5mm}

이렇게만 하시길.
양민들을 위한 글은 고2나 내년 시험 노리는 사람 용이고
위 1$\sim$8은 올해 시험치는 사람들의 최저가이드라인임.
\vspace{5mm}

저렇게 다해주고 사설강의에서 '기하와 벡터'와 '확률과 통계' 최상위 문풀강의만 하나 들어서 테크닉만 얻으면 될 것임.
\vspace{5mm}






\section{콕콕에 자주 들어와서 공부를 못 하겠습니다라는 분들을 위한 과제}
\href{https://www.kockoc.com/Apoc/719106}{2016.04.08}

\vspace{5mm}

그 분들은 행동양식을 바꿔야 하겠음.
일단 인터넷 접속이 너무 쉽다는 게 문제이온데
매일 공부한 교재 페이지의 사진을 \textbf{3장 찍어서 일지에 올리는 것으로 약속하시길 바랍니다.}
이 경우 하루라도 빠진다면 그건 공부 안 했다고 본인들이 실토하는 것이니 개망신이고  저 사진을 올렸다는 건 공부했단 증거니 인터넷 접속을 해도 되겠죠.
일지도 총회 이상은 '주간 일지' 작성으로 최적화하고  매일매일은 그냥 기록하는 것보단 맛폰으로 공부한 걸 찍어서 '갤러리' 게시판처럼 올리는 게 더 좋아보입니다.
그렇지 않고 인터넷 접속을 자주해서 공부 안 하는 양 해서 수능 끝나면 돌아온다.... 이거 절대 안 지킵니다.
농땡이 보존의 법칙은 어김없음, 콕 안 들어오면 그 시간에 딴 데 가서 노닥거리고 있을 게 뻔하죠.









\section{변명}
\href{https://www.kockoc.com/Apoc/732675}{2016.04.16}

\vspace{5mm}

\textbf{"처지가 불우해서 공부를 못 했습니다"}
\vspace{5mm}

대학도 마찬가지이지만 어디든 여러분들의 변명을 안 듣습니다.
돈 처발라서 높인 실력일지라도 \textbf{점수만 잘 나오면 우대해주는 것}이고
리어카 끌고 가족부양해서 공부하지 못 해서 \textbf{점수 안 나오면} 걍 씹어버립니다
그게 사회입니다... 가 아니라 인생 전체가 그렇습니다.
\vspace{5mm}

거꾸로 입장 바꿔서 님들이 물건을 살 때 악독한 놈이 만들지라도 그 품질 보지,
그럼 착하고 불우한 사람이 만들었는 데 영락없는 불량이다라고 하면 사주겠습니까.
\vspace{5mm}

가끔 상담할 때 "저는 노력했는데 왜 안 됩니까."란 질문 많이 받죠.
\vspace{5mm}

그냥 말하께요. 본인은 노력하는 수준을 너무 낮게 보아서 그런 겁니다.
공부 잘 하는 애들이 가령 1만시간까지 채워야한다고 본다면, 공부 못 하는 친구들은 300시간을 해놓고 그것도 많이 했다 생각합니다.
다소 과하다 생각하지만 객관적 기준을 대자면, 한 과목당 3000시간을 누적해서 투자해보았냐 그걸 재어보면 됩니다.
\vspace{5mm}

저럴 각오 없는데 자존심 챙긴다라면, 그냥 힘든 길 가실 필요 없습니다.
왜 준비도 안 되어 있고 각오도 안 되어있으며 병아리 오줌만큼 노력하고 죽겠다고 하려는데
최상위권 수준으로 가겠다라고 하는 사람들이 많은지 모르겠습니다만
결론부터 말하면 이런 사람들은 민주주의적으로 얘기해보았자 소용없어요. 죄다 자기중심적으로 가기 때문에.
좋게 달래면서 "그러니가 지금 졸라 하세요, 버티기라도 해야합니다"라고 해도 자기 자존심 때문에 붕괴되고 그러다가
나중에 시험결과 뜨면 또 자기가 잘못했다는 걸 알지만 그걸 인정 못 해서 타인 원망이나 합니다. 그게 그 사람들의 \textbf{그릇}입니다.
\vspace{5mm}

그 그릇을 객관화해서 고치는 거야 물론 타인 입장에서는 쉽지만 본인 입장에서는 힘든 일이겠죠.
그런데 그게 힘드니까 성공하는 사람들도 \textbf{소수}인 겁니다. 누구나 쉽게 극복하면 개나소나 다 성공했겠죠.
망하기 좋은 패턴 중 하나입니다. 이걸 알면 당사자가 알아서 뜯어고치는 수 밖에 없어요.
이런 사람들은 객관적으로 지적하면 자기를 비난한다고 하는 경우가 많은 걸 저도 아는데 답이 없습니다.
본인이 해온 노력이 성과를 못 맺은 건 그 성격 때문이니까요.
\vspace{5mm}

공부괴물들이 공부가 지겹다고 하는 건 다른 차원입니다. 얘들은 그냥 정말 지겨워서 지겹다고 하죠 $-$$-$
이 친구들은 어려운 문제 던져주면 눈을 빛내면서 결국 풀어댑니다.
양민들이 일주일에 할 것을 하루만에 끝내버리고, 심지어 풀이과정 50줄 쓸 것을 5줄에 쓰는 애들이 과연 없을 것 같죠?
그런데 얘들은 머리가 좋은 것보다는 틀이 정말 잘 잡혀있습니다.
자존심은 생각치 않고, 자기에게 도움이 되는 것은 필사적으로 자기 것으로 만들고 이야기합니다.
예컨대 여기 쓰는 칼럼의 내용을 자기가 생각한 양 말하는 경우도 많아요(즉, 이미 자기철학으로 흡수해서 좋은 건 다 한다 그것이죠)
\vspace{5mm}

양민들은 물론 양민의 방식으로 소박하게 가야합니다만, 자기들이 이루는 목표를 성취하려면 괴물들과 싸워야합니다.
자기가 이런저런 사정이 있어서 공부를 못 했다... 라는 변명이 괴수와의 경쟁에서 하나라도 참작될 수야 없죠.
이런 걸 모르면서 힘들다라고 한다면 그런 사람은 '입시'든 어떤 '경쟁'이든 다시 생각해보셔야합니다.
경쟁을 할 때에는 반드시 비정상적인 "괴물"을 상정해놓아야합니다. 어느 분야든 그런 괴물은 최소한 한명은 있습니다.
자기가 평범하게 시작해서 그 괴물을 사냥하려면 어느 정도로 렙을 올려야하는지, 그리고 어디까지 스트레스 받아야하나 정도는 감잡아야겠죠.
그래야 실제로 그 괴물들과의 경쟁 모드까지 가더라도 불연속을 겪지 않고 승리해나갈 수 있습니다.
\vspace{5mm}

가장 좋은 건 본인이 그런 괴물이 되는 것인데
괴물의 요건 중 하나는 \textbf{변태}입니다.
공부하는 고통 자체에서 쾌감을 느끼는 것이지요.
양민들은 공부 스트레스에서 고통을 느끼고 신음해서 그만둔다면,
괴수들은 공부하는 스트레스와 중압감 자체를 즐기기 때문에 사기캐가 되는 것입니다.
이건 의식적으로 그렇게 느끼는 게 아니라, 하다보면 사람이 그렇게 \textbf{변해}버립니다.
\vspace{5mm}

교재 차이가 필요없다는 게 사실 이것입니다. 중요한 건 어떤 교재를 보느냐가 아니라 저런 \textbf{괴물}이 되어가는 것이라서리
이야기해보면 괴물인 사람이 있고 아닌 사람도 있습니다. 어느 정도는 그게 티가 납니다요.
그런데 양민들은 괴물이 되는 것을 거부합니다. 그 사람들 입장에서야 그게 당연해보이겠죠.
그래서 어린 시절에 상급학교에 진학하는 게 장기적으로 좋은 전략일지도 모르죠. 어린 시절부터 괴수들을 겪다보면
그 괴수들의 눈높이가 정상이 되므로 본인도 그런 괴수가 되려고 자발적으로 노력할테니까요.
\vspace{5mm}






\section{다시 적는 인강에 대한 비판적 접근}
\href{https://www.kockoc.com/Apoc/732693}{2016.04.16}

\vspace{5mm}

처음에 개념서를 본인이 읽고, 그 다음 기초문제집들을 최소 2권 이상은 돌리고, 오답정리하고
기출 풀어보고 깨져본 다음에 인강을 발췌해서 들으시길 바랍니다.
\vspace{5mm}

처음에 오답정리를 해보고 깨져본 다음에 듣는 인강의 흡수율이 좋지
이런 것 안 하고 인강만 계속 돌리면 절대 실력이 \textbf{안 늘어납니다.}
인강을 듣고 있을 때야 강사가 신기한 정보들을 제공해주고 문제도 예술적으로 푸니까 공부가 되는 걸로 생각하는데
실제로 그건 \textbf{자기 실력이 아닙니다.}
실력을 키우려면 본인들이 직접 문제를 읽고, 그것들을 종이에 써보면서 정리해야합니다.
\vspace{5mm}

인강을 들을 때 공부가 된다고 하는 건 일종의 '마취 효과'와 비슷합니다.
들을 때야 고통도 안 느껴지고 흡수가 좋다고 생각해서 공부가 된다고 하나
문제는 그런 건 정착되기 어렵단 겁니다. 기분좋게 들은 건 뇌에서 기억을 잘 하려하지 않습니다.
남는 건 강사의 잡답이나 농담, 그리고 몇가지 신기한 스킬 정도일 것입니다. 투자한 시간에 비하면 효율이 정말로 낮아요.
\vspace{5mm}

강의가 좋다고 찬양하는 사람들은 많아도, 정작 그 사람들의 실적이 좋은 경우는 별로 못 보았습니다.
정반대로 실적이 좋은 사람들은 수험기를 보면, 인강을 안 들어도 성공했겠구나 느껴질 정도로 스스로 문제풀이를 많이 한 케이스입니다.
그럼 인강만 줄창 듣고 문제풀이를 안 한 케이스가 성공한 경우? 제가 아는 한 단 한건도 없습니다.
\vspace{5mm}

공부는 뇌를 길들이는 과정입니다.
뇌는 생존, 공포, 섹스, 고통 등에 관한 것은 정말 1번만 해도 잘 기억합니다. 그게 우리 유전자의 명령이니까요.
인강을 편히 듣는다는 건 그 정보가 뇌가 선호하는 것과 거리가 멀다는 겁니다. 그래서 들을 때는 기가 막혀도 그게 실력으로 이어지지 않아요.
정반대로 어떤 문제를 푸는 것이나 특정 지식이 자신의 절박한 상황에서 벌어졌다, 그건 악몽처럼 끝까지 기억하게 됩니다.
\vspace{5mm}

더 재밌는 사실은 인강을 들은 건 정착이 안 되지만, 자기가 남에게 가르치는 건 정착이 잘 된다는 것입니다.
외형상 보기에는 입력과 출력의 차이인데, 실제로는 출력을 해보는 게 더 도움이 된다는 것이죠.
왜 그럴까, 그건 가르치는 과정 자체가 쾌감이 있기 때문입니다. 이것 역시 유전자의 명령으로 가본다면
가르친다는 건 남보다 우월한 지위에 서있는 것이니 뇌는 이걸 선호할 수 밖에 없다는 걸로 해설할 수 있을 것입니다.
\vspace{5mm}






\section{수학 사교육이 학생 발목을 잡는 경우}
\href{https://www.kockoc.com/Apoc/734042}{2016.04.18}

\vspace{5mm}

\begin{itemize}
    \item[] \textbf{A란 문제를 풀 때에는 반드시 B를 써야한다.}
    \item[] \textbf{해당 문제 패턴에 쓰이는 스킬과 공식을 암기해라.}
\end{itemize}
가장 답없는 게 저 케이스다. 
왜냐면 자기가 공부를 하고 있고 각종 스킬을 있으니 잘 되고 있다고 생각한다.   
그러나 저것이 자기 발목을 잡는 것임을 모른다.   
수학 문제는 두가지다. 과거의 문제, 그리고 현재(지금 치는 시험)의 문제.   
학교 내신이나 교육청 모의는 과거의 문제 비중이 높다.
그래서 저런 정석, 스킬 중심의 접근을 하면 점수가 오른다.
그러나 과거의 문제 연연하지 않는 새로운 문제야말로 자신의 운명을 결정하는 것이다.
\vspace{5mm}

수학이 힘들다고 하는 이유는 다른 게 아니다.
그냥 접근방법이 잘못 되었기 때문이다.
특정 문제에는 특정 스킬을 무조건 써야한다고만 배우지, 그걸 \textbf{왜 써야하는지를 배우지 않는다. 아니 가르칠 사람도 별로 없을 것이다.}   
개인적으로도 이런 게 궁금해 여러 인강을 듣고 책을 찾아보았지만 해답은 지금 지진으로 고생하는 그 나라 책에 있었다.
\vspace{5mm}

이런 스킬암기에 주력하게 되면 본인이 문제해결능력을 상실해버린다.
아무 것도 모르고 도구도 최소화되어야 본인이 문제를 분할하고 조건을 분석하면서 생각이라는 것을 한다.
하지만 학원이나 교재에서 가르쳐준대로 $\sim$ 만 쓰면 된다라고 하면, 자기도 모르는 사이에 문제를 푸는 게 아니라 그냥 암기해버린다.
물론 현실적으로는 암기를 피할 수 없을 것이다.
그러나 수학공부의 목표는 문제를 암기가 아니라, 문제를 풀기 위한 \textbf{머리를 만드는 것}이다.
\vspace{5mm}

노력을 하는데 자꾸만 모르겠다. 라고 하는 케이스는 가만보니 저런 식의 암기형으로 머리가 맛이 간 케이스다.   
이런 애들은 생각훈련, 생각하는 방법에 관한 강의 같은 것으로 치유하는 게 좋겠으나 유감스럽지만 그런 책도 강의도 찾기 어렵다.   
게다가 그런 암기형 패턴을 폐기하지 못 한다.   
교과서가 좋다는 평가를 받는 건 교과서가 정말 좋아서가 아니라, 스킬이 덜 실려있기 때문이다라는 게 웃고 넘길 얘기만은 아니다.






\section{국어나 영어의 스킬적 접근이 문제인 경우}
\href{https://www.kockoc.com/Apoc/735647}{2016.04.19}

\vspace{5mm}

고교수학의 문제풀이적 접근은 폴리야로 집대성된다.
\vspace{5mm}

\href{https://en.wikipedia.org/wiki/George_P%C3%B3lya}{링크}
\vspace{5mm}

우정호 교수님이 번역한 폴리야의 책을 읽으면 어떻게 수학문제를 풀어야하는지 고전적으로 알 수 있다.
다시 말해서 이 경우는 문제풀이의 접근이 이미 일반론적으로 정리되었다는 것이다.
그 다음 문제풀이 접근법은 일본인들의 책을 구해다 읽으면 된다(번역된 것들이 있다)
\vspace{5mm}

주의해야할 사실은 문제풀이 접근법은 문제풀이 스킬과는 다르단 것이다. 과장해말하면 접근법과 스킬은 상극이다.
스킬 위주의 공부는 무조건 스킬을 써야한다는 강박관념에 빠진다. 그래서 스킬이 안 먹히는 문제풀이가 나오면 멘붕해버린다.
문제풀이 접근법은 초딩산수를 쓰더라도 그 어려운 문제를 논리적으로 분해해나가는 과정인 것이다.
\vspace{5mm}

국어와 영어에 대해서는 스킬이 중요하다는 것이 작년까지의 생각이었는데 이걸 수정해야겠다는 생각이 들었다.
2000년대 후반까지는 국어나 영어의 독해나 문풀 스킬이라는 게 꽤 유용했다.
그런데 지금 수험생들이 그 스킬을 모르느냐 하면 그건 아니다. 사실 이것도 '과다'하다.
그럼 이건 스킬이 문제가 아니라는 이야기다.
\vspace{5mm}

왜 그런가 생각해보니 매년 줄어드는 게 있다. 그건 바로 '독서량'이다.
인터넷 속에서 살다보니 책을 읽지 않는다. 책을 읽지 않으니 다양한 텍스트를 접하지 못 한다.
알고 있는 텍스트가 없으니 창의력의 재료가 부족해지고 분석의 연습대상조차 없다.
이 상태에서 스킬을 알아 보았자 소용이 없다. 왜냐면 처음보는 지문이 나오면 스킬을 쓰기 전에 뭔 소리인지 몰라서 포기해버린다.
\vspace{5mm}

스킬이 중요하다고 가르치는 사람들은 자기 세대가 독서량이 많았다는 걸 간과하고 있다.
무엇보다 가르치는 데 있어서 텍스트들을 하나하나 떠먹여주고 소화시켜주는 것은 상당히 품이 많이 드는 일이다.
독서량이 없어도 스킬만 가지고 풀 수 있다라고 하면 몸값이 비싸질 수 밖에 없다(어차피 입시결과는 책임지지 않아도 되기 때문이다)
\vspace{5mm}

"텍스트 소화량의 결핍"이 관건인데 문제는 이걸 현재 수험생들은 인식하지 못 한다.
애꾸눈의 나라에서는 애꾸눈이 정상이다. 절대적 독서량이 부족해도 다 똑같으니 그게 정상이라고 착각하는 것이다.
당연히 이런 체제에서는 어렸을 때부터 책을 많이 읽어온 애들이나, 하다 못해 기출지문 양치기를 한 애들이 점수가 나온다.
그러나 이런 애들은 소수이다. 무엇보다 지금 많은 독서를 해야한다니... 에서 다른 좋은 스킬이 없을까 고민하게 된다.
\vspace{5mm}

올해는 모르겠지만 내년 입시 준비하는 친구들은 정말 매주 책 한권씩은 읽는다는 강행군을 해야할지도 모른다.
수험 방향은 자기가 속한 경쟁집단의 약점을 보완하는 것이다. 현재 수험생들은 정말 책을 \textbf{지지리도 읽지} 않는다.
\vspace{5mm}

과장이 아니고 난 책을 읽지 않는 사람은 인간 취급을 안 하는 것은 아니라고 쳐도 그다지 존중하지는 않는다.
책을 읽지 않고 인터넷에 올라온 글(이 글도 마찬가지)만 보거나 사이트 게시판 가서 거기 글 보고 휘둘리는 게 병신이지 어디 사람새기인가?
어떤 문제가 있으면 다수의 쥐떼근성에 휘둘리지 않고, 자기가 추합한 정보나 느낀 바를 독서로 다져진 지성으로 스스로 가공해
누가 까다로운 질문을 해도 분명히 대답할 건 대답하고 모르는 것은 모른다고 해야 인간이지, 그렇지 못 하면 노예새기나 진배 없다.
\vspace{5mm}

좀 괜찬다 하는 사람들과 대화해보면서 질문해보면 오 이 사람은 책을 읽어왔군... 하면서 어떤 책을 읽었느냐 파악한다.
만약 그 책이 뻔하디뻔한 탑셀러라면 안심해도 좋다.
그런데 뭔가 말하는 내용이 예외적인 데다가 읽는 책도 대중적이지 않으면 눈을 비비고 다시 쳐다봐야한다.
책을 읽는 사람은 알 것이다. 사람이 가장 아름다울 때가 책을 읽을 때이고, 그 때 우리의 눈길도 그 사람보다 그 책을 향한다는 걸.
\vspace{5mm}

더 오버해서 쓰면 요즘 세대들이 헬조선하는 것도 웃긴 이야기가 그들이 까는 기성세대만큼 고생하는 것도 아니라고 보지만(586 제외)
무엇보다도 책을 안 읽기 때문이다. 만약 이 세대가 일주일에 책을 3권씩 읽고 부지런히 학습한다면 나이에 상관없이 내가 굽신거렸을 것이다.
그러나 맛폰질은 하면서 책을 안 읽으므로 '까도' 별로 후환은 없어보인다.
\vspace{5mm}








\section{어른들이 공부만 하라는 거}
\href{https://www.kockoc.com/Apoc/758269}{2016.05.03}

\vspace{5mm}

적어도 어른들 얘기 중에서 단 하나만 건지면 \textbf{"쓸데없는 짓 말고 공부나 하라"}
내 입장에서 나도 꼰대들의 메시지 중 참말과 거짓말은 구분하는데 저 말은 정말 진짜다.
무능한 젊은이가 정의감에 차서 움직인다고 쳐도 현실적으로는 별 쓸모 없는 경우가 대부분이며,
그런 정의감조차도 실제로는 오랜 사색에서 나온 진정한 철학이라기보단, 선동당하거나 혹은 성욕을 감춘 격정에 불과한 경우가 많아서이다.
\vspace{5mm}

우리가 공부를 하는 이유는 노골적으로 말하지만 \textbf{자신을 비싸게 팔기 위해서}이다.
이제는 대학을 졸업해 취업이라도 하면 정말 다행이라고 한다. 물론 앞으로는 취업의 개념조차 사라질 것이다.
이제는 일자리의 시대가 아니다 일거리의 시대다. 일거리를 스스로 찾아서 물고 와야하는 시대인 것이다.
일거리들을 물고오려면 본인이 비싼 몸이어야 한다.
\vspace{5mm}

그러면 사회 부조리가 벌어져도 침묵하고 공부만 하란 말입니까... 유감스러운데 이게 진리다.
본인이 능력이 없는 한 나서보았자 고기방패 빼고 뭔 가치가 있나.
일개 잡몹은 사회에서는 신경조차 쓰지 않는다. 네임드 히어로여야 그나마 신경써주는 척이라도 하지.
\vspace{5mm}

그러나 적어도 내가 관찰한 바로는 다들 사회 부조리에 항거하는 척 하지만 실제로는 호구가 되는 루트를 밟는 쪽이 많다.
특히 20대의 젊음이라는 건 그런 능력의 감가상각을 분식처리하는 점이 있다. 자기가 건강하고 젊음이 넘칠 때는 뭐든지 할 수 있는 양 생각한다.
10년도 지나지 않아서 곧 꺼질 텐데, 심지어 어떻게 사느냐에 따라선 5년도 못 넘길 수도 있다.
역산법적 사고 $-$ 자기가 서른살, 마흔살 ... 그리고 뒈지기 직전이라면 어떤 루트를 밟았을까 하고 가정해보는 방법으로 가면 답은 보인다.
나이먹을 수록 늘어나는 건 주름살과 후회일 뿐이다라는 말도 진짜다(반대로 머리털은 줄어든다)
\vspace{5mm}

자기가 대단하다라는 생각도 버려야하고 공부도 사실 별 게 아니라고 얘기해야 한다.
다만 자본주의 사회에서의 개인이 가질 수 있는 자본은 결국 능력으로 환원된다는 것을 알아야 한다.
\begin{itemize}
    \item 사회에서 바라보는 우리의 외모는 학벌과 성적표이다.
    \item 사회에서 바라보는 우리의 육체는 실무적인 능력이다.
    \item 사회에서 바라보는 우리의 마음은 교양과 전문성이다.
\end{itemize}
\vspace{5mm}

가장 큰 착각은 공부하지 않아도 나답게 살 수 있다....  공부 없이도 내가 존재한다... 라는 것.
조금만 생각해보면 이거야말로 망상이다. 공부 없이는 우리는 털없는 원숭이에 고깃덩어리에 불과하다.
배우지 않는다면 소말과 다를 바 없는 가축이나 노숙자나 거지 창녀들과 별 차이도 없다.
그 자아라고 하는 것조차도 탯줄 떼고 교육을 받기 전에 부모 등이 주입한 패턴이다.
참자아라는 건 본인이 경험하고 배우면서 학습한 그 자체이다.
\vspace{5mm}






\section{할 수 있기 때문에 인간이 아니다.}
\href{https://www.kockoc.com/Apoc/758394}{2016.05.03}

\vspace{5mm}

어른들이 하지 말라는 건 일단 안 하는 게 좋다. 그리고 왜 하면 안 되는 건가 스스로 생각하고 관찰하고 결론 내리고 생각하면 된다.
가장 간단한 건 자기의 아들 딸이 있어도 그걸 하게 허용할 것인가 자문자답하면 된다.
예컨대 포르노를 보고 히히덕거리는 사람이 그럼 내 딸이 야동을 찍는 것도 허락할 것인가... 생각하면 그냥 답이 나온다.
혹자는 이걸 가지고 가족을 언급하는 건 비겁하다고 항변하지만, 바꿔 말해서 자기 가족도 시키지 못 할 것이면 그게 문제가 아닌가 얘기하면 된다.
\vspace{5mm}

문제는 하지 말아야하는 건 일단 \textbf{저질러 버린 다음}에는 답이 없다는 것이다. 그래서 하지 말라고 강제하는 것이다.
자기 개인이 살인하고 간음하고 하는 것은 문제가 없다고 생각한다. 자기가 책임질 수 있다고 \textbf{멋대로 착각하기 때문}이다.
자기 자식이라면 그렇게 바라보지 않을 것이다.
\vspace{5mm}

아마 지금 20대들은 '우리는 $\sim$ 할 수 있다'라고 주입받았을 것이다. 그리고 그게 인간답다고 생각하겠지만 그건 틀린 얘기다.
왜 예수, 부처, 공자가 지금도 3대 성인인가. 사실 이들이 인간이었다면 지적 수준은 딱 지금의 고딩이었을지도 모른다.
그래도 종교적인 면을 떠나서 이 분들을 생각하지 않을 수 없는 건, 우리가 '인간'답다고 하는 것들이 다 이 분들의 가르침에서 왔기 때문이다.
그 이후의 인간적으로 산다는 건 이 분들 말씀에 주석을 다는 수준이라고 해도 지나친 얘기가 아니다.
\vspace{5mm}

그런데 이 분들의 가르침은 결국 "\textbf{하지 말라}"는 것이다.
할 수 있기 때문에 인간인 게 아니라, 하지 않기 때문에 인간인 것이다.
살인하지 마라, 간음하지 마라, 도둑질하지 마라.... 얼핏 보기에 사소해(?) 보이지만 이런 것들을 지키니까 인간 사회가 유지되고 발전해온 것이다.
저 분들이 위대한 건 "하지 말라"는 걸 가르쳤기 때문이다.
사실 그것만으로도 신(神) 대접을 받아도 좋다. 왜냐면 우리가 아는 역사가 제대로 쓰여진 건 그 가르침이 전파된 이후여서이다.
\vspace{5mm}

혹은 이렇게 반론할 것이다. 인류사는 자유를 쟁취함으로서 발전해 온 것이 아니냐고.
그런데그 원없는 자유는 문명 이전 야만 이전이 더 압도적이지 않았겠느냐. 그럼 그 시대가 이상향이냐.
"하지 말아야하는 것"을 지키면서 할 수 있는 것 하고싶은 것을 늘려왔으니까 자유로워진 것이다.
자유는 '하지 말라' 위에서나 성립될 수 있다.
\vspace{5mm}

그런데 그들이 장사하려면 저런 하지 말아야하는 걸 깨뜨려야 한다.
그리하여 똑똑한 어른들은 남의 자식에게 "한계는 없어, 너희들은 뭐든지 할 수 있어. 그러니 너희들은 뭐든지 파고 살 수 있어"라고 한다.
그래서 청소년들의 성도 상품화하고 노동력도 저렴하게 구입한다. 한편으로 그들을 협박해 온갖 상품을 팔아댄다.
이런 기본적인 것조차도 모르는 청소년들과 20대들이 환상 속의 기득권 탓을 한다라고 보이는 건 전혀 지나친 얘기가 아니다.
그런 사람들이 자기 자식들은 어떻게 취급하겠나.
\vspace{5mm}

가난만으로 모든 것이 보호받고 정당화되지 않는다.
가난하다는 사람들이 신나게 술을 마시고 돈을 마음대로 쓰는데 부자탓을 하는 건 이상하지 않나.
가난하다면서 하지 말라는 것을 즐기고 자기 학습을 게을리 하며 저축을 하지 않으면서 국가 탓을 하는 건 웃긴 것이다.
물론 이 글을 보고 자기는 빡세게 일하고 즐기지도 못 한다고 울분을 터뜨릴 분도 있을 것이다.
그런데 이런 사람들은 머지않아 곧 탈출한다. 그리고 자기를 이미 그 방종의 무리들과 차별화시킨다.
오히려 내가 보는 그들은 자기가 정말 가난하고 고통을 받고 힘들다라고 착각하지만 실제로는 즐길 건 다 즐기는 사람들이었다.
\vspace{5mm}





\section{왜 실패하는가}
\href{https://www.kockoc.com/Apoc/762686}{2016.05.06}

\vspace{5mm}

사람들은 말이지, 눈 앞의 푼돈 얼마를 위해서라면 웬만한 일은 다 견딜 수가 있다네.
부자들은 그 특성을 이용해, 평생을 시중받으며 안락하게 살지...
왕은 혼자서 왕이 되는 게 아니야. 왕이 혼자서 그 자리를 유지할 수 있다고 생각하나?
돈 따위는 필요없다는 천한 것들이 결속해서 반항을 하면 왕도 결국 사라지는 법일세.
하지만 가난한 자들이 왕이 되고자 돈을 바라면, 역으로 지금 있는 왕의 존재를 보다 견고하게 반석 위에 올려주지.
모두 그런 메마른 패러독스에서 빠져나오질 못해. \textbf{돈을 바라는 이상, 왕을 쓰러뜨릴 수 없네. 계속 매일 수 밖에 없지}.
왕도 폭동을 막기 위해, 다들 고만고만 윤택한 기분으로 있을 수 있도록 주의하고 있다네. 실제로는 얼마나 뜯어먹히고 있거나 말거나 말일세.
\vspace{5mm}

$-$ 도박묵시록 카이지의 진주인공 효우도 카즈타카 회장님의 말씀 $-$
\vspace{5mm}

관찰과 경험이 쌓이다보면 왜 망할 수 밖에 없나 하는 패턴들이 발견되는데
그 중 하나가 \textbf{우유부단}.
\vspace{5mm}

왜 우유부단이 문제냐면 이건 당사자가 스턴먹은 상황과 똑같아서 그렇습니다.
A할까 B할까 하면서 '시간은 계속 흘러가지'만 사실 아무 것도 선택 하지 못 하고 준비조차 하지 못 하고 그래서 기회를 날려먹죠.
속으로는 둘 다 가질 수 있을 거야라는 헛된 망상을 품습니다. 그리고 현실적으로 둘 다 얻지 못 하지요. 아니, 결국 전부 잃어버립니다.
\vspace{5mm}

"아냐, 운이 좋아서 둘 다 얻을 수도 있고 둘 다 할 수 있는 방법이 있을 거야"
\vspace{5mm}

물론 그 방법이 존재하는 경우가 대부분입니다. 그러나 그 방법도 '제때'에 공급되지 못 하면 아무 소용이 없는 것입니다.
당사자가 학과 공부도 하면서 수능까지 대비하는 방법이 없지는 않겠죠. 그런데 문제는 바로 그 때 당사자가 그걸 모른다는 것입니다.
그 방법이 존재한다고 한들 자기가 그걸 모르고 써먹지 못 한다면 아무 소용이 없어요.
\vspace{5mm}

손실을 인정할 때는 빨리 인정하고 정리해야합니다. 왜냐면 그래야 새로운 게임을 할 수 있기 때문이죠.
하지만 자존심 문제도 있고 무엇보다 자기가 낭비한다고 생각해서인지 그 손실을 인정하지 못 하고 죽은 자식 불알 만지는 사람들이 많습니다.
이것도 역시 장기적으로 보자면 '왕'과 '노예'가 갈라지는 분기점이겠죠.
\vspace{5mm}

자기가 애당초 의도하는 것에 도움이 되지 않는 건 황금빛이 나더라도 과감히 무시해버려야합니다.
예를 들어 시험보러가는데 길가에 1억 지폐 뭉치가 떨어져있더라... 하더라도 이걸 무시하고 가야한다는 것입니다.
물론 1억을 주으면 시험보는 것보다 더 현명한 선택일 수 있겠죠. 하지만 그로써 시험도 포기해야하고 앞으로의 선택이 문제가 됩니다.
자기의 목적을 망각하고 눈 앞의 이익만 좆는 거야말로 망하는 지름길이기 때문이죠.
손자병법에서도 이런 말이 나왔죠. 적에게 작은 이익을 줘서 유인하라.
그 말은 다시 말해 푼돈에 낚이는 사람은 뻔하다는 것입니다.
\vspace{5mm}

목표를 실행하는 건 전쟁과 같죠. 30:20로 싸우면 우리가 10이 남고 끝나는 게 아닙니다. 실제로는 9:4 비율 차이가 나서 우리가 20이 남는다고 하죠.
RTS 전략시뮬게임을 할 때에도 확인되지만 이기는 확실한 방법은 적보다 압도적으로 많은 병력으로 적시에 적을 치는 것입니다.
병력을 쪼개는 건 원칙적으로 미친 짓이지요.
\vspace{5mm}

마찬가지로 자기 일을 할 때에 안 그래도 시간과 체력이 제한되었는데 더 많은 걸 한다는 건 없는 병력을 더 쪼개는 것과 똑같은 짓이죠.
물론 계획을 세울 때는 자기가 그만큼 할 수 있다고 터무니없이 자신합니다만 그래서 성공한 사례는 제가 아는 한 없습니다.
한번에 여러가지 일을 해내는 사람은 이미 능력자이거나 아니면 타인의 도움과 협동 하에서 일을 효율적으로 처리하는 경우입니다.
\vspace{5mm}

좋지 않은 대학에 다니는데 수능치고 싶다... 그러면 정답은 간단합니다. 수능에 올인하는 것이죠.
다만 부모님 눈치가 있다라고 하면 학교 다니는 척 하면서 수능에 올인하겠죠.
그런데 여기서 꼭 '기왕 다니는 것 학교 졸업장도 따야지'라고 마음먹는 순간 패배는 확정되는 겁니다.
자기의 병력들이 수능이라는 거대한 군대와 싸우고 있는데 그 병력 중 일부를 뺀다.... 전쟁으로 치면 미친 짓이죠 사실
\vspace{5mm}







\section{인터넷의 수험정보}
\href{https://www.kockoc.com/Apoc/764299}{2016.05.07}

\vspace{5mm}

사실 불필요한 게 다수.
아무개 선생을 들어야 한다는 것은 정보가 아니라 공해죠.
쓸데없는 정보가 많으면 거기에 휘둘립니다. 그래서 피해 본 수험생들이 꽤 많아요.
어떤 인강을 들어야 한다거나 또 어떤 교재를 풀어야한다거나 하는 식의 정보를 가장한 광고글에 낚인 경우가 많죠.
\vspace{5mm}

그럴 바에는 나 자신의 "성격", "사고 스타일", "오답 유형", "취약 문제" 등을 기록하고 정리하는 편이 낫습니다.
아니 사실 수험판에서 굴러먹는 친구들을 보면 학원 홍보자 해도 될 정도로 참 이상한 분야까지 다들 알고 있으면서
정작 자기의 문제가 뭔지 전혀 모르고 있는 경우가 많습니다.
정보는 자기가 필요한 것만 얻으면 됩니다. 그리고 핵심적인 것만 있으면 되는 것입니다.
그것도 본인이 스스로 오프라인에서 탐문하고 곁눈질하고 캐내는 것이 좋습니다. 정말 중요한 건 인터넷에 올라오지 않죠.
\vspace{5mm}

쓸데없는 정보가 많으면 프로세스가 상당히 복잡해집니다. 프로세스가 복잡해지면 오류도 늘어나고 나중에 통제도 못 하지요.
개인이든 기업이든 구조조정과 혁신의 논리는 "불필요한 것을 없애고 핵심적인 것만 살리는" 것입니다.
핵심 프로세스만 잘 조합하고 싶으면 필요한 핵심 정보만 있어야 합니다.
그리고 그 핵심 정보는 상식과 통념과 '갈등'을 일으키는 것이어야합니다. 우리가 정보를 수집하는 건 '변화'를 알기 위해서이죠.
상식과 통념에 순응하는 정보는 변화를 가르쳐주지 않습니다. 그런 건 버려도 됩니다.
\vspace{5mm}

가령 지구과학을 선택해야 한다... 라는 건 작년까지는 매우 괜찮은 정보였습니다. 지금은 \textbf{과거}의 정보입니다만요.
지금 눈여겨보아야하는 건 지구과학 선택자가 많아졌다는 것이고, 그리고 이것만 믿다가 우리가 어떤 통수를 먹을까 하는 것입니다.
개정수학이 그 이전 수학보다 내용이 빠졌다라는 것 역시 과거의 정보입니다.
반면 개정수학 과정에서는 수포자가 나오기 힘들어서 문제가 더 어려워지는 경향이 있다는 게 핵심정보겠죠.
하지만 이것들도 시간이 지나면 쓸모가 없어집니다. 현실은 늘 바뀌죠..
\vspace{5mm}

그러니 수험생은 그냥 '하라는 공부만 하는 게' 정답입니다.
정보가 부족하기보다는 정보에 휘둘려서 \textbf{공부 방향을 못 잡아} 망하는 경우가 있습니다.
틀린 시험문제도 대부분 당사자의 교재나 학습커리로 커버되는 것이 다수입니다. 문제는 본인이 그걸 숙달하지 못 했다는 것이지만요.
\vspace{5mm}









\section{수학은 결코 쉬워진 게 아님.}
\href{https://www.kockoc.com/Apoc/767358}{2016.05.10}

\vspace{5mm}

양극화 사회에서는 평균적인 접근이 무의미함.
A는 1억을 벌고 B는 한푼도 못 번다고 하면 평균소득이 5천만원이 되는데 이게 정확한 자료라고 하지는 않음.
수학난이도가 쉬워졌다... 라는 건 걸러들을 필요가 있음.
그건 해당 응시생들이 어느 정도까지 공부했느냐하는 것까지 감안해야함.
\vspace{5mm}

20년 전에 저렇게 공부했으면 정말 신문기사에 나고 아주 천재라고 그랬을지도 모름.
그런데 이제는 중학교 때 정석을 다 마치고 간다라는 건 그렇게 놀라울 것도 아님.
선행학습은 이제 더 이상 문제가 아님, '할 놈은 거의 다 하고 있으니까'
여기서 유의할 건 상위권을 세분화시키면 그 내부에서도 격차는 매우 크다는 것이고
이건 IMF 이후에 태어난 세대부터 뭔가 더 당연하다고 여겨지는 바가 없지 않음.
\vspace{5mm}

과거 사람들보고 요즘 수학과 과학을 풀라고 해도 자신있게 풀 수 있을지는 의문임.
게다가 난이도보다도 더 힘들어진 건, 응시생들 수준이 높아지는 경향이 있다는 것임.
\vspace{5mm}

그렇다면 이미 수능 성적은 중학교 때부터 결정된다라고 해도 지나친 말이 아니게 되어버린다는 것.
왜냐면 그 때 다 끝내고 온 녀석들은 계속 공부할 테고 그럼 경험치나 레벨이 기하급수적으로 늘어나 격차가 벌어짐.
선행 안 한 친구들이 제 아무리 인강 듣고 실모 풀면서 간다고 해도 따라잡기 어려워지고
이 원인을 정확히 모르는 사람은 "머리" 이야기만할 것임.
\vspace{5mm}

물론 대치동에 가지 않더라도 방법은 있음. 학부모나 학생 본인이 중학교 때 저렇게 공부하면 되는 것이긴 함.
그러나 다수가 그래야 한다는 \textbf{생각을 하지 못 함}. 그냥 학교가 시키는대로 가야만 한다고 믿고 있기 때문임.
\vspace{5mm}

수1, 수2를 중학교 입학 전에 끝낸다는 건 남들보다 3$\sim$4년 앞선다는 이야기
그건 4$\sim$5수할 것을 미리 앞당겨 현역으로 끝내는 것과 똑같음. 즉, 이건 머리문제가 아니라는 것임.
누가 더 빨리, 많이 공부하느냐가 결국 좌우한다는 이야기임.
\vspace{5mm}











\section{수학강의에 휘둘리는 사람들}
\href{https://www.kockoc.com/Apoc/775832}{2016.05.16}

\vspace{5mm}

연구차 인강 많이 들어보았지만
개인적으로 정말 도움이 되었다 느낀 건 EBS 강의였음(...)
왜냐면 생각하는 법을 가르쳐주었고 그건 정말 나도 잘 써먹고 있음.
\vspace{5mm}

그리고 인터넷에서 강의 세세히 말하는 글은 걸러들어야죠. 공부하는 사람들이 그런 글 쓰겠나
그렇게 강의 소믈리에가 가능한 실력자면 좋은 대학 가서 공부하고 있어야지.
하지만 어떤가, 일지분석을 하건 콕콕 합격자 분석을 하건 '지독하게 공부한 사람'이 잘 나가지 강의도 사실 무차별한데
\vspace{5mm}

교재도 그냥 시중교재와 기출 신나게 풀고 교과서 연구하면 되는 것 아님?
지금 기출이라도 다 푼 사람 몇이나 있을까요.
작년 11월에 마플 잡고 그냥 달리라고 했는데 이거 다 푼 사람 있기나 하겠습니까만.
\vspace{5mm}

교과외적 내용 담은 게 소용이 있겠습니까. 교과외적 내용이라고 하면 그럴 바에는 대학수학까지 다 공부한 사람이 더 유리하겠죠.
중요한 건 교과외적 내용이 아니라 낯선 문제도 논리적으로 풀이하거나
특이점적 발상을 떠올리는 '생각하는 법'을 만들고 그런 습관을 들이는 것일터인데
정말 수학을 잘 하는 친구들은 교과외적인 것도 걍 신경도 안 써요. 수험에 필요한 것들은 어차피 저절로 유도되는 것들이고
교과외적 내용 암기해보았자 수능에서 교과서 내 내용으로도 참신하게 꼬아서 내면 그거 못 풉니다.
\vspace{5mm}

주어진 조건 도구가 20이라고 했을 때 100이라는 목적치를 달성하기 위한 80 $-$ 이건 스스로 생각하고 상상하고 전략짜는 걸로 채워야죠.
어떤 수학문제건 교과서적 기본원리를 문제의 답으로 연결시키기 위해서는 본인이 thinking을 해야하죠.
그런데 어차피 수학문제도 db화 가능한 점이 있다는 점에서 그걸 thinking이 아니라 pattern memorizing으로 해결하는 꼼수로 갑니다.
이게 내신까지는 어느 정도 먹혀요. 그런데 수능에서는 잘 안 먹힙니다, 왜냐면 수능은 꼭 새로운 걸 내니까.
\vspace{5mm}

매년마다 휘둘리는 호구들은 늘 생깁니다.
\vspace{5mm}

뭐 그건 중요한 건 아니고 올해 고2들이 참 역대급 실력자가 많다는 증거만 속속 확보되어서 걍 무섭다능.
아이엠에프 이후 세대들은 정말 계획적으로 임신, 출산, 교육해서 그런가. 부모의 지적 자산과 경험까지도 걍 상속.
시대적 환경이 꽤 무섭다는 걸 느끼죠.
\vspace{5mm}









\section{[임시 공지] 6평을 치르실 분들은}
\href{https://www.kockoc.com/Apoc/779984}{2016.05.18}

\vspace{5mm}

한시적으로 챗에서 6평 이야기를 나눌 수 있도록 조치가 있을 것 같습니다.
일지 작성하시는 분 분만 아니라 올해 6평 치르실 분들은 임시 공개챗방에서 전략을 이야기했으면 좋겠습니다.
\vspace{5mm}

개정교육과정 첫 6평이기 때문에 진득히 연구할 필요는 있어보이네요.





\section{수학교재 중간리뷰}
\href{https://www.kockoc.com/Apoc/780165}{2016.05.18}

\vspace{5mm}

즉흥적으로 쓰는 거라서리
\vspace{5mm}

아무튼 지금 와서도 수학 답이 없다 하는 부류를 위한 간략
\vspace{5mm}
\begin{enumerate}
    \item 수력충전 $\bigstar$$\bigstar$$\bigstar$$\bigstar$$\largestar$ : 확실히 수포자 구제용 맞고 필요한 계산 드릴 다 들어감. 서울우유급
    \item 올림포스 $\bigstar$$\bigstar$$\bigstar$$\bigstar$$\bigstar$ : EBS가 내놓은 진정한 야심작. 수수해보이는데 문제 하나하나가 서민수학의 걸작, 단원 김홍도 그림.
    \item 마플 $\bigstar$$\bigstar$$\bigstar$$\bigstar$$\largestar$ : 시중 야매교재 강의 다 필요없고 이걸로 일단 정리. 그리고 부족하면 마플 교과서 추가.
\end{enumerate}
\vspace{5mm}

지금 와서 쎈 보겠다는 분 시간없을 수도 있으니 쎈은 보충용으로 돌리고
아 올해 시험 포기해야하나 하는 분들은 위 3개만 돌리세요 그럼
\vspace{5mm}










\section{6월 지나면 힘듭니다.}
\href{https://www.kockoc.com/Apoc/784341}{2016.05.20}

\vspace{5mm}

또 이 이야기를 한다는 건 잔인하긴 한데 경험해보시면 아실 거예요.
그래도 이 이야기를 하는 이유는 시간낭비시키기 그래서 그런데
기온이나 계절상 제대로 공부할 수 있는 시기가 11월부터 5월까지입니다. 6월부터는 정말 양을 절반으로 줄여야하고 체력관리가야하거든요
\vspace{5mm}

그런데 6평과 9평 쓰나미로 정신적 충격 먹고 거기다가 온갖 잡서들이 유혹해서 자기 공부 제대로 못 합니다.
기본기가 되어있는 사람들이 그나마 잡서라도 소화시키지, 나머지들이야 휘둘리거든요.
이거 재밌는 게 당사자들은 제가 이런 말을 하면 반발합니다. 그런데 타자 입장 되어보면 바로 납득갈 겁니다.
\vspace{5mm}

6월 되어서 확신 안 선다 하는 분들은 깔끔하게 올해는 그냥 운빨 기대하면서 가고 내년 대비하는 게 낫습니다만...
경고드리면 올해 고2부터 이상하게 미친 듯이 잘 하는 경향이 있습니다(...) 개정교과 과정의 위력인지 학부모들 빨이어서 그런지 몰라도 그러합니다.
6월 되어서 힘들다 하는 분은 그냥 수학과 국어나 죽어라 파는 걸 권하고 싶습니다. 최상위권 실력이 아니면 언제 시험을 쳐도 답이 없으니까.
\vspace{5mm}

11월$\sim$5월에 공부를 제대로 안 한 사람이 과연 6$\sim$10월에 공부를 제대로 할 건지는 개인적으로는 심히 의문입니다.
\vspace{5mm}

이 글이야 작년에도 했던 소리이고 사실 개인적으로 트루라고 보기 때문에 다시 쓰는 건 괴로운 일이나
왜 이런 경고 안 했냐라고 욕먹기 딱 좋을 것 같아 다시 적습니다(...)
기출 돌리면 되지 않느냐... N수생도 풀지 않은 기출을 고2들이 다 풀었단 사례를 심심치않게 접하고 있습니다요(물론 상위권이겠지만)
차이가 커도 너무 크다고 생각해요.
\vspace{5mm}

사실 콕콕에서만도 몇몇 학생들이 쎈수학도 안 풀었다는 걸 보면서 한숨을 쉰 적이 있습니다만
진짜 최상위권들은 그런 건 거의 다 진작에 끝내고 공부 안 한 척 합니다. 인터넷에는 거짓 정보가 꽤 많이 올라오죠.
5월까지 그래도 공부하신 분들은 6월부터는 공부량 줄이고(더워지니까 당연합니다) 최대한 환경, 체력관리 신경쓰면서
모의고사를 포함한 문풀에서 오답정리, 분석, 그리고 불량해결을 철저히 하시길 바랍니다요.
\vspace{5mm}

6월부터 여학생들을 픽픽 쓰러지고 남학생들은 9월 정도 되면 다수가 멘탈붕괴되고
이과 하다가 올해는 안 되겠다 하면서 문과로 도망가는 일이야 한두건이 아니고
이제 애꿎은 교육청과 평가원 욕하고 무슨 모의가 좋다느니 아무개가 갓이라느니 허무맹랑한 '루저들의 사교파티'가 벌어질 겁니다.
\vspace{5mm}







\section{[수학교재] 풍산자 필수유형}
\href{https://www.kockoc.com/Apoc/784630}{2016.05.20}

\vspace{5mm}

선지와 문항만 다듬으면 쎈과 맞먹는데 그렇지 못 하다는 게 아까움
상 문제에 해당하는 건 신사고 라인이나 RPM과는 차별화되는 창의적인 문제들이 있음.
\vspace{5mm}

풀어보면서 참 잘 만들었는데 이거 출판사 뒷마무리가 부족하다라고 아쉬움.
\vspace{5mm}

올해 연구해보면서 꼽은 유망주.
\vspace{5mm}

\begin{enumerate}
    \item 올림포스 평가문제집
    \item 풍산자 필수유형
    \item 일품
\end{enumerate}
\vspace{5mm}

개정수학.
내용이 쉬워진 거지 문제가 쉬워진 것이 아니죠. 내용이 어려우면 오히려 문제는 쉬워질 수 있습니다(쓸 수 있는 잡기들이 많아지니까)
하지만 내용이 쉬워지면 논리가 명쾌하니까 더 꼬아낼 수 있죠.
\vspace{5mm}







\section{모 수학교육에 관한 책을 읽어보았는데}
\href{https://www.kockoc.com/Apoc/788825}{2016.05.23}

\vspace{5mm}

어느 학원인지 어떤 책인지 짐작은 다 가실 테고
\vspace{5mm}
\begin{enumerate}
    \item 기본교재는 정석과 블랙라벨 ; 쎈조차 부족하다
    \item 과학고, 자사고의 수업과정에서 쓰는 고급수학 등도 공부할 필요가 있다.
    \item 선행은 실속있게 해야한다
    \item 과학고생들이 보는 필독서 시리즈 같은 것을 읽어볼 필요가 있다.
\end{enumerate}
\vspace{5mm}

공부방법 면에서는 개념 복기를 해야한다 빼고는 새로운 게 없습니다.
주목할 것은 1번과 2번, 그리고 4번 같다고 보는데.
\vspace{5mm}

콕콕만해도 라벨은 커녕 쎈조차 풀지 않은 학생들도 널려있었고(그러면서 실모 얘기하는 것부터가 뭔가)
고급수학을 꼭 할 필요는 없지만 어느 정도 기본과정을 마친 분은 저걸 공부해둘 필요는 없지 않고(특히 수학적 모델링)
그리고 수학적 사고력을 키우는 데 과고생이 보는 필독서 시리즈가 다 좋은 건 아니지만 일부 책들은 확실히 인강 그 이상인 게 있습니다.
\vspace{5mm}



\section{더위 한방에 무너지는 공부}
\href{https://www.kockoc.com/Apoc/789055}{2016.05.23}

\vspace{5mm}

그러니까 사실상 5월까지라니까요.
지금부터는 의도적으로 학습량 절반 줄이고 체력보전에 신경쓰세요. 후회하지 마시고
에어컨으로 버틸 수 있다 해도 체력 떨어지는 건 못 막습니다.
이제부터 1순위는 무조건 \textbf{무더위 버티기, 그리고 6평을 치면서 '절망'을 일부러 맛보면서 인내하자}입니다.
6평 치고나서 그 결과 가지고 자살한다 뭐한다 그딴 드립치는 사람은 이미 그릇부터가 간장 종지만도 못 된다는 이야기입니다.
그냥 6평에서는 '절망감'의 예방주사를 맞는다고 생각하고 치세요.
\vspace{5mm}

남들은 열심히 하는데 그럼 어떡하느냐 그럴 건데 더위는 모두가 공평하게 겪습니다.
더위에도 불구하고 열심히 하는 사람들이야 원래 그런 놈들이니까 어쩔 수 없지만 이들은 극소수고
다른 사람들도 무너지긴 마찬가지이니까 '덜' 무너지는 게 이기는 겁니다.
괜히 만용부리다간 학습량이 저절로 줄어드니, 그냥 스스로 줄이시고 일부러 휴식시간 늘리세요
결과적으로 똑같지 않냐고 하지만 '컨트롤을 하느냐 못 하느냐' 차이는 상당히 큽니다.
학습량을 줄일 수 있어야 늘릴 수 있죠.
\vspace{5mm}

그리고 고2들은 사실상 겨울방학까지 다 끝내야한다는 걸 인지했을 겁니당.
\vspace{5mm}
