
\section{일격 레이드 들어갑니다.}
\href{https://www.kockoc.com/Apoc/246737}{2015.08.12}

\vspace{5mm}

http://kockoc.com/BookRaid/245328
\vspace{5mm}

제 팔은 절대 안으로 굽지 않습니다.
콕콕에서는 영리활동을 하지 않기 때문에 저야 거침없이 할 말을 합니다.
저런다고 일격이 나쁘다라고는 안 하겠지만, '아쉬운' 것은 당연히 지적하는 거죠. 그래야 내년에 더 좋아질테니까요.
제작자 분들이 단기적인 이익에만 눈이 멀었다고 보진 않아서요(불행하지만 그런 업자들이 너무나도 많습니다)
앞으로 좋은 교재를 만들고 싶으시면 하나하나 다 개선해나가셔야합니다.
\vspace{5mm}

불모지 탭에 보면 교재 사냥터와 교재 레이드가 만들어졌지요.
교재 사냥터에서는 마치 저평가 기업 고르듯 각자가 보는 소박한 교재 올리고 평가해보고 다른 사람들 반응 보시거나
아니면 지금 시작하고 있는 일격 레이드 등을 그대로 연동해서 따라가주시면 되겠습니다.
\vspace{5mm}

수험사이트들을 가보면 지나친 과장이나 혹은 까내리기가 상업적인 목적 하에 자행되고 있던데 그딴 건 당연히 할 리는 없죠.
그냥 차분하게 좋은 건 좋고 나쁜 건 나쁘고, 광고가 안 되었는데 괜찮은 건 평가해주고 하면 되는 것입니다.
\vspace{5mm}

현재 0회차 $-$ 즉 포장지 뜯고 해설지 훑어보면서 \textbf{개선점}을 제시했습니다.
사실 이건 만든 분들에게 꽤 엄하게 지적드리는 데, 저건 제작과정에서 조금만 신경쓰면 개선가능한 문제였다고 봅니다.
12회차까지 레이드 가면서 이제 개인 가격을 매기겠습니다만, 저것 때문에 우선 $-$10000원 계산을 할 것입니다.
\vspace{5mm}

그래도 적어도 손발이 사라진 건 다행(...)입니다. 뭐 그게 좋다고 하는 분들도 계시겠지만(...)
내일은 1회차 A, B형 간략 리뷰 올라가고 레이드 참여자들(현재 상원 국한)에 대한 지시가 있을 것입니다.
레이드에 참여 안 하시는 분들도 비슷하게 연동해서 가시거나 아니면 뜻맞는 분들끼리 진도 빼서 가시는 것도 괜찮겠죠.
\vspace{5mm}

그리고 다른 분들도 일격에 대해서는 $-$ 제가 제작자는 아닙니다만 $-$ 쓴소리를 아끼지 마시길 바랍니다.
\vspace{5mm}

+ 死만원으로 살 수 있었던 것
\vspace{5mm}

아, 물건너갔다.
\vspace{5mm}




\section{공부법 책은 사실 거의.}
\href{https://www.kockoc.com/Apoc/251344}{2015.08.14}

\vspace{5mm}

소위 공부하는 법 가르쳐줄께... 하는 케이스는 99$\%$ 장사치입니다. 그런 건 돈받고 파는 건 아니죠.
그러기 전에 본인들이 먼저 검증해야하지 않나 합니다만.
우리나라에서 공부 가장 많이 한 것으로 보이는 고시합격자 집단의 수기 읽어보면 새로운 것 그런 것 없어요.
결국 얼마나 시간낭비 줄이고 많이 (반복해서) 보느냐입니다.
\vspace{5mm}

일본의 공부법은 그나마 낫습니다만 이것도 별 것 없습니다.
일본의 한 미모(?) 변호사가 7번 읽기법 낸 거. 그걸로 끝입니다. 이것말고 다른 공부법 책들도 많습니다만 사실 무쓸모이죠.
뇌의 생리를 고려한다든가 더 효율적인 자료 정리한다든가.. 뭐 쓸모 없지 않은데 효율은 낮습니다.
그런데 이건 \textbf{"옆에서 잔소리해주고 빠따로 갈겨주는 사람"} 미만잡입니다.
\vspace{5mm}

그보다는 오히려
\vspace{5mm}

$-$ 전자파를 차단할 것, 컴으로부터 격리된 시간 가질 것
$-$ 자기가 약속을 어기면 고통을 입도록 시스템 짜놓을 것
$-$ 책상과 의자에 투자할 것
\vspace{5mm}

이런 게 중요하겠죠. 공부법 책 사둘 돈이 있으면 독서실을 끊든가, 컴퓨터 없는 방에서 쓸 수 있는 넓은 목재 table 사는 게 낫습니다.
사실 컴 얘기 나와서 그럽니다만 다 필요없고 \textbf{컴과 맛폰만 멀리해도 성적 올라간다는 게 불편한 진리죠.}
\vspace{5mm}

돌아다니다보면 별 것도 아닌 것 가지고 장사하려는 사람들이 꽤 많다라는 걸 봅니다. 수년 전부터 주욱 보였지만요.
\vspace{5mm}






\section{실모에 대해서 또 다른 생각 다시 적습니다만.}
\href{https://www.kockoc.com/Apoc/251556}{2015.08.14}

\vspace{5mm}

일격 1회 A,B형을 풀고 최대한 틀려보려고 하면서(어디서 실수할 것인가) 리뷰하고
해설 비교하면서 느낀 건데.
\vspace{5mm}

그 저자 분들의 생각과 달리 기존의 제 입장 $-$ \textbf{"야매교재 보지 말고 검증된 것 가지고 철저히 보아라"}하는 것
이 주장 고수해도 되겠네요.
\vspace{5mm}

기초가 탄탄하지 않은 상태에서 실모 보는 건 그냥 \textbf{자살행위}입니다.
문제가 좋네 안 좋네 그런 게 중요한 게 아니라
문제 하나를 틀리더라도 그걸 정리하면서 깨닫고 실력을 키우고 이런 게 중요하고
그런 차원에서 실모를 보아야하는 건데
\vspace{5mm}

뭔가 본말이 전도되어도 한참 전도되어버렸죠.
\vspace{5mm}

일전에 키배(?) 비슷하게 뜨면서 야매교재 옹호파(?)에서 그럼 일타, 일격은 왜 말이 없냐라고 하기에
그렇게 \textbf{입으로만 나대는 케이스} 경멸하는 차원도 있어서 지금 일격 하나씩 시간 들여가면서 리뷰합니다만.
1회차만 봐도 느껴지는데 전혀 제 생각은 바뀔 게 없을 것 같네요.
리뷰다는 것도 그렇자면 철저히 야매스러움을 벗어나고 품질강화하라는 조언입니다.
만약 판매량에만 신경쓰고 (이 시장은 정상적인 시장이 아닌 것을 아시겠죠) 자기가 스타라고 착각한다면
(본인들은 아니라고 하겠지만 제가 보기엔 딱인 걸 무슨) 그거 불행한 일이지요.
\vspace{5mm}

실모는 정 보려면 하나만 제대로 보시길요. 전 오히려 A, B형 1회 리뷰하는 것도 시간이 엄청 걸렸습니다.
\vspace{5mm}

+ 판매량 가지고 자랑하는 케이스 있던데 그건 사실 한심한 겁니다. 독자들이 평가해야지 업자가 그러는 건 웃긴 것임.
돌아다니다보면 그런 케이스도 있고 아마 여기도 예외는 아닐 것인데. 광고 차원이라면 모르겠고 출판사가 그러는 것까진 이해는 가는데
저자가 그러는 건 정말 쪽팔린 겁니다. 그럴 시간이 있으면 해설에 공들이세요.
\vspace{5mm}

+ 작년 말부터 시비 거시던 분들은 정작 하나라도 뭔가 기여한 건 없다는 것 재차 확인. 실천 안 하는 사람은 경멸하지 말입니다.
익게건도 그렇고 야매교재 논쟁도 그렇지만, 어설프게 시비는 걸지 자기가 상처입을 걸 두려워하는 그런 사람은 대놓고 경멸하지 말입니다.
\vspace{5mm}

+ 일격까로 읽힐지도 모르겠습니만, 더 정확히 말하면 까의 입장에서 지금 검토하고 문제삼을 건 다 삼아 지적하자 그 입장입니다.
1회차만 보면 문제는 킬러에 한해서 컨셉 괜찮은 것들이 있습니다.
물론 컨셉이 괜찮은 게 다는 아니고, 이게 정말 도움이 되는지는 더 신중한 입장을 취해야지요.
그런데 해설은 더 손봐야할 건 있어요. 노력한 흔적은 보이는데 그것만 가지고는 무리입니다.
최종 마무리가 약간 부족하다... 라는 아쉬움이 있습니다. 그 점에서는 제가 쓰는 보고서가 약간은 도움이 되었으면 합니다.
문제차원에서 보자면 보면 볼수록 괜찮다... 하는 것들은 있습니다. 그런 건 아예 가격을 주관적으로 산정해 넣었고 근거를 대답할 수는 있습니다.
\vspace{5mm}

이거 점수 안 나왔다고 좌절하시지 마시고, 약재 고아먹듯 문제 자체를 계속 여러번 풀고 생각해보세요.
\vspace{5mm}

+ 이것도 얘기해야겠는데 야매교재 논쟁 중 하나가
쎈수학으로 부족하다, 실모가 최고다. 그런 이야기인데. 이거 도무지 근거가(...)
그 분들 쎈수학이라도 제대로 공부하지 않았다 쪽으로 확신이 듭니다.
실모도 실모 장점이 있겠지만 내용이나 회독수 상승시 얻을 수 있는 것으로선 쎈수학, 정석 등을 따라잡을 게 없어요.
\vspace{5mm}

+ 더 공포스러운 게 실모 판매량이 증가한다.... 뭐 다 좋은데
판매량이 많다는 건 '기본적인 공부도 안 한 친구들이 실모를 본다' 그 이야기인데.
이거 업자들이야 그렇다 치고 소비자들은 자기들이 어떤 상황에 가는지 알지는 모르겠습니다.
그저 웹에서 문제 좋다 하는 것에 혹해서 자기 상황 모르고 구입하면 어떻게 되는 걸까.
\vspace{5mm}

+ 한편 쎈 등의 시중교재, EBS 본 사람에게는 일격 실모는 도움이 된다는 건 분명합니다.  겹치지 않는 부분들이 있어보여요
다만 그 겹치지 않는 부분은 시중교재와 EBS 등을 충실히 했을 때 그 \textbf{진가를 알 수 있다}는 게 함정이지만요.
\vspace{5mm}







\section{국어는 걍 답이 없습니다.}
\href{https://www.kockoc.com/Apoc/261711}{2015.08.19}

\vspace{5mm}

http://kockoc.com/column/261153
\vspace{5mm}

간만에 칼럼에 좋은 글이 올라와서 토론하다 느낀 것이죠.
\vspace{5mm}

반발하는 사람들도 많겠지만 일단 적으면
수학 실모는 만들기 쉬운 편입니다. 왜냐면 참조할만한 소스가 꽤 많고, 더군다나 조금만 꼬아내도 오류가 생길 일이 별로 없어요.
(당사자가 실력이 거품이 아니라면 말입니다)
\vspace{5mm}

그런데 국어는 그게 아니죠. 잘 내더라도 오답 시비의 가능성이 잠재하고 있습니다.
왜냐면 국어는 화작문을 제외한다면 절대 정답이 하나만 나올 수 있는 과목이 아니거든요.
그게 국어란 과목의 본질이기 때문.
\vspace{5mm}

그래서 국어는
$-$ 사설인강, EBS인강
$-$ EBS 문풀이나 실모
$-$ 기출
\vspace{5mm}

그 어느 것도 답이 되기 어렵습니다.
애매한 걸 굳이 억지로 얘기한다면 "상식적인 수준에서의 국어적 사고"라는 게 있긴 하겠고
그게 기출을 통해서 키워지기는 하는데 한계가 많죠.
\vspace{5mm}

간혹 수험사이트들 돌면 국어 실모나 문제가 얼마나 개판이냐... 라는 지적을 봅니다만
그거야말로 오만한다고 봅니다. 그거 제작자나 그런 사람들 제가 알 리는 없으니 옹호해주는 건 아니고(그렇다고 구입하란 얘기도 아니지만)
수학 실모가 난립하는 반면 국어나 영어는 별로 없거나 '변형'이라는 이름으로 짜깁기 가는 건 다 그럴만한 이유가 있는 거죠.
만들기 어려우니까 말입니다(다만 만들기 어렵다면 만들지 말든가, 제대로 만들면 되지 않나)
\vspace{5mm}

그래서 그동안 국어는 평가원에서도 꽤 난이도를 자제한 편이긴 한데
작년 시험 기점으로 수학 뿐만 아니라 국어도 $\sim$ 하게 출제하면 된다라는 것을 눈치깠다고 확인할 수 있는 정황이 포착되죠.
수학은 즉 쉽게 내면서도 나름 변별력을 줄 수 있다라는 것 $-$ 쉽게 내는 게 추세긴 하지만 $-$ 을 알아챈 것 같고
국어도 역시 어떻게 하면 오답시비 안 내면서 수준있게 낼 수 있는지도 파악한 것 같더군요.
여기 쓰긴 그렇습니다만 그런 정황은 역시 일본 쪽 입시문제에서 발견된 것이긴 한지라.
\vspace{5mm}

지금 고3이면 모르겠고 고2라면
요즘 독학용 교재도 많이 나왔으니까 이과라고 하더라도 인문논술의 사고법 정도는 익혀두시는 게 좋지 않나가
지금으로서 드릴 수 있는 최선의 조언일 겁니다.
비문학, 문학 독해는 문제를 보자마자 머릿 속으로 미니논술을 작성하지 않으면 안 되는 식의 문제풀이로 갈 가능성이 높습니다.
영어의 빈칸추론이 이미 그런 수준이지만요.
\vspace{5mm}

+ 상관없어보이는(?) 인문학 질문이 나와서 그러는데 제 대답은 그렇습니다.
"그 인문학자들이 헤게모니를 잡고 있던 국가나 사회는 부유해졌나"
\vspace{5mm}

소크라테스야 왕따당해서 사실상 살해당했고
플라톤은 군주 한명 과외제자로 잘 가르쳐보려다 배신당했고(그러기보다도 라톤이 형은 덕후였잖아)
아리스토텔레스는 제자가 무려 알렉산드로스여서 요새로 치면 마케도니아 스터디 원장으로 잘 나갔다가
물수능, 아니 외지인 추방정책으로 물 더럽게 먹었죠
\vspace{5mm}

애시당초 인문학만으로 모든 게 해결되었다면 옛날 국가들이 삽질을 겪지는 않았겠고
경제학이 나타날 이유도 없었겠죠.
조선왕조 500년이 왜 궁핍하고 가난하게 살았는지만 봐도 좋습니다.
\vspace{5mm}

+ 수학 교육의 의의는 저런 인문학의 '광기'를 막아주기 위한 브레이크라고 해도 사실 지나친 말은 아닐 겁니다.
아마 수학을 가르치지 않으면 '답이 없는 국어' 공부만 하면서 또 이 지옥불 반도는 21세기 예송논쟁이나 하겠고... 가 아니라
문돌이 어르신들이 쓸데없는 것 가지고 답없는 논쟁 하는 것 보면 끝이죠.
\vspace{5mm}








\section{공부 못 하는 애와 잘 하는 애의 결정적인 차이.}
\href{https://www.kockoc.com/Apoc/284872}{2015.09.02}

\vspace{5mm}

하위권이 중위권으로 올라서거나 중위권이 상위권이 되기 위해서
상위권이 최상위권이 되는 과정은 태권도의 다리찢기와 비슷한 단계를 반드시 거쳐야 한다.
\vspace{5mm}

가령 중위권이 하루에 푸는 문제량이 `100개라고 하자
그런데 하위권 애들은 공부하겠어요... 하지만 80개 정도를 일주일 정도 지속하면 짜증을 내거나 그만두려고 한다.
마찬가지로 중위권 애들을 억지로 상위권 코스에 맞춰 150문제에다가 킬러 10문항을 풀게 하자
머릿 속에 안 들어간다고 하면서 공부를 거부하려 한다
\vspace{5mm}

'말은 잘 하는 사람'은 신용하기 힘들 것이다.
그런데 그렇게 따지면 학생들의 신용수준은 개차반이다.
공부를 잘 하고 싶다라고 하지만 실제로는 조금만 공부하는 상태를 만들어놓으면 그 유지를 하지 못한다.
\vspace{5mm}

공부를 잘 한다라는 건 단지 좋은 인강과 교재만 있다고 되는 것이 아닌 것이다.
그걸 소화할 수 있도록 학습량이 늘어야 하고, 그 학습량을 소화시킬 수 있도록 그릇을 바꿔야 한다.
그 점에서 보자면 가장 중요한 건 '성격'이고, 그렇다면 \textbf{공부를 잘 한다라는 건 '성격'을 바꾸는 것이다.}
아마 학생 개개인들은 그걸 알기 힘들 것이고 사실 관심도 없겠지만,
여러 사람들을 $-$ 최소 5명 이상 경험하다보면 교재도 스킬도 강의도 아닌 '성격'(그리고 그걸 좌우하는 집안환경과 유전)의 중요성을 알게 된다
\vspace{5mm}

이상과 달리 현실은 "그 자식은 그렇게 망할 수 밖에 없어"라는 탄식이 나오는 것도 그런 것이다.
그 점에서 보자면 인강이나 대형학원 강의는 준비할 게 많다 해도 편한 것이 있다. 강의를 듣던말던 학생들의 성격에 신경쓸 필요는 없기 때문이다.
\vspace{5mm}

그럼 공부를 못 하는 애들이 머리가 나쁘고 철학이 없나
짤방대로이다. 머리가 좋은 케이스가 많고(오히려 좋으니까 공부를 안 한다는 게 문제)
개똥철학은 정말 많아서 부모님이 해주는 밥을 먹고 용돈을 쓰면서 개똥철학을 논하는 케이스가 많다.
그렇기 때문에 "의미없고 자유가 없는 공부"는 할 필요가 없다고 자기 정당화를 한다.
유감스럽지만 수험은 '자유를 포기하고 의미 같은 걸 추구하지 말아야' 성공한다는 진리를 모르거나, 알면서도 인정하지 않으려는 것이다.
\vspace{5mm}

그 점에서 보자면 중2병적인 것을 촉진하는 일종의 인문학적인 개똥철학이나 소설작품 등의 해악을 부인하기는 어려울 것이다.
각자 자아를 찾아라, 의미를 갈구하라, 그리고 자유를 추구하라... 감동적이긴 한데 사실 '생산$-$소비'의 틀로 본다면 허무맹랑한 이야기다.
정말 자아를 알고 싶으면 우리가 뭘 자유롭게 '생산'해서 그걸 '소비'와 '투자'로 연계시킬 수 있느냐 따져야지
그런 게 되어있지 않은 이상은 순전 말뿐인 개똥철학은 뇌내망상에 불과한 것이다.
그래서 일찍 철이 든다는 것도 좋은 건 아닐 수도 있다. 수험판에서는 고득점만 인정될 수 있기 때문이다.
\vspace{5mm}

결국 자기가 생각하는 것 갑절 이상의 학습량을 유지하다보면 필연적으로 찾아오는 그 '장애'를 넘어서는 수 밖에 없다.
한마디로 자기의 한계를 넘어서야하는 것이다.
50문제 풀면 공부하기 싫어지고 다 때려치우고 싶어하는 근성, 조금만 공부하다가 바로 인터넷 접속하거나 딴짓하는 습관.
중하위권들은 이런 습관을 고치지 못 하면서 좋은 교재와 강의만 찾는다.
\vspace{5mm}

그럼 사설강의는 어딴가. 결론적으로 어떤 것이든 '달콤'하다.
실제로 강의를 듣고 공부가 재밌어졌다하는 것은 인정할 측면이 있짐나 반면 위험한 것도 있다.
인기있는 강의란, 실력을 높여주기보다는 오히려 '들을 때 피로감이 덜 하고 재미있는 강의'일 가능성이 크다.
그런데 그런 강의가 재밌는 이유는, 특정 단원의 특정 내용을 공부할 때 반드시 거쳐야하는 '어려운 과정'을 생략하거나 꼼수로 넘어가기 때문이다.
또한 특정한 내용은 단지 이해만 할 게 아니라 정말 지루해도 암기해야 하며, 부단한 연습이 필요한 경우가 많다.
하지만 인강이 이런 것까지 책임져주기 기대할 수는 없는 것이다.
\vspace{5mm}

중하위권이 인강을 듣건 학원을 다니더라도 '성적'이 안 오르는 것이 이렇게 설명되는 것이다.
괴롭고 힘들고 짜증나는 대목, 즉 자기의 한계를 넘어서는 대목, 헌독을 깨고 새독을 빚는 대목
이런 것들을 3$\sim$4개는 거쳐야만 비로소 공부할 수 있는 사람이 되는 것이지 그렇지 않으면 '개똥철학 읊는 잉여'로 전락한다.
아마 이 글을 읽는 상당수는 자기는 안 그럴 거라고 생각하겠지만 정말?
\vspace{5mm}

분명한 건 수험은 자기 의미를 찾는 과정도 아니고 자유는 일찌감치 포기해야하는 비인간적인 경쟁이란 것이다.
이 점을 수긍하지 않는다면 그냥 수험을 포기하고 다른 길로 가는 게 낫지 않나... 하지만
자본주의 사회에서 $-$ 아니 인류사회에서 그런 동물의 왕국급 경쟁이 없는 곳이 어디있을까.
\vspace{5mm}






\section{2016년도 9평 A형, B형 수학 분석}
\href{https://www.kockoc.com/Apoc/285090}{2015.09.02}

\vspace{5mm}

B형 총평 : 30+0$-$5
팩트 : 평가원은 쉽게 내면서 실점 나오게 하는 방침 터득
\vspace{5mm}

B형 문항별 분석
\vspace{5mm}
$
1$\sim$6 : 생략. 그런데 이걸 내면서 출제자는 어떤 기분이 들까.
7 : 이거 식 말고 그림으로 푸는 게 좋죠. 1차변환 고수들은 행렬을 안 쓸 수 있으면 안 쓴다능.
아마 7번 틀렸으면 보나마나 3과 3root(2)를 헷갈렸을 겁니다.
8 : 이거 x 범위 제한 둬서 난이도 높이지
9 : 생략
10 : 이런 거 틀리는 애들도 있죠
11 : 익힘책 수준
12 : 좌표접선과 기울기접선 공식은 잘 활용. 그런데 이런 건 주관식으로 내면 더 좋았을 듯
13 : 생략
14 : 부피구하는 문제 :  출제자가 너무 착해서리 (e^2$-$1)pi 를 선지에 안 넣었죠.
15 : ★ 쉬운 확률 문제 : 단, 식으로 접근하려던 친구들 시간 많이 잡아먹었을 겁니다. 오답률 생각보다 높지 않을까
16 : 점화식 문제 : 그나마 중간 정도 되는 문제였는데 이것도 힌트가 많아서리.
17 : 9평에 마지막으로 등장하는 행렬 합답형. 어째 실전에서는 어렵게 내지 않을까, 역시 쉬웠습니다.
18 : ★ 오답률 높다고 보는 통계문제 P(Y>=26)>=0.5라는 조건 안 쓰면 엉뚱한 답이 나옵니다. 선지도 착하지가(?) 않습니다.
19 : ★★ 업그레이드해서 30번으로 내도 좋았을 2차곡선 문제 P의 위치를 하나로만 생각하기 쉽죠.
20 : 뭔가 무색해지는 무등비 쉽습니다.
21 : 19금 장면이 나올 찰나에 스탭롤이 올라가는 적분그래프 문제입니다(...) 이거 틀린 사람들 이불은 오늘 무사하실까?
22 : 생략
23 : 생략
24 : 생략
25 : 생략
26 : 삼각형 넓이 구하는 방법을 묻는 문제. 정사영은 그냥 장식품
27 : ★★ 그래도 꽤 흡족한 중복조합 문제였습니다. 요즘 중복조합은 케이스나누기 필수
28 : 불멸의 사인법칙
29 : ★★ 푸는 방식이 다양합니다. 법선벡터 풀이도 있고 아니면 평방 풀이도 있음. 평방 풀이시 '내분' 잘 이용하세요
30 : '$\sim$의 정리'만 잘 쓰면 끝나는 싱거운 문제
\vspace{5mm}

A형 총평 : 28+2$-$3
팩트 : B형이 어렵다고 A형으로 도망간 학생들 자살각
\vspace{5mm}

1$\sim$6 : 생략
7 : 계차수열 정의알면 한줄 풀이
8$\sim$11 : 생략
12 : 좌표 계산할 수 있느냐 하는 기본문제
13$\sim$14 : 생략
15$\sim$16 : 생략
17 : 문과 수열치고는 대략 중간 정도
18 : 합답형 쉽게 나왔네요
19 : 케이스 구분은 중복조합 문제에선 이제 필수
20 : ★ 대소관계 잘 파악해서 무등비
21 : ★★ t의 범위에 따라 함수, 도함수 정의한 뒤 문제에 제시된 부등식의 의미 파악하고 풀면 됨
22$\sim$26 : 생략
27 : ★ 계산 충실히
28 : 생략
29 : ★★ 정규분포의 정의를 아느냐 물어보는 문제
30 : ★★★★ 통수갑. B형에 냈으면 변별력 좋았을 문제임. 숨겨진 m의 범위를 찾아라
30번 문제는 저도 헤맸는데 '가수'는 자릿수와 독립해있지 본다면우리의 상식을 넘어서는 자릿수를 만들 수 있음.
B형 풀고 싱겁네 하는 분들도 A형 30번은 풀어보시길.
\vspace{5mm}$

\textbf{사견}
\vspace{5mm}

\textbf{​EBS 수특, 수완만 꾸준히 풀었어도 200점은 나오는 출제였습니다.}
(반론하고 싶으신 9평 문제와 직접 비교해보면 끝) 난이도로 비교하면 절반도 되지도 않았기 때문.
'기본'적인 걸 소홀히 하면 어이없이 틀릴 수도 있었던 문제들이 많았죠.
\vspace{5mm}

그런데 9평의 목적은 결국 막판 난이도 조절을 위해서 고3들 수준이 대략 어떤가 시험해보는 것이고
그런 차원에서 쉽게 내지 않았나 싶기도 한데, 제가 아는 한 올해 고3들은 수준이 낮지 않은 편이라서리.
그러면 본 시험은 쉽게 나오리라고 장담하기는 어려울 것 같습니다.
\vspace{5mm}

조언
\vspace{5mm}
\item \textbf{실점한 문제에 관해서는 까다로운 것 100문제 정도 풀어보시길}
예컨대 A형 30번에서 나가리났다하면 지표와 가수 관련된 4점짜리 문제만 골라서 풀어보라 그런 이야기
선택과 집중이 필요합니다.
\vspace{5mm}

\item 2. A 문제는 $\sim$ 하게 푼다는 스킬보다는 문제 리딩을 얼마나 잘 하느냐가 성패를 좌우하고 있습니다.
이건 제 실모비판론과도 맥락이 닿습니다만, 현재 자알 팔리는 교재들 중에서 '리딩'을 도와주는 교재는 별로 없어요.
틀린 문제에 한해선 교과서와 개념서 정독을 해보시고 어떻게 리딩을 할까 궁리해보시길 바랍니다.
B형 21번과 30번은 리딩만 잘 하면 그냥 풀었으니까요.
\vspace{5mm}

\item 3. EBS 무시하지 마세요. 아마 저보고 EBS를 맹목적으로 추종하느냐 어쩌냐하겠지만
제 경우도 매일 수학문제집을 풀고 검토하고 연구하고 있습니다. 그나마 평균적으로 나은 게 EBS라서 그렇습니다.
이번 9평이 본 시험과 일치한다고 보기는 어렵더라도, 출제 경향은 현재 EBS가 그나마 잘 반영하고 있다는 데 유의하시길요.
\vspace{5mm}

\item 4. 점수가 낮게 나왔다고 좌절하지 마시길요.
다시 풀어서 맞는 문제라면 그건 가망이 있습니다. 다만 컨디션이나 멘탈, 무엇보다도 실전연습이 덜 되어있어서 그래요.
꼭 실모 풀 필요 없이 시간 재서 수학문제 푸는 훈련을 계속 하십시오.
\vspace{5mm}





\section{수재의 조건}
\href{https://www.kockoc.com/Apoc/287758}{2015.09.03}

\vspace{5mm}

\item 1. 매일 예습복습을 꾸준히 한다.
\item 2. 거창한 계획을 세우지 않는다. 실천할 수 있는 단 한가지 약속만 지킨다.
\item 3. 자기 주장이 약하다, 그러나 실천은 강하다.
\item 4. 참고서 10권을 한번 보기보단, 1권을 10번씩 본다.
\item 5. 결과가 나올 때까지 최소 6개월, 길면 1년 반까지 기다릴 수 있다.
\item 6. 자기가 상위권에 속했다는 자부심 하나로 스트레스를 푼다.
7. 남들이 10문제를 풀면 50문제를 풀 준비를 한다, 결국 남들보다 많이 하려 한다
8. 필기, 메모광인 경우가 많다.
\textbf{9. 자기의 문제점부터 인식하고 타인에게 그런 지적을 가감히 듣는다.}
\vspace{5mm}

재수삼수해도 안 되면 저 체크리스트 중 과반은 날라간 거라고 보면 되지 않을까.
어른들이 사위나 며느리를 맞아들일 때에 상대 집안을 본다는 꼰대스러운 건 시대착오적으로 보이기도 하지만
사실 일리가 없는 건 아닌 게, '환경'이 실제로 그 사람의 진짜 '성격'을 암시하는 경우가 유감스럽게도 대부분이기 때문이다.
이걸 가지고 유전이라고 하는 경우도 있지만 그건 유전을 잘못 이해한 결과가 아닌가 싶고,
오히려 더 중요한 건 "성격"이라는 게 이 글을 쓰는 자가 현재까지 종합해 본 결론임.
머리가 나쁘다 하더라도 수험에 맞는 성격이면 어찌되든 잘 되지만, 반면 머리가 좋은데 수험에 안 맞는다면 소용없다는 생각.
\vspace{5mm}

이게 현재 중요한 이유는
과거에는 가부장 체제에다가 뭔가 군사독재 분위기였는데, 역설적으로 이게 학생들을 '수험생'으로 최적화시켜주었음.
의심하지 말고 시키는대로 받아적고 암기하고 시험쳐라, 못 하면 몇 대 맞고 숙제하고 다시 보라.
수험의 정답은 저것임. "단순하고 무식하게 공부"하는 것.
\vspace{5mm}

그런데 지금은 그런 구속적인 분위기가 사라짐.
적어도 과거에 비해선 정말 자유로워졌음, 그런데 이게 수험에는 좋지 않음.
그런데 문제는 '상류' 집안이거나 상위 '중산층' 집안은 그 부모들이 자녀들을 보수적으로 키움.
이게 마치 해저심층수처럼 장기간 그 학생의 수험에 영향을 주는 것임.
인내에 익숙하고 하라는대로 일단 할 줄 암.
반면 자유롭게 키워진 수험생들은 유감스럽지만 인내심이 정말 약하고,
자기는 안 그런다고 하지만 '상술'에 정말 잘 낚임.
\vspace{5mm}

똑같은 대한민국이니 뭐니하지만 아파트 단지별로도 소위 수준이라는 게 차이가 '극명히 나는' 게 슬픈 현실임.
대한민국 아줌마 극혐 치맛바람 왜 휘둘러 네이버에서 개념 댓글 달리지만 사실은
대한민국 아줌마의 최상위 종족인 복부인은 세계 최강이며, 또한 교육열 역시 세계에서 날리는 수준임.
이게 근거가 절대 없는 게 아님,
\vspace{5mm}

논지와 상관없는 썰 보충하면 복부인들은 정말 돈냄새 기막히게 잘 맡고, 기획재정부도 모르는 돈의 흐름과 향방도 직감적으로 예측함.
그래서 정부가 시키든말든 자기들이 느낀대로 투자함. 실제로 그들의 수익률이야말로 어떤 금융회사보다도 나을 것임.
\vspace{5mm}

이건 교육열도 마찬가지임. 자녀를 키우는 본능이라는 게 있어서 그런지 모르지만
남자들은 아파트 단지나 환경이 뭔 소용이냐 하지만 여자들은 그런 데 민감해서 그런지
돈만 있으면 조금이라도 더 좋은 환경으로 옮겨가서 자녀들도 업그레이드시키려고 하고 실제로 그런 성과를 거두고 있음.
물론 황새 따라하는 뱁새도 없는 건 아니겠지만 평균적으로 본다면 절대 무시할 수 있는 수준은 아님.
\vspace{5mm}

아마 본인들이 공부를 잘 한다라고 착각하겠지만
어떤 친일시인의 말대로 7할은 가정환경이 빚어낸 것임.
\vspace{5mm}

그럼 n수생으로 가봐서 관찰하면 실패하는 사례 예를 들면 그럼
\vspace{5mm}

첫쨰, 자기가 하고싶은 공부만 하려한다
둘째, 자기가 점수가 잘 나오는 과목을 공부하려는 경향이 있다.
\vspace{5mm}

독학이 위험한 이유이기도 함. 자기가 하고싶은대로 공부하기 때문에 실제로 점수를 높여주는 공부를 안 한다는 역설이 발생함.
국어가 약하면 국어, 탐구가 약하면 탐구에 바로 투입해서 남들의 10배는 해야하는데 그렇지 않고
그럼 수학은 어쩌죠? 영어는 어쪄죠? 하면서 또 우왕좌왕하다가 귀중한 시간을 날려먹음.
\vspace{5mm}

셋째, 자기의 스타일을 고집하기 때문에 개선을 하지 못 한다.
넷째, 문제풀이 자체가 결국 '자신의 성격'이 칠할을 좌우하는 것을 모른다.
\vspace{5mm}

가장 중요한 건 자신이 문제푸는 스타일을 객관적으로 관찰하고 교정받아야하는데 이게 쉬운 일은 아닐 거임.
그런데 공부를 해도 점수가 올라가지 않는다면 그건 '사고'와 '실행'의 문제고, 이걸 좌우하는 건 자신의 성격임.
예컨대 성급한 사람이 실수를 안 할 리가 없고, 처음 보는 문제에 겁부터 먹고 바로 공황장애 빠지는 경우면 어쩌겠나.
\vspace{5mm}

이것만 봐도 처음에 뭘 해야할지는 답이 간단한데...
문제는 이걸 고치라고 조언을 해도, 사실 고치는 경우는 그리 많지 않다는 것임.
"나는 너무 소중하니까".
그냥 공부만 많이 하면 다 해결될 거라고 착각.
\vspace{5mm}

물론 수재의 조건을 일률화시킬 수 없음. 그러나 최소한 지켜야 할 것들이 존재함.
특히 자신의 성격이 저런 것과 위배된다면, 당장은 호전될 수 없을지라도 바로 고치거나 보완할 필요가 있음.
\vspace{5mm}

어제 수학 B형의 경우만 하더라도 조금만 생각해도 풀릴 것을 어려워보인다고 손도 못 댄 케이스가 생각보다 많았는데
이게 지식의 부족 탓이기만 할까.
\vspace{5mm}








\section{수학 B→A형 돌릴 때 참조하실 것.}
\href{https://www.kockoc.com/Apoc/296976}{2015.09.06}

\vspace{5mm}

B형 수학은 온갖 무기가 동원되는 서바이벌 게임입니다.
그래서 교과외적인 무기도 $-$ 쓸모가 있다면 써도 되는 겁니다.
총 대신 탱크를 몰고가도 되고 무인기를 써도 좋습니다. 안 되면 주먹으로 가격해도 되서 꼼수나 스킬이 일정 정도 먹힙니다.
\vspace{5mm}

다만 요즘 와서는 화력보다는 '스나이핑'을 더 강조하는 분위기입니다.
꼼수나 스킬의 효용도 떨어지고 있어요. 문제가 쉽고 적고 떠나서 정말 교과서 개념 충실한 친구에게 유리해졌습니다.
\vspace{5mm}

반면 A형 수학은 맨주먹 빼고는 아무 것도 쓸 수 없는 격투기입니다.
격투기이면서 형식을 정말 잘 지켜야하기 때문에 멱살 잡는다거나 할퀸다거나 그런 거 안 먹힙니다.
흔한 \textbf{지수로그 지표 가수, 격자점, 그리고 문과 미적분에는} 꼼수는 더더욱 안 먹힙니다
A형 쉽지쉽지 그러는데 요새 킬러 수준으로 치면 A형 30번이 B형보다 더 어렵다고 느껴지는 경우가 많습니다.
B형 잘 푸는 친구들이 A형 격자점에서 헤헥대는 경우는 널렸습니다. 특히 성격 급하고 계산 실수 잘 하고 문제리딩 못 하면 답이 없죠.
\vspace{5mm}

본인이 자잘한 잡스킬 모르겠다, 기벡은 정말 공간감각 없어서 안 되겠다.
하지만 논리력은 명쾌하게 식을 통한 접근 정말 잘 하며 정리갑이다라고 하면 A형 가신 다음에
시중 교재에서 지수로그와 격자점만 확실히 정복하시면 100점은 어렵지 않을 것이지만
그게 아니라 잔실수가 많고 멘탈 개판이고 그렇다라고 하면 먼저 그 습관부터 고치시길 바랍니다.
\vspace{5mm}

최소한 제가 검증해본 바로는 수학은 지식보다는 이 역시 '성격'의 문제입니다.
중학교 때 숫자와 기하 감각을 졸라게 익하고, 고교 진학 전에 그리스 소피스트처럼 묻고 따지고 생각하는 싸가지없는 습관 들이면서
모든 것의 답을 궁구하려는 "왜?"라는 질문을 던진다면 고교수학을 잘 하는 것이고
저 중 하나라도 안 되면 운이 좋으면 모를까 그렇지 않으면 매우 힘들어지는 것이지요.
\vspace{5mm}






\section{이 시점에 감성파든 열심히 한다 하지 마셈.}
\href{https://www.kockoc.com/Apoc/418857}{2015.10.16}

\vspace{5mm}

\textbf{"아, 열심히 해서 n수 하지 않을 거야"}
냉정히 말하겠음. 부질없는 이야기입니다.
\vspace{5mm}

\textbf{"노력과 운과 상관없어"}
이거 수험이든 뭐든 겪어보지 않아서 하는 이야기입니다요.
\vspace{5mm}

저런 식의 메시지는 마지막 소중한 시간조차도 부질없는 감성에 젖게 만들며
본인의 실책을 극복 못 하고 결국 '정신승리'에 빠지게 하는 길입니다.
\vspace{5mm}

공부 초에는 만점받을 수 있어 얼마든지 공부할 수 있어라고 자신을 과대평가하다가
공부 말에는 이제 나 어떡하지라고 발 동동구를 거라고 작년 말에 \textbf{경고}했습니다.
\vspace{5mm}

그러니까 뽕맞지 말고 일찍 공부하라 나중에 시간없어서 힘들 것이다
그냥 생각없이 양치기 하고 빨리 오답정리하는 게 낫다... 라고 얘기했습니다.
\vspace{5mm}

시험은 아직 안 끝났으므로 단언은 안 합니다. 그러나 저 예측은 거의 다 들어맞고 있습니다요.
다만 '운'이라는 게 있기 때문에 아직까지 확언할 수 없지만
쪽지 보내시는 분 중에서 충고대로 하신 분들은 그래도 나름 모의에서 고득점 나오면서도 뭘 풀까라는 고민을 하시는 반면,
뒤늦게 공부시작해서 우왕좌왕하시는 분들은 자기 인생을 넘어 남들까지 탓하고 있습니다.
\vspace{5mm}

대한민국은 소말리아도 시리아도 아니지요. 님들이 공부를 못 한 게 아니라 \textbf{'안' 한 것입니다.}
공부를 안 해서 성적이 안 나온 것이라면 그건 정의로운 결과이니 거기에 슬퍼할 필요가 없습니다.
올해 목표를 못 이루었다면 다시 시작하거나 다른 길로 가면 되지, 별 이상한 감성에 빠질 이유가 없죠.
하지만 운이 나쁜 게 분명하다면 그건 올해 운이 안 좋은 것인데 불운이 내년에 다시 반복될 가능성은 높지 않으니
운이 나쁜 건 어쩔 수 없다, 다만 그 나쁜 운을 밀어낼 만큼의 노력을 하지 않았다라고 인정하고 다시 공부하면 되는 것입니다.
\vspace{5mm}

자기가 시험을 잘 치를지 못 치를지는 '무의식'이 정말 잘 알고 있죠.
그 무의식이 '야, 넌 올해 힘들 거야'라고 하면 그걸 인정하기 싫어서 다들 현 시점에 감성주의에 빠집니다.
마치 술을 마셔서 현실을 잃어버리듯 그렇게 '현실부정'을 하면서 난 잘 될 거야...
\vspace{5mm}

\textbf{'잘 될 거야'가 실제로 현실극복에 도움울 준 사례는 단 한번도 없습니다.}
\vspace{5mm}

오히려 '이래도 힘들어, 난 안 될 거야'라는 비관주의가 잘못된 자아를 바꿔 도움준 사례는 있을지 모르지만요.
마음을 비우고 수험의 본질을 돌아가십쇼.
\vspace{5mm}

\textbf{국어 $-$ 45, 수학 $-$ 30, 영어 $-$ 45, 탐구 $-$ 40을 8시간에 걸쳐 풀어내는 작업입니다.}
본인들의 인생을 좌우하는 건 어렵다는 킬러문제, 그리고 본인이 헷갈리거나 공부하지 않은 문제들입니다.
기본적인 문제를 다 풀어내고 저런 문제를 풀어내기 위해 공부하는 겁니다. 남은 시간이 촉박하면 가장 중요한 것에 할애하시는 것이지요.
물론 올해 공부를 열심히 했다고 자신이 생각하더라도, 전략이 틀렸으면 저 작업에서 실패할지도 모릅니다.
결과는 노력한대로 나옵니다. 다만 그 노력은 '타인'이 봐도 대단하다라고 느껴져야하는 거지, 본인'만' 인정하는 건 소용이 없지요.
\vspace{5mm}

\textbf{노력을 정말 한 사람이 노력을 까는 경우는 없습니다.}
\textbf{노력을 하지도 않은 인간들이 노력도 필요없다고 하죠.}
\textbf{노력 까는 사람들은 과거에도 현재에도 미래에도 노력과는 인연이 없죠.}
\vspace{5mm}

그럼 20일이 남았는데 어쩌냐 하는데
이 20일은 과거 100일의 효과를 누릴 수 있는 시기입니다.
일단 기온이 쌀쌀해서 공부하기 좋고 시험이 코 앞이라 긴장되며 그동안 공부한 게 쌓여서 가속효과를 발휘할 수 있습니다.
다시 말해서 본인들이 가속효과를 누릴 수 없으면 올해 시험은 그냥 마음편히 보는 게 낫다는 이야기일 겁니다.
\vspace{5mm}

아울러 이제 정말 웬만한 사이트 출입하면서 감정적으로 흔들리는 건 피하십시오.
하라는 공부는 안 하고 들락나락거리면서 잡글의 감성주의에 빠져서 올해 정말 낭비 안 했다라고 양심에 얘기할 수 있습니까?
전 격려글조차도 쓰지 않을 겁니다. 내용도 뻔하지만 이건 아무런 도움도 안 되기 때문입니다.
적어도 타인이 보기에 공부를 열심히 했다라고  평가받는 좋은 결과가 나와야하는 것이고
본인은 했다고 하지만 타인이 보기에는 저 놈은 왜 이렇게 쓸데없는 짓을 많이 하냐 하는 사람은 망해야 '정의로운' 것이 아닌가요?
시험공부도 대충 해놓고 나중에 내 인생 흐흐흑하면서 시험 잘 보길 바라는 사람은 도둑놈입니다.
그런 사람이 좋은 대학에 가면 우리나라는 그만큼 힘들어질 거라고 하는 건 과장은 아니라고 생각합니다만?
\vspace{5mm}

마음 비우시고 20일동안 후회 없는 공부하시길 바랍니다.
그리고 시험 당일날은 잘 보든 못 보든 '냉정'하세요. 절대 냉정하시길 바랍니다.
킬러 어려운 것 나오면 어쩌지 10평 잘 보았니 하는 것 다 개소리고
어차피 대부분은 시험 당일날 냉정을 못 지켜서 '틀리지 말아야 할' 문제도 틀리고
조금만 찬찬히 읽어봐도 맞는 문제 다 틀리지요.
\vspace{5mm}

그래도 위안을 원한다면 그냥 시험상황을 얘기해드리죠.
올해 시험은 응시자 풀이 최악인 상황입니다. 그리고 7차 교육과정 마지막이지요.
9평과 10모 컷만 보아도 문제가 아주 쉽다고 할 수 없는데도 그런 컷들이 나왔습니다. 현역조차도 평균적으로는 잘 하는 상황입니다.
물론 올해 시험 같은 경우는 기존 시험의 온갖 노하우와 시행착오까지 다 파악할 수 있으므로 특히 불리하다 할 수 없겠지만
본인이 정말 잘 치렀다고 하면 그보다 더 잘 치른 괴수들을 목격하게 될 것입니다.
\vspace{5mm}

따라서 올해 열심히 했더라도 원하는 결과가 안 나올 가능성은 높습니다.
노력을 안 하면서 요행이나 바라는 사람에게는 올해 같이 변명하기 좋은 해는 없겠지만요.
아울러서 이제는 취업이 되는 게 이상한 시대가 와서 대학졸업해놓고 또 수능치는 사람들도 늘어나고 있습니다.
즉 n에서 n+1이 된다고 해도 과거에 비하면 덜 부끄러운 시대이다라는 데 위안을 가져도 좋겠지요.
\vspace{5mm}

아무튼 냉정히 공부하시길요. 그리고 제가 보아도 '아 이 친구는 열심히 했어'라는 케이스는 대박날 거라고 봅니다.
그리고 목표를 달성하지 못 하는 사람들이 더 많겠죠.
중요한 건 이 대목인데, 바로 그 결과를 인정하고 다시 도전하면서 '공부 그만 하세요'라는 소리 들을 정도로 하는 사람은 내년에 대박나겠고,
귀중한 시간에 감성에 젖어서 하라는 공부 안 하고 그걸 핑계로 놀아제끼는 사람은 평생 힘들겠죠.
\vspace{5mm}










\section{콕콕에서 노력한다고 보이는 수험생}
\href{https://www.kockoc.com/Apoc/420994}{2015.10.17}

\vspace{5mm}

14명 정도입니다.
이 경우는 세가지로 확인하는데
\vspace{5mm}

\item 1. 일지 $-$ 일지 공부량을 보면 제가 생각한 것보다 1.5배 정도 공부한 케이스
\item 2. 학습란에 올라오는 문풀 글 $-$ 일지가 없어 모르지만 문제 보는 안목이나 풀이에서 내공이 확인되는 케이스
\item 3. 쪽지 $-$ 쪽지로 주고받는 경우에 확인되죠.
\vspace{5mm}

그런데 노력했다하는 건 3개월 이상 정말 다른 사람들과 차이가 날 정도로 공부한 경우만 말함.
적어도 제 기준으로 보아도 그 정도 역시 '서울대'급을 노린다면 평범한 정도입니다.
실상은 1$\sim$2개월 정도 하고 힘들어죽겠다하면서 가을의 수필가로 갑니다. 그래서 여간 안타까운 경우가 아니죠.
\vspace{5mm}

특히 그나마 공부한다고 하던 게 8월 이후면 이게 답이 없습니다. 이제 그나마 공부가 되려고 하는데 수능 코앞이면 좌절해서
공부를 싫어해버릴 수 있는 거죠. 1$\sim$3월달에 저랬으면 순상승기 탔을 건데 말입니다.
보통 슬럼프 주기는 2주, 1달, 3달, 6달에는 오게 되어있고, 이 시기를 넘기면 실력은 비약합니다.
그래서 무조건 일찍 공부를 시작하는 게 답입니다.
\vspace{5mm}

가끔 오는 한심한 질문이 6시간만 해도 되어요... 라는 건데 먼저 6시간 일주일을 해보고 질문하지도 않은 케이스여서입니다.
그런데 12월부터 6시간 꾸준히 하더라도 지금 모평이면 올 1등급은 맞고 있을 가능성은 매우 높습니다(특히 이번 교육청 이과라면요)
\vspace{5mm}

이건 논쟁이 붙긴 하지만 1년간 일지$-$상원 루트 나름 보면서 소프트하게 조언해주는 결과를 보면
이게 수능합불까지 어떻게 될지는 모르나, 결국 3개월 이상 꾸준히 해본 사람들은 모평에서 좋은 성적도 나오지만
수능 걱정을 하더라도 정말 건설적으로 합니다.
그 이야기는 결국 다른 여건 탓할 필요 없습니다. 100일 정도 웅녀 인간되기 프로젝트 따라하는 게 정답이란 것이죠.
\vspace{5mm}

다만 문제는 그거. 올해 수능은 2년 3년 바짝한 괴수들도 응시한다는 것. 원래 응시 안 하냐하겠는데 경쟁 과열이 높아진 듯.
\vspace{5mm}

그리고 앞으로 콕콕이 여학생 위주로 갈 거라고 보는 합리적인 이유는
\textbf{여학생들이 더 공부를 열심히 합니다}. 남학생들은 그에 비하면 너무 많이 흔들리네요.
\vspace{5mm}

커리가 정해지고 스트레스와 감정문제만 해소되면 정말 하루 6시간은 우습게 하는 게 있습니다.
필요한 건 '고급정보', 그리고 일종의 감정적 문제를 해결해줄 수 있는 멘토 아니면 게시판의 존재만 있으면 되지요.
\vspace{5mm}

노오력을 이야기할 때는 여학생들은 언급될 이유가 거의 없을 겁니다. 연애만 아니면 거의 다 공부를 시작하면 열심히 하니까요.
\vspace{5mm}

문제는 남학생들인데
\vspace{5mm}

이건 뭐. 칼럼을 써도 그 내용 너 마음에 안 들어 어쩌구 '공격'하는 한심한 종자들도 대부분 남학생들이지만
오래 관찰해보면 왜 선진국도 여풍이고 우리나라도 여풍이 강한지 그걸 알 수 있는 것 같습니다.
실속은 생각하지 않고 '자존심'에다가 '감정' 중심으로 행동하는 게 남자들입니다. 이 글을 쓰는 저도 예외가 아니려나
\vspace{5mm}














\section{10일 남았는데 입시에만 신경쓰시길 바랍니다.}
\href{https://www.kockoc.com/Apoc/456618}{2015.10.31}

\vspace{5mm}

나이 처먹어서 늙어서 그런지 모르지만
요즘 입시생들은 수험을 하기보단 수험코스프레질을 한다는 느낌이 강합니다.
\vspace{5mm}

시험 한달 앞두면 해야될 건 본인 틀린 것 철저히 점검하고, 기본서 다시 회독수 높이고,
아울리 시험 시각 맞춰서 시험 치는 연습하고 그 뿐입니다. 사실 이것만 제대로 하는 것도 어렵습니다.
그런데 상당수가 계획만 짜면서 수험업자들이 기획한 상술대로 놀아납니다.
제가 관여할 바는 아닙니다. 어차피 이건 수능치고 나서 결판날 문제라서요.
\vspace{5mm}

아니라고 하는 사람들도 있지만 \textbf{결국 공부한만큼 나옵니다}.
그런데 여기서 공부한다는 건 '집중'하는 걸 말합니다. 집중한다는 건 세가지 의미입니다.
첫째, 공부 외 다른 건 차단한다
둘째, 자기에게 필요한 것을 $-$ 본인이 공부하기 싫더라도 $-$ 공부한다.
셋째, 나를 잊어버린 상태 $-$ 정신들고보니 10시간 이상이 흘러가 있다.
\vspace{5mm}

이게 되는 학생과 안 되는 학생이 갈립니다. 그런데 문제는 안 되는 학생은 그 되는 상태가 뭔지 끝까지 모릅니다.
그게 되는 사람들이 쓰는 교재는 \textbf{매우 간결하고 핵심을 잘 짚죠}, 그런데 그게 안 되는 사람들이 쓰는 교재는 정말이지 장황합니다.
현실은 후자의 교재가 많이 팔리고 있고, 그만큼 많은 학생들이 입시에서 물을 먹죠.
그래서 공부해도 안 된다고 착각을 합니다. 자기가 집중을 정말 하고 공부하지도 않았을텐데 말이지요.
\vspace{5mm}

10일 남았으면 학습란에 제가 제시한 글이나 즐미님이 쓰신 글대로 가주세요. 그것대로만 하더라도 대단한 것입니다.
10일동안은 가능하면 다른 데 휩쓸리지 말고, 교재도 더 이상 따로 추가하지 말고, 갖고있는 것만 제대로 다 '실제 수능'에 맞게 푸시고
틀린 문제는 3$\sim$5회 이상 다시 풀고 점검하시길 바랍니다.
\vspace{5mm}

이런 것도 안 하고 자기가 공부했다고 착각하면서 부모탓, 환경탓만 하는 스랙이 되지 않기만을 바랍니다.
\vspace{5mm}

가능하면 콕콕 접속, 아니 웹접속을 아예 안 하는 게 좋을 수도 있습니다.
이 글을 보고 난 다음 수능 보기 까지 한번이라도 접속하면 떨어진다... 라고 주문을 거는 게 더 나을 수도 있습니다.
그렇지 않으면 10일이라는 황금같은 시간을 날려먹고 수험사이트나 오가는 스랙질이나 하고 있는 자기 자신을 보게 되겠죠.
\vspace{5mm}

작년말부터 학습일지부터 시작해 올해 상담까지 다 추려보았습니다.
콕콕에서 작년부터 이유없이 공부법 가지고 시비걸면서 스랙교래 추천하는 양반들이 있었는데요
그래도 혹시나 제가 틀렸나 다시 점검하고 실증적으로 갔지만 결론은 결국 '양치기' 최고이고 '집중하는 게' 답입니다.
이 글 읽는 몇몇은 작년에 제가 충고했음에도 그 이행 안 지키고 지금도 스랙질하거나
뒤늦게야 실천하고 아 진작 할 걸 하는 사람도 있을 것입니다(그래도 후자는 낫죠?)
상담글에서도 확인되고 제가 관여 안 했지만 최근 탈콕하시고 공부하러 간 분의 글에도 나오죠.
그냥 님들이 하던데로 양치기하고 집중하면 됩니다.
10일동안은 님들이 틀린 문제 약점 점검하고 극복하는 건 충분한 시간입니다.
물론 안 할 사람은 끝까지 안 하면서 머리탓, 세상탓, 부모탓, 우주인탓하겠죠. 어차피 이런 사람들은 대학을 안 가는 게 세상을 위해 좋습니다.
\vspace{5mm}

올해도 자기가 못 나올 거라고 하지만 의외로 대박치는 사람들은 셋 정도는 있을 겁니다.
신비주의는 아니고 일지 보거나 상담할 때 아 이 친구는 내가 본받고 싶을 정도로 공부했구나 하는 케이스가 있죠.
그런 경우는 시험에서 예상외로 잘 나옵니다.
\vspace{5mm}

하지만 평소에 모평이나 실모뽕맞거나 하면서 산만하게 공부하고 집중 안 하는 경우는 '역시나'가 될 수 있으니
10일동안 똑같은 물을 뱀이 마시게 해서 독이 될 것인지, 소가 마셔서 맛있는 우유로 만들 건지 본인이 잘 극복하시기들 바랍니다.
시험에서 만약 감점나온다고 하면 그건 시험 난이도가 아니라, 본인의 공부가 어디까지 완성되어 있느냐 그 차이일 뿐입니다.
\vspace{5mm}

시험날도 하루는 24시간이고 태양은 동쪽에서 떠서 서쪽에서 집니다. 잘 마무리하고 보시길 바랍니다.
\vspace{5mm}






\section{자기 머리를 믿으세요.}
\href{https://www.kockoc.com/Apoc/458132}{2015.10.31}

\vspace{5mm}

두가지 케이스입니다.
\vspace{5mm}

양치기를 한 사람은 시험날 신기하게도 문제를 보면 문제가 쑥쑥 풀립니다.
즉, 수험뇌가 완성되어서 자기는 아무 생각이 없는데 문제가 해석되고 풀리는 것인데요
이 경우는 본인이 할 건 자기 풀이를 의심하고 계속 검토하는 것입니다요.
\vspace{5mm}

공부를 열심히 해서 잘 돌아가는데 너무 빠른 나머지 브레이크가 안 걸려 문제를 잘못 읽는 불상사가  생깁니다.
이런 건 제가 경고드린 분이 몇몇 계십니다. 반드시 문제를 종이로 가려서 한줄씩 읽고, 조건 하나당 번호 붙이길.
국어를 풀 때는 특정선지가 답인 이유, 답이 아닌 이유 명기하고
수학은 반드시 조건들을 찾아내 번호 다 붙여서 그것들 다 썼는지 확인하시는 것.
\vspace{5mm}

양치기가 안 된 분들은 다소 의식적이지만 작위적인 생각을 하셔야할 것입니다. 사실 이게 맞긴 한데
문제는 이게 엄청난 피로도를 선사하며 속도면에서 불리할 수 있다는 것입니다.
양치기가 안 되기 때문에 문제 봐도 하나하나 자기가 써나가면서 풀어야하는 분들은 브레이크가 아닌 엑셀모드로 가시길 바랍니다.
즉, 이 경우는 스피드업을 하면서 문제를 보고 '연상하고 이미지'를 떠올리는 쪽으로 가야만 제 시간에 맞출 수 있습니다.
\vspace{5mm}

요약하면
양치기가 선행된 분은 자기 뇌의 엔진을 믿되, 브레이크를 잘 걸어주시면서 실수를 방지하시라는 것 $-$ 브레이크 모드
양치기가 안 되어있지만 의식적인 생각으로 풀겠다하는 분들은 연상을 많이 해서 문제 실마리를 찾는 엑셀 모드로 가시길 바랍니다.
\vspace{5mm}

브레이크 모드는 본인이 문제푸는 실마리가 너무 많이 떠오르기 때문에 본인이 논리적으로 그걸 검토해서 OX질을 잘 해야합니다.
반면 액셀 모드는 떠오르는 실마리는 터무니없는 것들도 다 적어야 합니다. 그게 없으면 문제를 아예 못 풀 수도 있습니다.
\vspace{5mm}

시험 망할까 어쩔까 그런 반응은 당연한데, 자기 뇌를 믿으시길 바랍니다
님들이 직접 뛰는 게 아닙니다. 님들의 경주마는 두개골 안의 그것입니다.
경주마를 잘 먹이고 다독이고 하는 쪽으로 신경쓰는 편이 나을 것입니다. 달리는 건 기수가 아니라 말이지요.
\vspace{5mm}






\section{학벌의 장점.}
\href{https://www.kockoc.com/Apoc/460316}{2015.11.02}

\vspace{5mm}

고등학교 졸업 이전에 부딪치는 악은 대부분 \textbf{폭력}임.
고등학교 졸업 이후에 부딪치는 악은 대부분 \textbf{사기}임.
\vspace{5mm}

폭력은 어떤 것이든 법망에 걸릴 수 있습니다. CCTV가 도처에 깔려있고 개인도 스마트폰으로 증거자료 삼을 수 있으니까.
그러나 사기죄의 경우는 잡아내기가 어렵죠. 상대가 속이면 $-$ 전문용어로 기망행위를 하면, 본인이 거기에 낚여서 행동함으로써 범행이 완성되니까.
\vspace{5mm}

사기의 원동력은 인간의 \textbf{탐욕과 무지}입니다.
그런데 탐욕이야 욕심을 줄이면 된다고 보면 되긴 하는데(라지만 말처럼 쉽지는 않음)
무지는 이건 정말 답이 없음. 우리가 아는 것보단 모르는 것이 훨씬 많기 때문에 $-$
사기에 걸리지 않으려면 정치, 경제, 법률, 자연과학, 기술 전반에 빠삭해야할 뿐만 아니라
상대방의 보조개만 보고도 심리를 파악할 수는 있어야한다고 보는데 이게 쉬운 일은 아닐 뿐더러
사기치는 사람들은 정말 똑똑한 데다가 냉혈한입니다.
\vspace{5mm}

그럼 사기꾼을 잡아내면 되지 않느냐.... 라고 하는 사람조차도 자기가 은근히 속고 있다는 걸 모르고 있고
이 글을 적는 저 자신도 제가 모르는 사기에 이중삼중으로 걸려있다라는 게 정확한 진술일 거예요.
공부하고 나서야 아 내가 속았구나를 뒤늦게 알았을 때는 이미 다 털린 뒤지요.
\vspace{5mm}

사기에 걸리지 않으려면 상대가 뭔가 제시할 때 "아니오"라고 거부해야하고
특히 다수결이 좌우하게 되는 소통에서는 백분토론할 때처처럼 설득력있는 자기 주장을 발휘해야하는데
문제는 여기서 먹히는 건 국어영역에서 말하는 논리가 아니란 겁니다. 첫째로는 감성, 둘째로는 외모, 셋째는 권위란 겁니다.
\vspace{5mm}

그런데 여기서 학벌이 유효하게 먹힐 수는 있음.
사실 사람들은 그 주장의 내용이 뭔지 이해하지도 못 하고 이해하려고 하지도 않음. 그냥 좋다 유리하다하는 걸 결정하는데
터무니없는 주장을 하는 사람일지라도 xx대 졸업이라고 하면 \textbf{보이지 않는 무언가 있겠군...} 이라는 게 사람입니다.
메시지 이전에 메신저의 스펙부터 보는 게 인지상정이기 때문에, 이 점에서는 학벌이 유효하게 작용할 수 있을 뿐더러
사기꾼일지라도 상대가 일단 xx대 출신이라고 하면 섣불리 먹으려고 하지 않습니다(물론 같은 xx대 출신이라거나 그 이상이라면 다르겠지만)
물론 xx대 출신이라고 밝혀졌지만 하는 짓이 영락없는 호구라면 얄짤없지만.
\vspace{5mm}

공부에 있어서는 어떤 주장이 오갈 때 \textbf{"나는 그럼 xx대인데 당신은 어디 다니십니까"}라는 걸로 데우스 엑스 마키나로 가는 경우가 가능.
매우 재수없는 상황이라고 할지도 모르겠지만 우리나라 사람들은 다들 학벌주의 저주에 걸려있다라는 건 똑같음.
당연히 이건 금전과는 거리가 멉니다요. 가능하다면 저런 식의 쓸데없는 싸움은 안 하는 게 현명하지만요.
\vspace{5mm}

그런데 개인적으로는 온오프에서는 그 덕분에 사기꾼들을 잡아내고 헛소리한 걸 잡아내는 건 가능했다고는 생각함.
그 이야기는 거꾸로 말해서 이런 케이스가 아닌 경우는 지금도 사기꾼들에게 놀아나는 케이스가 널려있을 거란 이야기죠.
\vspace{5mm}

아, 물론 수능 이후에 만화 검은사기는 보시길 바랍니다. 패턴이 늘 반복되긴 하는데 한번 볼 가치는 있음.
재밌는 건 이 만화에 실린 사기수법이 수년 뒤에 우리나라에서 반복되었다는 것이죠. 즉, 사기수법이 만화책을 통해 먼저 들어옴(...)
\vspace{5mm}






\section{수능출제는 통수}
\href{https://www.kockoc.com/Apoc/461252}{2015.11.02}

\vspace{5mm}

난이도가 높다 낮다... 그런 건 무의미.
\textbf{출제가 예상가능한 영역인가, 예상불가능한 영역인가. 이런 게 중요함.}
올해 수능이 불수능이냐 물수능이냐하는 건 중요치 않습니다.
\textbf{출제 경향이 2014년도와 2015년도 수능과 어떤 점에서 똑같고, 어떤 점에서 다르냐. 이게 가장 중요하죠.}
\vspace{5mm}

과거 복기해봅시다. 이과수학만 보자면 2014년도 역시 '통수'였고 2015년도 어떤 의미에서는 통수였습니다.
둘 다 당시 수험생들이 준비하던 방향과 다른 각도에서 출제했음.
전설의 2012년도 스타일에 맞게만 공부한 2014년도 수험생들은 29번에서 제대로 통수를 먹었고
그래서 기하와 벡터에 치중했던 수험생들은 2015에서 또 한번 통수를 먹었음(난이도가 하향되었다고 하지만 어려운 문제는 어려웠죠)
\vspace{5mm}

그리고 지금 수험생들은 \textbf{2014와 2015 반반무많이로 공부하고 있는 현실}임.
\vspace{5mm}

평균적 방향
\vspace{5mm}

국어 $-$ 어렵게 나올 것이다, 화작문 대비 철저히 해야징$\sim$
수학 $-$ 2014와 2015의 중간정도 보자. 으음, 이번에는 확통이려나?
영어 $-$ 쉽게 나오겠징. 그래도 불안하니까 듣기가 불안
탐구 $-$ 극헬이겠지.
\vspace{5mm}

그런데 문제는 다들 자기가 대비하는 방향으로 출제되고 있을 거라고 '착각'하는 것인데
과거 5년동안 수험생들의 의도대로 100$\%$ 출제된 경우는 없음. 수험생들이 대비하지 않은 방향으로 엿먹이는 건 지속됨.
당연히 사설학원이든 인강이든 실모든 그런 거 적중시킨 적도 없음.
\vspace{5mm}

이 얘기를 왜 하냐면 지금 시험이 불안한 사람들은 2014, 2015 기준으로 자기가 잘 볼 수 있느냐 없느냐 판단할 건데
실제 2016년도 시험 출제가 어떻게 될지는 아무도 모릅니다.
2014, 2015에 최적화된 사람이 열흘 후 수능에서 죽쑬 가능성도 높고,
2014, 2015에 맞게 공부되어있지 않았는데 조금 공부한 게 2016 스타일과 너무 잘 맞아 대박날 수도 있고.
정말 이건 아무도 모르는 것임.
\vspace{5mm}

예컨대 저래놓고 나서
갑자기 국어에서 문학을 어렵게 내고,
수학은 느닷없이 행렬이나 지수로그 격자점을 헬로 내버리고
영어 빈칸에서 고난이도 2문제, 순서잡기에서 헬난이도 내버리고. 탐구는 정작 쉽게 내면?
\vspace{5mm}

뭐 그대로 맛가는 거죠. 다만 이런 데도 안 흔들리는 사람들은 기본이 충실히 되어있는 케이스겠죠.
그리고 지금도 문제 점수 잘 안 나온다는 사람이 좌절하는 심경으로 치렀는데 대박날 가능성도 있음.
작년 이맘 때도 일지 체크해주면서 확인했는데 시험 전에 절망적인 심정으로 달린 사람이 대박 나기도 하고,
공부 거의 다 완료했단 사람이 문제 실수해서 날라가기도 하고.
엄밀히는 시험 전에 분명히 공부에 흠이 많은데 점수가 잘 나오니까 열심히 공부한 학생이 되고
반면 열심히 했는데도 점수가 안 나오니까 공부 안 한 학생으로 취급받고.
\vspace{5mm}

그러니 제가 드리는 말씀은 본인 뇌를 믿으라고 하는 수 밖에 없는 거임. 시험 칠 때 절감하실 것입니다요.
공부 안 되어있다라는 기준도 의식적으로 떠오르는 지식이 기준일 건데,
실제 수능의 문제풀이력은 무의식적으로 나오지요.
\vspace{5mm}

10일동안 기상시각 잘 조절하고 뇌 관리 잘 하면서 평정심 유지하는 게 나을 겁니다요.
예기치 않은 데에서 킬러문제가 나오면 어찌 풀까하는 걸 이미지 트레이닝하는 게 좋긴 할 것입니다만.
\vspace{5mm}

+ 생각해보니까 만약 탐구를 전부 다 물로 내버리고 영어를 불로 내버리면
그 때는 "영어 공부했어야하는데", "탐구 필요없네용$\sim$ 국영수나 할 걸"... 뭐 이럴 삘이고.
작년만 해도 수학은 저도 불로 나올 것이라고 생각했는데 다 실전미만잡이 아닌가 싶기도 하고.
어차피 정부는 퍼센테이지만 잘 조정하고 복수정답만 안 내면 욕 안 먹죠.
\vspace{5mm}

+ 실전주의적인 입장에서 말하면 탐구를 제외하고 국어영어수학은 각 과목당 2$\sim$3문제씩이 결국 발목잡고
그 문제 하나당 10$\sim$15분씩 소요되는 일이 벌어지죠. 출제자라면 이런 식으로 내지 않을까 싶은데 말입니다.
이 경우 수험생에게 중요한 건 차분히 문제, 지문을 읽고 '논리적'으로 정답을 유추하거나 확률 높은 걸 찍거나 하는 것일텐데.
\vspace{5mm}

+ 지금 수험생들에게 가장 큰 문제는 2015년도 출제 경향의 부채. 즉, \textbf{다 만점을 받지 않으면 안 된다는 강박이 가장 큰 문}제인 듯
하지만 예기치 않게 난이도가 헬이어서 만점의 저주라는 게 올해 깨질짇 모르는 일입니다. 그러니 절대 과거 경험으로만 접근하면 안 됨.
\vspace{5mm}

+ 최악의 시나리오는 국어나 수학에서 예기치 못하게 통수 맞고 그 뒤로 만점강박증에 자포자기해버리는 경우.
혹은 생각보다 쉽게 나온 국수인데 영어가 어렵게 나와 통수맞는 케이스
\vspace{5mm}






\section{교재퀴즈}
\href{https://www.kockoc.com/Apoc/462586}{2015.11.04}

\vspace{5mm}

Q. 가성비가 최악인 동시에 최고,  다수의 수포자 양성, 소수의 수학고수 양성한 책이 뭔지 기술하시오.
\vspace{5mm}

속내가 뻔히 보이고 가입일이 최신인 '선플'을 가장한 '악플'이 있어서 글지웠습니다.
이 사이트나 제가 마음에 안 드는 분들도 계시겠지만 활동을 하려면 \textbf{지능적으로 하셔}야죠.
아무튼 그 분들 덕분에 수능 이후 콕콕 사이트 운영에 대한 좋은 참조례가 마련된 건 감사.
\vspace{5mm}

그리고 전 올해 초부터 특정교재를 대놓고 '실명'으로 까지 않습니다. 그런 걸 가지고 고소미 먹이는 정신나간 놈들이 있기도 하지만
언급할 필요도 없는 교재는 대놓고 말하는 것만으로도 \textbf{타락하는 기분}이어서 말입니다.
특히 교재에 대해서 추궁하면 답 못 하고 도망가는 케이스라면 말입니다. 그런 사람이 없을 것 같죠?
\vspace{5mm}

A, B 교재가 궁금하신 분들 있는 것 같은데 힌트 드립니다.
그 저자들 \textbf{남자가  아닙니다}. 그 범주에서 찾아보시길
\vspace{5mm}

뭐 그건 그렇고 저기 Q에 해당하는 교재가 뭔지 한번 답들 해보시길.
사실 이 교재는 가성비가 대한민국 최악인 동시에 최고이기도 합니다. 제대로 정복하면 수학 최고수가 되죠.
너무 쉬운 문제인가?
\vspace{5mm}

그리고 이건 콕콕 사이트에서 활동하고 있고 하려는 사람들을 위해 본격하는 이야기입니다.
그리고 콕콕과 여기서 교재내는 사람들을 온갖 방법으로 공격하려는 사람들에 대한 경고성 메시지이기도 한데요
\vspace{5mm}

교재비판에 대해서 적죠
\vspace{5mm}

우선 교재명 언급만 하면 경찰서 출두 가능? 그렇습니다.
그럼 그런 거 가지고 먹이는 사람들 있나? 그렇지요.
벌금 싸게 나오는 것까지 고려해서 민사까지 감안해서 일부러 자기가 피해입었다고 썰까지 푸는 경우도 있죠.
그럼 이런 질문 하겠죠. \textbf{"일일히 자기 교재가 어떻게 언급되나 그거 찾는 사람도 있나요?"}
\vspace{5mm}

\textbf{예, 있습니다. 심지어 동료나 부하(?)들을 통해 알아보거나 제보받기도 할 걸요.}
\vspace{5mm}

그 사람들 하는 짓이 비겁하든 안 하든 어찌되었든 법의 논리는 다릅니다.
그리고 그 사람들은 일부러 '제3자'인 척 리플을 달아 질문을 유도해서 \textbf{법에 저촉되는 발언을 유도하기도 합니다.}
그 사람들은 품질이 아니라 명성만으로 먹고 살기 때문에, '진실된' 이야기일지라도 자기 교재에 관한 부정적인 평이면 개입하는 겁니다.
그리고 그런 노력 끝에 시장점유율을 유지하죠. 그래서 사람들은 '어라, 별로 좋은 교재가 아닌데도 뭔가 있나보다'라고 생각하고 구입하는 거죠.
\vspace{5mm}

응징하는 방법은 없습니다. 다만 그런 사람들이 알아서 시장에서 퇴출당하는 걸 기다릴 수 밖에 없지 않나 싶은데
제가 권하는 건 간단합니다. "언급하지 않는다고 고소미 먹이진" 못하니,
\textbf{첫째, 그런 교재들은 아예 언급하지 말고 좋은 교재만 언급한다.}
\textbf{둘째, 언급하고 싶으면 특정하지 말고 '단점'을 추상적으로 적는다(추상적 단점만 가지고는 특정 못 합니다)}
이렇게 가시길 바랍니다.
\vspace{5mm}

저저번에 올린 글도 참 흥미롭습니다만 저는 A, B가 어떤 교재인지 언급도 안 했고 그런 댓글이 안 달리길 바란다고 분명 적었습니다.
그런데 신기하게도 '단점'만 적었는데도 몇몇 분들이 교재 특정을 하기 시작했는데요
사람 의심하는 건 죄송하나, \textbf{평소에 활동 안 하던 분들이 댓글을 다는 건 좀 그랬습니다.}
게다가 특정하지 말라고 했는데, 일부러 특정하는 분위기로 몰아가는 것도요
\vspace{5mm}

그리고 오늘 아침에 가입일이 바로 11월 4일인 모 분께서 아예 B가 뭐냐고 멋대로 특정을 하기 시작하시던데
\vspace{5mm}

오늘 가입해서 '선플'을 가장해 B를 특정교재로 특정화시키려고 해서 엿먹이려한 사람, 아이디가 무려 \textbf{huntapoc}이더군요(조어관념 참 촌스럽다)
정보공유가 안 되는 것 같죠? 합법적이고 도덕적인 것에서는 유기적으로 다 돌아가고 있습니다.
여기 허혁재님이 콕콕의 비영리사이트적 특성을 강조하기 때문에 적극 안 나서고 있고, 영리적인 건 T 모 까페로 이전해서 그렇지
절대 망사이트는 아닙니다.
유감스럽게도 콕콕 사이트, 쉼터가 망해서(?) 그렇지 잘 돌아가고 있습니다.
수험에 관한 일종의 정의 관념이나 의리이지, 어디처럼 금전적 이해관계가 아니거든요.
\vspace{5mm}

앞으로도 여기 적지않은 공격이 들어올 것이라고 예상하고 있습니다. 저도 그래서 여러번 글로 낚시질을 해보면서 누차 확인한 바입니다.
이 사항은 허혁재님에게 제보되었으니 사이트 운영에 반영될 것입니다.
비겁한 질 해서 콕콕 사이트 흔들어보자라고 하는 사람은 그 시간에 공부나 좀 했으면 좋겠네요.
\vspace{5mm}

일부러 약자인 척하면 뒷공작하는 거 밝혀진 사람 꼼수는 과연 안 읽힐 것 같죠?
그리고 어떻게든 수험계에서 돈이나 벌어보자라는 사람은 이 사이트에 얼씬 안 하는 편이 낫습니다.
지금 검토하는 게 이제 어떤 뒷공작질하거나 선플을 가장해서 저렇게 나오거나 하는 경우들,
그걸 역으로 법에 얽게 만들어볼까 하는 것도 지금 생각하고 있습니다.
\vspace{5mm}

+ 그리고 예전에 제가 ㅇㅂ 대학게시판에서 활동했을 때, 거짓말아니고 거기 수험생들 흙수저들 불쌍해서(...) 이런저런 정보글 달았는데요.
아무튼 수험계가 참 웃기더군요(흙수저 얘기하지만 저야 금수저들과 놀아난 적이 있던 흙수저죠 $-$$-$ )
\vspace{5mm}

하도 웃겨서 거기서 졸업장 인증 때리고(그런 짓 쪽팔려서 이제 안 합니다만) 이런저런 정보글만 쓰는데
정보글만 써도 딴지거는 사람들이 있었습니다. 그리고 교묘하게 xxx 강사는 어떻게 생각해라고 하면서 간접광고하는 사람도 있었죠.
현실은? 그 때 거기 알바들 활약했다는 거 나중에 밝혀졌죠?
뭐 심지어 제가 거기 활동하는 것까지 파악해대는 모 학원도 있더구만요.
학원들이 머지않아 CIA에 진출할 모양입니다.
\vspace{5mm}

EBS 인강 칭찬하고 시중교재 부당하게 까이는 것을 변호한다고 까임.
심지어 EBS 알바란 소리까지 들었습니다(...)
그들이 뭔 생각인지 모르겠습니다.
\vspace{5mm}

더러운 세계죠. 만약 콕콕 사이트도 비슷하게 간다면 저도 그냥 떠나버릴 것입니다.
\vspace{5mm}






\section{개념서 함부로 쓰면 안 되는 이유}
\href{https://www.kockoc.com/Apoc/462860}{2015.11.04}

\vspace{5mm}

책 쓴다고 하는 사람들을 보면 '그냥 써볼까'일 건데요, 냉정히 말하면 책 쓴다는 건 의사의 수술행위에 비견된다고 봅니다.
책 한권이 사람의 인생을 바꾸기 때문입니다. 더 정확히 말하면 우리가 '사고'하는 건 강의나 책의 영향을 상당히 많이 받습니다.
\vspace{5mm}

우선 정석을 예로 들어봅시다. 전 정석을 매우 괜찮은 교재로 보고 있음.
학창시절에 썼던 교과서가 실력정석이기도 했지만, 지금 다시 봐도 책 5권 분량을 1권에 정말 빠짐없이 잘 넣은 책이라고 생각함.
게다가 예제$-$정석$-$유제라는 편집도 뭔가 논리적이거니와 이대로만 가면 일단 지식은 갖춰지죠.
\vspace{5mm}

그런데 정석의 저런 체계적 구조가 동시에 단점이 되어버립니다.
예제 아래 '정석'이 있는 구조로 학습해버리면, 학생들은 특정 문제는 특정'패턴'으로만 풀어야한다라고 학습해버립니다.
이건 일종의 강제이기 때문에 수학을 '싫어하게' 만드는 결과를 낳고, 실제로 부모님이 사준 정석 보고 수학이 싫어진 친구들도 많습니다.
책 자체는 괜찮은데 이 편집이 지닌 단점 $-$ 고수 입장에서는 정리가 잘 된 책이지만 고수가 아니면 '암기'로만 비쳐지는' 것
그래서 정석은 전국에서 가장 깨끗한 책 혹은 가장 더러운 책이 됩니다.
아예 안 보거나, 아니면 수십번 보거나 하기 때문에.
\vspace{5mm}

정석을 뛰어넘지 못 하고 정석에 익숙한 사람은 새로 보는 문제를 못 풉니다. 거기에 일대일대응하는 정석이 안 보이니까요(...)
\vspace{5mm}

수능 수학 기출 풀이를 보면 알겠지만 정작 쓰인 개념, 스킬은 교과서를 벗어나는 경우가 거의 없습니다.
(벗어난다고 보이는 경우도 출제 실수로 교과외적으로 풀리는 것이지, 교과서를 벗어나는 걸 의도하진 않았죠)
그래서 교과서를 잘 본 친구들이 수능은 잘 나오는데, 정석을 자기 것으로 만들지 못 하고 그 늪에 빠진 친구들은 안 나오는 경우가 생겨요.
\vspace{5mm}

실력정석은 시중 문제집을 다 풀고 1등급 나오고 만점 이상을 노릴 때 '보충서'로 보는 것이 좋다라는 게 제 소견입니다.
책 자체는 정말 나무랄 것도 없고, 그게 일본 것을 베낀 것일지라 하더라도 사실 상관이 없는 게, 이거 '좋은 것'을 제대로 베낀 경우여서입니다.
시작은 베끼는 것일지라고 하더라도 반세기동안 버텨온 역사라는 걸 가볍게 무시할 수는 없죠.
\vspace{5mm}

그런데 이건 다른 개념서들도 비슷하지 않을까 하긴 합니다.
내용이 문제가 아니라 개념이 '어떤 순서'로 학습되고 '어떤 관점'으로 조명되어야 하는 '독법'이 중요한데
그걸 책으로 제시하는 경우는 별로 없네요. 이래서 인강으로 가거나 하는 것이 아닌가 싶은데 말이죠.
\vspace{5mm}

문제를 풀 때에는 한 문장, 한 구절식 끊어 읽으면서 거기 주어진 조건에 어떤 논점들이 연결될 수 있는가 떠올리고 정리한 다음,
문제가 원하는 답에 도달하기 위한 과정을 '논술'할 줄만 알면 되는 건데
간단해 보이는 이런 과정을 제대로 설명해주는 국내 책은 찾기 어렵기도 하지만,
지금 생각해보면 그런 게 이런 건 '개념서'로 포섭하는 건 어렵다는 생각도 들고 있습니다.
4점짜리 문제를 풀기 위해서는 오히려 수리논술을 공부하는 편이 낫겠죠.
그냥 기출문제집의 해설이 기막히게 잘 쓰여있다면 그게 그냥 개념서보다 낫지 않나 하는 생각이 듭니다.
\vspace{5mm}

$-$ 수학적 도그마로서의 개념서 : 점
$-$ 실전문제의 논리적 풀이를 설명하는 개념서 : 선
$-$ 해당 수학 개념을 다양한 관점에서 실사례와 연결시켜 보는 개념서 ; 면.
이렇게 개념서의 용도를 구분해야지, 개념서 한권으로 다 해결하는 건 어렵다라고 보는 것이죠.
\vspace{5mm}

점 : 교과서도 좋습니다. 시중교재로 가면 형식면에서는 쎈, 감각면에서는 풍산자, 증명면에서는 셀파가 있죠.
선 : 기출문제집들을 보면서 본인이 스스로 개념서의 해당 페이지를 표시하는 식으로 책을 만들어나가야합니다.
면 : 수리 논술 교재(예컨대 남호영) 같은 것부터 시작해 각종 기출을 보면 됩니다.
\vspace{5mm}

당연히 이 글은 간혹 책을 쓰면 되지 않겠냐하는 허대장에 대한 디스인데요,
이건 허대장이 책을 쓸 능력이 없다... 가 아니라 너무 만만하게 보고 있다는 생각입니다.
일전에 제가 차트식 수학 구입을 권장해서 사보셨을 건데, 거기서 눈여겨볼 게 바로 '해설'입니다.
차트식 수학은 한문제 해설도 정말 일본인들의 장점이 묻어날 정도로 꼼꼼히 적어둬서 그걸로 개념서가 필요없게 했습니다.
이제 일격 A형 다 팔렸다니까 말씀드리면, 일격 A형 해설도 그런 식으로 더 보강할 수 있었지 않았을까 하는데요?
\vspace{5mm}

내년에 다시 치는 분이 없길 바라겠지만 다시 시작하신다면
기출을 보면서 본인이 참조하는 교과서나 검증된 기본서를 유기적으로 참조해보시는 습관을 들이시길 바랍니다.
\vspace{5mm}

+ 제시해보고 싶은 아이디어는 $-$ 뭐 여긴 훔쳐보는 사람도 많겠지만 제가 싫어하는 교재 저자라도 베끼라는 차원에서 제시합니다
학생들이 많이 보는 기본서 $-$ 정석도 좋고 교과서도 좋고 아무튼 그 중에 하나를 허대장님이 정하십시오.
그 다음 일타삼피를 내건 일격을 낼 때, 해설에 \textbf{책 참조 페이지를 표시하는 것}입니다.
가령 30번 기하와 벡터 문제,  쎈수학 기하와 벡터 p.xxx : 이렇게 말이지요.
아마 이렇게 참조만 하는 건 해당 출판사들이 확실히 반대하지 않는 한 오히려 고마워할 것입니다.
교과서에만 들어있는 내용이라면 뭐 그건 인용해도 되지 않을까 싶긴 한데, 그 외의 경우라면
학생들이 많이 보는 기본서를 가지고 표시해주면 이건 상호 윈윈이 아닐까 합니다.
\vspace{5mm}

굳이 억지로 기본개념서를 집필할 이유는 없습니다.
사실 가장 좋은 건 당해년도 EBS 수특, 수완과 연계시키는 것이 아닐까 합니다만.
\vspace{5mm}

+ 일단 이 글을 읽는 사람 중에서 난 수학이 싫다하는 경우는 머리보다 책을 의심하시길 바랍니다(만약 공부를 하는 경우라면요)
1등이 보는 교재 따라본다고 좋아지는 게 아닙니다. 소화 못 시키면 꽝나죠.
어려운 교재 보지 말고 쉬운 교재 골라보는 게 장땡입니다요. 정석을 굳이 볼 필요가 있느냐... 하면 전 없다고 말씀드리겠습니다.
\vspace{5mm}






\section{교재 이야기 ; 숨마쿰썰}
\href{https://www.kockoc.com/Apoc/464449}{2015.11.05}

\vspace{5mm}

올드 숨마쿰의 장점
\item \textbf{1}. 상위권
\item 2. 허세
\item 3. 어머니 안 계신 난이도
\vspace{5mm}

올드 숨마쿰의 단점
\item 1. 불연속
\item 2. 기초 부족
\item 3. 구하기 어렵다.
\vspace{5mm}

제가 아는 숨마쿰은 3가지임
7차 교육 이전 숨마쿰 : 올드 숨마쿰
7차 교육 과정 숨마쿰(현 고3까지) : 7차 숨마쿰
개정 숨마쿰(현 고2부터) : 개정 숨마쿰
\vspace{5mm}

개정과정은 약간 안타까운 게 있음.
숨마쿰의 단점으로 지적되던 게 하위권에게 너무 불친절하다, 그리고 기초 문제가 부족하다여서인가
이런 방향으로 개정했다고 보는데 그 결과 숨마쿰만의 장점이 희석되어 버렸음.
\vspace{5mm}

원래 숨마쿰은 안경 낀 수학남녀가 허세 부리는 용으로 딱 좋음.
이게 비꼬는 게 아니라, 실제로 대단히 중요함. 수학 시험의 불안감을 날려주는 게 '나 정말 어려운 문제집 풀었다'라는 허세임.
솔까 요즘 나오는 것 중에선 블랙라벨과 실력 정석 빼고 허세부릴 수 있는 건 별로 없음(아니면 사진 속 일본 교재 '붉은색' $-$ 아카차트로 가든가)
\vspace{5mm}

저기 숨마쿰 중에서 좌측이 올드, 우측이 7차임
사실 현재 수험생들을 대상으로 한다면 7차로도 상위권용 개념은 괜찮다고 생각함.
정석을 보는 것보다는 차라리 이 편이 낫다고 보는 게 있음.
올드 숨마쿰 자체가 과거에 정말 수학을 잘 했던 사람들이 자기가 공부한 내용을 제대로 짜깁기한 경우였음.
문제 난이도는 알보칠 수준임.
7차 숨마쿰은 약간 희석시킨 알보칠 정도인데 서술이 더 쉬워짐. 톡쏘는 맛이 약하긴 하지만 요즘 교재에 비하면 그래도 난이도 있는 편임.
개정은 보다가 필요없다라고 보아서 구매하진 않았음. 왜 그렇게 개정했을지 이해는 가는데 $-$ 고객들은 중하위권이 많으니까 $-$
이건 뭔가 좀 아깝단 생각도 들고 있음. 과거 교재의 어려운 것만 다 요약해서 별권 발행해주면 좋지 않았을까 싶음.
\vspace{5mm}

개정과정 생까버리고 올드 숨마쿰대로만 팔았어도 잘 팔리지 않았을까 싶음.
n수생이 많이 늘었거니와 수험생들 수준이 전반적으로 높아져서 고난이도 문제 수요가 높아졌기 때문.
사실 실모는 붕어빵 아니면 질소과자스러워서 $-$ 4점 3문제 풀려고 30문제 전부 구입한다는 딜레마 $-$ 상당히 비싸게 구입하는 것임.
\vspace{5mm}

저 당시 저자진들이 과거 모 사이트 레전드인 걸로 알고 있음. 이 경우는 인정할만하다고 생각하고 있음.
초기에는 서술이 너무 불연속적이고 어렵다라고 보는데 여러번 읽고 느낀 것은 어, 그래도 정말 상위권 허세부릴만하다였음.
교과서 따로 보란 말도 없음, 오히려 교과서상 개념을 자기들이 더 상세히 설명해주고 있음.
\vspace{5mm}





\section{그냥 평상시대로 하십시오.}
\href{https://www.kockoc.com/Apoc/465514}{2015.11.05}

\vspace{5mm}

제가 수험사이트 쪽에 꽤 비판적인 이유를 적어보겠습니다.
\vspace{5mm}

첫째, 수험이 뭔지 잘 모르면서 허세를 떠는 경향이 있다,
둘째, 별로 실전적이지 못 한, 비현실적인 낭만주의를 강조한다.
셋째, 실전에 도움이 되는 것보다는 장사에 혈안이 되어있다.
\vspace{5mm}

작년 말에 얘기했을지 모르지만 보통 이 기간은 공부기간에 가산하지 말라는 거, 이거 실감하시는 분들이 계실 것입니다.
왜냐고요? 멘탈이 알아서 부서지거든요.
이거 매년마다 보이는 현상이라서 별로 새로울 건 없어요.
\vspace{5mm}

그런데 이건 사실 마음의 문제입니다. 만약 님들이 수능날이 12월이라고 착각했다면 지금 긴장되거나 공부가 안 되거나 포기할 마음이 들까요.
갑자기 수능이 15일 뒤로 연장되면 우왕 기회다 하면서 달릴 사람은 달릴 것입니다.
즉, 이건 마음의 문제란 겁니다.
\vspace{5mm}

그런데 왜 마음이 압박이 되느냐.
\vspace{5mm}

전 시험 한달 전에 오프 모의든 실모든 그런 건 더 이상 추가하지 말라는 입장입니다. 콕콕도 올해 친 모양인데 그거 잘못된 거예요.
이게 실제 도움이 되느냐. 전혀 되지 않습니다. 이거 잘 나온다고 수능 잘 나오는 게 아니라, 원래 수능을 잘 칠 사람이니까 그것도 잘 나온 것이죠.
오히려 이런 것이 수험생들 마음에 엄청난 부담을 가져다주고 쓸데없는 낭만주의를 부추깁니다.
\vspace{5mm}

사실 시험은 그냥 '내일' 본다고 생각하면 됩니다. 이 얘기는 즉, 지금 일주일이 남았건 3일이 남았건 달라질 건 없다는 것입니다.
수능은 지식의 암기량으로 치르는 게 아닙니다. 얼마나 기본적인 걸 똑바로 알고 있는지, 그리고 시험날 얼마나 집중이 잘 되어있느냐 그것이죠.
그럼 눈 부릅뜨고 포스를 발휘해야 집중이 되는가. 그게 아닙니다, 그냥 평상시대로 보라는 겁니다.
그냥 일주일동안 시험 리허설만 줄창 하면서 시험날 최상의 컨디션을 발휘할 수 있게 조절하면 되지 지금은 뭔 교재 본다 그럴 때가 아니죠.
\vspace{5mm}

이렇게 긴장하고 하는 건 올초부터 여름까지 그랬어야하는 거지, 지금은 오히려 이완을 하면서 여유롭게 시험대비를 해야할 차입니다.
막판 정리한다 뭐다 하는데 지금 괜히 후까시 잡고 긴장해보았자 그거 일주일 못 갑니다. 오히려 소집일날 맥이 확 풀려서 더 맛이 간다니까요.
\vspace{5mm}

원양어선 냉동고 이야기 아시죠?
어떤 사람이 운나쁘게 냉동고에 갇혔는데 나오지 못 합니다.
자기가 죽어간다라는 기록을 남기고 정말 얼어(?) 죽습니다.
나중에 문을 열어본 동료들은 놀라죠.
자기 동료가 냉동육이 되어서가 아닙니다.
냉동고가 꺼져있어서 그 안은 '상온'이었거든요.
\vspace{5mm}

실제로 열심히 달려온 사람들은 지금 연초에 비할 데가 아닙니다.
실력은 객관적인 것입니다. 공부하신 분들은 뇌가 그만큼 단련되어 있지요.
왜 불안하나? 그거야 지금은 끊임없이 자기 비하를 해서입니다.
그럼 왜 비하를 하나? 시험이 부담되기 때문이죠. 그래서 자기가 공부 못 하는 못난 놈이라고 생각해야 안정이 옵니다.
사람은 불안할 때 자기 비하를 하면서 '나쁜 결과를 정당화시킬' 준비를 하죠.
하지만 이건 나쁜 방향으로의 진화이고 '노예'로 살아온 우리 조상들의 근성이 남아있는 결과죠.
\textbf{'나쁜 결과를 정당화시키'}는 건 \textbf{혼나기 싫어서}입니다. 즉 님들은 과거에 혼났던 망령에 사로잡혀서 핑계 댈 준비를 하고 있는 것이지요.
\vspace{5mm}

결과가 좋게 나오건 나쁘게 나오건 그렇다고 당장 죽거나 그런 건 아닙니다. 그리고 도대체 누가 혼낸다는 것이지요?
차라리 혼낸다면 내가 이렇게 열심히 공부했는데 도와주지 못 한 주변 사람을 혼내든가
시험에 나오는 내용을 빠뜨리거나 그걸 부각시키지 못 한 교재를 혼낼 준비나 하시지요.
정말 어이가 없는 게 돈도 자기들이 지불하고 공부도 해놓고, 나중에 인강 강사든 교재든 가족 눈치를 본다는 것입니다.
진정 갑(甲)은 공부하는 본인들이지, 공부 안 한 주변 사람들이 아닙니다.
공부해도 시험성적인 안 나올 수도 있는 것입니다. 그게 그렇게 놀라운 것도 아닌데 뭘 강박을 가지고 주변 눈치를 보십니까.
이렇게 되면 수능이 문제가 아닙니다. 죽을 때까지 남 눈치나 보고 살게 되어 있습니다.
\vspace{5mm}

그냥 평상시에 밥먹고 숨쉬고 화장실 가듯 그렇게 치세요.
어떻게든 돈이나 벌려는 업자들이나 이걸 이벤트화합니다. 그게 수험생에게 얼마나 부담을 주는지 모르면서요.
수능 끝나고 나올 때 매우 허탈하실 것입니다(n수생은 이걸 기억 하실 거예요). 그리고 아쉬우면서 아 그것만 더 공부했으면... 하는 생각이 들겠죠.
사실은 정확히 침착하게 문제 읽기만 해도 풀리는 것인데 괜히 긴장하거나 울거나 그래서 풀 수 있는 것도 놓치는 경우가 더 많습니다.
수능 시험 문제는 어려운 게 아닙니다. 단지 자기가 공부한 방향과 약간 '엇갈려' 있을 뿐입니다.
화제가 되는 문제는 늘 새로운 것이어서 그렇지요. 그렇다면 새로운 문제를 대비하려면? 침착해야 합니다.
하지만 가장 중요한 건 그 새로운 문제를 푸는 데 오히려 '방해가 되는 선입견이나 지식'이라면 버려야 합니다.
\vspace{5mm}

예컨대 수학의 경우만 하더라도 고정관념적인 패턴이 문제풀이에 방해가 되는 경우가 많습니다.
국어의 비문학 지문 문제든 영어의 빈칸도 수험생의 고정관념을 교묘하게 자극하는 케이스이지요.
이건 즉 지금 긴장하고 계신 분들이 강박적으로 암기하는 지식이나 풀이가, 오히려 수능에는 독이 될 수 있다는 걸 말하는 겁니다.
수능 시험을 치를 때에는 오히려 그동안 배웠던 것을 '잊어버리는' 것도 필요합니다.
뭔 뜬구름 잡는 소리냐 하겠지만 수능문제는 정말 뇌에서 알아서 푸는 것이고 우리 의식은 그걸 거들 뿐입니다.
\vspace{5mm}

자, 그렇다면 일주일 남긴 지금은 무엇이 필요할까요? 비장함? 단단한 각오? 아니면 감동적인 세러머니?
그딴 건 연초에 했어야하지요. 꼭 사람들이 연초부터 봄까지는 여유부리며 놀다가 지금 와서는 바짝 긴장하는데 이거야말로 웃긴 겁니다.
11월 11일에 예비소집이죠? 그 날은 친구들과 가볍게 대화하고 시험장에 다녀오고 그 다음 집에서 와서 가볍게 막판정리하고 푹 주무세요
그럼 11월 6일부터 11월 10일까지 가장 중요한 것?
\vspace{5mm}

건강관리입니다. 지금 마지막 수능이니 뭐니 그러면서 맹렬히 달리는 거 기특하긴 한데 그러다가 쓰러지면 아무 소용없습니다.
공부시간은 최소시간으로 잡고 수능시험 시각대에 맞춘 과목배정으로 문풀하면서 그냥 컨디션 관리를 하세요.
어떻게든 문제를 풀어서 점수 나오는 것 가지고 위안을 받고 싶어하겠지만 그건 시험 당일 컨디션에 그다지 도움이 되지 않습니다.
극단적으로 말하면 국영수탐 모두 약한 것만 골라서 아주 느리게 $-$ 수능시험 시각대에 맞춰서 훑어보고 여유있게 보내도 좋습니다.
주의하세요, 지금은 감기 시즌입니다. 게다가 수능날은 춥지요. 괜히 막판에 열심히 한다고 하다가 건강 날라가면 아무 소용 없습니다.
부담감은 문풀에 전혀 도움이 되지 않습니다. 강박관념에 시달리는 사람은 문제를 유연하게 읽지 못 하지요.
\vspace{5mm}

저 자신도 현역으로 대학에 들어갔고 나름 공부 잘 하는 인간들 틈에 있었으며
콕콕에서도 작년에 성공한 분들을 정리하면서 보면서 느낀 것 그대로 적는 겁니다.
합격하는 사람들은 뭔 실모 본다 어떤 교재 본다 이런 걸로 후까시 잡지도 않고 이상한 입시교주 숭상 같은 것도 하지 않습니다.
자기가 필요한 교재 정해서 그거 여러번 풀어서 자기 것 제대로 만들고, 정말 '핵심'적인 것만 제대로 스나이핑해버립니다.
이런 사람들은 온오프건 대화해보면 느끼지만, 절대 장황하지 않습니다.
오히려 일반인들에 비해 느슨하고 심지어 나사가 빠진 경우도 봅니다만, 집중할 때는 제대로 집중합니다.
평소에 느스한 상태로 있기 때문에 집중할 때에 제대로 집중할 수 있는 것이지요.
\vspace{5mm}

그에 비해 실패하는 사람들은 참 이것저것 많이 벌입니다. 그리고 막판에 감당을 못 해대죠
여러번 n수 하는 이유? 여러가지가 있겠지만 결국 이유는 '집중'하지 못 해서입니다.
본인은 머리가 나빠서... 라고 정당화하겠지만 틀린 이야기입니다. 정확히 말하면 '습관'이 잘못되어서 그런 것이지요.
습관이 잘못된 사람은 하나에 집중해도 모자랄 판에 여러군데 일 다 벌여놓고 수습을 못 해요. 대단히 산만하지요.
그리고 그 다음 해에 또 응시할 때에는 본전 찾겠다고 더 일을 크게 벌이다가 또 말아먹습니다.
\vspace{5mm}

이 정도면 메시지 전달은 충분히 되었다고 여기네요
자기가 계획안 것의 절반으로 줄이고, 11월 12일에 풀집중이 가능하도록 이제 심신을 이완시켜주시길 바랍니다.
시험 시각에 해당하는 시간동안만 공부하고 나머지 시간은 여유있게 보내시면서 '감기 걸리는 일'이 없도록 주의해주세요.
그리고 당황스러운 문제를 읽으면서도 침착하게 분석해서 그걸 풀어내는 자기 모습을 계속 상상하시길 바랍니다.
\vspace{5mm}






\section{교재 뒷담화 : C D E F}
\href{https://www.kockoc.com/Apoc/465559}{2015.11.05}

\vspace{5mm}

추정하지 말라고 했는데 이번에도 또 엉터리로 추정하는 사례가 없기만을 바란다.
이 글을 쓰는 이유는 특정 대상을 언급하기보다도, 책을 이딴 식으로 쓰면 안 된다는 '추상화된 문제'를 지적하는 것임.
간혹 이 글 보면서 자기를 씹는 건가하면서 도둑이 제 발 저린 케이스 있을지도 모르지만 그럴 시간에 공부나 하거나 책이나 고치길.
\vspace{5mm}

저번 알파벳은 A, B로 썼음. 그리고 내가 보기엔 그 저자들은 남자가 아니라고 분명 밝혔음.
(적어도 악플(?)다는 사람들은 내가 보기에는 잘 모르는 것 같아)
\vspace{5mm}

그럼 이번에는 C, D, E, F를 논하겠음
\vspace{5mm}

\textbf{C : 서문만 거창한 문제집}
\vspace{5mm}

이 문제집을 쓰는 사람은 집필을 위해서 일까지 그만두었다고 서문에 밝힘. 나야 컬렉터니까 구입
그런데 저자는 집필 기간동안 놀았나... 라는 생각이 든다.
문제 100$\%$ 모두 그냥 기출문제다. 그럼 기출문제를 그냥 무단으로 베낀 것이 아닌가.
해설은 그럭저럭 가독성이 있다, 썰을 잘 읽는 기분이긴 한데, 이 사람 강사라지? 그런데 책은 많이 읽은 것 맞나?
\vspace{5mm}

이 케이스는 강사가 함부로 책 써서 돈벌고 싶어서 출판 나선 케이스다.
그런데 강의와 문제를 만드는 건 별개의 문제지. 문제를 만든다라는 건 보통 어려운 일이 아니다.
이것도 나중에 썰 풀겠지만, 문제를 제대로 만드려면 정말 교수급은 되어야 한다. 그렇지 않은 나머지 자작문제는 한계가 있다.
그런데 이 저자는 문제 만들 능력은 없는 것이다. 그래서 기출만 몽땅 베껴넣었다.
그럼 해설은 어떤가?
\vspace{5mm}

뭐. 실모들보단 낫다. 이게 뭔 이야기인지 알겠지? 그런데 그 해설이 정말 돈주고 살만한 것인지는 의문.
비슷한 값에 10배는 많은 문제를 풀 수 있는 마플에 비하면 그다지 두드러지지도 않는다.
그런데 한권 가격이 참 더럽게 비싸다, 저번에 말했던 B 가성비만 최악이 아니었어.
그나마 B는 짜깁기 내용으로 새로운 걸 접할 수라도 있는데(수능에 도움이 되는지는 의문)
이건 뭐 기출 해설 소프트하게 해놓은 주제에 폭리를 받아먹고 있나.
\vspace{5mm}

내년에도 설마 나올리는 없겠지. 대형서점 런칭은 어느 정도 했던 모양인데 판매량은 글쎄.
\vspace{5mm}

\textbf{D : 고시 방법론을 응용한 문제집.}
\vspace{5mm}

서브노트가 좋아보여서 구입한 문제집이다. 이것도 지금 생각해보니 가성비 \textbf{최악}.
이 문제집은 저자들 학벌 스펙이 그럴싸하다. 그리고 문제푸는 방법론 $-$ 즉 형식적 측면도 새로운 건 있다.
그런데. 그게 전부다(...) 그리고 그 모든 엑기스가 서브노트 한권에 다 있다(...)
다시 말해서 서브노트만 챙기면 나머지는 볼 필요도 없다.
\vspace{5mm}

이 저자들도 C와 비슷한 케이스다. 그런데 C는 저자 한명이기라도 했지, 이 문제집은 저자가 여러명이고 다들 스펙이 좋은데도
모두 기출해설만 했다라는 게 문제다. 그리고 지금 가격 확인해보니까 이것도 가성비 F4 에 들 정도네
저자들이 문제 해설 썰만 그럴싸하게 풀지, 자기들도 \textbf{문제를 만들 능력은 '없다'.}
그런데도 서문에서 만들라고 고생했다라고 하는 건 뭘까.
\vspace{5mm}

문제푸는 방법론도 사실 냉정히 보면 그다지 새롭지는 않다. 이거 내가 알기론 고시 2차 답안작성 방식 그냥 베껴온 경우다.
수학문제 푸는 방식은 이채형이나 강필 인강을 들어도 좋지만 일본 책들 찾아봐도 잘 나와있다.
아니, 조금 부담되더라도 수리논술 양식을 참조해도 된다.
\vspace{5mm}

이거 뭐 책 내용도 없는 거 비싼 종이에다가 칼라풀 인쇄만 했는데 저자들이 뭔 생각이었나 궁금하다.
아무리 보아도 학벌빨만 더럽게 믿고 애들 현혹해 학원 수입이 많아지니까 욕심이 생겨 출판계로 나아가보자라고 한 모양이다.
책 함부로 쓰는 게 아니야, 그리고 전에 학생저자들 개념서 함부로 보지 말라고 했지만 '강사' 책도 이건 마찬가지라고 생각한다.
\vspace{5mm}

\textbf{E : 강사가 낸 낚시용 문제집}
\vspace{5mm}

참고로 이 강사는 꽤 스펙과 실력이 좋다. 그런데 책을 내려면 다 낼 것이지 일부만 내놓고 낼름 도망가버렸다(...)
이건 씹는다기보다는 그 강사의 안목이 참 근시안적이라고 탓하고 싶다.
우선 이 교재는 지수, 로그, 수열에서 나름 접근 스킬이 나쁘지만 않다, 특히 군수열적 풀이에서는 접근이 꽤 명쾌했다고 본다.
그래서 이 교재 후편을 기다렸는데 아무 소식이 없다, 그렇다고 이 강사가 그럼 일타로 잘 나가나. 그런 것도 없다.
걍 어차피 수명이 있는 것 실전개념서 출판이나 다 해서 출판계 점유율이나 높이지 뭐 했나.... 라는 생각이 든다. 지금도 일하고 있으려나?
\vspace{5mm}

\textbf{F : 명문대 합격을 핑계로 낸 질소과자}
\vspace{5mm}

이름부터가 노골적. 이거 학생저자가 낸 책이고 일단 스펙이 후덜덜해서 좋아보이지만 속지 말자.
수학 전분야를 다 터치하는 것 같은데 알고보면 기출문제 그럴싸하게 만만한 것만 추려내서 해설 적당히 함.
수험론적인 걸 강조는 하는데 그거 어차피 다 아는 내용, 그나마 챙길 게 '수험생들이 하는 xx'를 추려낸 것.
나머지는 볼 게 없다. 저자 스펙 내세워서 장사질한 사례고 내가 이래서 학생 저자들을 일단 불신하고 보는 것이다.
목차 구성은 나쁘지 않아서 이 녀석 머리는 좋구나 했는데, 이거 결국 설렁설렁 대충 내용 채워넣었다라는 의심을 지우기 어렵다.
\vspace{5mm}

여기서 E 빼고 나머지는 \textbf{도대체 왜 출판했나 하는 생각이.}
물론 부분적으로는 챙길 게 없진 않다.
C는 말투(...), D는 서브노트, 그리고 F는 목록. 시간 되면 이것만 다 추리고 나머지는 다 버릴 생각이다.
원래는 다 버리려고 했는데 지금 보니 저것들'만' 쓸모가 있다.
그런데 나야 뭐 급박하지 않으니까 저것 빼먹기만 할 수 있지, 적지않은 거금을 투자해 저것들을 구입한 수험생들 인생은 어찌된 것일까.
E는 괜찮다면서 왜 깠냐라고 하는데 후속편 수년간 기다려봐라. 짜증 안 나나.
(는 건 뻥이고 후속편 나중에 적당히 알아보았는데 생각보다 별 게 없더라는 함정)
\vspace{5mm}

가성비는 뭐. 지금까지 보면 C, D도 만만치 않게 최악이란 느낌이 들면서
그 돈을 모아서 미소녀 화보를 직구했으면 눈보신은 할 수 있지 않았을까라는 위험한 생각이 들기 시작하고 있다.
일단 C의 경우는 해설 말투, 그런 것 필요없고 그냥 EBS 기출강의 들으면 되는 것이고
D의 경우는 수학문제풀이와 관련된 책에 훌륭히 나와있거니와 고시 2차 답안 작성 같은 것 따로 구해보면 된다.
그리고 F의 xx목록은 본인이 직접 일기를 쓰면 해결할 수 있는 것이고 사실 풀이만 차분히 쓰면 다 해결된다.
그리고 E도 사실, 저 정도 팁은 풍산자에 다 있다(풍산자는 저자 분이 불편한 현실에서도 책 잘만 쓰시더라)
\vspace{5mm}

수학교재 구입하려는 분은 이 글의 추상적인 단점들 보고 구매할 때 신중히 고려하길 바란다.
C는 그냥 뭐랄까, 경기 어려운 시절에 살아남을 수 있을까. 이 경우는 내가 보기엔 걍 썰만 풀 줄 알고 자기 공부는 안 하는 도태 케이스인데
D는 ‥ 강의까지 내가 찾아서 다 확인해보았는데 나도 조심해야겠단 생각이 들었다. 이 역시 별로 발전은 없다, 그냥 학벌빨로 버티고 있어.
F는 이제 책이 안 나오지 않을까 싶은데 어떤 의미에서 잘 된 일이다. 그런데 이건 양심이 아니라 그냥 저자가 바빠서 도망간 것 같다.
일단 C는 그거 풀다간 같이 한심해져서 서민이 될 판이고
D는 형식이 그럴싸한데 형식만 만들고 나머지 개념과 기출은 걍 표절한 수준, 그래서 내가 블랙리스트에 넣었다.
F는 뭐. 그런데 다른 의미로는 리스트를 만들어서 조심해야 할 판이다. 이런 놈은 사기를 잘 칠 것 같다.
그럼 E는? 소심하지. 실력과 내용이 좋은데 통크게 장사 못 하니까 결국 뜨지 못한 듯.
\vspace{5mm}

그렇다능.
\vspace{5mm}






\section{운명}
\href{https://www.kockoc.com/Apoc/465783}{2015.11.06}

\vspace{5mm}

사주팔자가 다 맞나 안 맞나 그건 모르겠지만, 적어도 인터넷 덕분에 검증을 할 수는 있다는 것.
점집 50군데를 돌아본 사람이 정작 맞춘 곳은 2곳 밖에 없다라는 식으로 나오는데, 이 '확률'은 그냥 원숭이에게 OX를 맡기는 것보다 낮다.
\vspace{5mm}

그래도 사람의 운명이란 건 왜 정해져있다고 느끼게 되는 것일까에 대한 과학적(?) 접근을 시도해보면서 느낀 것.
그건 '익숙함'과 관계가 있어서라는 것이 요즘 드는 생각이다.
불행함에 익숙한 사람은 불행한 삶을 살지 않으면 불안해 한다.
공부 못 하는 것에 익숙해진 학생은 성적이 안 좋게 나오면 불안해 한다.
이게 뭔 개소리요 하는 사람들이 있을 수도 있겠다만 적어도 내가 관찰하고 경험한 바로는 이게 정말 만만치가 않다.
\vspace{5mm}

유전 이야기를 하는 경우도 많은데 이건 어설픈 유사과학적 접근이 아닐까.
과학적 접근이라는 건 관찰을 오랜 기간 해보면서 데이터를 축적해보고 가설을 검증해나가는 과정을 거쳐야하는 것이다.
그런데 유전적인 것이 실제로 검증된 적은 내가 아는 한 별로 없다. 부모가 유전자가 좋아서 자녀가 좋다면,
반대로 부모가 평범하거나 못난이인데 자녀가 좋은 경우는 어떻게 설명한단 말인가.
부모가 유전자가 안 좋아도 자녀가 좋을 수도 있다는 이야기라면, 부모 유전자가 좋아도 자녀가 안 좋을 수도 있단 얘기가 된다.
\vspace{5mm}

정말 제대로 검증해본다면 드라마처럼 의사가정과 서민가정의 자녀가 산부인과에서 바뀌었더라.... 는 걸로
'가정환경'이라는 통제변인 조절이 있어야하지 않나.
\vspace{5mm}

예전에 모 프로그램에서 쌍둥이 사주를 검증해본 적이 있다.
일란성 쌍둥이 자매였으므로 ⓐ 동일한 유전자 ⓑ 동일한 사주 $-$ 그런데 한명은 미국으로 입양되었다.
그 결과는?
\vspace{5mm}

한국에 남은 사람은 무속인이 되었고, 미국에 입양된 사람은 '대학교수'가 되었다.
이 당시 사주관련한 까페나 블로그에서는 대난리가 났다. 당연히 업자들의 그럴싸한 변명이.
이게 시사하는 바는 가장 중요한 건 '환경'이라는 이야기이다.
어느 지역에 사느냐, 어떤 가정에서 자랐느냐, 그리고 명문학교에 갔느냐 못 갔느냐 라는 것이 매우 중요하다는 이야기이다.
\vspace{5mm}

자기가 머리 탓을 하는 학생들은 우선 자기가 어떤 환경에서 살고있는가 그것부터 검증해보면 된다.
그냥 검증하지 말고 정말로 공부를 잘 하는 레전드들과 하나하나 비교해보면서 체크해보면 되는 것이다.
사소한 차이가 중대한 변화를 가져오는 것이다.
\vspace{5mm}

이렇게 보자면 내 입장도 수정해보긴 해야겠다. 내 경우는 어느 교재든 사실 그리 큰 차이는 없다고 본다.
비판한 A$\sim$F 교재도 사실 본인이 열심히 본다면 큰 차이는 없을 것이라고 보는 입장이긴 한데
생각해보니 이건 내가 어느 정도 많은 교재를 거쳐서 면역력(?)이 생겼기 때문이라고 할 수 있단 생각이 들어서 수정해야할 것 같다.
교재도 일종의 환경이라면 \textbf{'교재 잘못 선택해서 인생 말아먹은 케이스'가 많다}라는 게 논리적인 결론이다.
하지만 이걸 어떻게 얘기해줘야 할까.
\vspace{5mm}

공개적으로 교재 실명을 대면서 칭찬하거나 비판하기 힘든 이유는 현실의 민감한 이해관계 때문이다.
내가 양심과 노력으로 그거 검증해서 얘기해보았자, 당사자 입장에서는 $-$ 어차피 자기는 죄책감 없다, 돈만 벌면 장땡 $-$ 속이 타버릴 뿐이지.
또한 내가 좋다고 했는데 이걸 믿고 선택한 사람들이 정작 그 교재 때문에 망했다고 한다면(사실 교재보다는 다른 이유가 크다고 보지만)
이건 내가 할 말이 없어지는 것이기 때문에 교재 추천에 있어서 매우 신중한 태도를 취할 수 밖에 없다.
\vspace{5mm}

그런데 교재가 하향평준화되는 이유. 이건 시장의 축소도 축소지만 한가지 격세지감을 들어볼까.
합리적이 된다는 건 좋은 일인데 이게 과거보다 어떤 의미에선 이번 모 가수 제제 사건만큼 세기말이라고 느껴지는 게
가만히 보면 다들 눈에 $ $ 표시를 하고 다닌단 것이다.
이거 수험생들은 못 느낄 수도 있겠지만, 타락한 내 눈에야 훤히 보인다.
말로는 좋은 문제를 만들어서 공유하고 싶다... 라고 하는데 당연히 그건 공산당 독재 하기 전에 농민들 꼬실 때 하는 소리고,
현실은 나중에 자기 교재가 얼마나 많이 팔렸나 그런 걸로 자랑하고 위세떨고 다니는 참 거북한 광경을 보게 된다.
이런 것을 비판하면 "너 질투해서 그렇지"라는 반응이고 심하면 고소미를 먹기 일쑤인 것이다.
물론 그런 교재들이 훌륭한가.... 전혀 아니올시다이다. 벌거벗은 임금님이란 동화책이 괜히 고전인 게 아님.
그렇다고 그 사람들 실적이 훌륭한가, 내가 보기엔 그것도 아니야.
\vspace{5mm}

그런데 더 유감스러운 건 그런 사람들이 위세를 떨고 다니니까 지금 수험생들도 하라는 공부는 안 하고
어떻게 하면 자기들도 저 존경스러운(?) 선배님들처럼 되어서 교재 팔아서 수천수억을 벌어볼까 그 궁리를 하는 게 보인단 이야기다.
참교육이 따로 없다. 이런 것들에 영향을 받기 시작하면서 운명이 결정되는 것이다.
열심히 공부해서 수험에 매진해도 모자랄 판에 어떻게 교재 만들어서 그걸 팔 궁리해볼까라는 쪽으로 간다면 사실 심각하다.
그렇게 벌어댄 돈, 그 자신이 날려버리는 젊은 시절에 비하면 정말로 \textbf{푼돈}이다. 당연히 돈에 눈이 멀어댄 사람에겐 이런 충고가 안 들린다.
돈은 거짓말아니고 나중에 어떻게든 벌 수는 있지만, 공부와 경험에 투자했어야 하는 그 젊은 시절은 수천억을 줘도 돌아오지 않는다.
이 경우는 나중에 나이먹으면 정말 후회들 한다. 그 시절에만 할 수 있었던 중요한 배움이나 일이라는 게 있다.
\vspace{5mm}

환경이라는 게 이렇게 참 무서운 것이다.
인간이 운명을 바꾸기 어려운 것은 자기가 익숙한 환경이나 달콤한 금전적 수익을 버릴 수 없기 때문이다.
하지만 그게 정말 자기가 '패망하는' 길이라는 걸 안다면 과감히 벗어날 것이다.
2차 세계대전이 터지는 걸 알았다면 그 직전 프랑스나 이탈리아에 사는 유대인들은 과감히 미국으로 탈출했을 것이다.
하지만 모든 미래는 사실 $-$ 자기가 보고싶지 않은 미래일수록 $-$ 매우 터무니없어보여서 무시하기 좋다.
\vspace{5mm}

이번 시험을 치르고 결과가 시원치 않은 사람은 우선 '환경'부터 점검해보는 걸 추천한다.
\vspace{5mm}

운명을 바꾸는 법 $-$ 소위 개운이라고 한다면 그건 굿판이나 부적일 수도 있겠지만(사실 그리 효과는 크지 않다)
가장 좋은 건 사는 곳을 바꾸는 것이다.
사주팔자를 신봉하는 입장인 경우도 죽을 운일 때 외국으로 튀어서 겨우 악운을 피했다라는 믿거나말거나 글이 있다.
사는 곳을 바꾸기, 공부하는 곳을 바꾸기.... 이것만큼 영향이 가장 큰 것은 없을 것이다.
인강과 실강의 차이도 이런 데 있지 않을까.
\vspace{5mm}

인강은 혼자 듣는다. 따라서 자기가 공부를 안 하게 되더라도 그걸 파악할 수 없다.
하지만 실강에서는 공부하는 다른 친구들을 보게 된다. \textbf{자기가 조금이라도 안 하면 뒤처지는 것을 본다}(이게 가장 중요하다)
방에서 혼자 공부하는 건 인강에 준한다. 그러나 도서관에 가면 열심히 책읽는 사람을 보고 거기에 싱크로를 맞추게 된다.
\vspace{5mm}

교재가 감성적으로만 쓰여져 있다. 이런 교재를 본 사람은 감으로 문제를 풀 것이다.
교재가 스킬 위주로만 적혀있다. 스킬에만 의존할 것이다.
교재가 논리적이다. 시간이 걸리지만 그 수험생은 논리적으로 문제를 푼다.
\vspace{5mm}

하지만 공부를 열심히 하기는 쉬워도 자기 환경, 즉 관성을 바꾸기는 매우 어려워보인다.
공부하겠다고 다짐만 하면서도 1년 넘게 안 하는 사람들도 널렸다.
그 사람들은 다시 시작해야지 하면서 교재 주문 하고 계획짜는 걸 한 수십번은 반복했을 것이다.
\vspace{5mm}

운명을 바꾸기 힘든 이유가 이런 데에 있지 않을까.
\vspace{5mm}

사주팔자를 긍정한다고 하더라도 운명이 정해져있다면 이건 사실 필요가 없다.
운명을 바꾸고 싶어하니까 사주팔자를 따지는 것이다. 그런데 정작 사주팔자에는 운명을 바꾸는 방법은 없다.
단지 언제 좋을지 나쁠지 그것만 간략하게 제시되어 있다.
\vspace{5mm}

세계사는 사실 서양이 동양에 승리한 게 1부다. 지금은 그 서양을 배운 동양의 패자부활전 과정?
자본주의 산업혁명 여러가지 말이 많지만 그것들도 결과이지 근본적 원인은 아니다.
가장 근본적인 것은 바로 종교다. 그렇다고 우리나라 일부가 얘기하듯 하느님을 믿어서 어쩌구 그런 게 아니다.
기독교는 신과 인간이 계약 관계이다. 즉, 인간은 자기가 의무를 이행한만큼 신으로부터 권리를 얻어낼 수 있는 것이다.
이런 기독교의 사고방식이 적극적이고 개척적인 역사의 진보를 낳는데 기여한 것이다.
동양에는 이런 사상이 사실 없다(억지로 정신승리하려고 맹자 이야기를 끌어내거나 양명학을 얘기하지만 다들 한계가 뚜렷하다)
\vspace{5mm}






\section{교재 이야기 : 풍산자썰}
\href{https://www.kockoc.com/Apoc/465841}{2015.11.06}

\vspace{5mm}

http://news.naver.com/main/read.nhn?mode=LSD&mid=sec&sid1=001&oid=036&aid=0000004714
\vspace{5mm}

사지 멀쩡한 사람들도 교재 대충 쓰는데 이 저자 분은 불편한 처지에서도 교재를 꽤 잘 쓰는 편이다.
혹자는 이런 비판을 할 수도 있다. 저자 처지를 내세워서 마치 공정무역 커피 맛있다고 뻥치듯 광고하는 것 아니냐 그러는 것.
\vspace{5mm}

우선 이 교재는 '수포자용'이라고만 오해를 사고 있는데 절대로 아니올시다.
\vspace{5mm}

수포자용이라는 건 이 교재가 쉬워서가 아니다. 사실 이 교재의 유제나 연습문제는 어려운 건 대단히 어렵다.
다만 그 어려운 것이 조잡한 교재의 고난이도 문제나 스킬 요하는 그런 걸 필요로 하지 않는다. 즉, 수능에만 적합하다는 이야기다.
허세만을 좋아하는 수험생들은 "\textbf{어라, 수포자용이라고? 정말 쉽겠네. 그럼 내가 보면 안 되지"}라는 프로세스로 움직여서 그런 것이지
\vspace{5mm}

이 교재는 중하위권을 배려해서 그렇지,
수학 개념의 독창적인 해석이나 수록 문제는 상위권용이다.
사실 풍산자의 직관적이고 감각적인 개념 설명은 오히려 킬러 문제를 풀 때에 매우 요긴한 것이다.
이 교재로 공부해보신 분은 웬만한 수험수학의 팁 $-$ 다시 말해 수능에 필요한 정도 $-$ 은 다 들어가 있다는 데 동의할 것이다.
\vspace{5mm}

상위권을 배려(?)하는 교재는 사실 누구라도 쓸 수는 있다. 적당히 짜깁기하면서 문제를 대충 어렵게만 내놓고 해설 무성의하게 쓰고
이거 이해 못 하는 건 네가 공부 안 해서임 ㅋㅋ 이라고 구라까면 되기 때문이다.
진정한 교재집필의 고수라면 '중하위권'용 책을 쓴다.
게다가 이 교재의 개념 설명은 볼 때마다 감탄이 나올 정도이다.
다른 논란을 불러일으킬지 모르지만 쎈의 개념 설명이 갤xx이라면 풍산자의 개념 설명은 아xx가 아닐까.
그 이야기는 거꾸로 말해서 개넘 설명이 너무 직관적이고 감각적이므로, 논리적으로 엄밀한 부분은 취약할 수 있단 이야기인데
이건 다른 교재로 보충하면 된다.
\vspace{5mm}

이 글을 읽는 고1, 고2는 뭔 교재로 재기해야 하나... 하면 이 교재로 틀잡는 걸 권한다.
교재 권장에서 가장 중시하는 건 '부작용'이 있거나 혹은 잘못된 것을 엉터리로 배우는 게 아니냐는 것인데
물론 이 교재도 그런 게 없을 수는 없겠지만 적어도 내가 아는 한도에서는 발견된 건 없다.
유일한 단점이라면 양치기를 하기 힘들단 것인데 이건 쎈이나 RPM을 병행하면 된다.
\vspace{5mm}

개념의 감각적 해석 $-$ 문학적 비유가 마음에 안 든다는 사람들이 있을지도 모르겠지만 그건 몰라서하는 이야기다.
\vspace{5mm}

여기서 다시 교재뒷담화를 하면 나는 기본적으로 교재 저자들이 국어 실력이 없고 문과적 소양이 부족한 경우는 피해야 한다고 보는 입장이다.
수학 공부를 열심히 했고 계산이 빠르고 수식이나 그래프 구사가 뛰어난데도 수학 점수가 안 나오는 애들이 왜 안 나오는지 아나?
이런 애들, \textbf{국어 실력이 꽝이다. 문과적 소양이 없다}.
\vspace{5mm}

수학 문제는 수식이나 그래프로만 쓰여진 게 아니다. '한국어'로 쓴 것이다.
문제를 풀 때 가장 중요한 건 문제를 '해석'하는 것이다. 그리고 저자가 뭔 의도인가 그걸 알아내는 것이 중요하다.
그런데 이런 독해는 유감스럽지만 수학 교재들에서 누락시킨다.
수학교재들은 대체로 수리적인 풀이만 선사하지, 문리적인 해석에 대해선 언급을 피한다.
그래서 이른바 문제집은 많이 풀고 인강은 들었는데 결정적인 데에서 막히는 '괴혈병'에 걸리는 것이다.
\vspace{5mm}

이건 잘 나간다는 수학고수들 관찰해보아도 그렇다, 그 친구들의 계산, 발상력은 보통 학생의 3$\sim$5배는 된다.
그러나 그것 때문에 공부가 편중되어서 국어나 문과적 소양이 나가리나버린다.
그래서 문제를 잘못 읽은 채로 자기 확신에 빠져 엉뚱한 데 헤매기도 하고,
개념의 논리로 풀면 간단할 것을 자꾸만 화려한 기교와 수식을 연마하다가 맛가는 경우가 많다.
\vspace{5mm}

이런 이유 때문에 교과서를 읽으라고 하는 것이고, 가능하면 수학을 국어적으로 읽으라고 하는 것이다.
풍산자의 개념 설명이 이 점에서는 (다소 부족할지 모르나) 종합비타민제 역할은 해주고 있다.
수학을 잘 한다는 친구들일수록 이런 점을 경시한다. 그리고 실전에서는 터무니없는 점수를 맞는다.
이런 일이 벌어지는 이유는 간단하다.
\vspace{5mm}

평소의 모의나 사설모의, 그리고 시중교재는 패턴화되어있다. 그래서 문제를 대충 읽어도 뭔지 알 수 있다.
그러나 수능은 교수들이 새롭게 '내는' 것이다. 그래서 출제 방향을 다소 비틀거나 꼰다.
그렇기 때문에 문제의 해석을 잘못 하면 나가리나기 딱 좋다. 국어를 무시하는 친구들이 여기서 덫에 걸려드는 것이다.
\vspace{5mm}










\section{콕콕에서 교재평할 때 룰을 정해드리겠음.}
\href{https://www.kockoc.com/Apoc/467585}{2015.11.07}

\vspace{5mm}

과거 게시물 보면서 댓글(?)을 보면 제가 왜 교재평을 구체적으로 안 했는지 그 이유 이제야 이해가시는 분 많을 겁니다.
\vspace{5mm}

저야 허심탄회하고 가볍게 글을 써도 두가지 면에서 부담이 되어요.
\vspace{5mm}

첫째, 그냥 가볍게 쓴 것인데 조회수가 높아진다(... 이거 ㅇㅂ게시판에서도 그랬음 ... 그런데 그런 내용조차 조회수 높았으면 얼마나 막장이었냐)
둘째, 그거 해당 저자들이나 출판사가 보면 가만히 안 있음. 소통은 개뿔, 어찌되었든 비겁한 공격이 들어올 가능성이 있다.
\vspace{5mm}

그렇다고 교재 평을 안 할 수는 없겠고.
이건 안전한 룰이 없나 싶어서 생각해보는데 간단하더구만유.
다음과 같이 적으시길 바랍니다.
\vspace{5mm}

\item \textbf{1.}  칭찬하는 교재는 구체적으로 \textbf{장점과 단점을 평해준다 : 단, 이건 권해줄 수 있는 교재에 한한다.}
\vspace{5mm}

호의적인  비평글은 구매자의 구매의욕을 불러일으킵니다.
자기 교재  좋다고 하는데 뭐라하는 정신나간 업자는 없겠죠.
다만 명예훼손이나 모욕적인 건 그래도 삼가시는 게 좋습니다. 단점은 구체적으로 사실지적만 하면 되는 것이겠죠
\vspace{5mm}

\item 2. 칭찬하지 않는 싶은 교재는 그냥 \textbf{"언급을 하지 않습니다", "노 코멘트", "언급제외", 아니면 "잘 모릅니다"}라고 언급한다.
\vspace{5mm}

생각해보니 이게 그나마 무난하더구만유. 언급하지 않는다고 고소미 먹이면 그냥 이 대한민국이 정신나간 나라라서 말입니다.
사실 제가 교재 저자라면 누가 깐다고 하면 적극적으로 소통하고 해명할 것입니다. 이게 정상이죠.
그런데 이 나라에는 비정상들이 많아요. 똥이 더러우면 피해갈 수 밖에 없죠, 그렇다면 위와 같이 그냥 한줄로 정리하는 게 좋습니다.
물론 교재비평글을 쓰라는 게 아니고, 콕콕 회원 분들이 교재 질문을 받을 때는 저렇게 한줄로 답하란 이야기.
\vspace{5mm}

\item 3. 교재 비난을 하고 싶으면 \textbf{추상화}시켜라.
\vspace{5mm}

가령 "Z라는 양반은 탐욕스럽고 여자나 밝히고 한두번 쓰레기버린 게 아니고 인성 쓰레기다"라고 익명성으로 적는다면
이거 자기가 Z니까 문제된다고 나서는 정신나간 사람은 없겠죠. 설령 엉뚱한 사람이 자기 욕한 거지라고 갑툭튀해보았자 이건 법과 관계없어요.
다만 추상화라는 것은 정말 철저한 익명화를 얘기하는 겁니다.
\textbf{가령 '그 실전모의고사 개쓰레기야, 필적확인문구가 현 정권 비난하는 시사적인 것이었어. 신문기사에 난 적도 있었지'}
\textbf{라고 한다면 이건 안경 낀 이토 준지처럼 보이지만 실은 colorful bone으로 알려진 모 모의고사 저자로 특정될 가능성이 있어서 문제가 되죠.}
하지만 반면 '거창하게 광고했는데 기출만 박아놓았더라', '분명 스킬 다 소개해준다고 했는데 스킬 별로 없더라'
이렇게 기술하는 건 문제가 되지 않습니당. '교재값 비싸더라' 혹은 '팬들이 많다'도 그렇죠. 이건 엄연히 복수형인지라.
\vspace{5mm}

아무튼 최근에 있었던 모 사건을 보고나서 그렇다고 교재 비평 문화를 활성화 안 시킬 수는 없고 어떡하나 하다가
서로 낯붉힐 것 없는 룰을 정하면 된다고 해서 떠오르는 아이디어를 적어보았습니다.
\vspace{5mm}

가장 중요한 4번 추상화를 할 때에는 가령 교재 둘을 섞어서 가상의 교재를 가정해서 적어도 되겠고
혹자는 일부러 몇몇 사항들을 바꿔넣어서 아예 논란이 되지 않게 하는 방법도 있을 겁니다요.
하지만 가장 중요한 건 철저한 익명화죠. 그 경우 모 교재로 특정하는 사람 자체가 악플러 이거나 그런 걸 노리는 세력이라는 게 드러날 정도로 말이죠.
\vspace{5mm}

앞으로 콕콕 사이트는 계속 성장할 테고 들어오는 사람들도 많아지는 동시에 '악인'들도 적지않게 들어올 겁니다.
조만간 회원등급이나 권한에 대해서는 변경이 있을 거라고 예고 받은 바 있는데, 저런 룰을 관습화시키는 게 중요하다고 생각하네요.
특히 나쁜 교재야 특정할 필요없이 '나쁜 특징'을 가지고 추상화하면 그것만으로도 정보전달은 충분하기 때문입니다.
\vspace{5mm}

그리고 운영진 차원에서도 만약 그런 추상화된 비평글에 \textbf{특정교재나 강사 지목하는  댓글이 달리면}
\textbf{그건 삭제하고 그 회원은 권한 박탈하거나 차단시켜야겠죠.}
이렇게 하면 교재 비평글은 활성화될 수 있다고 생각하고
안 그래도 정보부족으로 시달리는 수험생들에게 시원한 사이다가 되겠군요.
\vspace{5mm}

아울러 문제가 많은 출판사나 법적인 문제가 있을 것 같은 교재는 '양해 구하고' 금지어로 지정하는 것도 사실 필요하다고 봅니다.
회원수가 늘면 어떤 게시물에다가 어떤 트러블일 날지는 모르겠는데 운영자 분 입장에서는 아주 번거로운 일이 되겠죠.
그렇다고 여기가 상업주의적으로 모 업체 교재만 아니면 광고 안 되는 그런 곳도 아니긴 하지만요.
\vspace{5mm}






\section{교재 뒷담화 : G H I J K}
\href{https://www.kockoc.com/Apoc/468220}{2015.11.07}

\vspace{5mm}

가르치는 사람은 배움을 게을리하지 않으면 안 되는데
이건 두가지 이유임.
\vspace{5mm}

첫째, 아는 게 있어야 가르친다.
둘째, 배우지 않으면 본인이 \textbf{교주가 되어버린다}.
\vspace{5mm}

그런데 첫째까지는 아는데 둘째를 경시하고 그렇게 괴물이 되어가는 사람들을 본다.
가르치면서 선생님 선생님 받들여지고 하면 처음에는 기분이 좋은데, 그게 1년동안 지속되면 안하무인.
그래서인가 강사들이든 저자들이든 인간성 측면에서는.... 언급을 하고싶지 않은 경우라고 정리할 수 있다.
\vspace{5mm}

교재 G : \textbf{지금은 스산해져버린 온갖 스킬 꼼수의 향연}
\vspace{5mm}

이 교재는 구하기 어렵다. 그리고 개인적으로는 버릴 생각도 없다.
왜냐하면 스킬 꼼수와 사파적인 개념에 대해서 이만큼이나 잘 모아놓은 기록도 없기 때문이다.
읽고 정리해보면서 오, 한 때 인기를 끄셨다는 분이 어떻게 독창적으로 해석했는지 그리고 이게 왜 인기를 끌었는지 그것도 알 수 있을 것 같다.
사실 이 저자에게는 나 개인적으로는 고마움을 표하고 싶다. 단지 이 분이 '인기'에 취해서 이미지 관리나 하다가 발전없이 도태된 것이 아쉬울 뿐이지만.
\vspace{5mm}

우선 저 스킬은 재밌게 가르치고 싶은 사람에게는 매우 요긴하다.
그리고 수험적으로 쓸만한 건 있다. 가령 문과 21번 킬러에서 다항함수 조건 주고 f(k)=? 하는 문제에 쓰일 스킬 같은 것이 다 나와있다.
하지만 그 뿐. 그 외에는 수능에 도움될 건 없다. 이게 상당히 유감스럽다.
끝까지 읽고 정리해보면서 느낀 건 한편으로 매우 감사하다는 것, 그러나 다른 한편으로는 사파로 가면 진화를 못 하는구나하는 경고였다.
(그리고 이것이 내가 강의에 너무 중독되지 말란 이유이기도 하다)
\vspace{5mm}

지금 수학교재 함부로 쓴다고 하는 사람들이 날고 기어보았자 저 분의 책을 따라가긴 힘들 것이다.
이 책은 노량진 홀로서기에서 구매한 책이다(pdf 파일 따로 복사하려고 커피 마시면서 노가리까다가 재밌어보이기에 구매)
가성비는 좋다고 할 수 있다. 매우 저렴했으니까 $-$ 수십만원 들어야 겨우 몇줄 챙기는 스킬들이 2만원 내에로 수십개는 나와있음.
어떻게 보면 사파 수학으로서 실력자인데 지금은 어디서 뭐하는지 알 수 없다(...) 그냥 은퇴하신 건가.
교주놀이 끝을 보는 기분?
잠깐, 이거 뒷담화 맞긴해?
\vspace{5mm}

교재 H : \textbf{대학교수들이 쓴 책으로 기대를 모았지만}
\vspace{5mm}

수학교육과 전공이 아니면 그다지 기대하지 말자, 걍 경문사 책이나 보거나 일본책을 보는 게 맞다는 것만 얻었다.
이 책은 가격이 꽤 세다(담긴 내용에 비하면). 일단 저자들은 스펙이 상당한 교수님들.
왕년에 나 수학 잘 했어 라는 심정으로 쓰신 책인 것 같은데 아뿔사.
기출 실어놓은 게 1990년대 초기 수능 기출. 그거 해설로 매진한 건 좋은데 책 내용 상당수가 걍 오일러 정리(...)
복소수 개념에 대한 회전변환 같은 거야 괜찮게 적긴 했는데 괴수님들, 이거 요즘 수능과 무관하다고욧.
스킬이 안 실린 건 아닌데 그 스킬들이 사실 별 소용이 없는 것들(...) 어떻게 된 게 저기 G만도 못 하냐(사회적 스펙은 더 좋구만)
이건 뭐 똥뱃살 더럭더럭 찐 왕년 권투선수가 이종격투기 게임에 나와 상대도 경로우대를 안 할 수 없는 그런 상황.
\vspace{5mm}

교수라고 무조거 믿으면 안 된다라는 선례를 남겨주었다.
수험용이 아닌 그냥 심심풀이 책이면 괜찮지만, 수험용으로 보았다면 그 학생은 +1 확정.
\vspace{5mm}

교재 I : \textbf{패턴정리 잘 되어있다라고 해서 기대했는데 영$\sim$}
\vspace{5mm}

이 사람은 학벌도 괜찮고 아예 이 분야에서 가르친다라고 해서 구매. 이거 일다 문과용인데 $-$
뭐랄까 상당히 실망스러웠다. 저자는 방송에 등장한 걸 보니 선량한 사람이고 게다가 이 분야만 적극적으로 판 경우이긴 한데
아무래도 강의만 하다보니까 책은 그냥 실망스럽게 만들 수 밖에 없지 않나 그런 느낌이 들었다.
문과 수학인데 문제에 대해서 '패턴적인 접근'만 하고 있다, 결국 탈패턴까지는 못 가고 있다.
다시 말해 $\sim$ 하게 풀어야한다라고만 기술하고 있지, 왜 그런 풀이가 나오나, 그리고 수학적 정의에서 어떻게 그런 자명함이 나오느냐 없다.
게다가 문과수학인데 편미분은 왜 동원했냐 도대체(...)
수학 외 다른 자전적인 얘기까지 쓰고 그건 감동스럽긴 한데. 결국 뭐야, 발전이 없잖아(...)
\vspace{5mm}

그래도 인간적으로 이 사람은 깔 필요는 없는 것 같다.
이 분야 세계를 보면 정말 '신사적'인 사람고 그렇지 않은 사람들이 있다.
그냥 아무 것도 아닌데 대충 올라가서 수험생들에게 위세떠는 천민들이 있고,
자기들이 원래 공부를 잘 했기 때문에 중하위권을 배려하는 신사들이 있다. 적어도 이 저자는 신사이다.
다른 건 몰라도 힘든 사람들을 위해 상담해주고 선량한 마음으로 소박하게 일하면서 교육 본연에 충실한 건 인정할 수 밖에 없다.
그리고 이 사람, 무엇보다 교재 추천 목록이 나랑 거의 흡사하다. 비슷한 교육코스다보니까 그런 게 아닌가.
\vspace{5mm}

G는 어떻게 되었는지 모르겠지만 책의 스킬은 감사. 이건 내가 따로 연구해보아서 콕콕에서 공개할 수도 있음.
다만 G로 공부하면 수학을 제대로 공부했다... 라고 하긴 힘들다는 점에서는 문제인 책이다. 실력자들이 보면 읽을 수 있지만.
H는 대학교 수학은 고교 입시수학과 다르다는 반례로서 충분치 않나 싶다.
그리고 사실 본고사 세대가 '암기수학'을 했다는 반증임.
\vspace{5mm}

I는 맨 앞에 쓴 그 교주가 되지 않길 바란다.
자전적 기록이나 고발서를 보면 이 사람 꽤 괜찮다 느낌도 있지만 안 타락한다라는 보장도 없지.
이 사람도 2d에 취향이 있다면 내가 적극옹호하지 않을까 싶은데 잠깐 뭔소리하는거지 내가?
\vspace{5mm}

... 그런데 여기서 끝나면 섭하지
\vspace{5mm}

\textbf{J : 교주놀이의 대가}
\vspace{5mm}

이 책도 지금 버려야할 것 같아서 보고 있다.
이 경우도 신랄하게 까야할 것 같다.
\vspace{5mm}

첫째, 저자가 내가 보기엔 공부를 안 한다
둘째, 교재가 제목이든 내용이든 자가당착이다(교과서를 강조하긴 하는데 해설은 why가 없지 이상한 패턴 위주냐. 이런 책들 꽤 있네?)
셋째, 실으라는 문제는 기출 대충 짜깁기로 실어놓고 북한에서 수령님 찬양하는 듯한 글들은 곳곳에 박혀있냐?
\vspace{5mm}

이 역시 가성비는 시궁창. 아, 이걸 내가 왜 샀지 $-$$-$
저자 스펙은 글쎄, 내 입장에서는 뭐 이래놓고 자랑질을 하고 있냐 그런 생각이 들고있다.
그나마 위의 G는 저자가 스킬 정리라도 했고, H는 교수답게 걍 수험과 무관한 신선놀음 볼만했고, I는 착하기라도 하지.
그나마 스킬이라는 것도 사실 스킬이라고 보기에도 민망한 수준이다. 걍 그렇게 풀어라 하고 적혀있다 $-$$-$
그렇다고 수학 전범위를 포괄하나, 이것도 전혀 아니올시다이다. 설명 필요한 대목은 다른 책을 참조하라고 버젓이 적혀있다.
그냥 기출 중심으로만 해설 그럴싸하게 해놓고 걍 마무리. 그런데 설명이든 해설이든 참 말투가 '교주'스럽다.
\vspace{5mm}

무엇보다 용서할 수 없는 게, 개념 엉터리로 인용한 부분이 한두군데가 아니다.
이게 왜 그런가보니까 교과서나 기본문헌을 제대로 인용한 게 아니다. 아무래도 어디 강의 베껴서 대충 적은 티이다.
강의 중에서 교과서 개념 부정확하게 인용하는 것만 골라서 그게 전부인 줄 알고 딱 적은 모양인데(...). 이렇게 잘못된 지식은 전파된다.
수학사랑에서 나온 박교식씨의 수학용어사전 같은 것도 안 보았나. 용어도 부정확하게 쓰고 있고 '야매용어'를 야매라고 언급도 안 하고 적었다.
\vspace{5mm}

두가지가 미스테리하다. 일단 이 책이 왜 잘 팔렸지? 그리고, 이걸 내가 왜 샀지(...) 이것도 구매하면 10만원은 넘어가는데 $-$$-$
어떻게 보면 나도 사기당한 셈인 것 아닌가 해서 지금 가슴이 갑자기 답답해져오기 시작한다.
열람본을 보고 문장의 거시기함을 직감했을 때 그 때 구매하지 말았어야하는데.
\vspace{5mm}

아마 다른 책들을 안 보고 이 책을 보았다면(뭐 그럴 가능성은 낮긴 하지만) 난 아무래도 중대한 인도적 범죄를 저질렀을지도 모른다.
수학책인지 아니 주체사상 경전인지 헷갈릴 정도의 책이다 $-$$-$
저자 소개부터 머릿말부터 중간중간 자뻑이 아무래도 '수령님 쓰시는 축지법$\sim$'이라는 노래가 딱 BGM으로 적절하거든 $-$$-$
시장을 믿을 수 밖에.
\vspace{5mm}

K : \textbf{아재 xx 서요?}
\vspace{5mm}

이 책은 마이너하다. 대충 고중숙 스타일의 교양서를 꿈꾼 듯 하다(고중숙씨가 쓴 책은 이과 학생이면 꼭 읽을만하다)
그런데 이 아재께서는 xx 서는지 안 서는지 모르겠지만 일단 스펙만 보면 최고대학 대빵이신데....
다른 건 모르겠고 미분과 극한에서 뭔가 썰을 장황하게 푸신다. 뭐 이건 재밌게 나도 공부를 했는데 $-$
중간부터 갑자기 xxxxx은 틀렸다라고 하면서 신림동에서 가끔 플랫카드 걸리는 유사과학 세미나 비슷한 이야기를 하시는 게 문제(...)
이거 아재 치고는 참 귀여우시다(...)라는 생각도 들면서
이공계가 생각보다 사이비 종교에 낚이거나 괴상한 자기 확신에 빠지는 경우가 많다는 걸 떠올리고 있다.
\vspace{5mm}

그래도 중간에 미적분에 관한 그럴싸한 썰은 읽을만해서 보는데. 아무래도 조심을 해야겠다.
유사종교 경전도 자주 읽다보면 정들어서 빠지거든(...)
그래도 이 아재는 J보단 낫다. J는 척 보아도 사기꾼인데 이 아재는 사회스펙상 돈욕심낼 처지는 아니고 정말 순수한 의지가 있는 것 같다.
\vspace{5mm}

지금 J와 K를 동시에 버릴까, 아니면 J만 불살라버릴까 고민 중.
J를 그냥 유포시키는 건 대한민국 학생들 IQ를 떨어뜨려 국력의 저하를 꾀하는 중대한 범죄로 보인다 말야.
K는 그래도 논리라도 분명하지, 종교적인 메시지가 강해서 문제지만(...).
\vspace{5mm}

+
그래도 다시 한번.... 이라고 훑었지만   J는 챰 구제불능이다. 얘는 대놓고 돈만 벌려고 책썼네.
그것도 강사 여러명이 협조해서 써도 힘든 판인데 뭔 깡으로 대충 쓰면서 주체사상을 피력한 건지 이해가 안 간다.
이 사람도 참 멘탈이. 어떤 면에서는 대단한 것 같다.
스펙 보니 이해가 가긴 하다. 확실히 I와 대조적이다.
\vspace{5mm}

++
내 이야기는 아니고(...)
과고 출신들은 참 선량하다는 게 문제.
이게 웃긴 게 과고 출신들은 수학, 과학 같은 걸로 사람들 계몽해야한다 그런 실속없는 선민의식(?)이 있어서 손해보고 있고
비과고출신들은 장사질해먹으려 하면서 실속 챙기고 있다.
\vspace{5mm}












\section{교재 뒷담화 : L M N}
\href{https://www.kockoc.com/Apoc/470245}{2015.11.08}

\vspace{5mm}

우물 밖을 나가보지 못 한 개구리는 우물이 우주인 줄 안다.
동전 밖에 만져 보지 못 한 아이는 아빠가 500원 내와, 1000원 줄께라고 하면 500원을 꽉 쥐고 잽싸게 도망간다.
\vspace{5mm}

한 때 동양적인 것에 대해서 비결에 대한 열풍이 불었던 적이 있다.
동양에는 뭔가 숨겨진 것이 있을꺼야... 그렇게 도사들은 밥벌이를 하였던 것으로 안다.
그러나 그 중 어느 하나도 제대로 검증된 건 없었다.
\vspace{5mm}

태권도가 고구려 무술이라고 믿는 바보는 없을 것이며
역시검도가 삼국시대로 거슬러 올라간다고 믿는 바보는 없을 것... 이었으면 좋겟지만 그렇지 않다.
속는 사람은 끝까지 속게 되어있다. 알지 못 하기 때문에 그럴싸한 말에 속아넘어가는 것이다.
\vspace{5mm}

비급이 사라진 자리 : 대용량 알고리즘을 해결할 수 있는 메모리.
\vspace{5mm}

\textbf{L : 배운 사람 눈에는 시큰둥, 안 배운 사람에게는 비급}
\vspace{5mm}

신비주의 마케팅으로 나온 케이스다.
(입시수학을 공부 안 한 사람)에게는 그럴싸해보이는 책이다. 왜냐면 패턴화, 유형화시켰는데 분량이 적어보여서다.
반면 공부를 제대로 한 사람에게는 턱없이 부족한 책이고 이게 어째서 비급으로 취급되나 의아스럽다.
하지만 우리는 대놓고 말을 할 수가 없다. 한국에서 태권도와 검도가 실제로는 일본 무술을 받아들인 것이라고 하면 맞아죽기 밖에 더 있겠나.
어찌보면 싸구려 마케팅과 과장광고에 속아넘어가는 수험생들이 죄다, 이건 누굴 탓할 이유가 없는 것이다.
비급이라고 하는 것의 내용을 대조해보았는데 풍산자와 비슷? 쎈과 일품 조합에는 그냥 쨉도 되지도 않았다
\vspace{5mm}

일부러 어렵게 꼬아낸 문제를 낸다 → 그걸로 애들을 공포감에 빠뜨린다 → 이 교재를 보면 해결할 수 있어.
그럼 그 교재로 일본 본고사 문제 같은 것을 풀 수 있을까. 전혀 아니올시다.
역대 수능 문제가 그럼 저런 식으로 어렵게 꼬아냈나. 올해까지도 어떻게 낼지 봐야 알겠지만, 꼬아내는 것과 새롭게 내는 건 다르다.
\vspace{5mm}

그 말많던 2014 기하 문제. 이거 제대로 예측하거나 풀이방법 제시한 강의나 교재가 시험 전에 있었던가.
작년 2015년 30번 문제만 하더라도 고1 수학 $-$ 짝홀 발상에다가 2x2 매트릭스 사고법이었는데 이거 작년에 얘기했었던가.
난 무턱대고 까지는 않는다. 정말 제대로 적중시키고 좋은 방법론을 제기한다면 칭찬할 것이다.
하지만 2년째 그런 사례, 내가 아는 한 단 한번도 없었다. 사실 있다고 한다면 그 강사나 교재는 광고없이도 대박났을 것이다.
\vspace{5mm}

\textbf{M : 책은 매우 좋았으나}
\vspace{5mm}

뒷담화 대상은 아니다. 만약 이 사람이 지금도 활발히 활동했으면 아마 콕콕 이상은 아니었을까 싶다.
책을 쓰는 법도 알고 있고, 무엇보다 돈보다도 수학을 좋아하는 인물이라고 확신할 수 있는 근거들이 많았다.
과거 책을 지금도 걍 출판하면 되는데도, 공포마케팅으로 가도 되는데도, 이 사람은 그리 하지 않았다는 게 신뢰의 근거다.
\vspace{5mm}

일단 A형 수학에 있어서 가장 직관적인 접근법을 제시해주는 책이라고 하겠다. 다만 꼼수는 꼼수다,
엄밀한 정의에 기초하기보다는, '감각'적인 풀이로만 간다는 한계가 분명히 있다. 이 점만 명심한다면 매우 좋은 보충서가 될 것이다.
다만 이 책을 연구해보다가 다시 쎈, 풍산자를 보고 다른 교양서를 공부하고 나서 느낀 건, 이 책의 스킬이 그리 약빨이 좋은 것만은 아니란 것이다.
학생 저자의 한계점이 지적된 경우다.
\vspace{5mm}

하지만 과거 책이 그렇다는 것이지 이 사람의 실력만큼은 대단히 좋다. 간헐적으로 이 사람이 올리는 기출 풀이 같은 것을 보면
요즘 수험가에서 떠도는 떠들썩한 것은 다 알고 있고, 관능적(?)인 언어로 풀어내는 것부터가 다음 책을 기약할만하다고 본다.
사실 이 사람의 신간을 개인적으로는 매우 고대하고 있다.
\vspace{5mm}

이 사람은 돈버는 걸 포기하는 대신 제대로 수학을 배우기 위하여 학업코스로 갔다. 정말로 현명한 선택이다.
그저 인기에 휘말려서 대충 책 내면서 결국 돈벌이의 마수에 빠져서 젊은 시절을 날리는 사람과 비교해보면 그렇다.
나중에 콕콕에서 이 사람의 책을 리뷰할 수 있길 바란다.
\vspace{5mm}

\textbf{N : 명문이공계를 졸업했다고 해도 그 당시 수학과 지금의 수학은 다른데}
\vspace{5mm}

이 책은 단행본이다. 제목은 참 거창하다. 중딩 버전까지 나왔다.
만화 xxx트를 원용한 건 뭔가 손발이 오글거리는데 수학과 관련된 어설픈 썸, 연애 스토리 도입은 뭐란 말인가.
수포자가 어떻게 해서 수학을 공부하게 되는가 하는 식의 접근은 나쁘지는 않은데, 이미 이 분야 선구자는 일본의 \textbf{'수학걸'}이 있다.
게다가 책값이나 두께에 비하면 담긴 내용은 보잘 것 없다.
\vspace{5mm}

도대체 이 저자들은 뭔 생각으로 그런 책을 썼을까...
전형적으로 자기들이 이과의 명문대를 갔으니 자기들이 수학을 잘 한다라는 포부 때문일 거다.
하지만 본인이 수학을 잘 하는 것과, 정말 입시에 도움이 되는 수학 썰을 푸는 건 정말 다른 문제다.
실제로 학원가나 과외 쪽에서 가장 잘 못 가르치는 사람들이 수학과라는 이야기가 괜히 나오는 게 아니다.
가르친다는 건 소통이지, 일방적으로 푸는 게 아니다.
그런데 이 책이 그런 경우다. 저자들은 자기들이 똑똑하니까 자기들이 말하는 게 전부인 줄 알고 있는 것이다.
\vspace{5mm}

얄궃지만 같은 제목의 다른 책은 매우 좋다. 다만 그 다른 책의 저자는 스펙만 보면 정말 지잡대란 소리를 들을 정도다
그리고 그 사람의 정치성향도 너무 노골적.
하지만 수학적 사고에 있어서는 그 다른 책의 저자에서 훨씬 더 많은 걸 배울 수 있었다.
\vspace{5mm}

우물 안 두꺼비가 되지 않기 위해서는 끈임없이 배울 수 밖에 없다.
왜 사람들이 학교를 졸업하면 머리가 돌이 될까.... 라는 것에 대해서 생각해보고 상상해보고 직접 겪으며 탐구.
교훈은 그거더라
학교에서는 매일이 사실 새롭다. 교육도 교육이지만 학년도 올라가고 시험도 치르고 진도를 강제로 나간다.
변화를 강제당하니 뇌가 자극을 받고 그러므로 머리를 쓸 수 밖에 없다.
그러나 학교를 졸업하면 대체로 반복적인 일상과 업무에 갇힌다.
새로운 자극이 없으니 그런 변화없는 삶에 순응해버리면서 머리는 점차 둔해진다.
\vspace{5mm}

그나저나 이 단계까지 가니까 결국 모른다... 라고 다들 그러시는데 그래야 당연하지만
생각보다 교재풀이라는 게 좁다.
아마 학생들은 누군가 마케팅한 교재가 정말 다인 줄 알고 그렇게 믿고 공부하겠지.
\vspace{5mm}

진정한 비급이라면 다음 주 목요일에 나올 문제가 들어있어야하지 않나.
\vspace{5mm}







\section{실패의 원인은 계량 실패}
\href{https://www.kockoc.com/Apoc/472231}{2015.11.09}

\vspace{5mm}

왜 성공하는 사람들은 자주 성공하고 실패하는 사람들이 자주 실패하는가... 는 한번쯤 고민할 문제가 있어요.
운이라는 게 있다고 봅니다만, 애당초 통제가능하지 않은 건 아예 고려하지 않는 게 나음
\vspace{5mm}

실패하는 사람들의 실패습관이야 여러가지 많습니다만.
가장 큰 문제는 밸런스가 깨져있단 것입니다.
\vspace{5mm}

A라는 프로젝트에 100이 필요, 보상은 1000, B라는 프로젝트에 50이 필요, 보상은 250, C라는 프로젝트에 300이 필요, 보상은 10,0000
그리고 자원이 1000이 주어져있다면 어떻게 하겠습니까.
\vspace{5mm}

여러가지 해답이야 나오겠죠.
그런데 오답은 지적할 수 있겠군요. 저기서 \textbf{2개 이상 선택하면 무조건 실패합니다}.
아니 1000이 있으면 100+50+300을 훨씬 초과하니까 상관없지 않냐 하는 질문이 따르겠죠
이런 질문을 하니까 망한단 것이죠.
\vspace{5mm}

현실은 덧셈이 아니기 때문입니다.
\vspace{5mm}

첫째로 프로젝트는 반드시 시행착오라는 걸 하게 됩니다.
명목상 A가 먹는 자원은 100입니다. 하지만 이건 '성공'했을 때의 기준이죠.
하지만 사람들은 평균적으로 3$\sim$5번은 실패합니다. 그래서 500 이상을 낭비해버리고, 이게 정상인 것입니다.
현실의 모든 작업은 반드시 실패하는 횟수를 전제해야하는 것인데 똑똑하고 욕심많은 사람일수록 이걸 고려하지 않죠.
실제로 이런데 A, B,C까지 다 했다면? 1000으로는 도저히 감당이 안 됩니다. 게다가 집중도 못 하고 실패는 더 늘어나니 자포자기하게 되죠
\vspace{5mm}

둘째로 자원소비는 산술평균적으로 이뤄지진 않습니다. 기하급수적으로 늘어난다는 것입니다.
수험 이야기를 한다면 2등급에서 1등급 올릴 때 10시간이 필요하면, 1등급에서 만점권까지 가려면 10^2 시간이 필요합니다.
C는 명목상 300입니다만, 실제로 진행하다보면 300이 더 늘어나게 될 때 이게 1차 함수가 아니라 지수함수 꼴로 늘어납니다.
\vspace{5mm}

셋째로 새로운 프로젝트가 계속 생긴단 겁니다.
A를 했다가 실패했다고 칩시다. 그럼 실패한 A를 정리하는 프로젝트 A1, 다시 재도전하는 A2, 그리고 실패 원인을 뿌리뽑는 D 프로젝트가 생겨납니다.
다시 말해 프로젝트 하나는 절대 하나가 아닙니다. 이것도 진행하다보면 꼬리치고 새끼치면서 일감이 더 많이 늘어나게 됩니다.
이걸 만회하는 건 '학습효과'겠죠.
\vspace{5mm}

n수생들의 실패 원인은 다른 게 아니라, 목표 계산을 잘 못 한 것이 가장 큽니다.
멀리서 보면 에베레스트 산도 손바닥 안에 들어가죠. 그러나 실제로 등반을 하면 ㅎㄷㄷ
예쁜 여자도 사귈 때는 행복한 것 같죠. 그러나 그 여자가 흔한 동네 미용실의 욕지거리 아줌마로 진화하는 건 삽시간입니다(그러니 2D로)
문제집은 한권이죠. 그러나 대략 500문제가 있고, 여기서 오답이 150문제이며 모르는 문제가 30개면 프로젝트는 180개로 증가합니다.
\vspace{5mm}

자, 그렇다면 해답은 간단하죠.
\textbf{가장 자원이 적게 드는 것을 빨리 끝내는 수 밖에 없습니다.}
보상은 낮지만 그건 확실히 내 자본이 되기 때문이죠.
\vspace{5mm}

한데 수험생들은 올해 수능을 치고 또 1년이란 기간이 주어진다고 착각들 하겠죠(자기 청춘을 깎아만든 시간인데)
아마 올해 학습하신 분이 있어 몇몇은 정신들 차리겠지만, 그래도 또 허송세월하며 3월까지 날리는 사람 분명 있습니다.
손바닥 안에 들어가는 에베레스트산이니 꼭대기까지 금방 오른다고 착각한 게 문제건만
그런 건 고려 안 하고 공부는 자기 적성에 안 맞는다는 둥 머리가 나빠서 그런다는 둥 원인을 엉뚱한 데 찾죠.
\vspace{5mm}

열심히 공부하신 분들 콕콕에 계시죠. 그리고 정상적인 과정들 보입니다.
제가 봐도 미친 듯이 공부했는데, 지금도 덜 공부한 것 같다고 그럽니다. 예, 이게 정상입니다.
이 분들은 에베레스트산 꼭대기에서 산소부족으로 허덕이는 것이거든요.
반면 얼굴빛 좋으면서 아, 이번 시험 어쩔까하는 사람들 있습니다. 산 언저리에서 쉐르파와 노가리까며 기도하고 있어요.
마음은 가장 편할 겁니다. 이 사람들은 내년에 또 치고 내후년도 갈테니까요. 그 빚은 이자쳐서 갚게 되어있죠.
\vspace{5mm}

필승해법은
\textbf{'적의 수를 줄이고 다수로 포위해서 섬멸한다'}
즉, 각개격파죠.
일대일로 싸우면 멋져보이지만 죽을 위험도 높죠
그러나 비겁해(?) 보일지라도 30대 1로 승부하면? 평화적으로 마무리지을 수도 있죠.
그런데 실패하는 사람들은 일대일 승부, 혹은 일대백 승부를 선호합니다. 당연히 죽죠.
\vspace{5mm}

공부든 뭐든 다 마찬가지입니다.
\vspace{5mm}

일찍일찍 합격해 가는 사람들은 \textbf{꿈은 크지만 욕심은 적은} 사람입니다.
그 사람들은 높이 올라가려 합니다. 하지만 성급히 올라가지 않죠. 차분히, 늦게, 소극적으로 계단까지 만들고 올라갑니다.
처음에는 시간이 걸립니다, 그러나 안전하게 올라가므로 추락하는 일은 없죠.
그 반대로 \textbf{욕심이 큰 사람들}
내, 이 사람들은 바로 실패란 말을 이마에 써두고 있습니다.
마린 2명을 히드라 30마리에 보내는 미친 짓을 하고 있죠.
그래서 이들에게 수험은 '낭만'이 됩니다. 왜냐고요? 실패를 포장이라도 해야하니까요.
\vspace{5mm}

상담쪽지 보내는 분들에게 공통적으로 하고싶은 이야기가 이겁니다.
크게 욕심내지 말고 \textbf{"작은 것"부터 확실히 끝장내라}.
너무 당연한 교과서적인 이야기지만 실천하기는 가장 힘듭니다.
실패하는 사람은 사자 한무리를 사냥하다 망하지만, 성공하는 사람은 사자 12마리를 한마리씩 공략해서 성공합니다.
\vspace{5mm}







\section{기싸움}
\href{https://www.kockoc.com/Apoc/470657}{2015.11.09}

\vspace{5mm}

"우리가 사는 곳은 현실인가 가상인가"
\vspace{5mm}

여러가지 썰이 있고 이와 관련해서도 수만장의 논문이 나오겠지만
자연과학적인 입장에서는
우리가 사는 세계는 엄밀히 따지면 현실 그대로가 아니죠.
\vspace{5mm}

우리의 뇌를 통해 인지하고 재구성한 \textbf{가상현실}이죠.
\vspace{5mm}

수능시험을 앞두고 왜 멘붕하게 되느냐.
싸우고 싶지 않아서 핑계를 대고 싶기 때문입니다.
\vspace{5mm}

만약 수험생이 군주, 장수라면 싸우고 싶어서 웅장한 bgm이 깔리면서 의욕이 올라가겠죠.
그러나 유감스럽지만 우리들은 대부분 양민, 노비들의 자손입니다(족보 그거 구라인 것 다 알죠? 진짜 양반 별로 없어요)
얻어맞고 학대당하던 그런 하층민스러운 기질이라는게 유전되어서인가 힘든 상황이 오면 포기할 준비부터 하죠.
왜 핑계를 대느냐, 그래야 \textbf{덜 혼나기 때문}입니다.
\vspace{5mm}

지금부터 세팅해야하는 건 '냉정하게 킬러문제를 스나이핑'하는 전투 게임으로 가야하는 것입니다.
\vspace{5mm}

\item 1. 공간의 장악
\vspace{5mm}

수요일에 예비소집한 다음 장소 통보 받겠죠. 아주 멀지 않다면 시험장소에 다녀오시길 바랍니다.
가서 내가 내일은 뼛속까지 다 짜내서 높이 올라가겠어라고 선포하고 오면 됩니다.
간혹 곳곳에 부적을 붙이거나 콩팥을 뿌리거나 심지어 몰래 노상방뇨까지 하는 케이스도 있던데 뭐 그건 제가 권할 바는 아니지만
이런 의식을 치르는 것부터가 호랑이가 영역권 표시하는 것과 비슷한 것이니 효과가 없다고 볼 수는 없겠죠.
시험 당일에는 무조건 일찍 가신 다음 시험치르는 교실의 복도를 다 걷고 화장실에서 용변보고 거울보고 썩소 짓고 하시길요.
주변 공간을 왜 스카우팅하느냐. 그래야 불안감이 사라집니다. 우리는 낯선 공간에 있으면 경계심을 발휘하므로 집중력이 떨어집니다.
자기가 정확히 어떤 공간에 앉아있으며 주변에 어떤 지형지물이 있느냐를 알고 나면 더 이상 그 주변을 의식하지 않게 됩니다.
\vspace{5mm}

\item 2.  시간의 장악
\vspace{5mm}

시험시각에 맞춰 해당 과목을 푸는 리허설도 하고 계실 것입니다.
오늘도 6시에는 일어나셨을 거라 믿겠습니다만 화, 수요일도 새벽 5시 30분에 일어나는 세팅을 해두시길 바랍니다.
그리고 시험시각에 맞춰 반드시 해당과목을 공부하거나, 설령 수업 때문에 못 한다고 하더라도 "지금은 국어시간이야"
"지금은 수학 29번으로 내가 n+1을 하면서 가슴앓이할 시각이지"라고 중얼거리는 것도 나쁘진 않습니다.
\vspace{5mm}

시험 전날은 10시에 무조건 주무세요. 화, 수요일에 일찍 일어났다면 잠이 안 올 리가 없습니다.
무조건 자야합니다. 그렇지 않고 11시 넘어간다... 새벽 1시, 2시까지도 못 잘 수도 있습니다.
시험날 발휘해야 할 정신력과 체력을 이 때 낭비하는 불상사가 벌어질지도 모릅니다.
이런 경우라면 차라리 더 일찍 자서 당일 새벽 4시, 심지어 3시에 일어나는 게 나을 수도 있습니다.
일어나서 문풀하고 시험 리허설 치고 수능장에 가는 게 나을 수도 있어요. 올빼미들은 일어난 다음 한참이 지나야 컨디션이 좋아지니까요.
수능시험이 끝난 뒤에 컥챗 와서 나 인생 어떡해 그러지 말고, 시험 종료가 되면 탈진 상태에 빠질 수 있도록 세팅해놓으시길 바랍니다.
\vspace{5mm}

\item 3. 음악의 장악
\vspace{5mm}

벌써 암욜맨 링딩동이 떠돌아다니고 있던데. 그보다도 아침마당 bgm이 더 위력적이죠.
\vspace{5mm}

정 안 된다 싶으면 평소에 듣던 음악을 틀어놓으시길 바랍니다.
그리고 수요일까지는 음악 중에서도 우울하고 슬픈 걸로 가시면 됩니다.
활기차고 긍정적인 음악을 들으면 오히려 뇌에서는 이런 데 쓸데없이 호메오 스타시스를 발현해 우울해지려고 하는 경향이 있지만,
슬프고 우울한(한편으로 차분한) 음악을 들으면 뇌에서는 긍정적으로 변하려고 하는 경향이 있습니다.
음악이 대리 비관을 해주기 때문에 역설적으로 우울한 게 사라지는 것이죠.
\vspace{5mm}

단, 노래가 들어간 음악은 절대 안 됩니다.
게임 bgm이나 military music 같은 것이 좋습니다.
\vspace{5mm}

물론 가장 좋은 건 음악 자체를 안 듣는 것입니다. 하지만 그게 힘들면 그나마 나은 걸로 덮어씌우란 것입니다.
\vspace{5mm}

\item 4. 수험사이트
\vspace{5mm}

올해 망했다고 다른 친구들 발목잡는 사례들 벌써 보입니다.
'개스랙'이라고 욕한사발 하고 접속하지 마시길 바랍니다.
\vspace{5mm}

\item 5. 잠재력의 발휘
\vspace{5mm}

내 무의식이 문제를 풀 수 있다라는 신뢰를 보여주길 바랍니다.
실제로 문풀에 있어서 더 중요한 건 무의식입니다. 문제를 푼다는 건 엄밀히 말해 뇌에서 알아서 풀도록 우리가 '잘 읽는' 것입니다.
문제해석을 정확히 한다면, 그리고 문풀절차만 지킨다면 문제는 저절로 풀립니다.
공부를 했음에도 문제를 못 푸는 건, 당사자의 뇌가 스스로 문제를 풀 수 있는 단계까지 공부가 되지 않아서이기도 하지만
무엇보다도 본인이 문제를 잘 해독하지 않아서 그렇고, 그 다음으로는 절차를 안 지켜서 그렇습니다.
해석과 절차 이게 안 되면 10년간 공부해도 수능은 절대 안 됩니다.
\vspace{5mm}

우리의 의식이 이렇게 100$\%$ 안전성을 보증하면 무의식은 편하게 발동함으로써 문풀 아이디어를 쏟아냅니다.
거꾸로 말해서 우리가 부정확하게 문제를 읽거나 절차를 안 지키면, 무의식은 발동하긴 커녕 사려버립니다. 그래서 문제가 안 풀리는 겁니다.
\vspace{5mm}

어차피 수능시험은 10문제 중 9문제는 님들이 아는 문제, 1문제가 모르는 문제입니다.
9문제 중 3문제는 그냥 푸는 문제, 3문제는 실수해서 망할 수 있는 문제, 나머지 3문제는 살짝 꼬아낸 문제입니다.
현재 출제 경향으로는 누구든 저 1문제에서 당혹스러워하긴 마찬가지입니다. 여기선 누구나 출발선이 똑같습니다.
아울러 실수할 수 있는 3문제, 살짝 꼬아낸 3문제는 시험당일날 맑은 정신으로 해석만 잘 하면 안전하게 대처할 수 있습니다.
\vspace{5mm}

이제는 참고서 볼 때가 아니라 위 1$\sim$5를 정확히 준수하실 때이니 잘 지켜주시기들 바랍니다.
믿건마말거나인데 구석기 시절 제가 수능을 칠 때, 전 그 추운날 시험장에서 반팔로 조깅을 했고
수돗가에서 일부러 머리를 감고 세수를 박박한 뒤 시험장에 들어갔습니다. 그렇게 해서 낯선 곳과의 기싸움은 선방했죠.
\vspace{5mm}

물론 저러란 이야기는 아닙니다만(....) 또 생각나서 적는다면
옷 두껍게 입지 말고 반팔+얇은 상의+얇은 외투. 이런 식으로 여러겹 입고 가시고 양말도 두겹 신고가시길요.
난방에 따라서 추워질 수도 있고 더울 수도 있으니 이거 가감해야할 것입니다.
그런데 가장 골치아픈 게 발이 추운 건데. 발은 따뜻할 수록 좋으니 양말 2겹 신고가는 건 매우 권장할만한 일이라고 하겠습니다.
머리는 차갑게, 발은 뜨겁게.
\vspace{5mm}





\section{교재 이야기 : 말이 필요없는 증명 썰}
\href{https://www.kockoc.com/Apoc/472719}{2015.11.10}

\vspace{5mm}






\section{교재 이야기 : 말이 필요없는 증명 썰}
\href{https://www.kockoc.com/Apoc/472719}{2015.11.10}

\vspace{5mm}






\section{교재 뒷담화 : O P Q}
\href{https://www.kockoc.com/Apoc/472974}{2015.11.10}

\vspace{5mm}

우리나라에서 수학 대중서들도 역시 마케팅의 힘으로 팔린다.
\vspace{5mm}

김용운 교수님의 "재미있는 수학여행"이 표준일 것 $-$ 이 시리즈는 좋다 : 단 김용운 교수님이 과거에 썼던게 더 좋다.
왜 이런 건 복간을 안 하는 걸까.
\vspace{5mm}

그런데 그 이후로 나오는 교양서들은 저 재미있는 수학여행을 못 벗어나고 있고
사실 하나마나한 이야기를 하는 경우가 많다.
\vspace{5mm}

그런데도 왜 잘 팔리는 걸까.
\vspace{5mm}

학부모들의 무지이다.
\vspace{5mm}

자기 자식들이 수학을 잘 하길 바라는 어머니들은 수험을 잘 모른다
그래서 수학에 도움되는 책이라고 알려지면 무조건 구입을 하는 것이다.
이 시장이 생각 외로 크기 때문에, "수학을 잘 할 수 있다"라는 표제만 달고 그럴싸하게 내용만 채우면 팔린다.
\vspace{5mm}

\textbf{O : 범죄뉴스 인용을 해도 관계없습니다.}
\vspace{5mm}

최근에 관악 성추행으로 검색해보시면 된다.
빙산의 일각.
걍 썰 나온 김에 조심스레 적어볼까?
이제야 정의가 구현되는구나... 가 아니라 '\textbf{에게}해'를 조망하는 기분이다.
대다수 의식있는 학생들이 왜 대학원 진학을 기피하는가, 그리고 전공보다도 교수님들의 인성과 성격을 따지나 그 이유란 게 있다.
꼭 말은 안 하겠지만 장년교수 $-$ 20대 여대생의  \textbf{I.얼레리꼴레리.YOU}는 쉬쉬해서 그렇지 원래부터 있었다.
\vspace{5mm}

더 듣고 싶겠지만 여기서 끊고(댓글로도 묻지 마슈. 그리고 난 기사만 찾아보라고 했수다)
일단 책만 보자면 내용이 역시 에게해 2탄이다. 그리고 저 사건이 터진 후 헌책방에 10권 이상 늘어선 그랜드 캐년까지 목격했다.
출판보다는 본인의 이미지 메이킹을 위한 책일수록 책 표지에 저자 사진을 크게 싣는다(그것도 미소짓은 걸로)
뭐 그건 모르겠고 내용만 좋으면 되잖아 하는데 입시수학에 전혀 도움이 되지도 않고,
그렇다고 다른 것이 쓸모가 있느냐하면 그건 아니다.
\vspace{5mm}

썰 나온 김에 적으면 대학교부터 아름다워질 것이다... 그런 건 없다.
고교까지 사교육은 그냥 돈만 챙기지 그래도 웬만한 경우는 지킬 건 다 지킨다. 퇴출이 그만큼 빠른 시장이어서이다.
것보다도 한낱 과외교사조차도 공부해야하기 때문에 딴 걸 신경쓸 여유는 별로 없다(막장사례가 없지 않겠지만)
그러나 대학 진학 이후는 캠퍼스의 가면을 쓴 사회 현실이다. 이게 좋은 게 아니다.
대학생은 보호받지 못 한다. '개인 책임'으로 산다. 하지만 대학이기 때문에 은연 중의 갑을 관계라는 건 존재한다.
\vspace{5mm}

상상도 못 하는 범죄가 일어나더라도 쉬쉬하는 경우는 많다. 그건 명문대일수록 심하다.
그 사건이 노출되면 피해자도 많은 걸 잃는다. 성범죄를 저지르는 남자들은 그것을 믿고 덫을 판다.
\vspace{5mm}

이전에 적었지만 '가르치는 입장'에 있는 사람이 교주놀이 하다가 더 막 가는 거, 이거 정말로 심하다.
사회에서 까이는 게 크리스천이던가. 한데 개인적 경험으로는 크리스천들이 차라리 이 점에서는 나았다.
물론 아닌 경우도 있긴 하지만, 신앙심으로 살기 때문에 굽힐 줄 알고, 굽힐 줄 아니까 그래도 최소한의 도덕이란 게 있는 거다.
반면 신 좆까 하면서 자연과학과 합리주의로 모든 게 설명된다.... 이런 사람들이 자기도 모르는 사이에 교주가 되고 막 간다.
\vspace{5mm}

P : \textbf{다작}
\vspace{5mm}

뒷담화 대상은 아닐지도 수학교양서 어디든지 올라와 계시더라.
그런데 내용은 그냥. 한번 읽어볼 수준인데 신기한 게 정말 어디든 이름이 올라와 있다.
좋게 말하면 열정적인데.... 중요한 건 이 분의 대표작이라는 게 있냐 보면 사실 기대한 수준만큼은 아니라고 보았는데.
\vspace{5mm}

지금 생각해보면 이렇게까지 다작하는 사람도 없고 정열적이라는 점에서는 높이 쳐줘야하지 않을까.
그게 실제 현재 입시수학에는 별로 도움이 되지 않는다 하지만, 수학의 대중화에서는 확실히 기여하는 게 있기 때문이다.
다만 아쉬운 건 그 본인만의 개성이라는 게 부족하다가 되겠음.
\vspace{5mm}

\textbf{Q : 사실 쓸모가 없는 직관수학}
\vspace{5mm}

한 때 휩쓸었던 책이다. 그리고 추천사나 스펙 등을 보면 으리으리하다.
그 직업인 분이 수학에 관한 책을 쓴 경우는 거의 없을 걸? 그런데 이 분, 원래 가르치는 일을 했었다.
P와 달리 이 분은 개성이 있다. 철저히 직관수학을 강조하고 있어서이다.
최근에 냈었던 그 책보다 훨씬 10년도 넘은 옛날에 '직관수학'에 관한 책을 낸 적이 있고 당시는 혁명적인지라 인상깊게 읽었다.
그래서 직관수학이라는 걸 나도 추종한 적이 있는데.
\vspace{5mm}

직관수학의 단점 : 그래서 뭐 어쩌라는 거야. 머리만 좋으면 된다고? 멍하니 보면 된다고?
\vspace{5mm}

만약 직관수학으로 가능하다면 뭐하러 사람들이 식을 만들고 그래프를 그리고 그랬겠냐.
그런 걸로 가능하다면 수학이란 학문이 필요없었을지도 모른다.
수학적인 사고가 되지 않기 때문에 이걸 보완하려고 만든 학문이 수학이다, 속칭 말해서 '머리 나쁜 사람들을' 위한 것이 수학이 아닌가.
\vspace{5mm}

밥 로스의 참 쉽죠$\sim$ 급.
물론 나쁜 책은 절대 아니다. 하지만 이걸 고교생들이 보았다하면 써먹기는 어려울 것이다.
이 책은 저자 분이 더 상세히 분설하고 해설하면서 한 여러권의 시리즈를 만들어서 난이도를 낮추고 정식교재화하면 대박났을 것이다.
한 때 잘 팔렸으니까 안 본 사람은 없겠지만, 끝까지 읽은 애들이 몇이나 있을 것이며, 읽었다고 해도 그걸로 도움이 되었을까.
\vspace{5mm}

사실 위 중 어느 것도 고교생에게는 도움이 되진 않는다고 본다.
굳이 수학적 사고에 대해서 가장 쉽게 접하고 싶다면 아래 책들을 읽길 바란다.
\vspace{5mm}

현직 교사가 고교생의 눈높이를 잘 헤아린 책이다.
입시에는 당장 상관관계가 없어보일 수도 있다. 그러나 고교수학의 접근법에 대한 '수학 철학'으로서 이만한 입문서는 없고
사실 그 이상도 필요하지 않다(그 이상 필요하다면 문제접근방법이겠지만 이건 차후에 논할 듯)
어째서 $\sim$ 한 풀이가 나오느냐, 그리고 $\sim$한 접근이 어떤 의미를 지니느냐 하기 위한 철학적인 교양을 쉽게 설명해놓았다고 할 수 있다.
\vspace{5mm}

무엇보다 중요한 건 쉽다. 그리고 요점은 웬만큼 다 담아놓았다.
\vspace{5mm}

그렇다고 저기서 스킬이나 무슨 수험 꼼수 기대하진 말길.
하지만 제대로 읽고나면 '수학적으로 사고한다'의 올바른 길을 걷는다는 건 보장한다.
\vspace{5mm}







\section{교재 이야기 : 교과서썰 1}
\href{https://www.kockoc.com/Apoc/476332}{2015.11.11}

\vspace{5mm}

교과서를 강조하지만 실제로는 교과서스럽지(?) 않은 사례들이 많다.
\vspace{5mm}

사실 이건 정말 하고 싶은 이야기다.
몇몇 저자나 강사들은 교과서가 중요하다, 교과서스럽게 공부해야 한다라고 한다.
이거 그럴싸한 말이긴 한데.
\vspace{5mm}

문제는 그 사람들의 강의든 교재든 그래서 교과서스럽냐... 하면 그건 아니었단 말이다.
교과서를 예로 들려면 출판사, 저자 라인, 페이지 등을 명기해야 하지 않나?
수능이 아닌 공무원이나 고시 참고서의 경우는 '참고문헌' 인용을 한다. 그거 집필자가 교과서를 읽고 짜깁기햇다는 근거다.
\vspace{5mm}

그런데 내가 보았던 어떤 수학교재들은 교과서가 중요하다고 말을 하는데
$-$ 교과서 출처를 표시한 것도 아니고
$-$ 그렇다고 교과서 내용 인용하지도 않았으며(그게 어려운 것도 아니고)
$-$ 내용은 내가 아는 한 전혀 교과서스럽지 않다는 것이다.
\vspace{5mm}

그냥 까고 말하지. 저거, \textbf{교과서를 안 보았단 이야기 아닌가.}
그리고 교과서를 강조한다면 나처럼 "두산, 미래엔, 성지 교과서 같은 것 보세요. 아, 물론 이건 7차 교육과정입니다"라고 하면사 컥챗이나 하면 그만이지
강의나 책을 팔아먹을 이유가 없지 않나.
\vspace{5mm}

이런 일이 왜 벌어지는 걸까  하기 전에
"도대체 교과서스럽게 공부한다는 게 무얼 이야기하는 걸까"라는 질문부터 답이 있어야하지 않을까 싶은데
\vspace{5mm}

이 경우도 냉정히 이걸 추리해야 한다.
\vspace{5mm}

\item \textbf{1. 교과서에 수록되어 있는 건가}
\item \textbf{2. 교과서 정신(?)을 따라가는 것인가.}
\vspace{5mm}

즉, 형식적 구분인가, 아니면 의도적 구분인가 이것도 명기해줘야지.
하지만 겉으로 교과서 강조하면서 실제로는 교과서도 잘 인용 안 하는 책들에서 이런 섬세한(?) 구분을 기대할 리는 만무하고
그냥 내가 정리해보면 당연히 2번이다.
교과서에 수록되어있느냐 안 되어있느냐 그게 중요한 게 아니라, 교육 당국아 강조하는 '실질적인 교과서 수준의 발상'이 중요하단 거지.
\vspace{5mm}

만약 수록이 문제라면 전체 출판사 다 검증해보아야하잖아.
\vspace{5mm}

하도 수험가에서 저런 식으로 앞뒤가 안 맞는 경우가 있어서 거금 들여서 교과서 익힘책 7차 과정 $-$ 무슨 고려출판사나 박영사까지도 다 구매해보았다
그리고 작년부터 올해까지 쭉 읽어보고나서 1번을 폐기할 근거를 찾았다.
\vspace{5mm}

수학익힘책 기준으로 친다면(설마 익힘책도 교과서가 아니라고 하면 할 말이 없다)
\textbf{로피탈도, 케일리 해밀턴도, 그리고 적분에 별 이상한 정리 같은 것들까지도 다 교과서 내용으로 보아야한다는} 결론이 나온다.
출판사들마다 저자 라인이 다르고, 저자 라인들도 조금씩 강조하는 게 차이가 있기 때문이다.
가령 적분과 통계에서 중앙교육에서 낸 책은 몇몇 내용은 시중교재도 초월하는 수준이었다는 것.
\vspace{5mm}

이론 뿐만 아니다. 단원 마무리할 때 나오는 퀴즈나 일화 같은 것들도 내용을 보면 준수리논술급도 있고
소위 교과서스럽다라고 알려진 걸 능가하는 사례들이 있다.
\vspace{5mm}

이렇게 실증해보면서 한가지는 알았다. 아, 교과서 강조해댄 사람들, 정작 교과서 보지도 않았구나(...)
\vspace{5mm}

그렇다면 여기서 발상의 전환이 바뀐다.
교과서는 최소한의 내용을 담고있다 (X)
모든 교과서 익힘책들을 다 들여다보면 시중교재를 능가하는 케이스도 있다 (O)
\vspace{5mm}

그렇다면 수록은 이제 무의미하다. 교과서에 실렸으니 그게 최고다라는 건 이제 폐기해도 되는 이야기인 것이다.
참고로 덧붙이자면 교사용 지도서 같은 것만 하더라도 대학교 수학의 내용까지 수록한 걸 본다면
가령 미래엔 지도서같은 경우만 하더라도 안드로메다는 저리 갔다는 걸 보면 형식설은 버려도 되는 것이다.
다시 한마디 : 교과서 강조해대는 사람들 그럼 교과서 제대로 본 것 맞냐? 아니, 뭐 보기라도 했다면 제대로 짜깁기 인용이라도 했겟지.
\vspace{5mm}

그럼  그럼 교과서적 정신은 무엇?
\vspace{5mm}

교과서에서 다들 안 읽고 넘어가는 : 단원의 취지, 학습목표, 공부하는 방향 같은 guide나 advice를 얘기하는 것이다.
가령 성지의 경우 단원 구성은
(제목) $-$ \textbf{(학습 목표) $-$ (생각 열기) $-$ (탐구활동)} $-$ (이론) $-$ (문제) $-$ \textbf{(생활 속에서 만나는 수학)}
이렇게 되어있다.
그런데 여기서 학생들이 공부하는 건 제목, 이론, 문제 $-$ 이걸 참고서로 강화해 푼다.
하지만 \textbf{학습목표, 생각열기, 탐구활동, 생활 수학} 같은 건 참고서에서 거의 언급하지 않거나 대충 넘어가지 않나.
그런데 이거야말로 사실 교과서의 정신이 아니겠나.
\vspace{5mm}

"저게 수능에 도움이 됩니까"
"수능이 무얼 측정하죠"
"사고력이죠"
"이론만 암기하고 문제만 푼다고 사고력이 늡니까. 자기가 공부하는 학습 목표를 알고 공부한 것이 현실에서 어떻게 응용되는지 알아야죠"
\vspace{5mm}

교과서에 들어간 이론은 사실 그리 새로운 건 없다.
그러나 저런 \textbf{'목표', '생각 열기', '탐구활동', '생활수학', '쉬어가는 이야기'} 등은 수험과 관계없어보이지만 실제로는 본질적인 것이다.
사실 저 가이드라인들이야말로 교과서 정신의 정수가 아니겠나.
\vspace{5mm}

그리고 현실에서 교과서스럽게 공부하라는 건 사실 '여집합'의 의미가 있다.
이건 즉, 사교육에서 강조해대는 어떤 꼼수라거나 스킬을 쓰지 말고, \textbf{순수한 정의나 이론만 가지고 문제를 풀라 그 이야기다}.
다시 말해서 2차 곡선 문제가 나오면 무조건 공식부터 쓰지 말고,
2차 곡선 $-$ 포물선, 쌍곡선, 타원의 정의를 떠올리고 어떤 성질이 있었나 연상한 다음
주어진 문제들의 조건을 저 2차 곡선의 성질, 정의 순으로 대응시켜보면서 실마리를 파악하라는 것이지
A 선생이 B한 문제는 C로 풀라고 했지 헤헷 ... 어엉 ... 안 풀리잖아... 라는 짓은 하지 말라는 이야기다.
\vspace{5mm}

그런데 앞에서 언급한 강의나 책들은 입으로는 교과서를 강조하는데 내용은 전혀 교과서스럽지 않다.
더 놀라운. 아니 놀랄 것도 없는 사실은 이런 모순을 수험생들이 지적하는 것을 못 보았다는 것이다.
하기야 붕어빵에 진짜 붕어가 안 들어갔냐 하는 게 더 어리석은 짓일지도 모른다.. 붕어싸만코야 설탕과 쵸콜릿맛으로 먹는 것이지 뭘.
그러나 사소한(?) 모순을 저지르거나 그런 걸 지적하지 못 하면서 수학을 공부한다는 건 어불성설이 아닌가.
\vspace{5mm}

이런 것도 지적 못 하면서 무슨 수학문제 하나 어려운 것 풀었다고 좋아하나.
\vspace{5mm}

그와 별개로 개인적으로는 교과서를 다 구입하면서 피눈물(?)을 흘렸지만 지금은 컬렉터로서의 실속없는 자부심(?)이라는 걸 갖게 되었는데
교과서에 실린 이론이나 문제보다는, 앞에서 말한 교과서 정신에 해당하는 내용들이야말로 저자들의 정성이 들어간 진국임을 맛보게 되어서이다.
특히 나 같은 독서광으로서는 강의보다야 그런 쓰잘데기없는(?) 내용들에서 더 얻는 것들이 많기도 하지만
고교수학이 일상에서 어떤 식으로 구현되는지를 보는 실마리로서 얻은 게 많아서이다.
\vspace{5mm}

다 늙은 머리로 킬러문제를 풀 때 도움되는 것은 사실 저런 것이지, 연구해본다고 들었던 사설인강 그런 건 아닌 것 같다.
인강은 지금 느끼지만 들을 때는 그럴싸했지만 실제 지금 생각해보면 가장 얄팍했다(이채형 강의만은 예외)
오히려 깊은 사고에 도움이 되는 건 저런 교과서 정신, 일본 책이었다.
\vspace{5mm}

시간나는대로 글을 쓰겠지만 $-$ 몇몇은 이런 것도 부풀려서 교재 내서 돈번다고 할지 모르지만 내가 보기엔 참 정신나간 사람들이고 $-$
수학은 어떤 특정한 스킬이나 패턴을 암기해야만 하는 과목이 아니다.
특정 스킬이나 패턴을 암기하는 과목일수록 오히려 웃돈을 주더라도 강의를 들을 필요가 있다. 시간과 노력을 단축시켜주기 때문이다.
\vspace{5mm}

하지만 수학은 '요리'와 비슷한 과목이다.
\vspace{5mm}

무턱대고 설탕을 부어대고 조미료 뿌리거나, 인스턴트 요리 가져와서 뜨거운 물 붓는 걸 제대로 된 요리라고 하나?
물론 급하면 그렇게 먹을 수도 있다. 그리고 그게 다수 수험생들의 현실이다.
하지만 수리논술이든 4점짜리 문제든 그건 "자, 여기 신선한 광어 한마리가 있으니 손님을 만족시켜봐"라는 수준으로 내는 것이다.
즉, 단서 몇개만 줘놓고 그걸로 문제의 목적을 달성하라고 하는 '해결과정'을 묻는 것이다.
그럼 여기서 스킬, 꼼수가 먹히나? 뜨거운 물만 부으면 조리되는 인스턴트 식품만 먹은 애들이 저걸 다룰 수 있겠어?
\vspace{5mm}

그런데 그게 공교육과정에 없는 게 아니엇단 거지. 바로 '날재료부터 요리하는 것'을 교과서에서는 분명히 제시해주었으니까.
\vspace{5mm}






\section{검증되었네요.}
\href{https://www.kockoc.com/Apoc/479357}{2015.11.12}

\vspace{5mm}

어제와 오늘 새벽에 계속 대화하고 상담하고 답변하면서
무엇이 옳은 공부법인가에 대해서는 어느 정도 검증이 완료된 것 같습니다.
자뻑으로 들릴 수도 있는데 제가 예측한 것이나 충고한 건 거의 맞아떨어졌습니다(그래서 말을 삼가야겟습니다. 이 때가 위험하니까)
\vspace{5mm}

1년 정도 지나면서 그 공부법을 실천해서 성적을 올리거나
목표를 이루지 못 했더라도 전진한 케이스들이 있으니 이제 이 분들이 '영리목적과 관계없는 순수한 학습공동체'를 일궈나가시면 되겠지요.
\vspace{5mm}

몇몇 분들이 돈과 관계없이 왜 이런 걸 상담해주느냐에 대한 답변은
\textbf{루저가 패자부활전 할 수 있는 올바른 방법을 확인하고 정착시키는 것}이 제 인생 목표 중 하나이기 때문입니다.
그리고 이런 걸 악용해서 돈을 버는 사람들을 매우 한심하게 생각하고 있기 때문입니다. 수험생을 위하는 척 하면서 곤경에 빠뜨리는 자들.
\vspace{5mm}

상담 질문글이 있는데 다는 답변해드리기 어렵고 한꺼번에 글로써 정리해 올리겠습니다.
사실 그 이야기가 다 그 이야기이고,
어제 시험에 응시해서 썰을 푸는 분들의 글이 훨씬 도움이 될 것입니다.
\vspace{5mm}

검증된 것
\vspace{5mm}

\item 1. \textbf{인강 좆까, 양치기 만세}
\item 2. \textbf{일지$-$상원 시스템의 효용성}
\item \textbf{3. 시험 일주일 전 컨디션 관리의 효과}
\item \textbf{4. 실모 무용성}
\item \textbf{5. 뇌를 믿어라.}
\vspace{5mm}

작년에도 비슷한 주장을 폈고 반론이 많았으나, 올해 콕콕 수험생들로 이건 확인되었네요.
특히 5번은 그걸 항의하는 사람조차도 본인이 철칙을 어긴 게 과학적(...)으로 밝혀져버렸습니다.
\vspace{5mm}






\section{장사철과 광고시즌 시작이군요.}
\href{https://www.kockoc.com/Apoc/489593}{2015.11.15}

\vspace{5mm}

속을 사람이야 속겠죠. 사실 그걸 뭐라고 할 수도 없겠죠.
4점짜리 어떻게 푸느냐 그러신 분들 많을 건데
\vspace{5mm}

$-$ 마플
$-$ 쎈 등의 개념서
$-$ 일품, 라벨 등의 좀 수준높은 문제집
\vspace{5mm}

이걸 다 풀어보고, 그 다음 교과서 읽어보시면서
'문제해결의 전략'이란 것을 쭉 생각해보면서 그걸 풀어보는 훈련 하는 것 빼곤 답이 없습니다.
\vspace{5mm}

적중적중거리는데 정작 그런 식으로 100$\%$면 마플이겠죠(별 문제들이 다 들어가있으니)
자꾸만 무슨 적중거리는데 그 논리면 적중 안 하는 문제집들 없고, 설령 그런 것 산다고 해도 공부 안 하면 아무 소용없으며,
무엇보다 그런 식의 말도 안 되는 적중 기준으로 따져도 적중 안 된 실모들이 훨씬 많으니까 내년에도 상술에 좀 휘말리지 마십시오.
\vspace{5mm}

실모야 보충용으로 푸는 건데 지금은 정말 무슨 인디아나 존스의 성궤, 성배처럼 숭상시된다는 게 문제입니다.
그럼 너는 마플을 왜 권하느냐 할 건데 간단합니다. 가성비가 가장 좋은 기출문제집이고 단점은 '분량이 너무 많다' 정도여서입니다.
\vspace{5mm}

그리고 광고하는 사람들은 성공한 학생들만 보지 말고 실패한 학생들부터 좀 돌아보시죠.
뭘 자기 것 봐서 잘 나왔다 그러시는 분들도 계시는데 그거 일부인 것 아시죠? 5명이 성공하면 45명이 실패입니다.
교재가 많이 팔렸다는 건 그만큼 그 교재 고객들 실패한 사람도 더 많아진다라는 걸 이야기하죠.
\vspace{5mm}

교재추천 함부로 안 하는 이유가 이딴 식으로 '교묘한 상술전략' 같은 게 있어서입니다.
그 분들은 이런 데 홍보하지 말고 다른 데서 좀 홍보해도 되지 않습니까?
이 사이트는 비영리성 분명 강조한 곳일텐데요. 영리적인 까페는 탑라인 가시면 되지요.
\vspace{5mm}

사이트 일궈온 사람들은 영리적으로 교재 홍보하는 사람들이 아니라, 그것과 무관하게 자기 공부 고민 털어놓고
이걸 극복하고자 한 사람들입니다. 장사하시는 분들이나 특정 교재 좋다라고만 하는 분들은 좀 자중했으면 좋겠습니다.
\vspace{5mm}






\section{문제 평}
\href{https://www.kockoc.com/Apoc/489714}{2015.11.15}

\vspace{5mm}

어떤 교재가 적중했더라... 는 건 적어도 제 기준에서 보면 없습니다(그 기준대로라면 기출은 100$\%$ 적중이겠죠)
올해 A, B형 킬러문제라는 것을 다시 풀고 훑어보고 정리해보았는데
홍보하느라 떠들석한 곳과 달리 네이버 블로그들 검색만 해봐도 벌써들 괜찮은 풀이들 올려두고 알아서들 하네요
\vspace{5mm}

\item 1. A형 30번
지수와 로그 같지만 실제로는 '부등식의 영역'(고1 수학) 문제입니다.
\vspace{5mm}

\item 2. B형 21번
역함수 미분, 그리고 그래프에 의존하지 않고 함수의 방정식화로 근을 구해 케이스 나눈다는 점에서 역시 고1 함수 문제입니다.
\vspace{5mm}

\item 3. B형 29번
14년도보다는 그래도 나아진 벡터 문제입니다. 이건 너무 오소독스해서리, 장점이 뭐냐면 14년도와 달리 단면화 찍기가 안 통했어요
\vspace{5mm}

\item 4. B형 30번
이 문제야말로 패턴, 업자들 풀이로 가면 힘들었을 거라고 생각합니다. 함수의 '설계'에 관한 문제이니까요.
\vspace{5mm}

간략히 말해서 새로운 건 없어보이긴 하지만 중요한 것이 없지 않네요.
고1 수학이 제대로 안 되어있으면 저 문제들 중에서 29번 제외하고는 풀기 어려울 수도 있습니다.
그런데 제가 아는 한 다수가 고1 수학을 대충 넘기고 이상한 스킬이나 특정한 풀이 같은 데 집착하다가 수능 뜨면 걍 죽어버리죠.
킬러 문제들은 패턴이고 스킬이고 안 먹힙니다. 원초적으로 문제에 쓰인 모든 조건들을 철저히 분석하고
연상할 수 있는 모든 기본 개념들을 동원하면서 모형을 설계해나가야합니다.
\vspace{5mm}

A형 30번의 경우도 범위 잘 잡아서 그래프들을 설계하고 그걸로 그림 잘 그린 다음
문제의 목적이 최단거리임을 알면 풀 수 있었습니다. 다만 숫자들이 기죽이는 게 있어서 격자점만 기대한 친구들은 나가리났겠죠.
B형 21번은 이거, 사실 \textbf{국어문제}입니다. 개인적으로 이 문제가 좋다고 보는 이유가 그것입니다.
수식적 표현을 의미적으로 파악하는 훈련이 되어있으면 바로 풀지만, 그게 안 되어있으면 패턴스킬질이나 하다가 안드로메다로 갔겠죠.
B형 29번은 사실 14년도에 나왔어야하는 문제라고 생각합니다. 이건 굳이 평할 건 없겠고
B형 30번도 매우 좋은 문제인 게 특정 논점에 치중한 스킬 풀이가 아니라, 논점들에 연연하지 않고
주어진 조건을 가지고 모형을 잘 만들고 검증해보는 접근 방법으로 가면 아주 쉽게 풀릴 수 있다는 것이었습니다.
\vspace{5mm}

다만 이런 것들은 학교든 학원이든 과외든 배우기 힘들 것입니다요.
특히 B형 30번은 보면 볼수록 좋다는 평이 나올 수 밖에 없는 게, 제대로 탈패턴화 아니면 정말 힘든 문제여서입니다.
본인이 문제에 현혹되지 않고 논리적으로 다양한 논점들을 자유롭게 논할 수 있어야 풉니다.
\vspace{5mm}

기출에서도 굳이 안 나온 거라면 30번 정도일 겁니다, 비슷한 문제는 올해 6평 30번 정도.
내년부터는 이 30번 비슷한 것들 $-$ 즉 단원에 얽매이지 않는데 학생들의 수리모형 만들기를 강조하는 문제가 나올 수도 있단 생각이 듭니다.
\vspace{5mm}

그럼 저걸 대비할 수 있는 교재가 있나.... 글쎄요.
함수를 직접 만들고 설계한다라는 일종의 함수모형화에 관한 직접적인 교재는 발견하기 힘듭니다. 간접적으로 제시한 건 있을지 몰라도.
(어째 그런 교재가 없지는 않다는 생각인데 좀 찾아보아야겠습니다)
고1 수학부터 바탕이 철저히 되어있고, 본인들이 직접 문제출제 같은 걸 해본 사람은 잘 풀었을지 모릅니다.
30번 자체가 국어로 치면 일종의 작문을 요구하는 즉, 논술적인 문제이기 때문입니다.
\vspace{5mm}

저런 문제가 나오면 '맨땅에서 어떻게 풀 수 있을까' 그걸 고민하는 게 중요합니다.
사실 맨땅 문제죠. 특별한 스킬이나 암기사항이 필요한 문제들은 전혀 아니었으니까요.
A형 30번이 요구한 건 그래프화 $-$> 케이스 나누기 $-$> 부등식의 영역이었고
B형 21번은 함수 그래프의 해석에다가 역함수 미분
B형 29번은 그냥 벡터 조작
B형 30번도 등장한 논점들로만 치면 깊은 건 없음.
\vspace{5mm}

어차피 이걸로 썰 풀기는 그렇고
공부들 하시려면 기출 신판 나오면 뭐 빨랑 정리하고
시중교재들 다 풀건 푸시는데
그 다음에 수리논술 문제들 한번 건드려들 보시기들.
\vspace{5mm}

현실적으로 인강 고르시겠지만 (추천해달라 가르쳐달라 댓글은 삼감)
패턴화, 유형화되는 건 피하시고 구태의연한 것도 거르시고 올해 30번 같은 걸 잘 대비해주는 것 고르시면 되겠죠.
\vspace{5mm}






\section{고1수학 투자하시기들 바랍니다.}
\href{https://www.kockoc.com/Apoc/490501}{2015.11.15}

\vspace{5mm}

일전에 고1수학이 중요하다고 한 글을 쓴 적이 있습니다만... 뭐 그건 예측이 맞다할 게 아니라 이거 너무 당연한 겁니다.
\vspace{5mm}

원래 고1 수학이 수능 범위에 안 들어가는 이유는
이게 '총론'이기 때문입니다.
7차 교육과정에서 행렬, 지수로그, 수열, 미적분, 확통, 기하는 "각론"이었죠.
\vspace{5mm}

총론은 고교수학 전체를 가로지르는 공통적인 내용,
각론이야 각 분야의 지엽적이면서도 개성있는 내용을 말하는 바입니다.
총론이 제외된 건 이건 '각론'의 내용으로 포함시켜 출제하겠다는 것이지요.
\vspace{5mm}

개정과정에서 수2는 문과수학 범위지만 이과수학 범위는 아닙니다.
그런데 착각하지 말아야 할 건, 수2가 이과수학에서 안 나온다는 게 아니라 '총론적'으로 출제된다는 것,
즉 다른 각론에 포함되어서 출제된다는 이야기이지 아예 안 나온다는 건 아니죠.
\vspace{5mm}

최근 3년간 수능출제에서 특기할 것.
$-$ 직관수학 멸종 : 직관수학으로 해결한다거나 하는 거 신기하게도 소리없이 사라졌습니다.
$-$ 스킬필요 급감 : 행렬이 사라지기도 했지만 케일리나 로피탈 논쟁도 없습니다.
$-$ 연속된 통수 : 국어나 영어도 그렇다 치고 수학도 킬러는 잘 보면 고1수학. 문과 수학 30번은 격자점 냈다가 \textbf{부등식의 영역}.
\vspace{5mm}

통수에 대비하려면? 그러니까 시중에서 말하는 출제경향대로'만' 공부하면 안 됩니다.
평가원이 이걸 정확히 알고 있기 때문에 저격질하는 것이지요.
\vspace{5mm}

이름난 실모들만 가지고 특정경향만 공부한 A
교과서나 시중교재로 전범위 다 기본을 공부한 B
물어볼 것도 없이 B가 유리합니다.
\vspace{5mm}

하도 문과 수학 30번이 어렵다라고 해서 오후에야 자세히 보았습니다만.
감상 3가지
\vspace{5mm}

1 $-$ "고 1 수학 정말 공부 안 하셨구나"
2 $-$ "A형 수학도 변별력 갖추긴 어렵지 않겠네"
3 $-$ "갓가원과 EBS가 사설 능가했구만"
\vspace{5mm}

지금 EBS 비난하는 사람들 없고, 올해 시험 이후로 평가원이 물수능냈다고 하던 입공부들 싹 들어가버렸죠.
\vspace{5mm}

바로 시작하시는 분들은 범위 상관하지말고 고1수학 철저히 하시고
특히 고1 수학 어려운 문제, 사정없이 풀어버리길 바랍니다.
\vspace{5mm}

공부하다보미면 위에서 말한 각론 $-$ 즉 지수로그함수미분적분확통 그런 것들은
결국 고1수학에 뒤집어씌우는 스킨에 불과하다라는 걸 알게 되실 겁니다.
\vspace{5mm}






\section{ebs 인강을 완강하고 사설 들으세요}
\href{https://www.kockoc.com/Apoc/491239}{2015.11.16}

\vspace{5mm}

지금 도저히 공부할 방법 모르겠다 하는 분들은요.
\vspace{5mm}

\textbf{그냥 ebs 들어가서 강의 따라가십시오.}
\item 1. 공짜이며 다운받을 수 있다.    2. 선별수강 가능하고 환승해도 된다   3. 강의력이 이미 일부 강의는 사설을 능가한다.   요즘은 안 듣습니다만 한참 인강 연구(?)할 때    처음에는 사설을 들으면서 오 거렸는데   나중에 EBS를 들으면서 오오오오오옷 거렸고 그 때 내린 결론은 '사설 함부로 듣는 게 아니다'라는 것이었습니다.   사설강의 뭐 들을가.   본인이 EBS 강의 완강이라도 해보고 고민하시길 바랍니다.   어떤 강의 듣느냐가 문제가 아니라, 자기가 수강한 강의 \textbf{'완강'을 할 수 있느냐가} 관건입니다.   그리고 인강의 문제는 딴짓입니다. 35분 강의 한 15분 들으면 살짝 쉰다고 웹서핑하다가 30분 날려먹고 그러고 있죠.
인강 듣지 말라는 이유 중 하나가 사실 이것 때문입니다.
이걸 막고 싶다?
\vspace{5mm}

파일 다운받은 뒤에 인터넷 끊고 보거나, 아니면 맛폰으로 옮겨서 보시길 바랍니다.
\vspace{5mm}

그리고 어느 강사가 좋냐.
그런 질문은 하지도 말고 사실 그런 것 따지지도 마세요.
그런 것 따지는 친구들이 성공하는 걸 본 적도 없습니다.
\vspace{5mm}






\section{수험판의 세뇌}
\href{https://www.kockoc.com/Apoc/491558}{2015.11.16}

\vspace{5mm}

교재와 인강 추천 글이 콕콕에 함부로 올라오지 않길 바라는 궁극적인 이유는.
제 눈에는 최소한 이 판은 제정신이 절대 아니란 겁니다.
\vspace{5mm}

인터넷 없이 동네서점에 나온 교재 꾸준히 공부하면 목표성취했을 친구들이
괜히 꿀교재 꿀인강 찾는다고 서핑질하다가 가랑비에 옷 젖듯이 특정 교재, 인강의 광신도로 전락해버리는 걸 많이 본다 그거죠.
\vspace{5mm}

작년부터 적지않게 상담은 했고, 콕콕 내에서 아니 저 대머리 늙은이는 왜 이렇게 사람을 단정지어... 그러실 건데
그런 이유는 간단합니다. 사람들이 서로 다를 것 같지만 '실패'하거나 '하류'로 전락하는 보편적인 패턴이라는 건 존재하고
그 중 하나가 저 \textbf{'세뇌'}입니다.
\textbf{용서할 수 없는 사실은 그 세뇌를 하는 자들은 많이 벌어들이고 있으며}
\textbf{그런 세뇌를 고발하는 것 자체를 일체 차단하려한다는 것이죠.}
\vspace{5mm}

만약 상담을 하려면 이런 질문이 와야죠
\textbf{"생1을 50 맞으려면 어떻게 해야해요?"    "올해 수능 B형 30번을 시간이 걸리지만 안정적으로 푸는 방법은 무엇일까요?}
\vspace{5mm}

그런데 정작 질문은
\textbf{"$\sim$ 교재 좋아요?"}
\textbf{"$\sim$ 인강 듣고 싶은데 어떡해야하죠?"}
\vspace{5mm}

\textbf{....}
\vspace{5mm}

수험사이트에 들어가지 않은 사람들이라면 저런 질문을 하지 않겠죠.
본인들은 부인하지만 결국 나쁜 정보에 노출됨으로써 그 자본의 먹이가 되기 시작한 겁니다.
자본주의 나쁘다 싫다 노오력해보았자 뭐하냐 가서 시위하자? 다 좋은데 한마디로 웃기고 있네요입니다.
본인들이 이미 상품의 노예이고 광고에 계속 영향받고 있는 것조차도 해결 못 하면서 뭘 자본주의를 극복한다는 거예요?
\vspace{5mm}

제가 해주는 상담의 요체는 재미없습니다.
비만환자들에게는 "기름진 것 덜 먹고 운동을 하라"라는 재미없는 조언이 유일한 해결책이겠죠.
마찬가지입니다. 공부환자들에게는 "걍 검증된 교재나 보고 인강 줄이고 문풀 오답 정리 충실히 해라"가 되겠습니다.
물론 양치기를 해도 안 되는 특수한 케이스도 있습니다. 이런 케이스는 그런데 그럴 수 밖에 없는 이유란 게 있더군요.
\vspace{5mm}

세뇌를 느끼면서 답답한 건 이것이죠.
공부는 열심히 합니다, 그런데 아무리 보아도 제가 보기에 성적이 안 오른 건
그 사람들이 마케팅에 세뇌당해 구입하게 된 '교재 탓'이 크거든요.
비만 환자들에게 야, 네가 지금 우걱우걱먹고있는 햄버거부터 스랙통에 넣으라고 하는 게 속시원한 해결책이지요.
하지만 이렇게 하면 다들 화내거나, 뺏기기 싫어서 더 처먹기 시작하겠죠.
\vspace{5mm}

방금도 인터넷을 서핑해보니.
역시나 아니나다를까 또 호구들을 낚기 위한 \textbf{밑밥깔기 언플이 시작된} 모양이더군요.
내년에 망하기 싫은 분들은 콕콕에 올라온 성공기든 실패기든 진짜 수험을 치열하게 하신 분들께 직접 여쭤보길 바랍니다.
\vspace{5mm}

그럼 이렇겠죠. "그럼 너는 왜 여기서 찌질대냐? 공개적으로 얘기하지"
저야 이렇게 답하죠. "어차피 호구들이 그런데 낚여줘야, 즉 깔아주는 사람이 있어야 님들이 올라갈 수 있음"
\vspace{5mm}






\section{중하위권이 희망이 없을 리는 없고}
\href{https://www.kockoc.com/Apoc/492004}{2015.11.16}

\vspace{5mm}

뭐 그렇게 여겨질 수도 있사온데 적어도 제가 관찰한 바는 그럼.
중하위권이 노력을 해도 안 되는 경우는 머리가 나빠서가 아니죠. 여러가지 이유가 있습니다.
\vspace{5mm}

\item 1. ADHD, 공황, 인지 장애가 있는 경우 : 문제는 이걸 본인과 가족도 모른다는 겁니다.
\item 2. 공부한 걸 자기 것으로 만들지 못 한 경우 : 이것도 복합적 문제로 나뉩니다만, "생각하는 법"을 배우는 게 필요한 경우라고 해야겠죠.
\item 3. 학은 하는데 습(習)이 되지 않은 경우 : 보통은 업자들에게 낚여서 학은 하는데 습은 못 합니다.
\vspace{5mm}

일지를 분석하든가 상담을 해보면
공부한 양과 질, 그리고 성적은 비교적 거의 정확히 일치합니다.
본인은 잘 모르겠지만 제 시각에서 보면 '잘 되는 이유'와 '못 되는 이유'란 건 존재한다는 것이죠.
\vspace{5mm}

다만 문제는 그거네요.
암세포도 생명인데... 라고 암세포에 정드는 일이 벌어진다는 겁니다.
시험지만 봐도 숨이 가빠오는 친구는 숨이 안 가쁘면 초조해하고
생각하는 법을 못 배운 친구는 생각하는 경험을 거부하는 경우가 많으며
한번 인강에 중독된 친구는 끝까지 인강만 따라가려 합니다. 그래서 문풀 경험이 부족하니 실전에서는 발리죠.
\vspace{5mm}

중하위권을 상위권으로 올리기 힘든 이유는
노오력이 부족해서도 아니고 머리가 나빠서도 아닙니다.
학습자 본인이 그 상태에 너무 친숙해져서, \textbf{변화라는 것을 거부}합니다.
즉, 기존의 자신을 바꿀 생각을 안 하는 것이죠. 고향 떠나기 싫다 그겁니다.
심지어는 스톡홀름 증후군도 아니고 그렇게 공부를 못 하거나 계속 실패하는 사람으로 걍 살다 죽겠다는 생각까지 하죠.
\vspace{5mm}

상담 청하는 분들은 부디 교재나 인강 그런 거나 질문하지 말고
자기가 바꿔야하는 것이나 청산히야 할 악습이 뭔지 그것부터 보시는 게 좋습니다.
그 악습을 고치지 못 하니까 \textbf{노력해도 안 됩니다.}
물론 노력을 해야만 그 악습의 정체가 비로소 드러납니다만요.
\vspace{5mm}






\section{+1수할 때 반드시 거쳐야 할 과정}
\href{https://www.kockoc.com/Apoc/493832}{2015.11.16}

\vspace{5mm}

입시 상담은 대부분 돈을 벌기 위한 수작인 경우가 많다는 거야 아시겠고
그런데 저는 그런 것보다는, 인간을 이해하고 그 인생에 개입해서 부정적인 걸 잡아내고 치유한다... 를 통해서
저 역시 인간을 탐구하고, 또한 그럼으로써 저 역시 제가 상실했다고 느낀 걸 치유하는 느낌을 받는다가 더 강합니다.
다만 제가 선택한 사람들과 대화하면서 깊이 들어가보면서 직설적으로 얘기할 때에는
이른바 감정이입이라는 걸 안 할 수가 없고, 뭔 중2병스러운 이야기냐 하겠지만 데미지를 입지요.
\vspace{5mm}

화두 중의 하나가 왜 장수생들이 피폐해지느냐입니다만.
\vspace{5mm}

원론적으로 말하면 장수생들은 잘 이끌어주는 구심도 필요하지만
1번 실패하면 사실 본인도 가늠할 수 없는 상처를 입고 그로써 온갖 성격적 변화를 거친다는 게 문제입니다.
콕콕 사이트가 이 점에서는 자부할 수 있는 게
아직까지는 남의 상처를 이용해서 장사를 하는 건 아직 없는 것 같고(홍보질은 강경하게 까고보는 것, 이거 유지되어야합니다)
그리고 정말로 공부한 사람들끼리 솔직히 사연을 털어놓기 때문에 힐링이 된다는 것이겠죠.
\vspace{5mm}

님들이 +1수를 고민한다면 이건 단지 다시 시작하는 게 문제가 아닙니다.
재수에서 삼수, 삼수에서 사수로 가면 반드시 상처를 입습니다. 실패는 실패이기 때문입니다.
그 실패한 이유를 분명히 직시하고 실패했다는 걸 인정하는 데도 사실 시간이 걸립니다. 하지만 이건 분명히 밟아둬야하는 과정입니다.
다시 시작하면 성공할 수 있는 가능성, 그리고 어떻게 하면 성공할 수 있느냐를 정말 제대로 계산하고 움직여야합니다.
즉, 실패의 인정과 고찰, 아울러 성공을 위한 과정의 치밀한 설계.
이게 없이 +1 수를 하면 좋은 결과를 기대할 수 없습니다. 왜냐면 치료하지 못 한 상처가 결국 곪아버리기 때문이지요.
\vspace{5mm}

교재를 추천해달라 인강이 뭐가 좋느냐.... 이런 질문을 하는 사람들 대부분은 \textbf{환자}들입니다.
반면 생각없이 무조건 교재, 인강 홍보해대는 사람들은 장삿꾼들이죠.
더 심각한 문제는 자기들이 정말 좋은 일을 하고 있다 착각하는 건데 절대 아닙니다.
나중에 얼마나 자기들이 나쁜 짓을 저질렀나 느끼면 죽고 싶어질 겁니다.
그들은 얼마나 많은 수입이 들어올까 즐거워하거나, 아니면 자기도 고소득자가 될 수 있다라는 것에 혈안이 되어있겠죠.
그냥 한마디로 제정신들이 아니죠.
\vspace{5mm}

상담하다보면 안 맞는 교재나 인강 구입하느라 아까운 시간 날리고 상처입은 케이스를 정말 많이 접합니다.
남들이 좋다고 하는 상품을 샀는데 성적이 그 따위로 나왔으니 이건 자기 탓이 아니냐고 자학하는 케이스들이 정말 많습니다.
이런 걸 접하다보면 정말 장삿꾼들에 대해선 좋은 감정을 품을 수가 없습니다.
+1수하는 사람들은 조금만 더 가면 '자살'까지 이를지도 모릅니다. 그래서 상담할 때에는(이것도 피곤해서 저도 어지간해선 피할 겁니다)
정말 조심스럽게 할 수 밖에 없고, 내가 그 사람이면 어디서 상처입었거나 실패했나 다 이입해보면서 스트레스를 안 받을 수가 없는 것이지요.
장삿꾼들은 한번이라도 자기들이 '자살까지 몇발자국 남은 학생들'을 가지고 장난질치나는 생각은 안 해보셨나보지요.
저는 얘기할 때마다 이 친구가 죽으면 어떡하나... 라는 걱정을 수백번은 해보았는데 말입니다.
\vspace{5mm}

+1수를 권유한 분들이 많습니다. 그런데 그걸 가지고 아마 오해한 분도 있겠지만요.
권유한 분들은 일지나 그간 행적으로서 제가 가능성을 보았기 때문에, 그리고 올해 실패했다면 \textbf{그 실패한 이유도 보였기 때문에} 그런 것입니다.
실패한 이유가 분명하다면 성공할 가능성은 높아집니다. 그 이유를 제거하면 되기 때문입니다.
그리고 사실 이 모든 상처를 극복하려면 일단 성공하는 수 밖에 없습니다.
제가 충고한 걸 듣고 바로 시작하는 분들은 \textbf{수능 보기 전 그 피말리는 기분}, 다시 살리시길 바랍니다.
1년이 늘어났다고(?) 해서 여유부리겟지만 그래선 안 됩니다. 11월 초에 느낀, 전신의 피가 메마르는 초조함을 다시 살려야합니다.
남들이 비아냥거리든가 가족이 눈치주는 거, 그거 아무 소용없습니다. 심지어 부모도 마찬가지입니다.
그 중 님 인생 책임져 주는 사람 아무도 없어요. 자기가 전신장애가 되더라도 끝까지 책임져 주는 사람이 아니면 무시해도 됩니다.
실패했을 때 비아냥대던 사람이 있으면 성공해서 다시 만나면 됩니다. 말 못 하고 억지로 칭찬하면서 속으로 부글부글대는 걸 즐기면 됩니다.
\vspace{5mm}

이 지겨운 수험을 더 해야하느냐. 이렇게 생각하면 안 되지요.
1년이란 시간을 더 가치있게 쓰느냐. 이렇게 생각하면 됩니다.
수험이 그럼 시간을 무가치하게 보내는 것일까요, 그건 아니지요.
자기가 진정 바라고자 하는 것을 노오력해서 현실적으로 도달할 수 있다면 그건 매우 가치있는 시간입니다.
더군다나 자기가 불가능하다고 생각하던 것을 이룬다면 그 뒤로 인생은 (당분간은) 매우 충만해집니다.
(왜 당분간이냐면 인간은 또 간사해서 언제 그랬다는 듯이 자만하고 다시 나태해지기 때문입니다)
\vspace{5mm}

+1수하시는 분들은 억지로 자기가 괜찮다라고 하지 마십시오.
올해 실패했다면 그 상처를 인정하십시오. 그래야 그 상처가 곪아서 자신이 잠식당하는 일을 막을 수 있습니다.
재수하는 사람들은 공부하는 종종 힘들 때마다 상당히 정신적으로 괴로워합니다. 자기가 실패자, 낙오자라는 생각 때문에요.
이걸 치유하는 방법은 "실패"를 긍정하는 것입니다. 실패를 제대로 했으니 그 반대로 나아가 성공도 할 수 있음을 차분히 바라보면 됩니다.
1년이란 시간을 가치있게 치열하게 보내면 시험 당일 \textbf{'뇌가 알아서 문제를 풀어주니'} 그걸 믿으시면 됩니다.
\vspace{5mm}

그러나 한편으로. 몇발자국 떨어진 곳에 있는 사신(死神)도 의식하길 바랍니다.
인간은 실패하면 모든 걸 리셋하기 위해 어리석은 선택, 즉 자살을 하는 충동을 받습니다.
자살하지 말기 위해 열심히 하라, 혹은 무한긍정하라 그런 이야기가 아닙니다.
\textbf{오히려 터무니없는 사건일수록 매우 익숙한 일상이 될 수} 있으니, 긴장하시라는 이야기입니다.
님들은 내년에 적어도 3$\sim$4번은 자살충동을 받을 것입니다. 다 포기하고 엉엉 울고싶다, 그냥 산에 들어가고 싶다 할지도 모르죠.
그런데 그건 냉정히 말하면, 공부하기 싫어하며 주인을 배반까지 하는 뇌의 '변덕'이기도 하고 '본능'이기도 합니다.
그럴 때가 오면 알아서 먼저 쉬거나 스트레스를 풀어주기도 해야겠지요. 시험볼 때까지 말 안 듣는 뇌와 계속 싸워야할 것입니다.
시간내서 자기가 내년 수능에 실패해서 안 좋은 충동을 받아 정말 자살했을 때... 를 한번 상상해보는 것도 나쁘지 않습니다.
그리고 왜 자기가 그런 안 좋은 충동에 휩싸였는지, 어떤 생활을 했는지 상상하면서 \textbf{미래를 과거형으로 복기해보는 것}도 권해볼만합니다.
\vspace{5mm}

1년 전과 달리 콕콕이 좋아진 것이 있죠. 이제 공부하는 사람들끼리 모여서 더 협조적으로 나아갈 수 있단 것입니다.
작년에는 일지를 일일히 체크해주고 그랬지만 지금은 그럴 필요가 많이 줄어들었습니다.
성공한 사람이든 실패한 사람이든 진지하게 자기 수험경험을 늘어놓고 조언해줄 수 있는 분들이 많이 늘어났습니다.
다시 시작하는 분들은 서로가 서로를 '관리'해주고 '따끔한 지적'을 해줄 수 있는 동료들을 찾으시는 것도 권해드리겠습니다.
사람은 자기에겐 한없이 관대하나 남에겐 엄격합니다.
본인은 시간관리를 못 하고 한없이 늘어져도 남의 공부에는 매우 엄격하고 현명한 관리를 해주지요.
이런 것을 이용하려면 타인과 손을 잡는 것도 매우 괜찮은 방법입니다.
아울러 상대가 괴로워하거나 스트레스 받는 걸 도와주다보면 자기의 문제도 간접적으로 해결할 수 있을 것입니다.
\vspace{5mm}

이제 내년에 어떻게 해야할지에 대해서 제가 눈여겨본 분들에 대해선 개입은 모두 한 것 같습니다.
나중에 나이를 많이 먹고 머리가 빠지다보면
합격보다도 이렇게 힘들게 노력한 순간이 더 인상깊고 가치있었다, 자기 인생이 바뀌는 중요한 시간대였음을 회고할 수 있을 것입니다.
저승사자도 공부를 열심히 하는 사람에게는 다가오지 못 합니다.
최소한의 학습량을 유지하면서 회독수와 문풀수를 늘리고 오답을 정리하면서 혼자 스스로 어려운 문제를 해결하는 '사고'를 하다보면
어느 순간 인간을 넘어 신과 하나가 된 듯한 몰입을 경험하실 것이고, 아 이게 바로 그것이구나를 느끼실 겁니다.
그 순간이 되어야 변덕스럽고 말 안 듣던 뇌가 비로소 제정신을 차리고 천상의 길로 나아가게 되는 것이지요.
\vspace{5mm}

제가 깊이 개입하면서 수험을 넘어 인생상담까지 간접적으로 하게 된 분들은 잘 될 것이라고  계산한 바입니다.
부정적으로만 흘러갔을지도 모르는 분들이 다시 바닥을 치고 올라가는 것을 보는 것은 매우 즐거운 경험입니다.
거액의 돈을 주고도 느낄 수 없는 희열감이지요.
\vspace{5mm}

그럼 모두 스타트하시길 바랍니다.
\vspace{5mm}






\section{교재 vs 놀이 하지 마세요}
\href{https://www.kockoc.com/Apoc/494964}{2015.11.17}

\vspace{5mm}

현대판 어부지리
\vspace{5mm}

조개 : 야, 난 쎈을 풀겠어
학     : 야, 난 마플을 풀겠어
\vspace{5mm}

이러면서 vs 놀이로 3달 허송세월
\vspace{5mm}

\textbf{어부 : ㅄ들. 난 다 풀었지롱}
\vspace{5mm}

교재를 한정해서 골라야하는 건
공부량이 원체 많은 고시 공부에 한해서이고
수능은 사실 공부량이 많은 과목이 아닙니다요.
\vspace{5mm}

단권화는 \textbf{머리에 하는 것}입니다. 원래 단권화란 개념도 교재가 변변치 못 한 수십년 전에 먹히던 것이고
지금은 웬만한 교재들도 양이 풍부합니다.
\vspace{5mm}

그런데 다들 착각하는 건. 교재에 내용이 많다고, 그게 공부하는 님들 머리에 지식이 많이 들어간 게 아니란 겁니다.
양이 1000인 교재도 5번 보면 님들 머리에 300 정도 들어갈까 말까죠.
반면 양이 500인 교재를 30번 보았다면 님들 머리에 2000이 들어갑니다.
\vspace{5mm}

단권화는 머리에 하는 것입니다.
노트 정리? 그것도 노트정리를 하는 과정에서 머리에 들어가기 때문에 추천되는 것이지,
노트 잘 만들었다고 해도 시험 시간에 가지고 못 들어가는데요
\vspace{5mm}

교재 뭘 풀어야하나, 중복문항 피해야하지 않나.
\vspace{5mm}

까놓고 말해서 낭설입니다.
\vspace{5mm}

기출은 공부하려면 정말 개개 문항을 다 설명할 수 있을 정도로 반복해 풀어야합니다.
그 과정에서 평가원의 의도라는 것, 출제원리라는 걸 읽을 수 있습니다.
한번 풀었다. 그건 무의미하죠. 중요한 건 그 문제를 내가 설명할 수 있느냐, 심지어 변형출제까지 가능하냐는 것입니다.
변형출제 가능할 정도면 실모는 필요없습니다. 실모가 사실 기성문제 변형출제 수준이기 때문이죠.
\vspace{5mm}

그냥 닥치는대로 반복해서 풀고 암기하고 그게 공부의 시작입니다.
\vspace{5mm}

공부 잘 하는 애들이 교재 vs 놀이 하는 것 단 한번도 못 보았습니다.
물론 '업자'들은 그걸 조장하죠. 그게 마케팅이 되기 때문이죠.
심지어 어떤 업자들은 자기 교재만 보라고 하죠. 학생들이 다른 교재 보면 자기 교재가 형편없다는 게 드러나거든요.
\vspace{5mm}

교재 뭐가 좋냐.... 라고 하는 게 장수의 지름길입니다.
그러니까 쓸데없는 생각 말고 그냥 다 푸세요.
다만 개념서만큼은 그냥 현직교사들이 쓴 것이나 교과서 보는 걸 권합니다.
vs 놀이 굳이하려면 개념서일 건데 이상하게 이건 안 하덥니다만,
\vspace{5mm}








\section{죽음의 절벽}
\href{https://www.kockoc.com/Apoc/497307}{2015.11.18}

\vspace{5mm}

성공하기 위해 필요한 노력은 $10^10$  인데
지수 계산을 할 줄 몰라서 10 x 10 으로 계산하거나
1010으로 보는 것까지 그렇다 치고 그냥 10만 해놓고 난 노력했는데 안 된다는 게 보이죠.
\vspace{5mm}

우물 안 개구리라는 표현은 이런 데 쓰는 것이죠.
동네 뒷산만 본 사람 입장에서는 백두산, 한라산, 에베레스트산이 어떨지 감이 잡히지도 않을 테고
산골짝에서 개울만 본 사람이 바닷가에 간다면 순간 압도당하겠죠.
\vspace{5mm}

바다를 본 적이 없는 사람은 기껏 가본 호수가 넓은 줄 알고 난 노오력을 많이 했다 합니다.
그런 친구들은 \textbf{바다에 데려가보는 수 밖에 없겠죠.}
한번 정말 시간이 난다면 서울대 도서관이나 신림동 고시촌 같은 데에서 $-$ 가능한지 모르겠으나 $-$ 그 사람들 어떻게 공부하나 보세요.
그리고 그들이 몇년간 공부했는지도 확인해보시면 되겠습니다.
\vspace{5mm}

이래서 환경이 매우 중요하단 이야기입니다.
부모가 판검사교수의사변호사인 쪽이 평균적으로 공부를 잘 하는 건 유전이 좋아서가 아니라
그렇게 공부한 사람들 밑에서 자랐기 때문에 공부에 들이는 노력에 대한 눈높이가 매우 높기 때문입니다.
\vspace{5mm}

상담해보면서 느끼는 건 다들 들여야하는 노력의 양을 잘못 계산하고 있단 것입니다.
1010   이 필요한데 10X10 정도 해놓고 안 된다라고 하고 있죠.
\vspace{5mm}

수학만 보더라도
유치원 때부터 조기 교육 시작해서 고3때까지면 가히 10년 넘게 교육받고 스트레스 받습니다.
좋은 집안에 태어나든 아니든 그런 노오력은 기울여온 것입니다.
그래서 제가 \textbf{노오력 까는 병신들은 스랙이라고 외치는 겁니다.}
이 스랙들은 저 금수저들이 '노오력을 안 하고 부모에게 얻어먹기만 했을 것'이라고 소설쓰면서 자기는 노오력한다고 소설쓰거든요.
말씀드릴까요?
\textbf{금수저들일수록 더 노오력을 열심히 하고, 게다가 인성조차 더 좋습니다.}
그럼 흙수저는? 말로만 노오력한다고 해놓고 말초적인 유흥에 빠진 경우가 더 많고 인성은 더 개차반인 경우가 많았어요.
그런 주제에 나중에 이 사회가 개판 어쩌구 말로만 지껄이겠죠.
그런 사람들은 이 사회 싫으면 그럼 '괜찮은 나라'로 이민 걍 가지 왜 아직도 안 가나모르겠습니다.
\vspace{5mm}

격한 말투일수록 진실성을 담보하니 이게 선행이라고 생각하고 더 적지요.
\vspace{5mm}

부모들이 왜 자기 자식들 내신이 불리하더라도 좋은 학교에 보내려하는 줄 아십니까?
물론 허영심 없는 부모들이 없는 것도 아니죠. 그런 부모들은 실패합니다.
잔인한 진실은 그겁니다. \textbf{자기 아이들이 수준낮은 애들 만나서 타락하지 않길 바라는 겁니다}요.
그럼 이 이야기에 풀발기하겠죠. 네가 흙수저와 서민 까고 있니?
\vspace{5mm}

집 밖에 나가서 근처 유흥가에다가 법망을 아슬아슬하게 넘어서는 업종 장사하는 생산자나 소비자들이 어떤 계층인지 보시죠.
물론 금수저들도 비밀스러운 소비를 하고 있죠. 그러나 규모로 치면 전자에 비할 바는 아닙니다.
\vspace{5mm}

신림동 고시촌이나 노량진 가면 공부하는 사람들만 보지 말고 근처에 발달한 다른 사업을 보시죠.
학원과 서점과 독서실 뿐만 아니라 온갖 종류의 유흥업종부터 심지어 유전, 생식을 이용한 이상한 사업까지 가장 먼저 생기는 동네입니다.
자기들이 수험에 실패한 사람들은 절대 자기들이 실패한 이유, 솔직하게 고백하지 않습니다.
정말 절망적인 상황이 되어서야 지킬 자존심도 없어서 그 때서야 실토하죠. 하지만 그래도 '그런 사소한 걸로 왜 실패해'라고 부르짖죠.
그게 정말 '사소한' 걸까요?
\vspace{5mm}

공부하다가 조금이라도 딴길 가면 그거 수험 실패라고 생각하지 말아야 합니다.
수험 실패면 다시 극복할 수 있다라고 착각하거든요.
님들이 공부하다가 럴을 하거나 연애질을 하면 그 때는 \textbf{"저승사자"와 가까워지는 겁니다.}
나이 처먹으면서 느끼는 건 죽음은 늘 그림자처럼 우리 뒤를 따라다니고 기회를 엿본단 겁니다.
이 글을 쓰는 저도 언제 뒈질지 모르겠지만, 저도 살면서 느낀 게 정말 오래 살 것처럼 생각하던 녀석이 가버리는 경우도 많고
심지어 행복해보이던 사람도 갑자기 부고가 뜨는데 나중에 알고보면 그럴만한 숨겨진 사연이 있었단 건데
특히 수험은 자기를 대수술하는 과정인지라 이게 실패하면 인생포기=자살이라는 게 결코 농담이 아닙니다.
남들이 보기에는 저거 왜 죽냐... 그럴지 몰라도 본인들은 그렇게 못 느끼죠.
\textbf{조금이라도 일탈한다면 그건 저승사자와 키스하는} 것입니다.
\vspace{5mm}

수험은 그럼 재정의되죠\textbf{. 승천을 목표로 죽음의 절벽 앞에서 도움닫기.}
무슨 개뿔 중2병급 표현이냐... 할지 모르나 저건 제가 보기엔 진실입니다.
절벽에 떨어진다고 해도 죽지는 않겠죠. 다만 절뚝거리며 살 각오는 해야합니다.
\vspace{5mm}

공부하다가 다 때려치우고 걍 기술이나 배우고 일이나 하자... 라고 말로만 그러지말고
그럼 일주일간 알바 뛰어보고 해보세요. \textbf{엿같아서 공부하는 게 낫다}는 걸 하루만에 느낄테고
그런 걸로 일할 노력이면 걍 공부해서 좋은 데 갈 수 있다, 내가 게을렀다라는 걸 일주일만에 납득할 겁니다.
\vspace{5mm}

운명을 바꾸는 건 매우 힘들다라는 건 동의하실 겁니다.
그런데 왜 그럼 공부를 힘들게 하는 건 싫어하시죠?
힘든 공부일수록 운명을 바꿀 수 있단 이야기인데?
\vspace{5mm}

+
\vspace{5mm}

최근 심각하면서도 재밌게 보는 게 경제의 지리학인데.
\textbf{사는 데 따라서 인생이 달라진다}는 것.
강남 강북 집값 격차가 커진 게 자본가의 음모?
너무 구태의연한 이야기죠. 오히려 그건 사는 사람들의 교육, 생활습관, 그리고 외모까지 달라졌기 때문입니다.
\vspace{5mm}

+
\vspace{5mm}

쉬운 예 들어드릴까요?
\vspace{5mm}

남자들이 헤벌쭉하는 흔한 요조숙녀.
그런 요조숙녀를 양성하려면 부모가 사자돌림이거나 대기업 중역 이상이어야 하고
사는 곳은 서초강남송파목동이나 여의도는 되어야 하며
학교는 정말 때묻지 않은 명문교여야하며
부유한 교회, 성당, 사찰. 그리고 먹는 것이나 의료서비스도 최상의 것으로 받아야 '완성'되죠.
\vspace{5mm}

여자를 완성시키는 건 비단 성형수술만이 아닙니다. 결국 환경이죠.
그런데 그런 환경은 그 부모나 조부모들의 '노력'의 결과죠.
그런데 남자들은 이런 여자들을 너무 날로 차지하려고 하죠.
\vspace{5mm}

집값이 비싼 곳은 저런 요조숙녀가 많습니다.
반면 집값이 저렴한 곳은 해맑은 여자애들도 때가 되면 술집에서 일하거나 못난 남자 만나 폭행당하고 있죠.
\vspace{5mm}

+
\vspace{5mm}

그럼 언제까지 노력?
노력 자체가 즐거워지는 순간요. Runner's High가 올 때까지.
이번에 액상탄마님이 인증한 것에 오, 잘했구만 했지만 아직 부족하다 느낀 게 그거예요.
아직은 노오력 자체에 쾌감을 느끼지 못 했군. 즉, 열심히는 하지만 결국 미치진 못 했군 정도.
더군다나 대화해보면 분명 1년 전과는 달라졌습니다.
1년 전은 걍 바바리안이었는데 지금은 조선시대 말까지는 왔죠.
그 때야 아이구 내 인생 그랬지만 지금은 몸에 좋은 보약을 챙겨먹으려는 적극성까진 띠고 있죠.
\vspace{5mm}

그러나 아직 수학적인 사고 $-$ 즉 합리적인 사고방식까지 체화된 건 아닙니다.
예를 들죠. 팀의 준에이스는 갔지만, '감독'까지는 못 되었다는 것이죠.
\vspace{5mm}

1년 공부해서 저 정도면 상당한 것이죠. 남들은 10년 넘게 공부해서 겨우 가는 수준인데 말입니다.
그러나
\vspace{5mm}

\textbf{1년 공부해서 저만큼 올라갔다는 건, 그만큼 또 빨리 추락할 수 있단 이야기죠.}
\vspace{5mm}






\section{문과 이과 전향에 대한 썰}
\href{https://www.kockoc.com/Apoc/499454}{2015.11.18}

\vspace{5mm}

여러가지 이야기가 있습니다만.
성공/실패를 떠나서 제가 드리고 싶은 이야기와 더불어 현실적인 얘기를 드리죠.
\vspace{5mm}

\item \textbf{1. 도전 자체는 새로운 것일수록 좋다.}
\vspace{5mm}

저야 현실적인 사람이고 다른 사람들이 커리 제시하면 안 될 것은 아니라고 보지만
전과에 대해서는 그다지 비관론을 펴진 않습니다. 수험은 말 그대로 도박이기 때문이지요.
문과<이과 라고 알려져있지만 이건 사실 제대로 추궁 들어가면 근거는 없습니다.
왜냐하면 여러가지 이유로 어린 시절에 수학을 잘 못 해서 문과 갔는데 지금은 머리가 트여 이과수학을 잘할 수도 있기도 하는 등
중요한 건 본인의 적성, 취향까지 감안해서 얼마나 잘 맞느냐인데.
사실 이건 아무도 검증해본 적이 없기 때문입니다. 직접 부딪쳐서 도전해보지 않는 이상은 정말로 알 수는 없습니다.
\vspace{5mm}

그보다도 '실패'를 생각하고 도전 자체를 기피한다는 것 자체가 제가 보기엔 가장 위험한 것 같습니다.
실패할 걸 각오하고서라도 성공할 수 있는 걸 염두에 두면서 도전해보는 사람이 살아남지,
"아, 나는 되지 않을 거야"하면서 칼을 뽑기도 전에 \textbf{포기해버리는 사람은 차라리 죽어버리는 편이 낫습니다}.
제가 그 시절로 돌아간다면 (물론 제가 다시 수험생이 될 이유는 없지만)
떨어질 것을 알면서도 그 낮은 가능성을 높이기 위해 별의별 수작을 다 부리면서 노력할 것이고
실패했을 경우에 충격을 줄이고 그 실패조차도 자산화하기 위한 장치를 마련해놓지
실패하면 어떻게 될까 하는 걸로 자포자기하지는 않을 거란 것이죠.
\vspace{5mm}

위기가 기회라는 말은, \textbf{자기가 직접 부딪쳐보는 위기가 기회가 될 수 있다}는 겁니다.
내가 위기에 도전하지 않으면, 위기가 나를 찾아옵니다.
저는 다소 운명론자이긴 합니다만, 그보다는 어떻게 하면 운명을 바꿀 수 있을까에 더 관심이 많습니다만
사실 운명이란 것도 일종의 기호적 매트릭스라는 점에서 그건 고정되있는 동시에 고정되어있지 않다는 것.
그리고 목숨을 걸고 도전하는 것 외에는 운명을 바꿀 방법은 아무 것도 없다는 겁니다.
이제야 뭔가 저주 같은 것을 깨기 시작한 콕콕의 한 사람이 도망친 곳에 낙원은 없고 색안경 껴도 낙원이 아니라 했는데 맞는 말이죠.
도망치면 운명은 못 바꿉니다. 도망칠수록 운명은 공고해집니다.
\vspace{5mm}

\item \textbf{2. 그럼 문과 → 이과는 무모하기만 한가?}
그건 아닙니다. 일단 분량의 차이는 열심히 하면 극복할 수 있기 때문입니다.     우선 이과수학도 편한 건 있습니다. 제대로 틀을 갖추면 그 다음에는 학문적 시스템을 이용할 수 있다는 것입니다.   문과수학이 경우는 이종격투기와 같아서 쓸 수 있는 무기가 별로 없습니다. 격자점 나올 걸로 생각한 사람들이 30번 고등수학에서 말아먹었듯.   그러나 이과수학의 경우는 온갖 화력전이라 개살벌하지만, 본인도 여러가지 다채로운 무기를 사용할 수 있습니다.   즉, 최소 공부량을 확보하고 제대로 체계적으로 접근해서 이과수학의 여러가지 tool들을 쓸 수 있으면 문과수학보단 낫다는 것이죠.   수리적 마인드나 설계 측면에서는 사실 별 차이는 없습니다. 이번 문과 30번이든 이과 30번은 "설계"를 할  수 있어야하는 문제였죠.   다만 문과수학의 경우는 평면좌표축과 정수론에 치중해있다면      이과수학의 경우는 공간좌표에다가 공간논리적 감각에다가 실수까지 확장된다가 차이 정도인데   이게 넘사벽인가... 하면 그건 아니라고 보고 있습니다.    오히려 사람에 따라선 이과수학을 제대로 공부해서 수학의 컴플렉스를 엎어버릴 수도 있다는 생각입니다.   수학의 경우는 결국 제대로 공부하느냐 그걸로 결판나는 것이죠.   예컨대 문제유형만 외운다거나 무슨 실모의 적중을 따진다... 가장 어리석은 생각입니다.   수학 공부는 결국 자기가 기본적인 것을 철저히 하고 tool을 능숙히 다루면서 논리기하대수 사고를 통해   새로운 유형이더라도 침착하게 대비할 수 있는 준비를 해놓는 거지, 예상유형을 암기하고 정리해놓는 것이 아니죠.   사실 이런 방식이면 이과생이더라도 문과 30번은 못 풀었을 겁니다.   \textbf{3. 과탐은 어떠한가?}   사실 과탐이 가장 문제입니다. 이건 다들 어렵게 내는데 어떻게 대비할지는 뚜렷이 안 가르쳐주기 때문입니다.   그러나 이건 이과생들이 과탐에서 겪는 고통이 문과생들이 과탐을 공부했을 때보다 덜하다는 건 아닙니다.   오히려 언어능력적인 측면에 있어서는 문과생과 이과생이 똑같이 과학을 공부한다면 문과생이 더 유리할 가능성이 높습니다.   과탐에서 난감한 게 바로 말장난을 까는 것인데 이건 이과생들이 약하기 때문입니다.   언어외적인 것 $-$ 즉 기하나 수리적인 과탐킬러의 경우는 사실 이과생들도 툴을 스스로 개발하거나 강의참조해야한다는 점에선 딱히.   \textbf{4. 그렇다면 왜 문과 → 이과는 어렵다 하는가?}   절반 정도는 과장된 허풍이 있다고 여깁니다만, 세뇌론에서 얘기하듯 이과>문과라는 추상적 고정관념이 다수의 무의식에서 실체화되어서입니다.   그런데 흥미로운 건 이게 직접 검증되었느냐... 그건 아니란 것입니다요.   흔히 드는 예가 이과에서 문과 가니까 등급이 올랐다는 것인데 이게 변인통제가 충실히 된 건 아니죠.   이과수학을 공부하다가 문과로 간 경우야 당연히 더 많은 내용이 선행되었거니와 당사자가 공부기간이 기니까 그런 것이지요.   이걸 제대로 비교하려면 정말 문과생과 이과생이 똑같이 둘 다 모르는 내용을 동일하게 학습했을 때의 학습성과로 봐야합니다만.   가령 집합과 명제를 똑같이 친다고 하더라도 이과생이 더 유리하다고 확언할 수 있을지는 의문입니다(공부량과 강의가 동일하다 가정 하에)   이과생이 우월하다고 하는 것이 원래 이과교육이 뛰어나서인지, 공부량이 정말 많아서인지 그것도 구분해야겠습니다만   이 역시 딱히 검증된 건 없다는 측면에서는 수험자본이 만들어낸 일종의 세뇌라고 생각하고 있습니다.   그것도 그렇거니와 저건 평균적인 수험생들을 염두한 것이고   수험생 개개인으로 치자면 또 역시 일반화시키기는 어려운 문제입니다.   수험생 개인이 공부할 의욕이 많으며 공부하는 방법을 깨달았으며 올바른 커리를 밟는다면 이과로 전과한 게 문제라고 할 수 있을까요?   사실 결과는 알 수 없는데 처음부터 단언하는 경향이라는 게 있는 것 같습니다.   \textbf{5. 그렇다면 이과 공부는 어떻게 해야하나?}   수리 감각이나 머리가 좋아야한다기보다도, 오히려 기본적인 교과서 개념을 더 철저히 암기하고 따져야하지 않을까요?   저 개인적으로는 오히려 문과수학이 감각을 요구하고, 이과수학일수록 이론적인 걸 더 많이 요구한다는 점에서   문과수학이 이과적이고, 이과수학이 문과적이라는 느낌을 받은 적이 많습니다.   다만 이과수학 과정의 텍스트는 언어적으로 풀이되지 않았습니다. 그래프와 수식으로 요약되어있죠.   만약 문과생들이 이걸 타당하고 합리적인 텍스트로 풀이하면서 이야기하듯 공부해나간다면 이거야말로 올바른 방법이 아닐까 싶더군요.   왜냐면 이과생들이야말로 문과적 소양이 없다보니 그런 텍스트 풀이를 하지 않고 생각없이 암기하다가    좀 심화된 출제로 나오면 대비하지 못 하는 경우도 많기 때문입니다.   수학공부에 있어서 중요한 건 어떻게 푸느냐 보다도,   자기가 공부하는 내용이 어떤 맥락에서 어떤 의미를 갖느냐를 정확히 이해하고 암기하는 것인데   이런 측면에서는 문과생들이 더 잘할 수 있다고 보고 있습니다.   다만 문제는 문과에서 그런 올바른 '리터러시'를 교육한다고 보장하지는 못 하는 것입니다만.   전과하신 분들이면 현역 고3 올라가는 이과생과 비교하면 됩니다. 사실 큰 차이는 없을 테고    황금의 3개월간 공부하시면 능가하면 능가했지 부족하지는 않을 것입니다.   나는 할 수 없어라고 생각하면 사실 아무 것도 바꿀 수가 없죠.   윗 글은 위로가 아니라 그간 제가 생각하고 분석한 바인데 아마 납득가는 분들이 꽤 있을 것입니다만   이런 것들을 따져서 어떻게 승부할 것인가를 계산하고 노력해야지   아, 그래도 이과는 무리야.... 라는 태도라면 사실 \textbf{앞으로 살면서 할 수 있는 건 아무 것도 없습니다}.   그리고 사실 우리는 남들이 그래주기를 기대해야합니다. 그래야 우리가 위너가 되고 남들이 루저가 되기 때문이죠.   세뇌론을 쓰게 된 경위이기도 하지만, 자기가 운명을 바꿀 수 없다면    그냥 죽을 때까지 노예로 살 것인지, 아니면 정말 적당히 살다가 삶을 정리할 건지 진지하게 고민해야합니다.   아무 도전도 안 하고 꿀빨다보면 편히 살 것이다... \textbf{우리가 가만히 있으면 불행과 사고가 직접 찾아옵니다}.   \textbf{우리가 능동적으로 찾아간 위기는 기회가 되지만, 수동적으로 맞이한 기회는 위기가 되죠.}      물론 말이야 쉽죠.의지대로 저걸 바꿀 수 있는 사람은 경험상 20명 중 1명 꼴입니다.    그런데 그런 사람들이 결국 상류층에 올라가는 거지, 그저 도망가면서 공부하고 적당히 학교 잘 가면 풀릴 거야... 글쎄요.   그런 경우가 있기는 하던지요?      마인드 차이라는 건 참 사소하지만 매우 중요한 차이를 가져옵니다.   이걸 언어로만 보면 못 느끼는데 사람들을 만나보면서 그들을 비교해보면 뼈저리게 느껴요.   남들이 고정관념에 갇혀있을 때 자기는 그걸 두려워하지 않고 뭔가 감행한 사람이 그게 부정적인 것일지라도 뭔가 해내기는 합니다.   +   여담적으면 그럼 과거의 그 훌륭한 이과천재들이 지금은 뭘하고 계시나라는 질문을 던지면 됩니다.  사실 다들 평범해졌죠.  사회에서 인정받는 천재(?)들은 정치가, 고위공무원, CEO 정도인데 이게 이과 공부와 관련이 있나 의문.    현재는 검증되었다고 볼 수도 있는 터라서.  사회적으로 성공한 사람들의 공통점은 뭐 설명할 필요도 없습니다.  첫째, 미소 짓고 노오력만 하는 척 하지만 실제로는 경쟁자들을 어떻게 짓밟을까 사악한 고민을 신나게 한 사람들이죠.  둘째, 위에서 말한대로 도전하는 것에서 브레이크가 없었단 겁니다. 뭐 무모하게 하면 망했겠지만.  12월 이전은 그래도 시간이 많이 남았으니 나는 해도 안 된다라거나 부모님 눈치 본다... 이건 글쎄요.  내년 10월 정도에 얼마나 쓸데없는 고민이었는지 뼈저리게 깨달으실 듯.  ++  그러고보면 수학 잘 하는 사람의 소위 현학적, 허세적 태도가 '넘사벽' 신화를 공고히 한 콘크리트였던 것 같은데.  이런 걸 뭐라고 해야할까나.




\section{교재 뒷담화 : R}
\href{https://www.kockoc.com/Apoc/500147}{2015.11.19}

\vspace{5mm}

진짜를 보지 못 한 사람들은 가짜에 환호하고
그 가짜에 환호하는 우매한 사람들을 보면서 비웃는 게 참즐거움이 아닐까 싶다.
\vspace{5mm}

일단 R의 저자는 $-$ 현재 e$-$book으로 밖에 못 구하지만 $-$
그리스 기하학과 현대 수학에 관한 꽤 탁월한 수필(?)을 냈다.
고교수학의 차원에서 뭔가 서양수학의 정신(?)이라는 걸 알고싶다면 꽤 읽어볼만한 책이다.
무엇보다 국내저자가 저런 책을 쓸 수 있었다는 점에서 놀랍다.
\vspace{5mm}

그런데 저건 수험서 아니잖아요. 그런데 왜 교양타령해요? 아니 뭔 책인지는 아나
그런데 R은 매우 탁월한 책이다. 이 책은 일본책을 누를 수 있는 마스터피스임.
그런데 저기 목록에는 안 올라갔더라는 것. 정말 저자나 일부 구매자 아니면 알 수 없는 꿈 속의 책이 되어버렸다.
물론 나야 맛있게 잘 먹겠습니다라고 얌냠거리고 있는데 누구든 소개해줄 생각은 없다.
궁금한 사람은 저 저자 분과 개인적 연락을 해서 구해보시든지(그렇다고 개인적 연락을 해서 구했다는 건 아니니까)
아니 e$-$book은 있을지 모르지만 뭐 알아서 찾아보시도록
\vspace{5mm}

일본 책과 다른 것 $-$ 일본 책은 마치 독일인들처럼 정교한 방법론을 제시하고 사실적이다. 그런데 도통한 맛은 없다.
반면 R은 세세한 지엽적인 건 빠져있는데, 문풀 방법을 고민해본 사람이라면 결국 도달하게 되는 도통한 경지를 적어놓았다.
앞의 두 책을 쓴 것도 본인이 열심히 공부해서여서일건데, 이 정도면 현직교사는 가볍게 넘어서는 내공이다.
게다가 우리나라 수학에서 어떻게 잘못된 방법론을 가르치는지도 세세히 적어놓은 건 현직 교사도 못 하는 거지.
\vspace{5mm}

아마 저자가 출판사 컨택을 잘 하면서 이 책에다가 현재 기출 킬러문제만 잘 융합시켰어도
그냥 컨셉만 잘 잡은 병신같은 책들을 누르고 잘 팔리지 않았을까도 싶은데
개인적으로는 이런 건 공유되어보았자 나도 재밌는 게 없기 때문에 그냥 내 머리칼 갯수만큼이나 비밀로 처리.
뭐 찾을 사람은 찾겠지, 정말 적당히만 힌트 던져도 다 찾더라.
\vspace{5mm}







\section{n수하지말라는 것에 대한 이야기}
\href{https://www.kockoc.com/Apoc/501678}{2015.11.19}

\vspace{5mm}

맨큐의 경제학 앞에 빌 게이츠나 농구선수가
대학을 포기하고 현업에 뛰는 걸 얘기하면서 기회비용을 언급하죠.
만약 본인이 대학에 안 가거나, 그리고 대학을 그냥 그런 데 나와도 더 가치있는 일을 할 수 있다면
당연히 삼수 이상은 할 이유가 없습니다.
\vspace{5mm}

그런데 문제는 삼수의 기회비용이 삼수를 하지 않는 기회비용보다 더 높은 케이스이죠.
즉 \textbf{삼수를 안 하면 더 문제인 케이스가 생각 외로 많다는 것}이겠죠.
\vspace{5mm}

저야 이런 걸 독려해보았자 딱히 수익 같은 것이 생기지 않기에 이해관계와 무관하게 말할 수 있는데
\vspace{5mm}

대학 안 가고 기술 배우면 된다..... 그럴 사람이면 \textbf{진작 중딩 때부터 그래왔겠죠}.
삼수보다 더 나은 도전이 있으면 공무원 시험 정도.
그런데 이것도 경쟁률은 수능 저리가할 수준이라는 문제가 있죠.
\vspace{5mm}

기술 얘기가 나와서 그러는데 기술도 '남이 돈을 지불해야' 의미가 있는 것이죠.
과거에 제자 한명이 걍 목공술을 배우면 대학 갈 필요가 없기에 한마디 했죠. 그럼 "누가 돈을 주는데?", "......."
기술도 어떤 분야 기술이냐, 수요가 많으냐, 그리고 진입해오는 라이벌이 있느냐 없는냐 다 따져보아야죠.
그리고 장사. 이거 정말 장사능력시험이라는 영역 새로 신설해야할 듯.
손님을 휘어잡고 기름칠 잘 하며 광고 신나게 해먹는 언변술 등은 국어영역,
공급수요에다가 재고납기 다 계산하고 수익율 계산하는 것은 수학영역,
거기다가 각종 행정적 규제, 법률, 세금 등을 탐구영역으로 하면 사실 이것도 공부할 것 많죠.
\vspace{5mm}

그것도 그렇거니와 가만히 분석해보면 재수해도 삼수해도 안 되는 게 아니라
재수, 삼수할 때 제대로 공부한 케이스도 그리 많지 않다는 게 문제임(...)
다시 말해서 올바른 공부 안 하면 n+1 해보았자 실패의 귀납법 완성이죠.
이거 어그로일지 모르는데 몇수 해도 안 되는데요... 이거 졸라 파고 들어가면 공부가 엉터리인 경우가 걍 대부분임.
그리고 이것도 알아야 함. n이 3 이상 넘어가면 실력이 늘어나는 게 아니라 허력(=실패를 조장하는 힘) 더 늘어나버립니다.
가끔 팔수 구수 해도 안 된다고 하는데 이 정도면 팔수 구수해도 안 되는 게 아니라, \textbf{실패하는 법을 5년 넘게 학습해버린 것임.}
\vspace{5mm}

걍 소주 노가리 까듯이 어이 살만하냐 하듯 분위기말하면 명문대에 꼭 가라할 건 아닌데
제가 명문대에 가라고 한다면 \textbf{일단은 합격을 해야 다년간 쌓인 체증이나 컴플렉스가 해소되기 때문}이라고 말씀드리겠음.
명문대 합격한다고 해서 효용이 그리 크냐 그건 아닌데, 사람이 달라지는 건 있음. 자신감이 생기고 나도 할 수 있다라고 느끼면서
내가 병신이 아니었구나, 단지 잘못된 방법이나 시스템을 타서 날 과소평가했구나, 날 손가락질하던 그 사람들이 문제였구나하는 것.
저거 명문대 대신 공무원이나 전문자격 시험으로 바꿔도 무방합니다만 아무튼 그렇다는 겁니다.
\vspace{5mm}

정말 눈여겨볼 사람들이 명퇴한 40대 이후임. 이 사람들은 님들의 부모님일 수도 있는데 잘 눈여겨보아야 함.
명퇴하는 사람들의 문제는 새로운 걸 학습할 능력이 없다는 거임. 회사형 인간은 쫓겨나면 답이 없음.
닭집 창업한다고 하지만 말이 좋아 창업이지 거액의 권리금 주고 프렌차이즈 가야죠. 아는 게 없으니까 $-$ 그 때부터 빚의 향연.
왜 개나 소나 닭집 창업하느냐 의심품겠지만 이건 당연합니다. 그나마 수요가 있고, 운좋으면 남의 치킨시장을 잠식해먹을 수 있으니까.
\vspace{5mm}

굳이 대학에 간다기보다도 "새로운 걸 학습하는 방법"을 알기 위한 측면에서 수험의 의의가 있는 겁니다.
10년 뒤, 아니 5년 뒤에 한국이 어떻게 변할지는 아무 것도 모름. 분명한 건 새로운 것을 배울 자세는 되어있어야 하고
시험이 없으면 우리가 시험을 만들어서라도 응시해야 할 판이죠.
\vspace{5mm}






\section{독학을 권유하는 이유}
\href{https://www.kockoc.com/Apoc/502091}{2015.11.20}

\vspace{5mm}

여기서 말하는 독학이란
$-$ 강의는 그냥 보충용, 가장 중시하는 건 "설명이 부족한 책도 자기가 설명을 하면서 읽기"
$-$ 국영수탐 하루에 수능시험 문제의 1.5배 풀기
를 말하는 것임.
\vspace{5mm}

일단 강의를 듣는 것은 혼자 책을 보다가 잘못 갈까 그래서 그런 건데
이 경우 간과하는 것이 있죠. 그럼 '강의'는 잘못된 게 없나. 유감스럽지만 많습니다.
\vspace{5mm}

독학의 단점은 시간이 처음에 많이 걸리고 불안하다는 것.
장점은 늦어도 6개월 버티고 나면 비로소 "나다운 것"을 알면서 자존심의 새싹이 돋아난다는 겁니다.
강사에 의존하면 끝까지 강의에만 빠져버리죠(이건 세뇌된 상태니까요)
아마 그 사람들은 첫키스를 할 때, 심지어 첫날밤을 보낼 때에도 인강을 틀지 않을까 싶을 정도입니다.
(그러다 강사가 정신교육하면 현자타임 오지 않을까)
\vspace{5mm}

학습의 목적은 \textbf{스스로 일어서는 것}입니다. 강의는 어디까지나 보충용이지요, 지금 잘못되어도 한참 잘못된 겁니다.
\vspace{5mm}

모르는 게 있으면 그걸 따로 메모하면서, 사이비 이론이라도 좋으니까 스스로 설명해보려고 하세요. 그래야 실력이 늘어요.
특히 수학은 한문제가지고 한달 정도까지 고민해 볼 수도 있습니다. 그런 고민과정에서 늘어나는 거지 강사가 찍어준다고 늘진 않아요.
강사가 찍어주면 처음에야 문제가 잘 풀리니까 오우 좋다 하겠죠. 세뇌론에서 말했듯이 쾌감의 원천은 \textbf{'논리적 사고의 중지'}거든요.
강의 듣고 기분이 좋은 건 모르는 것을 알아서가 아니라, '생각을 하지 않아도 되기 때문'입니다.
그게 독입니다, 나중에는 생각하는 방법을 몰라서 계속  강의만 의존합니다.
\vspace{5mm}

물론 생각을 하게 도와주는 강의도 없진 않죠. 하지만 그런 강의는 정말 찾기 어렵습니다.
\vspace{5mm}

지금 시작하는 분들은 도서관에 가서 역사책을 많이 읽어보세요.
어떤 책을 읽어야하느냐 물어보는 분들 많은데 저라면 서슴없이 역사라고 하겠습니다.
모든 책은 사실 모든 분야의 역사를 기록한 책입니다(미래도 가상의 역사죠)
역사는 기록자에 따라 조금씩 달라진다 해도 어찌되었든 FACT이기 때문에 다양한 함수관계가 응축되어있습니다.
그 역사를 읽고 해석하면사 생각하는 법을 배우는 것이죠.
\vspace{5mm}

책읽는 것도 내년 2월까지일 겁니다. 그 이후에는 읽고 싶어도 못 읽어요
\vspace{5mm}

물론 이건 평범한 케이스에 관해서이고
제가 학원가라고 하는 케이스는 무조건 학원 가세요.
혼자 공부할 수 있는 상태가 아니기 때문입니다.
\vspace{5mm}






\section{n수할 때 부모말을 들어야하나}
\href{https://www.kockoc.com/Apoc/503343}{2015.11.21}

\vspace{5mm}

n수가 문제가 아니라 전반적으로는 '참조'만 하지 들을 이유는 없음.
저 이야기는 부모님들이 자식의 인생을 다 책임져준다라는 걸 전제한 것인데
부모님들이 재산에게 '부양'의 대가로 재산을 물려줄 수 있을 지언정 \textbf{다른 거 할 수 있는 건 사실 없음}.
\vspace{5mm}

나이먹은 어른들이라고 하더라도 세상 어떻게 돌아가는지 아시느냐 그것도 아님.
그렇다고 철저하게 내 자식은 $\sim$ 하게 해야한다라고 조사하거나 고찰할 분도 없음.
보통은 다른 사람들 말 \textbf{: $\sim$ 카더라 수준을 보고 "얘야 $\sim$ 이게 좋다고 하던데"라고 하는 게 대부분임}
\vspace{5mm}

그리고 본인 인생은 본인이 사는 것임.
부모말 듣고 잘 되었다고 해도 자기가 잘했다고 미화할 것이고
부모말은 참조만 했을 뿐인데 나중에 망하면 이거 다 부모님 때문이예요라고 책임전가할 것임.
\vspace{5mm}

어떤 위로든 대화든 사실 \textbf{결과 외에는} 소용없습니다.
수험장삿꾼들이 정말 학생들 인생 신경쓴다고 생각함? 그거 개소리죠. 얼마나 벌 수 있을 것인가나 생각하지.
부모자식간도 비정함, 부모님들도 자녀가 말 안 들어처먹어도 \textbf{좋은 대학 가고 좋은 데 취업해서 두둑한 돈봉투 바치면 좋아하시지}
무능한 효자 효녀 좋아... 개뿔입니다. \textbf{입효도}를 누가 좋아하죠? 그건 누구라도 싫어할 것 같은데.
\vspace{5mm}






\section{여러가지.}
\href{https://www.kockoc.com/Apoc/504237}{2015.11.21}

\vspace{5mm}

\item 1. 좋아하는 교재일수록 독
\vspace{5mm}

참고서를 보는 건 궁극적으로는 '지적능력'을 키우기 위해서이다.
그러므로 자기가 부족한 것을 채워줄 수 있는 교재나 강의를 선택해야하는데
\vspace{5mm}

대개는 자기가 부족한 게 아니라, 자기가 '좋아하는' 교재나 강의를 선택한다.
국어를 잘 하는데 수학을 못 하는 친구는 수학을 공부해야겠지만 대개는 국어를 더 공부하려 하고,
문풀량이 부족한 친구들은 문풀을 많이 해야하는데도 계속 인강을 들으려고 하며,
반면 문풀이 충분한데 생각하는 법을 모르는 사람은 생각하는 방법에 관한 책을 읽거나 강의를 들어야하는데 관성대로 공부한다.
\vspace{5mm}

교재추천을 달가워하지 않는 건 여러가지 면이 있는데 이것도 한가지 이유다.
어차피 추천해보았자 자기가 좋아하는 교재만 본다. 그런데 이걸 알아야지, 자기가 좋아하는 교재는 \textbf{쓸모없는 교재라는 것을}.
봐야한다고 듣기는 했는데 보기 싫은 교재들이 실제로는 도움되는 교재들이다.
\vspace{5mm}

\item 2. \textbf{자존심}
\vspace{5mm}

수험생의 가장 큰 적은 자존심이다.
재수생까지는 자존심이 강하다. 그러나 삼수부터는 이게 무너지는데.
\vspace{5mm}

자존심이 적당히 무너지면 이것맘큼 좋은 게 없다. 그럼 선생이 하라는대로 자신을 개조하거나 적극적으로 변화하려하기 때문이다.
하지만 자존심이 너무 무너지면 폐인이 되어버리는데 역설적이지만 자존심이 높을수록 더 많이 꺼져버린다.
이 케이스가 서울대에서 참 많았던 것으로 안다(지금도 있겠지)
\vspace{5mm}

자기가 공부를 잘 한다라는 의식은 아무 소용이 없다. 뭘 하든 자기의 학습시스템이 얼마나 살상력이 좋으냐만 따져야지.
영원한 1등도 꼴등도 없는데 사람들은 과거 경험만 가지고 그걸로 자신을 과대, 과소평가하는 경향이 있다.
돌도끼 잘 던져 싸우는 족장님이 첨단 로봇이 전쟁하는 시대에도 위너가 될 리는 없지 않나.
\vspace{5mm}

\item 3. 성격
\vspace{5mm}

정말 수험에서 성공한 케이스는 \textbf{성격도 변하는 케이스}다.
물론 다 변한다는 건 아닌데 공부를 성공적으로 한다는 건 기존의 실패를 극복한다는 얘기고,
기존의 실패를 극복한다는 건 어리석은 행위를 낳은 성격을 고치는 데 성공했다는 이야기이기도 하다.
사람들은 수험에서 성공한 인간승리만 이야기하는데, 실제로 정말 중요한 건 성격이 바뀌는 것이다.
다만 이건 본인이 알아차리기가 어렵기 때문에 잘 모르는 것일 뿐.
\vspace{5mm}

개인적으로 조언주거나 할 때에 어떤 교재를 보느냐보다 강조하는 게 사실 이 대목이다.
성격이 바뀌면 사고방식도 바뀌게 되고, 사고방식이 바뀌면 문제를 푸는 스타일에도 영향이 있게 된다.
특히 수학의 경우는 절반은 성격이 먹고 들어간다. 성급해하거나 자뻑성이 강하거나 하는 친구들이 수학을 잘하기는 어렵다.
차분하고 꼼꼼하면서도 생각이 깨어있고 마무리를 잘 하는 친구들은 뭔 교재를 보든 점수가 잘 나온다.
거꾸로 말해서 수학공부를 한다는 건 수학뇌를 만들기, 수학에 어울리는 성격으로 바꾼다... 라는 것이다.
\vspace{5mm}





\section{개정 과정}
\href{https://www.kockoc.com/Apoc/505123}{2015.11.22}

\vspace{5mm}

고3 올라가는 사람은 모르겠는데
올해 시험 친 고3 이상이 그거 물어보면 걍 답이 없음.
\vspace{5mm}

수학 기출이 뭔 적중으로 푸는 게 아니죠. "사고력" 단련하려고 푸는 거지.
제2코가 들어간다 안 들어간다 행렬이 들어간다 안 들어간다 이게 중요함?
\vspace{5mm}

그런 개념들이 어떻게 변형해서 어떤 논리로 쓰이는가 하면 개정과정 상관없이 공부하는 거지.
고3 이상이면 어지간히 공부 안 한 사람 아니면 개정 전, 후 따질 필요 없음.
\vspace{5mm}

단 전반적으로 개정전이 난이도가 높음. 자, 여기까지 말해줍니다. 그 이상까지 다 떠먹여줘야하는지도 의문이고
이런 걸로 질문하는 사람은 자기가 스스로 내용비교도 안 해보았단 얘기인데(네이버 검색만 해도 뜬다) 이건 답 없는 것 아님?
그게 답답합니다. 이런 질문은 하지 마시고 찾아보세요.
\vspace{5mm}

그리고 저라면 제가 기존교육과정 배운 사람이면 기출 가리지 않습니다.
\vspace{5mm}




\section{수학문제풀이에 있어서 국어의 중요성}
\href{https://www.kockoc.com/Apoc/505651}{2015.11.22}

\vspace{5mm}

수학문제에 쓰이는 언어는 3가지이다.
\vspace{5mm}

\item 1. 문자, 수식
\item 2. 기하, 그래프,
\item 3. 개념, 성질
\vspace{5mm}

인강 강사들이 유행을 타는 게 1번으로 먹고 살던 사람은 2, 3번이 대세가 되면 날라가고
반면 2번으로 먹고 살던 사람은 1, 3이 대세가 되면 날라간다.
최근 추세는 2번을 죽이고 1번과 3번을 강화하는 쪽이다.
\vspace{5mm}

그런데 정말 중요한 건 3번이다.
왜냐면 1번과 2번은 사교육이 온갖 주입과 암기로 강화시킬 수 있기 때문이다.
그런데 3번은 그런 걸로도 먹히지 않는다.
\vspace{5mm}

최근 킬러 문제들은 반드시 저 (문자, 수식)이나 (기하, 그래프)를
국어적인 (개념, 성질)로 해석하는 작업을 반드시 거치는 걸 요구하고 있다.
물론 고수라고 하는 사람들은 그런 게 필요없다고 할지 모른다. 자기들은 그게 무의식적으로 체화되었기 때문이다.
하지만 평범한 사람들이 따라잡으려면 반드시 국어적 해석, 즉 개념, 성질로 포섭해보는 작업을 거쳐야 한다.
\vspace{5mm}

21, 29, 30을 아마 감각적으로 풀어댄 사람도 많겠지만
정말 똑바로 공부한 사람들은 머릿 속에서 교과서 목차$-$개념$-$성질을 검색하면서 어디에 해당하는지 분명히 짚었을 것이다.
그걸 똑바로 밟은 사람은 안정적으로 풀었을 것이다.
반면 아무리 수학문제를 풀었어도 저 3번 항목을 이용하지 못 한 사람은 힘들었을 것이다.
\vspace{5mm}

아마 저번부터 수학을 국어처럼 풀어라하는 것이 궁금한 사람이 있었을지도 모르는데 그게 이거다.
그리고 교과서를 보라는 이유도 그런데, 수학을 국어처러 공부할 수 있는 줄글을 그나마 명쾌히 쓴 경우여서이다.
가끔 수학고수라고 하는 사람들이 쓴 책을 보면 그 부분을 빼먹고 있는 경우가 많다.
\vspace{5mm}

국어적 해석 없이도 풀다보면 된다고 하는 사람도 있다. 그런데 그건 수화로 말이 통하는 것과 동급이 아닌가?
풀다보니까 된다라고 하는 건 그거 10명 중에 1명 정도만 먹히지 나머지 9명은 안 통한다.
다시 말해서 영어를 가르치지 않고 미국인과 눈빛과 손짓만 교환해도 의사소통할 수 있다라고 하면 누가 소통을 할 수 있나?
\vspace{5mm}

게다가 저 3번 항목을 제대로 짚어야
어째서 출제가가 그런 문제를 어떤 의미에서 냈고
풀이가 어떤 필연성을 갖게 되는지를 알 수 있다.
가끔 고수라고 하는 자들이 무조건 풀어준다...  하면서 풀이를 내고 환호한다 그러는데 그건 정말 바람직하지 않다.
왜 그런 풀이가 나오는지에 대한 '국어적 썰'이야말로 정말 중요한 것이다.
이런 과정없이 머리가 좋아서 푼다... 라고 하는데 과연 그런 방식이 한계가 없다고 보는가?
\vspace{5mm}

다른 이야기를 하면 수학문제를 풀 때에 꼼수나 스킬 이건 안 쓰는 게 좋다.
간단하다. 그런 꼼수나 스킬로는 풀 수 있는 게 '한정되어' 있어서이다.
교과서의 답답한 풀이는 소박해보이지만 사실 이거야말로 커버 범위가 넓다, 그리고 실수할 가능성도 거의 없다.
실수하는 케이스는 간단하다, '순서'를 안 지켰기 때문이다. 밟아아햐는 징검다리를 안 밟고 점프했다가 물에 퐁당 빠지는 것이다.
\vspace{5mm}

요새 학생들은 순수한 수리적 사고 $-$ 즉 식변환이나 그래프, 기하 패턴 파악능력은 좋다. 선행학습의 결과다.
그러나 처음 보는 문제를 어떻게 교과서상 개념으로 포섭할 것인가는 정말 약하다.
독서량이 부족해서 국어 실력이 좋지 않아서이다. 추상적 개념을 다룰 수 있는 지적능력이 부족하다보니 포섭력도 약한 것이다.
\vspace{5mm}

그래서 풀이 방향이 알려진 건 정말 빨리 푼다. 한데, 조금이라도 꼬아놓아서 풀이방향을 알 수 없는,
즉 개념과 성질로 차분히 생각해야만 접근 방법을 알 수 있는 건 건드리지 못 한다.
즉 문제의 독해력이 매우 부족한 것이다.
\vspace{5mm}

개념을 어느 정도까지 봐야하느냐라고 하시는데... 애국가 1절 암기하는 수준으로 줄줄 나와야한다고 본다.
교과서의 개념논리 흐름에 맞춰 풀이해야 낚시에도 안 걸리고 실수하지 않기도 하지만
그 개념들의 세부사항을 알아야만 낯선 문제를 수학적 개념에 포섭시켜 풀이방향을 잡을 수 있다.
\vspace{5mm}

수능수학이 과거보다 쉬워지는(?) 측면은 있다. 그런데 조심하자, 쉬워진다는 건 다루는 내용의 양적 측면이 줄어든다는 것이지
그 질적 측면도 쉬워진 건 아니다. 올해 시험도 과거에 비하면 쉽다고 하지만 풀이 방향은 다르다.
과거 2012년도까지의 문제는 발상 면에서 어렵다, 하지만 이것들은 아이디어를 잘 잡으면 논리가 없어도 맞았다.
논리가 없어도 맞다는 건 순서를 안 지켜도 일단 발상만 떠오르면 그럭저럭 답에 근접할 수 있었단 이야기다.
하지만 작년과 올해 시험은 '논리'를 강조하고 있다. 발상을 크게 요구하지는 않지만 순서를 못 지키면 풀 수 없거나
오답으로 끌려가도록 내고 있다
\vspace{5mm}



\section{그러니까 노력한 근거를 대야지}
\href{https://www.kockoc.com/Apoc/505846}{2015.11.23}

\vspace{5mm}

그런데 실증해보면 본인은 노력했다고 하는데 그게 '아닌 게' 보여서 말입니다.
\vspace{5mm}

추궁해보면 잠시 게임했다고 알고보니 하루에 2시간씩 게임한 게 6개월. 공부 시작한 건 고작 4개월인 경우.
살짝 연애했다고 하는데 더 추궁해보니까 그냥 1년의 절반 이상을 연애에 투자.
하여간 별의별 사례가 다 있음.
\vspace{5mm}

시간 검증은 그렇다 치고 그럼 풀어대 문제집 물어보면 횡설수설하다가 양치기 안 해도 되거든욧!
양치기 예외가 이번에 액상탄마님의 경우인데 수학은 그런 측면이 있었고 이 분은 사탐은 공전의 점수 나와서 오히려 검증된 경우임.
(수학은 공부하는 방법도 매우 중요하고, 특히 수리적 사고라는 건 1년만에 되지 않는다... 는 걸 보여준 좋은 경우인 것 같음)
\vspace{5mm}

작년 이맘때쯤인가 그래서 하도 짜증나서 제안한 게 일지 시스템.
그 이후로 교재 추천하는 질문은 몽땅 기각먹임.
일지 꾸준히 쓰는 사람들에게 조언 때려주고 어떻게 하나보았고
최근 10일동안 1명씩 다 읽어보고 확인해보았는데.
\vspace{5mm}

노력 했거든요 하는 거짓말까는 인간들은 걍 원양어선 타고 매일 참치나 먹고 살았으면 좋겠음.
일지 쓰는 사람이 다 합격하는 것은 아님.
그런데 잘 나온 과목과 못 나온 과목 비교해보면 \textbf{공부량에 확실히 비례함}.
수능시험이 그리 비합리적인 시험은 아님. 하긴 그럴 수 밖에 $-$ 인원수가 많기 때문에 우연성의 좌우도는 낮음.
교재 차이? 그런 건 없음. 어떤 교재를 보았냐보다도\textbf{, 기본교재'들'을 많이 보고 양치기했느냐}가 좌우함
강의 차이? 극단적으로 말해 강의 안 들어도 된다고 얘기해도 될 정도임.
그리고 작년말부터 공부한 사람들이 승률이 높음, 중간에 쉬는 기간이 있다고 하더라도 일찍 공부한 사람이 이김.
\vspace{5mm}

노력해보았자 소용없다는 사람들은 그럼 1년동안
일지 쓰면서 공부하면서 호출하고 질문하고 피드백해 본 뒤에도 그러나 보셈
수능시험은 그럴 수가 없음 시험임. 응시자가 도대체 몇명인데.
그리고 노력 까는 인간들의 문제는, 자기만 망하면 상관없는데
그런 개구라를 까서 다른 사람들까지 피해를 입힌다는 것임.
노력해도 안 되는구나 질려버린 후배들은
그래서 특별한 인강이나 교재부터 봐야하는구나 착각해서 해서 상술의 노예가 되어버리니까 문제임.
\vspace{5mm}

그런데 양치기만 해도 안 되는 경우는
\vspace{5mm}

국어 $-$ 독서량이라는 게 정말 많이 좌우함. 독서를 생활화한 애들은 사실 테크닉도 필요없음
수학 $-$ 수리적 사고, 문제 읽는 법, 그리고 문제 읽고 알고리즘 짜는 법은 알아둘 필요가(그런데 이거 정작 가르치는 인강 없음)
영어 $-$ 영어적 사고, 직독직해 필요함. 특히 전치사에 대한 감각이 강조됨
\vspace{5mm}

이라고 얘기하면 됨.
\vspace{5mm}








\section{개정 헤매는 분들을 위한 조언}
\href{https://www.kockoc.com/Apoc/506939}{2015.11.23}

\vspace{5mm}

원래 교재추천이나 코스 같은 걸 일일히 물어보는 것 자체가 안 좋은 결과를 낳는 경우가 많아서 그러나
그래도 불안해하는 분들을 위해 적습니다.
\vspace{5mm}

올해 시험을 친 고3을 포함한 n수생
\vspace{5mm}

\item 1. 기존 교재 버리지 마실 것 : 기존 교재가 난이도가 높습니다.
\item 2. 빠진 내용의 기출이라도 푸실 것 : 수학적 아이디어나 발상이 어디가는 것은 아닙니다.
\item 3. 개정과정에서는 확통, 기벡만 추가로 살필 것 : 그러나 크게 바뀐 건 없습니다. 재배치되었을 뿐이고 확률은 정수분할 정도만 보면 됩니다.
\item 4. 기출은 그냥 보던 것 보실 것 : 내년에 기출문제집이 나온다고 하더라도 올해 기출만 추가된 정도입니다.
\vspace{5mm}

올해 시험을 칠 고3
\vspace{5mm}

\item 1. 과거 교재 볼 필요는 없음, 님들 대상으로 한 교재를 보면 됨, 그 교육과정으로 7차 교육과정 따라갈 수 없음.
\item 2. 기출이 새로 나오면 보시길 바람 : 그런데 문제는 많이 푸세요.
\vspace{5mm}

그런데 둘 다 신경쓰실 건
\vspace{5mm}

\item 1. 고1수학의 비중 높이실 것. 올해 수능 기출 킬러는 스킨만 고2 수학이었지 실제 발상은 고1 수학 쪽이었습니다.
\item 2. 전과목 포함해 마지노선을 4월말까지로 당기시는 게 좋음, 개편 혼란 때문에 아마 여러 시행착오가 있을 겁니다.
\item 3. 시중교재와 교과서에 충실할 것.
\vspace{5mm}

1번의 경우는 이걸 왜 아무도 지적 안 하는지 모르겠습니다만 그 문제들이 변별력이 좋았던 건
다름 아니라 문제푸는 과저에 있어서 고1 수학을 물어보았기 때문입니다. 그 점에서라도 고1 수학 우습게 보지 마시길요.
이과의 경우도 범위에 직접적으로 안 나온다는 것이지, 어려운 문제를 만들 때 고 1 수학 원리 안 쓴단 말은 없었습니다.
2번의 경우는 그냥 겁주는 게 아닙니다. 올해도 5월 정도에 실력이 결정될 사람은 다 결정되었습니다.
안 믿다가 나중에야 네 말이 맞았다 그래 잘났다 어쩌구 반응이지만 이건 당연한 거예요.
방황하지말고 4월까지 졸라 달리시길 바랍니다. 노시려면 4월까지 마치고 노세요.
3번의 경우는 뻔한 소리가 아니라 지금 내년 출제가 어떻게 나올지 데이터가 없습니다. 예비평가 시행을 아직 안 했죠?
일차변환과 행렬이나 제2코사인이 빠졌다고 좋아할 게 아닙니다.
문과 수학은 오히려 늘어나버렸고(집합 명제 가지고 킬러내면 뭐 대박일 듯. 90년대 수능 한번 풀어보세요)
이과 수학의 경우도 외려 더 어렵게 낼 수 있는 여지가 커졌습니다.
\vspace{5mm}

그렇게 보자면 특정 출제 경향 따라서 공부한다... 아무 소용없습니다.
올해 실모 보면 된다 그거 경험해부신 분은 알지만 별로 의미없는 것 보았죠? 실모는 보충용으로나 먹혔죠.
또한 기출 역시 좋은 소스가 될지 모르나 이것만 가지고는 힘듭니다.
남들 보는 시중교재, 고1까지 포함해서 빨리 시작해서 더 많이 끝내세요. 그만한 대가는 분명히 보장됩니다.
\vspace{5mm}

쎈수학은 과거보단 쉬워졌습니다. 단 창의문제는 도전할만한 가치가 있고 경향이 다르니 풀어보시길
RPM은 바뀐 게 없죠(...)
블랙라벨도 크게 바뀐 거 없으나 개념서로서의 성격이 강해졌고 수험생들이 헷갈리는 부분을 잘 정리했습니다.
주목할 교재는 올림포스 평가문제집과 일등급 수학입니다. 새로운 변종을 경험해보실 수 있을 겁니다.
올림포스 평가문제집과 일등급 수학은 다 풀어보시는 게 좋을 것입니다.
\vspace{5mm}

그렇게 보자면 지금 시험 때까지 남은 기간은 '모자라'지요.
내년 수험생들은 시간부족에 더 시달리겠네요.
\vspace{5mm}

자, 이 정도로 말씀드리니 쓸데없는 질문이 없길 바랍니다.
\vspace{5mm}







\section{모의고사만 줄창 푸는 게 안 좋은 이유}
\href{https://www.kockoc.com/Apoc/509601}{2015.11.25}

\vspace{5mm}

모의고사를 내는 사람이건 푸는 사람이건
\textbf{해당 모의고사들이 특정 경향에만 치우쳐있다}는 것을 절대 모릅니다.
\vspace{5mm}

이른바 \textbf{평균의 함정}이죠. 그게 모의가 나빠서만은 아닙니다만
자기들이 생각하기에 \textbf{출제빈도가 높은 문항을 우선순위로 게재한다}는 게 문제임.
출제빈도가 높은 것을 풀면 그 다음부터 문제가 쓱싹 잘 풀리니까 이게 정말 효과가 좋은 걸로 생각하죠.
하지만 이건 결과적으로 우물 안 개굴짱을 만드는 것이죠.
\vspace{5mm}

그런데 수능은 \textbf{평균에서 꼭 벗어난 걸 내거든요.}
\vspace{5mm}

모의가 미래지향적이어야하는데 실제로는 과거지향적인 경우에 불과한 게 많다가 함정입니다.
물론 저자들도 적중시키고 싶어는 하죠. 그런데 그게 될 리가 있나.
그래서 수험생 입장에서는 모의고사만 믿는 건 현명한 전략은 아닙니다요.
\vspace{5mm}

작년에 모의고사에 비판적인 것을 보고 저 녀석 괜히 그런다 하실 분들도 있었을 건데
근거없이 저런 주장하는 건 아닙니다요.
\textbf{미래에 대비하고 싶으면 남들이 안 가는 방향, 커버하지 못 하는 것까지 다 준비해두는 수밖에 없습니다.}
더군다나 다수의 확신이라는 건 매우 높은 확률로 다수의 패배를 의미하기도 해서리
\vspace{5mm}

시중 기본서는 당장 수능과 거리가 멀어보이긴 하는데 이것만큼 미래를 준비하기 좋은 책은 없습니다.
반복하다보면 행간의 내용이 다 드러나죠. 그런데 모의고사는 그게 안 되는 것입니다.
\vspace{5mm}

나는 열심히 노력했는데.... 의 경우가 안타까운 것은 지나치게 다수의 확신을 따라가더라 그것입니다.
ABC에서만 나올 거라고 공부했는데 DEF에서 나왔다고 평가원을 욕하죠. 하지만 원래대로라면 A$\sim$Z 다 공부했어야하는 건데요.
올해 시험 치고 복기해보라고 한 기한 다 지나서 적는다면, 정작 자기가 집중적으로 공부한 것이 쓸모가 없다는 걸 느껴보라는 얘기였습니다.
\vspace{5mm}

내년도 4월 말까지 공부 해놓고 남들 어떻게 하나 보세요.
그리고 남들이 안 하는 것을 비밀병기 준비하듯 공부하면 됩니다.
\vspace{5mm}





\section{교재 : 한국사 강의없이 틀잡고 싶다면}
\href{https://www.kockoc.com/Apoc/511903}{2015.11.26}

\vspace{5mm}

역사신문
\vspace{5mm}

http://www.yes24.com/24/goods/2987183?scode=032&OzSrank=4
\vspace{5mm}

남경태의 종횡무진 시리즈
\vspace{5mm}

http://www.yes24.com/24/goods/17403469?scode=032&OzSrank=1
\vspace{5mm}

참고로 동서양사도
\vspace{5mm}

http://www.yes24.com/24/goods/17403476?scode=032&OzSrank=2
\vspace{5mm}

http://www.yes24.com/24/goods/17403471?scode=032&OzSrank=3
\vspace{5mm}

를 보면 지엽은 모르지만 틀은 잡을 수 있음.
\vspace{5mm}

그런데 개인적으로는 역사신문 시리즈만 구매를 권하고 나머지는 도서관에서 빌려볼 수 있으면 빌려보는 걸 권하겠음.
\vspace{5mm}

역사신문 시리즈의 장점이 절대 지루하지가 않다는 것이죠. 서술 방향은 오히려 좌파 쪽이긴 한데 술술 읽힐 것입니다.
\vspace{5mm}

그러고보니까 이원복의 먼나라이웃나라도 볼만하지 않을까 싶긴 한데.
\vspace{5mm}

그냥 강의를 듣는다면 사설강의 들을 필요 없이 EBS만 가는 걸 권함.
\vspace{5mm}

그런데 그 강의조차 듣기 싫다 난 걍 책 읽을 거야라고 하면 동네서점에도 있을테니 참조해보시길.
\vspace{5mm}






\section{몰입한 상태}
\href{https://www.kockoc.com/Apoc/512266}{2015.11.26}

\vspace{5mm}

세뇌론에 적겠지만 수험장에서의 멘탈을 이야기하는 분들이 있어서 그냥 미리 간략히 적으면.
\vspace{5mm}

A $-$ 몰입을 해본 적이 없음, 그래서 신유형이 나오면 긴장하고 평소보다 능력치가 떨어짐
B $-$ 평소에는 느슨하지만 몰입을 할 수 있음, 시험에 임할 때 자신을 잊고 몰입 상태에서 문풀을 시작함.
\vspace{5mm}

멘탈 붕괴된다라고 하는데 이거 본질에서 벗어난 거임.
낯선 상황에 긴장하는 건 누구나 마찬가지입니다.
그런데 A는 그 긴장에서 당황스러운 상태로 가는 반면, B는 바로 자기만의 몰입된 상태(일종의 세뇌된 상태)로 가죠.
\vspace{5mm}

우리가 호흡을 의식하면서 하는 건 아니죠. 무의식적으로 합니다.
우리가 말을 할 때에도 사실 생각없이 합니다. 일일히 생각하고 말하진 않죠.
아니 일상 영역에서의 행동은 의식하고 하는 게 없어요.
\vspace{5mm}

한 분야에서 공을 들인 고수들은 그 분야를 무의식적으로 해냅니다. 일일히 신경쓰지 않으니 피로감도 덜 하고 그래서 더 잘 해냅니다.
이 상태에 도달해본 적이 없거나 이걸 의식하지 못 하면 당연히 노오력에 회의감을 품게 될 것입니다.
그러나 이런 상태 $-$ 일종의 임장감이라고 하는 것에 도달해 기적적으로 뭔가 완수한 사람은 다른 분야도 마찬가지인 것을 알게 되죠.
\vspace{5mm}

이런 상태에 도달하려면 개인별로 차이가 있긴 하지만 무수히 많이 반복하는 수 밖에 없습니다.
비상 상태에서도 관련된 내용이 정확히 입에서 튀어나오도록요.
강도가 칼을 목에 대도 \textbf{그 내용이 다 정확히 나오도록 학습되어있어야 비로소 공부}입니다.
\vspace{5mm}

이게 멘탈과 관계있느냐. 글쎄올시다, 덜덜 떨고 파랗게 질리더라도 그 모든 지식을 복기할 수 있는 정도까지 가야 공부입니다.
그럼 이게 비현실적인 노력... 까진 아닐텐데 말입니다. 비상 상황에서도 정확히 임무를 수행하는 직업인들이 있을텐데요.
사실 제 입장에서는 그렇습니다. 멘탈 타령하는 사람들은 훈련을 덜 하고 싶어하는 사람으로 밖에 안 보이죠.
\vspace{5mm}

기출을 왜 여러번 푸느냐. 아는 것 반복할 필요 없지 않느냐.
이 때 피식 웃어주는 이유가 그겁니다. 이런 말을 하는 사람들은 공부를 그냥 데이터의 저장으로만 생각하는 거죠.
어떤 상황에서든 침착하게 학습한 바를  행하기 위해서 반복하는 것입니다.
\vspace{5mm}

시험 때 긴장 안 하고 싶다. 유일한 방법은 '실성'하는 거죠.
\vspace{5mm}

그리고 \textbf{약}
\vspace{5mm}

제가 싫어하는 사람이라면 권하겠습니다.
차라리 심호흡을 하고 자기최면을 하는 걸 익히세요.
\vspace{5mm}





\section{인터넷 강의 문제점}
\href{https://www.kockoc.com/Apoc/514702}{2015.11.28}

\vspace{5mm}

인강 없던 시절에도 그런 것 없이 서울대 가는 사람은 잘만 갔습니다.
개인적으로는 인강의 문제는 상당히 많다고 여기는데
근본적인 건 \textbf{"혼자 공부할 수 있는 힘"을 완전히 날려먹는다}는 것이고
그리고 거짓말이 상당히 많다는 것입니다.
\vspace{5mm}

일단 개소리 하나를 저격하죠 인강이 다 다뤄준다?
웬만하면 저격 안 하겠는데 저건 정말 개소리라서 한마디 합니다.
\vspace{5mm}

헛소리입니다. 저도 과거에 3년동안 듣고 정리해보았는데요.
일단 인강이 다 다뤄준다라고 느끼는 건, 본인이 \textbf{'책을 읽을 줄 몰라서'} 하는 이야기입니다.
수학에서의 문풀 사고법 빼고 나머지는 \textbf{모두 책에 있습니다.} 아니, 책에 더 많이 있다고 말씀드리지요.
궁금하면 님들이 신나게 필기해보신 다음 시중교재와 비교해보세요. 수능에 출제될 수 있는 건 거의 차이가 없습니다.
\vspace{5mm}

책을 읽는 게 힘들거나 아예 몰라서 처음에 입문용으로 듣는 건 추천할 만 합니다.
하지만 틀을 잡으면 바로 본인이 책을 읽고 스스로 정리하고 \textbf{추론}해야합니다. 그렇지 않으면 절대 늘어날 수 없습니다.
어디든 상담해보면 인강 때문에 흥한 경우보다 \textbf{망한 경우가 더 많습니다}.
그런데 흥한 경우는 혼자서도 독학이 가능하고 책을 읽을 수 있는 능력이 있습니다.
\textbf{무엇보다 조기교육을 받은 케이스가 많고 집안환경이 좋습니다(왜 이런 건 언급하지 않는 걸까)}
반면 집안환경이 좋지 않고 조기교육을 받지 않았는데 인강으로 가서 잘 되는 케이스는 없네요.
\vspace{5mm}

인강의 장점은 들을 때는 소화가 잘 된다는 것입니다. 단점은 소화는 잘 되는데 계속 배가 고프단 것이죠.
저런 거 그냥 필기노트로 해서 수학독본처럼 그냐 서술체로 가는 게 수험생의 시간을 단축시켜주지 않을까도 싶습니다.
예컨대 EBS에 보면 강의자막을 HWP 파일로 정리해주죠. 필기만 제공된다면 차라리 그것을 파는 게 훨씬 나을지도 모릅니다.
\vspace{5mm}

더군다나 인강의 문제점이 심하다고 보는 이유는 두가지가 있습니다.
\vspace{5mm}

첫째, 매우 안 좋은 수학책이 있습니다.
내용은 그럴싸한데 실제론 기본사고를 말아먹게 만드는 책이죠.
그런데 이 책, 온갖 \textbf{인강을 짜깁기}했더군요. 짜깁기한 저자도 문제가 많지만 이건 인강 내용 그 자체도 문제가 많다는 겁니다.
(그렇다고 잘 짜깁기한 것도 아닌 것 같지만)
\vspace{5mm}

둘째, 강의를 바쁘게 한다고 하는데 그럼 연구나 개발은 언제 하느냐는 것입니다. 매년 강의가 똑같고 심지어 틀린 말 하는 사람도 있죠
새롭다 신박하다 하는 풀이라는 것도 사실 학문적 체계가 의심되는 경우가 많으며 사실 수능에는  쓸모가 없습니다(...)
최근 5년간 수능에 인강이 큰 도움이 되는 경우가 있기나 한가는 의문이 들더군요.
\vspace{5mm}

흔히 이런 이야기를 하죠. x등급에서 x등급으로 올린다... 이런 광고도 참 의문이 많습니다.
가정환경이 좋고 조기교육을 받았으며 체계가 잡힌 사람들을 올리는 건 솔직히 어렵지 않습니다. 그 경우는 강의 없어도 올라갈 사람은 올라가죠.
하지만 정말로 사고방식이 막장이고 안 좋은 환경에서 자란 경우를 올린 케이스는 사실 본 적이 거의 없습니다(...)
원래 이런 것을 연구해보는 게 취미(?)이기도 해서 조언해주고 캐리해주면서 느끼지만
이거 단순히 공부의 문제가 아니라 푸념도 들어주고 쓴소리도 거꾸로 적절히 해주고 \textbf{성격} 고쳐야 하는데다가
가정\textbf{환경} 전체를 다 뒤집으면서 '사람' 자체를 바꿔야한다고 느끼는데 과연?
\vspace{5mm}

저런 걸 구체적으로 안 들어가본 사람이 머리 타령하겠죠. 아닙니다, 뭘로 가든 성격이 문제입니다.
다운증후군 환자거나 정말 뇌를 다쳐서 맛간 경우가 아니라면, 온라인 게임이나 야동이 가능하면 머리의 문제가 아니라 성격의 문제입니다.
그 다음은 공부환경의 문제죠. 그리고 이 점에서 인강이 또 문제가 됩니다. 인터넷 접속을 유도하거든요.
\vspace{5mm}

그리고 한가지만 그냥 제가 발견한 사실 적죠.
\vspace{5mm}

상담이든 대화이든 해보면 참 유약하구나라고 느껴진 케이스들이 많고 다 추적해보면
독서는 빈약한데 정말 모든 것을 '사교육'으로만 의존했고 그렇게 키워진 케이스입니다.
겉으로는 똑똑한데 급소만 찌르면 그냥 무너질 수 있는 사람들이지요.
이런 것 보면 한심하고 짜증나서 '독서'를 하고 스스로 공부하라고 하는 것입니다.
그리고 컴 접속 줄이고 도서관에 가라는 이야기입죠
\vspace{5mm}

사교육이나 인강으로만 머리가 채워진 친구들이 솔직히 과연 정상일까, 실제로는 멀쩡해보이는 '환자'들이라는 게 제 개인적 평가입니다.
스스로 읽고 생각하고 자문자답해본 친구들은 느립니다. 정말 튼튼하게 자아가 성장하고 있고 쓰러져도 다시 일어날 수 있죠.
그러나 선생이 가르쳐주는 걸 따라하는 것만 시도한 친구들은 스스로 일어날 수 없습니다. 입시까지는 그게 운좋아 먹히더라도 그 다음이 문제겠죠.
이 친구들 대화하는 것 대사 분석하면 "트라우마"에 비견되는 명제의 반복을 계속 확인할 수 있다는 것도 적어보겠습니다.
대화하면서 이 친구들을 치유하는 건 그 명제를 다른 명제로 바꿔주거나, 아니면 그 근원 자체를 해체시키는 것이지만 그건 다른 문제겠죠.
\vspace{5mm}

인강을 듣고 싶으면 사설 신청하지말고 EBS 입문강의나 열심히 들으세요.
그 다음 시중교재 풀면서 잘 안 되는 단원, 이해 안 가는 내용, 문제들을 따로 메모해놓고
그걸 스스로 한번 해결해보는 걸 해보십시오. 시간이 걸리더라도 이걸 스스로 해결해야 다른 것도 해결됩니다.
오래 고민한 다음에 인강으로 해결(될 건지는 모르지만)하는 건 권합니다. 스스로 고심하고 아파보아야 실력이 올라가지
남이 하라는 대로만 해서는 올라가는 게 아니라 올려지는 것입니다.
\vspace{5mm}

콕콕 내에서도 +1수를 권하거나 적극 하라는 경우는 두가지입니다.
첫째, 어떤 환경에서도 당사자가 공부할 수 있다고 생각하는 경우 : 사실 이 경우는 제가 더 이상 조언할 필요도 없을 정도입니다.
둘째, 여러모로 가능은 한데 본인이 뻘짓을 해서 말아먹은 경우 : 그럼 이건 뻘짓을 안 하면 되는 게 아닌가?
\vspace{5mm}

사실 그 외에는 알아서 하라고 싶을 정도입니다.
단, 모군은 \textbf{그 좋은 환경 버프업을 받았으면 하늘에 감사할 것이지 뻘소리는 적당히 했으면} 좋겠습니다.
가볍게 쓰는 글이라도 \textbf{그런 글에 현혹되어 정말 인생 날라가는 애들도 많다}는 사실을 알아두세요.
올해도 상담쪽지 받고 조언해주면서도 또 확인했지만 \textbf{대책없이 인강만 듣다가 망한 케이스가 90$\%$입니다요.}
혹자는 무슨 무료데이터... 개소리입니다. 일지쓰라고 하는 건 수험생들이 "저 열심히 공부했는데요"라고 징징대니까
그럼 얼마나 열심히 공부하나보자, 네가 공부하는 기록 써보고나 징징대라 하는 차원이 강하지, 사실 그 외는 도움될 것도 없습니다.
\vspace{5mm}

여기가 모처의 잘못된 사상이나 습관에 전염되는 건 강경하게 거부하는 바입니다.
돈은 돈대로 날리고 호구는 호구대로 되고. 이거 제 알 바는 아니라고 쳐도 너무 눈에 밟힙니다. 꼴불견이지요.
\vspace{5mm}

+ 학원가야하는 케이스 있죠.
\textbf{"혼자서 정말 공부가 안 되어서 인터넷 접속하는"} 그런 경우입니다. 컴 없는 데에서 정말 순수히 다른 친구들 따라 공부해야합니다.
그런데 학원조차도 \textbf{'집단적인 학습 분위기'} 그거 믿고 가는 겁니다.
이것도 독학가능하면 도서관으로 대체할 수 있습니다.
돈을 주고 '\textbf{환경}'을 구입하는 건 적극 권할 일입니다. \textbf{환경}은 상관관계가 뚜렷하니까요.
\vspace{5mm}

+ 다시 말씀드리지만 교재나 인강이 문제가 아닙니다.
먼저 환경, 그 다음 \textbf{습관}을 바꾸세요. 그리고 반드리 \textbf{라이벌}을 잡아야 합니다.
공부에 유리한 것만 늘리고 불리한 것은 제거하는 환경을 잡아야 공부가 됩니다.
그 다음 반드시 절박한 심정으로 습관도 바꿔야 합니다. 습관을 고치면 성격도 따라갑니다.
마지막으로 라이벌을 정해야합니다. 라이벌이 수험 스트레스의 진공 청소기입니다.
\vspace{5mm}

+ 그리고 다시 말하는데 남자 수험생들은 자기를 혐오하길 바랍니다.
이 색기는 망했다... 라고 하는 남자 1순위는 나르시스트입니다. 나르시스트가 극성맞은 엄마와 결합하면 마마보이가 되죠.
남자는 승부에서 지고 깨지고 다치고 심지어 기절까지하고 그러다 일어서고 싸우고 무기 바꾸고 흉터 생기고 그러면서 성장하는 거지,
나 잘 생겼다 존잘 이딴 드립치는 것이 아닙니다. 자기를 사랑하는 순간 더 이상 진화할 수 없고, 그 순간부터 내리막길인 것입니다.
왜 충고해줘도 안 먹히고 계속 그 지경인가 하는 케이스들 보면 공통점이 \textbf{못난 자기 자신을 극도로 사랑합니다}.
\vspace{5mm}

+ 현역으로 척척 붙어서 SKY 가는 친구들이 사회에서 잘 나갑니다.
이 친구들은 나르시시즘에 안 빠지거든요. 자기를 사랑하지 않으니까 과거의 자아를 버릴 각오가 되어있고 좋은 게 있으면 바로 갈아탑니다.
학습에서 중요한 건 '환경과 습관'이라는 걸 체득한 사람들입니다. 티는 안 나지만 성과는 무시무시하죠.
그러나 n수생들은 인간문화재도 아니고 자기만의 '전통'이라는 걸 고수하고 지키려고 합니다. 끝까지 그걸 안 버리려고 하죠.
자기 방식대로만 가겠다, 고집센 내 자존심을 지키겠다, 내가 했던 방식으로만 갈테야....
실패의 원인이 자기 자신이라는 사실을 인정하지 않습니다. 그래서 n이 커지면 더욱 그걸 인정하기 싫어서 포기조차도 정당화합니다.
\vspace{5mm}

+ 더 무서운 건 여학생들은 저런 나르시시즘은 없단 겁니다.
여자들은 남자들과 달리 출산 때 피를 보기도 하지만, 우선 '화장'을 하는 게 어색하지 않죠.
그건 자기를 사랑해서가 아니라 타인들을 현혹하고 유혹하기 때문입니다. 그래서 남자와 정말 마인드가 다릅니다.
필기시험에서 여풍이 강해지는 이유가 이 때문일 것입니다. 여자들의 단점이라는 정보와 체력 부족은 인터넷과 관리로 해결되니까요.
\vspace{5mm}




\section{영어에 관해서}
\href{https://www.kockoc.com/Apoc/518404}{2015.11.30}

\vspace{5mm}

형식이 다르다하지만 TOEIC과 TEPS가 도움이 됩니다.
국어는 PSAT/LEET가 정통코스가 되어간다고 생각해서 이 남자, 아니 이 영어는 어떨까 보았음.
\vspace{5mm}

최근에 영어가 어떻게 나오나 궁금해서 3일(...) 정도 공부했는데
\textbf{만만해보이는 토익조차도 오랜 세월동안 진화했다}는 것을 느꼈습니다.
3일 정도로 패턴화될 수 있을지는 모르겠지만
여러가지 차이를 순수하게 느낄 수 있었습니다.
\vspace{5mm}

우선 L/C조차도 간단한 추론, 순발력을 꽤 요구합니다. 호주발음이 듣기 개판이다 그게 문제가 아닙니다요,
문항과 답변이 직접 대응이 아닙니다.
\textbf{청해→ 판독 → 해석 → 여러가지 명제들 추론 → 답 고르기}
이와 같은 과정을 짧은 시간 내에 거쳐야합니다.
L/C 파트 2부터 파트 4까지 쭉 이어지는 과정이더군요.
\vspace{5mm}

R/C 쪽도 만만치 않습니다. 지문은 매우 쉬운데 선지에서 꼬아내던데
이 역시 위와 같은 영어적 추론을 해야하기 때문입니다.
\textbf{읽기 → 판독 → 해석 → 여러가지 명제들 추론 → 답고르기}
거기다가 복수지문들까지 보니까 대충 190문제부터는 이미 머리가 지쳐있음.
\vspace{5mm}

그리고 느낀 바 $-$ 아, \textbf{빈칸추론의 핵심이 저기 있었군.}
영어가 힘든 건 간단합니다. 단순히 해석을 넘어서 \textbf{"영어 자체로 사고해야하기 때문"}입니다.
그런데 아이러니컬하게도 이에 근접한 게 필요는 없지만 호기심에서 다시 쳐보았던 토익 시험에서 엿보이더군요.
어제 그래서 시험을 마치고 서점에 가서 토익과 텝스 교재를 죽 훑어보았습니다.
성인 대상으로 하는 시험이다보니 교재 질은 매우 좋더군요.
책이야 유명한 건 다 좋아보였습니다. 패턴 정리가 꽤 잘 되어있었으니까요.
\vspace{5mm}

올해 시험에서 영어 망쳤는데 다시 시작하실 분들은
어차피 대학가서 치러야 하는 시험이니 저런 시험들 응시를 해보시길 바랍니다.
냉정히 말해서 그냥 수능 수준의 영어가 도움이 되나.... 최근 강사추천도 올라오곤 하지만 그런 행태는 제가 혐오하는 것이고
기본적으로 English 자체로 사고한다는 점에서는 어휘나 지문 수준이 다르다고 해도 저런 공인영어시험을 공부하는 게 낫다는 결론이 되겠습니다.
\vspace{5mm}

그런 다음에 괴서 성문종합영어에 나온 명문들을 주로 읽어보시면 되겠죠.
읽어보란 이유는 간단합니다. 영어권 지식인들이 생각하는 건 우리 조선인들이 생각하는 것과 달라도 한참 다르기 때문입니다요.
여러 학생들이 골치아파하는 빈칸추론은 말이 빈칸이지 실제로는 '주제추론'입니다.
그런데 English Writing의 경우 topic을 부각시키는 매우 객관적이고 체계적인 방법론이라는 게 있습니다.
사실 이걸 공부하면서 빈칸추론 문제를 풀 때 선지를 보지 말고 직접 주제를 추론한 다음 그 다음 선지를 봐야 안 낚이지
그냥 보면 거의 낚인단 이야기죠.
\vspace{5mm}

뭐... 내년에 +1수 하실 분들은 국영수탐 모두 수능을 넘는 상위과정 다 공부해야할 것이다란 이야기가 되겠습니다.
정말로 상위권이 되고 싶다는 분은 성문종합영어 잘 도전해보세요. 지금 날린다는 영어강사들도 젊은 시절에는 이거 공부 안 한 사람은 없음.
재밌는 건 다 성문종합영어 까는 사람들도 나중에 자기가 강의하는 내용이 성문스럽게 변하고 있다는 것.
\vspace{5mm}










\section{빚개념에 대해서}
\href{https://www.kockoc.com/Apoc/518747}{2015.11.30}

\vspace{5mm}

5수생이 있다칩시다.
\vspace{5mm}

보통 이런 경우 어떤 관념이 문제나면
자기가 날려먹은(?) 4년만큼의 본전을 챙겨야한다는 \textbf{보상심리} 라는 게 있습니다.
그래서 목표치를 더 높게 잡으면서 자기가 수험고수이니 더 많이 하겠다 그래서 꼭 성공해야한다는 강박관념이 있죠.
\vspace{5mm}

사실 생각해보면 별 의미없는 자기학대에 불과합니다.
목표치를 높게 잡는다고 해보았자 그 4년이 빛나는 것도 아니죠.
4년동안 공부했다면 당연히 구력은 있습니다, 그러나 '실패'도 학습되었을 뿐더러 '해결되지 않은 원인'이 있단 거죠.
\vspace{5mm}

그럼 어떻게 해야하느냐.
\vspace{5mm}

일단 4년은 잊어버려야합니다. 그냥 4년동안 병원생활, 식물인간, 징역살이, 외계인에게 납치... 당했다고 생각하는 편이 나아요.
그 4년은 경제학적으로는 매몰비용입니다. 뭘 하더라도 사실 복구는 못 해요. 심지어 성공한다 하더라도 4년이 의미있느냐 그건 아닙니다.
다들 이런 매몰비용을 복구하겠다고 목표를 무리하게 잡는 걸 넘어 학습방법도 터무니없는 걸 선택하기 때문에 실패하는 겁니다.
\vspace{5mm}

저 4년은 안 돌아옵니다.
내년에 시험치는 사람이면 겸손하게 자기가 고3과 똑같다고 여기세요.
\vspace{5mm}

만약 개인의 성찰과 반성, 그리고 기본 지식을 쌓는 과정에서라면 유의미하다고 반문할 수 있긴 하겠죠.
그러나 이 경우 손해는 더 큽니다. 4년동안 해서도 되지 않는 \textbf{실패도 따라오기 때문}입니다.
시험을 여러번 쳐도 안 되는 이유는 공부가 부족한 것도 있지만, \textbf{실패하는 패턴을 반복하는 게 가장 큽니다}.
학원에서는 공부하는 방법이나 지식을 전수해주겠지만, \textbf{학생 개개인의 실패 패턴을 지적해주거나 잡아주진 못 합니다.}
본인의 과제죠.
\vspace{5mm}

하지만 이걸 하는 건 자존심을 포기하는 것에 근사하기 때문에 혼자 하기 힘들 수도 있습니다.
달리기를 잘 하는 친구에게 너는 달릴수록 불행해지니까 달리지 마라고 하거나
아주 얼굴이 예쁜 여학생에게 자네는 얼굴이 불행의 근원이니 차도르를 쓰고 알라후 아크바르를 외치도록 하는 것과 동급입니다.
하지만 그런 자존심을 포기하고 여태껏 살아온 방식을 과감히 바꾸지 않으면 실패는 또 반복되죠.
\vspace{5mm}

빚을 못 갚으면 파산신청하고 갱생하는 게 낫습니다.
내년에 다시 시험 응시할 분은 과거는 싹 잊으세요. 과거에서 챙길 건 오직 교훈, 그리고 자기의 실패하는 패턴에 대한 반성입니다.
그래서 다시 시작해야 하는 겁니다.
\vspace{5mm}

과감하게 구식무기를 버리고 신식으로 갈아타면서 자기를 잊는 사람은 살아남겠지만
계속 한탄만 하면서 자기를 너무 사랑하는 사람은 또 실패합니다요.
\vspace{5mm}




\section{최근의 동태에 대한 비판입니다만}
\href{https://www.kockoc.com/Apoc/518822}{2015.11.30}

\vspace{5mm}

강사 개개인이 착하건 악하건 관계없이
누구 강의가 최고예요... 라는 식의 글이 올라오는데 그건 바람직하지 못 한 행태입니다.
최소한 그게 검증되었다라고 볼 근거도 없지 말입니다.
\vspace{5mm}

여기서 믿을 수 있는 건 일지 꾸준히 쓰고 자기 성적 공개한 케이스이지
나머지는 믿을 것도 없습니다.
\textbf{사이트가 가장 힘들 때에도 꿋꿋이 활동하고 자기 공부하면서 소신껏 밀어붙여 올라간 케이스면 몰라도}
나머지는 솔직히 아니라고 생각하는데요.
\vspace{5mm}

그럭저럭 사이트가 잘 버티고 하한선 찍고 상승세 찍을 때나 와서 모 강사가 최고예요라는 행태는
그 강사가 제가 좋아하는 사람이건 안 좋아하는 사람이건 바람직하지 않은 행태죠.
아무리 좋은 강사도 맞는 사람이 있고 안 맞는 사람이 있습니다.
사설강의가 뭐 한두푼도 아니고. 듣고 싶으면 여름 때 정말 부족한 것만 골라 들으세요.
\vspace{5mm}

이제 또 강의 교재 홍보철이 시작되죠.
이 사이트도 졸라 모니터링당할 거예요. 자기 영업방해되는 글 있나없나
설마... 돈에 눈먼 사람들은 별 짓 다 합니다만요.
유감스럽지만 가장 한적해야 할 이 사이트조차도 제 눈에는 이미 \textbf{자본의 마수가 뻗쳐있습니다}.
\vspace{5mm}

약 1년 전인가. 일지조차도 시비 걸었던 사람들이 누구인지 전 기억하고 있죠(시비걸 이유가 없을텐데)
심지어 그 때 학습론도 하도 시비걸어대고 해서 짜증나서 지웠습니다.
그 사람들 누군지 몰랐다는 건 오산. 걍 싸우기 싫어서 냅둔 겁니다. 어떤 성향이고 어떻게 활동했는지 똑똑히 기억합니다.
어차피 인생 말아먹을 인간들이라서 말도 안 한 거예요.
그런데 그 사람들이 왜 '제가 영리사업도 안 취하고 그냥 애들 불쌍해서 이렇게 해라 지시하는 것'을 시비걸었을지는 다 알고있지 말입니다.
\vspace{5mm}

몇몇 영리활동 취하는 사람들은 일지조차도 무슨 목적이다 생각하실 건데 그딴 것 없습니다.
공부하는 패턴이든 방법은 님들이 상상하는 이상으로 다 알고 있으니까요. 해서는 안 되는 공부법까지도 다 정리했구만 무슨.
다만 질문하고 상담하려면 일지는 쓰는 성실한 사람이 아니면 안 받아준다는 건 당연합니다.
올해 실패한 사람들도 있지만(아마 조만간 연락받고 또 소통하겠죠) 대부분은 생각한대로라서.
이런 걸로 돈벌려는 사람이야 딱 한 사람만 집어서 성공사례라고 하면서 바로 영업질 들어가는 추한 짓이나 벌이겠죠.
그런 건 영 관심이 없지 말입니다. 핏덩이들 어떻게 성장하나 그거 보는 게 재미지 뭔 저런 걸로 수험재벌해서 나르시시즘 빠지게?
\vspace{5mm}

그리고 제발이지 사교육에 너무 매몰된 짓 좀 하지 맙시다. 나이가 몇살인데 강사 강의 교재찬양만 하고 있는지.
가정환경이 좋아서 경제적 어려움이 없이 사교육받은 걸로 강사추천하고 다니는 건 제 입장에서는 뭐 저 병맛 하기 전에
과연 그 사람이 경제적으로 정말 궁핍해지는 상황이 오면 극복할 수 있을지 의문이지 말입니다.
제가 저성장 가치주로 보는 사람들은 대부분 '가난'을 겪고 막장환경에서도 공부하는 사람들입니다.
이런 사람들은 각성해서 제대로만 캐리되면 남들 10년치를 1년에 달성할 수도 있어서 과거의 실패는 별 의미가 없죠.
벗어나기만 한다면.
\vspace{5mm}

일전엔가 어그로 끄는 모 분이 대화방에서 일지 결과 가지고 시비를 거시던데 수준 참 인증하시더군요.
뭔 일지 가지고 결과 공개해서 .... 다 자기들 수준이죠. 그 딴 건 관심도 없습니다.
저도 학교 스펙은 꿀릴 건 없는 사람이라서 걍 말하지만 웬 잡졸들이 별 것도 아닌 거 가지고 수험생들 공포심 이용해 장사하는 건 별 관심 없어요.
그보다 더 관심이 있는 건 \textbf{절망적 상황에 처했다고 생각하는 사람이 그 고난을 극복하고 올라가는 케이스}입니다.
그럼 여기서 노하우를 얻으려고하느냐? 아뇨. 노하우는 의미가 없습니다. 그리고 그런 노하우는 너무 넘쳐서 탈이죠.
중요한 건 본인들이 실제로 넘어서느냐이겠죠. 어디로 올라가느냐보다도, 본인이 '상승하긴 하느냐' 이게 관건입니다.
\vspace{5mm}

그리고 인생의 가치라는 건 저걸 넘어서는 게 없죠.
돈 많이 번다 그걸로 과시한다... 뭔 진주 물고 있는 돼지도 아니고.
어차피 누구나 다 죽기는 마찬가지입니다만 분명한 건 살아있을 때 얼마나 많은 도전하고 벽을 넘어서느냐하는 것이죠.
\vspace{5mm}

물욕이나 세속적인 것에만 사로잡히는 돼지는 인간들이 아닙니다. 아직 이걸 구분 못 하는 사람들도 널렸을 것입니다만.
살다보면 인간의 형체를 했지만 악마인 경우도 있지만 걍 짐승이나 가축에 불과한 경우도 많다는 걸 알게 됩니다.
가끔 보면 수험을 이런 식의 극복이 아니라, \textbf{'자본'의 먹이}로 타락시키는 분들이 있는데요.
왜 그렇게 사나 모르겠죠.
물질을 지배해야지 물질에 지배당하면 안 됩니다.
상품을 이용해야지 상품에 끌려다니면 안 되지요.
\vspace{5mm}

적당히들 해처드시길 바랍니다.
\vspace{5mm}






\section{고1수학 풍산자 풀고있는 팀}
\href{https://www.kockoc.com/Apoc/518860}{2015.11.30}

\vspace{5mm}

풍산자 다 풀거나 혹은 풀고 있을 때고 고난이 있으면
올림포스 문제집 사서
\vspace{5mm}

http://ebsi.co.kr/ebs/lms/lmsx/retrieveSbjtDtl.ebs?sbjtId=S20140000145&Clickz=C101
J 모 강사 강의 듣고 따라가시는 것 권하겠습니다.
다 따라가지 말고 해당되는 단원만 가지고 올림포스 사서 거기서 문제되는 것만 발췌해 들으세요
\vspace{5mm}

사실 수학 답 없으면 그냥 EBS 올림포스 따라가는 게 답이라고 생각함. 강사들도 그리 나쁘지 않고 무엇보다 공짜.
올림포스 문제도 어려운 건 상당히 어렵습니다.
그리고 적통과 기벡은 제가 유일하게 들을만하다라고 하는 모 선생님 강의는 아직도 EBS에 남아있습니다.
\vspace{5mm}

이렇게 하면 사설에 돈 쓸 필요 없습니다.
그 돈 쟁여두었다가 내년 여름방학 이후에 평 좋은 선생 것만 몰빵해 들으세요.
그 점에서 소위 프리패스라는 건 별로 권하지 않습니다. 이게 나중에 선택권을 엄청나게 제약합니다요.
\vspace{5mm}




\section{EBS에 꿀강의 많으니 그거나 들으셈.}
\href{https://www.kockoc.com/Apoc/520024}{2015.11.30}

\vspace{5mm}

소위 시중에 파는 비싼 것들도 별로 크게 나을 건 없음.
\vspace{5mm}

전에 이것 가지고 언쟁붙을 때 재밌는 게,
EBS 욕하던 사람들에게 그럼 누구 강의가 어떤 문제가 있느냐하면 거기서 \textbf{다들 어버버대더라} 그겁니다.
도대체 EBS란 말이 나오자마자 무조건 EBS 까기만 하던 사람들이 강사가 누군지도 모른다라 :)
\vspace{5mm}

특히 수학은 과거에 비해서 달라진 게 없고 $-$ 내용이 빠지면 더 빠졌기 때문에 $-$
적당히 강사조합만 잘 짜고 인강 들으면서 딴짓만 안 하면 저렴하게 양질의 수업을 받을 수 있습니다.
교재는 수특 쓴다고 하지만 강사는 정작 자기 프린트로 보충하는 경우가 많아서 별 문제는 되지 않아요.
인강 듣다가 딴짓 안 하는 것만 지키면 뭐 딱히.
\vspace{5mm}

이런 조언이야 '재벌'되고 싶어하는 사람이야 싫어하겠죠.
그래서인가 가만히 공개 게시판이나 사람들 많은 커뮤니티에 보면 "돈 안 들이고 가는 저런 방법"조차도 비난하는 병맛들이 있고
몇몇 지능파는 슬그머나 xxx도 좋지 않느냐 혹은 ○○○는 어때요라고 해서 홍보를 하는 경우도 있덥니다.
아마 여기 게시판도 슬그머니 모니터링당하고 있을 거예요. 거기 생리가 그런 곳이라서
\vspace{5mm}

장담하는데 EBS 강의만 충실히 들어도 딱히 사설보다 모자라다고는 못 느낄 것이고
오히려 몇몇 강의는 훨씬 좋다는 판단도 들 겁니다. 저기도 강의 대충하는 곳은 절대로 아니기 때문에.
\vspace{5mm}

저기서 이채형, 손광균, 고동국, 서정원, 정승제도 뭐 들을만한 강의고
교재 집필진으로 유명한 김원중 선생님도 수리 가형 찍고 있네요.
들을만한 강의 없다고 핑계댈 그런 시대는 아닙니다.
\vspace{5mm}

이제 또 장사치들 난리치고 홍보하고 겁주고 그럴 시기가 왔는데
마음에 안 드는 친구들이나 그런 거 소비하라고 하고
님들은 소박하게 시중교재나 열심히 풀고 강의는 EBS로만 해결해보시길 바랍니다.
그래도 충분히 넘치니까요.
\vspace{5mm}

설마 EBS 추천했다고 비추먹는 그런 어이없는 일이 벌어지진 않겠지?
하기야 내년에도 어떻게 팔아먹고 부자될까 하는 돼지들 입장에서는 꾸에엑 소리가 나오는 대목입니다만.
\vspace{5mm}






\section{선택과목 고를 때는}
\href{https://www.kockoc.com/Apoc/521445}{2015.12.01}

\vspace{5mm}

백분위나 표점이 아니라
'상위권이 얼마나 적으냐', '난이도는 얼마나 되느냐'
이거 기준으로 가야합니다요.
\vspace{5mm}

아까 챗방에서 지2 백분위 96이라고 하면서 갓폭이 아니라 조폭(...)이라고 하는 아우성이
전체 2과목의 백분위가 공개되면서 간사하게 바뀌는 드라마가.
\vspace{5mm}

일단 비교가 되지 않죠.
올해의 경우 생2는 100일을 돌려도 2등급 따기도 힘든데
지2는 EBS 강의 하나 빨리 들어주고 한달 돌리면 정말 공부 안 한 게 아니면 만점 받기는 어렵지 않았습니다.
여기서 절약되는 시간만 보더라도 지2가 압도적으로 나았죠.
혹자 생1이나 화2 백분위 가지고 어 좋은 과목 아냐.... 그것들은 이미 수학보다도 어려워진 과목들입니다.
왜 생2 하는 것 안 말렸냐고 저에게 그러면 할 말 없습니당(...)
지2 해서 망했다고 하시는 분은 생2 가면 100$\%$ 망했죠.
사문과 생윤 선택은 제가 올해 극도로 말렸죠. 결과야 뭐.
(그러니까 아말듣 인생망 이래보았자 소용없음. 난 분명 충고했음)
\vspace{5mm}

제가 조언드리는 건 중하위권 대상이지 상위권은 아닙니다. 전 수험에 있어서는 공산주의자(?)라서리.
우선 생1, 생2, 화1, 화2를 기피하는 이유는 간다. 노력으로'만' 되지 않기 때문입니다.
시험 킬러문제를 공략할 수 있는 교재나 강의가 거의 없습니다.
사문과 생윤도 마찬가지요. 변별력 가를려고 온갖 장난질 다 쳐놓기 때문에 노력과 결과가 비례하지 않습니다.
내년에 선택하실 분들은 인기가 덜하면서도 상위권이 덜 포진한 것 고르세요.
다시 말해서 사교육 시장이 덜 형성되거나 특목고나 자사고에서 좋아하지 않는 과목을 고르는 게 낫습니다.
\vspace{5mm}

수능이 쉬워졌다... 그거야 수학만 전부인 줄 아는 구시대의 잔재나 하는 말이죠.
실제로는 더 어려워진 것입니다.
과거에 서울대 합격한 사람이 요즘 수능 치면 과연? 솔직히 저도 장담 못 해요.
\vspace{5mm}

무조건 제 말이 다 맞다는 건 아닌데
노인네가 충고하는 건 그만한 이유가 있다는 것 정도는 알아두시길.
\vspace{5mm}









\section{강사 오개념 발생하는 이유}
\href{https://www.kockoc.com/Apoc/524777}{2015.12.03}

\vspace{5mm}

강의 2시간을 준비하려면 20시간을 쏟아부어야하는데
현실적으로 강의 많이 하는 사람이 그럴만한 시간이 있을 건가.
오히려 수학이 가르치긴 더 수월할 겁니다. 교과서적 개념을 넘어서 원시적인 것까지 뚫고나면
그 다음이야 어떤 문제가 나오든 고교과정 수준에서는 해설할 수 있기 때문.
\vspace{5mm}

하지만 국어, 영어, 탐구는 아니죠.
특히 탐구는 그 내용들이 논리적으로 다 추론되는 게 아니라 일일히 자료 찾아보고 가야하는 것이라서리.
강사가 박사급이거나 강의를 적게 하는 대신 엄청 공부하는 스타일이라면 믿을만하겠지만 그게 아니면 저라면 안 믿을 것입니다.
\vspace{5mm}

수능 시장이 이상한 것이죠.
공무원 사법 행시 CPA 쪽은 강사들이 자기 자랑할 시간도 없고 사실 홍보질도 필요없습니다.
저긴 정말 실력대로 검증되고 있어서리. 교재 오탈자에도 더럽게 욕먹기 때문에 강사들이 짜깁기 책이라도 열심히 만듭니다.
만약 문제 하나라도 빗나갔다... 그대로 퇴출당하거나 복귀하는 데도 상당히 많은 시간이 걸리죠.
\vspace{5mm}

강사든 교재든 공부하는지 그거보고 고르세요. 대부분 강사, 교재평은 3/4가 걍 홍보에다 알바질이라고 보는 게 정확합니다.
그러니까 강의도 솔까 사설들을 필요 없고 EBS 들으면 된다고 얘기하는 겁니다.
EBS는 거기 담당 PD도 졸라 까다로운 걸로 알고 있어서 오히려 검증 면에서는 신뢰도가 높은 편입니다.
\vspace{5mm}

그리고 교재는.
저라면 EBS만 보겠고 사실 신사고, 천재, 두산동아 등에서만 내는 교재 위주로 가겠습니다.
저자보다도 출판사를 믿겠음. 여기 교재들도 오류가 없는 건 아니지만 그나마 수용할 수 있는 수준임.
그냥 수험생을 돈으로 보고 대충 책부터 내자하는 경우라면 볼 필요가 없다고 봅니다.
이거 신사고 알바란 얘기 나올 것 같은데 사실 신사고에서 낸 책들이면 웬만한 것 다 커버가 됩니다.
거기다가 EBS까지... 이거 다 볼 시간이나 있을까?
\vspace{5mm}

탐구에서 물지와 삼사 조합 얘기하는 이유는 별개 아닌 게
물리는 애당초 지엽 문제로 오답시비 걸릴 게 별로 없고,
지구과학도 사실은 과학이라기보단 매우 정밀한 시나리오라서 여기 나온 지엽도 실제 지엽은 아닙니다요.
그에 비해 화학과 생물은 고교과정의 것들이 전부가 아니라는 문제가 있죠.
삼사조합도 마찬가지입니다. 역사가 해석에 따라 달라지긴 하더라도 '기술'로서의 역사가 바뀌는 건 아니거든요.
누가 타임머신 타고 가서 조작질 안 하는 이상 오답시비 걸릴 일이 별로 없음.
\vspace{5mm}

재작년 이맘 때쯤인가 모 게시판에서 탐구가 수능을 좌우한다라고 했을 때 비웃던 사람들이 지금은 뭐할지 모르겠음.
그 당시에 EBS가 좋다라고 얘기해주고 앞으로 탐구가 정말 중요하다 생명과 화학 분산시키려고 어렵게 낸다라는 예측은 맞았기도 하지만
아마 야매교재도 점유율이 떨어질 겁니다. 그것들이 엉터리라는 게 참 많이 검증들 되어서리.
\vspace{5mm}







\section{자존심을 버려야 자신감이 생긴다.}
\href{https://www.kockoc.com/Apoc/524793}{2015.12.03}

\vspace{5mm}

둘 다 성을 自로 써서 그런가 같다고 여깁니다만 실제로는 그렇지 않죠.
\vspace{5mm}

자존심이 강하면 자신감을 잃기 쉽죠.
반면 자존심을 버리면 참된 자신감을 얻을 수 있죠.
\vspace{5mm}

고민상담해보면 대부분 문제가 자존심입니다.
여전히 자기가 잘났다, 내가 사랑스럽다... 라는 것을 말하죠.
그리고 실제로는 조언보다는 그런 자존심을 네가 지켜주었으면 좋겠다라는 메시지를 읽습니다만.
\vspace{5mm}

해결책은 두가지이죠.
하나는 자존심을 완전히 버리는 것 $-$ 쉽지가 않음
다른 하나는 자존심을 높이고 실제로 그렇게 자기 스펙을 높이는 것 $-$ 한계가 있음.
\vspace{5mm}

사람 보면서 아 이 인간은 망하겠군이라고 평가하는 첫째 기준은 자존심이 크냐 아니냐 입니다.
자존심이 큰 사람은 반드시 망합니다요. 자존심이 조금이라도 상하는 일이 있으면 이성을 잃고 어리석은 선택을 하거든요.
이 사람들은 자기 실력을 키우기보다는 다른 사람의 평가, 자기의 외모, 아울러 어떤 부나 권력의 과시를 합니다.
그렇게 안 하면 \textbf{못 견디기 때문입니다}. 그리고 그 맛에 빠지다보면 나중에는 겉은 화려한데 속은 파삭 쪼그라드는 거죠.
\vspace{5mm}

남을 칭찬할 줄 알고 때로는 고개를 숙일 줄도 알아야 행복한 게 별 게 아닙니다.
자존심 자체를 최소화하면서 자기가 얼마나 윤리에 부합하게 사는가 그거 하나만 보면 되지
나머지는 걍 신경 꺼버리면 애당초 걱정할 것 자체가 사라져버리기 때문이죠.
\vspace{5mm}

자살시도 자체도 자존심 때문이죠.
사실 객관적으로는 별 것도 아니지만 본인 입장에서는 프라이드에 상처를 입으면 죽고싶어하는 거죠. 마치 세상이 다 끝난 듯
\vspace{5mm}

\textbf{만신창이가 되더라도 나중에 이기면 된다..}. 라는 마인드로 가야지, 내가 잘 나갔었는데 생각하는 건 아무 소용도 없습니다.
\vspace{5mm}






\section{메모 : 시스템}
\href{https://www.kockoc.com/Apoc/528655}{2015.12.05}

\vspace{5mm}

순수히 경험과 관찰에 의한 짤막한 기록이니 걍 믿거나 말거나
\vspace{5mm}

\item 1. 나이먹는다고 사람이 성장하는 것은 아님, 오히려 나이에 맞게 성장 못 하면 좌절함.
성장은 연속함수가 아니라 불연속함수, 경험이 쌓이면서 내적모순이 심화되면
어느 순간에 이를 정리하기 위해 각성하는 순간이 있는데 이 때야 비로소 성장하기 시작한다고 할 수 있음.
각성하지 못 하면 성장은 없다.... 보아도 좋음. 남자들이 평생 철이 들지 않는단 이야기는, 평생 성장할 수 있단 이야기.
\vspace{5mm}

\item 2. 다수에게 강의하는 것보다도 한 사람을 제대로 가르치는 게 더 어려움.
다수는 50명이 있다고 하면 그 중 최소 20$\%$는 알아서 올라가기 때무네 그걸 자기 실적으로 가산하면 됨.
그러나 한 사람은 정말 제대로 공부한다는 건 '인생'이 바뀌는 문제임. 상당한 난이도가 있음.
가르치거나 상담할 때는 반드시 충격을 주지 않으면 안 됨. 그렇지 않으면 각성할 수 없고 각성하지 못 하면 성장은 안 함.
\vspace{5mm}

\item 3. 대기만성은 맞는 말이긴 한데...
조기에 성숙한 사람은 크게 성장할 수 없음. 초기에 성공만 한 사람은 한번 실패하면 계속 곤두박질함.
반면 오래 실패하면서도 스케일을 넓힌 사람이 그 벽을 넘어서기 시작하면 그 때부터는 승승장구.
다만 이건 사람마다 시기 차이도 있거니와 각성하지 못 하면 글쎄, 그리고 큰 그릇이 만들어지기 전에 작업이 중단되면?
\vspace{5mm}

\item 4. 성공과 실패는 2가지 유형이 있음.
좋은 시스템에 편승해서 성공하는 A, 다른 하나는 좋은 시스템을 자작하는 B.
나쁜 시스템에 올라타서 실패하는 C, 다른 하나는 나쁜 시스템을 자초하는 D.
전반적으로 A, D 비율이 높음. B와 C는 적음.
수험 뿐만 아니라 전반적으로 성공하는 사람들은 B임. 당연히 희소할 수 밖에 없음.
수험 상담을 해보면 D가 많음.
잘못된 시스템은 본인의 자존심과 직결되어있어서.
그래서 자존심이 완전히 날라갈 때까지 실패를 자초함.
\vspace{5mm}

\item 5. 나쁜 시스템을 자초하는 사람은 그냥 군대에 가거나 현강 학원 가는 게 나음.
군대에 가면 기존의 잘못된 시스템을 리셋할 수 있는 장점이 있고
현강 학원에 가면 공부잘하는 애들을 모방해서 좋은 시스템을 주입당할 수 있기 때문임.
독학재수는 스스로 좋은 시스템을 만들 수 있거나 그런 시스템에 길들여진 사람이 해야 효율이 쩔지,
공부 시스템이 안 잡혀있고 본인이 실패를 반복하는 사람이면 자살행위임.
\vspace{5mm}

\item 6. 자기가 B라고 착각하는 경우가 많음. 그런데 B 정도면 이미 '창업주' 역량이 있음. 공부할 필요가 있나?
SKY만 보아도 보통 \textbf{A→D 케이스}가 많음.
엄마의 skirt wind로 용케 좋은 대학은 갔는데 그 다음은 엄마도 커버 못 쳐주고 자기도 자존심 하나로 잘못된 시스템 고집하다...
그 다음으로 \textbf{C→A}는 많이 관찰\textbf{, C→B 케이스는 딱 두 건}만 확인\textbf{.}
\vspace{5mm}

\textbf{7.} 과거에 사람들이 텍스트를 반복해서 암송, 필사하게 한 것은 성공적인 시스템의 틀을 주입한 것이라 보면 됨.
지금 생각해보면 과거시험치는 사람들이 사서삼경 암송한 게 뭔 바보같은 일이냐 하겟지만
사실 그 텍스트들이 대단한 것은 시스템의 모듈에 해당.
그런데 요즘 학생들이 특정 텍스트들을 반복학습하면서 시스템 구축할 수 있긴 있나.
보통 입시 상담은 어떤 좋은 시스템에 올라타야하느냐 묻는 정도니까.
\vspace{5mm}

8. 언제까지 콕콕에서 상담질을 할지는 모르지만 상담가질을 하고 싶은 사람은 위에 준해서 생각해보시면 됨.
\vspace{5mm}



\section{올라가는 게 중요한 것 아닌가}
\href{https://www.kockoc.com/Apoc/529739}{2015.12.06}

\vspace{5mm}

A : 금수저 환경으로서 어린 시절 조기교육, 그 추이로 연고대 합격 가능 : 서울대 합격
B : 막장 환경에서 뒤늦게야 공부시작하고 5수 넘김, 대학도 못 갈 경우였는데 혼자 힘으로 공부해서 서성한 합격
\vspace{5mm}

이 경우 누굴 선택할지는 뻔하다. 나라도 B를 고른다.
\vspace{5mm}

A는 저 상태에서는 이제 더 이상 상승할 수는 없다. 만약 그가 B의 역량이 있었다면 이미 유학가서 미국에서 한가닥했을 것이다.
그러나 그는 투입된 것치고는 효율은 낮다.
반면 B는 저 결과만으로 초라할지 모르지만 투입된 게 마이너스인데도 끝내 그걸 플러스로 돌리고 올라갔다.
대학수험만이 전부는 아니다. 이런 친구는 그 이후에도 꾸준히 도전해서 올라가고 실제로 내가 관찰해보는 사람들이 이런 케이스다.
\vspace{5mm}

물론 학벌로 치자면 어그로 끄는 이야기지만 국내대학에서는  '서울대'에 들어가면 그 이하는 대학으로 안 보인다.
그러나 인생 전체로 치자면 서울대가 중요할까, \textbf{장기적으로} \textbf{꾸준히 성장할 수 있느냐}가 더 중요한 것 아닌가?
20대에 서울대 현역으로 합격했는데 거기서 성장이 멈춘다면 이건 보통 심각한 게 아니다.
반면 그냥 똥통대학에 들어갔다쳐도 그 사람이 엄청난 포텐셜을 지니고 있으면 이 사람은 가히 새로운 창업주가 될 것이다.
\vspace{5mm}

개인적으로 소위 "머리가 결정한다"거나 "양극화론" 같은 걸 까는 이유는
이것들은 엄밀히 말해서 객관적 기술보다는, "그러니까 하류층은 공부하지 말고 거지같이 살아라"하는 걸 조장하기 위한
카스트 제도에 가깝기 때문이다.
한국사 공부하다보면 왜 왕족과 귀족이 불교를 환영했는가 하는 설명으로 "현 신분제도를 정당화"하는 것임을 알게 되는데.
진짜 기득권층은 절대 하류층들이 노력 따위는 못 하도록 한다. 모든 것을 \textbf{다 '유전자'로 돌려 바꿀 수 없는} 것이라고 세뇌한다.
\vspace{5mm}

여기서 상담하는 사람들 $-$ 이제 일지쓰는 사람 빼고는 거의 다 하지 않겠지만 $-$
얘기하다보면 문제가 되는 건 못 배우고 몰라서가 아니라, 배우지 말아야 할 것을 배우거나 잘못된 저주에 세뇌당한 경우.
말도 안 되는 이야기에 집착한다거나 본인에게 별 도움도 안 되는 강박적 메시지를 광신하는 경우가 정말로 많다.
\vspace{5mm}

한번에 서울대에 갈 수는 없다(굳이 서울대에 갈 필요도 없지만)
하지만 여러번에 걸쳐 올라가면 갈 수 있다고 확신되면 시간이 걸리더라도 체계적으로 공부하고 올라가면 되는 것이다.
\textbf{요컨대 과거보다 더 높이 올라가면 그걸로 일단 스타트하는 것이고, 그 과정에서 자기를 묶은 정신적, 물리적 족쇄를 벗으면 된다.}
공부해서 과거보다 올라간다, 나아진다라고 확신만 들면 그대로 공부하면 되는 것이다.
그렇게 하나하나 성공사례를 스스로 만들다보면 가속이 붙어서 나중에 급상승하는 것이지
처음부터 다 해먹을 수 있을리는 없지 않은가.
\vspace{5mm}

실패하는 이유는 두가지이다.
첫째는 서두르기 때문이다.
둘째는 어려운 것부터 하려고 하기 때문이다.
\vspace{5mm}

마음이 초조할수록 오히려 더 오래 걸리는 쉬운 길을 택해야 한다.
빨리 가야한다고 암벽등반하다가 추락해버린 사람들이 얼마나 많은가.
\vspace{5mm}

물론 막장환경에서 공부가 안 되는 사람들도 계속 안 되는 굴레라는 게 있다.
그런데 엄연히 말하면 이건 유전자가 아니다. \textbf{환경 문제가 정말로 크다.}
환경 개선을 하지 않거나, 아니면 그런 환경에 순응해버린 노예의 마인드를 청산하지 않으면 이게 계속 발목을 잡는다.
어느 쪽이든 마음만 먹어서는 곤란하다. 조그마한 과제라도 성공시켜나아가야만 변화가 있다.
\vspace{5mm}





\section{정말 돈이 없다면 모르겠지만.}
\href{https://www.kockoc.com/Apoc/530358}{2015.12.06}

\vspace{5mm}

정말 돈이 없으면 모르겠지만 그게 아니면 이 시기는 알바 뛸 게 아니라 다시 공부 달리고 있어야 할 시즌이죠.
사실 조금만 생각해보아도 알 수 있음.
\vspace{5mm}

11월 $\sim$ 2월 : 대학생 인력공급 넘침, 업자들의 착취가 가능하고 추워서 일하기도 고달픔.
이런 시기에 공부 안 하고 일하면 올해 5$\sim$11월까지 달렸던 \textbf{공부 감각이 모두 소실되어버리죠}.
그래서 3월달에 예열한다고 해보았자 감각 찾는데 3개월 날림.
그래서 실제로 6월달부터 시작. 그대로 반년이 증발되어버림, 이래놓고 또 입시 실패했다고 엉엉댐.
\vspace{5mm}

차라리 지금 공부해서 내년 3월까지 달리고 그 때 알바 뛰는 게 낫지 않나 생각이 들죠.
\vspace{5mm}

그런데 이제 알바 뛰고 공부하겠다... 그 알바비 많아보았자 3개월 300만원.
그런데 11월부터 2월까지 공부해서 얻는 게 300만원의 10배는 넘어서지 않을까 싶은데.
이거 조금만 생각해도 도달하는 결론임. 본인이 정말 돈이 필요하다면 모를까 그게 아니면 '뻘짓'하는 거죠.
\vspace{5mm}





\section{인강을 들을 때 처음부터 배속 높이진 마실 것.}
\href{https://www.kockoc.com/Apoc/530936}{2015.12.06}

\vspace{5mm}

빠른 배속으로 돌려 듣는 것이 뇌활성화에 좋다는 것은 과학적으로 검증되긴 했으나
그건 어디까지나 "로직을 제대로 이해했을 때"라는 걸 전제합니다.
\vspace{5mm}

인강을 듣는 이유는 "생각하는 방법을 보고 따라하기" 위한 겁니다.
단순히 듣고 보는 게 아님, 탁월한 강사들은 어떻게 문제를 읽고 분석하고 생각하고 풀어나갈지 그걸 복기해주죠.
일부 머리가 빠른 사람들은 그걸 지겨워하면서 2배속 이상 돌리는데
\vspace{5mm}

이렇게 하면 내용이 일단 해마에 저장은 됩니다만 아마 곧 잊어버릴 겁니다. 임팩트가 없기 때문이죠.
실력있는 선생일수록 중요한 부분은 매우 '슬로우'하게 저음 깔면서 강조합니다.
이걸 그대로 따라가줘야 제대로 인지되면서 내 것이 됩니다. \textbf{정확히 말해 그 강사의 사고 프로세스를 내 머리에 복사하는 것이죠}.
그런데 이걸 재생속도를 높인다는 건 야동을 10배속으로 돌리는 것과 똑같은 참사(?)를 불러일으킬 수 있습니다.
\vspace{5mm}

일단 강의를 들으려면 제대로 들어야합니다. 정말 중요한 부분은 1배속으로 여러번 반복해 들어야하죠.
대략 1.2$\sim$1.4배속으로 가야지 처음부터 2배속으로 가면 현명할 것 같지만 그만큼 흡수율이 떨어지는 것입니다.
강의가 세뇌라고 해서 무작정 부정적인 것만은 아닙니다. 강의를 듣는 목적이 '제대로 세뇌당하기' 위해서입니다.
세뇌당하고 나서 그 다음 배속을 높여 반복해 들어야 내 것으로 만드는 것이죠.
시간 단축하고 이 사람 저 사람 다 듣겠다고 빨리 듣고 음미는 안 합니다. 그러니까 발전이 없는 것이죠.
\vspace{5mm}

처음에는 배속을 적당히 하면서 들어주고, 나중에 복습용으로 빨리 들어주는 게 낫습니다.
처음에 제대로 들었다면 그 다음에야 1.8배속 이상 가더라도 뇌에서 인지하면서 숙달이 되는 것이죠.
이렇게 활용한다면 사설강의도 별 필요도 없습니다. EBS 강의로 쏠쏠한 효과를 누릴 수 있죠.
사설강의는 여름 이후에 필요한 것만 골라서 알짜만 챙기면 됩니다.
\vspace{5mm}

그리고 강의를 들을 때는 처음에 비판적인 태도는 갖지 마십시오. 그냥 내 머리에 강사가 말한 것을 덮어쓰세요.
어느 정도 세뇌당한 다음에 비판해도 늦진 않습니다.
강의를 들을 때는 강사가 말하는 방식, 속도, 어조부터 시작해 제스처까지 그냥 흉내내는 것도 좋습니다.
\vspace{5mm}






\section{하나마나한 이야기}
\href{https://www.kockoc.com/Apoc/531614}{2015.12.07}

\vspace{5mm}

자꾸만 쓸데없는 걸로 고민하는 사람들이 많죠.
\vspace{5mm}

계획이 실패하는 이유는 간단합니다.
꿈에 부풀어서 한시간 동안에 다 자란 돼지 한마리를 다 먹을 거야... 문학적으론 가능하겠지만 현실적으로 가능하나요?
아무리 배가 고프더라도 저걸 먹을 수 없죠.
계획이 실패하는 사람들은 사실 실천도 안 해본 사람들입니다.
\textbf{계획량  < 실제 업무량 이라는 부등식을 지켜야하는데 보통 계획을 자기 능력치의 5배 이상 잡아놓고 공부가 안 된다고 그러죠}
지금 알바 뛴다 공부 3월에 시작한다는 사람들도 이런 케이스예요.
자기가 공부를 많이 할 수 있다고 착각하는 것입니다.
\vspace{5mm}

그런데 실패하는 인간들은 특징이 있죠. 계획만 참 거창하게 짜기 좋아한다는 것.
왜냐? 계획 짤 때는 즐겁거든요. 그리고 계획 못 이루면 다시 우울증. 그러다가 계획 짜면 또 즐거움.
공부 안 하고 쾌감을 누릴 수 있음, 이것도 어떤 면에서 마약입니다.
\vspace{5mm}

그것도 그렇거니와 쓸데없는 걱정 $-$ 즉 기우도 마찬가지입니다.
자기가 공부를 못 한다거나 못 생겼다거나 하는 생각은 백날 해보았자 아무런 이득 자체가 없어요.
그거 한다고 나아지는 것도 없습니다. 그럴 시간이 있으면 차라리 강의 듣거나 책을 읽거나 피부관리, 표정연습이라도 하는 게 낫습니다.
푸념 늘어놓는 게 정말 자기들이 걱정해서... 헛소리입니다. 그것도 역시 쾌감을 얻기 위한 거죠.
\textbf{자기가 못 났다라고 해서 다운되어있다가 남에게 격려, 칭찬을 듣고 다시 쾌감 누리고 또 우울해하고}.
자기가 정말 진지하게 못 났다고 생각하면 그런 말조차 안 해요. 학원가거나 바로 병원가는 거지.
이것도 그냥 어찌보면 중독적 행태입니다.
\vspace{5mm}

가끔 보면 남자 녀석이 무슨 기생오래비도 아니고 자기면상 올리면서 존잘 이러는 병신같은 케이스도 그렇죠.
그거 보면 "이 녀석 정신적으로 맛이 갔구나"라는 걸 느끼는 경우가 많죠. 뭔 남자가 못 났으면 얼굴 팔아먹고 있냐란 생각도 들지만,
\textbf{남들이 어떻게 반응해주나 그거에 쾌감 느낀다면 그건 다른 데에서는 스트레스를 엄청 받고 있으며 자아도 흔들리고 있단 이야기거든요}.
\vspace{5mm}

위 세가지 유형. 병적인 쾌감에 중독된 케이스입니다.
자기 공부에 바쁜 사람은 저럴 여념도 없어요.
그래서 전 저런 사람에게는 그냥 진실된 평가 내려주고 알아서 하라고 합니다.
한번 받아주면 그 다음부터 '쾌감' 얻으려고 병적인 행태를 계속 반복하거든요.
그건 근절하고 끊어내야한다는 게 제 생각입니다.
이거 받아주면 끝도 없어요 정말이지.
\vspace{5mm}

저 공부 계획 어때요... 라고 할 시간에 문제하나라도 더 풀고
나 못 생겼어... 할 시간이 있으면 어떻게 꾸미고 코디할 것인지 연구하고
자기 얼굴 사진 올릴 시간이 있으면 가서 이웃이라도 도우면 됩니다.
\vspace{5mm}

자기들이 정말 중요하다고 생각하는 게 4$\sim$50대 이후에도 중요한지 생각해보세요.
얼굴 타령 참 지겹게들 하는데 그거 어차피 30대 이후에 노화 안 되는 사람 없습니다.
나이먹을수록 빛이 발하는 건 얼마나 많이 공부했느냐, 현명한 판단을 하느냐 하는 겁니다.
자기가 못 생겼다 어쩌구... 매력이라는 것은 외모도 외모지만
그 전에 "스트레스를 덜어주고 마음을 편안히 해주는" 것에서 비롯되는 겁니다.
무엇보다 계획충. 자기가 실패한 기간동안 그냥 소박하게 공부했어도 합격했을 거라고 계산하면 답 나옵니다.
\vspace{5mm}

정신승리 얘기가 아니고 남자든 여자든 공부하면 매력이 늘어납니다.
눈빛이 정말 달라지거든요.
아무리 잘 생기고 예쁘면 뭐합니다. 눈빛이 흐리멍텅하면 기둥서방 술집여자지.
\vspace{5mm}








\section{공부시간 산정법}
\href{https://www.kockoc.com/Apoc/539674}{2015.12.11}

\vspace{5mm}

가장 현실적이고 탁월한 견해.
\textbf{"암기하는 시간}만 공부시간으로 정한다"
가령 12시간 책상에 앉아있다, 그 중 암기한 시간이 2시간이면 공부시간은 \textbf{2시간으로만 잡아야 한다.}
\vspace{5mm}

그런데 수능은 암기와 거리가 멀잖아요.
\vspace{5mm}

\textbf{ⓐ 문제풀이를 하고 해설을 읽고 정리하는 시간}
\textbf{ⓑ 필기한 것을 '읽고 암기하는' 시간}
\vspace{5mm}

사실 이걸 제외한 나머지 시간은 공부시간에 빼야하지 않을까.
\vspace{5mm}

강의를 들어도 기억이 되는 경우가 있긴 한데 이건 좀 애매하다.
A 강사 강의는 한번 듣기만 해도 기억될 수도 있고, B 강사 강의는 안 그럴 수 있는데 이걸 정형화시킬 수 있을까.
\vspace{5mm}

그리고 국어, 수학, 영어에서 킬러문제를 해결해나갈 때의 "논리적 접근"이라는 건 단순 암기가 아니라
체험$-$체화를 거쳐야하는데 이건 어떻게 잡을 것인가(사실 이걸 강사들이 해결해줘야하는데 이런 강의는 정말 찾기 어렵다)
\vspace{5mm}

아무튼 공부시간이 많다고 해도 소용없는 게
\vspace{5mm}

\textbf{공부시간이 12시간이라고 해도 '암기 등에 쓰는 시간'이 0시간이면 그건 공부한 게 아니기 때문이다.}
열심히 장사를 했다. 하루에 1000만원 어치를 팔았다. 그런데 비용이 1000만원이다, 이런 무슨 소용이 있을까.
인강을 열심히 들었다. 그래서 정리를 안 하고 거의 다 망각해버린다.... 이게 공부인가?
\vspace{5mm}

남는 건 결국 '기억' 밖에 없다.
단지 그 기억이 시각적 기억이냐 청각적 기억이냐,
그리고 심상 기억이냐 경험 기억이냐 추상 기억이냐 그 차이일 뿐이다.
역으로 말해서 인간이 기억을 못 한다면 학습이 필요가 있을까.
\vspace{5mm}

다들 공부했다고 하소연하지만 이렇게 구체적으로 학습'회계' 관점에서 들어가보면 인과관계는 매우 잔혹해진다.
강사들은 '이해'만 하면 된다고 이야기하지만 실제로 자기들이 그 지식을 암기하고 있단 사실을 망각해버린다.
\vspace{5mm}

그렇다면 암기는?
\textbf{반복}이지.
\vspace{5mm}

그렇다면 공부의 실패는?
\vspace{5mm}

ⓐ 암기할 대상을 잘못 고르거나 누락
ⓑ 암기를 제대로 하지 않았다.
\vspace{5mm}

이걸로 정리되지 않나?
\vspace{5mm}

입시기간이 길어지고 똑똑한 게 많은 사람들이 실패하는 경우.
A 관점 : 우와 수능 어렵나봐, 저 사람들 저렇게 실력좋은데
B 관점 : 암기시간 재보면 얼마나 나올까.
\vspace{5mm}

유감스럽지만 B 관점이 옳을 가능성이 높다.
\textbf{공부할 대상을 잘 선정하고 많이 반복하고 암기한 것 가지고}
\textbf{자기가 남들보다 머리가 좋고 우월하다라고 과시하는 허세들이 많은 것도 현실이다.}
\vspace{5mm}

1번 읽고 기억하면 머리좋은 것이 아닌가요?
10번 읽고 기억하면 그건 머리 나쁜 게 아닌가요?
\vspace{5mm}

그럴 수도 있다. 그런데 10번 읽는 게 나쁘단 말인가.
9번 더 읽는데 100년도 걸리는 것도 아니고.
만약 10번 읽어야 기억하는 사람이라면 그냥 쿨하게 10번 읽으면 된다.
1번 읽어서 기억하는 사람이라면 그만큼 예민하고 날카로우며 학습하지 말아야할 것도 학습하는 문제가 생길 테니까.
그런데 문제는 10번 읽어야하는 사람들은 2번 읽고 포기한다는 것이다.
\vspace{5mm}

학습 비결이라는 게 별 게 없다. 사실 그 노하우라는 건 대형서점에 전시된 책들에 나와있다.
물론 '비결'이 아닌 가짜 비결들도 많다. 그리고 수험시장은 그런 가짜비결조차도 비싼 값에 파는 경우도 많다.
실제로 문제가 되는 건 비결을 몰라서 아니라, 비결에 집착하다가 \textbf{가짜 비결에 낚여서 허송세월한 케이스가 레알 많다는 것이다}.
그 친구들이 "암기시간만 공부시간"이라는 극단적이지만 매우 간결한 것을 받아들일까, 그럴 리야 없지 않나.
\vspace{5mm}

+ 사례
\vspace{5mm}

내가 괜찮다고 보는 모 콕콕러는 특정 과목 인강을 들으면서 이해가 안 간다고 할 것이다.
그런데 저건 매우 정상적. 이과 처음 시작해서 이제야 인강보면서 바로 이해간다면 그게 거짓말이나 사기이지 정상이겠는가.
그 인강도 반복해서 3$\sim$5번 듣고 책도 10번 읽어야 한다. 그래야 이해되는 게 정상이다.
내가 흡족한 건 "모른다"고 분명히 고백한 것이다. 허세들의 문제는 모르는 걸 아는 척 하다가 발리는 것이다.
제대로 안다는 건, \textbf{자기가 무엇을 모르는가 그걸 분명히 아는 것}이다.
자기가 모르는 것, 못 하는 것을 제대로 알아야 올라간다.
\vspace{5mm}

금수저들은 자기가 머리가 좋다고 생각하지만 그건 개소리.
자기들이 처한 환경에서 돈걱정없이 공부하는 것도 그렇지만, \textbf{환경 자체에서 그런 학습지식이 수도없이 반복주입된 걸 본인들이 모른다.}
환경은 에스컬레이터나 달리는 기차와 같다. 가만히 있어도 움직이긴 움직인다.
그러나 어느 순간에는 스스로 걸어가야 할 순간이 온다.
모 의원 아들 청탁 사건 같은 게 벌어지는 게 그 때문이다.
사교육 빨이 먹힐 때야 환경이 좋은 걸 모르고 지 머리가 좋은 걸로 안다.
\vspace{5mm}

그러나 스스로 헤쳐나아가야 할 때는? 그 때는 정말 흙수저 미만잡이다.
\vspace{5mm}

+
\vspace{5mm}

사실 또 한편으로 흥미로운 것은 요즘 x스쿨 사태에서 보다시피
자기들이 금수저라고 대놓고 말하는 금수저는 없단 것이다. 다들 흙수저인 척 하지.
그리고 이건 공공연한 사실이다. 금수저들도 \textbf{자기들처럼 공부하면 안 된다는 것을} 안다.
x스쿨이 정말 제대로 공부시켰어도 x시 존치 반대했을까. 자기들이 공부를 안 하고 실력이 없으니까 무서워하는 거지.
\vspace{5mm}

간혹 보면 건물주 미만잡 금수저 미만잡 거리는 애들. 그럴 시간에 공부나 하지 뭐하나 모르겠다.
\vspace{5mm}






\section{개인적 검증.}
\href{https://www.kockoc.com/Apoc/540628}{2015.12.11}

\vspace{5mm}

A. 인강 30시간짜리 들어봄
\vspace{5mm}

들을 때는 개꿀.
복습 안 했음(...)
3일 지나니까 다 까먹음.
\vspace{5mm}

B. 인강은 살짝 듣고 기본문제풀이해본 것
\vspace{5mm}

반복횟수 보니 대략 5번은 넘어감
어렴풋이 기억남
\vspace{5mm}

C. 지금도 백지 복기 가능한 과목(뭔지 물어볼 필요는 없을 듯)
\vspace{5mm}

가끔 뻘짓하긴 하는데 이건 백지 주면 처음부터 끝까지 설명가능.
연구용으로 인강을 들은 적은 있지만 그건 별로 도움 안 되었음. 이미 그 이전에 독학으로 완성
다만 상세한 이론 연구를 위해 온갖 책을 찾아 읽고 스스로 연구해보았음.
\vspace{5mm}

황금의 3개월에서 1/9가 지나감.
이 시점에서 경고하고 싶은 건, 시간 많이 남았다고 여러 과목 동시에 진행하지 말고
2$\sim$3과목으로 좁히고 목표량을 100이 아니라 10+10+... +10으로 쪼개서 그 10 하나를 끝낼 때마다 스스로에게 상 주도록 쪼개고.
처음부터 큰 것을 하려고 하지 말고 작은 것을 수도없이 반복하라는 것.
\vspace{5mm}

10page 분량이 있다고 칩시다. 그럼 이게 정말 10page일까요?
실제로는 100page여야합니다. 10회독은 해야하니까요. 그런데 실제로도 10page에서 100page 분량 내용이 나옵니다.
우리가 배우는 교과서, 참고서 내용이란 것은 '요약, 압축'된 것이죠. 공부하다보면 요약, 압축된 것이 풀리기 시작합니다.
\vspace{5mm}

2월까지 공부할 때 만약 1회독만 하는 거라면 그 계획은 폐기하고 다시 시작하시길.
예를 들어서 탐구과목을 2개 해서 2회독하는 것과 탐구 과목 1개를 해서 5회독한다면 후자가 낫습니다.
분량을 줄이고 회독수를 늘리는 것이 공부의 비결입니다.
왜 실패하는지 아십니까? 시간 많다고 전과목을 다 건드려보면서 인강까지 다 듣고 학익진(...)으로 가기 때문입니다.
이거 양도 많지 성과도 단기간에 안 나오지, 일주일 지나면 다 때려치우고 싶어집니다.
타겟을 철저히 좁히세요. 황금의 3개월동안 한과목만 봐도 좋습니다. 범위를 좁혀서 회독, 반복을 높이시길 바랍니다.
\vspace{5mm}

이것만 하면 됩니다. 무슨 아무개 강의 듣는다거나 특별한 교재 봐야한다거나 그럴 필요 단 하나도 없습니다.
공부가 안 되는 건 회독수가 적고 연습이 부족해서입니다. 그 다음으로 본인의 사고 프로세스가 문제있거나 잘못 알고 있는 것도 있지만요.
하지만 대부분은 회독수가 적어서 안 올라가는 것입니다.
이런저런 방황하다가 모 선생 강의 듣고 깨우쳤다하는 케이스.
물론 모 선생이 잘 가르쳐서 그럴 수 있지만, 그 모선생 강의를 듣는 시점에 '회독수와 연습'이 임계점에 도달했기 때문인 경우가 많습니다.
\vspace{5mm}

그럼 몇번을 반복해야하나?
하루$\sim$사흘에 1번이라고 하면 10번은 최소한 봐야한다고 봅니다.
이제는 정말로 제대로 알지 못 하면 시험에서 날라가기 때문이죠.
\vspace{5mm}







\section{첨단장비의 노동환원}
\href{https://www.kockoc.com/Apoc/540961}{2015.12.12}

\vspace{5mm}

공부 노력은 안 중요해, 그냥 xx 강사 강의만 들으면 된다.
그냥 가볍게 반문하겠음
\vspace{5mm}

제 친구 중에서는 당시 특목고$-$서울대 라인 밟은 사람들 많습니다(아시는 사람은 챗에서 들었을 테니 말 안 하겠음)
그런데 그 친구들 중에서 누구 하나라도 공부방법을 깨달았다, $\sim$ 안 해도 된다라고 하는 단 한명도 없었고
저 역시 '올바른 공부방법이 무엇인가'라고 하면 목숨 걸고 말하라고 하면 말을 못 하겠음.
\vspace{5mm}

그런데 \textbf{"지금은 놀고 그냥 나중에 하면 되지 않겠느냐"}라고 깨달았다고 하는데 그게 참.
거기 댓글 단 사람 중에서 몰x군은 작년 이맘 때에 제가 늦으니까 빨리 했는데 안 했고
래x은 플래너 인증 요청하면 알겠지만 죽어라 달린 분이 그런 댓글 다는 건 좀 그렇지 않나.
\vspace{5mm}

자기 약점이나 방향이라는 건 '죽어라 공부하고서도 한계에 부닥쳤을 때'에 비로소 느끼는 것이고
그렇다면 그 전제라는 건 죽어라 했을 때를 전제하는 건데
웃긴 건 공부에 방향이 중요하다라고 하는 사람들을 잘 보면 정작 그렇게 \textbf{죽어라 공부하는 경우}가 아닙니다.
\vspace{5mm}

자기가 약점이나 방향이라고 느끼는 것이 정말 \textbf{'진짜 약점이나 방향'}이라고 생각하는 건지?
가장 위험한 생각이 이거죠.
\textbf{1000시간 공부할 바에는 걍 약점과 방향 잡고 10시간 공부하는 게 더 똑똑해.}
아마 다들 이런 생각하고 있을 것임. 래서 +1수를 늘리는 겁니다.
왜냐? 약점과 방향을 잡는다는 건 훌륭함. 그런데, \textbf{자기가 아는 그 약점이 진짜 약점이고 그게 전부라는 보장이 어디있죠}?
빨리 양치기하라는 이유가 별 게 아닙니다. 그래야만 그나마 \textbf{진짜 약점과 방향을 알 수 있어서}입니다.
그건 본인이 알지 남들은 아무도 몰라요.
\vspace{5mm}

오히려 불편한 진실은 "아, 빨리 시작하면 뭐해. 아, 양치기하면 뭐해. 걍 xx 인강 듣고 방향 잘 잡으면 되지"
이 메시지조차도 \textbf{공부량을 줄이거나 공부를 안 하려는 아주 교묘한 핑계}라는 거죠.
정작 실적이 좋은 사람들은 자기 확신은 안 합니다. 공부하면 할 수록 자기가 더 모자르다라는 걸 깨닫거든요.
졸라 달릴 수 밖에 없지 O, X 확 떨어지는 것 아니라는 걸 알기 때문입니다.
\vspace{5mm}

혹은 이런 이야기를 하겠죠. 첨단장비 걍 훔쳐쓰는 게 더 효과적이지 않나.
\vspace{5mm}

그런데 훔쳐쓰는 건 한계가 있죠. 고장나면 어떡할 거임? 그리고 자가 생산은 언제할 것임?
님들이 가볍게 쓰는 맛폰조차도 하드웨어$-$모듈$-$부품, 소프트웨어$-$모듈$-$코드.
이거 하나하나 분해해보면 \textbf{수십년, 아니 수백년전부터 진행되어온 지식, 육체 노동자의 노동이 집약된 결과입니다}.
그냥 하늘에서 뚝 떨어진 게 아니란 것이죠. 어디든 사람의 '손길', 즉 노동이 압축되어 있습니다.
우리야 그걸 천박한 자본주의적 거래로 너무 가볍게 소비하고 있습니다(그래서 살기좋은 세상인 것이죠)
\vspace{5mm}

그런데 공부가 돈을 주고 상품만 딱 구입하면 능력치가 올라갑니까.
그렇다면 누가 공부를 합니까, 걍 돈주고 능력치를 사지.
하지만 현실적으로 그런 경우는 없다는 데 유의하시길 바랍니다.
비싼 돈 주고 컨설팅을 한다... 아니 그게 효과가 있으면 다 그걸 구입하겠죠.
하지만 실제로 수율을 보자면 그것들도 형편없는 걸로 압니다.
\vspace{5mm}






\section{잡담}
\href{https://www.kockoc.com/Apoc/543390}{2015.12.13}

\vspace{5mm}

\item \textbf{1. 정석}
\vspace{5mm}

정석부터 잡으려는 분들이 계시는데
정석이야말로 1등급 넘어서면서 수리적 사고가 숙달되어서
눈에 보이는 것들을 집합$-$명제로 함수관계로 환원시키거나
모든 운동을 벡터와 변환으로 보거나
심지어 소리까지도 삼각함수로 근사시켜서 주기를 논하거나.
\vspace{5mm}

... 그 정도가 아니면 처음 보는 건 그냥 자살일 수 있으니 유의하시기들 바랍니다.
\vspace{5mm}

수능시험 수학이 쉬워졌다 어쩐다하지만 그건 피상적인 이야기입니다.
적어도 과거보다 나아졌다고 생각하는 것 $-$ 특히 최근 경향을 보면 소박한 "수리적 사고"를 검증하는 건 성공했다고 보이는데요
이런 시험이면 정석을 주교재로 삼는 건 실익이 적습니다.
과거 수학시험이야 DB 승부다보니까 우리보다 수학 선진국인 일본의 온갖 패턴이 들어가있던 정석을 그냥 공부해도 동
\vspace{5mm}

최근 수학은 심지어 내신조차도 선생들이 업그레이드되었기 때문에 '암기만 해서' 푸는 것은 안 냅니다.
중요한 건 \textbf{문제를 독해하는 능력}입죠.
거기에 정석이 직접 도움을 주진 못 한다라고 얘기하겠습니다. 정석은 독해능력이 있는 사람에게만 도움을 주는 책입니다.
\vspace{5mm}

\item \textbf{2. 돈을 주면 특효약을 얻을 수 있다?}
\vspace{5mm}

공부에서 참 흔한 이야기입니다.
콕콕에서도 그렇게 일종의 야매교재를 칭찬하거나(그래서 올 수능은 적중하셨나)
특정 강의나 학원만 지독하게 신성시하는 경우가 있었사온데.
\vspace{5mm}

그러니까 그 특효약이 뭔지나 가르쳐주었으면 합니다.
자기만 따라오면 공중부양이 가능하다 기경팔맥이 뚫려서 200살까지 산다 그렇게 드립치면 뭐해요.
실제로 그게 뭔가 \textbf{보여줘}야지.
\vspace{5mm}

다 필요없고 올 수능만 봅시다. 그래서 \textbf{적중한 교재가 있었습니까}.
그렇게 시험전까지 실모 안 보면 망한다 누구 실모 좋다고 했는데 실제 적중도는?
수학에서 고1수학 발상 나온다라고 한 사람 단 한명이라도 있던가.
영어는 쉽게 나오니까 걍 상관없다고 하더니만 어렵게 나오니까 평가원이 그렇게 출제하면 안 된다 드립치질 않나.
이거 다들 보면 걍 제정신이 아닌 것 같음.
\vspace{5mm}

\item \textbf{3. 멍청한 짓을 반복하면 바보다.}
\vspace{5mm}

상담하면서 늘 느끼지만 실패하는 가장 큰 원인은 "멍청한 짓을 반복해서"입니다.
그런데 본인은 그걸 잘 못 느낍니다.
저야 상담해줄 때야 온화하게(?) \textbf{"$\sim$ 하는 게 낫지 않느냐"}라는 권유형을 취하지요.
그러나 그게 정말 내용상 권유형은 아닙니다. 제가 상담해주는 방식은 간단합니다.
\textbf{"피상담자가 어떻게 하면 제대로 말아먹는가"하는 걸 먼저 가정하고 시나리오를 써보고 이야기드리는 것이죠.}
피상담자가 A라는 코스로 가면 망할 게 분명하면 $\sim$A에 해당하는 B, C, D를 권유하는 것입니다
과거에는 대놓고 A로 가면 망한다... 했는데 정작 A로 가면 망한 사람들이 "너 나 비난했지"라고 해서 권유형 취한 거지 내용 바뀐 건 아닙니다.
\vspace{5mm}

상담이야 책임감있게 해줄지는 몰라도 상대방이 "에이 안 그래도 되잖아요"라고 해서 잘못 가도 그건 \textbf{제가 알 바도 아니죠.}
지금 황금의 3개월 1/6 지났죠. 자, 지금이야 마음 편하죠? 2월 말 되면 또 사람들 마음 바뀝니다.
상담이야 해주겠고 정말 상대가 그렇게 한 경우에는 나름 응답을 하지만 또 자기 멋대로 하는 경우는 제가 생각할 경우는 아니죠.
더군다나 대놓고 말 안 하지만 익명성을 이리저리 바꾸거나 시험해보거나 하는 한심한 케이스도 많던데 그건 당사자들이 알겠죠?
\vspace{5mm}

덤으로 상원이 아닌 경우 상담은 늦게 대답해줍니다.
신중히 답할 것도 있지만 '먹튀'만 하는 케이스는 제가 좋게 보지 않아서요.
\vspace{5mm}






\section{다수의 선택}
\href{https://www.kockoc.com/Apoc/545560}{2015.12.15}

\vspace{5mm}

출제경향은 다수의 생각을 배신하는 경향이 있다.
작년 수학과 국어 출제 수준도 그랬고 올해 영어 역시 마찬가지였다.
\vspace{5mm}

수험사이트에 올라오는 코메디성 글이 많지만 가장 웃긴 것은
일개 학생이나 대학생이 \textbf{"평가원 너희가 그렇게 출제하면 안 된다'라고} 훈장질을 하고 있다는 것이다.
변별력을 갖추려는 평가원으로서는 다수 학생들을 통수먹이는 건 당연하지도 않나
\vspace{5mm}

저렇게 말하는 사람들은 실제 입시 결과도 기대 이하지만 입시를 떠나서 앞으로 사회 생존도 염려된다.
그건 수능시험의 본질조차 생각하지 않고 있기 때문이다.
대학수학능력시험은 다수를 위한 시험이라면 다수가 하는대로 가도 된다.
예컨대 운전면허시험이라거나 토익 700점 넘기는 수준이면 다수론대로 따라가도 된다.
그 정도의 결과는 "다수"를 위해 열려있는 것이기 때문이다.
\vspace{5mm}

하지만 수학능력시험은 \textbf{소수를 위한 시험}이다.
개나소나 만점을 외치니까 우와 나도 만점을 받을 수 있어 하지만, 실제로 그런 \textbf{만점을 받는 건 소수다.}
그것도 그냥 \textbf{소수}가 아니다. \textbf{"\textbf{다수}"의 생각과 행동을 읽고 있는 \textbf{소수}}여야한다.
\vspace{5mm}

그렇기 때문에 최소한 다음 조건들을 충족해야 한다.
ⓐ 다수가 보는 시중 교재나 강의는 빠삭해야 한다.
ⓑ 출제자 수준을 넘어선 소수여야 한다.
\vspace{5mm}

둘 다 충족해야 실패하지 않는 것이다. 하나라도 X 가 되어버리면 거기서 날라가버린다.
본인은 소수에 속한다고 하지만 시험만 치면 어이없이 실수하거나 날라가는 경우는 ⓐ가 안 된 것이다.
본인은 시중교재를 충실히 풀었지만 어려운 문제에서 막히는 건 ⓑ가 되지 않은 것이다.
\vspace{5mm}

그럼 다수와 소수를 둘 다 선택하라... 그래서 힘든 것이다.
분명 \textbf{다수의 행렬에 속해서 영화를 관람하고 있어야} 한다,
하지만 결정적일 때 \textbf{나 혼자 비상구로 도망갈 준비를 해놓아}야 한다.
다수의 행렬에 묶여있으면 극장 화재 시탈출하지 못 하고 압사당할지도 모른다.
그렇다고 처음부터 비상구에 있으면 영화를 보지 못 하고 허송세월해야한다.








\section{과거의 문제와 미래의 문제}
\href{https://www.kockoc.com/Apoc/551104}{2015.12.18}

\vspace{5mm}

선행을 미리 했던 학생들이나 특목/자사고 간 친구들이 공부를 '잘' 하는 건 맞다.
그러나 신기한 건 그들 모두가 수능을 잘 치르진 않는단 것이다. 수시로 잘 갈지는 몰라도 정시는 생각보다 수율이 낮은 경향이있다.
이것도 왜 그런가 생각해보면 간단하다.
우리가 하는 공부 $-$ 특히 사교육+기출 중심의 공부는 "과거의 문제"를 푸는 공부이다.
물론 과거의 문제를 학습하고 익히는 건 기본이다. 하지만 수능 문제는 "미래의 문제"다.
과거의 문제에만 집착하면 \textbf{미래의 문제를 풀기 어렵다}.
\vspace{5mm}

그래서 입시판은 늘 이변이 발생한다.
그냥 기대 안 했던 A가 고득점을 받는다. 그 때까지 A를 천대하던 사람들이 "너는 잘 할 줄 알았어"라고 입장을 바꾼다.
유망주이던 B가 터무니없는 점수를 받는다. 사람들은 싹 침묵해버린다.
그렇다고 A가 잘하고 B가 못 한다라고 단언할 수 없다. 단지 출제'경향'에 안 맞았을 뿐이다.
이런 경향을 무시하는 사람들은 모든 것을 "운"이라고만 치부하기 좋다.
물론 운이라는 건 없지 않다. 그러나 운은 '인과관계'를 정확히 모를 때에 그 현상을 설명하기 위한 도피적 개념에 불과하다.
\vspace{5mm}

잘 할 줄 알았는데 못 나온 케이스들의 특징은 다음과 같다.
\vspace{5mm}

$-$ 모평 등급에 연연한다 (모평은 어디까지나 과거 문제의 짜깁기일 뿐이다, 물론 6,9평은 다르지만)
$-$ 기출과 인강을 돌리려고만 하지 생각을 하지 않는다(막연히 푸는 것과 생각하는 건 다르다)
$-$ 자신의 단점을 개선하려하지  않는다(수험은 자기를 바꾸는 과정이지 수험상품 소비가 아니다)
$-$ 환경을 바꾸지 않는다(공부는 의지대로 하는 게 아니다. 환경이 도와주는 것이다. 단지 우리가 의지로 환경은 바꿀 수 있다)
$-$ 절박하지 않다(사실 이게 가장 크다. 내가 느낀 n수 실패의 가장 큰 원인은 당사자들이 지나치게 편하다는 것이다)
\vspace{5mm}

분명한 건 2017 수능은 2016 수능이 아닌데도 사람들은 올해와 마찬가지로 \textbf{그렇게 바라볼 것}이란 사실이다.
\vspace{5mm}

대충 이런 체크리스트로 나눠진다.
\vspace{5mm}

1$-$1 수능을 수능으로 보는가 / 1$-$2 수능을 학력고사처럼 보는가
2$-$1 어떤 문제도 원점으로 돌아가 풀 수 있도록 하는가 /  2$-$2 과거 기출문제를 암기하는데 그치는가
3$-$1 기본적인 것을 설명할 수 있는가 / 3$-$2 잡기에 맹목적인가.
\vspace{5mm}

공부를 열심히 해왔다 안 해왔다를 떠나서 저것들로 한번 점검해보면 대답하기 싫은 방향으로 답이 나온다.
분명 공부를 열심히 하면 좋은 성적을 거두는 건 맞는데.... 문제는 n수생 이상부터는 \textbf{실패하는 방법도 학습해버린다라는 게} 문제.
실력 1000을 높이는데 동시에 허력을 900을 높이면 1000$-$900=100이 되는 셈이다.
그런데 사람들은 \textbf{자기의 허력을 공개하는 건 꺼린다}.
\vspace{5mm}

여담이지만 벌써 한달이 지났다. 2016 수능까지면 1/12가 지난 셈이다.
다들 시험 전까지는 아 일주일만 있으면 공부했을텐데... 라고 간절히 호소하지만 지금은 그딴 건 없다.
인간이 다 그렇게 간사한 동물인 것이다.
\vspace{5mm}

전 분명 일찍 시작하라고 했으니 나중에 원망하지 마셈.
\vspace{5mm}







\section{[콕콕운영제언] 콕콕사이트의 보수적 운영}
\href{https://www.kockoc.com/Apoc/552129}{2015.12.18}

\vspace{5mm}

일단 1년간 주욱 관찰해보면서 확인한 것은
'비공개'로 "상호신뢰가능한 회원들끼리 소통하는" 것이
공개된 공간에서 서로 믿지 못 하는 것보다 나았다는 것을 확인했습니다.
\vspace{5mm}

\textbf{ⓐ 심리적 위안} : 사실 이게 가장 크죠. 불안하거나 스트레스 받을 때 호소하는 것
\textbf{ⓑ 질서있는 갈등} : 사람들이 모인 곳이야 으레 다툼과 갈등도 있지만 상호 신뢰에다가 중재절차가 반영된 터라
\vspace{5mm}

\textbf{ⓒ 안전한 소통}
\vspace{5mm}

아까도 챗방에 보니까 '아무 말도' 없는 회원이 호출을 해보니 바로 나가버리던데.
안 그럴 거라고 믿고 싶지만 부적절한 의도로 대화 내용을 캡처해서 악용해버리는 경우가 불가능하지 않습니다.
\textbf{겉으로는 정정당당한 척 하면서 뒤로는 이중 아이디나 부계정을 써서 타인 행세하는 비겁한 인간들이} 늘 있기 때문입니다.
요즘은 이게 진화해서 일부러 명예훼손이나 모욕죄로 엿먹이기 위해 발언을 유도하는 질떨어지는 자들도 있습니다.
\vspace{5mm}

콕콕이 너무 친목화되어가느냐 하는 경우도 있습니다만 일단 알려둘 것은
여기는 영리 사이트가 아니란 것이죠. 핵심멤버가 자기 책을 팔아먹는 경우도 있겠지만 그건 사이트 목적과는 다소 거리가 있습니다.
이 사이트의 목적은 n수의 안식처 \textbf{수험생들끼리 소통하고 정보교환을 해서 자기들의 벽을 넘어서는 극복의 축적}이지요.
저도 '상업주의'와 무관하게 그런 게 가능할까 하는 방법의 검증과 시스템의 확립을 위해서 여기 있는 거지
천박하게 자기 광고해서 컨설팅 비용이나 야매교재로 재벌이 되는 게 아닙니다. 오히려 그런 것을 '배격'하는 걸 목적으로 하죠.
\vspace{5mm}

그러나 돈이 걸린 곳은 어디나 더러운 인간들이 있습니다.
그런 사람들에게까지 무단으로 개방해버리는 경우 문제가 터집니다.
올해 있었던 몇가지 사건들만 보더라도 사실 그 진실을 알고 있는 사람으로 말하건대 이 사회에서 승리하는 건 '정의로운' 사람이 아닙니다.
윤리 따위는 개나 주더라도 돈으로써 법을 이용할 줄 아는 사기꾼이 일단 우세합니다. 현실이 그런 건 어쩔 수가 없죠.
그렇기 때문에 선량한 사람들을 보호하기 위해서라도 '성벽'을 쌓는 건 불가피합니다.
\vspace{5mm}

일단의 비공개 게시판이 있는 건 양해하셨으면 합니다. 꾸준히 활동하고 글을 쓰며 공부하려는 분들은 환영하고 받아주니까요.
그리고 실제로 그런 비공개 게시판이 '스트레스'와 '불안감'에 시달리는 수험생들에게 도움이 된다는 게 개인적인 판단입니다.
공개 게시판의 경우는 수험생들이 \textbf{자기를 방어하기 위해 일부러 성적을 과대발표하거나 무리한 공부를 하는 식으로 파멸해가는 경향이 있습니다.}
즉, 우리가 알고있는 소위 수험생들의 허세라는 건 자기 방어적 성격이 강하단 것입니다.
만약 수험생이 남 눈치를 보지 않고 비슷한 처지에 있는 사람들끼리 소통하고 대화할 수 있다면 저런 허세를 부리지 않아도 되었겠죠.
콕콕에서 중시하는 건 누가 좋은 대학에 갔느냐가 아닙니다.
\textbf{수험에 실패한 사람도 다시 일어설 수 있는 시스템의 완성이죠.}
우리가 모든 게임에서 이길 수는 없습니다. 승승장구하는 사람도 한번은 지게 되어있습니다.
중요한 건 졌을 때 그대로 쓰러져있지 않고 어떻게 일어나느햐는 법을 배우고 실천하는 것입니다.
\vspace{5mm}

이런 점 때문에 비공개 게시판이 늘어나는 것이고, 적어도 외부인이 보기에 친목성이 강화되는 것으로 비치는 건 양해해주셨으면 합니다.
챗방조차도 사실 '안심하고 대화할 수 있도록' 하려면 자격요건을 제한해야하는구나를 확실히 느낀 새벽이었습니다.
미리 말씀드리지만 관리자 허님을 포함해 몇몇 사람들은 대화 참여자들의 ip를 확인할 수 있습니다.
오늘 새벽에도 제가 호출했을 때 말없이 나간 분도 예외가 아닙니다.
이런 불편한 일을 막기 위해서는 대화방도 재편해서 lv 1 이상이 아니면 $-$ 즉 게시글을 쓰는 등 소정의 절차를 취하지 않으면
대화방의 출입을 제한하는 조치가 필요합니다.
\vspace{5mm}

분명 이 사이트는 작년보다 나아졌습니다.
슬프지만 이건 자본으로부텨 견제를 당할 위험이 크다는 것이죠.
무슨 무협지를 쓰느냐.... 하겠지만 슬프게도 그렇지 않습니다. 제가 이 사이트 와서 경험한 것들도 내막 알아보니 장난아니었어요.
교재에 대한 솔직가감한 평가조차도 업자들에게는 치워버릴 방해물로 밖에 보이지 않거든요.
실제로 수험시장은 '여론'이 중요합니다. A 교재나 B 강의가 좋다고 하면 다들 우르르 소비하죠. 액수도 장난이 아닙니다.
콕콕러들이 선량한 의도로 수험시장을 평가한들 그건 타인들 눈에는 "자기 장사를 방해하는" 것으로 밖에 안 비칠 겁니다.
\vspace{5mm}

돈에 눈이 먼 사람들은 이 사이트는 좀 피해주셨으면 좋겠습니다.
그리고 이거 나름대로의 사이트 문화로 정착시키겠지만
피해야 할 강의나 교재를 직접 언급하는 경우는 없을 겁니다. 무조건 그 경우는 \textbf{"언급제외",} 즉 볼드모트급으로 분류할 것이니까요.
바보가 아닌 이상 자기 상품이 언급제외되었다고 나서는 업자 분은 없을 것이라고 믿겠습니다.
추천해야 할 강의나 교재도 거의 공짜거나 가성비가 좋은 것으로 한정하는 쪽으로 조정할 것입니다.
\vspace{5mm}

+
\vspace{5mm}

챗방 이용자 분은 처음보는 닉이거나 말없는 경우는 말을 걸어보시길 바라고.
왕관이용자 분들이나 lv이 좀 높으신 분은 수상한 닉이 있으면 제보해주시길 바랍니다.
씁쓸하지만 이런 장치가 아니면 안전한 챗도 불가능해진 상황이 오는 것 같군요.
대화방에서 제재하는 방향은 "벙어리" 아니면 "ip밴"이온데
벙어리는 사실상 대화 내용을 열람할 수 있어 ip 밴 밖에 없습니다. 퇴장기능이 따로 있는지는 모르겠습니다만.
결국 번거로운 걸 막으려면 가장 많이 쓰는 콕방을 일정 레벨 이상으로 자격제한하고 그런 식으로 가는 게 불가피해보입니다.
\vspace{5mm}

+
\vspace{5mm}

오늘 새벽에 대화방에서 제가 말을 걸었던 "x토x" 회원은 바로 탈퇴했군요.
그냥 말을 걸고 자기 소개를 하라고 했을 뿐인데 바로 탈퇴라.
물론 그 전에 ip로 대략 어느 지역인지는 확인해보았긴 했습니다만.
\vspace{5mm}






\section{여러가지}
\href{https://www.kockoc.com/Apoc/556008}{2015.12.21}

\vspace{5mm}

\item 1. 실모, 그리고 교재
\vspace{5mm}

그 저자들조차도 시중에 있는 문제집을 넘어서 과거 본고사 것이나 일본 것까지 다 연구했을지는 의문입니다(저야 보유만 했습니다만)
쎈이나 EBS 문제가 안 좋다라고 하는 분들에게는 블라인드 테스트 던지고 싶음.
\vspace{5mm}

일단 실모는 "수리적 사고가 잘 잡힌 친구"들에게는 좋을 수도 있습니다. 하지만 그게 안 되어있으면 보지 말아야 합니다.
이 사이트와 관련있는 일타삼피 $-$ 고득점 N제 $-$ 조차도 교과서나 시중 기본서 양치기가 안 된 친구들은 \textbf{보면 안 됩니다}.
\vspace{5mm}

일타삼피를 포함한 실모는 실제로 참신해보일지 몰라도 본질적으로는 "학원가" 냄새를 못 벗어나죠.
저는 이런 걸 흔히 '사파'라고 부릅니다. 물론 사파 내공을 쌓아도 시험에 합격하기만 하면 나쁠 건 없지요.
그런데 정파 $-$ 즉 교과서나 기본 틀이 잡혀있는 친구들이라면 실모의 독을 마셔도 소화시키기나 하지
그게 안 된 친구들은 상담해보면 안 본 실모가 없는데 기본적인 개념을 물어봐도 답을 못 하는 경우가 있습니다.
\vspace{5mm}

문과 이야기가 있는데 사실 적중도는 안 좋죠. 30번 제외한 나머지는 그냥 쎈으로 넘치고, 30번 자체는 고1 수학 꽝이면 못 풉니다.
상당히 많은 친구들이 격자점만 나올 거라고 생각하고 그것만 대비하다 털린 케이스도 있었을 것이고
이과 수학은 더욱 그렇습니다. 30번 문제는 아예 사파적인 것은 대놓고 꺼져라고 외치지 않던가요?
문이과 불문하고 고득점 킬러문제는 '패턴'이 먹히지 않고, '생각'을 여러번 해야하며, 매 과정마다 정확한 식과 그래프 전개 요구한다.... 였습니다.
무릇 어떤 교재가 좋다고 하려면 근거는 분명히 대야한다고 생각합니다.
물론 무조건 "좋다"라는 썰만 듣고 잘못된 수험 전략을 짜는 건 본인 책임입니다만 치러야 할 대가가 상당히 크네요.
\vspace{5mm}

지학사에서 나온 풍산자 약점공략 시리즈 주목.엄지손가락 듭니다.
구성이 색달라서 저자진들을 보니 으음, 에이스들이고 교재 접근이 매우 훌륭하네요.
그리고 신사고에서는 특작이 부활했습니다.
그리고 마플은 잽싸게 팔리는 모양이더군요.
여기다 EBS까지 추가되고 하면 사실 교재가 없어서 문제가 아니라 시간이 없어서 골치아플 것입니다.
\vspace{5mm}

\item 2. 인강
\vspace{5mm}

인강의 문제는 지나치게 길다는 것입니다. 한 코스를 완료하려면 하루에 3시간이어도 기본 2주일까지 가는 경우가 많습니다.
하나하나 필기를 다 해야하고 여러번 들어보아야하죠.
그런데 문제는 이렇게 정작 하고 나면 그 지식들이 다시 '증발된다'는 것입니다.
귀로 듣는 것보다는 읽는 것이 20배는 더 빠릅니다. 반복해서 읽고 암기하는 것이 학습의 정도죠.
사실 이 때문에 인강을 들을 때나 아하 하는 사람이 점수는 잘 나오지 않는 것입니다. 인강듣다보니 복습하고 문풀할 시간과 체력이 날라가죠.
\vspace{5mm}

인강은
$-$ 과목의 감이 없어서 큰 흐름을 잡는다거나
$-$ 이해가 안 가는 대목만 설명을 듣고 이해한다거나
$-$ 문제푸는 큰 틀을 익힌다거나
이 정도에 국한해서 보는 게 낫다는 생각입니다
굳이 알차게 활용하고 싶으면 mp3만 따서 이어폰 끼고 듣고다니면서 시간 절약하는 방법이 있을 것입니다만.
EBS 수능개념 강의 정도만 듣고 문풀 하시다가 5,6 월 되어서 자기의 약점이나 취약 과목 및 단원에 관한 것만 골라서 듣는 게 낫다고 충고드립니다.
\vspace{5mm}

\item 3. 왕따
\vspace{5mm}

여기서까지 이런 이야기가 들리는데 참 \textbf{미개하고 비겁한 짓}입니다.
물론 왕따를 시켜야만 하는 불가피한 이유가 있다면 모르나, 그 이유는 당당히 공개할 수 있어야죠.
해도 되는 왕따라는 건 그 대상이 주변인들에게 악행을 저질렀는데도 대화가 안 통하는 케이스 정도인데 10대에 이런 케이스가 있나요?
\vspace{5mm}

\item 4. 환경
\vspace{5mm}

서울 강남 한복판에 산다면야 걸어가기만 해도 온갖 문화를 누릴 수가 있죠.
그러나 본인이 독도나 마라도에 산다면?
수험생들은 자기 환경이 얼마나 축복(저주)받았는가를 모릅니다.
공부를 자기만 한 줄 알죠. 70$\%$는 부모님이 해주신 것일텐데 말이지요(특별한 예외를 제외하곤)
부모가 어린 시절부터 투자한 녀석들이야 기출만 풀고 학원만 다니는데 왜 고득점이 안 나오냐 하겠죠.
\vspace{5mm}

공부는 사실 환경이 전부입니다. 환경에 적응한다는 것은 \textbf{사고방식도 달라진다}는 걸 의미하죠.
지적환경 구축을 위해서는 \textbf{독서도 많이 해야하고 공부 잘 하거나 머리를 많이 쓰는 사람들 근처에 있어야 합니다}.
공부 열심히 할거야라고만 부르짖지 말고 환경을 바꾸는 게 좋습니다.
환경을 바꾸라는 건 방해받지 말고 공부할 수 있는 장소와 시간을 확보하는 것도 중요하지만
꾸준히 지적자극을 받을 수 있는 '독서'와 '강의', 그리고 '경쟁'까지 포함하는 개념입니다.
\vspace{5mm}

\item 5. 고레카와 긴조
\vspace{5mm}

헌책방에서 다시 겟해서 읽는 자서전입니다. 일본의 전설적인 투자자 $-$
이 양반은 부모가 부자도 아니고 머리 좋은 사람도 아닙니다니다.
20세기 초 일본의 젊은이들이 그랬듯이 초등학교만 졸업한 뒤 옷가게 사환으로 일하다가
'책'을 읽고 중국$-$유럽으로 건너가 장사를 하기로 마음먹은 게 14살이더군요(...)
정말 홀홀단신으로 건너가 일본군도 따라다니고 들개에게 죽을 뻔하고 굶어죽기 전까지도 갔다가
부기(회계) 실력으로 기회 잡아서 일본군에 납품하다가 나중에는 중국인들의 동전을 녹여판 주괴를 수출해
수억씩 벌어들였는데 쑨원의 혁명군에 투자했다가 지는 바람에 쫄딱 망해서 자살을 생각했던 게 19살 $-$ 미성년.
(주괴 수출이 허가되지 않을까봐 세관장을 권총으로 협박한 장면도 인상적인데 생각해보니 고2가 그랬다는 게 흠좀무)
그 이후로도 대박과 쪽박을 반복하다가 3\textbf{년간 도서관에서 온갖 경제 서적과 자료를 탐독하여 고수}가 된 뒤에
31살에 경제연구소를 신설해 교수까지 제자로 삼고(초졸이면서) 사업하면서 고위층과 연줄 맺고... 그 이후는 알아서 책구해서 보시길.
(이 사람이 우리나라에서 최초로 용광로 고로를 설치했을 것입니다. 그게 포스코 박물관에 있다던데)
\vspace{5mm}

자꾸만 유전거립니다만 그건 별 의미가 없습니다. 본인이 환경을 적극적으로 바꾸는 게 더 중요하다는 일례가 되겠습니다.
나는 한 때 성적이 잘 나왔는데(중학교 때겠지만) 지금은 왜 그럴까 하기 전에 책 한권이라도 더 읽고
최대한 보수적으로 실력테스트해보면서 조금이라도 문제가 있는 건 철저하게 바꾸고 개선하는 편이 나을 것입니다.
\vspace{5mm}

\item 6. 공부를 한다 $-$
\vspace{5mm}

는 것으로는 안 되고 \textbf{미쳐야합니다}.
수학에서 미적분을 공부하면 사회의 모든 현상을 미적분으로 생각해보아야합니다(실제로 그렇게 쓰이고 있지요)
국어에서 이해가 안 가는 지문을 읽는다면 강의에만 의존하지 말고 네이버 검색이라도 해서 관련된 화제들이 어떤가 다 찾고 생각해보아야하고
영어는 아예 외국인에게 내가 구사할 수 있는 대사 목록으로 암기해버려야겠죠.
\vspace{5mm}

머리좋은 사람도 \textbf{"미치도록 좋아하는 사람"}을 못 이깁니다.
과탐의 화학과 생명과학은 "공부를 잘 하는 사람"이 아니라 정말 그 문제에 환장한 '마니아'들 아니면 꺼져라고 소리치고 있죠.
관점을 바꿔 보자면 생1의 경우 열심히 한 사람에게는 억울하겠지만, 본인이 유전성애자(...)였다면 아니 이런 천국이라는 소리가 나왔을지도 모르죠.
\vspace{5mm}

소위 지능지수 천재 $-$ 에 대한 열광은 1990년대에나 유명했던 걸로 압니다만
지금은 nerd의 시대죠.
머리좋냐 안 좋냐보다도 본인이 얼마나 거기에 \textbf{미쳐있느냐}가 더 중요합니다.
단지 고득점을 맞는다... 로는 분명 실패합니다. 중요한 건 내가 그 과목에 얼마나 미쳐있느냐는 겁니다.
\vspace{5mm}

7. 사교육의 미래
\vspace{5mm}

뭐긴요 인류의 기원인 \textbf{아프리카}로 돌아가는 거겠지.
가 아니라 실제로 아프리카 방송 스타일이 '학습효율' 면에서도 훨씬 낫죠.
지금 가장 앞서나간 게 EBS 인강이죠.
사설 인강은 처음부터 끝까지 다 들어라, 강사님에게 세뇌당하여라, 교재에 애정을 품고 5000원짜리 50000원에 사라 하는 거면
EBS  인강은 발췌해 들을 수도 있고 콜라보 강의도 있지만
\vspace{5mm}

이 어느 쪽도 피드백은 약하죠.
\vspace{5mm}

전국에 입시고수들은 늘어나니 머지 않아 그 사람들이 출판사와 모종의 협약을 맺거나(사실 맺을 필요가 있나, 부수 늘려주는데)
직접 아프리카로 문제풀이를 하면서 모르는 것 설명해주고 피드백받고 별풍(...) 받는 식으로 가겠죠.
피드백도 피드백이지만 채팅창에 여럿이 들어온다는 것부터가 이미 학원분위기 연출인지라.
그러고보니 콕도 어차피 대화방 있으니까 나중에 일격, 일타 저자 분들이 공지하면서
채팅창 띄우고 같이 문제푸는 타임 갖는 것도 좋을 것 같긴 한데. 생각해보니까 이거 가능하잖아?
\vspace{5mm}





\section{괴담}
\href{https://www.kockoc.com/Apoc/556127}{2015.12.21}

\vspace{5mm}

흡혈귀나 좀비 화의 문제 $-$
\vspace{5mm}

흡혈귀나 좀비가 하루에 2배씩 개체를 늘린다고 한다, 한달 뒤는?
2의 30제곱은 약 1000,000,000 = 10억
사실 흡혈귀나 좀비 자체가 과학적으로도 말이 되지 않지만.
\vspace{5mm}

\item 1. 수학괴담
\vspace{5mm}

하$\sim$상위권 통틀어 한문제 푸는 데 평균 5분이라고 치자. 어려운 문제와 쉬운 문제를 합쳐서 대략 가정
그럼 2000문제라고 하면 10,000분 = 대략 166시간이 나온다.
그럼 하루에 수험생들이 공부할 수 있는 수학시간은 2시간이라고 가정하면 이건 83일이 된다.
일주일에 6일 공부한다고 하면 14주이고, 그렇다면 대략 3개월이다.
자, 그렇다면 쎈만 하더라도 대략 7$\sim$8000문제가 되는데 그럼 1년.
그런데 내가 듣는 괴담은 xx고 애들은 쎈, RPM, 라벨 다 푼다... 인데
그렇다면 2000문제가 아니라 15,000문제를 넘어가는 꼴인데 그럼 몇년이 걸린단 이야기인가.
\vspace{5mm}

이게 시사하는 논점은 꽤 많다.
첫째로 한문제에 5분당 걸릴 일은 없다. 많이 풀다보면 시간이 상당히 단축. 그러나 고난이도를 대비하면 한계가 있으리라
둘째로 실제로 시중교재까지 저렇게 제대로 다 풀어대는 경우는 별로 없다. 만약 있다면 그건 문제가 중복되어서 시간이 별로 안 걸려서이다.
셋째로 저런 괴담은 공부하지 않는 학부모들이 자녀 압박용으로 퍼뜨린다/
\vspace{5mm}

이렇게 수치화해서 접근하다보면 말도 안 되는 괴담들이 있다는 걸 알게 된다.
그렇다면 그렇게 괴물적으로 처리한다는 xx고 분들이 다 입시 결과가 좋으신가?
\vspace{5mm}

\item 2. 실모
\vspace{5mm}

의아스럽운 것 : 실모가 좋다, 많이 팔린다라는 이야기는 듣지만
정작 검증해보면 정작 제대로 적중한 적은 없는데다가, 그렇게 많이 팔리는 데 왜 '많이 성공은 못 하시느냐'이다.
가령 10,000부가 팔린다면 그럼 그 중 몇명이 드라마틱하게 올랐다거나 그 검증이 있어야 하지 않나. 그런데 그런 적은 단 한번도 없다.
\vspace{5mm}

그것도 그렇지만 실모 양치기 이야기가 나와서 그렇다면 $-$ 실모 1만원당 4회분 치면 여기서 의미있는 문제는 1회당 3문제 정도
그럼 1만원에 12문제 정도가 유의미하단 얘기다, 나머지 문제야 시중교재들이나 기출에 널려있다.
한 문제당 1,000원인데 이것들이 적중한다면 그리 비싸지는 않지만 적어도 실모 양치기란 말은 뭔가 이상하다,
풀 대신 고기를 뜯어먹는, 웬지 개처럼 날렵한 양들을 몰다가 알퐁스 도데의 별 같은 분위기에서 아가씨 대신 아저씨를 만나는 기분?
\vspace{5mm}

수학문제 풀 때에만 논리적일 게 아니라(아니 그것도 논리적이지 않은 것도 문제지만)
그냥 실모 좋다 나쁘다 떠나서 이런 것도 논리적으로 따져보면 좋을 것 같은데 수년 째 이런 논의가 안 된 것 자체가 신기하다.
모처에서 모 강사 교재 비싸다... 라는 논의보는 기분임. 메시지는 좋다 그래, 그런데 그 메시지와 메신저가 모순이면 이상하지 않나?
\vspace{5mm}

\item 3. 모두가 의사가 된다면
\vspace{5mm}

그나마 의료계통은 공급통제가 이뤄지고 있다고 하지만 이것도 의아스러운 건 많다.
흔히 하는 이야기가 고령화 덕분에 노인들 시장으로 한 의료산업이 발달할 것이다...
그런데 이건 뭐 흙파서 장사하는 것도 아니고 가장 중요한 전제가 있지. 그 노인들이 '돈'을 지불할 의사와 의향이 있나.
막연하게 고령화라고 표현하면 그럴 듯 하지만, 실제로 2$\sim$30년 후 호구가 되어주실 수 있는 부유한 노인분들이 얼마나 되나가 중요하지 않나.
\vspace{5mm}

6$\sim$70년대 사람들은 이렇게 생각했을 것이다. 나라가 부유해지면 결혼을 많이 하고 출산률이 늘어나니 이 분야로 투자하자.
지금 현실은 어떤가? 혹자는 가난해서라고 하지만 사실 195$\sim$60년대와 비교해본다면 아무리 헬조선이니 뭐니 해도 비교될 수가 없다.
지금 저출산인 이유는 간단히 말해서 '민주주의', '남녀평등', '자아실현', '개인주의'다.
이게 정당하고 정당하지 않고를 떠나서, 사람들 가치관이 더 이상 가족중심이 아니며 출산과 육아 자체를 '행복에 반하는 것'으로 보고 있어서디ㅏ.
\vspace{5mm}

그럼 앞으로 의사가 무조건 잘 나갈 거라고 하는 것도 이런 식의 섬세한 검증이 필요하지 않나?
내일의 주식시장도 모르는데 먼 미래의 흐름이라는 것을 정확히 예측한다는 건 어렵다.
그나마 선형적이고 기술적인 예측 하나만으로 보는 건 정확성이 있다 해도 '개인의 장래'와 관계된 건 세부적이고 미시적인 건 예측불가이다.
\vspace{5mm}

\item 4. 인생 재단
\vspace{5mm}

자식이 재수하거나 삼수하는 등 20대 때 실패하는 경우에 비아냥을 많이 받는다.
그런데 그 비아냥거리는 사람들은 왜 위인전에 으레 등장하는 실패 에피소드에 대해선 말을 하지 않을까.
성공한 사람들을 보면 정말 죽음 직전까지 실패한 경우도 많다.
오히려 큰 성공 직전에는 무시무시한 실패가 있는 패턴이 많지 않나.
\vspace{5mm}

중요한 건 실패를 했다기보다도 그 실패한 시점에서 가만히 무릎꿇고 노느냐, 아니면 그걸 \textbf{극복하려 하느냐 그게 아닌가}.
좋은 기회는 아무 것도 아니면 위기가 되지만, 위기는 잘 대응하면 기회가 되는 것이라는 건 표어가 아니라 실증 사례이다.
다들 정주영 이병철 돈 많다 어쩌구저쩌구만 하지 그 사람들이 정작 겪었던 실패나 불행, 그리고 뭔가 새로 시작할 때 받은 비아냥은 신경쓰지 않는 듯.
\vspace{5mm}

사실 성공하는 사람들을 찾는 방법은 다음과 같지 않을까.
\vspace{5mm}

첫째, 남들과 다르고 터무니없는 것을 공부하고 준비한다.
둘째, 적극적으로 일을 벌이면서 실패를 자주 한다. 그런데 실패해도 일어나려 한다.
셋째, 그를 비웃는 사람들이 대단히 평범하고 멍청하다.
\vspace{5mm}

성공을 미래형으로 보느냐 과거형으로 보느냐 그 차이다.
성공'한' 사람에게 박수가 의미가 있을까, 성공'할' 사람에게나 의미가 있지.
하지만 대중들은 성공'한' 사람만 쳐다본다.
\vspace{5mm}






\section{학생 모의고사는 단 한번도 검증된 적이 없죠.}
\href{https://www.kockoc.com/Apoc/558076}{2015.12.22}

\vspace{5mm}

http://kockoc.com/column/556875
\vspace{5mm}

첫째, 저 짤방의 학생은 실모를 푼 게 아닙니다. 사설모의고사들을 많이 풀었던 것이죠(...)
현재 언급되는 실모들은 학생 모의고사들이 주류입니다.
글쓴이가 제목대로 검증되었다고 하려면 저 여학생이 학생모의고사들을 300회 풀었느냐를 보여줘야죠.
둘째, 글쓴이께서는 정작 이과 수학을 모른다고 하시더군요.
정작 본인께서 이과수학을 공부해보시고 갖가지 교재들을 풀어보시고 저런 말씀을 하시면 모르겠습니다만
그렇지도 않은 데 그렇게 말씀하시는 건 상당히 문제가 많습니다.
\vspace{5mm}

그래서 저 글은 제목부터 고쳐야합니다.
일단 글쓴이께서는 수험도 수험이지만 '검증'이 무엇인가, '근거'가 어떤 건가 그것부터 확실히 하셔야하지
이과 수학도 모르시고 거기다가 모의고사를 많이 풀었단 방송 인터뷰를 "학생모의고사를 양치기하면 된다 검증되었다"하는 글을
실모양치기 효용검증사례(이과편)이라고 제목을 붙이는 건 문제가 있다고 지적드립니다.
\vspace{5mm}

문과수학에서 글쓴이 성적 이상을  '실모 양치기' 안 하고 거둔 사례도 있습니다.
어떤 암이 있다 칩시다. 허강탕을 먹어서 나은 케이스도 있고, 허강탕을 먹지 않고 나은 케이스도 있습니다.
그런데 허강탕을 먹어서 나은 케이스가 허강탕이 가격창렬이지만 최고다라고 하다가
허강탕은 먹되 그 이전에 항암수술은 받아야지... 라고 하면 그건 매우 우스운 논의가 될 것입니다.
\vspace{5mm}

그리고 이 사이트가 왜 실모에 비판적이냐... 그냥 이 사이트 혼자 '정상'일 거라는 생각은 안 하시나 모르겠습니다.
실모찬양론이 올라오는 건 어른의 사정이 있죠.
첫째, 학생들이야 다 보지 않았으니 실모가 좋다고만 말하겠죠.
둘째, 실모들을 판매하는 사람들이 그럼 실모가 나쁘다라고 말할 리는 없죠.
그런데 이번 칼럼란에 재밌는 일이 벌어졌죠.
\textbf{미래 허강탕 판매업자이자 현재 실모 저자인 분이 실모에 매우 비판적인 입장을 취하시고}
반면 글쓴이 같은 분께서는 실모를 많이 풀면 된다라는 이야기를 하십니다.
사실 이 정도면 교통정리는 된 것 같습니다만.
\vspace{5mm}

실모든 사설인강이든 적당히만 이용하면 나쁠 거야 없죠.
그런데 문제는 별로 검증되지도 않았는데 무작정 좋다라는 이야기 때문에
1등급을 받지 못 하는 다수의 중하위권 학생들이 그 호구가 되어서 자신의 공부까지 망치는 경우가 많다는 것입니다.
흔히 실모찬양자들은 이렇게 얘기하죠. "뭐 1등급 안 나오는 애들은 기본교재부터 보라고"
\vspace{5mm}

그런데 말입니다. 실제로 '1등급' 대상으로만 판매할 리는 없잖습니까.
1등급 대상으로만 한다면 돈을 벌 수 있을 리가 없지 않습니까. 1등급이 안 나오는 친구들에게도 팔아야 재벌이 되는 것 아닌가요?
그럼 학생저자가 아닌 실제 교사, 학원강사 및 원장, 박사급 이상은 능력이 없어서 어려운 문제집을 안 내셨을 것 같나요.
그 분들은 현재 학생저자들 저리가라할 실력의 소유자들입니다.
어렵게 못 내는 게 아니라 더 많은 독자들을 위해서 중간 정도의 난이도로 내신 것입니다.
무엇보다 가장 중요한 건 그 분들이나 그 출판사들은 허위과장광고 따위는 안 한다는 것이죠(이게 뭘 시사하는지는 아실 것입니다)
만약 그 분들도 "이 모의고사만 보면 1등급이 나온다."라는 식의 광고를 하는 게 안 부끄러웠다면
훨씬 더 어려운 문제를 실은 모의고사를 내서 버셨겠죠.
\vspace{5mm}

학생모의고사라면 그냥 풋풋한 아마추어리즘이 생명력이어야하는 것이 아닌가 싶은데
마치 지금은 \textbf{"공인된 필수과정"}처럼 인식되었다는 게 문제입니다.
그렇다고 적중이 되느냐 물어보면 올라오는 답변은 "그러니까 걍 실전경험을 누리기 위해서"이라는데
실전경험이면 그냥 다른 저렴한 파이널 풀거나 시간재서 풀거나 복사집에서 학원사설모의고사(이것도 복붙성이 많지만)들을
제본받아서 풀어보아도 충분한 겁니다.
\vspace{5mm}

그리고  검증.
여라가지 차원이 있지만 적중도로만 보자면 그냥 역대 기출 '분석'만 해보면 됩니다.
시중에 어려운 문제가 없어서 학생모의고사를 풀어야한다는 논거의 맹점이 여기서 드러나는데
사실 수능 기출에서 문제가 되는 것들은 단순히 어려운 게 아니라, "패러다임이 다른 문제"를 냈다는 것입니다.
가령 적군이 화살을 쏠 거라고 생각하고 갔는데 레이저총을 들고 온다거나
이번 미술전에서는 정물화로 승부보아야지 하고 갔는데 갑자기 추상화를 그리라고 한다거나 하는 게 문제였습니다.
학원가 모의고사나 문제집은 보통 창의성이 없습니다.
A라는 기출이 있으면 그 A만 가지고 A', A", A"', A"'' 이렇게 꼬아내죠. 당연히 이렇게 가면 어렵습니다.
그러나 실제 수능기출은 A 시리즈를 기대하고 간 학생들에게 B라는 문제를 냅니다. 그러니까 못 푸는 것이죠.
\vspace{5mm}

학생모의고사에 대해서는 저는 이 글 보는 저자들이 혹시 상처(?)라도 받을까봐 해서 말 아낍니다만
사실 일필까지를 포함해서 과연 기출의 그 참신한 패러다임 쉬프팅까지 간 경우는 없다고 생각하고 있습니다.
추앙받는 실모들을 몇달 전에 서재정리하다가 과감히 버렸는데 도대체 그것들이 왜 추앙받는지는 3일동안 보아도 모르겠더군요.
그냥 아마추어리즘 학생모의고사라고 하면 어, 해설이 이렇게 부실하고 문제도 꼬아낸 정도는 인정할 수 있다.. 정도였지만
정말 이게 수능에 도움이 되는 걸까라고 하면 고개를 갸웃거릴 수 밖에 없겠더군요.
\vspace{5mm}

...
\vspace{5mm}

그리고 수학문제가 과거보다 쉬워졌다....
"어렵다"와 "쉽다"의 구분 자체가 참 애매하고 부정확한 것 같습니다.
쉬워졌다고 하면 그 어려운 방식으로 공부한 사람들이 무조건 다 96, 100이 나와야겠지요.
하지만 실증 사례를 보면 평소 모평에서 잘 나오거나 실력자라고 하던 친구들의 결과가 기대 이하인 경우가 많습니다.
혹자 이걸 실수라고 할 수도 있겠지만 우연도 반복되면 필연으로 수렴하는 것입니다.
\vspace{5mm}

사실 수학문제가 쉬워졌다... 라는 것은 어느 한가지 주장에 불과한 것이지요.
중간과정을 생략하고 말하면 요즘 수학은 쉬워진 게 아닙니다.
단지 문제를 꼬아서 내는 게 아니라, "근본적이고 기초적인 과정으로 풀어야만 패러다임 쉬프팅에 대처할 수 있게 낸다"는 것입니다.
예를 들자면 그 사람들은 멍게새우쵸콜릿맛이 나는 짜장면을 요구하지 않습니다.
그냥 짜장 아이스크림을 요구합니다. 다만 이 짜장 아이스크림은 짜장면의 기본을 잘 지키면 만들 수 있습니다.
\vspace{5mm}

점수가 깎이는 이유는 두가지입니다.
\vspace{5mm}

하나는 21, 29, 30에서 내는 생각하는 문제가 수리적 사고가 잘 박혀있지 않으면 풀 수 없게 낸다는 것입니다.
수리적 사고가 체화된 친구들에게는 그리 어렵지 않은데, 그냥 어렵다는 문제를 컬렉팅하는 친구들에게는 매우 어렵습니다.
풀이과정을 보면 생각보다 간단하죠. 하지만 시험장에서 떨리지 않고 이 신문제를 풀어낸다는 건 컬렉터들에게는 난감합니다.
\vspace{5mm}

다른 하나는 터무니없는 실수입니다.
21, 29, 30을 풀어댄다는 친구들이 쎈수학 B등급도 안 되는 것에서 터무니없는 실수를 해서 감점당합니다.
기본적인 개념과 연산의 문제인데, 이런 것은 쎈이나 RPM 같은 것만 충실히 했으면 그냥 충분히 대비할 수 있었지요.
그런데 이 어느 쪽이든 학생 모의고사를 꼭 풀어야만 대비된다... 라는 건 상관없습니다.
오히려 학생 모의고사에 집착하거나 그것들에 의존해버리면 악화시킬 수 있다라고 할 수 있죠.
\vspace{5mm}

앞의 것을 대비하려면 어려운 문제를 잡다하게 풀 게 아니라
시간 제한 걸어놓고 4점짜리를 스킬없이 순수히 교과서상 개념으로 본인이 '과정'을 분설해보는 훈련을 하면 됩니다.
자기가 생각 못 했던 새로운 유형을 "교과서"에서 배운 것으로만 도전해보는 것이 중요한 것이지
"많이 풀다보면 그래도 유형의 교집합이 있어서 대처가능할까"라는 식은 먹히기 힘들죠.
\vspace{5mm}

뒤의 것을 대비하려면 역시 시중교재로 양치기를 하면 됩니다.
학생 모의고사들은 생각보다 빠진 게 많아요. 출제 경향 좆는다고 하다가 어 F 개념은 나오지 않아라고 하다가
정말로 F 개념이 나와버리면 다 침묵해버리는 게 현실입니다. 이건 유형이 다 망라된 교재로 양치기하는 게 낫습니다(가격도 착하죠)
\vspace{5mm}

+
\vspace{5mm}

더불어 노골적으로 말하면 모의고사라는 것들도
창작한 문제를 공유하고 평가받고 싶어서 쓴다.... 라가 아니라 그냥 $_$ 이거 아닌가요?
돈을 버는 건 나쁘지 않습니다. 아니 오히려 제대로 문제를 만들어서 효용을 준다면 재벌이 되든 뭐하든 그건 욕먹을 게 아니죠.
문제는 터무니없이 비싸다는 것, 그리고 '아마추어리즘'으로 도피입니다.
가령 학생모의고사라서 이윤추구 동기가 적다면 대단히 싸게 팔 것입니다.
돈욕심이 없다면 문제가 다소 표절끼가 있다거나 해설이 엉터리여도 양해할 수 있죠.
그건 정말 아마추어리즘이니까요.
\vspace{5mm}

반면 값이 비싸지만 프로의식을 추구한다면 품질이 좋아야합니다.
정말 문제가 순수 창작이고 하나하나 개발하는데 엄청난 시간이 걸리며 해설도 백종원 요리방송만큼 이해할 수 있어야 하죠.
돈값을 한다면야 비싸게 팔든 누가 뭐라고 하겠습니까.
그런데 문제는 돈버는 건 프로인데 품질이 아마추어리즘이라는 것이죠.
다시 말해서 \textbf{돈은 프로처럼 벌겠다는 건데, 품질 문제가 지적되면 아마추어로 도피한다는 것}입니다.
\vspace{5mm}






\section{해설은 읽는 것이죠}
\href{https://www.kockoc.com/Apoc/559722}{2015.12.23}

\vspace{5mm}

수학의 해설을 보는 게 문제느냐 하는데 사실 그건 "읽는 방법"의 문제가 있습니다.
이런 질문들을 해보셨을 것입니다.
사실 유명강사의 강의 내용이나 해설은 그리 큰 차이는 없는데(접근방법 차이나 디테일은 있을지 몰라도)
\textbf{강의를 듣는 것은 괜찮고, 해설을 읽는 건 안 되느냐 말인가.}
\vspace{5mm}

읽는 것은 \textbf{대상을 입체적으로 조명한 뒤 논리적으로 타당한 순서로 밟아가는 과정}입니다.
\vspace{5mm}

수학 강의는 똑같은 내용을 더욱 상세히 쪼개서 '순서'대로 납득이 가도록 이야기해줍니다.
그 문제의 취지가 무엇인지 설명해주면서 기본 정의, 성질, 공식에서 어떻게 실마리를 잡아 풀어가는지 \textbf{순서대로} 이야기해줍니다.
그렇기 때문에 학생들은 '읽으려는 시도'를 하지 않아도 그 '절차와 과정'을 주입받을 수 있습니다.
그래서 강의가 도움이 된다는 것인데 이것도 일정 시점에서는 한계에 부딪치겠죠. 남이 생각해준 것이지 자기가 생각한 게 아니라서 그렇습니다.
\vspace{5mm}

반면 교재 해설은 지면과 분량의 한계상 풀이에 도움이 되는 내용을 '압축'해서 썼기 때문에 그걸 바로 알 수가 없습니다.
문제의 의도라든가 요건 해석 → 관련된 교과서상 개념 찾기 → 이용할 수 있는 조건과 단서 → 알고리즘 구성 같은 게 간략히 나와잇죠.
그래서 본인들이 읽을 실력이 없거나 읽으려는 노력을 하지 않고 해설을 보면 "푸는 패턴"으로 전락해버리는 게 해설입니다.
하지만 본인들이 읽으려고 하면서 그 문제나 해설을 쓰는 사람의 의도가 무엇인지 생각하면서 행간까지 쪼개서 순서를 잡으려고 하면
강의 이상의 무엇인가를 선사해줍니다.
\vspace{5mm}

요컨대 해설을 \textbf{본다}와 \textbf{읽는다는} 건 다릅니다.
단순히 보려는 자에게는 해설은 쓰레기일 수도 있습니다.
그러나 읽으려는 자에게는 해설은 무궁무진한 소스가 될 수 있죠.
단순히 보려거나 들으려는 자는 사실 아무 것도 알 수가 없습니다.
그러나 독해하거나 청해하려는 자는 쓰레기 교재를 금은으로 바꿀 수 있습니다.
\vspace{5mm}

해설을 읽지 않고 무작정 풀어대니까 되었다... 그리고 칭찬하는 댓글이 달렸는데 글쎄요 모르겠습니다.
만약 교재 해설이 정말 문제가 많다면 그래도 됩니다. 최소한 제가 자비로 구입해 명성대로인가 확인해보았던 학생 실모들 몇몇은
해설을 차라리 안 보는 게 낫다고 생각하는 경우가 있었으니까 그래도 되었을지 모르니까요.
또한 온갖 문제를 다 모아내서 보충용으로는 좋은 RPM의 경우도 도대체 이건 해설이 맞냐하는 경우도 있습니다.
하지만 그게 아닌 쎈이나 다른 교재들 $-$ 즉 대중적으로 많이 팔리고 저자진들도 검증된 경우의 해설은 그리 나쁘지 않습니다.
지금 서점에 등장하는 유명한 기출 문제집들도 해설을 잘 꼽아 씹으면 인강 못지 않을 것입니다.
\vspace{5mm}

그럼 수학문제를 풀기만 하면 되는가. 아닙니다, 푸는 데 성공한 경우라도 반드시 해설과 비교해서 읽어보아야합니다.
그래야 자기가 어디서 부족한가 또는 논리적 문제점이 있는가 없는가를 확인해볼 수 있습니다.
풀지 않고 해설을 보는 건 어리석은 짓입니다.
5번 정도 풀어보는 시도를 하고 그 실패의 과정을 남긴 다음 해설과 비교하며 읽어야 합니다.
해설을 수천번 보아도 깨달을 수 없습니다. 그러나 해설을 1번 제대로 읽으면 깨달을 수 있습니다.
본다는 건 현상을 그냥 긍정하고 아무 생각도 안 하는 것입니다.
반면 읽는다는 것은 현상을 의심(부정)하고 그래서 "왜?"라는 물어보는 것입니다.
회의하고 부정하고 가정하고 하면서 현상을 해체하다보면 '납득할 수 밖에 없는' 명제 단위까지 도달합니다.
참과 거짓이 분명히 드러나는 단계까지 가면서 성장하는 것입니다.
\vspace{5mm}

그렇기 때문에 역설적으로 수학실력은 수학공부가 아니라 국어공부를 통해서 늘어난다고 할 수도 있을 것입니다.
중학교 때 성적이 좋다가 고등학교 때 추락하는 현상을 설명하는 공식으로 "수학 = 국어 + 산수"를 들기도 합니다.
연산하고 답을 낼 줄만 안다면 그건 산수이겠지요. 사실 다수가 수학을 산수로 착각하고 있습니다.
반면 수학은 왜 그런 답이 나오느냐 하는 과정, 그리고 그 과정은 왜 나왔느냐하는 발상, 또한 발상은 어디서 나왔나 ... 묻는 과정입니다.
\vspace{5mm}

한국이 안전사고에 둔감하다라는 것을 가지고 기성세대만 욕할 것도 없습니다.
안전사고라고 하면 고속성장이니 불감증이니 말이 많습니다만 간단합니다. "논리적 지침을 안 지켜서"입니다.
우리나라 사람들이 문제인 것은 빨리 빨리가 아니지요. 빠른 것은 좋은 것입니다.
빨리 빨리가 문제가 아니라 '지켜야 할 것을' 안 지키고 skip해버리기 때문입니다. 그러니 매주매주 버라이어티한 사건사고가 터지는 것이죠.
작년에 선박 침몰 사건이 있었습니다. 그런데 이 사건과 관련된 것 어느 것 중에도 "체계적인 논리"란 것은 없습니다.
문명사회의 모든 사건은 사람이 만든 것입니다. 그럼 그 사람들은 어떻게 공부했을까요?
수학이란 과목에서 "풀면 된다"에만 집착하는 사람들은 그냥 평범하게 살아야합니다.
이런 사람들이 요직에 앉거나 큰일을 벌이면 분명 많은 사람들을 다치게 할 게 분명하기 때문입니다.
\vspace{5mm}

+
\vspace{5mm}

수학은 본디 유럽들 것이었지 우리 것이 아니었습니다.
중국인들은 순환론적인 특정이데올로기(ex 음양론)로 모든 것을 정당화하려 했습니다. 그래서 동양은 2000여년 발전이 지체됩니다.
하지만 그리스인들은 항상 부정하고 의심하고 회의했습니다. 기독교도 신이란 어떤 존재인가 논쟁을 하고 싸우고 그랬습니다.
그래서 16세기 경부터는 이미 비교할 수 없는 격차가 벌어집니다.
\vspace{5mm}

자기비하도 필요할 때는 해야합니다. 우리 전통에 수학은 없다고 단언할 수 있습니다.
회의하고 성찰하고 검증하는 것이 없기 때문에 수학을 배워서 '근대인'이 되는 것입니다.
근대인 = 즉 시민이 되기 위해서는 자본과 교양이 필요합니다.
많은 학생들이 이야기하는 흙수저 금수저는 '자본'이죠. 하지만 자본만 갖추면 뭐합니까. 생각할 줄 모르는데
다들 인터넷 글, 댓글만 보고 베끼고 짜깁기하면서 그게 '교양'이라고 착각하겠지요.
그런 교양의 기본으로서 수학을 배우는 것입니다.
\vspace{5mm}

문제를 많이 풀어댔다 어떤 것이든 풀 수 있다... 라고 말하는 친구들은 나중에 분명 큰 낭패를 볼 것입니다.
이건 입시 수준이 문제가 아닙니다. 사고, 행동 방식의 문제입니다.
자기가 밟고 있는 과정이 정말 문제가 없는가, 합당한가 따지지 않고 답만 구하면 된다..
이런 방식으로는 분명 초기 성공을 거둡니다. 실속 위주로 과감하게 승부하는 사람이 아무래도 승률은 높으니까요.
그러나 중요한 필수과정들을 스킵해버리면 챌린저호 꼴이 나버립니다.
분명 입시를 위해서는 점수가 가장 중요한 기준이겟지요.
하지만 명문대에 가는 것이 "더 크게 망하기 위해서"라면?
가능하면 "근대인"이 되는 방향으로 공부하는 게 바람직한 것이고
원래 우리 한국의 것에는 수학이 없으니 수학을 제대로 공부해서 그런 결함을 보완해야겠다고 해야할 것입니다.
\vspace{5mm}

입시에서 목적을 달성하지 못 한 경우도 아 나는 모자라단 말인가... 라는 감상은 그냥 한달 전에 끝냈어야하는 문제고
냉정하게 자기가 왜 실패했나 수학문제를 차분히 풀 듯이 그 해를 발견해나가면 되는 것입니다.
지금도 공부가닥을 못 잡았다면 역설적으로 그건 본인들이 입시수학을 풀 줄만 알지, 그걸 체화시키지 못 했다는 이야기이겠죠.
\vspace{5mm}

++
\vspace{5mm}

사실 이게 가장 중요한 것인데 뒤늦게 언급하자면
\vspace{5mm}

가애든 동네학원이든 인강이든
"그냥 이렇게 풀면 된다"라는 인스턴프 풀이야말로 잘못 가르치는 것입니다.
\vspace{5mm}

패턴 문제풀이라는 건 인스턴트 풀이가 가능합니다. A라는 문제가 있으면 ⓐ로 푼다, 이런 문제이죠.
그리고 사교육에서는 대개 이렇게 가르칩니다. 간편하고 쉬워서 다들 선호합니다.
그러나 수능 경험해보신 분은 아시죠. 어려운 3점이나 4점에서는 인스턴트 풀이가 안 먹힌다는 것을요.
\vspace{5mm}

ㄱ에서 ㅎ까지 가려면 ㄱㄴㄷㄹㅁㅂㅅㅇ ... ㅈㅊㅋㅌㅍㅎ 까지 \textbf{순서}대로 밟는 게 맞습니다.
그런데 이게 매우 번거롭기 때문에 "자음"이라는 개념으로 퉁치고 ㄱ$\sim$ㅎ로 요약하는 것이지요.
하지만 이런 요약을 하려면 먼저 자음체계를 제대로 배우고 납득하여 머릿 속에 체계를 단련시킨 다음에 가야합니다.
문제는 이렇게 교육시키는 경우가 적단 것이죠. 학교에서 엉터리로 대충 가르치는 경우도 많고 학원도 가애도 예외는 아닙니다.
그나마 인강은 다수가 보고 평가하기 때문에 덜할지 모르나 사실 심사하면 문풀에만 치중해 대충 넘어가는 경우가 많기도 하지만
저렇게 체계적으로 가르치면 '재미'가 없고, '재미'가 없으면 수강생이 줄어든다는 점이 가장 큽니다.
\vspace{5mm}

그래서 수학을 잘 한다 = 인스턴트 풀이가 가능하다... 라고 착각하는 수험생들이 헤매는 것입니다.
무엇이든 그래서 "하나"로 다 끝낼 수 있다는 컵라면스러운 것을 요구합니다.
수학은 컵라면처럼 3분 내에 물끓여 넣으면 완성되어야하는데 왜 연애하는 내 친구는 잘 되고 나는 안 될까
이걸 설명하기 위해 "쟤는 머리가 좋고 나는 머리가 나쁘다"라는 카스트제도 가설을 세우고 믿고 그렇게 망해갑니다.
\vspace{5mm}

수학을 못 하는 이유는 상당수 그렇습니다. '푼다' = '한큐에 해결해야한다'로 착각합니다.
그런데 어떤 수학문제든 한큐에 해결되는 건 사실 없습니다. 2점짜리 계산도 실제로 냉정히 들어가면
그 숫자가 어떤 집합에 속하는가, 그 연산은 xx 법칙이 성립하는가, 주의해야 할 조건은 없나 다 면밀히 따져야합니다.
이런 게 귀찮기 때문에 대충 넘어가는 식으로 가르치는 게 우리들의 잘못된 교육이죠.
수학은 컵라면이 아닙니다.
순서대로 지켜야 할 논리 과정들을 모두 떠올리고 그걸 순서대로 순열, 조합시키는 개념들의 이항정리라고 보면 됩니다.
\vspace{5mm}

공부를 못 한다는 친구들이 노력을 해도 안 되는 건 '논리적 체계'가 안 잡혔기 때문입니다.
체계가 안 잡혔다라는 건 "사고나 행동, 즉 일의 순서"를 못 지킨다는 것이지요.
논리는 결국 '순서'입니다. OX, 즉 참과 거짓도 근본적인 것들을 순서대로 나열하면서 가리는 것이지요.
\vspace{5mm}

제가 가르칠 때애는 절대 빨리 풀지 말라고 합니다. 그럼 제가 스피드를 혐오해서?
왕년에는 세자리세자리 곱도 암산으로 해댔습니다. 지금은 두자리두자리곱은 가능한 수준이지만 저도 왕년에 암산마니아였고 급했습니다.
그렇기 때문에 스피드의 폐해를 알고 있습니다.
스피드를 높이는 손쉬운 꼼수는 "과정"을 생략하는 것입니다. 즉, 다 순서대로 밟지 않고 중간에 뛰어넘는 것, 즉 스킵해버리는 것이죠.
열심히 숙련하고 올바른 기법을 개발하면(올바르다라는 건 그 기법이 모든 순서를 지키면서 근거를 갖추었다는 겁니다) 속도는 높아집니다.
그러나 이거에 많은 노력이 들기 때문에 많은 학생들은 스킵을 해서 스피드를 높이려는 유혹에 빠집니다.
\vspace{5mm}

결과는 본인들이 아실 것입니다. 사실 이건 글을 읽거나 컥챗에서 대화해보아도 알 수 있습니다.
뭔 궁예질이냐 하겠습니다만. 스킵하거나 물타기 해서 점수 억지로 올리는 친구와, 그렇지 않은 친구들은 말과 글에도 차이가 납니다.
실제로 N수까지 가는 경우는 운빨도 없지 않겠습니다만, 알고보니 기존에 공부 잘 한다고 했던 것이 스킵으로 이뤄낸 거품 실력이기 때문입니다.
반면 찬찬히 기초부터 한 친구들은 올해 시험도 그렇지만 천대받다가 정작 시험점수는 잘 나오는 경우가 있습니다.
\vspace{5mm}

남들보다 빨리 시작하자, 이게 황금의 3개월 모토입니다.
그런데 공부는? \textbf{남들보다 천천히 하자}. 다들 이걸 모르시더군요.
N수가 실패하는 이유는 3$\sim$4월에야 공부를 서둘러서 하기 때문입니다. 서둘러서 하니 부실공사가 되어버리고 스킵으로 완성된 실력입니다.
이런 친구들일수록 더 천천히 개념을 읽고 더 천천히 문풀을 해야합니다. 대신 스킵하는 것 없이 논리적으로, 답답하게 해야하죠.
처음에는 이래도 되나 할지 모르지만, 모든 순서를 다 밟으면서 훈려하다보면 '참속도'가 올라갑니다.
이렇게 천천히 공부해야하기 때문에 빨리 시작하라는 것입니다.
느리게 하는 공부를 빨리 시작하라는 것인데, 대부분은 늦게 시작해서 성급하게 공부하려합니다. 이래서 실력이 오르겠습니까.
\vspace{5mm}








\section{과잉언급되는 천재들}
\href{https://www.kockoc.com/Apoc/571900}{2016.01.01}

\vspace{5mm}

\textbf{$\#$ 중수와 고수를 나누는 구분은 다음과 같다}.
\vspace{5mm}

\item \textbf{1. 고1 수학 최고난이도까지 숙달되어있느냐 $-$ 블랙라벨, 실력정석 문제를 껌으로 풀 수 있느냐.}
\item \textbf{2. 국어실력이 탄탄한가}
\item \textbf{3. 성격이 급하지 않고 차분한가}
\vspace{5mm}

고1 수학이 매우 잘 잡혀있으면 미적분도 사실 한달 내면 수능 수준으로 거의 다 정복할 수 있다.
(여기서 머리타령할 사람이면 고1 수학을 반복하는 걸 권하겠음. 고1수학은 정말 총론 중 총론이다).
그러나 인강을 줄창 들고 진도를 마친 친구라고 할지라도 고1 수학이 잘 안 잡혀 있으면 계속 문제가 터진다.
국어실력이 꽝이면 문제를 읽을 줄 모르거나 조건을 누락한다는 것도 그렇고
가장 중요한 건 성격인데 $-$ 수학 공부에 있어서 노력이란 "차분한 집중이 가능한 성격"의 형성까지 의미한다.
성격이 불안하거나 매우 급하거나 해서 문제를 해부하기보다는 답만 구하고 넘어가려고 하면 절대 발전할 수 없다.
\vspace{5mm}

그럼 위 3가지가 잘 된다고 천재인가. 그렇다고 할 수는 없다.
환경만 잘 갖추고 트레이닝 코스를 잘 만들면 1$\sim$3은 불가능하지는 않기 때문이다.
\vspace{5mm}

천재의 요건은 \textbf{대량생산은 불가능해야하지 않나.}
그런데 우리나라 사람들은 조금만 뛰어나도 천재라는 딱지를 붙이는 게 문제인 것 같다.
보통 천재라고 하려면 그 혼자만의 두뇌를 가지고 나머지 세계인구를 상대하는 수준,
즉, 집단이 이뤄놓은 문명을 개인이 바꿀 수 있을 정도여야하는데 우리나라는 이런 기준에 관대한 듯?
\vspace{5mm}

그것도 그렇지만 주제파악 못 하는 경우도 있다.
만약 본인이 개막장 환경에서 공부에 전혀 도움이 되지 않는 환경에서도 공부하고 싶어서 올라간 경우,
이런 경우라면야 '수재(秀材)' 정도로 칭할 수 있다. 아직까지는 인간의 영역이라는 이야기다.
천재(天材)에서 천(天)이라는 게 뭘 의미할지 생각해보아도 야 저런 표현은 함부로 쓰면 안 되겠구나라고 알 수 있다.
그 뛰어나다는 것도 선천적인데 생각해보면 수험지식은 선천적인 것과는 거리가 멀며, 오히려 선천적인 게 방해가 되는 경우도 많다.
\vspace{5mm}

\textbf{$\#$ 노력은 [log(a)X]}
\vspace{5mm}

노오력을 가장 잘 설명해주는 것은 로그함수일 것이고, 정확히 말하면 여기다가 가우스 기호까지 처리는 해야한다는 것.
다만 밑인 a는 사람에 따라 달라진다. 환경, 성향, 성격, 취향, 절박함 등이 어우러진 것.
될 놈은 된다 안 될 놈은 안 된다라고 하거나
노오력해보았자 소용없다는 케이스는 메시지보다는 '메신저'를 우선 보아야 하는데
전자의 경우는 가르치는 사람인 경우가 많은데 이건 결국 "성적은 유전자가 좌우하니 난 모르겠다"라는 책임방기와 똑같고
후자는 나가서 노가다도 안 뛰면서 엄마가 주는 밥이나 챙겨먹는 댓글충인 경우가 많다.
어차피 이건 간단히 반박되는데 남의 자식보고는 어차피 유전자가 좌우한다고 하는 인간도 자기 자식은 노력시키려고 하겠고
댓글 달면서 헬조선 싫어 노오력해보았자 소용없어하는 놈은 집안 다 망하고 굶어죽을 지경가면 그 때에는 또 먹고살려고 노오력하고 있다.
\vspace{5mm}

핵심은 노오력의 성과는 아주 천천히 나타난다는 것이다.
만약 a가 10이라고 하면 10, 100, 1000, 10000.... 이런 식으로 가야먄 겨우겨우 성과가 나타난다는 것이다.
그래서 여기서 참을성이 없는 사람은 '물타기'를 한다. 바로 [log(a)X] + F : 즉 F라는 상수를 일시적으로 더해 결과를 높이는 것인데
그럼으로써 사실상 거품으로 성적을 올리고 자기가 공부를 잘 한다고 착각하다가 입시에는 죽쑤는 사람들이 생겨나게 된다.
저기서 F라는 것은 사교육의 족보나 꼼수 혹은 야매교재를 보면서 일시적으로 점수를 올리는 경우.
이 친구들은 수험경향이라거나 무슨 평가원 코드라거나 어려운 이야기는 잘 하다가도 정작 시험성적은 개차반이거나 쉬운 것도 대답 못 한다.
\vspace{5mm}

노오력은 정말 정직하게 해야한다. \textbf{자기가 스스로 한 노력의 성과는 안 사라진다.}
그게 혹자 재수삼수라고 할지라도 그렇다. 자기 스스로 노력해서 실패의 늪에서 일어난 인간은 정말 빨리 성공의 낙원으로 가기 때문이다.
순수한 노력 X를 기울여서 만든 [log(a)X]=S는 X^S로써 작용한다.
정직하게 노오력한 사람들은 일확천금의 유혹이나 아주 멍청한 사기에 당하지 않는 이상은 잘 나간다고 보면 되겠지만,
반면 자식사랑한다고 부모가 사교육시켜주는대로 거품실력을 올린 친구들은 사실 그 이후로는 잘 나가기 어렵다.
\vspace{5mm}

다만 현 입시가 순수히 노력해서 '현역'으로 갈 수 있는 수준인가에 대해서는 약간 이견이 있다.
대략 짐작하면 평범한 애가 혼자 공부해서 하려면 +2년은 더 추가되어야하지 않나라는 생각.
왜냐면 학교는 도움이 되긴 커녕 방해가 되는 케이스도 많고, 지금 입시가 참 구조가 엿같아서 공정하지 않은 측면도 있기 때문이다.
그러나 이걸 본인 노력으로 이겨내야한다는 것만큼은 올바른 정론이라고 얘기하고 싶다.
\vspace{5mm}

\textbf{$\#$ 그럼 천재는?}
\vspace{5mm}

입시는 천재를 원하지 않는다. 실제로 천재가 있더라도 현행 교육제도에서 매장당한 케이스가 많을 것이다.
그리고 대부분의 천재썰은 부풀려진 케이스가 많다.
실제 입시에서 원하는 건 두뇌가 아니라 '엉덩이'라는 게 중요.
\vspace{5mm}

단도직입적으로 말하면 가장 중요한 건
\item \textbf{1. 환경}
\item \textbf{2. 습관}
\item \textbf{3. 기초}
이 세가지이다. 이 중 하나라도 되어있지 않으면 이게 심각한 애로사항을 초래한다.
\item 4. 성격
이 역시 만만치 않지만도 들 수 있겠지만 사실 1$\sim$3만 제대로 되어있으면 덩달아 치유되는 것이다.
\vspace{5mm}

부모가 우리 아이 천재예요하는 경우나 자기가 천재라고 착각하는 아이들은 실제로 머리가 좋냐... 하면 그건 아니다.
수험과목이 신체 스펙을 요구하는 스포츠도 아닌데.
그러나 \textbf{기본기가 정말 잘 되어있는 경우. 남들이 한 패턴 떠올리려면 30초 걸릴 걸, 본인들은 1초 내에 3개 정도 떠올린다는 것.}
한마디로 지식로딩 속도가 좋다는 것인데 이게 천재라고 할 수 있는 건가... 충분히 훈련과 숙달로 가능한 것이다.
그리고 그건 본인들이 정말 학습에 도움이 되는 환경에서 자라서 그렇다.
하지만 그렇다고 할지라도 이 글을 읽는, 자신이 불행하다고 믿는 n수생들이 그런 환경을 자기가 스스로 조성 못할 것은 아니다.
\vspace{5mm}

습관은 두고볼 것도 없다. 적어도 내가 보았던 공부 잘 하는 친구들도 그렇고 내가 그에 속했던(...) 때도 그랬지만
정말 엄격한 청교도적 생활 습관 유지하면서 학습량을 꾸준히 달성해서 목표량 혹은 목표량의 30$\%$ 초과달성을 하루도 빠짐없이 3달 내내,
기상시각은 일정히 유지하고 공부에 방해되는 것은 멀리하면서 그렇게 살 때 성적이 잘 나오는 것.
기초. 모두가 망각하지만 실제로 수능 어떤 과목이건 어려운 문제가 안 풀리는 건 그와 관련된 기초가 잘 안 잡혀있어서 그렇다.
수험에서의 창의력이건 순발력이건 그건 가장 원론적이고 총론적인 기초가 숙달되어있느냐 뿐인데
그런 기초적인 것일수록 '쉽다고' 무시하는 케이스들이 많다만. 물론 그 무시하는 사람들이 공부를 잘 하느냐 하면 그건 아니다.
\vspace{5mm}

천재 따위는 수험에 필요하지도 않다. 그러나 '수재'가 될 필요는 있다고 보고
그 수재가 되기 위해서는 환경, 습관, 기초는 분명히 갖춰야 한다.
\vspace{5mm}

\textbf{$\#$ 부모}
\vspace{5mm}

서글픈 이야기지만 성적이 나쁘거나 공부하기 싫은 케이스는 가정환경과 무관하지 않더라는 것
뛰어난 학생이군요라고 할 수 있는 경우는 정말 부모들이 이런 '깨인 분들이 계시다니'하는 경우가 많고,
반면 내가 봐도 머리는 좋은데 왜 공부는 못 할까하는 경우는 정반대.
그리고 여기 추가하자면 조기교육을 시켰느냐 안 시켰느냐하는 게 참 오래 가는 것 같다.
5$\sim$6살에 어떤 조기교육을 성공적으로 시켰다면 그게 10년 이상 복리로 늘어나는 셈이니 그 차이는 무시하기 어려운 게 아닐까.
\vspace{5mm}

이래저래 상담하는 경우 내용 절반 이상이 사실 부모님들 문제다.
그리고 내 조언은 간단하다. 성년자가 되었다면 부모님에게 정신적 의존을 하지 말라는 것.
아니 그리고 나도 나이처먹으면서도 느끼는 건데 어른들 하는 말이 다 옳은 건 아니다(내 말도 다 옳은 건 아니지 않나)
잔인한 진실을 적으면 자식들보고 n수 그만두고 대학가라하는 경우는
진지하게 자녀 인생을 고민하기보다는, \textbf{"댁 아이는 어디 갔나요"}라는 질문을 회피하기 위한 게 많다.
자기 자식의 인생보다는, \textbf{자기 체면을 신경쓰는} 그런 부모들이 많다.
\vspace{5mm}

한데 나중에 "엄마가 N수하지 말라고 했잖아요. 나 그래서 그렇게 살았는데 이게 뭐야 책임져"라고 하면 반응은?
\textbf{"그럼 공부하지 왜 내 말 들었냐. 네 인생은 네가 알아서 할 것이지"}
따지고 본다면 스무살 넘어서도 부모님 시키는대로 사는 것도 웃긴 일이다.
그 때부터 하지 말아야할 것을 고르는 건 본인의 '윤리'로 해야하는 것이다.
\vspace{5mm}

\textbf{$\#$ 대학가도 별 거 없다는데 왜 공부해야 하나}
\vspace{5mm}

\textbf{엄밀히 말하면 공부하는 사람이 되기 위해서이다.}
실제로 이십대 중반을 넘어서 공부하는 사람 그리 많지 않다.
대학은 어떻게 보면 본연(?)의 기능에 돌아가고 있는 것이다. 취업기관이 아닌 학문기관으로(...)
하지만 대학이 취업을 시켜주든 말든 자기가 노오력해서 좋은 대학에 들어갈 수 있었다라는 경험은 다른 성공들의 가능성을 높여준다.
\vspace{5mm}

10년 전까지야 일단 좋은 대학에 들어가서 취업해서 조직에 뼈를 묻고... 아니 걍 조직 기생충으로 들러붙어 살아간다는 게 먹히긴 했다.
그러나 지금은 모든 연령이 다 먹고사는 것, 그리고 내가 뭘 공부해야 살아남을 수 있을까를 고민한다.
공부할 수 있는 걸 고민할 수 있으면 그나마 행복이다. \textbf{공부하고 싶어도 공부 못 하는 사람들이 더욱 많다}.
수능을 포기하고 다른 시험을 공부한다거나 바로 돈버는 노선에 뛰어들면서 한달만 지나면 느낄 것이다.
그나마 국가와 사회의 전폭적 지원이 이뤄지고 참고서 가격도 저렴한 편인데다가 정보얻기 좋은 게 수능이었음을.
아마 이런 걸 가지고 "저 늙은이는 팔자좋은 소리하고 있네"라고 할 사람도 있지만 내 반응은 간단
"자기들이야말로 아직까지 배가 불러서 팔자좋은 지 모르지. 왜 진작 내 말 안 들었을까 후회할테니 구경이나 해야지 ㅎㅎ"
\vspace{5mm}

대기업 취업이라는 개념도 아무리 늦어도 10년 내에는 사라지지 않을까. 인류문명의 경제 시스템 자체가 바뀌고 있는데 무슨.
아프리카 BJ 들이 잘 나간다는 것을 당연하게 여기면서, 왜 그들이 잘 나가게 되었는가하는 근본적인 시스템의 변화,
그만큼 그들이 돈을 벌었으면 누가 돈을 잃어쓸까하는 생각을 해봐아야 하지 않을까.
조직들이 무너지고 초개인들 $-$ 즉 부지런히 노오력하고 공부해서 개인 능력을 키워 영업하는 개인들이 잘 나가는 세상은 현재완료.
어떻게 보면 공부할 건 더 늘어나버린 것이다.
예컨대 자기를 연예인으로 내세워서 이미지 팔아먹으면서 돈벌어보았자
본인이 책을 안 읽고 공부한 게 없어서 무개념 발언을 하거나 상식도 없어서 엉뚱한 답변하면 한순간에 날라가는 거지.
\vspace{5mm}

$\#$ 황금의 3개월 중 1/3
\vspace{5mm}

콕콕 내에서도 공부할 사람은 하고 안 할 사람은 안 한다.
초가을 정도 되면 반응은 달라질 듯.
여기에 대해선 이견이 많지만 참 답답한 듯. 작년에 안 겪어보셨나.
시동 걸고나서 본격 공부가 되려면 최소 3개월은 지나야 한다.
(물론 하루 12시간 내내 공부만 하는 독종 케이스는 제외. 그런데 독종이라면 황금의 3개월 말하지 않아도 공부하지 않았나)
\vspace{5mm}

수험은 남을 이기는 것이고, 남을 이기려면 \textbf{더 많이, 그리고 더 일찍 공부하는 수 밖에 없다}.
찔릴 놈들이 많겠지만 적어볼까
\vspace{5mm}

\textbf{$-$ 남들보다 늦게 시작한다}
\textbf{$-$ 앞서서 공부한 사람들을 압살할 수 있는 꿈의 교재나 강의가 있을 거라 착각한다}
\textbf{$-$ 수험사이트 검색하면서 그런 거 없나하는데 시간 허비한다.}
\vspace{5mm}

그런 게 있을 턱이 있나.
머리 좋은 놈은 노오력하는 놈 못 따라간다. 물론 노오력하는 놈은 좋아하는 놈 못 따라가지.
그런데 노오력하는 인간들끼리 비교하면 일찍 시작하고 반복을 많이 한 놈을 못 이긴다.
\vspace{5mm}

내가 EBS 강의 빠는 가장 좋은 이유는 그건데
첫째, 발췌해서 들을 수 있다.
둘째, 정말 기본적인 것만 설명하고 기교가 적다(기교는 남는 게 없다. 기초적인 것만 남지)
셋째, 다운받은 다음에 반복청취할 수 있다.
\vspace{5mm}

황금의 1개월 날린 사람들도 딴 생각하지 말고 자기가 취약한 과목 EBS 다운 받은 다음 2월말까지 돌리길
수능개념 강의로 xx 과목이 있으면, 그 과목과 관련된 선생들 인강 올해판을 다 다운받고 이해 안 가더라도 끝까지 시청한다.
필기까지 끝냈으면 그 인강의 mp3 버전을 다운받은 뒤, 스마트폰에 넣고 dice player 같은 걸로 1.5 배속 재생하면서 계속 듣고 다니셔라.
그렇게 해서 최소 3회청을 달성해도 힘들면 그 때 사설들으면 되는데
보통 강의를 1번 다 돌리고 그 다음 또 돌린 다음 3회청까지 하면 그 과목 체계는 거의 다 잡힌다.
\vspace{5mm}

이것도 본인들이 '안' 해서 그렇지 뭘.
\vspace{5mm}






\section{인강 활용법}
\href{https://www.kockoc.com/Apoc/572680}{2016.01.02}

\vspace{5mm}

$-$ EBS 강좌 기준 $-$
\vspace{5mm}

\textbf{$\#$ 국어, 영어}
\vspace{5mm}

동영상강의 : 한번 듣고 필기만 할 것,
필기할 때에는 \textbf{곰플레이어의 캡쳐 기능을 이용하는 게 편함}. 즉, 강의 들을 때는 캡처만 하고 듣고 나서 필기하시라는 이야기
(이것이 EBS 강의 추천 이유, 일단 다운받은 다음 곰플 같은 것으로 배속수 조절하면서 필기는 캡처하면 되므로 흐름이 끊기지 않음)
그 다음은 mp3 강의 다운받으신 다음 통학, 산책할 때 음악 대신 들으실 것. 최소 3회청 이상하면 강의 뽕을 뽑을 수 있음.
언어과목의 경우는 순수한 음성만으로 알고리즘 강화가 가능
\vspace{5mm}

\textbf{$\#$ 수학}
\vspace{5mm}

천천히 들으실 것. 무엇보다 필기가 중요함.
역시 곰플 캡쳐 기능을 이용하는 게 편리할 것임. 강의를 한번 들어준 뒤 필기 제대로 하실 것.
그런데 필기를 어디할 것이냐가 문제일 건데 이 경우는 노트 아니면 A4에 따로 필기하는 것을 권함.
A4에 필기하는 경우라면 나중에 바인더링 제본할 경우를 고려해 좌측 란은 비워두시길 바람.
수학강의는 mp3 강의에서 얻는 것은 없을 것임, 필기한 것을 반복해서 보시거나, 아니면 그냥 동영상강의를 빨리 돌려보는 걸 권함.
역시 한번만 들어보면 아무 소용없다는 걸 강조
\vspace{5mm}

\textbf{$\#$ 과탐/사탐}
\vspace{5mm}

수학에 준함, 단 음성강의를 들을 가치는 있음. 국어, 영어 강의가 지겹다면 탐구강의 음성만 듣고 다니는 것도 도움이 됨.
문제는 필기일 것인데 이게 강사가 올려준 pdf만으로도 커버링이 되지 않는 경우가 많음.
시간이 걸리겠지만 역시 노트나 A4에 따로 필기할 것을 권함.
강사마다 케바케이긴 한데 그 필기를 개념노트나 시중기본서에 단권화할 수 있는 경우가 있고 아닌 경우도 있겠지만
그냥 나중을 생각한다면 단권화하지 말고 따로 노트를 만들어서 제본할 것을 권하겠음.
어차피 단권화는 머리에 하는 것이지 책에 하는 것이 아님.
\vspace{5mm}

\textbf{$\#$ 사설강의가 더 나은데요?}
\vspace{5mm}

처음에는 그렇게 느껴질 것임, \textbf{처음 듣는 맛이 다르니까}.
그러나 최후에 남는 건 기교가 아니라 '기본'임을 강조하고 싶음, 수능에 준해서라면 기교가 먹히는 경우는 드뭄.
기본 지식이 반복, 압축, 집적, 세밀, 융화되어가면서 실력이 되는 것임.
재미없는 강의라도 여러번 들어서 그걸 거의 암송할 수 있을 수준으로 만드는 게 훨씬 나음.
잘 고른 강의를 나중에 2배속으로 들으면 그 강의속도대로 뇌가 움직임.
\vspace{5mm}

\textbf{$\#$ 강의를 인상깊게 듣는 법?}
\vspace{5mm}

손과 발을 움직여주는 게 핵심임, 자세한 원리는 나도 모르겠지만 들으면서 손가락으로 강사 하는 말을 필사하는 흉내 내거나
강사의 리듬에 맞춰 발가락을 꼼지락거리는 사소한 것도 집중도를 높여줌.
실강은 강사들이 일종의 연극을 하는 것과 유사해서 몰입할 수 있음, 그런데 인강은 영화를 보는 것과 같아서 몰입도가 떨어짐,
가능하면 1.3$\sim$1.4 배속으로 돌리고(말이 빨라지면 듣는 사람도 긴장하니까) 캡처는 곰플 기능을 이용해서 간편히 하길
자막이 지원되는 경우 혹은 그게 아니라면 강사의 말을 그대로 돌림노래식으로 반복하는 것도 좋음.
아니면 자기가 좋아하는 음악을 BGM으로 깔고 강의를 들어도 좋지 않을까 싶기도
\vspace{5mm}

\textbf{$\#$ 강의 자막 파일}
\vspace{5mm}

정 강의 듣기 싫으면 간혹 올라오는 강의자막 hwp를 다운받아 출력해보는 것도 답임.
읽는 것을 좋아하는 사람이면 더 나은 선택일 수도 있음.
물론 자막이 지원되는 강의라는 전제
\vspace{5mm}

\textbf{$\#$ 강의는 한번 들어도 되지 않나요?}
\vspace{5mm}

아무개 강의를 따라가서 고득점 나온 건 아무개 강의가 훌륭해서일 수도 있지만
$-$ 사실 요즘은 다 공부하고 노력하기 때문에 강의 질은 큰 차이는 없음 $-$
오히려 똑같은 이야기를 반복청취하고 필기한 것이 더 중요한 듯. 즉 반복이 중요
강의는 스스로 생각하지 않고 그냥 세뇌당하는 과정이라는 한계가 있지만,
'반복세뇌'당하는 과정에서 지식 주입당하는 효과는 무시 못 함.
손이가요 손이가 새우깡, 어르신의 인사돌, 이가 탄탄 이가탄.... 등의 광고가 대중들을 휘어잡는 힘 : 반복
\vspace{5mm}






\section{수학에서 꿀교재 찾으려는 망상은 버리시길.}
\href{https://www.kockoc.com/Apoc/575074}{2016.01.04}

\vspace{5mm}

상대적으로 수험에 좋은 책이 있을지 몰라도, \textbf{한권으로 다 대비되는 꿀교재는 없습니다.}
\vspace{5mm}

엄격히 말하면 실력정석도 본인이 아주 뛰어난 실력자가 아니라면 '안 보는 편'이 낫습니다.
정석은 풍부한 수학적 소스를 담고 있습니다.
문제는 그것을 '주입식'으로 적어놓았다는 것이고 사고하는 방법을 가르쳐주는 것과 거리가 멀다는 것입니다.
\vspace{5mm}

그래서 중급자까지가 실력정석을 보면 생각을 하기보단 \textbf{'암기'해버리}는 방향으로 가고,
암기해버리는 방향으로 공부하다보면 \textbf{시험에 나올 문제만을 담는다는 꿀교재}를 찾는 망상에 빠지게 됩니다.
\vspace{5mm}

교재로 치면 현 수능은 교과서'만'으로도도 대비가능합니다.
하지만 교과서만 봐서 무리잖아,
당연하죠. 교재도 중요하지만 더 중요한 건 \textbf{문제를 해부하는 '행동' 영}역이니까요.
\vspace{5mm}

아무 것도 모르는 사람이 인강을 들으면 좋은 건, 인강에서는 그 문제 해부 방법이 소개되어있기 때문입니다.
그러나 중요한 건 본인이 직접 문제 해부를 해봐야하는 것이지, 백날 해부방법만 알아보았자 소용없다는 것입니다.
아무 인강이나 듣고 (EBS 인강으로도 충분합니다) 문제 해부 절차를 양식화시킨 다음,
어려운 문제를 해설을 보지 않고 수일 걸리더라도 혼자서 끙끙 수리논술 풀 듯이 해부해보는 경험을 해보아야합니다.
이런 해부에 여러번 성공하면 수학 실력만 올라가는 게 아니라 '인간'이 바뀌게 됩니다.
\vspace{5mm}

수능에 나오는 역대 기출에 쓰이는 도구는 모두 교과서 수준에서 끝납니다.
더 좋은 방법이 있을 수도 있습니다만 그런 건 지양되어가는 분위기이며, 아울러 교과서 개념만으로 스스로 풀 수 있게 공부해야지,
자꾸만 잡스킬이나 잡개념에 의존하려하면 \textbf{실력은 절대 늘어나지 않습니다.}
궁금하시면 기출분석을 해보시면 되는 데 그 어떤 것도 야매 교재나 학생 모의고사가 반드시 필요하다... 그런 것은 없습니다.
개정과정으로만 치면 시중에 좋은 교재는 많이 나왔고(그걸 다 풀 수 있을지도 의문)
정말 문제해부와 직접적으로 관련이 높다면, 작년 기출 경향으로 본다면 수리논술 문제들을 그 절차에 맞게 풀어보는게 더 낫습니다.
\vspace{5mm}

위와 같은 것만 지키면 수학은 그리 무서울 것도 없습니다.
마치 수학을 잘 한다는 것이 유전이라거나 어떤 특정한 교재만 보아서 된다는 식으로 여론을 조성하는 장삿꾼들이 있죠.
당연히 그런 교재나 방식은 개인적으로 분석완료되었습니다만,
'도움이 되기는 커녕 학생들 사고방식까지 말아먹을 수 있다'는 게 제 결론입니다.
교과서상 개념들을 아주 정확히 익히고 그걸 순서있게 나열해나가는 '논리적 양식'을 습관화시킨다,
이게 중요합니다.
\vspace{5mm}

예를 들면
A $-$ 중학교 수학과 고1 수학이 철저히 완비되어있지만 고2 수학부터는 모른다
B $-$ 중학교 수학과 고1 수학은 그저 그렇지만 고2 수학 진도가 선행되어있다라고 하면
이 경우 A와 B 중 누가 잘 하느냐... 논할 필요가 없고 A가 그냥 무조건 잘한다고 보시면 됩니다.
왜냐면 A는 수학의 총론적인 부분이 매우 잘 잡혀있어서 수능 범위 내용들은 'skin' 정도로 인식되고 진도나 이해가 매우 빠릅니다.
이런 친구들은 고2  수학 진도를 $-$ 수능 기출 문제까지 병행하면서도 $-$ 현행 과정으로 6개월 정도로 끝냅니다.
반면 B의 경우는 선행은 되어있지만 제대로 이해할 힘도 없고 논리적으로 사고하는 게 되어있지 않아 지루한 암기로 수학을 받아들입니다.
유감스럽지만 많은 사람들이 B에 속하고 그래서 입시에 망할 수 없다고 해서 장삿꾼들 배나 불려주고 있는 게 현실이지요.
\vspace{5mm}

21, 29, 30이 도저히 안 된다는 분들은 오히려 속도를 늦추세요.
그냥 시중교재만 착실히 풀면서 하루에 2문제식 그냥 저런 문제들을 애써 답구하려하지말고, 문제를 끊어읽고 어떤 개념을 써야할까 생각만 하십쇼.
그렇게 가다보면 어느 순간에 에라 모르겠다 하면서 문제를 해부하다가 유레카 하면서 답까지 도달한 자기 자신에 희열감을 느낄 것입니다.
\vspace{5mm}

$\#$ 잔소리 $\#$
\vspace{5mm}

+ 정석은 거의 30년 넘게 발전이 없었다... 라고 해도 지나치지 않았다고 봅니다.
이건 장점인 동시에 단점이기도 합니다. 장점이라고 하면 그래도 정석은 부정할 수 없는 어떤 굳건한 위상을 지키고 있다는 것이죠.
그러므로 수학실력이 되는 사람은 정석을 읽으면 얻을 수 있는 게 많습니다.
하지만 초심자들이 정석을 본다면?
혼자 억측하는 건 무리이므로 거금 들여 일본 책과 교재들을 연구했지만 현재는 정석같은 방식을 고수하는 책은 드뭅니다.
차트식 수학조차도 해설은 매우 친절한 편이고, 어떻게 문제를 풀어야 할지 알고리즘을 부드럽게 적시한 편입니다.
\vspace{5mm}

+ 문제를 암기하지 말고 개념을 암기하시길 바랍니다.
개념 암기라고 하면 문제쎈에 나온 개념 + 각 공식과 정리의 증명 + 유의사항 정도면 충분합니다.
한데 학생들은 개념 암기는 안 하고 자꾸만 야매교재에 나온 이상한 스킬이나 문제패턴을 암기하고 있더군요.
수능에서 21, 29, 30을 제외한 나머지는 쎈 B형 수준에서 거의 다 풀립니다. 21, 29, 30은 거의 다 신패턴아니면 기출반복이죠.
\vspace{5mm}

+ 그리고 간혹 보다보면 수험을 하기보다는 xxx님의 xxx 강의를 듣는다, xxx 교재를 푼다라는 수험코스프레를 더 즐기는 사람이 많습니다.
고득점을 받아 원하는 대학 원하는 학과에 합격하기 위해서 xxx 강의나 xxx 교재를 듣는 것입니다. 그것들은 목적이 아니라 수단입니다.
그런데 어떤 언플이나 광고가 자행되었는지 몰라도, 수험의 목적을 잃고 \textbf{xxx 강의나 xxx 교재를 보는 것을 목표로 하는 학생들이 있습니다(...)}
\vspace{5mm}






\section{일지 뉴비들에게}
\href{https://www.kockoc.com/Apoc/575186}{2016.01.04}

\vspace{5mm}

일지에 대해서 무슨 장사를 한다거나 자랑질(...)을 한다거나 하여간 그 당사자의 인격을 보여주는 반응들이 있습니다만.
일지 제안한 것은 별 개 아닙니다.
칼럼을 쓰니까 별별 질문들이 다 날라오는데 \textbf{'실천적인 공부'와는 매우 거리가 먼 것}이어서였습니다.
공부를 정말 한다면 날라올 수 없는 질문들이 귀찮아서(...) 일지 쓰라고 한 것입니다.
\vspace{5mm}

수험생들에게 상담해준다고 글 쓰다가 며칠 지나지 않아
배너 광고에 올라오면서 장사하는 사람이 한두명이 아니라 그럴지 모르는데
개인적으로는 그런 건 별 관심도 없습니다.
상담해주는 것을 기화로 노골적으로 장사하는 인간들은 매우 천박하다고 여기고 있어서입니다.
그렇게 돈벌어보았자 곧 날려먹죠. 그런 사람들의 그릇이야 뻔하니까요.
제가 관심이 있는 건 "공부를 못 하는 사람들이 극기(剋己)하고 성공하는 과정"의 반복과 일반화일 뿐입니다.
입으로만 헬조선이니 흙수저이니 그러지 말고 본인들이 어떻게 해서 금수저들을 이길 수 있을까 하는 것을 실제로 모색하는 것이지요.
\vspace{5mm}

이번에 일지 쓰시는 분들은 대략 2주차 되셨을 것이고 본인들이 양심이 있다면 느끼셔야합니다.
공부를 안 할 때에는 뭐든지 다 공부할 줄 알앗지만, 실제로 해보니까 \textbf{하루 최소 공부량도 채우기 힘들구나라는 것을요}.
그리고 사실 이게 정상입니다. 막연히 하겠다라고 생각하는 것과, 실제로 해보는 것은 전혀 다른 문제이기 때문입니다.
학습량을 정말 늘려야하는 경우는 간혹 보다가 지적해드리겠지만,
지금 10일차 넘게 쓰신 분들은 일주일에 하루 정도는 정말 원없이 노는 시간 가지시고
나머지 6일은 정말 보수적인 학습량을 최소한으로 책정하고 그것이라도 100$\%$ 달성하려고 하시길 바랍니다.
\vspace{5mm}

이건 콕부심(...)이고 뭐고 그럴지 모르겠는데
현재 콕콕 사이트가 1년동안에 공부를 그래도 하려고 하는 알짜 회원들이 늘어나는 것은
사이트 주인장이든 저든 아무 상관이 없습니다. 오로지 정말 공부하려는 사람들만 우대하고 활동하게 하는 시스템이죠.
게다가 여긴 상업주의와는 거리가 멀고 일격도 졸라 까는 곳이기 때문에 특정교재를 보아야한다는 것에 구애받지 않고
정말 좋은 게 어떤 건지 판별하면서 공부 경영을 계속할 수 있습니다.
\vspace{5mm}

일지를 쓰실 때에는 하루에 공부한 것과 놀았던 것, 몰라서 질문해야할 것과 깨달은 것등을 중심으로 적고
일주일에 최소 한번 정도는 다 '합산'해 정리시간을 갖고.
중요한 고민이나 질문에 대해서 콕콕사이트 네임드들에게 호출해보시기들 바랍니다.
\vspace{5mm}

+++ 딴소리 +++
\vspace{5mm}

이 사이트가 실모나 사설인강에 대해서 매우 비판적이다 어쩐다 그러는데.
이게 정상인 겁니다. 솔직히 제가 돌아보면 타 사이트는 지나치게 '상업주의'에 몰두해있고
xx를 안 듣거나 xx를 안 풀면 망한다... 라는 공포분위기가 조성되어있습니다.
물론 상품이 좋다면야 그걸 권하는 건 안 말립니다만, 곰곰히 보면 장단점이 제대로 검토된 적은 없습니다.
\vspace{5mm}

학생인 척 해서 xx 좋다 하는 알바들은 꽤 많습니다.
그걸 방지하려면 거의 공짜로 누릴 수 있는 컨텐츠 빼고 나머지는 정말 엄격하게 비판하고 이야기해야 합니다.
그런 비판을 감수하지 않으려는 강사나 교재라면 그냥 '배제'해도 됩니다. 그건 정말 자기들이 자신이 없다는 이야기거든요.
인터넷 글 보고 우왕 좋겠다 하면서 구매해보니까 생각보다 별로라서 후회하는 사람 한둘이 아닐 건데요?
\vspace{5mm}








\section{표절}
\href{https://www.kockoc.com/Apoc/578553}{2016.01.07}

\vspace{5mm}

아마 언젠가 대한민국 교육 전체가 표절이라고 된통맞는 날이 오지 않을까 싶은데.
\vspace{5mm}

\item 1. 사교육 분야는 원체 고수가 널림,
지금 수험사이트들을 보면 자기들이 고수라고 하면서 돈 많이 번다 어쩌구 하지만 그건 다 개소리고
고수급으로 치면 상상을 초월하는 노인(...) 분들도 많습니다.
그런데 이 분들은 인강도 잘 찍지 않고 그냥 오프라인 강의로 활약하거나 아니면 업계를 떠났거나 그럼
왜 그런가 생각해보면 간단한데
벌만큼 벌어서 그런 것도 있지만
어차피 그런 걸 \textbf{인강으로 공유해보았자, 그리고 교재로 내보았자 100$\%$ 표절당합니다.}
\vspace{5mm}

\item 2. 우리나라는 지나치게 표절에 관대.
물론 합리적인 기준은 있겠죠.
가령 모의고사 어려운 것 100문제를 1년 걸려서 만들었다 하면 그건 창작이라고 볼 여지는 있는데,
수학 전분야 다룬 것을 500쪽 넘게 쓰는 게 1년 정도 밖에 안 걸렸다 하면 100$\%$는 아니어도 일단 표절 의심은 갑니다.
뭘 이런 걸 표절이냐 아니냐 하기 전에 '국제적 기준'으로 따지면 되고, 게다가 '참고문헌'과 '출처표기' 제대로 해보면 답 나오죠.
고교생이면 모르겠는데 대학에 들어간 사람이 '이런 게 뭔 표절이예요'라고 하면 이미 '공부할 권리'는 포기했다고 보면 되겠죠.
대학에 들어가서 등록금을 낸다는 것은 본인도 이미 '지적재산권의 거래'에 관여하고 있다는 것이고
더 높은 학위를 받는다는 것은 표절에 있어 매우 엄격한 논문까지 쓰겠다는 것인데 말입니다.
\vspace{5mm}

\item 3. 그럼 표절하지 않은 사람이 어디있느냐.
책수집하다보면서 느끼는 건데 '표절했다고 보기 어려운, 그리고 저자 본인이 정말 창작했다고 할 수 있는' 책들이야말로
일찍 절판될 뿐더러 빨리 사라지더라는 것. 특히 국내의 수학 책들은 유전자풀로 말하면 자가복제 갈라파고스 열화복제되었다 보시면 됩니당.
왜냐면 저작권이 인정되지 않을 뿐더러 노력해서 성과 공개하면 '이런 데 표절이 어딨냐 걍 하는 거지'라고 베껴버리니
충분히 독창적인 수학책을 쓸 수 있는 사람들도 집필을 안 하는 거죠. 수년 걸려 제대로 써보았자 보상 못 받는데 뭐하러 공개함?
대신에 그런 것들을 짜깁기 표절해서 적당히 구라까는 교재, 아니면 문제수로 승부보는 교재들만 살아남는 것이죠.
\vspace{5mm}

\item 4. 표절이 아닌 책들은 분량이 컴팩트한 대신에 그 문장 하나하나마다 논리가 잘 잡혀있습니다.
제가 수집하는 책들이 주로 이런 것들임, 이런 책들을 찾아 읽어야 생각하는 법이 바로 잡히기 때문입니다.
이런 경우 저자 스펙을 보면 30대 중반 이상은 넘어서고, 박사 이상이거나 아니면 그만큼 필드에서 활동했거나 뭐 그렇죠.
거기다가 참고문헌을 제대로 적시합니다. 이런 저자들은 자존심이 있기 때문에 남의 저작물도 존중할 수 있는 것이죠.
반면 표절한 책들은... 설명 안 해도 되겠죠? 나중에 나이 먹으면 자기가 그런 책을 냈다는 것에 부끄러워하지 않을까 싶기도 함.
\vspace{5mm}

\item 5. 아무튼 표절하는 현실 인정한다 치면, 적어도 남의 표절을 까려면 자기나 자신이 속한 쪽의 표절도 다 까야죠.
Clean Hands의 법칙은 최소한 지키라는 것이죠. 그런데 그런 걸 까지는 않음,
그렇다고 해서 자기들이 생각하기에 피해입었다는 쪽은 그럼 오리지널인가?
절대 그렇지는 않을 건데 말입니다. 우리나라 사교육의 오랜 역사를 보면 현재 인기있는 사람들이든 교재든 찰나의 포말에 불과할 건데?
게다가 사실 우리나라 교육이 태생부터가 말이 좋아서 일본에서 받아들인 것이지 일본 것들을 수도없에 베끼지 않았나?
\vspace{5mm}

\item 6. 동대문 보따리 장수가 프랑스, 이태리 명품 짝퉁을 만들어 돈을 버는 것까지는 좋은데(.. 라고 하지만 이거 상표법 위반이죠)
거기서 자부심을 갖고 "내가 명품을 만들려고 얼마나 고생했는 줄 알아"라고 착각해버리면 답이 없죠.
처음에야 '생존'하려고 표절한다고 하겠지만 그 표절로 부를 쌓으면
슬그머니 "나 원래 잘난 놈인데"라는 자아실현(?)으로 생각이 바뀌고 분수를 잃는 것입니다만.
본인이 창작실력이 없는데 성과는 내야 하니 남들이 표절이라고 확인할 수 없는 쪽을 찾아 표절하게 되겠지요.
\vspace{5mm}

7. 그래서 사실 교재 고민하는 사람들도 "짜깁기 표절한 것들 중 어느게 좋아요"라고 고민하는 것이니 웃길 따름인 것임.
어차피 강의나 교재나 다 베낀 것들 투성이면, 그냥 양많은 것 선택하면 되는 데 뭘 고민하는 건지.
수능에서 강사들 오류 저지르는 것이 뭐겠음, '짜깁기의 오류'인 것이죠.
강사들도 그냥 지식 정리해주는 업자 정도로 보면 되는데 슬그머니 '참선생'의 반열에 서려고 하는 것도 웃기지만
강의라는 것도 방만한 지식을 정리해준다... 정도로 보면 그냥 그런 것 잘 해주는 강의 하나만 아무거나 들어야 함.
표절문제 제기되면 이 바닥이 원래 그렇다 하는 곳인데 xxx 들어야한다하는 건 남대문, 동대문에서 뭘 사야할까 고민하는 것과 똑같음.
어차피 최종정리는 자기가 해야하는 것이 아닌가?
\vspace{5mm}

8. 교재도 마찬가지죠. 여기서 사설이니 EBS니 따지는 건 걍 웃긴 짓이긴 함.
그러나 EBS 권하는 건 그나마 이제 무난하게 정리되어있어서 그런 것이고 여러가지 지원받을 수 있는 게 많아서 그런 거죠.
어차피 다 짜깁기면 저렴한 가격에 많은 지식을 체계적으로 정리할 수 있는 교재 고르면 되는 것임.
이게 무슨 박사급 논문쓰는 것도 아니고 걍 다 짜깁기한 것 가지고 노는 것인데
이건 뭐 패스트푸드 점에서 햄버거 브랜드 따지고 편의점에서 삼각김밥이나 컵라면에 까다롭게 굴려는 것과 별로 다를 바도 없죠.






\section{확통에 추가된 분할}
\href{https://www.kockoc.com/Apoc/580053}{2016.01.08}

\vspace{5mm}

누가 도입했는지는 모르겠지만
수험의 정석 $-$ 스케줄링이 바로 \textbf{저 자연수의 분할}에 있다.
이게 나중에 경영과학 같은 데 쓰는 것이기도 하는데 내용은 잘 도입하신 듯.
\vspace{5mm}

자연수 분할의 수 : P(n,1)+P(n,2)+...+P(n,n)
성질 : P(n,k)=P(n$-$k,1)+P(n$-$k,2)+...+P(n$-$k,k) , P(n,k)=P(n$-$1,k$-$1)+P(n$-$k,k)
\vspace{5mm}

고교 확통의 흥미로운 점은 2가지.
\vspace{5mm}

첫째, 제목은 명색이 확률과 통계인데 정작 제목에 없는 '경우의 수'가 진주인공이라는 것(경우의 수만 잘 마쳐도 사실 90$\%$는 끝남)
둘째, 미적이나 기벡과 달리 '현실의 영역'에 절반 정도 걸쳐있기 때문에 학생 본인이 스스로 풀이방법을 개발할 수 있고 해봐야하는 것.
그래서 확률과 통계는 스스로 경우의 수를 만들어보고 그걸 본인의 현실 문제에 대입해보면 되는데
자연수 분할을 일종의 '학습량' 분할이라고 생각하면서 접근해보기 $-$ 가령 n=100 문제라고 하고 k=스케줄 단위라고 하면
합리적인 공부 계획을 세우는 데 도움이 될 수도 있다(물론 발상이 그렇지 그렇게 다 고지식하게 접근하라는 이야기는 아님)
\vspace{5mm}

그러고보니 경제수학의 전조인가.
\vspace{5mm}






\section{[일지 가이드 160108] 가이드 제시}
\href{https://www.kockoc.com/Apoc/580103}{2016.01.08}

\vspace{5mm}

2월말까지
\vspace{5mm}

국어
최소 : 최근 3년치 기출 문제를 풀거나 강의만은 들어놓으실 것
평균 : 화작문 문제집 한권은 제대로 돌려서 오답정리할 것
\vspace{5mm}

수학
최소 : 풍산자 등을 모두 다 풀 것
평균 : 쎈 또는 마플까지 다 풀 것
\vspace{5mm}

영어
최소 : 국어에 준함
평균 : 문법서, 구문서 한권을 제대로 떼거나 그에 준하는 강의 하나 완강, 영어어휘집 돌리고 있을 것
\vspace{5mm}

탐구
최소 : 수능개념 강의 과목마다 돌릴 것
평균 : 기출 다 분석해볼 것, 그리고 개념서 한권 돌릴 것
\vspace{5mm}

이것만 2월말까지 다 한다면 그 다음 일주일간 놀러갔다 오셔도 됩니다.
11월에 시험치고 나서 공부할 땐 저 정도야 하겠지... 라고 망상을 품는데
사실은 4월되어서도 못 하는 케이스도 많습니다.
\vspace{5mm}

일단 평균치는 해야 승산이 있고, 최소는 해야 '인간'대접은 받습니다.
막말하자면 남학생들은 최소 못 하면 그냥 군대 가시고 (군필자라면 빡센 삶의 현장 찍으시거나)
여학생들은 그냥 연애$-$결혼 코스 가는 게 어떻냐고 권해드릴 수도 있습니다.
\vspace{5mm}

국영수탐 중에 2개 정도는 평균, 2개 정도는 최소 유지 정도가 낫겠죠.
그리고 돈이 많은데(!) 공부가 안 된다하면 '오프라인' 사설학원, 학생수준 높은 데 가십시오.
\vspace{5mm}

아마 여기 와서 일지 쓰는 분들이면 어느 수험사이트건 다 돌아다니면서 글 읽겠지만
결론은 결국 "양치기"라는 건 확인했을 겁니다. 양치기 한다고 다 성공하는 건 아니지만, 성공한 사람 중에 양치기 안 한 사람은 없다는 것.
일지들을 쭉 보면서 늘 확인하지만 본인이 '교재나 강의 고를 때 목표한 학습량'과 실제 학습량은 다르다는 것.
그러니까 선택장애 겪지 말고 2월말까지 '최소'는 다 끝내주고 인간대접받으시길 바랍니다.
\vspace{5mm}








\section{카스트 제도}
\href{https://www.kockoc.com/Apoc/581753}{2016.01.09}

\vspace{5mm}

인터넷은 신(新) 중세시대를 도래시킨 것 같다.
\vspace{5mm}

말의 영향력이 커진 사회다.
조회수와 추천수가 많으면 그 말은 사실이 되어버린다.
열람자들이 그걸 일일히 검증해보고 믿는 것이 아니다. 그냥 조회수, 좋아요가 많으면 믿는다.
초기에 인터넷이 도입될 때만 하더라도 이것이 정보격차나 폐쇄사회를 해소할 수 있을 것이라는 낙관적 전망이 있었다.
현실은 거품정보 양산, 그리고 기존의 폐쇄사회를 붕괴시킨 것은 맞지만 새로운 폐쇄사회를 낳고 있다는 것.
\vspace{5mm}

수험의 경우는 웃긴 게 있다.
무슨 갓$\sim$$\sim$ 시리즈가 돈다. 10대 애들이야 철이 없으니까(?) 그런 컬트에 열광한다 치자.
어차피 수험이라는 게 결국 사교육에서 지적재산권 전수받고 그걸 정리해서 팔아먹거니와
수험 잘 한다고 천재도 아니고(수재라면 몰라도), 그리고 올바른 방향으로 노력하면 되는 것인데 거기서 왜 함부로 갓$\sim$$\sim$ 을 붙이지?
누구나 처음부터 공부를 잘 한 것은 아닐텐데?
\vspace{5mm}

수험이 의미있는 건 '노력'으로 가능한 승부여서이다.
천재들만 시험을 잘 보고 엉덩이로 공부하는 애는 못 본다면 이 게임이 재미가 있겠나.
그런데도 마치 수험에서 천재가 존재하는 양 그런 식으로 이야기를 하면서 갓$\sim$$\sim$ 어쩌구 하는 것,
소꿉장난이나 서바이벌 게임 치고는 이미 '업자들' 논리까지 개입한 것 같기도 하고 아무튼 꼴불견이다.
인터넷에서 이런 컬트에 빠지지 않고 그냥 착실히 공부하면 인생 폈을 친구들이 이런 데 휘말려 시간낭비하는 것을 보면 더욱 그렇다.
\vspace{5mm}

대체로 흐름이
\vspace{5mm}

\textbf{갓아무개 등장 → 모 과목 몇분만에 다 풀고 만점 → 갓$\sim$$\sim$은 $\sim$$\sim$$\sim$ 강의나 교재를 좋아해 → 수험생들을 위해 전격판매}
\vspace{5mm}

그런데 과거에 저런 것 없어도 공부할 사람은 했고 잘 나갈 사람은 잘 나감.
그래도 혹시나 해서 저런 게 정말 도움이 되나 다년간 보지만 내 기준에서 보자면 저게 썩 도움이 된다고 보지는 않았다.
일단 교재들을 보면 4$\sim$5등급 애들은 전혀 이해할 수 없는 것도 많지만
저런 것들을 보았다는 친구들이 이미 1$\sim$2등급이라면 그런 걸 보지 않아도 잘 나올 수 있었단 얘기가 되기 때문이다.
\vspace{5mm}

특히 수학과 과학을 좋아한다는 사람들이 왜 수험에 대해서만큼은 카스트제도식 미신을 전파시키는지 모르겠다.
될 놈은 되고 안 될 놈은 안 된다... 라는 것이야말로 하나마나한 이야기고(저런 말한 녀석이 자기가 망하고 있을 때도 저 말을 추종할까)
머리가 좋다라고 하는 건 수험에 있어서는 큰 상관관계는 없어보이고(환경이나 선행이라면 모르겠지만 글쎄.)
예컨대 아인슈타인이나 퀴리부인 같은 사람들이 한국의 수험에서 좋은 성과를 거두었을까? 에디슨이면 몇등급 떴을까.
\vspace{5mm}

갓이과 어쩌구하는 녀석들이 극혐인 이유는, 수험에 대해서만큼은 이상하게도 \textbf{이과적 분석을 하지 않는다}.
마치 수학과 과학을 잘 하는 게 천부적인 재능이고 신의 영역이라는 식으로 중세시대 마인드로 돌아가는데
이걸 보면 사이비 종교 집단 $-$ 가령 일본의 오움진리교에서 도쿄대 나온 이공계들이 있더라하는 게 이해가 안 가는 것도 아니다.
고교수학문제를 빨리 풀면 뭐하나, 이미 사고방식이 르네상스가 오기도 전 중세 봉건영주 수준인데
\vspace{5mm}

수험에 미신이 있을까.
작년 수능도 콕콕 내에서도 성공한 사람과 실패한 사람이 있다. 그렇다고 실패한 것이 범죄는 아니지.
물론 실패함으로써 당사자는 징역 1년 이상에 벌금 3000만원 이상에 처해지는 죄수가 된다.
사실 재수생들부터는 죄수생인 게 맞다. 그래서 수험은 프리즌브레이크 석호필의 탈옥인 것이다.
하지만 탈옥한다고 예수부처알라를 외치는 바보는 없을 것이다.
기도를 열심히 하니까 웜홀이 열려 나갈 수 있더라... 이걸 믿는 바보는 없지 않나.
\vspace{5mm}

노오력하면 보상이 오나요 그러는데 이건 OX 문제가 아니라 부등식 문제가 아닌가.
중학교 과정까지 마친 친구가 서울대급으로 가려면 사실 5년은 걸린다고 보는 게 내 생각이다(평균적이고 일반적인 기준)
그런데 이미 선행해서 중학교 때 고교과정까지 미리 보는 녀석은 그 5년이 3년으로 감해지는 것이고
고등학교 3학년 때까지도 정심없이 놀다가 아 이제 공부해야하겠다하면 5수는 당연한 것이 될 수도 있다.
그런데 노력해도 안 되는데요... 라는 친구들을 얘기하다보면 노력은 분명 했다. 하지만 그게 '기준치 미달'이라서 그렇지.
노력하다가 중도포기하는 친구들의 토테미즘이 곰이 아니라 타이거라는 것만큼은 확실한 것 같다.
쑥과 마늘을 먹고 버티라고 했는데 이런 것 해서 뭐해 하고 뛰쳐나간.... 뭐 종족번식에는 성공하시긴 한 것 같다.
\vspace{5mm}

길게 공부해야하는 게 당연한데도 실패하는 사람 다수는 "아, 이걸 왜 해야 하는데"라고 하다가 저기 갓$\sim$$\sim$$\sim$ 시리즈들을 본다.
그리고 그 갓$\sim$$\sim$$\sim$ 들의 노하우만 알거나 그들이 보았던 교재만 보더라도 금방 따라잡을 수 있을거라는 착각을 하는데
그런데 있으면 이 글을 쓰는 내가 먼저 낼름했을 것인데 \textbf{내가 아는 한 그런 건 '없다'}.
문제를 더 빨리 풀 수 있는 툴과 스킬이라는 것도 수능에서는 안 먹히는 경우가 태반이지만
소위 지식의 효율적 가공 같은 것 $-$ 2x2 매트릭스나 MCSE 같은 건 이미 공개된 것이고
무엇보다 저런 것들은 반드시 부작용을 수반하기 때문이다.
\vspace{5mm}

올해 수능을 응시해서 성공할 수 있을까요 없을까요... 이건 무의미한 이야기가 아닌가, 그걸 누가 아나?
목사, 스님, 무당들도 수능은 무서워한다. 합격에 대해선 다 침묵한다.
심지어 밤마다 잠을 못 이루게 하는 악귀들도 어려운 수학문제 냅다보여주면 된다, 수학문제도 못 푸는 악귀는 병신처럼 보이지 않겠나.
뻔한 이야기지만 중요한 건 본인이 '공부에 미쳐있는 상태'를 스스로 자초하는 것이고
우선 수능과 관계없이 숨쉴 때마다 본인이 공부한 텍스트나 문제가 연상되는 그런 충만한 상태까지 가면 되지 않나.
\vspace{5mm}

학벌이 무의미한데 공부해서 뭐하냐는 질문도 있다.
$\sim$벌이 의미하는 것은? 학벌, 재벌, 문벌, 군벌 등의 특징은 "경쟁에서 자유롭다"는 것이다.
과거에 서울대만 졸업해도 먹힌 이유는 간단하다, 서울대에 합격한 뒤에는 \textbf{공부를 안 하더라도} '갓서울대'니 뭐니 해도 칭송받아서이다.
그러나 지금은 서울대를 가던 노인대를 가든 \textbf{공부를 안 하면 살아남을 수가 없다.}
서울대에 들어가더라도 본인이 공부하지 않으면 쓰레기가 되기 때무네 학벌이 무의미해지고 있는 것이다.
명문대에 가면 '너 공부 쫌 했네' 정도만 본다, 사실 그 정도면 족한 것이다. 그 뒤에도 공부할 건 우글우글하다.
세상의 변화속도가 점점 빨라지고 활동무대가 넓어지고 있기 때문에 공부하지 않으면 따라잡을 수 없다.
\vspace{5mm}

수능을 치건 안 치건 어찌되었든 공부는 계속하고 있을 수 밖에 없는 것이다.
동네의 한적한 공인중개사도 지역 아파트 브랜드동호수견적 다 외우고 어떤 매물이 오가는지 금융시장이 어떤지 딱 꿰뚫고 있어야 살아남는다.
하다 못해 노점상조차도 어느 거리에서 몇시에 사람들이 많이 오가는지 단속은 어떻게 해야 피하는지 그런 건 다 연구한다.
\vspace{5mm}

그런데 오죽 수험판만 신기하게도 카스트제도적 인식이 남아있다는 게 신기할 뿐이다.
공부 못 하는 친구는 죽을 때까지 못 하나?
석달 붙잡아놓고 국영수 문풀 1000개 이상 시키고 인강 빨리 돌리고 하면 스트레스는 받지만 호전이 있을 건 당연하다.
문제는 이걸 \textbf{'안' 한다}는 것이다.
왜 안 하냐? 해도 실패한다는 것이다. 그럼 왜 실패하냐? 여기서 어물쩍댄다.
자기들은 했다고 하지만 실제 기록을 보면 공부를 안 한 경우가 많다.
그나마 이건 낫다, 그런데 더 심각한 건 그냥 수험사이트보니까 반드시 $\sim$ 강의 보고 $\sim$ 교재 안 풀면 안 될 것 같단 것이다.
그런데 더 심각한 건 $\sim$ 강의 보고 $\sim$ 교재 풀어도 이해가 안 간다는 것. 그냥 자기들이 바보라는 것을 인증했으니 공부해보았자 소용없다란 결론.
\vspace{5mm}

인생을 포기하는 매우 합리적인 결론이 아닐 수 없다.
\vspace{5mm}






\section{[일지 가이드 160110] 강의}
\href{https://www.kockoc.com/Apoc/582726}{2016.01.10}

\vspace{5mm}

사설 들으실 분은 가도 좋은데 효과야 딱히.
\vspace{5mm}

국어
\vspace{5mm}

기출강의만 빠른 배속으로 돌려들어보는 것 권하겠음.
선생들이야 뭐 다 한가닥하는 사람들인데 중요한 건 어느 선생을 듣느냐가 아니라
A 선생은 $\sim$ 하게 보고, B 선생은 $\%$ 하게 보는 구나... 라는 \textbf{차이}를 본인이 발견하는 것임.
국어는 정답을 찾으려는 태도 때문에 말아먹음.
국어 과목의 특징상 정답이란 게 존재할 수 없습니다, 그런데도 불구하고 시험에서는 하나의 정답을 강요하죠.
왜 정답이 존재할 수 없어서 C도 답이 되고 D도 답이 되는가, 그런데 왜 시험에서는 C만 인정해주느냐
라는 논리프로세스를 익혀야 합니다. 이럴려면 '각각의 입장'에 따라서 결론이 달라진다는 차이를 봐야하는 것이죠.
\vspace{5mm}

정답을 찾는 태도보다도, 왜 오답이 정답보다 더 타당한가라고 소피스트적으로 억지주장을 할 수 있는 능력이 더 중요합니다.
아니 뭔 소리야, 오답이 타당하다고 하면 국어점수가 나올리 없잖아.
자기가 억지주장을 하기 때문에 왜 그게 억지인지 스스로 알게 되거든요.
\vspace{5mm}

수학
\vspace{5mm}

나는 풍산자도 뭐도 모르겠다하면 그냥 EBS 수능개념강의만 따라가는 것도 권하겠음.
그런데 이거 90강에 육박하니 만만치 않습니다. 하루 3강은 들어야 한달만에 따라잡을 삘,
그리고 수능기출강의까지 들어주기만 하더라도 으음.
사실 이 정도도 안 해요. 그래놓고 나중에 자기 인생이 운이 안 좋아서 꼬였다 헛소리나 하고 있지.
아랫 글에서 적었지만 수학점수 잘 나오니 어쩌니 하면서 결국 상술로 연결시키는 그런데 속지말고
남들이 뭐라 비웃든 개무시하고 풍산자(쎈)$-$마플로 가거나, 아니면 수능개념$-$기출강의라도 꼬박 따라가세요.
실제로 EBS 수학강의 다 따라간 사람도 별로 없습니다(따라간 사람이 깔 리도 없고)
게다고 올해는 3,4점 코스까지 마련할 모양이니
\vspace{5mm}

영어
\vspace{5mm}

윤연주, 윤장환은 믿고 들어보는 강의라고 생각.
개념강의 목록 보아도 넘치기만 하지 부족하진 않음. 저게 마음에 안 들면 작년강의 찾아 들어도 좋고.
그런데 영어는 결국 보편지문 같은 걸 많이 읽고 양놈들 사고방식을 체화시키는 게 더 중요하죠.
빈칸추론 안 되는 건 간단, 한국인의 정서와 양놈 논리가 불일치하는데 우리 정서대로 풀면 오답 나오기 딱 좋죠.
역시 이것도 제대로 안 하는 사람도 있을 것임.
\vspace{5mm}

탐구
\vspace{5mm}

말이 필요없음. 그냥 개념$-$기출강의 따라가면 그만.
다만 시중교재로도 답이 없는 것들은 어쩔 수 없이 사설 들어야할 건디
그런 건 대충 3, 4월 이후로 다른 사람들 평가 들어보면서 좋은 것만 골라듣길 바라고
그 이전까지는 듄 강의 및 교재과 양치기 문풀로.
물론 화생방 훈련이나 화투질, 무두질로 가시는 분은 없을 거라 믿습니다.
\vspace{5mm}

일단 위의 것들은 자기가 정말 xx 과목에 뭐할지 모르겠다는 사람만 따라가면 됩니다.
듄 수능개념, 기출강의 우습게 보지말고 쭉 따라간 다음에도 개판이면 저 욕해도 상관없습니당.
괜히 xxx 강의 들어야한답시고 웹질이나 하면서 3월까지 놀다가 그 때부터 코스프레해서 n+1이나 하지 마시고요
\vspace{5mm}

그럼 저것들을 다 들으면 뭐하냐.
3월부터 EBS의 논술강의 들으시면 됩니당.
수능과 관계없잖아?
아뇨. 논술강의는 반드시 도움이 됩니당.
시중에 쓸데없는 내용만 담고 비싼 야매교재 같은 거 찾지말고 EBS 강의나 충실히 들으세요.
그래도 모자라면 그 때 가서 사설강의 들으시면 되는 것입니다. 괜히 프리패스 혹해서 인생 저당잡히지 말고요.
\vspace{5mm}









\section{[일지 가이드 160111] 수학 개념 증명해보기}
\href{https://www.kockoc.com/Apoc/584086}{2016.01.11}

\vspace{5mm}

풍산자, 쎈, 마플, 일품, 라벨에는 문제에 쓰여야하는 정의, 성질, 공식, 팁 등이 나와있다.
\vspace{5mm}

풍산자 $-$ 감각적이고 직관적으로 설명
쎈 $-$ 정말 필요한 것만 알뜰하게 적어둠(처음에는 왜 이렇게 빠진 게 많아 하겠지만 나중에 내공늘면 알 것)
마플 $-$ 수능에 쓰일 수 있는 수준으로 응용, 심화시켜놓음
일품, 라벨 $-$ 고난도 풀이를 위한 수준으로 정제해놓음
\vspace{5mm}

그런데 저기 빠진 것들이 바로 '증명'임.
쎈 기준으로 가자면 개념 설명의 공식들은 증명되어있지 않은 것들이 많음.
\vspace{5mm}

2월말까지 최소공부량 달성하시면
본인들이 보신 책의 개념 설명에 빠진 증명들을 스스로 하거나 아니면 교과서, 다른 참고서, 인강을 통해 얻은 걸 가필해보시길 바람.
특정 단원의 특정 공식이 어떤 맥락에서 어떻게 나온 것인지 그걸 복기해보고 설명할 수 있어야 문풀실력이 늘어남.
예컨대 확률과 통계에서 중복조합 공식, 자연수 분할 공식, 집합 분할 공식의 경우는
$-$ 수식적 설명, $-$ 국어적 설명이 양쪽 모두 가능한데 이것들을 찾아보고 채워넣는 게 중요함.
(특히 확률과 통계는 해당 공식이 왜 나왔는지 생각 안 하고 풀면 '산수'가 되어버림. 그 공식이 나오게 되는 과정을 분명 복기할 줄 알아야 함)
중복조합은 작대기를 이용한 증명이라거나 아니면 a+b+c=k 와 같은 증명으로도 족하고 자연수 분할 등도 이에 준함.
\vspace{5mm}

수학이 양이 많은데 왜 이리 내공을 들여야하냐, 걍 빨리 풀면 안 되냐 뭘 과정 쓰고 그러냐 할지 모르지만
이게 국어, 영어, 탐구 문풀에 막대한 영향을 끼치기 때문에 함부로 할 수 없는 것입니다
얼마나 빨리 푸느냐도 중요하지만, 그 이전에 얼마나 '정확한 논리를 정교하게 구사하느냐'가 더 중요하다는 점을 일러드리고 싶음.
자기가 보았던 참고서에 나온 개념은 강도에게 포박당한 상태에서도 줄줄 암송할 줄 알아야 하며
모든 공식이 다 어떻게 유도되는지 그 전제, 조건, 과정, 맥락도 술술 풀어댈 줄 알아야함, 수리적 사고력이 여기서 출발하는 것임.
\vspace{5mm}

그럼 이걸 진작하지 왜 양치기를 하고 하느냐....
저런 증명은 어느 정도 패턴화가 되어있지 않으면 받아쓰기만 하더라도 당장 이해가 되지 않기 때문임.
선문풀하고 후증명해보고 나서야 자기가 어떻게 공부했나 반성해볼 수도 있으며 잽싸게 교정이 가능함.
그러면서 논리라는 걸 세울 수 있음.
\vspace{5mm}

본인이 논리가 잡히면 그 다음부터 고난도 문제에 스스로 부딪쳐서 '자기 논리로 문제를 해부해보는' 것을 경험해보시면 됩니다.
이걸 직접 해보아야 실력이 늘지, 다른 것 해보았자 절대 안 늘어요.
이걸 늦어도 5월부터는 해야하는 것입니다. 당연히 웬만한 양치기는 5월 이전에 끝내놓아야 함.
6평부터는 소위 킬러문제에 대해서 본인들도 풀고 토론해보고 그래야만 안 무서워하지 안 그러면
시험 끝날 때까지 꿀교재니(그런 게 어딨어), 특정 강사 강의 들어야하니 그러는 겁니다.
\vspace{5mm}





\section{문과의 시대가 다시 오지요.}
\href{https://www.kockoc.com/Apoc/584304}{2016.01.11}

\vspace{5mm}

그게 누가 뭐라고 해도
권력, 돈, 그리고 이성을 휘어잡는 것은 문과를 졸업했건 이과를 졸업했건
\textbf{"말글"}을 다루는 사람입니다.
관념적인 이야기이긴 합니다만
인간은 '기술'을 지배하고, \textbf{마음}이 인간을 지배합니다.
그럼 그 마음을 이공계 학문으로 대체할 수 있느냐. 지금까지 그런 숱한 시도가 벌어지고 있고
경제학 같은 경우도 투자심리를 계량화하려는 시도를 하지만 아마 불가능할 것입니다.
아직까지 심리가 어떤지 마음이 무엇인지도 불분명하기 때문입니다.
\vspace{5mm}

수학계통이 다시 인기(?)를 얻은 이유가 기업에서 수리적 사고능력을 중시해서라고 하는데.
그렇다고 과거동안 '비즈니스'에서 써야하는 수학적 사고의 툴이 새로운 게 생긴 것도 아닙니다.
현재의 이공계 인기는 간단합니다.
원래 취업시장에서 이과는 그냥 능력있는 노예, 문과는 잘리기 쉬운 '동지'로서의 인식이 있었는데
문과 취업시장은 헬 중의 헬이 되어버렸고(루피, 동료는 필요없다), 이제는 기업에서 원하는 건 노예 아니면 아웃소싱이어서입니다.
문과가 망한 게 아니라, 문과 쪽이 매우 중요한데도 우리나라의 문과 교육이 진보도 발전도 없어서 그래요.
대졸하고 나서 4개 국어 기본인데다가 법률, 경제, 경영, 심리에 쌈박해서 개인창업이 가능하고 바로 해외진출 가능한 다음 욕먹었을지.
의치한이 열심히 공부한다고 하지만 선결 조건은 '인원수 제한'이죠.
만약 면허가 없었다면, 그리고 인원수가 과다배출이었어도 그랬을까 하냐면 그건 아닌 것입니다.
\vspace{5mm}

인터넷 덕분에 화이트칼라 일감이 사라졌다... 는 건 맞는 말입니다만
그렇다고 해서 우리가 쓰고 있는 언어가 사라진 건 아닙니다. 언어의 영향력은 훨씬 더 강해졌습니다.
기존문과교육이 관리자를 위한 것으로만 교육되어서 그렇지, 'CEO'를 위한 문과 교육의 수요라면 줄지 않았을 겁니다(공급이 없어서 그렇지)
\vspace{5mm}

다수가 좋다고 하는 건 너무 맹목적으로 좆지 마세요. 그런 것 없습니다 $-$ 다 돌고 돌게 되어있어요.
먹고살기 위해서 의치한이나 공무원에 간다는 건 타당한 이야기지만, 성공하기 위해서라면 이건 다른 이야기가 되어버립니다.
지금 10대 분들이 기성세대와 차이가 없는 게 '대학 간판'이 모든 걸 결정한다는 것에 너무 매몰되었다는 것인데
대학은 근본적으로는 '공부하러' 가는 곳이지, '취업하러' 가는 곳이 아닙니다.
부득이하게 간판을 보는 현실 때무네 대학 간판을 높일지 몰라도 그 간판 하나로 결정되는 것은 아닙니다.
본인이 치열하게 공부하고 연구해서 뭘로 하든 그걸로 독보적인 실력을 보여주겠다... 그런 각오로 가는 곳이라고 봅니다.
남들이 카더라하면서 답은 [   ]다 하든 말든 그건 5년 뒤에는 언제 그랬냐 얼마든지 말바꿀 수 있습니다.
\vspace{5mm}

컴퓨터나 인터넷이 등장해서 정보처리를 해주니 공부할 게 별로 없을 것이다... 라고 믿었던 시대가 있었지만 현실은?
그 컴퓨터나 인터넷도 믿지 못해서 모든 걸 다 꿰어차고 암기하고 공부해야해서 공부량은 기하급수적으로 늘어나고
죽을 때까지 학생이어야하는, 그리고 죽기 전에는 죽음이 뭔지도 공부해야하는 그런 세상입니다.
현재의 경기불황 경제위기는 더 이상 기존의 싸이클이 아닙니다.
'질적'으로 뭔가 변한 것입니다, 즉 세상이 또 바뀌어버리고 만 것입니다.
\vspace{5mm}

미래예측서에 공통적으로 강조하는 건 전문직이 아니라 '영업능력'입니다.
A라는 상품을 팔려고 하는데 한국에 수요가 없다... 그럼 아프리카 남미까지 뒤져서 팔아버리는 게 더 중요해진 것이지요.
꼭 국제적이지 않더라도 뭔가 '팔아대는' 것 자체가 사실 재화와 화폐의 흐름을 촉진시키기도 하지만 말입니다.
\vspace{5mm}

+
\vspace{5mm}

믿거나말거나 모르지만 진짜 무서운 부자들은 있는 척도 안 하지만 아니 스스로 검소하게 삽니다.
파리떼들을 막고 싶어서도 그렇겠고 돈이 가족들을 타락시킨다는 것도 그렇겠지만
\vspace{5mm}

근본적으로는 "배고파야만 볼 수 있는 게 있다"
\vspace{5mm}

부자도 오래가는 부자가 있고 그렇지 않은 부자가 있는데 후자가 더 많습니다.
이 경우 그가 갖고있던 돈은 정말 그의 돈이었을까, 아니면 돈이 의지를 갖고 당사자 품에 들어왔다 나가버렸을까.
명백히 소유권은 내가 갖고있지만 정말 내가 감당할 수 없는, 즉 부릴 수 없는 돈이라면 그냥 세상이 나에게 맡겨둔 것에 불과한 겁니다.
과분하게 들어온 돈은 내 것이 아니죠, 그냥 내가 돈의 노예가 되어버릴 뿐. 언제든 나간다고 해도 이상한 게 없는 겁니다.
남이 돈 갑자기 많이 벌었다 질투할 것 없습니다. 그 돈은 다시 나가거든요.
눈 앞에 잔칫상이 차려져있는데 그게 무려 500인분입니다. 내게 남겨진 시간은 3시간입니다, 어쩌겠어요?
아까워서 꾸역꾸역 처먹다가 배터져 죽는 게 해피엔딩일까요?
\vspace{5mm}

믿거나말거나이지만 사주분석해보면 '재운'이 있는 사람들은 정말 신기할 정도로 돈이 잘 들어오는데
다시금 신기할 정도로 돈이 무섭게 나가버린다는 예측이 있습니다. 이게 다 맞는다는 건 아닌데 그런 패턴은 있더구뇨.
재극인이라고 해서 재운이 강하면 인성(=배움, 학습, 인내)이 극당하죠. 사주가 맞고 아니고 떠나서 이건 시사하는 바가 많아요.
돈에 눈이 먼다... 라는 게 참 시사하는 바가 많습니다. 어떻게 보면 자기가 쓰지도 못 할 돈인데 집착하다보니 자기 오성마저 망가지는 거죠.
돈을 번다... 라기보다 돈을 의지있는 생명체로 여기고 "돈에게 사랑받는 존재가 되기", "돈을 컨트롤 할 수 있는 능력 갖추기"로 바꿔야겠죠.
일본과 미국의 거부들도 그렇지만 우리나라 재벌들만 보더라도 그 사람들의 도덕성 여부를 떠나보자면
자본주의 사회가 아니더라도 어디서든 한가닥 해먹었을 그런 사람들입니다. 미래를 읽고, 남들이 안 하는 걸 착수하며, 사람을 부렸으니까요.
\vspace{5mm}

물론 빵을 외면할 수는 없겠습니다만 반드시 돈을 좆기 위해 공부한다고 하면 이건 '먹고살기 위해 몸을 팔아도 된다'와 똑같은 얘기가 되죠.
공부할 수 있는 시간은 한정되어 있습니다.
\vspace{5mm}






\section{[일지 가이드] 30번 쓰신 분들 이상}
\href{https://www.kockoc.com/Apoc/586339}{2016.01.12}

\vspace{5mm}

일지 게시판에 쓴 게 지워져서리 여기 올립니다.
어차피 들어와서 공부할 회원들은 다 모인 것 같아서.
\vspace{5mm}

가칭 '하원' 게시판은
\vspace{5mm}

$-$ 칼럼 작성한 콕창
$-$ 일지를 쓰는 학생(30번 이상)
\vspace{5mm}

에게 권한을 드립니다.
\vspace{5mm}

그리고 거기서 일지를 쓰셔도 좋고 아니면 그냥 현재 공개 일지를 쓰셔도 좋은데
핵심은 올해 시험 대비하는 분들끼리 $-$ 상위권 허세 신경쓰지 말고 $-$ 서로 조언해주면서 친목질하며 수험을 영위해가는 것이 되겠습니다.
일지 피드백은 '일주일에 한번씩 합산'해서 올린 것을 보고 드림.
작년은 매일 했는데 시간도 많이 걸리고 비효율적이더군요.
\vspace{5mm}

수험사이트들이 지나치게 상위권 명문대 의대 간다에 치우쳐있는데
그런 건 별로 관심없다고 말씀드립니다. 개인적으로는 그런 걸로 허세 피우는 건 정말 싫어해서리.
어차피 한번 살다가 가는 인생 각자가 얼마나 열심히 공부해서 성과 거두고 자기 인생 펴나갈 수 있나 그런 것이 취지죠.
작년과 달리 수험고수(...)들도 많아졌고 조언 줄 사람들도 늘어났으니 사이트에 기여하고 자기도 고수가 되겠다(라지만 붙는 게 좋겠죠)
고 마음 먹으시면 일지 꾸준히 쓰시고 들어오시면 되겠습니당.
\vspace{5mm}

사실 수험생들에게 필요한 건 정보보다는 '안심하고 소통할 수 있는 공간'인데
그런 게 부족했으니.
\vspace{5mm}

조건은 어찌되었든 일지에 하루에 1번꼴로 30회씩 올렸냐는 것이고
내용상 허위거나 형식적인 게 아니라 진짜 공부했냐하는 것입니다.
가칭 하원은 대신 어그로를 끌거나 공부에 방해주거나 하면 자격박탈도 꽤 쉽다는 것이 제약조건으로 붙겠네요.
\vspace{5mm}

+
\vspace{5mm}

덧붙이면 수험정보는 어느 정도 비공개성이 필요하다일 건데 뭐 그런 것도 감안하지 않을 수 없겠죠.
\vspace{5mm}









\section{자위권 해결 실천편}
\href{https://www.kockoc.com/Apoc/587686}{2016.01.14}

\vspace{5mm}

\item 1. 도움(?)이 되는 온갖 컨텐츠들을 외장하드에 넣은 다음 꽁꽁 싸맨 뒤 깊숙한 곳에 박아넣을 것
\vspace{5mm}

\item 2. \textbf{욕망이 발동하면 무조건 외출}할 것 $-$ 이어폰 낀 다음 휴대폰에 저장한 EBS 인강을 플레이하면서 나가 15분 이상 걸을 것
\vspace{5mm}

\item 3. 그게 힘들다 싶으면 그냥 수면 취할 것.
\vspace{5mm}

\item 4. 그런 식으로 해서 달력에 표시해나갈 것. 마치 쿠폰카드처럼 그게 30개 모이면 놀러갈 것이다라고 약속할 것.
\vspace{5mm}

\item 5. 충동을 이기고 일주일차 이상 가면 몸상태가 호전되는 것을 느낄 것임. 2주차를 넘기면 자위에 빠져있던 때와 달라짐.
\vspace{5mm}

+ 산책할 때는 그냥 나가지 말고 인강 다운 받은 것이나 음성녹음 같은 걸 끼고 움직일 것.
국어, 영어, 탐구 같은 것은 음성녹음만 들으면서 산책하는 게 학습효과에 도움이 되는 경우가 많음.
\vspace{5mm}

+ 적절한 자위는 도움이 된다.... 라는 논리면 적절한 술담배마약도 도움이 된다고 할 수 있음.
조금이라도 도움이 안 되는 건 그냥 안 하는 게 나음. 그리고 그것도 못 할 바에는 그냥 공부를 안 하는 게 바람직
\vspace{5mm}

+ 공부가 안 되고 피곤하면 소설이나 만화책을 보아도 좋지만 이것도 자위에 도움(...)이 되는 일이 있어서리.
그냥 이어폰$-$인강 들으면서 나가거나 아니면 수면이나 보충하는 게 나음.
\vspace{5mm}

+ 보통 청소년 상담에서 "신중한 검토 끝에 합의보거나 조절하는 게 낫습니다"의 의미는 \textbf{"걍하지 마 병신들아"} 이 소리임.
\vspace{5mm}






\section{실패가 두려운 게 아니라 도전을 못 하는 것이 더 무서운 것입니다.}
\href{https://www.kockoc.com/Apoc/587914}{2016.01.14}

\vspace{5mm}

제가 n수 조장한다는 여론이 있는데 여기서 해명(?) 비슷하게  하지요.
\vspace{5mm}

일단 여러분들은 '실패'를 두려워하면서 도전 자체를 포기하려하겠죠.
사실 이거 아무 것도 안 해도 해결됩니다.
나이먹으면 \textbf{도전도 못 하거든요}.
\vspace{5mm}

그런데 나이처먹은 사람으로서 말씀드릴 수 있는 건.
실패조차도 장기적으로 보자면 플러스가 됩니다. 그게 다른 도전에 도움이 되기 때문이지요.
\vspace{5mm}

무인도에 살면 웬수조차도 마주치면 반갑다란 말이 있죠.
지금 여러분들은 성공 vs 실패 라는 틀로만 보려고 하겠지만 사실 이건 불완전한 구분입니다.
성공, 실패의 전제는 '도전이 가능하다'라는 것입니다.
그런데 도전이 항상 가능한가.... 그렇지 않다는 게 문제죠.
\vspace{5mm}

실제로는 "도전할 수 있나 도전할 수 없나"가 더 중요합니다.
우리가 윤리적이다 비윤리적이다 따지든, 가난하냐 부유하냐라고 하든, 혹은 금수저냐 흙수저냐 하는 것. 이건 거의 다 유치한 것이죠.
이건 어디까지나 우리가 \textbf{'살아있을' 때에나 의미있는 것}입니다.
배가 부르고 몸이 편하면 자기가 잘 생겼느니 돈이 많으니 권력이 높으니 하면서 참 헛된 자랑을 하면서 그걸로 만족을 느끼려하겠죠.
그러나 본인이 말기암이라거나 아니면 사고가 나서 죽기 직전이면 "근심없이 숨쉬는 것 자체가 행복임"을 느끼는 겁니다.
\vspace{5mm}

자꾸만 시험실패하면 어떡해요... 라고 하는데 올바른 공부방법으로 빡세게 하면 시험으로 승부볼 수 있는 것이면 3$\sim$4년 내면 붙습니다.
(다시 말해 저걸 초과하면 그건 본인의 방법이 문제가 많거니와 제대로 공부 안 했다는 이야기입니다. 그래서 제가 오수썰에 비판적입니다)
님들은 떨어지면 어떡해... 라고 하겠지만 사실 이건 매우 배부르고 한심한 고민입니다.
지금 지구상에 님들과 나이가 비슷한 사람들은 공부하고 싶어도 못 하는 사람들이 대부분입니다.
강제 노역당하지 않고 총들지 않고 수험에 몰두할 수 있는 것 자체가 행복이고,
시험에 도전해 볼 수 있는 것 자체가 축복인 걸 모르는 사람들이 많습니다.
헬조선 거리는 사람도 해외 여러국가 다녀오면 현재 우리나라는 살기 나쁜 곳이 아니라고 말하게 됩니다.
\vspace{5mm}

도전할 수 있을 때에 남 눈치보지 말고 도전하십시오.
이 말에 수긍을 하든 안 하든 그건 제가 알 바는 물론 아닙니다.
그건 본인들이 나이먹으면서 느끼는 것이니 제가 뭐라할 건 아니기 때문입니다.
하지만 본인이 가고싶은 길이 있다면 그 길을 못 가면 그건 평생 한이 되는 것만큼은 트루입니다.
\vspace{5mm}

다만 도전을 한다면 자기에게 쓴소리해줄 수 있는 사람들 2$\sim$3명은 확보해놓고 매일 잔소리를 들으면서 도전하십시오.
\vspace{5mm}






\section{[공지] 콕콕 총회}
\href{https://www.kockoc.com/Apoc/589984}{2016.01.15}

\vspace{5mm}

ⓐ 일지를 30일 이상 작성한사람
ⓑ 칼럼을 3편 이상 작성한 올린 사람
ⓒ 운영진 및 기존 상원멤버
ⓓ 콕콕 사이트와 유관한 교재를 출판하거나 교육활동을 하는 사람
\vspace{5mm}

일단 이 중 하나에 속하면 저에게 알려드리면 총회 자격을 드립니다('그리고'가 아니라 '또는'입니다)
저 요건을 만족시키면 뭔가 특별한 사정이 없는 한 총회 게시판에다가 카타콕 일지(비공개 일지)로 쓸 수 있습니다.
다만 들어오는 게 쉬운 반면 나가기도 쉽습니다. 즉, 어그로를 끈다거나 비속어를 쓴다거나 하면 쉽게 방출됩니다.
공개적으로 일지 쓰는 것에 부담을 느끼시거나, 그냥 비슷한 처지에 있는 사람들끼리만 서로 얘기하고 싶다고 하거나
일종의 친목 도모하면서 오순도순(...) 수험에 매진하겟다는 분들은 저 요건 중 하나를 충족시키면 댓글로 신청해주시길 바랍니다.
가볍게 확인한 다음 바로 사이트 관리자에게 보고해서 출입권한을 드리겠습니다.
아무래도 공개 일지는 다 공개하기 그런 것도 있고 불안한 것도 있사온데 그런 건 줄어들 것입니다.
총회의 1차 기능은 수질관리(...)이겠죠. 그리고 제 경우는 당연히 상원, 총회 순으로 관리하고 피드백합니다.
\vspace{5mm}






\section{[일지 가이드 160116] 지금 공부가 되고있는 증거}
\href{https://www.kockoc.com/Apoc/590446}{2016.01.16}

\vspace{5mm}

\textbf{우울하다}
\textbf{힘들다}
\textbf{공부하기 싫다}
\vspace{5mm}

라고 하는데 꾸준히 교재는 풀고 있으면 지금 이건 정말 공부를 하고 있단 증거임(가을에 겪으실 걸 지금 미리 겪고 있음)
반면 공부가 너무 잘 된다거나 아무 생각 없다면 좀 의심을 해보아야할 듯.
\vspace{5mm}

공부를 하다가 우울해지는 건 아래 얘기한 '역금단증상' 때문입니다.
\textbf{원래 뇌는 공부하기 싫어한다 → 공부 상태가 지속된다 → 뇌에서 공부를 안 하기 위해 우울한 상태에 빠진다}
\vspace{5mm}

그럼 언제까지 하느냐. 2월 첫째주까지만 달리셨으면 합니다.
그 다음 주는 어차피 \textbf{'설날'}이 있어서 공부 하라고 해도 못 할 각이죠.
\vspace{5mm}

아무리 공부해도 이해가 안 된다... 싶으면 질문을 때리거나 인강, 과외를 이용하겠습니다만
대부분의 의문은 "탐구"나 "반복"으로 해결됩니다.
생소한 지식체계에 자신의 뇌를 적응시켜나가는 게 중요하다는 점에서는 힘들더라도 규칙적으로 반복해주는 게 가장 좋습니다.
반복하다가 뇌에서 적응하는 순간 저절로 이해가 되는 유레카 순간이 오면 이게 하나로 끝나는 게 아니라 쌓여왔던 의문이 거의 다 풀리기도 하는지라.
\vspace{5mm}

아 그리고 총회자격부여되신 분들은 상단 안전구역에서 총회에 나오는 '안식처' 게시판
그리고 기록실에 나오는 총회일지를 이용하실 수 있으실 것입니다.
안식처 게시판은 칼럼이나 일지를 일정량 이상 쓰신 분들께서 자유게시판처럼 이용하면서 부담없이 글을 쓰시면 되겠고(수질관리합니다)
현재 일지가 노출되는 게 거북하다 하실 분들도 많으니 그런 분들은 총회일지를 이용해주시면 되겠습니다.
\vspace{5mm}






\section{각자에게 맞는 배우는 방식은 다릅니다.}
\href{https://www.kockoc.com/Apoc/592462}{2016.01.17}

\vspace{5mm}

A는 독서가 최적이고 강의가 쥐약인 반면
B는 강의가 최적이고 독서가 안 맞을 수도 있습니다.
C는 강의$-$복습 주입방식을 좋아하는 반면 D는 토론하거나 자기가 가르치면서 깨닫는 걸 선호할 수 있죠.
(사실 흥미로운 건 D입니다. 분명 자기는 그 내용을 모르는데 남에게 가르쳐야한다는 과제가 주어지면 제대로 공부하고 가르치면서 깨달으니)
\vspace{5mm}

이게 사람마다 참 방법이 다 달라요.
제가 xx만 들으면 된다, xx만 보면 된다라는 것이 위험하다고 보는 이유인데
사람마다 이렇게 접근방식이 다른 만큼 학습의 보편타당한 원리는 정말 신중히 접근해서 추출해야하는 것일지언대
수험 컨설팅을 하면서 특정 상품만 좋다라고 하는 케이스가 많습니다.
\vspace{5mm}

이건 금단, 역금단증세와 다릅니다.
본인이 공부를 안 해서 성적이 안 나오는 것과, 현재 듣는 강의나 교재가 안 맞아서 성적이 안 나오는 건 정말 다른 문제입니다.
졸라 노력했는데도 성적이 안 나오는 경우는 특수한 경우를 제외하면(다년간 껌씹고 오토바이타고 다녔다더라하는 것)
방식에 문제가 있는 것이기 때문에 이건 신중히 얘기하면서 접근 방법을 고려해보아야합니다.
본인에게 맞는 방법을 찾기만 한다면 다년간 고생했던 것이 3개월 내에 해소될 수 있습니다.
\vspace{5mm}

게다가 공부머리라는 것은 절대 1차 함수가 아닙니다. 상당히 불규칙한 f(x)를 베이스로 깔고 있는 가우스 함수이지요.
그게 일찍 트인 학생도 있지만 늦게 트인 학생들도 많습니다.
수학을 문자로만 접근하는 학생도 있지만 반면 이미지 $-$
그것도 자신의 운동이미지로 연상해서 가는 학생도 있는 등 접근방식은 참 다양합니다.
\vspace{5mm}

수학에만 한정해 말하면 참 재밌습니다.
국내 수학 사교육은 마치 자기 강의만 들으면 된다, 모 교재만 보면 된다는 식으로 만병통치약을 강조한다 그건데
경문사에서 나오는 수학교육에 관한 책이든, 일본에서 나오는 양서들을 보면
수학을 왜 학생들이 싫어하게 되었나라는 걸 진지하게 고찰하면서 나름의 해결방식들을 내놓고 있다는 것입니다.
그럼 누가 거짓말을 하고 있는 것일까요?
\vspace{5mm}

분명한 사실은 1등급도 거품 1등급이 있단 것입니다. 잘 나가다가 어느 순간 2$\sim$3등급으로 떨어지는 케이스가 있죠.
당사자는 자기는 열심히 하는데 왜 그런가 부르짖습니다. 그런데 이걸 '기반공사가 약한 마천루 부실공사'라고 하면 싹 설명될 터인데 말입니다.
사실 수학공부는 출제 경향이 바뀌더라도 점수 변동이 적은 게 당연할 터인데 말이죠.
아마 이 점에서만큼은 수학의 정석도 할 말이 많을 겁니다. 수학의 정석 입장에서 국내수학 교육은 자기를 중심으로 공전하는 것들로나 보이겠죠.
물론 우리가 아무 기초도 없이 정석을 지향하면 타죽을지도 모릅니다.
\vspace{5mm}

이런 것 다 무시하고 빨리 가고 싶다고 소위 합격기에 나온 커리만 따라가는 건 위험합니다.
n이 3이상 되어버린 사람들이 사실 이런 부류라고 봅니다. 자기 문제점을 해결하고 가야하는데 그렇지 못 한 채 계속 부실공사만 한 것이죠.
\vspace{5mm}






\section{계획을 짜고 실천하는 법}
\href{https://www.kockoc.com/Apoc/596969}{2016.01.19}

\vspace{5mm}

여러 번 언급한 이야기이겠지만 고민하는 분들이 많아서 다시 한번 올리지요.
\vspace{5mm}

\item 1. 목표달성에 필요한 작업량을 산출한다
\item 2. 주어진 시간을 산출한다.
\item 3. 작업량을 시간단위로 나눈다. (하루 필요작업량)
\item 4. 하루에 할 수 있는 작업량을 냉정하게 계산한다 (하루 가능작업량)
\item 5. 3번과 4번을 비교한다, 만약 하루 필요작업량이 하루가능작업량보다 크다면 하루 가능작업량을 늘릴 수 있는 방안을 모색한다.
\vspace{5mm}

그런데 보통은 위와 같은 단계를 거치지 않고 그냥 하죠.
\vspace{5mm}

사실 그러기 힘든 게 어떤 교재까지 공부해야하냐 그걸 몰라서 그런데 해답은 간단합니다.
2016년 5월까지 EBS 수특, 시중교재, 기출을 다 끝낸다고 잡으면 되기 때문입니다.
그리고 계산해보시면 아시겠지만 저거 다 끝내려면 11월 말에 시작했어도 힘들다는 결론이 나옵니다.
\vspace{5mm}

이것도 머리가 아프다면?
\vspace{5mm}

그럼 수능 시험 당일에 풀어야 하는 문제량 : 국어 40, 수학 30, 영어 40, 탐구 40을 보정해서
\textbf{국어 120, 수학 100, 영어 120, 탐구 120 문제 가량을 꾸준히 할 수 있느냐}만 측정해보시면 됩니당.
당연히 처음에는 그게 불가능합니다만 진정한 고수라면 저 정도는 하루 내에 풀어낼 수 있어야하겠죠(거의 다 아는 문제여야하고)
\vspace{5mm}

이 시기 되어서 업자들이 또 광고질해대고 장사하자... 하겠지만
그런 건 라이벌들이나 권해주시고 님들은 그냥 기출, 시중교재나 풀어서 저런 양적달성이 가능한지나 검증해보세요.
특히 수험교재들은 적어도 사견상으로는 신사고, 천재, 미래엔 등에서 나온 것으로도 충분하다고 여기며
수학의 경우 특작, 수능다큐, 풍산자의 그 테마별 시리즈, 블랙라벨 수능전략서에다가 교과서만 봐도 다른 야매교재는 볼 필요 없다 생각합니다.
이러고도 세뇌당해서 자꾸만 이상한 교재들 보겠다는 분들 계시는데 인생 그렇게 낭비해도 제 알 바는 아니지만
최근 3년동안 정말 적중한 사례나 있는지 한번 검증해보시면 됩니다.
\vspace{5mm}

제가 빨리 풍산자와 쎈, 마플 풀어야한다고 했는데 벌써 1월 중순입니다. 한달 반 지나면 3월이네요.
냉정히 말해서 그거 다 끝낸 사람 그리 많지 않을 겁니다.
\vspace{5mm}








\section{부채도사}
\href{https://www.kockoc.com/Apoc/603292}{2016.01.21}

\vspace{5mm}

재작년부터 이미 평가원의 출제를 '사설'이 못 따라잡기 시작했다는 것.
\vspace{5mm}

개인적으로 예측한 것 중에서 맞은 것은 화생하지말라, EBS가 다시 중심이 될 것이다.
결정적으로 틀린 것이 바로 2015 수능 수학 난이도.
\vspace{5mm}

전자는 대충 수험생들 현황이나 온라인에 올라오는 정보를 취합, 그리고 무엇보다 저 자신이 교재 연구(?)라는 걸 하기에 알 수 있지만
후자는 평가원에서 독자적으로 내리는 것이나 사실 뭐라고 할 수가 없었죠.
물론 소가 뒷걸음질치듯 작년 수능에서 영어가 다소 어렵게 나올 것이다라고 하는 건 맞았으나,
그건 평가원을 해킹해서가 아니라 다들 영어를 물로 본다면 통수작렬 가능성 있다고 보았기 때문입니다.
\vspace{5mm}

시험은 경쟁과 출제의 교향곡입니다.
대체로 경쟁이 어떤 상황인지 출생년도에 따른 수준은 짐작가능합니다.
가령 현재 고2$\sim$고3들은 꽤 현실적이고 선행 3년치는 기본이라서 정시로 따라잡으려면 수년 걸린다... 정도는 IMF라는 대사건으로 요약할 수 있죠.
그러나 이에 대해서 평가원이 어떻게 나올지는 아무도 모르는 것입니다.
변별력이 없다고 하면 아예 물로 내버려서 실수하는 걸 노릴 건가, 확 마그마로 내버릴 것인가. 이건 부채도사입니다.
\vspace{5mm}

그런데 간혹 보면 후자가 가능하다고 보는 케이스가 있는데 그런 분들은 그럼 일주일치 주식이나 맞춰보라고 말씀드리고 싶습니다.
수험생들의 대략적인 수준이나 사교육 공급$-$수요 같은 것들이야 일정한 경향성을 띠고 있기에 그런 경쟁 정보에 의한 예측은 개연성이 있지만
평가원이 정말 어떻게 낼 것이다라는 건 사실 제대로 맞춘 사람은 단 한번도 없거니와
그런 식의 이야기를 하면서 장사를 해대는 업자들은 수능 이후 몇달동안은 그냥 침묵해버립니다.
\vspace{5mm}

다만 지금 하나 머리털 걸고 자신있게 말할 수 있는 건
올해 실패하는 수험생들은 '학습량' 부족 때문에 아작나기 참 좋을 것이라는 사실입니다.
이건 그럴 수 밖에 없어요.
첫째, 사교육 어느 쪽도 2017 수능이 $\sim$ 하게 나올 거다라고 자신하거나 준비하지 않고 있다는 것.
둘째, 이 역시 학생들도 몰라서 그냥 꾸역꾸역 공부하고 있는 수준이라는 것
\vspace{5mm}

특히 재밌는 건 이와 관련해서 기존 교육과정에서 썰 풀었던 사람들도 지금은 유의미한 이야기를 못 하고 있단 사실입니다만...
사실 이건 안이할 수 밖에 없는 게 고2들 모의고사도 그렇고 작년 수능에서도 '신호'라는 것을 분명히 던져주었다는 것이고
그건 정말 왕도를 통해 공부한 사람이면 앞으로 어떤 식으로 애들을 엿먹일 수 있겠구나라는 실마리를 대충 잡을 수 있다는 것입니다.
그리고 뻔한 이야기지만 N수생들에게는 더 가혹해질 수 있습니다.
아니, 정확히 말하면 교과서와 기본을 무시하고 모의고사나 강의에만 의존하는 N수생들에게는 더더욱요.
\vspace{5mm}

2015 수능에서 평가원은 "우린 쉽게내면서도 너희 엿먹일 수 있어"라고 경고했고
2016 수능에서 평가원은 "논리적이지 못 한 녀석, 모델링 못 하는 녀석은 꺼져. 탐구는 복불복"이라고 얘기했죠.
각자 어떻게 공부해야할지 이것만 봐도 방향은 잡히지 않나요? 국어 영역 공부해서 어디 쓰나. 이런 데에 써야지.
사실 2015와 2016 연속으로 '업자'들의 그건 털렸죠.
\vspace{5mm}

올해 다시 N수하겠다는 분들은 작년보다 보수적으로 기간 잡고 학습량은 3배 이상일 거다 각오하고 가세요.
그리고 이럴 때일수록 그냥 기본적인 교재나 푸세요. 적어도 제가 돌아본 바, 업자들도 감 못 잡기는 마찬가지입니다.
그에 비해서 어떤 과목이든 낼 수 있는 방향이나 소스는 참 무궁무진하거든요.
\vspace{5mm}






\section{수학 기본서 평가}
\href{https://www.kockoc.com/Apoc/604542}{2016.01.22}

\vspace{5mm}

평도 점수도 주관적
\vspace{5mm}

수학의 바이블 ★★★★☆☆☆
\vspace{5mm}

장점
$-$ 해설이 친절함
$-$ 문제의 등급별 구분이 과외에 좋음(숙제내주기 좋은 구조)
$-$ 어려운 문제가 괜찮은 것들이 있음.
\vspace{5mm}

단점
$-$ 처음에는 좋은데 나중에 보면 깊이가 없음. 초기에 비해서 발전이 없음.
$-$ 문제 유형이 생각보다 망라적이지 않음
$-$ 독자적인 사고를 길러주는 책은 아님, 교과과정을 쉽게 설명해주기 위한 눈치
\vspace{5mm}

수학의 정석(실력정석) ☆ or ★★★★★★☆
\vspace{5mm}

장점
$-$ 제대로만 본다면 $-$ 가령 5회독 이상이라면 포스 작렬함
$-$ 행간에 숨겨져있는 내용들이 장난 아님, 개념과 연습문제 하나가 최종보스급인 게 있음.
$-$ 이 책을 제대로 정복하고 수학 못 한다는 친구는 못 보았음
\vspace{5mm}

단점
$-$ 대부분 베개로 라면 받침대로 씀, 사놓으면 뭐하나 읽지를 못 하는데
$-$ 문제를 해설이 못 따라감, 그리고 초심자가 보고 이해할 수 있는 해설이 아님
$-$ 소수의 실력자를 낳은 동시에 다수의 수포자를 양산. 다들 자기가 소수가 될 거라고 해서 달려들지만
\vspace{5mm}

숨마쿰라우데(구판, 7차, 신판) 평균해서 ★★★★☆☆☆
\vspace{5mm}

장점
$-$ 실력정석과 온갖 학원가 수학을 적당히 비빔밥화해놓았으며 개념 설명이 납득가는 수준
$-$ 편집이 매우 괜찮은 수준이거니와 적어도 학생들이 쓴 교재 치고는 억지가 아니라고 느껴질 수 있음.
$-$ 적어도 몇몇 설명이나 문제는 최상위권 지향
\vspace{5mm}

단점
$-$ 구판이 가장 좋다는 건 개정해나가면서 하향되었다는 이야기가 아닌가, 정체성을 잃어버린 듯
$-$ 실린 문제가 어디서 많이 본 느낌이 남, 수학교재가 안 그런 게 어딨냐만 사실 실린 문제나 연습문제는 아쉬움
$-$ 처음 풀 때는 뭔가 괜찮은데 나중에는 실력정석이 끌리기 시작하는 이 느낌은 뭘까.
\vspace{5mm}

풍산자 수학 ★★★★★★☆
\vspace{5mm}

장점
$-$ 개념 설명이 쉽지만 핵심을 잘 찌르고 있고, 실린 문제도 쉬워보이지만 사실 어려운 것들이. 저자가 신경쓴 티가 역력함
$-$ 시중 나온 책 중에서 돌리기 가장 좋은 책임. 일단 수학 기본서는 빨리 돌릴 수 있는 것이 우선이 아닐까
$-$ 독학이 가능한 몇 안 되는 책
\vspace{5mm}

단점
$-$ 저자가 최신 수능 경향을 반영 못 하거나 생각 안 하는 걸로 보임. 이건 다른 교재로 보충하길 바람.
$-$ 이 시리즈만으로는 수능 대비할 수는 없음. 그냥 기본서 중 기본서로 보는 게 좋음
\vspace{5mm}

문제 쎈 ★★★★★★☆
\vspace{5mm}

장점
$-$ 가장 망라가 잘 되었으며 단권화하기도 편리함. 등급 구분도 잘 되었고 개념 '편집'도 괜찮음. 탄탄한 안정성
$-$ 수험 동향을 가장 잘 반영하고 있음. 다수가 보는 무난한 책이면서도 단점을 찾아볼 수가 없음.
$-$ 실린 문제수에서도 타의추종 불허함.
\vspace{5mm}

단점
$-$ 개념 쎈을 보기도 어중간한 개념 설명. 개념 설명에서 증명은 타참고서보거나 인터넷 검색해서 알아서 채워넣을 것
$-$ 해설이 간혹 납득 안 가는 것들이 있음. 그리고 일부 해설은 너무 억지 티가 나는 것도 없지 않음
$-$ 가장 좋은 풀이 엑기스나 유형조합을 하필 해설에만 넣어서. 다수 학생들이 그걸 놓치고 있음(...) 이거 따로 편집해서 핸드북으로 팔 것이지
\vspace{5mm}

마플 : 기출 ★★★★★★☆, 개념 ★★★★☆☆☆
\vspace{5mm}

장점
$-$ 기출 : 유형별 정리 잘 해놓음, 양이 꽤 많은 편, 문제 선정 괜찮음,
$-$ 개념 : 기본 개념에다가 기출을 잘 노임
\vspace{5mm}

단점
$-$ 기출 : 해설이 다소 억지인 게 있음, 다른 단원에 있어야 할 문제가 잘못 섞인 경우도 있음.
$-$ 개념 : 보충 개념서일지는 몰라도 이걸로 처음부터 가면 독학가능할까?
\vspace{5mm}

유난히 설명이 잘 된 수학  ★★★★★☆☆
\vspace{5mm}

장점
$-$ 개념서면에서 특화. 설명이 매우 상세함, 그리고 신선한 접근이 돋보임
$-$ 도해적 설명 자체가 매우 괜찮음.
\vspace{5mm}

단점
$-$ 업데이트 되지 않음, '개념보충서'로만 간주하는 게 좋음, 특정개념이 정말 이해가 안 갈 때에만 봐도 됨
$-$ 문제선정이 아쉬움(업데이트가 안 된 탓이 큼). 상세한 설명은 해법셀파나 교과서를 참조해보아도 된다는 점에서 대체성.
\vspace{5mm}

수학의 원리  ★★★★★☆☆
장점  $-$ 개념이나 문제가 꽤 크리티컬한 것들을 잘 겨냥해놓았음. 핵심 사항의 핵심적인 설명을 알고 싶으면 보시길  $-$ 저자의 내공이 느껴지는 책. 심화개념서를 보고싶은데 정석이 싫다면 이 책으로 가는 게 좋다고 권함  단점  $-$ 중심이 되기는 뭔가 아쉬움. 상위권용임을 차라리 표방하는 게 좋지 않았을까. 서브로는 좋지만 메인으로는 그닥임  $-$ 생각 외로 독학은 어려울 수 있음, 책이 애당초 중상위권 이상을 겨냥하고 있기 때문.제가 보지 않았거나, 별로 수험과 상관없다고 느껴지거나, 비평할 필요가 없는 것 등이면 언급은 안 합니다.참고로 위에서 보통은 2$\sim$3 종류는 보지 않을까 싶은데 최근 수능은 위의 것들로도 넘칩니다. 수능에서 커버가 안 되는 문제들은 사실 어떤 강의를 들어도 힘들 것입니다. 본인들이 부지런히 연습해서 수리적 마인드를 안 키우는 이상은.메이저 출판사가 낸 것들은 거의 다 괜찮으니 어떻게 보느냐가 관건이지 뭘 보느냐는 딱히 중요하지 않죠.어차피 국내수학은 거의 다 짜깁기이죠. 대부분 수학의 정석에서 파생된 것이고, 정석조차도 뿌리는 일본인지라.물론 주체적으로 그 이후에도 계속 공부해서 교재에 반영하는 새로운 흐름이 있어서 다음과 같은 걸 잘 구분해야합니다.A $-$ 일본 것을 열화복제한 것이 국내 사교육 시장에서 진화된 경우B $-$ 일본 본토에서 최근까지 발전해 온 것을 국내에 유입시킨 경우그런데 짜깁기 스킬이 수능이나 수리논술에 먹히지는 않죠.교과서가 재부각된 게 그런 이유. 왜냐면 교과서가 역설적으로 B 경향을 잘 반영해서리그 저자진 분들이 경문사 등에서 낸 책을 보면 연구를 엄청 열심히들 하셨죠.사실 한권으로 정리되는 게 있을지는 의문. 완성시키는 건 자기 머리이지 책이 아닐텐데.
\vspace{5mm}







\section{[교재글] 개정교육 과정인데 비싼 교재 살 필요가 있나요.}
\href{https://www.kockoc.com/Apoc/605445}{2016.01.22}

\vspace{5mm}

분명 이런 가상의 질문이 있을 것 같아서 적습니다만.
\textbf{자연수와 집합의 분할 빼고는} 사실 구입할 필요가 별로 없습니다.
\vspace{5mm}

그럼 자연수와 집합의 분할은?
쎈 기준으로만 치면 내용에 비해서 쎈 역시 문제 해설이 부족하다는 걸 느낍니다.
머리 쓰시다보면 자연수, 집합 분할의 추가정리는 스스로 발견하실 수 있겠죠.
\vspace{5mm}

지금 참고서 시장이 웃긴 게
메이저 출판사 교재조차도 그냥 기존 과정들 개정에 맞게 재배치하고 문제수준 '하향'한 다음 신유형과 기출 덧붙인 수준입니다.
그럼 실제로 올해 수능이 어떻게 나올 건지 대비되어있는가... 하면 이건 \textbf{아무도 모릅니다}.
그렇다면 학생들이 할 수 있는 것이라곤
시중 교재와 기출 문제집이나 열심히 풀다가 EBS 수특, 수완 나오면 그거나 풀면서
모의고사 나올 때마다의 경향을 매번 주시하면서 눈치작전 들어가는 것이죠.
\vspace{5mm}

제가 수험생이라면
풍산자, 쎈, RPM, 일품, 라벨, 일등급, 실력정석 중 택 3하고 교과서 꾸준히 풀다가
이해가 안 가는 건 EBS 강의 아니면 사설 메이저 강의 공신력있는 것 하나만 따라가면서
돈 안 쓰고 모아두고 있다가 6평 치고 나서 좋다는 게 확인된 문제집이나 강의를 구입할 것 같네요.
실제로 시험 경향에 맞게 보정한 교재들은 그 때 정도야 나오겠죠.
학생들도 어리둥절할 뿐만 아니라 교재 만드는 사람들도 다들 어리둥절해야 정상입니다.
제가 교재 만드는 사람이면 가만히 눈치보면서 모평 어떻게 나오나 본 다음에 7월달에야 낼 듯.
\vspace{5mm}

콕콕에서 공부하시는 분들이면 현재로서는 출제경향은 모르겠으니 그냥 기본교재나 다 푼다... 와 같은 농부모드로 가시길요.
EBS에서 출시하기로 한 교재 보니까 3, 4점도 있는 걸 보니 풀만한 교재들이 부족해서 망할 일은 없을 겁니다.
오히려 6월에 가서 학습량이 부족해서 교재들을 소화 못 해서 내년을 기약해야하는 불상사의 가능성이 높을 뿐이죠.
작년까지의 교재 보신 분들이라면 굳이 바꾸진 마세요. 달라지는 것 생각보다 없고, 오히려 하향된 측면도 없지 않으니.
정말 꼼꼼히 만드는 업자들이라면 아마 6월 정도는 지나야 낼 것입니다. 그래야 경향성이 거의 다 파악되니까요.
\vspace{5mm}

아마 올해는 교재 리뷰를 쓰지 않을까 싶은데 $-$ 다룰 가치가 있는 것들만 $-$
교재와 관련된 글은 댓글란을 닫기로 합니다.
본문은 별 문제도 없는데 생각없는 사람들이 교재를 특정하거나
일부러 대놓고 모 교재 아니냐라고 특정하려하던 수상한 아이디들도 있어서(부끄럽지도 않냐?)
그런 건 아예 차단해버릴 생각이어서입니다.
\vspace{5mm}

적어도 제가 언급하는 교재들은 '단점'에도 불구하고 공부할 가치는 있는 것들이라고만 아시면 됩니다.
\vspace{5mm}

+ 다만 기존 이과였는데도 삼각함수 같은 것이 힘들다거나  문과였다가 이과 갈아타시는 분들이라면 그냥 개정 교재로 가는 게 편합니다.
\vspace{5mm}

++ 현명한 학생이라면 6평 전까지 시중교재 풀 수 있느 것 다 풀고, 그 때 가서 출제 경향 파악되면 거기에 집중하겠죠.
\vspace{5mm}








\section{수험계의 착취}
\href{https://www.kockoc.com/Apoc/615234}{2016.01.29}

\vspace{5mm}

편의상
\vspace{5mm}

A $-$ 업자
B $-$ 상위권
C $-$ 하위권
\vspace{5mm}

으로 잡자.
\vspace{5mm}

A라는 업자는 B를 겨냥한 상위권 교재나 강의를 낸다.
그런데 여기서 주의할 것은 상위권을 가르치는 것이 하위권을 가르치는 것보다는 더욱 손이 덜 가고 쉽다는 것이다.
왜냐면 심화로 갈수록 낼 수 있는 건 한정되어 있거니와, 상위권 학생들은 핵심만 짚어주면 알아서 따라오거나 청출어람하기 때문이다.
당연히 합격률이 높게 나온다.
\vspace{5mm}

그러나 여기서 주의할 건 저건 그 상위권들은 A를 거치지 않았어도 좋은 결과를 내었을 가능성은 매우 높다는 것이다.
\vspace{5mm}

그런데 문제는 C의 선택이다.
C는 자신이 하위권인 걸 머리로는 알면서도 가슴으로는 인정하지 않는다.
결과에 더욱 절실하기 때문에 몸에 좋은 보약 아니 머리에 좋은 것은 빚을 내서라도 하려고 한다.
그래서 A 업자가 B로써 결과를 내보인 그런 비싼 상품을 구입한다.
물론 그 결과가 어떨지야 우리는 너무나도 잘 알고 있다.
\vspace{5mm}

하위권들이 상위권이 되는 코스들은 분석하기 쉽다. 왜냐면 없으니까.
그런데 사실 분석하고 말 것도 없다. 이 사람들은 일단 n이  3이상 넘어가는 경우가 거의 대부분이고
수험과정을 보면 1$\sim$2년 정도 시행착오를 하다가 아주 기본으로 돌아가서 '쉬운' 것부터 \textbf{반복을 엄청 많이 하다보면 드라마틱하게} 오른다.
소위 머리가 좋다는 친구들은 대부분 환경이 좋고, 환경이 좋다는 건 어린 시절부터 예습, 복습 등 반복이 습관화된 경우가 많다.
학교와 학원이 시간적 효율성이 떨어질 것 같지만 효과가 좋은 건?
수업이 아무리 엉터리여도 그리고 야자가 야만적인 것처럼 보여도 그 시스템은 반복을 보장해주거든.
\vspace{5mm}

그럼 A가 파는 상품이 도움이 되는가?
유감스럽지만 그렇지 않다. 하위권들에게 도움되는 건 없다.
그러나 A가 버는 돈은 \textbf{'하위권'들의 눈물이다.}
\vspace{5mm}

기성세대고 뭐고 다 욕할 것도 없다니까. 자기들이 속한 수험판에서의 착취도 스스로 극복 못 하면서 무슨.
하위권들이 상위권이 되는 방법은 \textbf{훨씬 더 쉬운 교재와 강의를 더 많이 반복하는 것} 뿐이다.
반복을 하다보면 이해와 암기도 최소량이 보장되고, 그래서 자신감이 생기면 뇌에서 공부의 쾌감을 느껴 계속 공부하게 된다.
반면 상위권들이 본다는 코스 가면 당연히 이해부터 될 리는 없고, 그래서 자기 머리가 나쁜가보다라고 연속좌절하는 것이다.
\vspace{5mm}

단원 내용이 이해가 안 가면 그 전 단원이나 기초 과정에 돌아가서 더 많이 연습한 다음 다시 돌아와야 한다.
그래도 이해가 안 가면 관련 내용에 관한 EBS 강의 여러개를 3번 이상 들어보면 된다.
그래도 이해가 안 간다? 거짓말할 것도 없다. 저렇게까지 반복했을리도 없거든.
뇌를 길들이는 방법은 반복 뿐이다. 이 사실만 알고 있으면 호구가 될 위험도는 낮아진다.
\vspace{5mm}

물론 자기들이 어려운 수험생일 때는 업자들을 욕하다가,
자기들이 그 입장이 되면 수험을 신비화시켜서 장사하려는 더욱 악랄한 인간들도 있지만.
뭐 인간세상 돌아가는 게 다 그렇고 그런 게 아니겠나.
운이 좋아서 $-$ 가령 찍은 게 맞아서 $-$ 수험에 성공한 사람도
자기가 머리가 좋아서 그리고 정말 실력이 좋은 선택받은 존재라고 과시하고 싶어지는 게 인지상정일 것이다.
물론 그딴 것은 더 나이먹어보면 알겠지만 없다.
수험만큼 평범한 것도 없으니까.
\vspace{5mm}




\section{여러가지 잡담}
\href{https://www.kockoc.com/Apoc/619855}{2016.02.01}

\vspace{5mm}

\item \textbf{1. 환경}
\vspace{5mm}

어제 총회챗에서 상담하면서 종합한 건
다들 공부머리가 있고 노력할 여지는 있는데 환경이 문제입니다.
\vspace{5mm}

집독학은 가능하면 피하세요. 집은 '내가 지배할 수 있는 공간'입니다. 그래서 공부가 잘 되지 않습니다.
인간은 원래 공부하기 싫어하는 동물입니다(공부를 함부로 해버리면 자기정체성을 상실하는 탓입니다)
공부는 강제당해야합니다. 그러니 공부를 강제하는 환경을 우선 조성해야합니다.
집에서 하면 시간낭비를 안 하고 편히 집중할 수 있다하겠지만 실제 그렇게 해서 성공한 사례 거의 없습니다.
\vspace{5mm}

학원을 가도 좋지만 도서관이 가능하면 도서관에 일찍 출근하세요.
도서관에서 최소 4시간을 버티면 됩니다. 4시간 하고 공부가 질렸다면 책을 빌려 봐도 좋고 도십시오.
심심하다하면 인강 mp3 끼고 여행(?)을 가거나 몰링을 하세요
특히 집에 있다가 폐인되신 분들은 하루에 1시간 30분은 쏘다녀야합니다.
\vspace{5mm}

\item \textbf{2. 진로가 뭔지 모르겠다.}
\vspace{5mm}

제가 가장 싫어하는 게 코 앞의 일도 처리 못 하면서 왜 지구반대편을 걱정하느냐입니다.
수능이 코 앞이면 수능을 잘 치고 그 다음을 고민하시면 됩니다.
무슨 입결상담이니 그런 것도 다 소용없습니다. 점수 잘 받으면 끝나는 문제 아닌가요?
\vspace{5mm}

다들 자기들이 합리적이라고 착각하겠습니다만 솔직히 한시간 뒤에 뭔 일이 일어나지도 모르는데 미래를 기정사실화하고 고민하고 있죠.
수험생들이 이런 경우는 대부분 공부하기 싫은 뇌의 핑계일 뿐입니다.
\vspace{5mm}

ex) 한의사 안 망해요?
\vspace{5mm}

한의대 가서 졸업한 다음에 따지세요
\vspace{5mm}

\item \textbf{3. 과고의대 떡밥}
\vspace{5mm}

가장 한심한 논쟁입니다.
우선 세금을 이야기하는 경우를 보면 대한민국 사람들은 특정 주제가 되면 그 전문가가 된다는 게 떠오릅니다만.
\vspace{5mm}

일단 기본적으로 왜 과학자 꿈에서 의사 꿈으로 바꾸느냐 그것부터 고려해야죠.
간단히 말해서 똑같이 착취당해도 과학자들은 대우를 못 받습니다. 그런데 의사들은 대우는 받습니다.
그렇다면 이 문제는 과고 출신들이 특정 커리큘럼 밟고 자격 요건 통과하면 의사만큼의 경제적 대우 해주는 걸로 해결할 문제죠.
그러나 10년 넘게 흘렀어도 바뀐 것이 없습니다. 그럼 뭐 어쩌란 건지?
\vspace{5mm}

만약 직장인이라면 "유학갔다오는 대신 우리 기업에 5년간 복무해 그 지식을 활용해야한다"라고 계약하는 건 가능합니다.
그런데 그걸 왜 특목고생들에게 '강요'하는 건지 모르겠네요. 거기 직접적인 법률 상의 권리 관계가 존재하는지도 모르겠고 말입니다.
경제적 보상이나 그런 걸로 유인할 생각 안 하고 "투자한 만큼 뱉어내라" 이런 한심한 이야기나 하고 있으니 문제죠.
가령 지금 변호사가 과거만큼 인기가 좋나요? 떨어지고 있죠. 보상이 낮아진다는 걸 아니까 그런 겁니다.
의사로 몰리는 걸 막고 싶다? 그럼 이공계 처우 높여주든가, 아니면 의대 정원 늘리든가 하면 됩니다.
\vspace{5mm}

왜 본질적인 해결은 간과하고 그냥 '만만한 학생'들만 두들겨대는지 모르겠음.
이 나라가 헬조선인 이유는 간단해요. 법 지키고 노력하는 사람들만 두들겨대니까 헬조선이죠. 애당초 그 어원도 그런 데서 유래된 것이고
그런데 이 나라는 가만히 보면 '공부 열심히 하려는 사람'부터 작살내려고 하지요.
\vspace{5mm}

과학고 죽이기가 아마 초반에 있었죠. 비교내신제 날려먹기가 어디서 찌른 결과더라하는 루머가 있었고
그 당시에는 왜 카이스트가 아니라 서울대에만 가느냐(...)라고 겐세이먹인 걸로 기억하는데
어떻게 보면 공익적인 메시지 같지만 지금 보면 그냥 어이없는 공격입죠.
\vspace{5mm}

이공계 대우가 좋다면 의대로 빠지는 경우 드물건데 말입니다. 그런데 세월 지나도 이거 나아진 게 있나요?
장학금 지원해줄 테니까 노예나 되라고 하는데 바보가 아닌 이상 누가 여기에 고분고분 따름?
이공계 살리자고 하는 사람들이 사실은 살릴 생각도 없어요. 이공계가 나라를 살린다하면서 드립치면서 자기 자식은 다 고시, 의대로 보내던데 뭘
자기 자식들을 그런데 보낸다는 건 결국 자기 자식들에게 유리하게 움직이겠다는 뜻이죠.
\vspace{5mm}

그럼 거꾸로 과학고 등에도 세금지원을 안 하면 이제 누가 과학고에 가나요. 다 일반고 가서 내신 학살하고 돈 많이 버는 데 가겠지.
그래놓고 나면 이 놈의 나라가 헬조선이니 이공계가 망한다 또 그딴  드립치고 있겠죠.
과학고 가서 열심히 공부해서 의대가면 욕먹고, 일반고 가서 의대 가면 욕 안 먹고. 이건 문제없나보죠?
\vspace{5mm}



\section{국어에 정답이 있을까.}
\href{https://www.kockoc.com/Apoc/619599}{2016.02.01}

\vspace{5mm}

문법(文法)이야 확실히 O, X 를 가릴 수 있죠.
왜냐면 문법은 어떻게 써야하느냐 지시하는 것이니 이랬다저랬다하는 게 곤란하니까.
\vspace{5mm}

문학/비문학 독해 문제에서 100$\%$ 정답은 사실 존재할 수가 없음.
1번 선지가 정답이고 2번 선지가 오답이라고 하는 건 1번 선지가 더 타당한 것이고,
더 타당하다는 것은 여러가지 관점에서 1번 선지가 2번 선지보다 우세하기 때문에지
모든 관정메서 1번 선지가 O고 2번 선지가 X여서가 아닙니다.
\vspace{5mm}

다시 말해서 문제를 풀 때에는 다양한 패널들이 있다고 칩시다.
패널 유재석, 김구라, 강호동, 조혜련 등
그리고 그 패널들이 각 선지마다 O나 X 패널을 들면서 지지나 반대를 표시하겠지요.
그래서 가장 많은 지지를 얻는 게 답이 되는 것입니다.
\vspace{5mm}

그런데 이걸 모르고 국어에서 O, X가 분명히 갈린다라고 착각하는데 사실 이건 매우 위험한 겁니다.
거꾸로 이것도 답인 것 같고 저것도 답인 것 같은데 어쩌냐하는 경우는 점수는 안 나올지 몰라도 사실 이게 정상인 것이죠.
\vspace{5mm}

국어에서 100$\%$ 정답과 오답이 갈리는 경우는 문제를 아예 그렇게 명쾌하게 낸다면 모르지만
사실은 그 수험생 자신의 가치관과 관정미 \textbf{그런 수험국어에 맞게 '토르소'가 되어버린 것}입니다.
문제는 그런 수험국어의 관점이라는 건 정상적인 사고와 거리가 멀거니와, 나중에 사람을 정말 꽉 막힌 선비로 전락시켜버린다는 것입니다.
수학과는 다릅니다, 수학의 생명력은 창의력이 아니라, 참과 거짓을 분명히 가를 수 있다는 것입니다.
형이상학적인 학문이니만큼 무엇이 진리이고 무엇이 거짓인지 밝히지 못 하면 쓸모가 없지요.
\vspace{5mm}

그러나 국어의 독해 쪽은 참과 거짓이 명쾌히 밝혀진다는 게 사실 거짓말인 것입니다.
그렇기 때문에 초기에 국어 독해를 풀면서 왜 이게 답이고, 저게 답이 아닌지 모르겠다하는 거야 말로 실제로는 정상인 것입니다.
\textbf{이 단계에서 제대로 공부하려면 왜 특정 선지가 정답인지, 오답인지 검사와 변호사 입장에서 주장해보는 식의 "나홀로논쟁"을 해봐야합니다.}
\textbf{그리고 여러가지 관점에서의 O, X 합산으로 다수결을 해보는 것이 맞습니다.}
적어도 수능기출은 다수결은 가능하게 출제해놓으니까요.
\vspace{5mm}

한데 시중 참고서든 강의든 국어에 무조건 100$\%$ 정답이 있다라고 해버리니까 여기서 상식있는 학생들이 헷갈리는 것입니다.
A라고 생각할 수도 있고, B라고 생각할 수도 있지 뭔 소리야 하면서 말이죠.
국어 = 답이 여러 개일 수도 있다 = 다양한 관점에서 O, X를 검토해보면서 나홀로논쟁으로 다수결을 해볼 것
수학 = 답은 하나다 = 다양한 관점에서 다양한 접근법으로 단 하나의 결론에 도달할 것.
\vspace{5mm}

그런데 지나치게 수학이 강조된 결과, 수학에서의 접근법 그대로 국어로 가져가는 경우가 많습니다.
그리고 글을 읽거나 이야기해보면 순수한 국어적 관점을 몰각해버린 케이스도 적지 않아요. 우선 독서량부터 절대적인 결핍상태가 많지요.
\vspace{5mm}





\section{실패하는 애들은 다 이유가 있음.}
\href{https://www.kockoc.com/Apoc/621373}{2016.02.02}

\vspace{5mm}

이 시기에
\vspace{5mm}

\item 1. 논쟁
\item 2. 게임
\item 3. 컥챗
\vspace{5mm}

과거에는 왜 어른들이 "하라는 공부는 안 하고"라는 말을 했나 반항하기도 했는데
지금 나이처먹고보니 그냥 그게 다 \textbf{진리}다.
기성세대나 상류층이 다 해처먹어서 그렇지 않느냐고 했을 때에도
지금 생각해보면 개념파에 해당하는 어른들이 '다 자기 탓이다'라고 해서 화냈던 기억이 나는데
자기 탓이라는 건 비단 '노오력' 뿐만 아니라 '학습'까지 포함한다는 것을 뒤늦게야 알았다.
\vspace{5mm}

예컨대 열심히 노오력하는 노예가 있다고 치자, 노예가 해방되는 방법은 하나다.
세상을 전복시키든가, 아니면 도망쳐버리든가, 아니면 몰래 공부를 해서 권력을 얻어 노예문서를 불태우든가
저렇지 않고서 열심히 일한다고 해도 기약은 없는 것이다.
하다 못해 부패하고 썩어빠진 어른일지라도 \textbf{"공부하지 말라고 한 적"}은 없었다.
젊은이들을 어떻게 착취해먹을지 고심하는 사장이라도 공부 이야기에서는 1$\%$는 착해질 수 있는 것이 아닐까.
\vspace{5mm}

망하는 애들은 다 그만한 이유가 있다.
본인이 돈을 버는 것도 아닌데
2월이 되었는데 게임하고 있거나 이상한 논쟁이나 벌이고 게임을 한다면 공부를 안 하고 있으면
당연히 망하는 거지 그럼 흥할 일이 있나?
지금 공부 안 하고 막판에 몰아치기 하면 된다... 그래서 그 몰아치기가 성공한 예를 알고싶다.
\vspace{5mm}

실패한 이유 가지고 하늘 탓할 것 없다. 공부한 것 따지면 되는 것이지
상담해보면 결국 구구절절한 사정도 \textbf{"공부하기 싫어서 꾸며낸 구라"}가 대략 95$\%$는 된다.
공부할 놈은 정말이지 목에 칼이 들어와도 책 읽고 문제풀고 있다.
자기가 내일 죽을 것을 알면서도 사과나무 대신 기출문제를 암송하고 있으면 된다.
\vspace{5mm}

그래놓고 시험치고 나서는 자기들이 열심히 공부했다고 자기 스스로 말한다.
공부했는지 안 했는지는 주변 사람이 평가하는 것이지 본인이 스스로 평가하는 게 아닐텐데?
그리고 공부 열심히 한 사람은 \textbf{절대 자기가 열심히 했다고 말하지 않는다}.
겸손도 있지만 사실 저게 맞는 말이거든, 아무리 해도 해도 부족하다고 느껴지는 게 공부임.
\vspace{5mm}

분명 집 떠나서 도서관에서 꾸준히 공부하라고 얘기했고, 게임 손에 잡지 말라고 강력하게 경고했으며
황금의 3개월 날리면 힘들 거라고 했는데도 이걸 어기는 친구들을 내가 어떻게 봐야할까?
\vspace{5mm}














\section{우유부단한 게 가장 최악}
\href{https://www.kockoc.com/Apoc/624579}{2016.02.05}

\vspace{5mm}

주식에서도 돈잃는 가장 최악의 패턴은
\vspace{5mm}

\textbf{오르기 직전에 못 견디고 팔아버림,}
\textbf{떨어지기 직전인데도 오르고 있다고 사들임.}
\vspace{5mm}

기술보다 마음이 중요하다는 전형적인 예다.
\vspace{5mm}

마찬가지로 $\sim$ 할까 하는 사람들의 패턴은 세 가지이다.
\vspace{5mm}

\item \textbf{1. 부모님 충고를 듣고 $\sim$ 하는 경우}
\item \textbf{2. 부모님 무시하고 망하든 말든 해보겠다라고 하고 소신것 나가는 경우}
\item \textbf{3. 어느 것도 못 하고 시간만 낭비하는 경우}
\vspace{5mm}

저 중 최악은 3번이다.
\vspace{5mm}

인생은 턴제 RPG가 아니라서 선택을 보류하는 동안에도 시간이 흘러간다.
한여름에 아이스크림 2개 중 하나를 택일강요받는다면 어느 걸 먹을까 눈돌릴동안 다 녹아버린다.
\vspace{5mm}

그럼 꼭 이런 이야기를 하지. \textbf{"실패하면 남는 게 없잖아요"}
아 뭐 이런 병신들이 다 있나.
\vspace{5mm}

실패가 남는 게 없긴 뭐가 없어. \textbf{"지혜"와 "교훈"}을 얻는데.
책에 쓰여진 그런 것 말고 본인이 괴로워하면서 체득한 자기만의 지혜와 교훈인데
아니 무엇보다 실패할 것을 알면서도 도전하는 사람이 남과 다른 게 있지. 그게 \textbf{'용기'} 아냐?
가슴 근육 키우고 배에 군주제 실현한다고 용기 있는 게 아님, 실패할 것을 알면서도 그래 해보자 나서는 게 용기 아녀?
용기 키운다고 해병대 갈 필요가 없다. 손해보는 것을 알면서도 그리고 심리적으로 위축되어도 해보는 게 용기지.
남들이라면 무섭다 손해본다라고 할 때도 \textbf{'에잇, 경험이다'라고 하는 게 용기 아냐?}
\vspace{5mm}

지혜도 없고 교훈도 없고 무엇보다 용기도 없다면 $-$ 특히 그게 남자면 $-$
그 사람이 좋은 대학 가더라도 인생은 별로 기대할 건 없다.
\vspace{5mm}

물론 용기와 만용은 다르다.
용기있는 사람은 실패를 하더라도 적어도 '안전선'은 마련해놓는다, 하지만 만용을 부리는 사람은 자살도 서슴지 않는다는 것.
그런데 만용은 '용기없는 사람'들이 궁지에 몰렸을 때 빠지는 극단적인 상태다.
용기를 강조하는 이유는 간단하다, "만용"을 부리지 않으려면 용기가 필요하기 때문이다.
\vspace{5mm}

예컨대 똥통대학 걍 다닐까요 아니면 재수할까요.
사실 20대 전체로 보면 어느 것이든 큰 차이는 없다.
똥통대학을 다니더라도 본인이 영업력이 출중하고 인맥 잘 잡아서 그 분야에서 먹거리를 잡으면 되는 것이고
재수를 하면 죽기살기로 해서 학벌 높이면 되는 것이다.
그 어느 쪽이든 실패하더라도 본인이 복기하면 지혜, 교훈, 용기를 모두 얻을 수 있다.
\vspace{5mm}

하지만 상당수는 선택이 모든 걸 좌우한다라고 믿고 있는 것 같은데
나 역시 그렇게 믿었지만 요즘 생각은 달라졌다.
우리 스스로가 용기도 없고 신중하지도 않고 노력하지도 않은 것 가지고 괜히 '선택' 탓을 하는 게 아닌가.
그럼 당시 선택은 그 당시에는 나름 신중히 숙고한 결과 아니었던가.
선택을 어느 쪽을 하는 게 문제가 아니라, 한쪽을 선택했으면 그냥 그걸로 밀고 나가는 게 답이다.
\vspace{5mm}

선택을 잘한 사람이 돈 많이 벌고 떵떵거린다, 나도 저렇게 살고싶다"라는 망상.
제가 말씀드리겠음, 그런 사람들이 \textbf{정말 나중에 제대로 망합니다}.
미신론적인 추명학으로 말하면 별 노력도 안했는데 돈이 굴러오는 사람은 재운이 들어오는 건데요,
그거 절대 공짜 아닙니다. 재운이 나갈 때는 정말 신기하게 잘 망합니다
게다가 그런 사람들은 실제로는 지혜도 용기도 없고 무엇보다 따라온 사람들이 다 돈보고 따라온 사람들이라서
재운이 나가면 정말 비참하게 망하고 아무도 안 쳐다봅니다.
\vspace{5mm}

이야기한 김에 더 적으면 '비정상적으로 좋은 운'을 자기 능력으로 착각하는데
능력과 실력은 "나쁜 운"이라도 걷어내는 것을 말하는 것입니다.
진짜 실력자들은 운이 나쁜 사람들에게 있습니다, 그리고 이런 사람들이 나쁜 운이 걷히고 좋은 운이 오면 대성하죠.
\vspace{5mm}

괜히 어설프게 선택 잘 하면 인생 트인다 그딴 망상 갖지 말고. 소신있게 결정하세요.
결과는 어떤 선택을 했느냐보다도 어떤 노력을 했느냐로 좌우되니까요.
\vspace{5mm}






\section{콕콕에서 연구할만한 주제들}
\href{https://www.kockoc.com/Apoc/626685}{2016.02.07}

\vspace{5mm}

$-$ 어떤 볼펜이 수학풀이에 더 적합한가?
$-$ 풀이과정에서 연습장 공간을 어떻게 분할하여 나눠볼 것인가
$-$ 시험장에서 떠올릴 수 있는 과목 '두문자'는 어떻게 개발할 수 있나?
$-$ 카페인을 이용한다면 어떤 타이밍에 먹는 게 좋은가?
$-$ 적정 수면 시간과 타이밍은 어떤 타임이 좋나
$-$ 과도한 공부 이후 반드시 오게 되는 스트레스 관리는 어떻게 할 것인가
$-$ 공부할 때 들어도 무해한 음악은?
$-$ 주의 환기를 위해 봐도 좋은 사진이나 그림은?
$-$ 문제를 읽을 때 어떤 순서로 어떤 개요를 그려나가야하는가?
\vspace{5mm}

...
\vspace{5mm}

찾아보면 주제는 참 다양하다.
\vspace{5mm}

....
\vspace{5mm}

수험칼럼 따위가 아니라 사실 저런 걸 공동으로 연구하고 발표해보는 게 과학의 단계가 아닌가 싶은데
자뻑은 아니고 스스로도 학습일지를 읽거나 상담해보면 어느 정도 패턴을 파악해볼 수 있지만 이건 아직 '미신'에 불과하다.
왜냐면 맞기도 하고 안 맞기도 하니까.
\vspace{5mm}

그런데 로켓트에 들어갈 엔진이라거나 인공 심장에 들어갈 판막에 들어갈 '부품'의 정밀성.
그런 정밀성을 갖춘 학습공학이나 학습시스템을 정리해서 한 사람의 인생을 바꿀 수 있다면 이건 대단한 일이 아니겠나.
사실 사소한 필기습관이라거나 볼펜 종류만 가지고도 성적이 바뀌어 그걸로 인생이 갈라지는 경우가 많다.
\vspace{5mm}

특정 강사 강의나 인강만 들으면 잘 할 수 있어라는 토테미즘 부족사회 제정일치 그런 시대가 아니라
어떤 하드웨어나 소프트웨어로 효율성있는 시스템을 밟아 바뀌어나가느냐 하는 과학의 시대가 이미 왔어야하지 않나 싶기도 하지만
요즘 드는 생각이 인간이 만들어낸 최고의 자본은 황금도 화폐도 아닌 결국 \textbf{'지식' 밖에 없다}는 것이다.
그리고 그 지식도 뇌로 숙달된 것이 아니면 죽은 것에 불과하다.
\vspace{5mm}

그리고 이것이 만약 수능에만 집중된 것이면 나 역시 업자 장사치에 지나지 않는다는 얘기를 들을지 모르지만
사실 저건 사람이 태어나서 죽을 때까지의 전과정에 적용될 수 있다는 걸 요즘 느끼고 있다.
한 개인이 실천할 수 있는 이상적인 자가교육시스템이란 어떤 것일까.
\vspace{5mm}

일종의 독설적 칼럼이라면 나도 신나게 쓸 수 있고 심지어 주작할 수도 있다. 그리고 이게 얼마나 미신투성이인지는 안다.
그렇다고 학습법이라는 게 정해져 있느냐하면 그런 건 아니다. 미연구된 분야가 많다.
가령 A4 용지만 가지고도 학습에 어떻게 활용할지에 대한 참 방법이 많은데 국내에서는 이것조차 정리된 것도 없다.
콕콕에서 꾸준히 활동하면서 총회까지 들어와 글을 쓰는 분들이 이런 주제를 가지고 서로 연구하고 결과를 대조해보았으면 좋겠다.
\vspace{5mm}






\section{모 강사 자서전(?)을 읽고}
\href{https://www.kockoc.com/Apoc/626687}{2016.02.07}

\vspace{5mm}

유명하다는 강사 자서전(?)을 읽었다. 물론 거기서 뭔가 기대한 건 아니고 어떤 식으로 사람들을 휘어잡았나 보기 위해서이다.
\vspace{5mm}

꽤 괜찮은 요식업 만화로 국내번역명 '라면요리왕'과 '라면서유기'라고 있는데 참 통념을 깨는 만화다.
장사해먹기 위해서는 대중들의 싸구려 입맛에 맞춰야한다라는 무시무시한 진실이 통념없이 드러나있다.
진주인공 대머리 세리자와부터가 은어로 맛을 낸 진짜 라면을 내도 안 팔리자
에라 모르겠다라고 기름기 듬뿍인 라면을 냈는데 그게 잘 팔려서 대박난 케이스다.
\vspace{5mm}

모든 이를 만족시키는 최고의 요리를 내면 돈을 번다.... 그건 거짓말인 것이다.
마찬가지로 오늘 들으면 내일 죽어도 좋은 그런 강의가 정말 인기가 좋은가? 사실 그렇지 않다.
수억대 연봉 강의는 들어보고나면 내 개인적으로는 실망한 경우가 많았다. 혹시 내가 부족해서 그런가 했는데 그건 아니다.
그런 강의들이 잘 팔리는 건, 생각하는 게 \textbf{싸구려인 학생들이 원하던 싸구려 내용이기 때문}이다.
사실 무턱대고 강의듣는다는 친구들은 혼자 공부할 줄 모르는 사람들이다. 아니, 그보다 공부하는 것 자체를 싫어한다.
공부하는 것 자체를 싫어하는 사람들을 만족시키는 방법은 두가지이다.
하나는 강의 내내 유머를 적절히 넣고 쉬운 내용으로서 공부했다라는 포만감을 주는 것이고
다른 하나는 욕설이 섞인 카리스마 강의로서 '마조히즘적'인 것을 일깨워주는 것인데
\vspace{5mm}

저기서 성공한 건 후자다.
내가 읽은 자서전도 후자를 참 적절히 써먹은 케이스다.
그 내용을 읽고난 것은, 컴플렉스 덩어리였던 사람이 참 이런저런 경험을 하고나서 사람 다스리는 법을 알고나서
학생들을 어떻게 갈궈야 속으로 좋아하는지에 대한 '마조히즘적'인 진실을 일찍 간파했었구나라는 것이다.
(물론 오해사기 싫어서 말하면 나는 이런 걸 대단히 혐오한다. 일단 성격도 이상하다고 느끼는 데다가 변태같아서 그렇다)
\vspace{5mm}

강의기법이나 내용이야 뭐 별의 별 것은 없고.
\vspace{5mm}

그럼 왜 폭력적이거나 독설적인 강의가 인기가 좋을까.
\vspace{5mm}

\textbf{첫째, 그런 폭력적인 것을 경험하고 나야 비로소 교육받았다고 착각하는 구슬픈 유전자 때문이다.}
\textbf{둘째, 사람들에게 있어서 교육이란 뭔가 '학대당하는' 것이다. 조선인들에게 민주주의적인 교육은 필요없다.}
\textbf{셋째, 힘있고 폭력적인 메시지야말로 역설적으로 주입이 잘 되고, 이게 실제로 고득점에 도움이 되기 때문이다}
\textbf{넷째, 강사의 카리스마를 돈을 주고 소비하는 건 터무니없는 가격인 걸 알면서도 해외 명품을 소비하는 것과 비슷한 만족감을 준다.}
\vspace{5mm}

그래서 저런 강사를 한번 듣고나면 xxx 들어라 하는 자발적인 전도사가 된다.
물론 이런 카리스마가 먹히는 건 길어도 한 5년?
특히 이런 강의들은 이상하게 현강이 더 강조되는 것 같은데 하기야 '폭력'은 인터넷보단 직접 마주보는 게 더 실감나지 않겠나.
\vspace{5mm}

그런데 문제는 저런 달콤한 폭력에 중독된 사람은 \textbf{끊임없이 저런 강의만 찾아다닌다는 것이다.}
왜냐? 카리스마있게 야 이 xx야 욕하면서 주입하는 그런 것에 길들여지고 나면, 민주주의적으로 차분하게 '생각한다'는 것을 잊어버리거든.
게다가 혼자 밍밍한 느낌으로 사고하는 것보다 카리스마 강사가 이렇다 저렇다하는 메시지가 더 선명하니까 그걸 다 흡수해야겠단 절박감이 오지.
\vspace{5mm}

슬프지만 이런 수법은 앞으로도 꽤 먹힐 것이다. 가르치는 걸로 일단 호구지책하려는 사람은 위와 같은 걸 잘 감안하시길.
그 당사자들이야 숨기고 싶은 노하우일지 모르지만 솔직히 노하우치곤 뭐 이런 싸구려가 다 있나 싶을 정도다.
이런 걸 부러워해서 '앞으로 강의하고싶다'고 하는 케이스는 타락하기 전에 정신차리거나 다른 버전으로 가는 걸 권하고 싶다.
솔직히 말이 카리스마지, 실제로는 시험에 절박해서 공포에 질린 사람들을 걍 협박하는 것과 큰 차이가 있나?
그리고 내가 아는 한 그런 식의 강의하다가 그만 둔 사람들, 다른 분야에서 일 제대로 못 한다. 그만큼 그 사람들도 비정상이 되었단 얘기거든.
\vspace{5mm}

나도 한 때 강의라는 걸 꽤 많이 들어보았지만 그런 강의가 도움이 되는 건
그런 강의 내용을 내 스스로 반박할 때가 아닌가 싶다.
혼자 책 찾아 읽어보면서 내가 들었던 강의 내용이 틀렸구나를 하나하나 찾고 논증하는 것.
그럼 그게 돈이 되어서? 아니, 그런 걸 하나하나 잡아내는 것 자체가 그냥 \textbf{'재밌어서'} 그렇다.
돈이 안 되는 학문이든 개발의 동기는, 과거에 몰랐거나 불가능했던 걸 내가 눈충혈되고 밤새면서 '알게 되거나', '가능하게 만드는 그것'인데
소위 독재자처럼 군림하던 사람들의 그런 내용이 실제로는 출처가 어디서 나왔는지 그리고 어떤 게 구라인지 간파하는 게 그냥 재밌는 것이다.
\vspace{5mm}









\section{라이벌}
\href{https://www.kockoc.com/Apoc/626811}{2016.02.07}

\vspace{5mm}

혼자서 20km를 뛰라고 하면 못 뛴다
그러나 하프마라톤 대회라면 뛸 수 있다. 다른 사람이 뛰는데 내가 멈출 수는 없기 때문이다.
에어로빅 등의 GX도 그러하다.  혼자서는 하기 힘들지만 같이 하니까 할 수 있는 것이다.
독학재수, 특히 집독학을 비추하는 이유도 그렇다.
혼자서 하면 사실 공부를 하다가 자의적으로 중단하더라도 아무도 못 막고 본인이 수치심을 못 느낀다.
똑같이 공부하는 '라이벌'이 있어야만 공부할 수 있는 것이다.
\vspace{5mm}

일지를 쓰고 남의 일지를 보라고 하는 이유가 그 때문이다. 다른 라이벌들이 \textbf{어떻게 공부했나 보고 자극받고 따라가는 게 효과가 좋다}.
이 일지조차도 콕콕에서는 참 오해하는 사람들이 많은 것 같은데 내 입장에서는 그게 오히려 더 이상해보였다.
공부에서 '경쟁'은 정말 필요악 중 필요악이다, 특히 가르치는 사람이 서로 비교질하는 게 그게 즐거워서가 아니다.
비교를 시켜서 마음에 상처받게 하고 열등감을 느껴야만 본인이 공부한다,
그런 말을 하면 "네가 안 해도 하잖아"라고 하는데 그런 애들은 다른 점에서 문제가 많다.
경쟁을 좋아한다고 다 공부를 잘 하는 건 아니지만, 공부를 잘 하는 애들은 \textbf{경쟁 자체를 즐기고 당연시 한다}.
친구 아들 딸을 이야기하는 엄마가 옳다는 건 아니다. 그런데 그 엄마들도 사실 경쟁하라고 부추기는 건 나쁘지는 않지
다만 부추기되 도움을 실제로 안 주니까 문제일 뿐이다.
\vspace{5mm}

서로 공부를 자기가 몇시간 했으니 어떤 교재를 몇회독햇느니 하는 걸 자랑하는 것이 갈등을 낳는다고 하더라도 이건 권장될 얘기다.
\textbf{저런 얘기를 들어서 빡친다고 하더라도 그런 빡침 자체가 본인에게는 매우 도움이 되는 것이기 때문이다.}
남이 10시간 정말 공부한 것을 보고 슬프거나 우울한 감정이 들더라도 그것 자체는 본인의 기준치를 높여주므로 결국은 좋다.
\vspace{5mm}

\textbf{최근에 왜 비교질을 하고 싸움붙이느냐라는 지적을 받은 적이 있어서 내가 한소리해야겠다.}
그런 걸 문제 삼으니까 당신들이 발전이 없는 거라고.
애들끼리 경쟁하고 참고서 뭐 보나 서로 산업스파이질하고 공부시간 속이고 하던 것?
그거 내가 중1 때 하던 짓이다. 다시 말해서 옛날부터 공부하던 사람들은 저런 건 너무 당연시했단 것이다.
심지어 공부 잘 하던 사람들끼리 서로 속사정 잘 알면서도 8시간 공부한 걸 2시간 공부했다고 농담하던 게 겸손으로 통하던 시절도 있다.
그리고 이건 현재도 똑같은 보편적 진리다.
공부 잘 하는 애들이 스트레스를 푸는 건 라이벌을 어떻게 이겨먹느냐 연구하고 노력하는 것이다.
프린세스 메이커 게임에 스트레스가 0이 되는 이벤트가 라이벌 등장인데 이건 정말 현실적인 것이다.
\vspace{5mm}

자기는 저런 비인간적인 경쟁이 싫다고 하면 \textbf{그냥 '평범한 대학' 가서 '평범하게' 사시면 된다.}
그런데 저런 것도 비판하기나 하고 실천도 못 하면서 남들보다 잘 살거야 그딴 드립은 치지 않았으면 좋겠다. 세상에 그런 게 어딨냐?
겉으로는 성인군자숙녀인 척하는 우등생 남녀들이 속으로는 얼마나 많이 연구하고 노력하고 남들 하는 것 벤치마킹하는지 모르나?
\vspace{5mm}

소위 공부 잘 한다는 친구들의 칼럼도 읽을 때는 주의해야하는 게 그거다
얘들이 말하는 많이 푸는 게 그냥 많이 푸는 게 아니다. 다른 애들보다 최소 1년 이상, 그리고 2배 이상은 가는 게 평범하게 푸는 것이다.
\textbf{자동차라고 하면 롤스로이스가 기본이고 간식이라고 하면 거위 간, 그리고 술이라고 하면 발렌타인 두자리 년수}
이걸 평범이라고 한다. 그래서 여기서 빈익빈부익부 가중되는 것이다.
\vspace{5mm}

일지 쓰시는 분들은 자기의 복제 캐릭터 아무개를 가정해보자.
일지를 하루 단위로 쓰던 일주일로 쓰건 그 아무개가 어디까지 공부했을 것이다라고 소설을 써보길 바란다.
그렇게 비교해보면 자기가 얼마나 태만하게 공부하는지가 느껴질 것이다.
\textbf{그리고 총회에서 상담하는 분들 중 가능성있다는 분들은 의도적으로라도 경쟁하고 비교시킬 것이다.}
물론 그게 싫으면 빠져도 되지만, 저건 그냥 놀이가 아니다. 저렇게 해야 사실 오래 버틸 수 있고 올라갈 수 있다.
\vspace{5mm}










\section{계획을 짜는 알고리즘을 간략히 적어보자.}
\href{https://www.kockoc.com/Apoc/628832}{2016.02.08}

\vspace{5mm}

계획을 짜는 건 자기 컴플렉스를 푸는 게 아니다.
계획은 "100$\%$ 실천가능한 공부단위"를 작성하는 것이다.
\vspace{5mm}

가령 쎈을 다 풀겠습니다... 이건 계획이 아니다, 그냥 목표다. 이걸로는 절대 실천을 할 수 없다.
반면 \textbf{"하루에 쎈을 30문제 풀기로 하겠습니다. 그럼 3달이면 100$\%$ 완료합니다, 30문제이니 지치지 않고 풀 수 있습니다"}
라고 해야 이게 계획인 것이다.
\vspace{5mm}

비유하면 좀 그렇지만 '나라는 가축에게 사료를 언제 얼마나 배급할 것인가'
이게 계획이다. 그런데 대부분은 계획을 짠답시고 비싼 사료만 사놓고 이걸 한번에 먹으려고 하다 배터져 죽는 상황을 반복하고 있다.
\vspace{5mm}

다시 말해 계획은 \textbf{단위 공부량을 줄이는 것}이다.
대신 양을 줄이되 \textbf{현실성을} 담보하고, 그럼으로써 \textbf{짧은 기간에 많은 양을 끝내는 것}이다.
다들 착각하는 게 하루 공부량을 많이 잡기만 하면 할 수 있다고 하는 데 그건 틀린 이야기다.
자기가 할 수 있는 능력 범위 내에서 공부하는 게 우선 전제되어야 한다.
\vspace{5mm}

그리고 반드시 계획은 '자본'을 늘려나가야한다.
예컨대 쎈, 마플, 급품벨을 한꺼번에 푸는 것과, 쎈 먼저 그 다음 마플, 그 다음 일등급, 일품, 라벨 순으로 풀어나간다고 하자.
총량은 별로 다를 것 없어보이지만 실제로는 다르다.
전자의 경우 학생은 정말 아무 자본도 없는 상태에서 쎈, 마플, 급품벨을 모두 상대해야한다.
\vspace{5mm}

하지만 후자의 경우 학생이 쎈을 끝내면, 그 쎈이 자기 아군이 되어서 마플 공략을 도와주고, 마플을 마치면 쎈과 마플이 급품벨 공략을 도와준다.
즉, 하나하나 완성해내가면서 자기의 자본을 늘려나가는 것이 전제되어야 한다.
'순서'대로 하나하나 공략해나가는 것은 한단계 한단계 아군(자본)을 늘려서 그 다음 적을 처리해나가는 가장 현명한 방법이다.
\vspace{5mm}

그렇다면 대략 1주일이나 2주일 정도 잡고 2과목의 교재 한권씩을 끝내나가면 된다.
예컨대 수특을 잡는다면 매일 하루 30문제 수학, 하루 영어 10지문은 기본으로 보면서 월수금은 국어, 화목토는 탐구.
이런 식으로 진행해나가면서 3월말까지 끝내는 걸로 기한을 잡는 것도 훌륭한 전략이다.
공부량이 적다고 할지 모르지만 그건 틀린 생각이다. 3월말까지 이것만 완수하더라도 4월초에 수특이 이미 내 '아군'이 된 상태다.
그만큼 실력도 늘고 부담이 준 상태에서 다른 교재들을 공략해나가면 되는 것이다.
\vspace{5mm}

아마 이런 걸 잘 모르는 사람들은 머리 탓을 할지 모른다
그러나 본질적으로는 '순서'대로 일처리를 하지 않은 것이 문제다.
공부든 일이든 순서대로 처리해나가면서 자기 자본을 조용히 늘려가 나중에 규모를 키우는 것이 정석이다.
그렇기 때문에 아무 실력도 없는데 처음부터 실력정석이나 어려운 실모를 푸는 건 바보같은 짓이다.
절망적일수록 가장 쉬운 교재를 정해서 그걸 확실히 내 걸로 만들고 차근차근 나가는 게 공부다.
\vspace{5mm}

그럼 이렇게 쉬운 계획을 누가 방해하는가?
그건 자신의 \textbf{과욕}과 \textbf{불안감}이다.
욕심을 부리는 건 좋은데 그걸로 자기 능력을 과대평가하면서 짧은 기간에 너무 많은 걸 '순서없이' 끝내려 한다.
게다가 불안감 때문에 이것저것 풀지 않으면 안 될 것 같다.
이 글을 보는 다수 n수생들이 사실 이런 코스로 세월을 날렸을 것이다.
\vspace{5mm}

명심하시길, 스피드와 물량도 중요하지만, 그 \textbf{모든 건 '순서'가 지켜져야만 의미가 있다는} 것을.
순서를 지켜나가면서 조금씩 10원, 20원 쌓다가 나중에 10,000원, 20,000원까지 모으다보면 그것이 기하급수적으로 불어나는 것이다.
남들이 어떻게 한다더라 신경쓰지말고, 인내심으로 버티면서 순서대로 자기 자본을 늘려나가야 한다.
그렇기 때문에 공부에서 기술보다 '마음'이 중요한 것이다. 기술은 시간과 노력을 단축시켜준다. 이건 \textbf{'인내심'}의 적이다.
하지만 너무 늦으면 안 되기 때문에 일단은 '빨리' 시작해야하는 것 뿐이다.
\vspace{5mm}


\section{슈퍼 마리오로 설명하는 입시}
\href{https://www.kockoc.com/Apoc/633676}{2016.02.13}

\vspace{5mm}

급식충(?)들은 모를 수도 있는 수퍼마리오 초판입니다.
보통 2시간은 걸리는 게임인데 4분 57초만에 피치공주를 구하는 동영상입니다.
현재 수능난이도에서 만점이 나와야하는 과목은 대체로 이렇게 문제를 풀어나가면 된다고 보면 됩니다.
\vspace{5mm}

저런 플레이가 가능한 건
\vspace{5mm}

$-$ 수도 없이 연습했기 때문이다.
$-$ 적과 함정이 어디서 나오는지 암기했기 때문이다.
\vspace{5mm}

그럼 실제 시험은 어떤 식으로 나오나
\vspace{5mm}

+ 초고수 플레이
\vspace{5mm}

잘 하는 친구들은 저 정도는 한다가 보면 되겠습니다.
그럼 저게 어떻게 가능할 것인가.... 당연히 연습과 암기죠.
물론 처음보는 게임이라고 할지라도 '보편적인 게임 진행'에 숙달된 사람이라면야.
\vspace{5mm}

강의에만 의존하는 건 프로게이머 방송만 보고 게임을 잘 할 수 있다라고 믿는 것과 같습니다.
\vspace{5mm}





\section{공부를 하면서 스트레스를 받아야 하는 이유}
\href{https://www.kockoc.com/Apoc/636254}{2016.02.15}

\vspace{5mm}

믿거나말거나
고교시절에 의무적으로 과학 보고서를 써서 내야하는 게 있어서 햄스터 실험을 한 적이 있습니다.
스트레스를 주었을 때 햄스터의 지능에 어떤 영향이 있을 것인가... 라는 것인데 참고서는 전파과학사에서 나온 스트레스 어쩌구.
일단 가설은 스트레스가 지능을 떨어뜨린다... 였는데
\vspace{5mm}

엉성한 실험이었다고 하지만 오히려 실험결과는 스트레스를 가한 햄스터가 더 똘똘했단 것입니다.
이거 실험을 엉터리로 했나 했지만 기한은 다가와서 그냥 그렇게 냈는데
이게 생물선생님에게 우수하다라는 평가를(... 아니 다른 녀석들은 어떻게 쓴 거야 도대체)
\vspace{5mm}

...
\vspace{5mm}

일반적으로 공부할 때 스트레스를 받아야하는 이유는 그래야 머리가 '좋아'지기 때문입니다.
한계를 넘어서는 자극을 받으면 기존의 인지구조가 변형되고 뇌에서는 그런 충격을 줄이기 위한 변화를 모색하는 경향이 있습니다.
하루에 30문제가 한계량인데 50문제를 풀게하면 스트레스를 받고 고통을 느낍니다.
이 경우 관찰되는 방향은 세 가지입니다.
첫째는 문풀을 줄이는 방향으로 가는 것 $-$ 즉 공부를 회피하거나 공부하지 않기 위해서 핑계거리를 만드는 경우
둘째는 문제를 푸는 인지구조가 바뀌어버리는 것 $-$ 즉 머리가 좋아지는 경우
셋째는 그 고통 자체를 즐기는 것 $-$ 즉 공부변태가 되는 경우
\vspace{5mm}

대체로 둘째와 셋째가 바람직(?)하다고 할 수 있습니다. 사실 셋째를 권장하는 이유는 '천재'를 이기는 건 변태 빼고는 없으니까요.
하지만 대부분은 첫째로 갑니다.
이유없이 짜증내거나 화를 내는 경우 $-$ 물론 당사자는 그럴 싸한 이유를 만듭니다 $-$ 이력을 분석해보면 '공부 스트레스'를 받은 케이스죠
\vspace{5mm}

어떻게 보면 여기서 범재와 수재 갈리는지도 모르지요.
\vspace{5mm}

망치로 사정없이 두들겨서 변성시켜야하는데 대상이 고정되어있지 않으면 멀리 튕겨나가버리겠죠.
그리고 거기서 학습한 효과로 망치를 피해나갈 것입니다.
하지만 제대로 묶여있다면 망치로 얻어맞으면서 아주 단단해지고 치밀해지겠죠.
감금, 수감되어있다면 지적자극에 얻어맞아야되고 그렇게 하면 뇌가 바뀔 수 밖에 없습니다.
\vspace{5mm}

독학으로 하면 공부가 잘 된다고 하는 경우는 이런 본질을 모르는 경우죠.
사실은 하기싫은 공부야말로 진짜 공부입니다. 그런 공부를 해서 엄청 스트레스를 받으면서 변성 단계에 이른 다음에,
적절한 시기에 다른 공부를 하거나 휴식을 취하면 그동안 뇌가 재성형되고 그 다음에 다시 공부하면 이렇게 쉬웠냐하는 느낌을 받죠.
\vspace{5mm}

.....
\vspace{5mm}

실패하는 이유는 별 게 아닙니다. 하기 싫은 공부를 안 해서입니다.
하고싶은 공부를 해놓고 공부량이 많다고 해보았자 소용이 없습니다. 그걸로는 뇌가 안 바뀝니다요.
\vspace{5mm}








\section{슈퍼 마리오로 설명하는 고수}
\href{https://www.kockoc.com/Apoc/636264}{2016.02.15}

\vspace{5mm}

포도님이 링크시킨 영상의 주인공이 더 막 나가는 플레이를 소개
\vspace{5mm}

$\#$ 개조(Kaizo) 마리오 3 스피드런
\vspace{5mm}

절대 지루하지 않습니다, 보는 도중에 여러번 감탄사를.
물론 여러번 죽기도 하셨지만, 컨트롤과 공략이 파이브 스타 스토리즈의 파티마급이 되시네용
대충 국어, 영어, 수학의 고수라고 하면 저런 느낌이라고 보시면 됩니다. .
\vspace{5mm}

$\#$ 원판 마리오 공략 스피드런
\vspace{5mm}

스테이지를 뛰어넘는 마술피리 안 쓰고 그냥 공략한 건데 이걸 1시간 내에.
그것도 그렇지만 마지막 5분 영상이 압권입니다. 저런 식으로 클리어가 가능하구나
아예 닌텐도 코드를 다 꿰뚫고 있었네.
\vspace{5mm}

이제 공부의 패러다임은 다름아닌 '게임'이죠.
남에게 자기가 문제푸는 걸 어떻게 뽐낼 수 있느냐 그 자체로 동기부여하는 것도 매우 중요하다는 것.
수학 양치기를 한계량까지 한 사람은
평범한 학생이 30분 끙끙거리는 걸 3$\sim$5분 내에 예술적으로 풀이하죠.
그게 처음보는 문제일지라도 $-$ 그리고 그 맛에 공부합니다.
\vspace{5mm}






\section{통과의례}
\href{https://www.kockoc.com/Apoc/636843}{2016.02.15}

\vspace{5mm}

외모지상주의를 조장하는 건 아니지만 그냥 비현실적인 예를 들겠습니다.
A란 사람은 못 생겼는데 열심히 노력해서 돈을 벌었습니다. 그리고 그 돈으로 성형을 합니다.
B란 사람도 못 생겼는데 돈이 없어서 혼자 운동하고 얼굴요가(...)를 해서 저절로 미남/미녀가 됩니다.
\vspace{5mm}

외모지수는 똑같습니다. 그런데 이런 경우 사람들이 선택할 건 B일 겁니다.
'돈'을 주고 타인의 행위로 완성시킨 걸 A의 것으로 보지 않기 때문입니다.
하지만 B의 경우도 어떻게 보면 자가 성형인데 저런 경우는 자연미로 인정받습니다.
\vspace{5mm}

그럼 공부도 마찬가지겠죠
만약 C와 D라는 공부 못 하는 사람이 있다고 칩시다.
C는 거액의 돈을 들여서 오버테크놀러지 기계로 뇌에 데이터를 전송받아 똑똑해집니다.
D는 거액의 돈으로 사교육을 받아 노가다 수험을 한 다음에 똑똑해집니다.
둘 다 거액의 돈을 들였다고 하더라도 누굴 선호하겠습니까. 대답할 필요는 없겠죠.
이게 참 신기한 것입니다. 우리가 사람을 평가하는 기준이라는 것은 무의식에 있을 것 같은데 나름대로 논리가 서있단 것이죠.
원래 배우자를 선호할 때 받는 유전자의 암묵적 명령이라는 게 여기에 있을지도 모릅니다.
\vspace{5mm}

수험에서 배우는 게 쓸데없다고 하더라도, 수험 자체가 유의미한 이유가 여기에 있을 것입니다.
주인공이 정말 진정한 주인공이 되기 위해서는 '고난'과 '역경'을 거쳐야하기 때문이라는 것은 '신화 코드'의 하나죠.
그런 통과의례를 거친 사람이야말로 종족번식에 유리하다는 유전자의 가르침일지도 모릅니다.
수험에서 배우는 건게 쓸데없더라도, 그것이 수험생 본인을 '단련'시켜준다는 건 바뀌는 게 없는 것입니다.
하다 못해 운이 나빠서 결과가 안 좋다고 하더라도, 그렇게 단련된 뇌가 어디 가는 게 아니죠.
\vspace{5mm}

모든 게 유전자 덕이다하는 사람들을 보면 세계사 공부도 안 했는가 싶죠.
황족, 왕족, 귀족, 하다 못해 천재의 가계도 같은 걸 추적해보면 그거 3대 이상 가는 경우는 별로 없습니다.
총회 게시판에서 언급된 합스부르크 왕가의 경우도 끼리끼리 혼인으로 금수저 극대화 전략 폈다가 유전병으로 말아먹었죠.
영국 왕실이나 일본 천황가도 상징적 존재가 아니면 이미 축출당했을 것입니다.
\vspace{5mm}

반면 흔히 언급되는 록펠러나 로스차일드, 발렌베리 일가의 경우는 '교육'을 \textbf{후덜덜하게 시킵니다.}
아니 무엇보다도 유대인이나 화교만 보아도 공통점이 있죠. 혈통은 사실 별 것 없는데 교육이 장난이 아니라는 것.
유대인들은 분파도 다양하지만 현재의 유대인들은 구약성서에 나오는 그 유대인과는 정말 거리가 멉니다만
그들은 유대교 전통에 따른 탈무드라는 교육자본을 갖고 철저히 교육에 매진해왔으며
화교들 역시 이런 점에서는 마찬가지였습니다.
\vspace{5mm}

요즘 수험가에서는 의치한 가면 월 얼마 번다 이런 걸로 무조건 거기 가야한다.
라고 하는데 개인적인 생각은 저런 바람은 이뤄지지 않을 거라고 여깁니다.
\vspace{5mm}

흔히 하는 이야기가 수요와 공급이라 하는데 이건 자본주의 경제를 절반만 언급한 것입니다.
만약 특정 직종이 공급이 통제되어서 그만큼 차익을 누린다고 하는 걸 시장경제는 가만히 냅두지 않습니다.
기술이 발달하건 여론으로 제도가 바뀌든 해서 그걸 반드시 날려버립니다. 그런 게 유지되면 사실 자본주의 사회가 아니죠.
게다가 정작 꿀빤다는 의료직종에 종사하는 분들도 그 분들이 대학교 갈 때 돈보고 간 것은 거리가 있을 것입니다.
(정작 그 때 돈보고 대학간 건 공대나 경영대가 아니던가요)
\vspace{5mm}

오히려 지금은 남들이 외면하지만 미래에 성장가치가 있는 전공에 가는 게 나을지도 모릅니다.
똑똑한 놈들이 몽땅 의치한에만 매진한다면, 타 전공 분야를 10년 이상 바라보고 자기가 거기서 일인자가 되는 전략이 나은 것이죠.
그럼 그런 일인자가 되기 위해 필요한 건?
\vspace{5mm}

수험이라는 고통스러운 의례를 자력으로 통과하는 것입니다.
\vspace{5mm}

제가 싫어하는 사람이라면 돈보고 가라고 얘기하겠지만
제가 생각하는 사람이라면 너무 돈을 보지 말라고 이야기하겠습니다.
요즘 느끼는 것이지만 거액의 재산이든 돈에 대한 탐욕이 사람의 눈을 멀게 한다는 것을 느끼고 있기 때문입니다.
검소하게 살아야하는 이유는 저축을 위해서 혹은 청빈이 아름다워서가 아닙니다.
가난하게 살아야만 더 많이 보이기 때문입니다.
\vspace{5mm}









\section{선택을 못 하는 이유}
\href{https://www.kockoc.com/Apoc/636857}{2016.02.15}

\vspace{5mm}

흔한 고민이 반수냐 아니면 그냥 재수냐 그건데
이럴 때에는 본인이 합리적이라는 생각을 버려야하며
아울러 선택안은 다시 정리하면
\vspace{5mm}

\textbf{1 $-$ 대학교도 대충 다니면서 운좋게 반수}
\textbf{2 $-$ 여전히 고졸, 하지만 남는 시간으로 딴짓하면서 재수하기}
\textbf{3 $-$ 이것도 저것도 못 하고 그냥 끌려가기}
\vspace{5mm}

이렇게 해야 정답일 겁니다.
\vspace{5mm}

체크리스는 ⓐ 할 수 있는 것, ⓑ 해야하는 것, ⓒ 하고싶은 것 인데
여기서 할 수 있다는 건 과정이 아니라 '결과'입니다.
가령 학교 다니면서도 수능 칠 수 있다고 하는 건 애매합니다.
전과목을 B를 맞을 것인가 A를 맞을 것인가 분명히 얘기하고 과반이 A가 나올 수 있다라고 하는 등의 기준을 짜고 결정해야죠.
그게 아니면 이것도 저것도 못 합니다.
그런데 재수를 하자니 본인이 롤을 한 경력이 있고 하루 공부시간이 5시간 넘은 경우가 없으며 이걸 하면 고졸이다라고 하면 골치
\vspace{5mm}

반수 : 할 수 있는 것도 꽝, 해야하는 것 애매, 하고싶은 것 꽝이면 이 경우는 그냥 재수로 빨리 돌려야겠죠
재수 : 할 수 있는 것 애매모호, 해야하는 것 꽝, 하고싶은 것 오케이.
\vspace{5mm}

이렇게 나열해서 비교한 뒤 과감히 빨리 선택하는 게 나음
\vspace{5mm}

이건 거꾸로 말해서 뭘 '버릴지' 확실히 결정하라는 이야기임. \textbf{버리지 않으면 얻지도 못 합니다.}
다들 얻는 것에만 환장해서 버리는 것을 모릅니다, 버리지 못 하니 얻지도 못 하고 시간만 버리는 것이죠.
그런데 N수생들 보면 실패한 이유가 별 게 아닙니다.
\textbf{자존심이 강해서 버리지 못 하고, 버리지 못 하니까 얻지도 못 하고, 그렇게 시간은 날라가고.}
\vspace{5mm}

쓴소리하자면 벌써 2월 끝나가죠. 황금의 3개월 공부하는 사람은 이미 공부했죠.
아마 공부한 사람도 느낄 겁니다. 3개월도 금방 지나가는데 이거 공부할 것 더 널렸네.
하지만 공부 안 한 사람은 앞으로의 기간이 참 길다고 착각들 하죠.
\vspace{5mm}

다시 강조합니다만 버리세요. 그리고 그 놈의 자존심 제발.
이게 가장 큰 적입니다.
어그로끌자면 현역으로 유수 대학 못 갔으면 그 사람이 자존심 챙길 자격이라도 있습니까.
바로 부족한 것 인정하고 백의종군하고 노예생활해야 인간이 되는 거지.
그런데 그 놈의 자존심 때문에 그 자존심을 채우는 '낭비적인' 방향으로 움직이니까 시간만 더 갉아먹는 거죠
그리고 장사치들 배나 신나게 불려주고 말입니다.
\vspace{5mm}

일지쓰라는 게 일단 공부하는 사람들이나 질문하라는 것도 있지만
지금 고백하는 또 하나도 있죠.
\textbf{일지를 꾸준히 쓰면 작성자가 자존심이 저절로 무너집니다,}
\textbf{자기가 공부를 생각보다 게을리하는구나, 엉터리로 하는구나를 확인하니까요.}
\vspace{5mm}






\section{수험에 대해서 착각하는 것.}
\href{https://www.kockoc.com/Apoc/638785}{2016.02.17}

\vspace{5mm}

\textbf{공부 + 경쟁 = 수험} 입니다.
공부를 잘 하는 것만으로는 불충분합니다. 결국 경쟁자들을 내리찍고 자기가 올라가야 끝나는 거예요
내가 봐서 좋은 건 남이 봐도 좋습니다. 그렇다면 그 남을 물리치지 않고서는 성공할 수 없어요.
그래서 $\sim$ 해도 되나요... 라는 질문을 보면 이건 십중팔구 실패하겠구나 보는 겁니다.
다들 으르렁거리면서 앞서나가려고 미친 듯이 공부해도 실패해서 3$\sim$4년 공부하는 경우도 있는데
보통 공부를 안 하시던 분들이 이제 공부를 하면 자기가 드라마 주인공이라도 되는 줄 안다는 것이죠.
\vspace{5mm}

씁쓸한 이야기입니다만
그렇다고 요즘 서민이나 하류층이 헝그리 정신이라도 있는가하면 그것도 아닙니다.
한명씩 잡아 추궁해보면 충분히 공부할 수 있는데 안 하고 본인이 게으름 피우거나 게임이나 환락에 빠진 경우가 대부분입니다.
\textbf{헝그리 정신은 정작 부모 스펙도 괜찮은 부잣집 자제들이 갖고 있다는 게 더 절망적인 사실}이죠.
심심하시면 정말 명문대 진학 성공한 사람들 표본을 모아서 가정환경 확인들해보시길요.
집안이 하류이면서 자기가 서민인데도 불구하고 영화나 드라마처럼 뭔가 반전이 있을 거라는 근거없는 믿음에 빠진 케이스가 많습니다.
어느 수험사이트 가던 성공한 케이스들 가정환경 분석해보시죠.
가정환경이 안 좋은 경우는 정말 본인이 악바리 헝그리 정신을 발휘한 케이스입니다.
\vspace{5mm}

그리고 냉정히 말하면 올해 시험이라면 이제는 사실 전 일지 써온 사람들 빼고는 조언할 필요는 없다 여깁니다.
황금의 3개월은 이제 다 지나갔습니다. 그리고 여기서 이미 절반 이상은 \textbf{'결판'났다}고 보고 있습니다.
과장하는 게 아니라 실제로 이 3개월동안 공부한 것이 복리효과가 붙어서 정말 막판을 결정합니다.
11월부터 2월까지 금방 지나가죠? 이제 5월까지도 순식간이지요.
사실 11월부터 2월까지 안 한 사람이 3월$\sim$5월에도 할 가능성은 낮죠.
자기들은 할 수 있다 착각합니다만요.
\vspace{5mm}

특히 웹검색하면서 꿀교재 꿀강의 찾으면 된다..
헛짓하지 말고 문제집이라도 꾸준히 다 푸십시오. 그거 공부하기 싫어서 결국 책, 강의 수집한다하는 뇌의 발작 그 이상 그 이하도 아니니까요.
황금의 3개월 기간동안 공부 안 햇다, 그런데 게임을 잡았다 하는 남자분이면 그냥 병역 빨리 처리하라고 얘기하고 싶고
그게 아니라고 하면 반드시 집독학하지말고 도서관에 가든 하다 못해 스파르타 학원에 가서 개고생하는 걸 권하겠습니다.
그것 외에는 방법이 없을 겁니다.
\vspace{5mm}

그리고 올해 아니면 내년이다.... 아마 내후년 혹은 내내후년이 될지도 모릅니다.
\vspace{5mm}






\section{공부는 자기 좋으라 하는 겁니다.}
\href{https://www.kockoc.com/Apoc/639054}{2016.02.17}

\vspace{5mm}

"돈은 잃어도 머릿 속에 든 것은 잃지 않는다" 기억상실 크리는 어쩌고요.
도 그렇지만 기본적으로 \textbf{공부는 나 좋으라고 하는 것}이지 남 좋으라 하는 것이 아닙니다.
\vspace{5mm}

\textbf{공부 힘들어죽겠다 미치겠다 노오력은 무슨 노오력이냐 하는데}
솔직히 부모든 누구든 공부 강요할 이유가 없습니다. 부모자식도 남남이거든요.
\vspace{5mm}

최근에 와서 어떤 미친 놈들인지 몰라도 의대 돈 많이 번다라는 선동을 하고 있더군요.
그래서인가 개나소나 의대간다 어쩐다라는 식으로 병신들 같이 선동당하는 사람이 늘지 않았나 싶은데
그런데 말입니다, 공부 안 하는 사람이 의대 가고싶다고 하는 건 '걍 날로 돈벌고 싶다'라는 저속한 욕망 그 이상 그 이하도 아닌데
공대도 마찬가지이지만 공부 안 한 사람이 운좋게 의대간다 칩시다, 재앙이죠. 그 사람 손에 \textbf{한두명 목숨 잃을 게 아니니까요}.
돈만 바라보는 것은 미개한 '단식부기'적인 사고입니다.
현대사회의 틀은 복식부기죠. 차변$-$대변, 의무$-$권리, 임차$-$임대, 거액의 돈$-$무거운 책임
\vspace{5mm}

이렇게 보면 공짜라는 건 사실 없습니다.
\vspace{5mm}

그러나 공짜가 없는데도 보면 사람들 사이에 격차라는 게 생기죠.
타고난 선천적 능력 때문이기도 하지만, 결국 누가 시간을 덜 '낭비'했느냐입니다.
만약 게임을 했다고 칩시다, 순간적으로야 즐겁겠죠. 그런데 그 게임을 한 즐거움은 절대 '적분'되지 않습니다.
공부는 매일매일 해도 그것이 효과없는 것 같지만 조금씩조금씩 쌓여서 복리효과를 발생시킵니다.
그러나 게임이나 환락은 할 때에는 매우 즐겁지요. 그리고 조금씩 인생을 갉아먹기 시작합니다. 그것의 효용? 사실 없어요.
없는 걸 떠나서 그런 걸 즐기는 평범한 사람들을 '병신'으로 만들지요.
\vspace{5mm}

부모님이 무능하건 유능하건 이런 건 압니다. 왜냐면 자기들의 성공, 실패가 저렇게 좌우되었다는 것을 알기 때문이죠.
그래서 최소한 한번 이상은 자녀들에게 공부하라고 독려합니다.
그러나 어느 순간부터는 독려를 안 하는데 물론 부모가 소라넷이나 가고 등산불륜이나 하는 개쓰레기인 경우도 있지만,
'내 자식은 공부할 녀석이 아니다'라는 확신을 받았기 때문입니다.
그럼 이 확신을 바로 잡으려면? 그거야 자기가 미친 듯이 변태적으로 공부한다는 걸 '실천'해서 보여줘야죠.
그럼에도 불구하고 다수의 젊은이들은 '오랄 스터디'만 강합니다. 입으로만 $\sim$ 하겠다, 그러니 돈 내놔.... 이런다는 것이죠.
\vspace{5mm}

부모님이 자기 공부 도움 안 준다라고 하는 사람은 가슴에 손 얹고
정말 한번이라도 게임하지 않았나, 게으름피우는 모습을 안 보여주었나.. 등을 돌이켜보길 바랍니다.
정말 공부를 제대로 하는 사람은 눈빛이 살벌하거나, 변태적이거나 그렇기 때문에 꽤 강력한 기를 발산합니다.
하다 못해 강도조차도 '절도'로 끝내야겠구나라는 느낌을 줄 정도죠. 공부에 빠진 사람은 그냥 거기에 미쳐있거든요
\vspace{5mm}

그래서 상담 요구하는 사람에게는 제가 냉담한 겁니다. 공부에 미쳐있지도 않으면 그냥 그걸로도 과반은 실패한 건데 뭐 어쩌란 거야.
그리고 공부하기 싫어? \textbf{자기 좋으라고 하는 걸? 그럼 안 하면 될 것 아냐. 나이가 몇인데 어리광부려?}
\vspace{5mm}







\section{재종학원이 나은 이유}
\href{https://www.kockoc.com/Apoc/640034}{2016.02.17}

\vspace{5mm}

재종학원이 만능은 아닙니다만 적어도 인강보단 나은 이유는
바로 \textbf{호손 이펙트}로 설명됩니다.
\vspace{5mm}

https://ko.wikipedia.org/wiki/$\%$ED$\%$98$\%$B8$\%$EC$\%$86$\%$90_$\%$ED$\%$9A$\%$A8$\%$EA$\%$B3$\%$BC
\vspace{5mm}

간단히 말해서 \textbf{사람은 "타인의 관찰"을 의식하는 순간 행동이 달라진다}는 것입니다.
\vspace{5mm}

$-$ 돼지우리에서 살던 여자가 남친이 라면먹고 가고싶다고 하니까 방을 치운다
$-$ 사단장님이 방문하신다니까 부대 전체가 깨끗해졌다
$-$ 혼자서 못 뛰던 장거리 코스를 마라톤 대회에 참여하니까 완주할 수 있었다.
\vspace{5mm}

만약 강의 내용을 받아적고 정리한다면 그건 인강을 못 따라갑니다.
그러나 성과는 '실강'이 더 좋은 이유는 저걸로 설명됩니다.
인간은 \textbf{타인과의 '관계' 속에서 긴장하고 집중}하니까요. 혼자 냅두면 대단히 태만해집니다.
\vspace{5mm}

생활습관 안 잡히는 사람들은 백날 계획 세울 필요 없이,
그냥 조직 속에 들어가거나 타인과 경쟁하는 모드로 가는 게 직빵입니다.
그리고 더불어 '집단 속'에 있다는 것 때문에 안심이 됩니다. 혼자 죽지는 않거든요
\vspace{5mm}

그러나 \textbf{반전}
\vspace{5mm}

다만 들어간 이상 거기서는 반드시 선두를 유지해야 합니다.
처음에야 좋다고 하지만 3, 4월 지나면 또 다시 거기서 태만해지죠. 파레토의 법칙이 어김없이 작용
결국 공부 안 하는 80에 속해서 집단으로 태만해지는 일이 벌어질 수 있습니다.
그 내부에서도 20 안에 들지 않으면 소용이 없는 거죠.
이른바 80의 함정에 빠지면 돈은 돈대로 내면서 작살날 수도 있습니다.
조직도 결국 평안함이 계속되면 "혼자 죽지는 않는다라는" 느낌이 오히려 독으로 작용하는 것이죠.
\vspace{5mm}

비싼 돈 들여 갔다면 라이벌 정한 다음에 그 라이벌을 성적으로 작살내야 영화보러간다거나 술마신다거나(...) 하는 식으로 잡으시길 바랍니다.
그냥 재종 갔으면 따라가면 되지.... 라고 하다가 관료주의 함정에 어김없이 빠져버리니까요.
\vspace{5mm}








\section{수험은 중국무술이 아닙니다.}
\href{https://www.kockoc.com/Apoc/640886}{2016.02.18}

\vspace{5mm}

무술 중에서 '기'를 발산하고 '내공'을 발휘한다.... 는 것은 뻥이죠
실제로 중국무술의 경우는 기껏 해보았자 그 기원이 명말청초 때입니다. 그것도 대부분 명맥 끊김.
기를 모은다거나 장풍 쏜다거나 하는 것은 무협지나 일본만화에서 나온 것이지 실제로 검증된 게 없어요.
\vspace{5mm}

그런데 특히 그런 무협지든 일본만화는 두 가지가 있죠.
첫째, 주인공이 알고보니 혈통이 금족보였다
둘째, 주인공이 개고생하다가 기연을 만나 내공과 필살기를 전수받는다.
여담이지만 이걸 깨부순 게 헌터헌터인데 연재가 참 불안정해서리(특히 그 핵무기 장면은 소름이 쫘악)
\vspace{5mm}

그런데 공부 안 하다가 꼭 어떻게 공부해요하는 사람들이 저런 중국무술적인 발상에 빠져 있는 것 같습니다.
공부 잘 하는 사람들이 보는 특수한 책을 자기가 보면 바로 깨달을 수 있다.... 빨리 갈 수 있을 것이다라는 것.
\vspace{5mm}

그런데 그딴 건 단언코 말해서 없어요.
일부 저자들이 자기들 책이 굉장하다 어쩐다 해서 비싼 값에 팔아치우는데 까고 말해 '무안단물' 수준입니다.
수학으로 치면 가장 좋은 책은 제가 보기에는 쎈수학입니다. 그리고 쎈수학을 권했습니다,
다들 아폭이 무슨 신사고 알바냐 어쩌구 하는데 더 놀라운 사실은, 쎈수학 C 스텝까지 다 끝낸 경우 생각보다 없어요.
국어요? 시중교재 다 고만고만합니다. 가장 좋은 건 본인이 깊이있고 어려운 책을 꼼꼼히 읽어왔냐는 것입니다.
영어요? 강의와는 무관해보입니다. 이것 역시 어린 시절부터 원서 읽고 미드 듣고 회화 꾸준히 해온 애들이 잘 합니다.
재종학원 찬양? 본인이 얼마나 좋은 환경에서 특혜받고 자랐는지 모르는 경우입니다.
\vspace{5mm}

저도 많은 표본을 보앗다고 단언은 못 하겠지만 정리되는 진리는
\vspace{5mm}

\textbf{$-$ 양치기를 꾸준히 한다고 성공한다고 할 수는 없다, 그러나 성공한 사람 중에 양치기 안 한 녀석은 없다.}
\textbf{$-$ 머리가 좋은지 안 좋은지는 모른다, 그러나 두뇌회전이 빠른 경우는 다 교육환경이 대단히 좋은 케이스였다.}
\textbf{$-$ 어려운 문제 잘 푼다고 자랑하거나 소위 수험경향 따지는 녀석들이 시험 당일 쉬운 문제에 털리고 색다른 문제에 통수맞는다}
\textbf{$-$ 수험서나 강의 명품 따지는 놈들은 최소 3년 내내 그러고 있다.}
\textbf{$-$ 성적은 회독수에 비례한다. 1번만 강의 듣거나, 책을 읽거나, 문제를 풀어보았자 소용이 없다, 10회독은 해야한다.}
\textbf{$-$ 수능에서는 출제 난이도 따지는 건 무의미하다. 무조건 난이도 높다고 생각하고 대비해야한다.}
저기 어디 지름길이 있는지 제가 궁금합니다. 그런 걸 알면 \textbf{진짜 부자가 될 수 있거든요}.   종합격투기 나가서 목숨을 건 싸움을 해야하는데 무슨 $\sim$ 경을 외우고 기를 모으면 이길 수 있어...
이런 것도 아니고 말입니다요.
\vspace{5mm}

나중에 과학기술이 발달해서 뇌를 이식 혹은 복제할 수 있다거나
사람끼리 서로 USB로 연결해서 기억을 몽땅 전송해서 신경망도 복제할 수 있다면 모를까.
그게 아닌 이상 나머지는 다 \textbf{'거짓말'}입니다.
\vspace{5mm}

중국무술이 망한 이유 중 하나가 문화대혁명이라죠.
문화대혁명 때 홍위병들이 다구리치니까 고수도 별 것 없었다고(...)
아니, 그 전에 무술이 대단하다면 다구리도 맞서 싸울 수 있어야하잖아.
\vspace{5mm}

현역 때 잘 하는 애들이 있죠. 그런데 그 친구들은 어린 시절부터 공부를 꾸준히 해온 케이스입니다.
선행을 미리 한 애들은 딴 친구들이 3$\sim$4수할 것을 미리 3$\sim$4년 앞당겨 공부함으로써 현역으로 가는 것이고
선행을 하지 않은 친구들일지라도 초, 중학교 때 공부하는 습관이나 틀이 잘 잡혀있어 한번 보면 바로 익힙니다.
\vspace{5mm}

적어도 수험에 대해서는 분명 과학적으로 정리되는 것들이 이미 있는데
공부를 하지 않은 분들이 여전히 '무속적'으로 접근하는 것을 보면 참 안타깝습니다.
무속이 과학으로 바뀌기 힘드니 결국 '종교'로 진화하는 것이고 그러니까 특정교재만 보면 된다, 특정강의만 들으면 된다로 바뀌죠.
\vspace{5mm}

적어도 교재 질문은 최소한 국영수탐 남들이 많이 푸는 걸 3회독은 한 다음에 했으면 좋겠음요.
질문하는 것 답변해주기 싫은 이유는 다른 걸 떠나서 본인들이 공부를 안 하는 케이스여서입니다. 안 할 건데 왜 질문을 하지?
실천하고 질문해주셨으면 좋겠어요.
대답이야 뭘 해보았자 남들 많이 푸는 교재 그냥 달달달 외우고 반복하고 하는 걸로 끝납니다.
솔직히 말해서 쎈수학만 가지고도 10번 이상 돌리고 부족한 내용 검색질을 하든 타 수학교재로 보충하든지 질문하든지 채워도
그걸로도 수리영역 150$\%$는 대비할 수 있습니다.
\vspace{5mm}

+
복싱, 씨름, 스모가 차라리 현실적이죠.
특히 스모는 우스꽝스러운 뚱보들 때문에 겉이미지는 그래보입니다만
그게 살찌우는 게 오히려 현실적. 뭘로 가든 체중 키우는 것 못 따라간다....
사실 테크닉은 '물질'의 부족을 메우기 위한 것입니다. 물질이 풍부하다면 테크닉도 크게 필요가 없죠.
지금 이 세계를 지배하는 건 동양의 신비인가요? 아니면 미국의 물량주의인가요?
\vspace{5mm}

++
수험서 보아도 참 수요와 공급의 원리.
아마 쎈수학 같은 책이 소수만 갖고 있고 복제가 불가능했다면 한권당 가격은 꽤 어마어마했을 것입니다.
실제로도 1권당 1천만원의 가치는 있습니다. 그런데 이게 대량복제되어서 판매되니까 사소해보이는 것이죠.
하지만 무엇보다 수험서는 자기가 공부하지 않는 이상은 폐품입니다.
\vspace{5mm}






\section{아주 흔한 수험패망의 루트}
\href{https://www.kockoc.com/Apoc/640984}{2016.02.18}

\vspace{5mm}

A라는 책은 내용이 풍부하고 크게 손색이 없음,  그런데 이 책은 어디서든 쉽게 구할 수 있고 무엇보다 저렴합니다.
B라는 책은 사실 내용이 틀린 게 많고 지나치게 치우쳐져 있습니다, 하지만 신비주의적으로 광고되고 잘 알려져 있지 않습니다.
\vspace{5mm}

어찌되었든 붙을 학생이라면 A나 B나 모두 보겠죠. 그런 걸 가릴 시간이 있으면 걍 다 보는 게 맞으니까
하지만 B에 대한 안 좋은 이야기를 하겠고 그 결과 B가 도태되어야하겠죠.
\vspace{5mm}

한데 재밌는 건 학생들이 B를 꽤 신경쓴다는 것입니다.
신경쓰는 이유는 간단합니다. '좋아보인다'는 것이죠
그런데 왜 좋아보이느냐, 읽어보긴 했느냐 하면 그건 아니라고 합니다.
남들이 B가 좋다고 하니까 왠지 보지 않으면 큰일날 것 같다고 이야기하지요.
\vspace{5mm}

그래서 A와 B를 모두 구입합니다. 하지만 수능 직후 A나 B나 깨끗합니다(...)
\vspace{5mm}

자, 이런 일이 왜 벌어질까요?
\vspace{5mm}

수험이란(그리고 사실 삶이란) "뇌와의 끊임없는 싸움"입니다.
뇌는 그것의 자극을 추구하는 경향이 있습니다. 식욕, 수면욕, 성욕... 등 자극적인 것을 추구합니다.
그래서 어느 일본 과학자가 쓴 책에서 그러더군요. 뇌는 주인을 배신한다고(반면 대장은 주인에게 충실하다나)
그런데 여기서 '불안감'이라는 게 중요합니다.
\vspace{5mm}

수험생은 누구나 불안합니다. 당연히 이 불안감은 공부로 이겨내야하죠.
하지만 공부도 괴롭습니다, 그렇기 때문에 뇌에서는 공부가 아닌 다른 방식으로 불안감을 해소하려하죠.
그건 바로 "남들과 다른 것"을 소유하는 것입니다.
그렇기 때문에 불안한 수험생들일수록 남들이 보지 않거나 검증되지 않은 \textbf{새로운 교재를 사려고 하지요}.
마침 그런 교재일수록 1등급을 보장한다거나 명문대 합격생의 그럴싸한 추천사(사실 글 내용은 별 게 없어요)가 달려있습니다.
자위행위를 하고 순간적인 쾌감에 행복해지듯 수험생들은 이런 새로운 교재를 구입하고 포장을 뜯을 때까지는 행복합니다.
다만 책페이지를 펼치면서 '역시 공부해야하는 건 마찬가지구나'하는 순간 \textbf{현자타임}이 도래하죠.
"결국 꿀공부는 없었어"라는 걸 깨닫죠. 그리고 공부해야하는 참고서가 늘어납니다. 부담감도 곱절이 됩니다.
\vspace{5mm}

더 심해진 불안감에 다시 또 새로운 교재 없나 기웃거리면서 '공부 안 하는 핑계'를 만들고
그렇게 교재 모으기 하다가 시간이 촉박해지면 "강사님만 믿숩니다 T_T"하면서 거금 들여 인강결제를 시작하죠.
그런데 그 인강도 듣다가 다시 회의감이 들죠. 그게 인강이 별 게 아니다라는 걸 알아서가 아니라, 이걸 듣고 공부하는 게 힘들어서입니다.
그래서 다시 기본교재로 돌아가 독학해야할까.... 이렇게 우왕좌왕하는 가운데 가을이 옵니다. 그리고 n+1이 달성되죠
\vspace{5mm}

이걸 읽으면서 본인 이야기라는 데 찔리는 사람이 한두분이 아닐 걸로 알고 있습니다.
이건 상상이 아니라 실제로 관찰, 상담에서 적지않게 확인된 사례이기 때문입니다.
\vspace{5mm}

그럼 비극이 어디서 시작된 걸까요?
그거야 \textbf{인내하지 못 해서였죠}.
\vspace{5mm}

저 친구는 그냥 교재 늘리지 말고 시중교재 하나만 가지고 여러번 돌렸으면 진작 그 n수의 고리를 끊어낼 수 있었습니다.
괴로운 것을 참아내면서도 참고서를 다 풀어내고 틀린 것 오답 체크하고 모르는 것 정리해서 질문하고 하다가
어느 순간에 일정한 경지에 이르는 순간 스트레스가 풀리면서 공부를 더 하고 싶다... 하면서 공부가 쾌감이 되는 순간,
즉 공부하기 싫어하는 자기 뇌를 굴복시키고, 불안감을 긴장감으로 바꾸는 경지에 이르렀을 것입니다.
\vspace{5mm}

그런데 그걸 공부의 고통을 못 참으니까 시간과 돈을 있는대로 날려먹는 것이죠.
학원 $-$ 실강을 그나마 권하는 이유? 적어도 거기선 공부가 강제되고 도피할 수는 없으니 말입니다.
비싼 돈을 지불했기 때문에 아까워서라도 다니게 되고 그래서 저런 충동구매는 적어도 안 합니다. 학습량이 최소한 쌓이는 건 있단 것이죠.
\vspace{5mm}

여기 와서 기웃거리는 분들은 일지를 보고 그냥 따라하세요.
꿀루트가 있다? 그런 쓰레기 같은 생각을 하니까 인생을 그렇게 살아오신 겁니다. 그딴 건 없습니다.
\vspace{5mm}

위의 수험서 소비 패턴.
전형적인 사치된장패턴이라고 할 수 있죠.
공부도 안 해놓고 EBS 강의를 깝니다, 그런데 사설강의 찬양하는 이유는 결국 "비싸서'라는 결론에 도달하죠.
물론 사설은 꽤 재밌게 즐겁게 합니다. 그러나 뭘로 가든 종착점은 마찬가지일터인데 말이지요.
공부에 지불하는 건 그런 수업료보다는, "시간"과 "인내"입니다.
\vspace{5mm}

혹자는 이런 질문을 하겠죠. 당신의 이야기는 알겟는데 그건 꽤 평범하지 않느냐.
\vspace{5mm}

거시적으로 볼까요. 그런 평범한 것조차 제대로 해내는 학생들도 소수입니다.
누구나 계획은 거창하게 세우지요. 그런데 그 계획을 50$\%$ 이상이라도 주어진 기간에 해내는 사람은 10명 중 1명 될까말까입니다.
여기서 등수가 갈리는 겁니다.
\vspace{5mm}

어떤 교재나 강의가 좋은지 또한 어떤 수험생이 잘하는가 하는 가쉽에 능통해서 계획을 그럴싸하게 세우면 뭐해요.
\textbf{실천을 안 하는데}
아무리 좋은 계획도 실천 안 하면 망상이죠.
그냥 공부하다보면 계획 세우기도 귀찮아집니다. 책 읽고 문제푸는 것으로도 정신없이 시간이 지나가거든요.
\vspace{5mm}

계획은 이러쿵저러쿵 세우는데 지금 풀어놓은 교재가 없다.... 그 순간 본인은 그냥 쓰레기인 것입니다.
계획 세울 때야 즐겁게 하죠. 그런데 정작 이런 사람들이 일주일을 넘기는 경우도 드뭅니다.
중간중간에 스트레스 받는다 하기싫다 이거 꼭 해야해 우리나라는 너무 공부에 미쳐있어... 이렇게 또 가기 시작하죠.
\vspace{5mm}






\section{과외 구할 때}
\href{https://www.kockoc.com/Apoc/641984}{2016.02.19}

\vspace{5mm}

적당히 알아서들 구하시겠지만 학생 입장에서 선생을 검증해보는 것은 필수죠.
\vspace{5mm}

\item \textbf{1. 기본 개념, 공식, 성질, 정리에 대해서 증명할 수 있는가 확인해보자.}
\vspace{5mm}

예컨대 수학이라면
dy/dx의 정확힌 정의, '미적분학의 기본정리'라거나 '통계의 각종 정리'에 대해서 왜 그런 것이 나왔는지 증명가능한지는 확인해봅시다.
저걸 모르고 문제를 푼다면 그런 과외는 받을 필요도 없습니다. 단지 문제푸는 기계일 뿐인데 그럼 EBS 강의로도 충분합니다.
그런데 상당수 문제 잘 푼다는 친구들이 저런 것도 설명 안 하고 '걍 하면 된다'라고 하는데
걍 하면 된다 수준이면 평범한 문제는 풀지 몰라도 새로운 문제나 통수 문제는 못 풉니다.
어려운 문제를 푸는 힘 자체가 가장 기본적인 개념을 사유해보고 하나하나 증명하는 데에서 나오는 것입니다.
적어도 그런 걸 해주지 않으면 아무 의미는 없습니다. 이런 걸 한 줄 안다는 건 적어도 그 선생이 꾸준히 공부한다는 이야기이죠.
\vspace{5mm}

\item \textbf{2. 어떤 유형의 학생이고 어디가 문제이며 과거 몇학년 때 공부가 안 되었는지 정확히 짚어야한다.}
\vspace{5mm}

과외가 학원과 다른 건 개인에 대해서 바로 문제가 뭔지 지적해주고 해결책을 제시해줄 수 있어야한다는 겁니다.
심각한 문제의 경우라면 해결하는 데 시간이 걸릴지 몰라도 방향은 바로 나와야 합니다. 거의 점술가 수준으로 나와야 합니다.
사실 공부가 망한 패턴은 한정되어있기 때문에 이걸 정확하게 짚는 건 어렵지 않습니다.
하다 못해 학생의 방 $-$ 특히 서재나 학부모 이야기 5분만 듣더라도 대충 과거 이력이 짐작이 가야합니다
물론 요새는 학부모나 학생들도 이런 건 다 검증합니다. 하다 못해 과외 처음 한다고 구라까고 하나하나 검증해나가죠.
\vspace{5mm}

\item \textbf{3. 특정과목 전문이라는 건 말도 안 되는 이야기다.}
\vspace{5mm}

국어와 영어는 못 하지만 수학은 잘 한다.... 개소리입니다. 수학만 잘해서 어떻게 명문대에 가나요?
무슨 의사 전문의도 아니고 수험과목에서 그런 건 없습니다. 공부 잘 하는 친구라면 국영수탐구 골고루 다 잘합니다요.
수학문제 풀이에 있어서 국어실력은 상당히 중요하고, 마찬가지로 국어 독해에 있어서 수학적 문제해결 능력이 필요해요.
탐구는 말할 것도 없습니다, 영어가 다소 동떨어지긴 했지만요.
그럼 석박이 잘 하는가 그것도 아닙니다. 수험은 주식이나 부동산 투자와 비슷합니다, 결국 누가 더 연구를 했느냐로 갈려요.
기출만 보아도 평가원이 어떤 의도인지 대략 짚을 수 있고, 수험생들 대부분이 어떤 경향에 치우쳐져있는지,
아울러 시중 교재, 강의가 어디서 부족한지 짚으면 그럼 답이 나옵니다.
수능에 올림피아드가 필요하나? 필요 없죠. 그러나 이과 수능에 수리논술이 필요한가?
기출 분석을 해보았다면 필요하다라고 말해야할 것입니다
특히 학생 한명만을 본다면 수학을 가르치면서도 이 친구가 국어에 약하다는 걸 눈치까고 국어 공부하라고 하거나
자투리 시간에 영어 질문도 바로 받아줄 수 있어야합니다.  그런데 그게 아니다, 별로 기대 안 하는 게 좋을지도 몰라요.
국어 영어는 못 하지만 수학은 잘 하니까 국영수 골고루 잘 하는 것보다 더 뛰어나다?
그럴싸해보이지만 사실 헛소리죠.  그런 논리면 국영수 골고루 잘 하는 친구가 수학만 했으면 더 잘 나왔을 것입니다.
\vspace{5mm}

\item \textbf{4. 무료 보충 가능해야.}
\vspace{5mm}

시간제 뿐만 아니라 진도에다가 난이도까지도 보증해줘야합니다.
제 시간에 못 끝내준다거나 진도가 밀린다고 하면 적정한 수준에서 보충해 줄 준비까지 되어있어야합니다.
적어도 책임지려는 사람은 보충을 해줍니다, 하지만 책임질 준비가 안 되어있다면 이건 생깔 것입니다
물론 교통비는 알아서 익스큐즈해야할 문제겠지요.
과외선생 중에서도 돈만 바라보는 사람이 있고 반면 자기 자존심이 강한 사람이 있습니다.
후자는 무료 보충을 기꺼이 해줄 것입니다. 물론 그만큼 학생이 실력이 올라가길 바라겠지만요.
\vspace{5mm}

\item \textbf{5. 당연히 본인이 교재 없이도 즉석에서 문풀, 설명 가능해야}
\vspace{5mm}

설명 불요라고 봅니다. 쎈C스텝이나 기출 킬러를 제외하고는 그냥 즉석에서 풀고 설명해줄 수 있어야 합니다.
그리고 쎈C스텝이나 기출 킬러라고 하더라도 답지를 그냥 읊는 게 아니라
그 문제가 교과서상 어떤 개념을 어떻게 변형시켰으며 출제자가 어떤 의도였는데 어디서 삑사리났는가도 얘기해줘야합니다.
\vspace{5mm}

\item \textbf{6. 상위권 전문은 피하시길 ★★★}
\vspace{5mm}

사실 이게 가장 중요합니다. 얼핏 보면 상위권 전문이 잘 할 것 같지만 실제 그렇지는 않습니다.
상위권 전문이라는 건 상위권 학생만 받겟다 내지 상위권 내용만 가르치겠다는 건데, 정작 이건 가르치는 입장에선 어렵지 않습니다.
상위권 학생은 알아서 공부하는 습관이 잡혀있거니와 스스로 깨우쳐 따라옵니다.
원래 좋은 대학에 갈 수 있는 게 탄탄하기 때문에 사실 냅둬도 알아서 갈 수 있을 가능성이 높아요.
그래서 이 경우 선생의 역할은 별 게 없는데 실적이 상당히 부풀려집니다.
\vspace{5mm}

진짜 실력자들은 최하위권을 최상위권으로 만드는 경우이겠으나 이 경우는 단 한번도 본 적이 없습니다(이거 로또급 아녀?)
보통은 하위권을 중위권으로 만들거나, 중위권을 상위권으로 만드는 경우가 그나마 현실적입니다. 이건 손이 꽤 많이 가는 작업입니다.
문제를 풀어주는 것보다 더 중요한 건 공부의욕을 북돋아주고 학생이 겪는 슬럼프나 고민을 해결하는 걸 넘어 미리 예견해 줄 수 있어야죠.
그런데 상위권들은 이런 애로사항이 별로 없어요. 그래서 과외 선생이 정작 하는 것도 없고 진화도 못 합니다.
하지만 하위권을 중위권으로 올렸거나, 중위권을 상위권으로 올린 선생들은 실력도 있어야하지만 대화능력이나  사람통찰도 잘 합니다.
안 그러면 사실 성적을 올릴 수 없으니까요. 그걸 알아서인지 모르나 학부모들은 '졸업생'을 요구하는 경우가 있습니다.
아무튼 문제 잘 풀어준다... 이것만 하자면 과외 할 필요 없이 그냥 인강만 가면 되는 겁니다.
\vspace{5mm}

\textbf{7. 가격}
\vspace{5mm}

가성비로 보아야겠는데 일단 가격과 수업 수준이 비례하는 건 아닙니다.
선생들도 보면 돈에 눈이 어두운 사람이 많고, 반면 가르치는 것 자체를 좋아하는 사람도 있죠.
어찌되었든 돈만 벌면 된다는 케이스도 있지만, 반면 먹고살기 위해 이 짓 하지만 자존심은 지키자는 케이스도 있어요.
\vspace{5mm}

비싸게 부른다고 잘 가르친다... 그런 경우는 없어요. 오히려 학부모들의 허영심을 알고 비싸게 부르는 케이스가 있겠습니다만
이 경우는 그냥 척 보아도 사기꾼이 아닌가요?  가격을 높이 받을 수도 있겠지만 그건 그만한 근거라는 게 필요합니다.
저도 이런저런 이야기 많이 들어왔습니다만 정말 학벌 별 것 없는데 '부르는 게 값이다'라는 케이스 적잖게 있더군요.
어떻게 보면 밑바닥이니까 그런 이야기 나온다라는 생각이 들덥니다.
정작 잘 가르치는 고수들은 사람을 별로 안 받으려 해요. 오래 할 일이 아니라는 것도 알고 있고 대개 \textbf{자기 공부하는 게 있습니다}.
그리고 보통 피곤한 일이 아닙니다. 적당히 시세대로 받으려고 합니다.
물론 그런 사람들을 찾는 것 자체가 꽤 어렵습니다만 어머니들이 잘 찾죠.
\vspace{5mm}

이런 점들을 감안해 알아서 잘 구하시기들 바랍니당.
물론 과외 구한다는 건 본인이 '병자' 상태라는 걸 느낄 때나 도움 되는 거고
그게 아니면 그만한 돈으로 차라리 집이나 차를 사거나 해외여행 가자 하고 EBS 강의 졸라 듣고 매질하는 게 더 도움될 겁니당.













\section{계획이 안 맞는 사람도 있음}
\href{https://www.kockoc.com/Apoc/645403}{2016.02.22}

\vspace{5mm}

계획을 세워야한다고 하지만, 사실 그런 계획이 안 맞는 사람들도 있다.
\vspace{5mm}

$-$ 즉흥성이 강하다
$-$ 성격이 격정적이거나 산만하다
$-$ 뭔가 얽매이는 걸 싫어한다.
\vspace{5mm}

하여간 이외에도 여러가지 특징들이 있겠지만, 결론적으로 계획을 짜도 실천도가 낮은 사람들이 있다.
이런 사람들은 계회을 아주 간소화시켜버리거나 차라지 짜지말고
\textbf{"과제"를 설정하고 " 3$\sim$5일 정도 데드라인" 잡은 뒤 그 기간동안 과제에만 몰두하는 방식}으로 가는 게 낫다
실천도가 낮은 사람에게는 계획이 실적으로 '둔갑'해버린다. 그래서 계획만 짰는데 그게 공부한 것처럼 느껴져버리면 문제다.
이런 사람은 차라리 계획을 포기하고, "과제" 하나를 잡은 다음 단기간 내에 스퍼트해서 그걸 끝내는 방식이 나을 수도 있는 것이다.
\vspace{5mm}

어제도 충고했지만 실천도가 낮은 사람은 전과목 골고루하기보다는
아무개 선생 인강 50강이라고 하면 그걸 20강 1부, 30강 2부라고 나눠서
1부는 4일동안 끝낸 뒤 하루 놀고, 2부는 5일동안 끝내기라고 가는 게 낫다. 물론 다른 과목은 건들지도 말고
그럼 열흘이 지나면 그 아무개 선생 인강은 다 청취해버린 것이기 때문에 아무튼 '성과'는 생기고, 이로써 선순환 루트를 탈 수 있다.
\vspace{5mm}

자신이 계획을 세웠을 때 일주일에 3일 이상 날라가버린다거나, 실제 이행률이 50$\%$ 미만이라면
계획을 포기하고 \textbf{단기과제 스퍼트형으로 가는 것}도 시도해볼만하다.
이거 문제가 많냐고 할지 모르지만 사실 그런 지적이 바로 아랫 글에서 말한 '대안없는 비판'의 전형이다.
한 과제만 잡아서 그것만 몰두하는 것도 문제가 없는 건 아니지만, 적어도 이건 실천과 성과는 담보해준다.
거창하지만 실제로 실행이 40$\%$ 미만인 것이야말로 최악인 것이다.
\vspace{5mm}

+ 그리고 이건 조심스럽게 말하면, 오히려 공부 잘 하는 애들은 계획형보다는 이런 단기스퍼트형이 더 많지 않나 생각되는 것도 있다.
실제로 계획을 본인이 꼼꼼히 세워서 그걸 실천에 옮긴 경우야말로 찾아보기는 어려웠기 때문에.
물론 다수는 학원이나 과외가 시키는대로 따라가므로 적극적 계획형이라기보다는 수동적 계획형이 많다라고 할 수도 있겠지만
벼락치기의 마수는 사실 피해갈 수가 없다.
\vspace{5mm}

++ 비유하자면 이런 것이다 10명을 모두 공략하면 1:10으로 싸워야 한다.
반면 한명한명 공략해서 승리해 아군을 만들어나가면 1:1, 2:1, 3:1, ... , 9:1로 싸울 수 있다.
하나씩 처리하는 경우의 장점이 그렇다. 일단 달성한 과제는 그 다음 과제 공략에 도움을 준다.
반면 한번에 여러개를 처리하는 경우 본인은 그 여러 과목과 상대하느라 조기에 탈진해버릴 수가 있다.
\vspace{5mm}

+++ 두달 잡고 간다... 사실 이건 힘들다. 두달 동안에 지쳐버릴 수도 있고 뭔일이 날지도 모르기 때문에
뭐든지 일단은 짧게는 이틀, 아주 길어도 9일 내로 끝내는 게 좋다. 즉, 공부할 대상을 잘 세분화시켜서 그렇게 하나하나 처리하란 말씀.
수학 전범위를 일주일동안 할 수는 없지만, 가령 "미분1"만 본다고 하면 일주일이면 가능하다.
두달 이상 가는 독종도 없지는 않지만 드문 편이다. 사실 이런 성격이면 그냥 알아서도 현역으로 좋은 대학 잘 간다.
그런데 안 그러니까 지금 고민이지 않겠나
\vspace{5mm}












\section{데드라인}
\href{https://www.kockoc.com/Apoc/650700}{2016.02.25}

\vspace{5mm}

이제 황금의 3개월은 끝났죠.
\vspace{5mm}

그 다음 분기는
3월$\sim$5월 까지의 100일
6월$\sim$8월 까지의 100일
9월$\sim$10월 까지의 60일
\vspace{5mm}

그나마 제대로 공부하는 기간이 3월$\sim$5월까지인데 솔직히 이 때도 안 할 사람은 안 합니다.
시동이 걸리는 대략 3월 중반 정도. 3평 보고 나서 일부는 3뽕.
그리고 4$\sim$5월에 좀 공부를 하고 6평을 치르죠.
\vspace{5mm}

그리고 6평 보고나서부터는 무조건 흔들립니다. 그리고 이 때부터 새로운 교재들도 무진장 늘어나죠.
기초 실력을 쌓는 건 거의 포기, 이게 시험 당일까지 발목을 잡습니다.
\vspace{5mm}

데드라인을 제시하면
\vspace{5mm}

\textbf{EBS 수특 $-$ 늦어도 4월말까지}
\textbf{기출 정리 $-$ 늦어도 4월말까지}
\vspace{5mm}

이렇습니다. 사실 이 정도는 해놓아야 안심합니다. 이걸 넘어서면 올해도 물건너갔다고 보셔도 되겠죠.
그럼 내년은? 역시 힘들 겁니다. 지금 고2들이 여간 잘 하는 게 아니라서(IMF 베이비라서 그런지 몰라도)
솔직히 지금 와서 학원 고르고 $\sim$ 한다라고 했을 때 가능성은 꽤 낮아지죠.
\vspace{5mm}

여전히 인기강사 상품 쇼핑하는 분들도 있는데 일단 신청했으면 제발 빨리 끝내세요.
수험이란 결국 남들보다 '많이', '신속히' 정리하는 놈이 이깁니다. 강사 믿고 끝까지 간다고 해도 확률은 20$\%$ 내외입니다
지금 와서 쎈 봐도 되냐 풍산자 봐도 되냐 그딴 헛소리하지말고(하려면 황금의 3개월동안 했어야지)
기출이라도 빨리 정리하세요. 이거 할 수 있을 거라고 다들 착각하실 건데 망상과 실천은 다릅니다. 실제로 해보면 경악하실 것입니다.
\vspace{5mm}

강의만 들으면 된다... 황금의 3개월동안 기본교재와 기출 푼 경우라면 먹히겠지만 그게 아니면 강의 들을 때는 행복하겠지만
문풀 들어가면 비명을 지를 겁니다. 지나치게 강의환상에 빠져서 게을리한 결과 개판오분전이라는 걸 꼭 본인 인생 걸고 확인하는 사람들이 있죠.
지금 막 시작하는 사람들은 4월말까지 기출 빨리 정리하시길 바랍니다.
\vspace{5mm}

솔직히 말해서 입시는 지금 시점에서 50$\%$는 이미 결정.
공부 힘들어 죽겠다 어쩐다하는 사람도 11월부터 2월까지 꾸준히 공부했으면 여름에 우위를 확인하실테고
지금부터 학원강의만 따라가도 되지라는 사람들은 그냥 '형벌'받는다고 생각하고 죽어라하시길.
나머지는 걍 답 없어요.
\vspace{5mm}










\section{수학 커리}
\href{https://www.kockoc.com/Apoc/651201}{2016.02.25}

\vspace{5mm}

\item 1. 풍산자
\item 2. 쎈 (복습용 RPM은 선택적으로)
\item 3. 마플(마더텅이나 자이로 가도 좋음)
\item 4. EBS 시리즈(수능특강부터 시작해서 N제, 수능완성까지 포함)
\item 5. 급품벨(일등급수학, 일품 수학, 블랙라벨) 중 택2
\item 6. 실력정석(바이블이나 수학의 원리 도 좋음) + 교과서(구할 수 있으면 교사용 지도서도 좋음)
7. 선택적 실모, 혹은 구할 수 있으면 과거 본고사 문제
\vspace{5mm}

이렇게만 해도 11월까지 다 풀 수 있을지도 고민인데 교재 모자라서 혹은 강의 안 들어서 망하는 일은 없음.
사실 쎈과 마플까지만 다 풀어도 문제를 풀고 오답정리하는 과정에서 개념의 전반적 틀과 논리는 다 갖춰집니다.
마플까지 가면 안정적 2등급은 무조건 나옴(왜냐면 마플까지 끝내는 인간도 생각보다 적음)
\vspace{5mm}

EBS 시리즈 넣는 건 간단함, 시중교재를 살짝 비틀거나 색다른 문제를 군데군데 박아넣습니다.
급품벨은 당연히 \textbf{풀어야}합니다. 다 풀 필요는 없고 2개만 골라서 풀어도 좋죠.
\vspace{5mm}

1, 2, 3을 대략 4월까지. 4번은 나오자마자 순삭.
그리고 5번은 8월까지
\vspace{5mm}

그렇게 한 다음 실력정석을 보면서 필요한 것만 발췌해 읽고 풀면 됩니다.
정석은 초보자에게는 맹독이지만, 고수에게는 만독불침의 명약입니다. 그러니까 5까지 다 하고 보면 됨
그 다음 교과서들을 주문해서 군데군데 잡다한 것들을 봅니다, "이렇게 놀라운 출제소스들이 있다니"에 다시 놀라게 됩니다.
마지막으로 선택적 실모 보시면 됩니당. 출제자들의 마인드나 성향이 읽혀질 것이고 해설 보면서 실력의 한계도 읽어낼 수 있을 겁니다.
\vspace{5mm}

그런데 이제 곧 3월이니 1, 2부터 시작하면 시간이 주욱 밀리죠.
지금 기출도 중간 정도는 풀고 있어야 합니다.
콕콕 내에서도 빨리 하라는 잔소리 듣고 하라는 대로 해서 실력이 늘어난 걸 절감한 케이스가 있고
반면 3월이 되었는데 이제 와서 저거 할까요 하는 사람들도 있습니다(...)
\vspace{5mm}

저대로만 다해도 안정적 1등급은 뜹니다. 저걸 해내기 힘들어서 그렇지.
저걸 다 하고도 모자라면 그 때부터는 일본 문제집을 수입해보거나 경문사에서 나온 수학교양서를 읽으시면 됩니당.
기타 이상한 야매교재들은 볼 필요가 없습니다(그런 것 보다간 기본적인 사고가 망가집니다)
\vspace{5mm}

비밀이랄 것도 없어서 이건 그냥 걍 공개합니다. 사실 다 아는 내용 아닌가 싶은데
무엇보다 고2 분들은 저거 빨리 밟으시길. 몇몇 상위권 표본들이 벌써 5까지 병행하는 케이스를 들었습니다(...)
님들은 수학을 빨리 정리하고 국어로 몰두하셔야할 것이니다. 2018년도 수능의 핵심은 '국어'인 것이 명백관화해졌습니다.
\vspace{5mm}

수학을 일단 저렇게 가야하는 이유는 '컷'이 높기 때문입니다.
컷이 높은 과목은 "실수'를 안 하는 게 핵심입니다. 심화의 필요성이 줄어들죠, 특히 학생들이 쓴 출처불명 야매교재 심화인 척 하는 건 자살행위.
그럼 컷이 낮은 과목은? 일부 탐구과목들이 있겠고 혹은 국어가 해당될 수도 있습니다(아마 수학 컷을 낮출 일은 없어보입니다)
이런 과목들은 '심화'해서 공부해야합니다.
\vspace{5mm}









\section{공부를 해야하는 이유}
\href{https://www.kockoc.com/Apoc/658157}{2016.03.01}

\vspace{5mm}

친구 : 겉으로 위로하는 척 하지만 뒤로는 킬킬킬 웃으며 행복해한다.
가족 : 역시 위로하는 것 같지만 사실 별로 도움은 주지 않으며 나중에 왜 xx는 하는데 너는 못 하냐 나온다.
\vspace{5mm}

끝까지 배신 안 하는 건 자기가 공부한 것임.
돈은 본인이 능력이 없으면 사정없이 떠남.
낭비도 무식한 놈이 하는 것임.
\vspace{5mm}

어차피 이런 얘기를 해도 다수는 그럴 리가 없어
당신은 세상을 너무 비관적으로 보는 게 아니냐 하지만
결국 저렇게 되어있음, 진실의 맛은 씁쓸함,
거꾸로 말해서 씁쓸한 맛이 나는 것은 믿어도 좋고
달콤한 맛이 나면 의심해보아야 함.
\vspace{5mm}

아마 다수는 부유하고 화려하게 살고 싶어서 공부할 것임, 그리고 그건 첫동기로서 나쁘지 않음.
그러나 중간부터는 사실 그것도 만만치 않구나 깨달아야하지만
공부가 어려울수록 진입자가 적기 때문에 \textbf{이 놈 덕분에 앞으로 생존할 수는 있겠구나...} 그런 마인드로 가야함.
잘못된 걸 공부한 게 아니라면 이 놈은 어지간해선 배신하지 않음.
물론 잘못된 것을 학습하면 (얼치기 운x권 사상이라거나 사이비 종교 교리) 그런 건 정말 배신하지 않고 확실히 인생을 조져줌.




\section{반도의 수학교재}
\href{https://www.kockoc.com/Apoc/659705}{2016.03.02}

\vspace{5mm}

개정되었다는 교재들을 보면 그냥 개정이 되었는지 개정하는 흉내를 냈는지 의심이 가는 게 많음.
소감은 딱 : \textbf{'이런 꿀장사가 따로 없겠구나'.}
조금만 교과서만 봐도 교정할 수 있는 오류가 10년째 그대로인 경우도 많음.
\vspace{5mm}

그래서 교과서가 낫다는 말이 나오는 것임.
왜냐? 교과서 저자진들은 계속 연구하는 분이거든
그나마 제대로 공부하고 연구하며 해외수학교재도 번역 공부하는 분들이 보이니까.
일부 교과서 저자 분들은 '번역자'도 하고 계심, 그럼 번역 과정에서 좋은 책 내용을 흡수해서 교과서에 반영하는 건 당연하지 않겠음?
\vspace{5mm}

반면 학생이나 학원강사가 잘 쓸 수 있을까.
솔직히 학생교재는 볼 필요는 별로 없다고 생각함. 그 학생이 아무리 잘 써보았자 자기가 배운 사교육 내용을 그냥 짜깁기한 것임.
그리고 학원강사들도 강의하느라 연구하고 정리할 시간은 \textbf{별로 없음}.
그렇기 때문에 오히려 사설 쪽의 내용은 풍부한 것처럼 보여도 그것 자체가 저기 섬나라 것을 베껴온 경우가 많으며
실제로 진보하지 못 했다라는 문제가 있음.
\vspace{5mm}

이래서 역설적으로 교과서만 진보한다는 이야기가 나오는 것임
교과서를 보라고 하는 결정적 이유가 이것임. 정확히 말하면 \textbf{교과서 저자나 공교육 교사가 쓴 책을 보는 게 낫다}고 말해야 함.
쎈이나 EBS 보라는 것도 포괄해 말하면 '제대로 뭘 알고 있을 가능성이 높은 분들이 쓴 것'을 보면 되는 거라고 얘기함.
\vspace{5mm}

그리고 교과서의 장점은 잡스킬이나 지엽이 아님. 바로 "읽는 방법"을 선사해준다는 것임.
교과서에서 풍부한 내용을 기대할 필요는 없음, 어디까지 교과서에는 기본, 기본의 기본, 기본의 기본의 기본 .... 이 나와있음.
그런데 최근 3년치 수능에서 특정 문제를 못 풀었다면 그건 스킬을 몰라서가 아니라, \textbf{그 문제를 읽는 법을 몰라서 그럼}
그럼 왜 읽는 법을 모를까. 그거야 기본도 안 되어있어서 그런 것임,
미적분이 어디서 출발했는지 확통의 기본 프레임이 뭔지 그게 안 되어있으니까 문제를 조금만 꼬아내도 못 푸는 것임.
평소에야 맹물은 맛이 없어서 탄산음료나 커피를 마시겠지만 힘든 운동을 한 뒤에도 과연?
\vspace{5mm}

신기한 것은
일본인들은 책 한권 쓰는데도 수년걸리고 정말 제가 봐도 잘 쓰는데도 그걸로 재벌타령하거나 그러지 않는데
우리나라 사람들은 머리가 정말 뛰어난지 책도 정말 쉽게 쓰고 그걸로 참 잘 벌어들이지 말입니다.
그런데 왜 모 아이돌 경쟁 프로그램이 일본 프로그램과 유사한 그런 광경이 떠오르는지
\vspace{5mm}

그리고 그런 천재(?) 저자에게 출처 물어보면 답을 한 경우가 없었음. 그래, 장사하셔야지.
\vspace{5mm}

+
뭐 하기야 학창시절에 취미가 괴수님들이 쓰신 책이 뭘 표절했나 확인하는 거였는데 이것도 참.
제가 확인한 사람만 무려 5분이었음. 일본책이고 미국책이고 아니 베끼려면 제대로 베끼지 그것도 원생들 시켜서 대충 번역(...)
그러신 분이 제자들에게는 똑바로 살라고 그러던데.
\vspace{5mm}

++
위에서 말했지만 최근 3년치 기출보면 스킬이 문제가 아니라 문제를 '해석'하고 해결의 '프레임'을 짜는 게 더 중요한 문제입니다.
이미 평가원 문제는 진보했어요. 그런데 아직도 거액을 지불하고 잡다한 스킬이나 이상한 문제 푸는 것에 집착하는 사람들이 많죠.
문제 "독해력"는 인강이나 잡교재가 간접적으로 해결해 줄 수는 있으나, critical path는 결국은 혼자 끙끙대며 철학자질하는 것입니다.
\vspace{5mm}

+++
일본 것 베끼면서 대한독립만세라고 해보았자 아무 소용없죠. 입으로는 한민족 만세하면서 일본 걸 참 잘도 표절한다는 게 거참.
가장 미련한 게 자기들이 배우거나 애용하던 게 뒤늦게 일본 것임을 깨닫고 이걸 '순화'시킨다고 억지로 명칭 바꾸거나 하는 것인데
그냥 일본식 표현쓰는 게 낫습니다. 우리가 쓰는 한자어가 대부분 일본산인 걸 알면 어휘 자체를 다 갈아엎을 것도 아니고.
\vspace{5mm}






\section{승부는 곧 결정납니다.}
\href{https://www.kockoc.com/Apoc/662562}{2016.03.04}

\vspace{5mm}

현실의 논리가 사람의 직관을 배신하는 경우가 많은데 그건 수험 스케줄도 마찬가지.
이제 6월까지 가면 그 때부터는 공부를 더 하기 힘들죠. 뭐 한다고 \textbf{말은 잘 하는데} 그런 사람은 20명 중의 1명 정도.
\vspace{5mm}

거의 다 9월까지는 끝낼 거야... 라고 하지만 9월 되어서도 못 끝내서 10월까지 잡았다가 멘붕해서 양줄이려고 하다가 말아먹고.
거기다가 6월$\sim$8월은 무더위가 오죠. 지구온난화 덕분에 5월 중순부터 그럴지도.
그래서 많은 학생들이 이 때에 넉다운당합니다.
\vspace{5mm}

이렇게 계산하면 실제로 승패는 지금부터 100일 이내, 즉 5월말이면 95$\%$는 결정난다고 보아도 무리는 없죠.
그럴 리가 없어, 네가 소설쓰는 거야... 라는 분도 계시겠는데 저야 한마디. 당신은 그렇게 생각했으니까 여태껏 그렇게 살아오신 것임.
\vspace{5mm}

지금 이 시기에도 인강 뭐 들을까 고민하면 사실 답이 없다고 보는데... 왜냐면 공부량도 양이지만 그냥 마인드나 습관이 실패하는 유형 그대로라서.
그냥 아무 생각없이 하나 잡으면 '끝내는 결과' 모드로 가야지, 자꾸만 시작해야지... 하면 영원히 시작만 하게 됩니당.
시작에만 능한 사람은 '시작'만 하면 다 되는 줄로 착각하죠. 시작이 반이다라는 말은 시작만 하면 영원히 '반' 밖에 못 한다는 이야기이도 한데.
\vspace{5mm}

황금의 3개월동안 어영부영하다가 지금 공부하는 사람은 이제 \textbf{하루 순공부시간 8$\sim$9시간 잡고} 5월까지만 공부한다... 라는 마인드로 가는 게 현실적일 듯
그것도 안 되는 사람들이야 그 때부터 또 이상한 잡다한 교재에다가 작년에도 별 쓸모없던 실모 같은 것 보겠다고 돈쓰고나 있겠고.
물론 11월부터 공부하신 분은 하루 순공부 6시간만 유지하시면 됩니다. 일주일 하루 쉬는 건 당연하고.
\vspace{5mm}

이번 한달도 공부 그렇게 못 하면. 그냥 내년 기약하는 게 나아요. 자신감 떨어뜨리고 협박하는 게 아니라 현실적인 이야기를 하는 겁니다.
왜냐면 3월달도 공부 안 한 사람이 그 이후에도 공부 제대로 할 리는 만무해서리. 이런 사람들은 걍 알바 뛰고 개고생하고 정신차리는 게 먼저임.
\vspace{5mm}






\section{친구나 가족 다루기}
\href{https://www.kockoc.com/Apoc/663083}{2016.03.04}

\vspace{5mm}

\item 1. 자기보다 잘 나가는 친구와는 의외로 오래 유지할 수 있습니다.
만약 여러분이 n수생이고 친구가 잘 나가는 대학생이거나 직장인(...)이면 자기가 폐를 끼친다고 생각할 수도 있겠습니다만.
실제로는 본인이 친구에게 \textbf{더 많은 행복}을 선사합니다.
다른 게 아니라 친구는 여러분을 보면서 '우월감'을 느끼기 때문입니다.
그 친구가 다른 잘 나가는 사람들과 같이 있을 때 못 느끼는 행복한 감정을 여러분을 통해서 느낄 수 있죠.
눈치없는 사람들이 이런 걸 가지고 "아니 제 친구들은 안 그러던데요"할지 모르겠지만 그거 참 순진한 이야기임.
\vspace{5mm}

\item 2. 부모님들은 실제로는 여러분이 잘 되는 것보다는
여러분을 통해 어떻게 자랑을 할 수 있을까, 아니면 얼마나 등골 덜 뽑힐까 하는 데 관심이
그러다고 뭐라 할 수는 없습니다. 부모님이 금전적 지원을 한다면 이 분들이 님들 인생의 오너죠.
그렇다고 님들이 배당을 한다거나 아니면 빚을 갚는 채권자 관계가 되는 것이 아니라면 사실 할 말은 없음.
명문대에 갈 필요없다, 씩씩하게 자라다오... 라는 걸 믿는 순진한 사람은 아무도 없습니다.
어차피 배당(...)은 불가능하고 오히려 '투자'해달라고 할 게 뻔하다면 체면이라도 세워주야하지 않겠습니까.
\vspace{5mm}

\item 3. 자기가 망하고 있을 때에는 부모님보다는 친구의 '악담'이 더 진실에 가깝습니다.
부모님은 주주이므로 내부인이지만 친구는 경쟁자이기도 하기 때문에 손님의 관점, 즉 객관적 입장에 서 있기 때문이죠.
자기 문제점은 자기를 싫어하는 사람이 훨씬 잘 알고 있습니다.
그렇다면 친구가 자기를 싫어하는 사람에 속하냐 하면 \textbf{그럴 수도 있고 그렇지 않을 수도 있습니다.}
\textbf{적당히 거리를 두고 있는 친구가 필요한 이유입니다.}
\vspace{5mm}

\item 4. 부모와의 관계 설정이 문제일 건데 주주의 의견은 존중만 하되 들을 필요가 없습니다.
다시 말해 \textbf{투자받은 만큼 성과를 내서 체면을 세워준다...} 빼고 여러분이 할 수 있는 건 아무 것도 없습니다.
흔히 말하는 진로나 직업 설정에 대해서는 부모님이 아주 잘 나가는 사람으로서 세상 돌아가는 걸 아는 분들이 아니라면 무시해도 됩니다.
예컨대 부모 말 들어서 진로 선택했는데 그게 망하는 길이었더라.... 그거 아무도 책임지지 않습니다.
반대로 님들이 부모의 반대 무릅쓰고 원하는 길로 가서 죽어라 노력해서 대박 터졌다, 그럼 부모님은 자기가 기여했다라고 생각합니다.
원래 인간사 돌아가는 게 다 그렇게 되어있습니다.
\vspace{5mm}

\item 5. 간혹 가다보면 부모님이 자기 의사를 존중해주고 .... 라는 식의 동화를 보는데 그딴 건 없습니다요.
의사존중이라는 건 존재할 수 없죠. 의사를 존중한 척 하면서 교묘히 설득하는 것이지.
자식 뜻대로 하게 해준다라는 건 그냥 '무관심' 아니면 '무관심을 가장한 관심' 양쪽인데
무관심은 사실상 포기입니다. 이래서 잘 된 케이스는 없어요.
반면 관심이라고 하는 건 부모가 자기가 희생해서라도 자식을 위해 투자하겠죠.
\vspace{5mm}

가장 좋은 건 자녀도 열심히 공부하고, 부모도 그렇게 투자해주는 경우입니다.
그럼 가장 안 좋은 건 둘 다 X냐 하겠지만 실제로는 그게 아니죠. 둘 다 X면 사실 서로 원망할 껀덕지가 없어서 그나마 낫습니다.
최악인 것은 자녀가 열심히 공부하지 않는데 부모가 투자한 경우죠.
\vspace{5mm}






\section{사설 인강 따라가고 있으면 그걸 그만둬야하느냐.}
\href{https://www.kockoc.com/Apoc/663118}{2016.03.04}

\vspace{5mm}

수험 끝났다고 훈수 두는 사람들이 이래라저래라가 먹히는 건 11$\sim$12월이고
지금은 하던 것만 끝까지 가야합니다.
\vspace{5mm}

가장 최악인 건 자기가 하던 걸 중단하고 \textbf{다른 게 좋지 않을까... 기웃거리는 것입니다}. 그럼 결국 어느 것도 하지 못 하기 때문입니다.
EBS갔다면 그냥 EBS 가시고, 사설 간다면 그냥 사설 가시면 됩니다. 그럼 둘 다 유의미한 차이는? 그다지 없습니다.
자기 통제 안 되어서 재종 간다면 재종 따라가는 게 답입니다. 다만, 재종에서 탑을 달려야합니다.
\vspace{5mm}

이 시기에 와서 저에게 쪽지보내서 강의 뭐 추천해달라 뭐하라하는 사람들 보면 이제는 짜증납니다만
그냥 자기가 끌리는 것 하면 그냥 그거 가시라는 것입니다. 그리고 이제 3월인 이상 제가 잘 충고해줘보았자 그건 별 소용이 없어요.
시작이 중요한 게 아니라, 자기가 하던 게 있으면 그걸 완료, 정리하고 계속 반복하라는 것입니다.
\vspace{5mm}

강의나 교재 같은 거 평론이야 저 같은 사람은 할 수 있죠. 왜냐면 과목 내용을 그나마 알고 있고
심각하게 학생이 해당 교재나 강의와 안 맞는다 하면(이건 노력해도 안 되는 정말 극악의 케이스입니다) 그 때야 충고할 수 있으니까요.
\textbf{그런데 대부분 안 맞는다는 건 뻥이고 그냥 공부하기 싫어서 핑계대는 것이죠.}
강의나 교재 때문이라고만 생각하지 자기가 쓰레기인 건 절대 인정하지 않으려고 하죠.
그냥 하는 것 꾸준히 하면 되는 거지 뭘 일일히 물어보고 하는지 모르겠네요.
\vspace{5mm}

뭐가 좋느냐 따지지 말고 하나라도 잡으면 꾸준히 끝까지 가시기 바랍니다.
이제는 뭘 따질 시점도 아니고 아울러 교재나 강의 추천해달라 그런 댓글 달면 상대가 아주 심각한 상황이거나 병자가 아닌 이상은 씹어버립니다.
자기들이 공부를 꾸준히 할 거라고 정말 진정 믿고 뭔 강의가 좋냐 따지는 것이면 착각도 이만저만이 아니죠.
\vspace{5mm}





\section{불행중독증}
\href{https://www.kockoc.com/Apoc/665182}{2016.03.06}

\vspace{5mm}

여기 콕콕에도 있고 어디든지 널렸습니다만.
\vspace{5mm}

자기가 불행하게 살았고 힘들다.... 라는 걸 자꾸만 강조해대면서 거기서 자존감을 얻으려는 케이스가 생각보다 많습니다만.
간단히 말해서 지금 행복한 사람도 언젠가는 불행해지고, 불행한 사람도 언젠가는 행복해집니다.
새옹지마라는 말에서 눈여겨볼 건 고대, 중세에도 인간살이는 마찬가지였을 거란 사실입니다.
\vspace{5mm}

우선 과거에 자기가 피해를 보았다... 라는 것에 대해서는 가족이나 친구나 지인을 원망하겠습니다만
과거에 집착하는 것은 그냥 현재 노력을 하기도 싫고 미래를 위해 준비하기 싫다는 "고급 핑계"에 불과합니다.
왜 고급 핑계냐고요? 본인 스스로도 이게 핑계인지 모르니가 고급 핑계지요.
\vspace{5mm}

과거에서 얻을 건 교훈 밖에 없습니다. 그런데 이 교훈은 결국 "자기 반성"으로 이어집니다.
부모가 내 인생 말아먹었다? $-$ 그럼 왜 그 부모 말을 그냥 듣거나 거부하지 않았을까 하는 자기 반성으로 결국 이어집니다.
아무리 부모라고 하더라도 기본은 남이고,
분명히 내가 그렇게 따라가지 않을 수 있거나 다른 길을 갈 수 있었다고 생각하면 내 문제가 되지요.
\vspace{5mm}

무엇보다도 잘 찾아보면 자기보다 불행한 사람은 훨씬 많습니다.
딴 이야기지만 입으로만 서민타령하면서 못 살겠다 하는 거... 저는 그건 일단 의심해요
정말로 먹고살기 힘든 사람들은 하루하루 입에 풀칠도 하느라 일하느라 정신없고 그렇게 목소리 낼 여유조차 없어요.
\vspace{5mm}

마찬가지로 정말 힘든 사람들은 두가지입니다.
아예 맛이 가서 말을 하기도 힘들어서 통각신경이 마비된 상태로 하루하루 살아간다거나
아니면 그걸 극복하려고 부지런히 일해서 시간이 모자라죠.
\vspace{5mm}

그럼 너는 왜 이렇게 단정적으로 글을 쓰냐하는데 제가 이런 쓰레기짓을 해보았기 때문에,
그리고 온오프에서 이런 걸로 적발한 사례가 많아서 압니다.
확신적으로 말하는데 '과거타령'하면서 남과 비교질하는 건
오히려 여유가 넘친다는 이야기라고 해도 틀린 이야기는 절대 아닙니다.
이런 경우 실패한 건 '타인' 때문이기도 하지만 결국은 '나 자신'의 스타일 문제입니다.
\vspace{5mm}

이런 케이스는 한두번은 상담을 들어줄 수 잇지만 그것도 자꾸 하다보면
'당사자는 계속 그것에 중독되어서' 또 쓰레기짓을 하죠.
과감히 혼낼 때는 혼내고 지적할 때는 지적해야 합니다.
이런 사람들은 기본적으로 "과거타령"하는 것에서 뇌가 쾌감을 얻고 있으니까 문제입니다.
아니, 이렇게 스트레스 푸는 것도 나쁘지 않냐....
불행한 것으로 쾌감을 느끼는 뇌라면 그 주인의 인생을 어떤 방향으로 이끌지야 명백하지 않습니까?
\vspace{5mm}

상상을 초월하는 n수생들 많아요. 요즘은 실패를 하면 기본 5년은 넘어갑니다.
그런데 이런 사람들 분석을 해보면 의식주 고민은 오히려 덜 하고, 인간의 기본 욕망은 다 채우고 있습니다.
생각보다는 공부할 여건이 다 갖춰져 있는데 계속 실패를 반복하려고 하죠.
왜 그럴까... 라고 하며 다방면으로 해답을 찾았는데
지금 내린 결론은 당사자의 뇌가 그렇게 무의식적으로 이끌고 있다는 것입니다.
\vspace{5mm}

마약중독에 빠진 사람, 즉 뇌가 마약으로 쾌감을 얻는 사람은 계속 마약을 할 수 밖에 없죠
이게 게임이든 알콜이든 성적인 것이든 마찬가지입니다.
심지어 남자에게 학대를 당하는 여자들조차도 그런 학대상황에 안 벗어나는 경우도 일부분은 이렇게 설명할 수 있을 것입니다.
마찬가지로 수년동안 진전이 없는 사람들을 보면 "과거에 xx 때문에 망했다"라는 걸 강조한다는 공통점이 있었습니다.
그리고 계속 그런 이야기를 수차례 반복해요. 본인도 괴로운 이야기일 건데 왜 그럴까.
간단하죠, 고통과 쾌감은 통하기 때문입니다.
\vspace{5mm}

이 글을 읽는 사람 중에 상당수 $-$ 특히 ㅇㅂ에서 오는 사람들은 $-$ 는 여기 해당할 거라고 확신하고 있습니다.
단언코 말하지만 이건 '치료'의 대상이지 위로의 대상이 절대 아닙니다. 말이 좋아 과거지, 그거 그냥 자기 자신도 모르는 고급 핑계죠.
비교해서 말하면 성공해나가는 사람들이나 부자들이 과거 타령은 안 합니다. 그 사람들은 늘 미래를 바라보고 있죠.
과거에 이래서 망했는데 ... 할 시간이 있으면 미래에 어떻게 하겠다라고 생각해보고 구체적인 준비를 하는 편이 낫습니다.
\vspace{5mm}

개인적인 얘기지만 정작 제가 타인에게 과거 불행이나 고민을 털어놓은 적은 별로 없습니다.
왜냐면 살아오면서 정말 두자리 수 되는 사람들에게 저런 식의 불행중독증에 걸린 두자리식 '인생한탄'
"나도 한 때에는 잘 나갔는데 $-$", "이게 다 ○○ 때문이다"... 라는 식의 이야기는 지겹게 들었기 때문입니다.
그리고 그 사람들은 절대 그거 한번만 이야기하지 않습니다. 수십번을 이야기하고 그리고 수년동안 안 변합니다(...)
그냥 정신차렷 소리 지르면서 공부하기도 싫고 일하기도 싫으니까 한탄하는 거지 윽박지르는 게 명쾌한 해결입니다.
\vspace{5mm}










\section{미래준비자}
\href{https://www.kockoc.com/Apoc/666823}{2016.03.07}

\vspace{5mm}

가난한 집안, 막장 가정 등의 문제는 그런 부모의 가치관이 자녀에게 그대로 전파될 가능성이 높다는 것이죠.
아래 불행중독증에 적었지만 그런 불행에 중독된 사람은 계속 불행해지려고 하는 성향이 있습니다.
하지만 미래준비자들은 실패조차도 미래의 성공을 향한 자산으로 삼으려고 하지요.
\vspace{5mm}

아울러 불행중독자들은 자꾸만 타인과 비교하면서 나는 왜 이렇게 못 났을까, 저 녀석은 그래도 ○○가 작을 거야라고 위안하지만
미래준비자들은 뛰어난 사람이 있으면 어떻게 그 재능을 배울 수 있을까, 아니 그 사람을 부하로 삼거나 동지로 삼을 수 있을까 생각합니다.
\vspace{5mm}

간단히 생각해봅시다.
불행에 중독된 사람이 지도자가 되면 그 사람 혼자 망하는 일로 끝나지 않습니다.
과거에만 집착하지 미래를 준비하지 않는 사람은 자기 책임이란 걸 지지 않고 어떻게든 타인에게 전가시키려고 하죠.
그리고 이건 실제로도 20세기 공산권 지도자들(이 사람들은 그 쪽 분야 대가였습니다)이 정말 잘 보여주었죠.
\vspace{5mm}

그리고 지도자들은 직접 일하는 게 아닙니다. 타이밍 잘 맞춰 적재적소에 인재들을 배치할 줄 알고 갈등도 조절할 수 있습니다.
인재들이 자기 전문분야를 공부하는 데 그친다면 지도자들은 사람을 공부하고 부릴 줄 알며
동시에 어떤 미래가 전개될지 생각해보고 가능성있는 시나리오를 짜서 그 방향으로 리스크를 감수하고 움직입니다.
\vspace{5mm}

불행에 중독되어있지 않고 항상 미래를 준비하니까 실패할 가능섣도 낮지만
실패했을 때도 손절매를 빨리 하며 그리고 그 손실을 넘어서는 교훈을 얻어냅니다.
\vspace{5mm}

....
\vspace{5mm}

이 정도는 책에 다 있는 내용이기도 하지 않을까 싶은데, 문자로 쓰여있는 것과 실천하는 것은 다른 것 같습니다.
저 역시 실천을 못 하는 건 마찬가지가 아닌가 싶은데
아무튼 상담하다보면 수험상담이 야매성 정신상담으로 이어지는 경우가 많고
그래서 귀납적으로 발견하는 건 성공하는 사람과 실패하는 사람의 \textbf{마인드가 정말 다르단 겁}니다.
\vspace{5mm}

입시에 성공한 사람들은 쿨한 성향이 있습니다. 구체적으로 말하면 과거사에 집착하지 않으며 실패를 바로 인정해 손실을 최소화한다는 것이죠.
거기다가 열심히 공부한다... 와는 다릅니다. 그것보다는 자기 인생을 열심히 경영한다... 가 더 맞는 이야기이겠죠.
대화하면서 느끼는 일종의 공감각적 이미지는 "사로잡혀있지 않다"라는 것입니다.
반면 N수생들은 대화하면서 이입을 하려고 하면 뭔가에 사로잡혀 있고 거기에 집착하고 있다는 걸 느낄 때가 많습니다.
저주에 걸린 건데 그 상태에서만 노력해보았자 저주의 약효가 커지지 않을까, 그런 생각도 들기도 합니다.
\vspace{5mm}

자기 과거가 어두우니까 집착을 하면 그만큼 보상도 클 것이다라는 미신이 작용하는게 아닌가도 싶은데 말 그대로 미신입니다.
과거가 힘드니까 앞으로 잘 될 것이다... 라는 건 어디까지나 서사문학의 틀일 뿐이고
실제로는 그런 힘든 과거에서 벗어나는 것, 즉 \textbf{역경을 극복해야 잘 되는 것이지} 역경에 집착하면 평생 노예가 되어버립니다.
\vspace{5mm}









\section{[수험교양 001] 장우석, "수학멘토" & "수학 철학에 미치다"}
\href{https://www.kockoc.com/Apoc/668096}{2016.03.08}

\vspace{5mm}

읽으면서 수학적 사고를 키울 수 있는 몇 없는 책 중 하나.
사실 이 시점에 수험생이 읽는 건 좀 늦지 않을까 싶지만 수학에 관심이 있다면 쉬는 동안 읽으면서 사고를 체화시키는 것도 좋다.
우선 이 책부터 소개하는 건 간단, 괜찮은 수학책들은 주로 일본, 미국, 독일 책이다.
(여담이지만 좋은 책을 고를 때에는 나중에는 저자고 출판사도 다 제끼고 국적부터 확인하는 습관부터 들게된다).
그런데 위 책들은 우리나라 사람이 쓴 것 치고 대단히 철학적이거니와 핵심적인 것만 건드리고 있어서이다.
\vspace{5mm}

내용 자체는 수험과 거리가 있을 수도 있다. 그러나 그런 건 수험서에 맡겨야 하는 일이고.
진정 수학적으로 사유하는 것이 무엇인가에서는 개인적으로 위 책들의 영향을 적지 않게 받았다고 생각한다.
수학멘토는 저자가 나름 수학과 철학을 접목시키면서 공부한 노트이고
수학 철학에 미치다는 것은 유명 수학자들의 연대기를 따라가면서 어떻게 수학이 발전했는가를 고딩이 이해할 수 있게 정리.
이 2권을 둘 다 읽고 고민하다보면 아주 간략하게 '근대 수학'의 가치관... 이라는 것을 형성할 수 있을 것이다.
\vspace{5mm}

수험 수학을 공부하다보면 처음에는 문제가 풀리지만, 고수 단계에서는 문제를 '풀어야' 한다는 것을 알게 되고,
나중에는 문제를 풀고 안 풀고가 중요한 게 아니라 그 문제가 일각에 불과한 '눈에 보이지 않는 빙산'을 읽어야 한다는 것을 알게된다.
그럼 그 빙산은 어디서 떨어져 나왔을까라는 데까지 이르게 되면 거대한 남극대륙을 떠올리게 되면서
과거 수학자들은 어떻게 사유했을까에 대해서 궁금해지게 된다(이 지점부터가 바로 29번, 30번의 시작이라고 할지도 모른다)
물론 위 책들을 읽는다고 풀 수 있는 건 아니지만 수험생에 있어서는 초고수의 영역에 해당하는 분야에 대한 개관서로는 적합하다고 할 수 있다.
인강이 맞지 않는 사람들은 이런 책들에서 의외의 깨달음을 얻을 수도 있지 않나 싶어 소개한다.
\vspace{5mm}







\section{입시 분석이 무의미할 수도 있는 이유}
\href{https://www.kockoc.com/Apoc/673033}{2016.03.11}

\vspace{5mm}

\item 1. 서울대에 들어가는 사람들은 그런데 큰 관심도 없음.
특목고, 자사고, 일반고 최상위권들은 어차피 끼리끼리 다 알고들 있어서 자기들끼라 우열관계로 분석들 하고 있음.
저런 걸 좋아하는 학생들은 자기가 공부를 잘 한다고 착각하지만 실제로는 그저 그런 케이스이고
가장 좋아하는 게 부모들인데 알고보면 공부가 뭔지 잘 모르거나 이며 현역이 아니며 저런 이야기에서 심리적 위안을 얻는 부류
\vspace{5mm}

\item 2. 속칭 스나이핑해서 들어간다고 하지만 이것도 정말 효용이 있는지는 의문.
대학에 들어가는 이유를 두가지로 분류하면 하나는 실속이고 다른 하나는 체면치레인데
실속있는 곳은 스나이핑한다고 해도 어차피 못 들어가고, "나 대학갔다"하는 식으로 체면치레하는 곳이나 가끔 빠지는데
이런 데 들어가면 그 때야 좋고 지금 3$\sim$5월 봄날 캠퍼스 생활에 싱글벙글이지만 가을이 오면 현실을 체감하죠.
\vspace{5mm}

봄날의 햇살에 꾸벅꾸벅 졸다가 머리가 빠지지 않았던 과거의 꿈을 꾸었는데
정작 그 때에 저런 등급컷을 보았느냐... 하면 그런 것도 존재하지 않았지만 공부 정말 잘 하는 애들이 저런 걸 신경쓰던가 하는 생각이 불현듯.
냉수 마시고 생각해봄에 저런 걸 보는 친구들은 결국 공부량이 부족하거나 알고리즘이 잘 잡히지 않아서 고민하는 친구들 아닌가.
\vspace{5mm}

최상위권이 되지 않으면 이제는 정말 무의미할 수 있는데, 정작 최상위권이 저런 걸 신경쓰는 건 '초중딩' 때까지라는 게 함정.
왜냐면 정말 전국에서 놀겠다고 하는 친구들이야 선행으로 끝낼 걸 다 끝내든가, 아니면 어린 시절에 영재교육 비슷하게 받아서
선행을 안 하더라도 한번만 듣고 열을 깨우치는 괴물들인데 그들에게 저런 분석이 의미있는가 하는 생각이.
게다가 저런 분석은 어디까지 점수가 나오고 난 다음에야 검토하는 거지,
저런 분석 자체가 \textbf{점수 자체를 올려주지도 못 하는데} 필요한 것이긴 함?
점수 자체를 올리려면 본인들이 죽어라 공부하든가 그게 힘들면 스파르타 시스템 들어가서 본인을 개조하든가 해야지.
\vspace{5mm}

참 쓸데없는 짓들 하는 거죠.
\vspace{5mm}

상담하다가 저에게 질책 들으신 분도 있죠. 가령 실력정석 한권만 보면 된다... 정석, 좋은 책이죠.
그런데 공부 잘 하는 애들이 교재를 줄이려 하는 케이스는 제가 아는 한 단 한번도 없음.
정말 한권만 본다고 하는 경우라고 해도 수십번을 봐서 마스터하는 케이스인데 개인적으로는 저는 이건 비추.
왜냐면 여러권을 보면서 그런 복습효과도 생기지만, 무엇보다 수학교재는 각각 빠진 게 조금씩 있어서 이걸 보충해나가야함
그래서 "양을 줄인다"거나 "특정 강의만 들으면 된다" 이딴 건 없는데 참 신기하게도 이런 게들 있다고 착각하고 물어보고나 있음.
수학교재를 8권 풀어도 내신 1등급 장담 못 하고 특정 강의만 듣고도 인생 조진 애들이 많은데 그런 판타지를 쓴다는 것 자체가.
\vspace{5mm}

그냥 고3 중위권 수준이 쎈 열심히 풀면 격려는 해줍니다. 어차피 이런 친구는 상위권 끝자락이라도 들어가면 희망이 있는 거니까.
그러나 고2가 고3들 푸는 킬러까지 죄다 풀고 풍산자 쎈 마플 급품벨 실력정석 연습문제 껌으로 풀어도 그건 '모자라다'고 얘기합니다.
그거 푸는 친구들이라면 최상위권을 노릴 건데 전국구 최상위권은 경시대회 문제는 기본이고 상상을 초월하는 수준으로 다 끝내놓은데다가
두뇌 수준도 엄청 비상한 것이 아니기 때문에 저것만으로도 부족하죠.
\vspace{5mm}

그런데 이런 후자들에게 입시분석이 의미있을지?
\vspace{5mm}

간략히 말해 다들 참 쓸데없는 밈에 사로잡혀들 있죠.
\vspace{5mm}

기본적으로 저는 평범한 학생들을 위한 과목 테크트리 정리에 관심이 있지만
정말 최상위권이라면 그 때는 인격이 바뀌는데 $-$ 솔직히 지금 수험사이트에 올라온 것들이나
거기서 명사라고 하는 사람들이 정말 최상위권?
\vspace{5mm}

그건 아니라고 생각합니다.
정말 최상위권인인 친구들은 몇마디만 던져보아도 반응이 일반인들과 다릅니다.
말에 가시가 돋혀있다라는 건 기본인데다가 그들의 말 한마디한마디는 상대방의 뇌세포 집적도를 테스트해보려는 게 깔려있죠.
그러니 어느 교재 봐야하나요 무슨 인강이 좋아요하는 질문은 걍 단세포이죠.
최상위권이라면 안 끝내놓은 게 없는데 그런 질문 따위 던지겠습니까.
오히려 그런 친구들이 눈으로만 싹 훑고 문제 푼 걸 가지고 어떻게 풀었냐 제가 질문하면서 논리적 흠결이 없나 확인해보아야하는 구만.
\vspace{5mm}

여담으로 머리좋다는 친구들이 싹 훑고 문제 푸는 건 "초고화질 이미지"로 푸는 것이고
뭘 모르는 사람들이 유전타령하지, 사실 이건 다년간 사교육받고 훈련하면 가능합니다. 이렇게 하는 사람이 적어서 그렇지
(다시 말해서 여기서 유전타령하는 사람들을 전 ㅄ으로 봅니다. 정말 아는 게 없구나 싶어서)
이런 친구들에게 논리 문제 내면 일반인들보다도 더 버벅거리죠. 왜냐면 이미지 접근은 논리에는 안 먹히거든.
\vspace{5mm}






\section{3평 가형문제 평}
\href{https://www.kockoc.com/Apoc/673299}{2016.03.11}

\vspace{5mm}

1 $-$ 새로운 출제경향이 안 보임, 기출의 반복 $-$ 그냥 교육청은 이에 대해선 방관하거나 손놓은 듯.
2 $-$ 눈여겨볼 것이 "함수" 강화야 작년 수능 출제 경향이기도 한데, 이거 시중 사설이나 그걸로 대비되는 건 절대 아닐 건데?
3 $-$ 돌아보면서 점수대들 보니 올해 고3에 대한 개인적 평가는 그리 틀리지는 않은 것 같습니다.
\vspace{5mm}

교재가 문제가 아니라 학생이 문제야... 라고 하고싶은데 내년에는 그런 이야기는 안 나올 삘이고
(정확히 말하면 현재 고3들은 공부 안 하는 걸 고2들은 공부하고 있다라는 딜레마가)
올해판 마플이나 급품벨 잘 풀었으면 무조건 96 이상은 나왔을 문제들인데 점수 보고를 보면 정말 어지간히들 공부 안 한 듯.
물론 n수생이 점수가 안 나오면 더 심각하겠죠.
\vspace{5mm}

상세문항 분석 올릴까 보았는데 전혀 그럴 가치가 없다고 보여서 걍 넘어갑니다.
문항분석보다도 공부 안 하는 학생들 멘탈 분석을 하고 싶다고 하면 독설치고 지나칠지 모르겠지만.
\vspace{5mm}









\section{실력정석은 막판에 봐야 진가를 발휘}
\href{https://www.kockoc.com/Apoc/675627}{2016.03.13}

\vspace{5mm}

실력정석 1권으로 뭉개고 보자... 라는 케이스가 있는데
다시 말하지만 이 책은 국내 수학책에서는 탑일지 몰라도, 렙이 안 되는 사람이 손잡았다간 기빨려 미라되기 좋은 책이다.
\vspace{5mm}

개념 $-$ 예제 $-$ 유제 $-$ 연습문제
\vspace{5mm}

다소 해설이 구린 것만 빼고보면 문제 하나하나가 일독할 가치가 있다.
특히나 이제는 기출만 무작정 짜깁기하거나 변형하면서 '수학 최고'를 자부하는 사이비가 널리는 시대에 실력정석의 가치는 상대적으로 높아진다.
\vspace{5mm}

날 보고 정석혐오자라고 착각할지 모르는데
애당초 수학교과서를 실력정석으로, 영어교과서를 성문종합영어로 배웠는데 무슨(...)
그런데 그렇기 때문에 정석의 단점이 뭔지도 안다. 이 책은 수학이 어느 정도 반열에 들지 않으면 정말 수학을 극혐하게 만들기 좋다.
\vspace{5mm}

반면 교과서적인 것도 거의 다 마스터하고 시중 문제집에다가 기출까지 다 보고 나서 보면
정석 문제 하나하나가 거의 몇쪽에 달하는 설명을 할 수 있는 떡밥투성이라는 것을 알 수 있다.
그 이야기는 공부가 되어있지 않은 상태에서 정석을 봐보았자 별로 효용이 없다는 것이다.
거기다가 정석이 직접적으로 수능출제 경향에 도움이 된다... 그건 아니다.
정석을 통해서 '수학적 사고력에다가 계산력'을 극강의 엽기적인 수준으로 늘릴 수는 있는 것이다.
\vspace{5mm}

가만 보면 하수일수록 정석부터 잡는 경향이 있다. 그거야 한권으로 끝내겠다는 망상 때문(...)
\vspace{5mm}

실제로 일본의 차트식 수학은 오히려 난이도 안배를 잘 하고 해설을 풍부히 달았다. 이에 가까운 책은 오히려 해법 셀파이다.
교재가 어디서 기원했으며 또한 어떻게 변천했으며 그 원산지에서는 어떤 트렌드인지 모르니까 정석 한권만 보면 되는구나라고 착각하는 것이다.
특히 그 일본 트렌드를 잘 반영한 것이 역설적이지만 교과서이다.
교과서란 형식에 얽매이기 때문에 교과서가 좋은 것을 느끼면서도 그걸 말로 설명 못 하는 건데
당연하지, 그 중에서 일본 책까지 수입해 보면서 평가하는 사람이 몇이나 있겠냐.
\vspace{5mm}

지금 급부상하는 수학교재는 마플 출판사 정도. 기출도 원숙해졌지만 교과서와 시너지도 참신한 시도를 많이 하고 있다.
\vspace{5mm}

아무튼 정석은 맨 나중에 봐도 되는 책이다.
시중교재 검증된 것만 풀고 교과서 마스터하고 기출도 여러번 돌리고 정석을 보면
어떤 걸 패스해도 되는지, 그리고 어떤 문제가 심오한 의미가 있는지 스스로 알 수 있다.
그럼 보는 시간이 단축되는 것인데 왜 처음부터 실력도 안 되면서 정석을 보는 고집을 피우는지 알 수가 없다.
\vspace{5mm}









\section{내신따기 힘들어진 건 사실이죠.}
\href{https://www.kockoc.com/Apoc/675938}{2016.03.14}

\vspace{5mm}

원래대로라면 특목자사고 갔을 애들이 \textbf{내신 노리고 그냥 일반고에 잔류하는 케이스가 증가}했는지라.
그래서 원래대로라면 내신이 정말 잘 나왔어야 하는 친구들이 이상하게 나오지 않아서 왜 그런가 싶었는데.
\vspace{5mm}

특목자사고 간다고 해도 탑을 달리지 않으면 깔아줘야하기 때문에 노력은 노력대로 하고 스트레스만 줄창 받기 딱 좋고
그렇다고 일반고에서 수능에 관한 노하우 같은 것이 부족하지는 않기 때문 $-$ 어차피 다 학원에서 배운다는 분위기이고
\vspace{5mm}

그래서 과거와 같이 굳이 좋은 학교에 가야한다라는 분위기는 가라앉은 것 같습니다.
이 역시 시장의 자정작용(?)이라고 볼 수 있을지도 모르지요.  대신에 일반고 양민들이 내신 따기는 어려워졌다는 부작용이(...)
\vspace{5mm}

다만 교육당국에서 뻘짓을 한 것이 바로 중딩 자유학기제인데 당연히 그 기간에 선행을 하지 뭘 하겠습니까(...)
교육시장의 움직임이라는 걸 모르지 않으면서도 정부는 룰을 개정하는 뻘짓을 하죠.
\vspace{5mm}

아무튼 그래서 고2 올라가는 사람은 결단을 내려야합니다.
가장 좋은 시나리오는 내신도 좋고 수능 공부도 되는 경우입니다. 하지만 몇이나 이걸 걸머쥐려나
그 다음으로는 내신이라도 잘 따놓는 경우입니다. 그런데 이게 수능공부보다 더 힘듭니다.
차악은 내신은 망하더라도 수능 공부라도 끝내놓는 경우입니다. 정시는 매우 힘들다 쳐도 수능으로 만회할 수 없는 건 아니니까.
\vspace{5mm}

하지만 대부분은 내신도 망하고 수능도 망하죠(...)
그래서 1학기 여름방학 때 결단을 참 잘 해줘야합니다.
\vspace{5mm}






\section{올해도 똑같네}
\href{https://www.kockoc.com/Apoc/677337}{2016.03.15}

\vspace{5mm}

하라는 걸 11월에 안 하고 계속 딴짓하다가 3월이 되어서야 이제 어쩌면 좋냐 하는 패턴은 매년 다를 바 없는 듯.
그리고 그 상태로 6월이 된다. 그냥 올해 시험은 물건너간 겁니당.
\vspace{5mm}

마플을 지금부터 풀면 되냐고 하는데
저는 \textbf{고2들에게 풀라고} 합니다. 그래도 \textbf{늦기 때문}입니다요.
얘들 경쟁상대는 그냥 커리 다 밟은 애들이 아닙니다.
양민들이 10줄 풀이써야 푸는 걸 눈으로 정확히 계산해 푸는 괴물들이 진짜 라이벌이라는게 문제죠.
속칭 머리 좋은(정확히 말해 머리가 잘 만들어진) 애들을 상대하려면 빨리 가는 수 밖에 없습니다.
그걸 알기 때문에 콕콕에서도 11월부터 빨리 시작하라 한 것인데
\vspace{5mm}

일지를 보니 그나마 저걸 지킨 사람들은 허덕대면서 자기 목표달성을 30$\%$ 가능성 정도는 할지 모르겠지만
나머지는 글쎄, 모르겠습니다.  아무리 꿀교재, 꿀강의라고 할지라도 여러번 반복하는 것은 못 따라가며
이들에게 필요한 건 '시간'일 터인데 말입니다요.
\vspace{5mm}

아마 4$\sim$5월에 허황된 광고하면서 교재 내는 사람들이 있을 겁니다.
말하지만 본인이 초고수가 아닌 한 그걸 잡는 건 스케줄 망가뜨리며 망하기 딱 좋은 짓입니다(교재검증이야 차치하고서라도)
수학교재는 자기가 다 풀고 오답정리하고 중요한 걸 떠올릴 수 있지 않는 한 그냥 '악성재고', '비계덩어리'입니다.
\vspace{5mm}

초짜이거나 잘 모르는 사람일수록 무슨 꿀교재나 꿀강의 의존하는데 정작 그거 결과 물어보면 '기대보다 별로'라는 데 실망하고,
왜 그딴 걸 믿어서 양치기를 안 했을까라고 후회하는 경우가 태반입니다(그 반대는 사실 한번도 본 적이 없네요)
\vspace{5mm}

이렇게 보면 양민들의 문제는 머리가 안 좋다 그게 아니죠.
스케줄 자체를 잘 못 짭니다. 다시 말해 본인이 수험기업이라고 하면 \textbf{기업경영을 못 하는 셈}이죠.
냉정하게 상황판단을 하고 플랜을 잘 짠 다음에 어떤 과정을 밟은 것인가 잡고 시행해야하는데
남이 이거 좋다더라... 하면 거기에 휘둘리면서 낭비를 합니다.
\vspace{5mm}

이제 들어올 수 있을지 모르나 콕챗에서 콕콕수험고수들이 하는 말들은 서로 약간 다를지 몰라도 공통점은 있는데
최소한 시간 관리 측면에서만큼은 진짜 '보수적'으로 가야한다는 것, 즉 최악 상황을 감안해야한다는 것만큼은 진실이고
수험 최고의 거짓말은 '열심히 공부하겠다'라는 것입니다.
\vspace{5mm}

그런데 올해 시험 물건너가면 사실 내년 시험도 힘들 거라고 보는데
그나마 올해 고3들은 작년 고3에 비해 실력이 낮은 경향이라도 있는데 내년 고3들은 아주 잘할 거라고 보고 있어서 말입니다.
(거기다가 영어 절평, 아주 역대급이 되지 않을까 싶음)
\vspace{5mm}








\section{수험커뮤니티는 자칫하면 사이비 종교화되기 쉽죠.}
\href{https://www.kockoc.com/Apoc/680137}{2016.03.17}

\vspace{5mm}

오프라인에서 수험사이트 아무개나 수험 컨설팅 어쩌구 이야기를 들으면 이렇게 반론해줍니다.
"제가 같이 수업들었던 친구들 절반 이상이 서울대 갔고 어쩌구저쩌구해도 그렇게 교주짓하는 사람들은 없습니다"
"그 친구들 진짜 학벌이나 경력이 정말 객관적으로 대단한가요?"
\vspace{5mm}

이렇게 추궁하다보면 왜 그런 데를 언급하느냐 하면 '말빨이 좋아서'라고 하더군요.
학생이든 학부모든 다 입시공포증에 걸려있습니다. 죽음에 대한 공포가 종교의 기원이라지요.
\vspace{5mm}

냉정하게 말해서 인터넷에 올라온 수험썰들이 도움이 되는 경우는 별로 없습니다.
인터넷을 끊고 그냥 현강으로 학원수업들으면서 시중교재 열심히 파도 입시에 성공할 가능성은 높을지 몰라도
인터넷에서만 언급되는 것을 맹신하면 5년 이상 날리는 것도 일이 아닙니다.
\vspace{5mm}

현상을 예측할 때는 변증법을 주로 사용하면 됩니당.
그런데 이 변증법을 쓸 때의 정반합에서 반(反)은 물질적 불균형에서 찾을 수가 있죠.
과연 수험시장이 정상적으로는 거액을 벌 수 있는 구조인가.
냉정히 생각하면 그럴 수는 없습니다. 그런데 그런 일이 벌어진다는 것은 이 판도 정상이 아니란 얘기죠.
\vspace{5mm}

적지않은 수험생들인 결국 거액을 주고 '재고'를 떠안습니다.
재고라는 것은 소화시키지도 못 하는 참고서를 말하는 것이지요.
재고가 생겼다는 것은 이미 공부를 제대로 못 했다는 이야기입니다. 그럼 수험에서 실패할 가능성이 높지요
\vspace{5mm}

현재 일지를 써보시면서 총회까지 들어오신 분들도 열심히 하는 분들 많습니다.
그런데 느끼셨을 건데, 시중 교재조차도 단기간에 끝내기 힘들다라는 것을 깨달으셨을 것이고
그럼 어떤 교재 보느냐라고 물어보는 게 얼마나 어리석은지 아셨을 것입니다.
\vspace{5mm}

그럼 왜 사람들은 그런 꿀교재물어보는 것에 빠질까요. 그거야 그 때는 수험이 과학이 아니라 \textbf{종교여서 그렇지요}
이 때 그들이 원하는 수험서는 '부적'입니다. 그것만 갖고 있으면 온갖 액운을 피할 수 있고 복을 얻을 수 있다고 믿는 것이지요.
하지만 구매한 다음에 서재에 꽂은 참고서는 풀지 않는 이상 애물단지입니다.
\vspace{5mm}

하지만 재밌는 건 구경하다보면 이 책만 보면 수능 1등급이 나온다라는 식으로 '부적'을 파는 사람들도 많죠.
아울러 자기들 말만 들으면 자녀들이 좋은 대학에 간다고 약파는 '무속인'들도 있습니다.
현실의 무속인조차도 갓 신내림 받은 사람이면 모를까, 2$\sim$3년 지나면 약빨이 다 해서 그 다음부터는 콜드리딩으로 말맞추기하는 현실일텐데.
\vspace{5mm}

더 좋게 말해보았자 그들은 \textbf{수험떴다방} 혹은 \textbf{수험브로커} 정도겠죠.
저는 그 사람들이 수능을 쳐서 정식으로 관악에 들어가는 걸 보여주는 게 가장 확실하다고 생각합니다요.
그런데 그렇지 않고 인터넷 글이건 책에서든 핵심은 피한 채 온갖 미사여구로 인간극장을 찍고 있더군요.
\vspace{5mm}

어차피 저는 그런 걸로 사업하는 건 별 관심도 없기 때문에 그냥 과학적으로 이것만 얘기할 수 있습니다.
\vspace{5mm}

첫째, 남들보다 훨씬 더 정확하게 신속하게 돌아가는 두뇌를 후천적으로 만드는 커리가 필요하다
둘째, 시중의 상위권들조차도 알지 못 하는 특수한 교재는 분명히 존재한다.
\vspace{5mm}

첫째의 경우는 저도 그런 사람들을 만나보았고 같이 공부했으며 겪어보았으니까 말할 수는 있습니다.
다만 선천적인 것보다도 오히려 후천적인 게 더 중요하다고 보이는데 과연 그런 환경들의 엑기스만 뽑아내서 어떻게 구현할 수 있을까.
한가지 분명한 건 여기서 작은 것의 반복이라는 게 키라는 것입니다. 소위 '구몬' 시스템 같은 것
물론 이 이외에도 엄청나게 많습니다. 무엇보다 어린 시절에 어떤 자극을 받았느냐하는 게 상당히 큰 역할을 끼치는데
그걸 2, 30대 양민들에게도 어떻게 적용할 수 있을까 하는 게 제 관심사입니다.
\vspace{5mm}

둘째의 경우는 사실 입증된 적이 있죠. 과거에 과고, 외고에서 집중적으로 설대에 간 적이 있습니다.
지금이야 힘든 일이지만 그 당시 가능했던 이유 중 하나가 그 때는 인터넷 강의가 등장하지 않았고 주요소스를 독점했기 때문입니다.
이게 유의미한 이유는 현재 특목자사고에 가는 친구들도 머리가 상당히 좋은 편인데도 그런 성과를 못 거둔다는 것이지요.
2000년대부터 특목고에서 독점하던 것이 전국적으로 퍼졌습니다. 그래서 지금은 비급이라는 것이 보편화된 것입니다.
농담처럼 한 이야기가 아니라 쎈수학 같은 책이 소수만 보았다면 이건 정말 한권에 100만원 넘어가는 비급 취급받았겠죠.
\vspace{5mm}

비급이 존재하는지 안 하는지 모르겠지만 적어도 수학 29번과 30번, 그리고 과탐 킬러를 시중인강이나 교재로 커버 못 할 수 있다는 것.
그런데 출제자는 그걸 낸다는 점에서는 직접적이든 간접적이든 그런 비급을 만들어낼 수 있다고 할 수도 있을 것입니다.
그런데 재밌는 건 그런 비급은 '출판되면' 더 이상 비급이 아니란 것이죠, 대중적인 인강도 마찬가지입니다.
폐쇄적인 집단에서 소수만이 교육되고 타인에게 전파하지 않는다.... 과거 특목고는 그랬던 적이 있었습니다.
\vspace{5mm}

머리와 비급이라면 이건 수백, 아니 수천만원을 지불해도 아깝지 않을 것입니다. 인생을 바꾸는 거니까.
하지만 시중에 많이 팔리는 것들이 이런 것을 담보한다는 건 거짓말이죠.
마약을 못 구하면 마라톤을 뛰면서 러너스 하이로 천연마약인 도파민 분비를 촉진시키듯이 공부하는 수 밖에 없을지도 모릅니다.
\vspace{5mm}

그런데 정작 사이비 종교들은 저 두가지를 얘기하지 않죠.
그런 것을 알고 있었다면 굳이 호구들을 현혹해서 장사할 필요도 없었겠지만.
\vspace{5mm}







\section{공부에 있어서 회독수}
\href{https://www.kockoc.com/Apoc/682149}{2016.03.18}

\vspace{5mm}

기본서 $-$ 10회독
객관식 $-$ 5번씩은 풀 것
기출 모의 $-$ 3번 정도
\vspace{5mm}

평균화시킨 레시피(?)가 저 정도는 됩니다.
학교나 학원에 다니는 열심히 공부하는 친구들도 저 정도는 자기도 모르는 사이에 달성되는 것이죠.
\vspace{5mm}

그런데 인강만 듣는 경우 한번 듣고 복습을 안 하고 끝납니다(피로도가 높아서, 거기다가 딴짓을 해서)
1회독만 하면 90$\%$를 망각하죠. 짧은 시일 내에 다시 복습을 해야 그 망각도를 절반 정도 줄일 수 있습니다.
\vspace{5mm}

강남 모 지역의 학원들은 실적이 좋습니다. 하지만 실질 실적은 갸우뚱할 수 밖에 없는 게, 애시당초  '시험을 쳐서' 우수한 애들을 뽑으니까요.
우수한 애들이란 어린 시절부터 관리가 되어서 머리가 만들어진 애들입니다. 이런 친구는 회독 습관이 들여있어 학습도가 높습니다.
그래서 학원 입장에서도 한번만 가르쳐도 열을 깨우치는 애들이니 가르치기도 편하거니와 성과도 좋은 것이죠.
\vspace{5mm}

반면 회독수가 안 된 친구들을 저런 머리로 만드는 것?
시간과 노력도 많이 걸리지만 성과도 기대만큼은 아닙니다. 그래서 학원에서는 이런 친구들은 걍 버리거나 아니면 은폐시켜버리죠.
그 점에서 인강만큼 편리한 것도 없습니다. 왜냐면 인강을 들어도 효과가 없는 경우는 책임을 질 필요가 없기 때문입니다.
\vspace{5mm}

그럼 과연 10회독이 쉬운 건가.
평균적으로도 그렇고 일지도 그렇지만 고3 기준으로 해도 5월까지 기본교재 2회독만 해도 다행입니다.
물론 초상위권들은 능력이 좋으니 더 많이 돌려 7회독은 해놓으며 시험 직전에 15회독까지 하기도 하죠.
수학문제 풀이 뿐만 아니라 출제가 어떻게 되는지도 그리고 출제자가 어떤 함정을 까는지도 달달 암송할 정도입니다.
하지만 하위권들은 이런 경지를 모르니까 모 선생만 잘 따라가면 된다... 라고 생각하고 계속 우물안을 맴도는 것이죠.
\vspace{5mm}

아무튼 10회독도 해보지 않은 사람들이 머리나 수험을 탓하는 건 아니라고 생각합니다.
그리고 진정 뛰어난 학원은 이런 회독수를 확보해주는 곳입니다.
\vspace{5mm}

제가 쎈이 좋은 교재라고 하는 이유는 A, B 스텝에서 내재적 회독수가 확보되는 편집구조여서입니다.
정석도 비슷할 수 있지만 정석은 난이도 격차나 편집에서 회독수를 달성하기 어렵습니다.
\vspace{5mm}

반면 야매교재들에 대해서 까는 이유는 간단합니다,
자기들의 노트만 짜깁기해놓은 수준이고 어떻게 공부해야할지 저자들도 모르는 게 보여서요.
그 저자들 이력보면 왜 그럴 수 밖에 없는지 이해가 안 가는 게 아니죠.
\vspace{5mm}

수학의 경우 제가 양민들에게 권하는 코스가 (이건 수도없이 반복해서 이제 다들 외우실 겁니다)
기본적으로 쎈(풍산자)를 보고 그 다음 복습용으로 RPM 선택 가능,
그 다음으로 마플이나 마더텅을 보면서 급품벨 중 2권 선택, 그리고 EBS 따라가라는 것입니다.
실모중독자들이나 광신도들은 이걸 가지고 EBS가 안 좋다 깔 것입니다(라고 하지만 참신한 문제는 정작 EBS에 있던데)
이렇게 권하는 것은 저런 과정으로 가면 회독수가 확보되어서 양민들이 고수가 될 수 있는 현재로서 가장 안전한 나선형 코스여서 그렇습니다.
물론 지금은 쎈과 풍산자는 1$\sim$2회독 완료하고 마플을 풀고 있어야하는 시점이지요.
\vspace{5mm}

지극히 평범하고 '실천가능한' 안을 분명 제시했습니다.
물론 이건 여흥이지만 이렇게 해도, 안 되는 친구들은 여전히 수험신비주의에 사로잡혀
아무개 강사를 반드시 들어야 한다, 이상한 야매교재 봐야한다 그러면서 또 올해 1년 공치는 걸 야동 보듯 몰래 훔쳐다보고 있죠.
악취미(?)라고 할지 모르겠지만 제가 쓰는 xx론의 근거는 "실패'입니다. 실패하는 과정의 반대로 가면 성공이 있기 때문.
\vspace{5mm}

어떻게 보면 공부를 하기 싫어하는 그 친구들의 뇌가 무의식적으로 '비현실적인 안'을 일부러 고르는 것일수도 있어요.
수험실패를 일부러 하기 위해서, 공부하기 싫다를 '넌 공부할 수 없어'라고 자포자기하게 미리 까놓는 것이죠.
\vspace{5mm}






\section{회독학습을 오해하시는 것 같은데 간략히 적죠.}
\href{https://www.kockoc.com/Apoc/683063}{2016.03.19}

\vspace{5mm}

아래 글에 댓글로 적었습니다만.
\vspace{5mm}

쎈을 기준으로 한다면
\vspace{5mm}

1회독 : 개념 읽기 $-$ 개념에서 유도하고 싶은 건 네이버 검색이나 교과서 참조 무방 $-$ A형 풀고 오답정리
2회독 : B형 대표문제, 그리고 각 유형별로 난이도 中 문제 절반 풀고 오답정리
3회독 : B형에서 中 문제 나머지 풀 것.
4회독 : B형에서 上 문제만 풀어나갈 것
5회독 : C형 절반 풀 것 (짝홀 나눠서)
6회독 : C형 나머지 풀 것
\vspace{5mm}

이렇게 나눠 가시면 6회독 달성이실 텐데. 그리고 그 반복 과정에서 얻어질 것이 충분히 많을 텐데 말입니다.
저렇게 나눠가는게 처음부터 C스텝까지 다 푼다고 하는 것보다 훨씬 쉽고 기억하기에도 좋으며 책을 내 것으로 하는 데 도움이 됩니당.
Divide and Conquer 교범에 충실한 방법이고, 7번 읽기 공부법의 야마구치 마유도 비슷한 이야기를 했죠 아마.
\vspace{5mm}

쎈 한권 기준으로 한다면 1회독은 7일, 2회독은 14일, 3회독은 10일, 4회독은 10일, 5회독은 10일, 6회독 10일 정도로 잡으시면 됩니당.
\vspace{5mm}

이거 아는 줄 알았는데 예상 외로 모른다는 것을 파악해서 요령 알려드립니다. 아니 이런 건 안 배우셨나들 다 $-$$-$
\vspace{5mm}

그럼 마플의 경우는?
\vspace{5mm}

그거야
\vspace{5mm}

1회독 : 행복한 1등급 제외하고 홀수번만 풀어나걸 것
2회독 : 행복한 1등급 제외하고 짝수번만 풀어나가기
3회독 : 행복한 1등급 절반 건드리기 + 경찰대/사관 기출 절반 건드리기
4회독 : 행복한 1등급과 경찰 사관 기출 아작내기
5회독 : 행복한 1등급 최고난도 풀어버리기(분량이 별로 없어서리)
\vspace{5mm}

이것도 비슷하게 스케줄화하면 됩니당.
그리고 교재 한권만 집중적으로 팔 필요 없음
\vspace{5mm}

가령 쏀과 마플을 본다면
위에 제시한 쎈 1, 2회독 $-$ 마플 1회독 $-$ 쎈 3회독 $-$ 마플 3회독 $-$ 쎈 4회독 $-$ 마플 4회독 $-$ 쎈 5회독 $-$ 마플 5회독 $-$ 쎈 6회독
이런 식으로 조합해나가도 됩니다. 실제로 저도 이런 식으로 하라고 지시하고 있는데 말이지요
즉 공부해나가는 난이도의 함수를 미분가능하게 설계하는 게 묘미임
\vspace{5mm}

그런데 적극적으로 회독수 공부가 뭔지 물어본 사람은 오늘에야 나왔죠(...)
자기들이 적극적으로 물어보거나 핵심적 질문을 안 하고 저보고 '말바꾼다'라고 하면 할 말이 없음.
저렇게 쪼개서 공부하는 건데 그걸 모르셨단 말인가(...)
\vspace{5mm}

그런데 말은 조금씩 바뀔 수 밖에 없는 것도 있음, 왜냐면 상황은 계속 변하니까.
가령 xx가 좋은 문제집이더라 하면 xx은 당연히 추가해야죠. 왜냐면 그래야 다른 수험생들과의 경쟁에서 이기니까.
예컨대 인수(는 보지 않았지만)가 좋은 교재라고 하면 이것도 추가하면 해야하는 것입니다.
그래서 처음부터 완벽한 계획 짜는 건 위험하죠. 시간은 한정되어 있는데 풀 문제집은 생기니까 말입니다.
\vspace{5mm}

그런다고 하더라도 쎈과 마플만 제대로 풀면 4점짜리 대비책 여름에 따로 세울 것 빼고는 망해도 2등급은 뜰 수 밖에 없음.
일지 쓰는 분들도 조사해보니 이 정도까지 한 사람도 거의 없고, 다른 데도 스텔스 잠행해보니 다들 입공부하고 있죠(그러고 망했다고 울겠지)
\vspace{5mm}

다만 이걸 알아두셔야 함
\vspace{5mm}

자기 등급이 올라갈수록 싸워야하는 적은 줄어들지만, 대신 그 적의 수준은 후덜덜하게 높다는 것.
예컨대 100위권에 든다면 100명만 승부하면 되겟죠. 그런데 그 100명이면 정말로 공부로는 일당백하는 괴수들입니다.
일당백이 수사적 의미가 아니라 국수영 문풀이나 사고 흐름으로는 30명 양민을 연결한 것보다 낫다는 겁니다.
\vspace{5mm}





\section{1인자와 2인자}
\href{https://www.kockoc.com/Apoc/684760}{2016.03.20}

\vspace{5mm}

성공은 반추할 수 없지만 도취하기 좋습니다.
실패는 기뻐할 수 없지만 반추할 수 있습니다요.
\vspace{5mm}

젊은 시절 많은 경험을 해보라고 하지만 그게 구체적으로 뭔지 말해주진 않습니다.
개인적으로 그게 뭘까 찾아보고 읽어보고 내린 결론은
"경험"을 통한 깨달음이란, "실패를 수습하고 그로써 교훈을 얻고 자신의 단점을 보강해나간다"는 것입니다.
\vspace{5mm}

실패는 '기출 문제'입니다.
나의 실패는 \textbf{나만이 풀 수 있는 기출문제}입니다.
\vspace{5mm}

첫째, 문제의 존재가 확실하다
둘째, 무엇이 오답인지, 그리고 어떤 것이 정답인지 알 수 있다.
\vspace{5mm}

상담하다보면 누구나 자기만의 기출문제가 있습니다.
상담해주는 의도가 뭐냐라고 하겠지만 다른 게 아닙니다. 타인의 기출문제들을 풀어볼 수 있기 때문입니다.
그런데 똑같이 실패를 겪고도 기출문제를 반복해 푸는 사람이 있는 반면, 외면해버린 채 다른 문제집을 풀려하는 사람이 있습니다.
정확히 말하면 전자는 소수고, 후자는 다수입니다. 시간이 지나면 전자가 성공하고 후자는 풀어야 할 기출문제집이 늘어납니다.
자신의 기출문제를 정복한 사람은 타인의 기출문제도 풀어보며, 아울러 자기와 타인을 위한 예상문제도 만들어냅니다.
그런 예상문제를 미리 만들어 풀어주는 사람이 지도자입니다.
\vspace{5mm}

극단적으로 말해서 문제를 푸는 건 누구라도 할 수 있습니다만, 어떤 문제가 나올지 정확히 예측하는 건 아무나 하는 게 아닙니다.
본인 스스로 가상의 문제를 내보아야할 뿐만 아니라, 운명이란 이름의 출제위원이 내는 의도까지 읽어내야 합니다.
본인이 실제로 매우 평온한 삶을 살아갈 수 있는데도 불구하고 문제들을 가정해보고 해결하려한다면
그 사람은 어떤 문제가 나오더라도 고득점을 거둘 수 있을 것입니다.
\vspace{5mm}

지식이 문제해결을 위한 재료라면, 지혜라는 건 그 문제를 해결해나가는 '실천'을 말하는 것입니다.
지혜는 책만 읽는다고 되는 것이 아니라, 적극적으로 지금 벌어진 문제를 해결해나가고 미래의 문제까지 미리 끌어다 해결해야 늘어납니다.
학교나 학원에서 가르치는 건 오직 정보입니다, 정보 자체로는 아무 것도 할 수가 없습니다.
그 정보를 체계적으로 정리하고 경중을 가리면서 의미를 부여해야 비로소 문제해결에 도움이 되는 지식이 됩니다.
그러나 지식 자체만으로는 문제해결을 할 수 없습니다, 지식이 없어도 문제를 해결하려는 실천적 지성인 '지혜'가 있어야죠.
이 지혜는 본인이 직접 겪어보지 않는 이상은 늘어나지 않습니다.
\vspace{5mm}

수능에서 N수한다 어쩐다고 하겠지만 냉정히 말해서 이 분들은 보고를 놓치고 있죠.
물론 저는 입시에 한해서는 현역으로 좋은 대학 간 것이 진짜라고 봅니다만, 이런 케이스는 고생과 실패를 별로 겪지 않기 때문에
역설적으로 매우 위험해집니다. 다시 말해서 수동적인 지식 축적은 가능해도 적극적인 지혜를 키우는 것까진 힘들다고 생각합니다.
(사실 이런 사람들 인생을 추적해보면 3인자까지는 몰라도, 1인자나 2인자는 찾기 어렵습니다. 우연만은 아니지요)
하지만 1인자나 2인자에 해당하는 사람들은 학벌이 생각 외로 보잘 것 없고(?), 거기다가 실패한 적이 꽤 많습니다.
즉 이건 그들이 시간이 걸리더라도 자신들의 기출문제를 정리했으며, 지식은 비천할지 모르나 지혜가 쌓였다는 걸 말하는 것입니다.
\vspace{5mm}

공부를 잘 해보았자 소용없다라는 건 물론 공부를 안 한 사람들의 공격일 수 있지만
선해해본다면 '공부를 잘 해보았자 우두머리, 즉 리더가 되지 못 하면 무의미하단' 이야기입니다.
(돈을 많이 버는 것도 우두머리입니다. 부하가 되어보았자 벌지 못 합니다)
그런데 우두머리의 조건은?
남들이 쉬쉬하고 기피하는 문제를 본인이 해결하고 주도하며 남의 문제를 해결해주려는 것입니다.
우리는 그런 사람들에게 고개를 숙이게 됩니다.
자기 목숨까지 걸고 리스크를 감수하면서 부딪쳐가는 사람이 1인자고,
자기 목숨까지는 걸지 않되 지혜를 제공하며 보좌하는 게 2인자입니다.
\vspace{5mm}





\section{몇가지 질문에 대한 답변}
\href{https://www.kockoc.com/Apoc/685888}{2016.03.20}

\vspace{5mm}

\item \textbf{1. 작년 수능이 기대보다 망했는데 또 도전해야겠느냐.}
\vspace{5mm}

다른 것을 떠나서 실패를 했다면 왜 실패했나 그 원인부터 파악해야 합니다.
과학적인 분석이 없는 수험은 곧 종교행위가 됩니다. 그게 실패든 성공이든 종교가 되어버리면 그 다음부터는 답이 없습니다.
운좋게 성공해놓고 왜 성공했는지 이유를 모르면 어차피 곧 망합니다.
실패했지만 그 원인을 철저히 분석해서 개선하면 그 다음에는 실패할 수가 없습니다.
\vspace{5mm}

우선 이것부터 해놓으면 도전해야 할지 말아야 할지가 보입니다.
실패를 좌우하는 건 상당히 '사소한 원인'에서 비롯됩니다.
그런데 그 사소한 원인은 알고보면 '빙산의 일각'이죠. 그걸 분석해보면 자기가 인식 못 했던 거대한 문제가 발견됩니다.
그 문제를 잡아내어야 합니다.
\vspace{5mm}

재수가 삼수 사수로 이어지는 건 그 문제가 해결되지 않아서입니다. 우리는 그 미지의 대상을 모두 '운'으로 돌리죠.
운이라는 것도 우주적 차원에서는 \textbf{필연}이 됩니당.
\vspace{5mm}

\item \textbf{2. 영어 처음부터 단어 외워야하느냐}
\vspace{5mm}

이것도 고정관념이 있는 것 같은데 그냥 말씀드립니다.
마더텅이나 자이 같은 기출 가지고 그냥 어려운 지문부터 풀고, 답 낸 이유 적고, 틀린 다음 해설과 비교하고 해설 자세히 보세요.
즉 영어는 어려운 지문부터 걍 공략하고 까이고 극복하길 바랍니다(이건 국어도 마찬가지입니다)
\vspace{5mm}

수학은 쉬운 것부터 회독수를 늘리는 게 좋습니다. 왜냐면 수학의 진정한 실력은 기초에서 나오기 때문입니다.
수학의 기초 개념이야말로 사실은 '고차원 수준의 보물'들입니다. 그렇기 때문에 그 기초들을 익혀야하는 것입니다.
어려운 문제를 $\sim$ 하게 풀면 되죠라는 식으로 인기 강사가 아무리 지껄여보았자 본인이 확실히 기억하고 있는 교과서 개념만도 못 합니다.
\vspace{5mm}

하지만 영어나 국어는 다릅니다. 영어나 국어의 사고법은 어려운 지문을 잘 해석하고 그 문제를 푸는 것이 오히려 기본일 수도 있습니다.
수능에서는 자잘한 걸 요구하기보다도 실제로 학생이 어려운 지문을 잘 분석하고 분해해보느냐를 따져보길 때문입니다.
이런 건 본인들이 어려운 지문에 도전해보는 걸로 늘어납니다.
\vspace{5mm}

다시 말해 수학이 찰흑을 빚어 뭔가 만드는 것이라면, 영어나 국어는 거대한 대리석을 깎아나가는 조각과 같습니다.
엄선된 글들을 여러번 읽으면서 거기에 내재되어있는 사고법들을 익히고 이것들을 독해하는 걸 훈련하는 역삼각형 방법으로 가는 게 낫습니다.
어떤 식으로 푸느냐보다도, 각자의 '독법'이라는 걸 만들어야 합니다.
국어나 영어는 절대 객관적일 수가 없습니다. 원래 주관적인 논리에서 주관성을 최대한 배제해 객관성에 근사시킨 것일 뿐,
그것들이 객관적일 수는 없는 것입니다. 그러므로 국어나 영어의 어려운 지문을 읽고 풀어가는 것은 표준화된 방법이 나오기 힘듭니다.
글을 읽는 건 독자 자신의 철학과 성격에 종속적입니다.
\vspace{5mm}

덧붙여 말하면 문과 분야의 객관성이란, 주관적인 것들이 서로 충돌하고 갈등을 빚다가 나오는 타협적인 것에 불과하지
애초에 객관적인 것이란 존재할 수 없습니다.
형식논리학으로 참 거짓을 실제로 분별하는 건 어렵습니다. 애초에 세상이 연역논리로 설명되기는 힘들기 때문이죠.
어떤 주장이 참이다 거짓이다보다는, '더 옳은 점이 많다'라는 개연성, 타당성으로 가는 게 현실입니다.
실제 수능 국어나 영어의 독해는 그 개연성과 타당성을 전제로 한 조건부 확률적 문제를 내고 있죠.
\vspace{5mm}

\item \textbf{3. 시험 망하면 어쩌냐}
\vspace{5mm}

평생 잘 먹고 잘 살 줄 알았던 4, 50대 아재들조차 모가지 잘리는 그런 세상입니다.
젊었을 때 실패해본 적이 없기 때문에 우물쭈물하다가 거액의 빚을 지고 한순간에 망해버리는 케이스가 적지 않네요.
승부를 했다고 이길 수는 없습니다. 그런데 본인이 탁월한 투자가라면, 이기는 방법보다는 '지더라도 안 망하는 방법'을 궁리하겠죠.
다시 말해 공격력과 방어력 중 선택하라면 방어력이고, 딜러와 탱커 중 택해라면 탱커가 낫다는 것입니다.
\vspace{5mm}

만약 본인이 특정 시험에 모든 것을 걸었다면 그것 자체를 걍 바꾸시길 바랍니다.
수능을 쳐도 안 되면 원칙적으로는 다른 길로 갈 수 있도록 하는 것도 나쁘지 않습니다.
다만 그 다른 길에 대한 의식이 수능까지 말아먹는 거라면 이건 제3의 안을 택해야겠지만요.
\vspace{5mm}

최근 헌책방에서 정말 엄청난 책을 구했습니다. 시세 잘 쳐주면 30만원짜리일 건데 5000원에 구했죠.
(그 저자는 당시는 평범한 엔지니어였지만 지금은...)
그런데 거기 명문이 적혀있더군요. 아무 것도 모를 때는 세상이 모순 덩어리인 것 같지만
알고보면 자기가 잘못 알고 있어서 착각하는 경우가 많다고.
\vspace{5mm}

수능에 응시하는 것도 마찬가지입니다. 실제로는 수백만가지의 길이 있는데 우리는 의도적으로 하나만 보고 있는 거죠.
\vspace{5mm}






\section{진도}
\href{https://www.kockoc.com/Apoc/692514}{2016.03.24}

\vspace{5mm}

아부하기 싫어서 적는다면 지금 상당수가 '늦습'니다.
일단 이 시점에 뭘 봐야하느냐 물어본다면 이미 상당히 밀린 것입니다요.
이제 3월말인데 기출도 절반 이상은 다 돌렸어야합니다. 즉, 쎈이든 풍산자든 하나는 끝냈어야한다는 얘기죠.
\vspace{5mm}

상당수가 자기들이 교재를 잘못 선택해서 혹은 강의를 안 들어서 아니면 잘못 들어서라고 착각할 건데
이건 무의식적으로 교재나 강의 탓을 하는 '책임전가'입니다.
실패 요인은 \textbf{'진도'를 일찍 빼지 못 해}서입니다.
즉, 스케줄을 못 맞추었기 때문입니다.
\vspace{5mm}

거의 다 1, 2월까지는 시간이 꽤 많을 거라고 착각에 빠지면서 설렁설렁하죠.
그런데 남은 시간은 자기가 생각하는 것의 1/3도 되지 않고, 해야할 건 자기가 생각하는 것의 3배를 넘어갑니다요.
이게 맞는지 틀린지야 본인들이 경험해보시면 됩니다.
작년 일지 사례와 콕콕 입시 성과로 보건대 성적이 좋은 쪽은 정말 해야할 것을 남들보다 다 빨리 빼고 일찍일찍 돌린 경우입니다.
소위 수험사이트나 야매교재업자들이 강조하는 방법으로 성과가 좋은 건 거의 못 보았습니다요(좋으면 제가 그걸 추천했죠)
\vspace{5mm}

뭔 교재를 보아야하느냐가 아니라, 자신의 속도와 밀도를 높이는 데 신경쓰시길 바랍니다.
자기가 풀지 않은 교재는 그냥 재고입니다. 뭐든 일단 끝까지 풀면서 뇌를 단련시키는 과정입니다요.
\vspace{5mm}

진도가 밀리지 않은 사람과 그렇지 않은 사람들과 얘기해보거나 글을 읽어보면 '마인드'도 확실히 차이가 납니다.
전자는 자기 할 게 바빠서 사소한 질문은 하지 않습니다. 오직 힘들어죽겠다 그 한마디입니다.
후자는 여전히 뭔 교재가 좋아요라는 질문을 병적으로 던집니다. 그리고 힘들다라는 말을 하지 않아요,
그러다가 6월 넘어가면서 또 포기각 나오겠죠.
\vspace{5mm}









\section{뭔 교재 보았냐 물어보지 말고}
\href{https://www.kockoc.com/Apoc/694072}{2016.03.25}

\vspace{5mm}

\item 1. 어떻게 실패했냐
\item 2. 어떤 환경이었느냐
\vspace{5mm}

이 2가지만 보시면 됩니당.
\vspace{5mm}

SKY 붙거나 의대에 간 사람들은 겸손을 가장해 자기가 머리가 좋아서... 라고 하겠지만 그건 개뻥이고
대부분은 천부적 환경이 좋거나, 아니면 본인이 노력해서 환경을 잘 만들어서 기존의 나쁜 환경을 극복한 케이스입니다.
당연히 '실패하는' 코스를 피해간 것이죠
성공하는 방법을 모르면 실패하는 방법을 제대로 알면 됩니다. 그리고 그 반대로 가면 일단 망하지는 않으니까요.
아울러 사람을 규정하는 건 환경입니다. 환경이 의식과 행동에 영향을 주기 때문이죠.
\vspace{5mm}

동양에서는 일을 도모할 때 따져보는 것으로 \textbf{천시지리인화}가 있습니다.
요즘 와서는 천시 = x수저로 바뀌는 것 같고  인화라는 것은 어떤 네트워크에 들어가느냐 하는 걸로 결정되는 걸 본다면
역설적으로 남은 것은 지리인데, 이 지리라는 것이 실제로는 우리가 의지대로 바꿀 수 있는 것일지도 모릅니다.
시간의 가치도 자기가 어디서 사느냐 혹은 어디서 일하느냐에 따라 달라집니다.
같은 시간이지만 번화가 마천루 로얄층에서 보내는 것과 어디 달동네 폐가에서 보내는 게 똑같을 수는 없죠.
그런 점에서는 천시조차도 지리에 지배되는 감이 없지 않으며 또한 사람들도 번화가에 몰리는 것을 보면 인화 역시 마찬가지입니당.
\vspace{5mm}

뒤늦은 감이 없지 않지만 올해 수험 시작하는 사람 중에서 현명한 사람이라면 어디서 공부하느냐 하는 걸 먼저 물어보았고 실천했겠죠.
집에서 공부하지말라하는 것도 마찬가지입니다. 공부가 잘 되는 사람이 질문할 리는 당연 없겠고
보통 안 되어서 질문할 건데, 그럼 그건 '집의 영향'도 만만치 않기 때문입니다.
\vspace{5mm}






\section{집착}
\href{https://www.kockoc.com/Apoc/694074}{2016.03.25}

\vspace{5mm}

최상위권과 상위권 : 3년
상위권과  중위권 : 3년
중위권과  하위권 : 3년
\vspace{5mm}

너무 그럴싸하게 근사한 것 같은데 대충 이 정도의 격차가 발생합니다.
그럼 왜 저런 격차가 생기느냐 하는 건 "환경" 차이가 가장 크지만, 누가 먼저 일찍 공부했느냐도 중요하겠죠.
초딩 때 고1 수학까지 건드렸다면 남들보다 6년 앞서간 겁니다. 무리하지 않게 공부하면 당연히 최상위권이 될 수 밖에 없습니다.
\vspace{5mm}

그런데 비극은 '공부해야하는구나'라는 걸 너무나 뒤늦게 깨닫고 있는데다
그걸 깨달았을 때에는 공부에 방해가 되는 요소들이 많아졌다는 것이죠.
거기다가 마음은 급하지 그래서 올해 당장 합격하고 싶지.... 하지만 현실적으로 그게 가능하지가 않죠.
공부 못 하는 친구들의 특징인 집착하는 게 너무 많다는 겁니다.
과거에도 집착하고 현재 자기가 갖고 있는 교재나 질러버린 인강에도 집착하고.
\vspace{5mm}

집착한다는 것부터가 이미 행동양식이 비효율적이라는 얘기죠.
사고와 행동이 효율적인 사람은 미래에 방해가 될 것은 과감하게 정리해버립니다. 이것들의 효용이 적다는 걸 알고 있어서입니다.
자기 교재나 인강에 지른 돈이 많더라도 그게 도움이 안 된다고 합리적인 확신이 든다면 과감히 버립니다.
이 사람들이 신경쓰는 건 경쟁자와의 승부죠.
\vspace{5mm}

상위권들은 집착을 해도 미래의 순위에 집착합니다. 그래서 낭비 요인이 없으니 더 쭉쭉 나가는 것이죠.
계속 그 자리를 지켰으니 트라우마 될 것도 없고 더 좋은 방식이 있다고 보면 과감히 옮길 수도 있습니다.
전진해왔기 때문에 밑천 건져야한다 그런 생각도 덜 합니다.
\vspace{5mm}

그래서 제가 업자들이 문제라고 보는데. 다수의 중하위권들이 집착을 하는 요인 중 하나가 업자광고에 세뇌된 게 있어서입니다.
xxx 강의나 xxx 교재 보면 안 되나요... 공부 안 한 친구들이 저런 강의나 교재가 좋다고 말하는 '배경'이 무엇이겠어요?
그런 걸로 돈 번 인간들은 지옥에 가지 않을까 싶습니다.
\vspace{5mm}









\section{교과외적(?)인 증명과정이 필요없는 게 아닐텐데}
\href{https://www.kockoc.com/Apoc/706283}{2016.04.01}

\vspace{5mm}

수학을 열심히(?) 공부해도 성적이 안 나오는 이유에 대해서 나름대로 가설을 세워보고 찾아본 결과는
수학은 다른 과목과 달리 \textbf{공부한 문제가 그대로 시험에 출제되지는 않는다}는 것입니다.
다시 말해 A라는 문제를 연습했다고 치면 시험에 나오는 것은 A가 아니라 A'' 혹은 A''B'''라는 것입죠.
학생 중에서 난감한 경우가 "무조건 암기하라", "패턴을 외우라" 하면서 이 문제는 이렇게 푼다라고만 암기하도록 배운 케이스인데,
이렇게 푸는 방식을 외우는 케이스가 점수가 잘 나오는 경우는 매우 드뭅니다, 사실 정상적이라면 나올 수가 없어요.
\vspace{5mm}

암기할 대상은 '정의와 성질' 정도면 족합니다. 그리고 그 암기도 문제를 푸는 과정에서 저절로 이뤄지는 것이죠.
\vspace{5mm}

그럼 수학공부를 왜 하느냐라고 하면
생각하는 방식을 연습하기 위해서입니다, 수학공부의 잇점은 딱 하나. "예술적으로 사고한다" 그 정도입니다.
잠시 딴 이야기하자면 공대에 가서라도 그리고 대기업에서도 유감스러우나 수학을 쓸 일은 그리 많지 않습니다(공업수학이 수학?)
수학이 정말 중요하다라고 하는 건 어디까지나 그 전공자나 교육자들이나 하는 이야기죠.
\vspace{5mm}

일본의 전설적인 로켓공학자인 이토가와 히데오씨가 자신의 책에서 아래와 같이 얘기한 적이 있죠.
이토가와 히데오는 전투기 설계로 유명했고(하야부사), 그리고 펜슬로켓 아이디어도 대박친 분입니다.
그의 이름이 소행성에도 붙을 정도죠(https://namu.wiki/w/$\%$EC$\%$9D$\%$B4$\%$ED$\%$86$\%$A0$\%$EC$\%$B9$\%$B4$\%$EC$\%$99$\%$80)
공대 진학하셔서 뭔가 뚝딱 만드실 분이라면 이 분이 쓴 책은 읽어보시길 권하며.
\vspace{5mm}

나는 학교를 졸업하고 비해기 회사에서 10년 간 전투기와 폭격기의 설계도를 그렸지만 피타고라스의 정리를 쓴 적은 단 한번도 없다. 또 그 후에는 음향항과 의학기기를 10년 했는데, 이 사이에도 한번도 없다. 그리고 로켓 개발 10년 사이에도 없엇다. 또한 조직공학 연구소에서 10수년 있는 동안 전혀 한 번도 없다. 그것 뿐인가, 초등학교에서 배운 산수, 중학교의 대수, 기하 고등학교의 미분 적분에서부터 대학의 수학까지 현재까지의 일에서 실제로 사용한 경험이 없다.
한번도 없다는 것은 조금 극단적이고, 2번은 있다. 한 번은 공학 박사학위 논문이었는데 '수학을 넣지 않으면 학위를 받을 수 없다'고 해서이고, 또 한 번은 대학에서 강의를 할 때 '수학을 넣지 않으면 학생들을 바보로 만드니까'가 이유였던 2가지 기회 뿐이다.
선입견으로 비행기 설계에서는 수학을 사용할 것이라고 말하지만, 비행기 설계의 좋고 나쁨은 그 비행기에 타는 파일럿에게 좋은지 아닌지가 최대의 포인트이다. 조종사들이 어떤 비행기를 조종하고 싶어하는지 그걸 발견하는지 발견하는 것이 설계의 가장 큰 뜻이고, 수학 등을 억지로 사용해도 훌륭한 비행기는 생길 수 없는 것이다. 따라서 왜 그렇게 몇천 시간이라는 시간에 걸쳐 수학을 공부해야하는 것인가 이해하기 어렵다.
(중략)
수학을 하면 머리가 좋아진다는 것이다. 이것은 일을 절차대로 생각하게된다는 얘기인데 찬성할 수 없다. 경험상으로 대학교수회의 때 수학과 교수의 의논이 가장 사리에 맞는 이야기였다고 생각할 수 없기 때문이다. 만약 만에 하나 수학을 하여 머리가 좋아진다면 미국의 대통령도 수학을 전공하는 사람이 되어야하지 않을까.
(중략)
수학은 과학이 아니다. 기술도 아니다. 오히려 철학 분야에 들어가야 하는 것으로 뛰어난 수학자의 업적을 보면 수학이라는 무기질보다도 예술적인 향기조차 풍기는 느낌이 든다. 산수, 수학을 공부하는 것은 절대 불가결하게 필요한 단계를 하나하나 올라가는 과정을 가르키기 위한 것이다. 예를 들어 3+6은 얼마인가하면 양손의 손가락을 이용해 9를 답한다. 그런데 다음으로 56이면 양손의 손가락으로는 부족해진다. 아이는 !?라고 하다가 발가락이 있다는 것을 곧 알 수 있다. 이 \textbf{'오를 수 있다', '됐다'}는 것이 상당히 중요하다. 할 수 있으면 기쁘고 자신감도 생긴다. 또한 자신감이 생기면 의욕이 생긴다. 즉, 어떤 단계라도 하면 된다.
오른다는 적극적인 성격의 인간을 만들기 위해 수학이 있는 것이다. 말하자면 머리를 좋게 하기 위한 방편이다.
\vspace{5mm}

인용하다보니 수학공부의 의의까지 넣었는데 저는 저 생각에 동의합니다.
역설적인 이야기지만 수학공부를 많이 해야 들어갈 수 있는 학과에서는 정작 수학을 쓰는 일은 별로 없습니다.
그러면 수학을 왜 크게 반영하느냐. 그건 위에 인용한 대로입니다. 수학 자체가 좋은 머리를 보장하는 건 아니나
수학공부가 \textbf{좋은 머리를 만들기 위한, 일종의 계단식 상승의 방편}이 될 수 있기 때문입니다.
공부하면서 좋은 머리로 상승해 온 학생이 아니면 어려운 수학문제는 풀 수가 없습니다.
수능에서 수학을 제외한 나머지 과목은 미친 듯이 암기시켜서 고득점을 보장받을 수 있습니다.
그러나 수학만큼은 그게 먹히지 않습니다. 그래서 정말 '좋은 머리로 성장해 온 학생'인지 확인할 수 있고,
아울러 입시의 공정성도 일정 부분 보장할 수 있다고 할수 있습니다.
\vspace{5mm}

수학공부를 열심히 하는 건 그 문제가 그대로 출제되어서가 아니라,
그런 문제들을 풀면서 \textbf{'머리'를 만들어가기 위해서}입니다.
3+6을 손가락으로 세다가 5+6을 제시받은 아이는 발가락을 쓰겠죠.
그런데 그 다음 30+14라고 하면 엄마아빠형누나친구의 손가락발가락을 다 동원하다가
그것들을 숫자로 추상화할 수 있다는 걸 알게 되면 비약적인 발전을 하게 되는 것입니다.
\vspace{5mm}

물론 이런 식의 문제는 수학에만 있는 건 아닙니다. 국어에도 영어에도 과학에도 있습니다.
다만 수학이 그 점에서는 더 적합하다... 정도이죠. 그런데 바로 이게 학습자의 머리를 좋게 하는 것입니다.
\vspace{5mm}

그런데 입시에 최적화한다고 이런 '계단식 상승'을 제외하고 나올 것만 공부한다.... 당장 문제는 맞춥니다.
그러나 분명 벽에 부딪쳐서 못 뛰어넘습니다. 문제를 어떻게 푸는지 그 방법을 알지, 머리가 좋아진 게 아니기 때문입니다.
자기가 모르는 게 조금만 섞여도 거기서 생각을 하지 못 합니다.
수학문제를 잘 푸는 친구들은 자기가 모르거나 난해한 문제도 스텝바이스텝으로 추리하면서 실마리를 찾아갈 수 있는 사고를 할 줄 압니다.
\vspace{5mm}

입시에 나오지 않는다고 어려운(?) 증명을 할 필요가 없지 않느냐는 건 이걸 모르는 것이죠.
물론 과한 증명을 할 필요는 없다.. 가 아니라 사실 그걸 가르쳐줄 선생도 없을 것입니다만
교과서에 나온 공식이나 성질이 어디서 비롯되었느냐하는 증명을 해보는 건 필요합니다. 그 과정에서 머리가 좋아지기 때문입니다.
수능 수학은 그 정도가 아니기 때문에 그렇게 할 필요가 없다.... 그게 어디 정해지기라도 했답니까.
\vspace{5mm}






\section{잔인한 4월}
\href{https://www.kockoc.com/Apoc/706818}{2016.04.01}

\vspace{5mm}

11월부터 공부해 온 사람들은 이제 지쳐서 공부하기 싫은데도 관성 때문에 공부하고 있을 때임.
이런 분들은 최대한 버티다가 불꽃놀이 시즌 막판에 좀 쉬고, \textbf{수면}시간을 평소보다 늘려주시고(6시간이면 7시간 잔다거나)
그래서 5월 초까지 설렁설렁하셔도 됩니다. 사실 그래야합니다, 5월은 어차피 6평 때문에 긴장 바짝할 터인데 그 체력과 정신력 비축해야하는지라
\vspace{5mm}

보통은 수면시간을 줄이라하는데 왜 늘리냐 $-$ 거기 답변은 간단합니다.
공부 외 활동에서 우리 정신 건강/육체 건강에 도움이 되는 건 사실 수면 뿐입니다.
잠을 개운하게 잘 자고 나면 피로와 스트레스가 풀려서 공부를 기분좋게 할 수 있습니다.
그 수면시간을 줄여서 자기학대를 하는 건 멍청한 짓이죠. 멍한 상태에서 공부할 테니까.
물론 가을이 되면 수면시간을 저절로 줄이게 됩니다. 그러니까 봄날에 괜히 피로 쌓아두지마시라는 이야기.
\vspace{5mm}

그리고 공부를 2개월 이상 한 사람은 독학하는 사람은 학원 다니고 싶어할 테고, 학원 다니던 사람은 독학하고 싶어할 것인데
그게 정말 진지하게 공부방법을 반성하는 것인지, 아니면 지금 공부하는 게 힘들어서 뇌에서 핑계대는 것인지 확인해보시길 바랍니다.
물론 독학으로 공부를 했는데 공부시간이나 학습량이 나오지 않았다면 진지하게 학원으로 갈아타야합니다.
다만 "학습량"이 많이 나오는데도 그걸로 스트레스받아서 공부하기 싫다 힘들어 그만두고 싶어라는 메시지가 주인을 기만하는 경우가 있습니다.
\vspace{5mm}









\section{개정수학이 개정 전 수학과 다른 것.}
\href{https://www.kockoc.com/Apoc/708462}{2016.04.02}

\vspace{5mm}

궁극적으로는 경쟁을 더 가중시킬 듯.
\vspace{5mm}

개정 전 수학은 고 1 때부터 삼각함수와 순열, 조합을 박아넣어 그 때부터 다수의 수포자를 양산할 수 있던 데다가
행렬, 지수로그, 순열, 방정식과 부등식, 함수의 극한과 연속, 미분 곳곳이 초심자에게는 대단히 진입하기 어려운 구조였는데
\vspace{5mm}

현행 과정은 삼각함수와 지수로그함수를 미적분 2로 밀어넣어버렸고
초월함수나 로그함수 몰라도 일단 미적분1까지 끝낼 수 있게 해놓은 구조라서 수포자 양산이라는 비극을 초래하진 않을 듯.
다시 말해서 이전 과정은 원래 수학을 잘 할 수 있는데도 교과 과정에 치여서 중도포기하는 비극도 없지 않았으나
현재는 그렇지 않다는 것이죠.
\vspace{5mm}

그런데 이게 역설적으로는 경쟁을 더 빡세게 할 것입니다.
첫째로는 시험 출제는 어차피 경쟁이 좌우하는 거라서 교과 과정이 쉬워지는 것과 별 상관없다는 것.
둘째로는 수포자들이 줄어들기 때문에 경쟁이 심화되고 평가원으로서는 불수능 출제를 해도 무리없다는 것.
\vspace{5mm}

교과과정이 쉬워졌으니까 만만할 것이다... 라고 하면 없는 코를 빌려와서 다치게 생겼다는 것.
\vspace{5mm}

특히 이번 과정이 인상깊은 게 \textbf{'선행'이 무의미}합니다. 공부하거나 연구해보신 분은 알 것임.
수1, 수2를 제대로 하지 않고는 미적분 1을 못 하게 해놓았고, 미적분 1을 못 하면 미적분 2도 힘들고, 미적분 2를 못 하면 기벡도 못 하는 구조임.
선행한다고 진도 빨리 나아가보았자 별 실익이 없음. 고2 이거나 고2 올라갈 사랑이라면 학교 진도에 맞춰가면서 고난도 문제 푸는 게 바람직.
\vspace{5mm}

반면에 이전 과정은 선행을 하지 않으면 수포자되기 참 좋은 구조였죠. 교과과정 자체가 지나치게 눈높이가 높았음.
그래서 재능이 없어도 선행해서 감을 잡으면 이득을 보았고, 재능이 있어도 선행을 안 하면 교과과정에 치여 수포자되기 좋았는데
그런 부조리(?)한 일이 현행 과정에서 벌어지진 않을 것 같습니다.
\vspace{5mm}








\section{양민들을 위한 수학교재테크트리}
\href{https://www.kockoc.com/Apoc/717659}{2016.04.08}

\vspace{5mm}

초심단계
\vspace{5mm}

01 쎈수학 A형 + 수력충전이나 연개수문(선택사항)
02 쎈수학 B형 下, 中
03 RPM (고난도 문제 제외)
04 마플(마더텅 자이도 무방) 중난이도 문제
05 쎈수학 B형 上
06 RPM 전부 다 풀 것 $-$ \textbf{RPM 끝}
\vspace{5mm}

중간단계
\vspace{5mm}

07 일품수학 개념만 정리
08 풍산자 필수유형 고난도 빼고 풀 것
09 마플(마더텅 자이) 최고난이도 문제 빼고 4점짜리 절반 풀기
10 일품수학 1등급과 수능문제 절반 풀기
11 풍산자 필수유형 고난도 문제 절반 풀기
12 EBS 올림포스 고난이도 빼고 다 풀 것.
13 일품수학 1등급과 수능문제 다 풀기 $-$ \textbf{일품 완료}
14 풍산자 필수유형 고난도 문제 다 풀 것 $-$ \textbf{풍필유 완료}
15 EBS 올림포스 다 풀 것 $-$ \textbf{올림포스 완료}
\vspace{5mm}

고수단계
\vspace{5mm}

16 수학의 바이블 고난이도 문제 빼고 정리 $-$ 단, 이미 아는 문제는 스킵하고 읽어도 됨
17 실력 정석 : 예제와 유제는 그냥 읽고 기본 연습문제 다 풀 것
18 블랙라벨 스텝 1 풀기
19 마플 최고난도 문제 풀기 $-$ \textbf{마플 완료}
20 \textbf{수학의 바이블} 고난이도 문제 다 풀 것 $-$ \textbf{수학의 바이블} \textbf{완료}
21 실력정석 실력문제 : 절반 정도 시도해볼 것. 단, 모르는 경우 별표치고 답지보고 정리
22 블랙라벨 스텝 2 절반 풀기
23 그동안 소박맞은 쎈수학 C 스텝 절반 풀기
24 실력정석 실력문제 : 나머지 절반도 시도, 역시 모르는 문제는 별표치고 답지보고 정리 $-$ \textbf{실력정석 형식적 완료}
25 블랙라벨 스텝 2 나머지 절반 풀이
26 쎈수학 C 스텝 나머지 절반 풀기 $-$ \textbf{쎈수학 완료}
27 블랙라벨 다 풀기 $-$ \textbf{블랙라벨 완료}
\vspace{5mm}

고수단계까지 완료 후, 자기가 틀리거나 별표친 문제는 다시 풀이 읽고 정리해 볼 것.
위와 같은 스텝은 1단원마다 해도 되고, 혹은 3$\sim$4단원별로 해도 됨.
가령 미분을 다룬다면 미분계수만 저렇게 돌려도 되고, 아니면 미분 전체로 다 돌려도 좋음.
\vspace{5mm}

그 이후
\vspace{5mm}

중간고수 점검단계
\vspace{5mm}

28 교과서 구해서 풀어볼 것(선택적 : 안 해도 무방)
29 기본 개념과 원리 증명해볼 것 : 미적분학의 기본정리, 지수의 확장 밑의 축소, 이항분포 공식 증명 등
30 일등급 수학 그냥 쫙 풀기 : 일등급 수학이 가장 어려워서가 아니라, 난이도 배분을 잘못해놓은 게 있어서 이 때 푸는 게 걍 유리함.
\vspace{5mm}

최강고수 단계
\vspace{5mm}

이스터에그 :  콕콕의 모쏠 아무개를 납치해 자료 내놓으라고 한 뒤 그걸 풀 것
31 고1 수학이 제대로 안 된 경우 위와 같은 과정으로 고1 수학 정리할 것 : 단, 4월부터는 권하고 싶지는 않음.
32 수리논술기출 구해서 풀어볼 것 ; 정답에 집착하지 말고 일단 푸는 훈련을 하는 게 좋음.
33 과거 본고사 문제 구해서 풀어볼 것 ; 구하기 어려우면 천일수학을 구매한 뒤 도전.
\vspace{5mm}

그런데 다수가 고수단계까지도 가지도 않고 그냐 실모 봐야하느니 강의 들어야하느니 그러고 있다는 게 함정.
어떻게 공부해야하지 말고 저기 적힌 단계별로 고수단계까지 다 완료해놓고 오셈.
고수단계까지 완료하고 나면 \textbf{질문을 하는 입장이 아니라 받는 입장이 된다는 게 함정}
\vspace{5mm}

저 단계로 해도 약 30회독임. 아무리 바보여도 공부가 안 될 수가 없음.
초고수 단계까지 간 다음에야 경문사 책에 각종 서적을 읽으면서 "중요한 것은 사고"라는 것을 알게 될 것임.
이 단계까지 가면 패턴화고 뭐고 필요없음, 본인이 패턴을 만들고 있을 것임.
\vspace{5mm}

그리고 저런 접근은 다른 과목도 마찬가지입니다.
제 입장에서는 저런 식으로 스케줄 짜서 공부하는 건 '상식'이었는데, 요새 친구들은 그게 상식이 아니라는 걸 뒤늦게 알고 놀랐음(...)
그게 인강의 폐해에다가 야매교재 문제가 아닌가 싶음.
\vspace{5mm}

스킬? 심화개념?
저것까지 하다보면 본인이 알아서 깨닫고 기억합니다.
저게 너무 많다고 해도 10회독 단계는 되도록 줄이실 것. 뭘 놓고 빼야할지 그딴 질문하는 인간은 걍 떨어지라고 저주할 것임.
그런 건 알아서들 하시길. 다만 올해 치는 사람들은 30회독은 무리일수 있으니 좀 빼야할 것임.
\vspace{5mm}

저렇게 공부하고 나면 이 판에서 사기치는 인간들 때문에 짜증날 것이고, 왜 진작 이렇게 안 했느냐에 지나간 세월이 한스러울 겁니다.
\vspace{5mm}

그리고 이런 질문 : 그럼 실력정석의 실질적 완료는?
\vspace{5mm}

그거야 사람에 따라 다릅니다. 실력정석이 좋은 점은 문제를 잘 선별해놓았단 겁니다 $-$ 물론 고수들을 위해서.
실력정석은 실력이 안 되는 친구가 처음에 보면 기 빨아먹혀 망합니다. 그래서 10회독을 초과하지 않은 단계가 아니면 안 보는 게 낫죠.
그러나 해당 단원 문제를 정말 많이 풀어서 귀신에게도 그 내용을 설명할 정도가 되면 매우 좋은 책이 됩니다.
나중에 되면 앞에서 공부한 내용 대부분이 실력 정석으로 압축정리가 저절로 될 것입니다. 그걸 실질적 완료라고 평하겠음.
\vspace{5mm}

저 정도 공부하면 당연히 점수가 나올 수 밖에 없지 않느냐 할 것입니다. 예, 맞아요. 다들 저 정도는 공부해야합니다.
당연히 저런 학생들을 시험쳐서 뽑는 학교가 실적이 좋겠고
저런 학생들을 이용해서 자기 교재가 실적이 좋다고 하는 야매들도 많은 겁니다.
\vspace{5mm}

저 과정의 의의는 일반 양민들이 할 수 있는 것이란 겁니다. 저거도 힘들면?
그럼 초심 단계의 문제집들을 늘려서 회독수를 늘리거나, 저 단계에서 인강을 선별적으로 들어주면 됩니당.
\vspace{5mm}






\section{올해 시험치는 분들을 위한 테크트리}
\href{https://www.kockoc.com/Apoc/719082}{2016.04.08}

\vspace{5mm}

과정은 아래 준해서 하되 정말 양 줄여주면
\vspace{5mm}

\item 1. 마플(마더텅, 자이도 괜찮음) $-$ 필수
\item 2. 쎈 $-$ 필수
\item 3. 급품벨 $-$ 하나만 보면 좋음(권하자면 일품)
\item 4. EBS 수능특강 $-$ LV 2, 3만 발췌해서 풀 것
\item 5. EBS 수능완성 $-$ 기출 빼고 풀 것
\item 6. EBS N제 $-$ 작년에 준해서 보자면 가성비 좋을 것이므로
7. EBS 올림포스
8. 실력정석 $-$ 연습문제만 발췌독할 것, 실력정석이 싫으면 숨마쿰을 보아도 좋음.
\vspace{5mm}

이렇게만 하시길.
양민들을 위한 글은 고2나 내년 시험 노리는 사람 용이고
위 1$\sim$8은 올해 시험치는 사람들의 최저가이드라인임.
\vspace{5mm}

저렇게 다해주고 사설강의에서 '기하와 벡터'와 '확률과 통계' 최상위 문풀강의만 하나 들어서 테크닉만 얻으면 될 것임.
\vspace{5mm}






\section{콕콕에 자주 들어와서 공부를 못 하겠습니다라는 분들을 위한 과제}
\href{https://www.kockoc.com/Apoc/719106}{2016.04.08}

\vspace{5mm}

그 분들은 행동양식을 바꿔야 하겠음.
일단 인터넷 접속이 너무 쉽다는 게 문제이온데
매일 공부한 교재 페이지의 사진을 \textbf{3장 찍어서 일지에 올리는 것으로 약속하시길 바랍니다.}
이 경우 하루라도 빠진다면 그건 공부 안 했다고 본인들이 실토하는 것이니 개망신이고  저 사진을 올렸다는 건 공부했단 증거니 인터넷 접속을 해도 되겠죠.  일지도 총회 이상은 '주간 일지' 작성으로 최적화하고  매일매일은 그냥 기록하는 것보단 맛폰으로 공부한 걸 찍어서 '갤러리' 게시판처럼 올리는 게 더 좋아보입니다.  그렇지 않고 인터넷 접속을 자주해서 공부 안 하는 양 해서 수능 끝나면 돌아온다.... 이거 절대 안 지킵니다.  농땡이 보존의 법칙은 어김없음, 콕 안 들어오면 그 시간에 딴 데 가서 노닥거리고 있을 게 뻔하죠.









\section{변명}
\href{https://www.kockoc.com/Apoc/732675}{2016.04.16}

\vspace{5mm}

\textbf{"처지가 불우해서 공부를 못 했습니다"}
\vspace{5mm}

대학도 마찬가지이지만 어디든 여러분들의 변명을 안 듣습니다.
돈 처발라서 높인 실력일지라도 \textbf{점수만 잘 나오면 우대해주는 것}이고
리어카 끌고 가족부양해서 공부하지 못 해서 \textbf{점수 안 나오면} 걍 씹어버립니다
그게 사회입니다... 가 아니라 인생 전체가 그렇습니다.
\vspace{5mm}

거꾸로 입장 바꿔서 님들이 물건을 살 때 악독한 놈이 만들지라도 그 품질 보지,
그럼 착하고 불우한 사람이 만들었는 데 영락없는 불량이다라고 하면 사주겠습니까.
\vspace{5mm}

가끔 상담할 때 "저는 노력했는데 왜 안 됩니까."란 질문 많이 받죠.
\vspace{5mm}

그냥 말하께요. 본인은 노력하는 수준을 너무 낮게 보아서 그런 겁니다.
공부 잘 하는 애들이 가령 1만시간까지 채워야한다고 본다면, 공부 못 하는 친구들은 300시간을 해놓고 그것도 많이 했다 생각합니다.
다소 과하다 생각하지만 객관적 기준을 대자면, 한 과목당 3000시간을 누적해서 투자해보았냐 그걸 재어보면 됩니다.
\vspace{5mm}

저럴 각오 없는데 자존심 챙긴다라면, 그냥 힘든 길 가실 필요 없습니다.
왜 준비도 안 되어 있고 각오도 안 되어있으며 병아리 오줌만큼 노력하고 죽겠다고 하려는데
최상위권 수준으로 가겠다라고 하는 사람들이 많은지 모르겠습니다만
결론부터 말하면 이런 사람들은 민주주의적으로 얘기해보았자 소용없어요. 죄다 자기중심적으로 가기 때문에.
좋게 달래면서 "그러니가 지금 졸라 하세요, 버티기라도 해야합니다"라고 해도 자기 자존심 때문에 붕괴되고 그러다가
나중에 시험결과 뜨면 또 자기가 잘못했다는 걸 알지만 그걸 인정 못 해서 타인 원망이나 합니다. 그게 그 사람들의 \textbf{그릇}입니다.
\vspace{5mm}

그 그릇을 객관화해서 고치는 거야 물론 타인 입장에서는 쉽지만 본인 입장에서는 힘든 일이겠죠.
그런데 그게 힘드니까 성공하는 사람들도 \textbf{소수}인 겁니다. 누구나 쉽게 극복하면 개나소나 다 성공했겠죠.
망하기 좋은 패턴 중 하나입니다. 이걸 알면 당사자가 알아서 뜯어고치는 수 밖에 없어요.
이런 사람들은 객관적으로 지적하면 자기를 비난한다고 하는 경우가 많은 걸 저도 아는데 답이 없습니다.
본인이 해온 노력이 성과를 못 맺은 건 그 성격 때문이니까요.
\vspace{5mm}

공부괴물들이 공부가 지겹다고 하는 건 다른 차원입니다. 얘들은 그냥 정말 지겨워서 지겹다고 하죠 $-$$-$
이 친구들은 어려운 문제 던져주면 눈을 빛내면서 결국 풀어댑니다.
양민들이 일주일에 할 것을 하루만에 끝내버리고, 심지어 풀이과정 50줄 쓸 것을 5줄에 쓰는 애들이 과연 없을 것 같죠?
그런데 얘들은 머리가 좋은 것보다는 틀이 정말 잘 잡혀있습니다.
자존심은 생각치 않고, 자기에게 도움이 되는 것은 필사적으로 자기 것으로 만들고 이야기합니다.
예컨대 여기 쓰는 칼럼의 내용을 자기가 생각한 양 말하는 경우도 많아요(즉, 이미 자기철학으로 흡수해서 좋은 건 다 한다 그것이죠)
\vspace{5mm}

양민들은 물론 양민의 방식으로 소박하게 가야합니다만, 자기들이 이루는 목표를 성취하려면 괴물들과 싸워야합니다.
자기가 이런저런 사정이 있어서 공부를 못 했다... 라는 변명이 괴수와의 경쟁에서 하나라도 참작될 수야 없죠.
이런 걸 모르면서 힘들다라고 한다면 그런 사람은 '입시'든 어떤 '경쟁'이든 다시 생각해보셔야합니다.
경쟁을 할 때에는 반드시 비정상적인 "괴물"을 상정해놓아야합니다. 어느 분야든 그런 괴물은 최소한 한명은 있습니다.
자기가 평범하게 시작해서 그 괴물을 사냥하려면 어느 정도로 렙을 올려야하는지, 그리고 어디까지 스트레스 받아야하나 정도는 감잡아야겠죠.
그래야 실제로 그 괴물들과의 경쟁 모드까지 가더라도 불연속을 겪지 않고 승리해나갈 수 있습니다.
\vspace{5mm}

가장 좋은 건 본인이 그런 괴물이 되는 것인데
괴물의 요건 중 하나는 \textbf{변태}입니다.
공부하는 고통 자체에서 쾌감을 느끼는 것이지요.
양민들은 공부 스트레스에서 고통을 느끼고 신음해서 그만둔다면,
괴수들은 공부하는 스트레스와 중압감 자체를 즐기기 때문에 사기캐가 되는 것입니다.
이건 의식적으로 그렇게 느끼는 게 아니라, 하다보면 사람이 그렇게 \textbf{변해}버립니다.
\vspace{5mm}

교재 차이가 필요없다는 게 사실 이것입니다. 중요한 건 어떤 교재를 보느냐가 아니라 저런 \textbf{괴물}이 되어가는 것이라서리
이야기해보면 괴물인 사람이 있고 아닌 사람도 있습니다. 어느 정도는 그게 티가 납니다요.
그런데 양민들은 괴물이 되는 것을 거부합니다. 그 사람들 입장에서야 그게 당연해보이겠죠.
그래서 어린 시절에 상급학교에 진학하는 게 장기적으로 좋은 전략일지도 모르죠. 어린 시절부터 괴수들을 겪다보면
그 괴수들의 눈높이가 정상이 되므로 본인도 그런 괴수가 되려고 자발적으로 노력할테니까요.
\vspace{5mm}






\section{다시 적는 인강에 대한 비판적 접근}
\href{https://www.kockoc.com/Apoc/732693}{2016.04.16}

\vspace{5mm}

처음에 개념서를 본인이 읽고, 그 다음 기초문제집들을 최소 2권 이상은 돌리고, 오답정리하고
기출 풀어보고 깨져본 다음에 인강을 발췌해서 들으시길 바랍니다.
\vspace{5mm}

처음에 오답정리를 해보고 깨져본 다음에 듣는 인강의 흡수율이 좋지
이런 것 안 하고 인강만 계속 돌리면 절대 실력이 \textbf{안 늘어납니다.}
인강을 듣고 있을 때야 강사가 신기한 정보들을 제공해주고 문제도 예술적으로 푸니까 공부가 되는 걸로 생각하는데
실제로 그건 \textbf{자기 실력이 아닙니다.}
실력을 키우려면 본인들이 직접 문제를 읽고, 그것들을 종이에 써보면서 정리해야합니다.
\vspace{5mm}

인강을 들을 때 공부가 된다고 하는 건 일종의 '마취 효과'와 비슷합니다.
들을 때야 고통도 안 느껴지고 흡수가 좋다고 생각해서 공부가 된다고 하나
문제는 그런 건 정착되기 어렵단 겁니다. 기분좋게 들은 건 뇌에서 기억을 잘 하려하지 않습니다.
남는 건 강사의 잡답이나 농담, 그리고 몇가지 신기한 스킬 정도일 것입니다. 투자한 시간에 비하면 효율이 정말로 낮아요.
\vspace{5mm}

강의가 좋다고 찬양하는 사람들은 많아도, 정작 그 사람들의 실적이 좋은 경우는 별로 못 보았습니다.
정반대로 실적이 좋은 사람들은 수험기를 보면, 인강을 안 들어도 성공했겠구나 느껴질 정도로 스스로 문제풀이를 많이 한 케이스입니다.
그럼 인강만 줄창 듣고 문제풀이를 안 한 케이스가 성공한 경우? 제가 아는 한 단 한건도 없습니다.
\vspace{5mm}

공부는 뇌를 길들이는 과정입니다.
뇌는 생존, 공포, 섹스, 고통 등에 관한 것은 정말 1번만 해도 잘 기억합니다. 그게 우리 유전자의 명령이니까요.
인강을 편히 듣는다는 건 그 정보가 뇌가 선호하는 것과 거리가 멀다는 겁니다. 그래서 들을 때는 기가 막혀도 그게 실력으로 이어지지 않아요.
정반대로 어떤 문제를 푸는 것이나 특정 지식이 자신의 절박한 상황에서 벌어졌다, 그건 악몽처럼 끝까지 기억하게 됩니다.
\vspace{5mm}

더 재밌는 사실은 인강을 들은 건 정착이 안 되지만, 자기가 남에게 가르치는 건 정착이 잘 된다는 것입니다.
외형상 보기에는 입력과 출력의 차이인데, 실제로는 출력을 해보는 게 더 도움이 된다는 것이죠.
왜 그럴까, 그건 가르치는 과정 자체가 쾌감이 있기 때문입니다. 이것 역시 유전자의 명령으로 가본다면
가르친다는 건 남보다 우월한 지위에 서있는 것이니 뇌는 이걸 선호할 수 밖에 없다는 걸로 해설할 수 있을 것입니다.
\vspace{5mm}






\section{수학 사교육이 학생 발목을 잡는 경우}
\href{https://www.kockoc.com/Apoc/734042}{2016.04.18}

\vspace{5mm}

\textbf{A란 문제를 풀 때에는 반드시 B를 써야한다.}
\textbf{해당 문제 패턴에 쓰이는 스킬과 공식을 암기해라.}
가장 답없는 게 저 케이스다. 왜냐면 자기가 공부를 하고 있고 각종 스킬을 있으니 잘 되고 있다고 생각한다.   그러나 저것이 자기 발목을 잡는 것임을 모른다.   수학 문제는 두가지다. 과거의 문제, 그리고 현재(지금 치는 시험)의 문제.   학교 내신이나 교육청 모의는 과거의 문제 비중이 높다. 그래서 저런 정석, 스킬 중심의 접근을 하면 점수가 오른다.   그러나 과거의 문제 연연하지 않는 새로운 문제야말로 자신의 운명을 결정하는 것이다.   수학이 힘들다고 하는 이유는 다른 게 아니다. 그냥 접근방법이 잘못 되었기 때문이다.   특정 문제에는 특정 스킬을 무조건 써야한다고만 배우지, 그걸 \textbf{왜 써야하는지를 배우지 않는다. 아니 가르칠 사람도 별로 없을 것이다.}   개인적으로도 이런 게 궁금해 여러 인강을 듣고 책을 찾아보았지만 해답은 지금 지진으로 고생하는 그 나라 책에 있었다.   이런 스킬암기에 주력하게 되면 본인이 문제해결능력을 상실해버린다.   아무 것도 모르고 도구도 최소화되어야 본인이 문제를 분할하고 조건을 분석하면서 생각이라는 것을 한다.   하지만 학원이나 교재에서 가르쳐준대로 $\sim$ 만 쓰면 된다라고 하면, 자기도 모르는 사이에 문제를 푸는 게 아니라 그냥 암기해버린다.   물론 현실적으로는 암기를 피할 수 없을 것이다.    그러나 수학공부의 목표는 문제를 암기가 아니라, 문제를 풀기 위한 \textbf{머리를 만드는 것}이다.      노력을 하는데 자꾸만 모르겠다... 라고 하는 케이스는 가만보니 저런 식의 암기형으로 머리가 맛이 간 케이스다.   이런 애들은 생각훈련, 생각하는 방법에 관한 강의 같은 것으로 치유하는 게 좋겠으나 유감스럽지만 그런 책도 강의도 찾기 어렵다.   게다가 그런 암기형 패턴을 폐기하지 못 한다.   교과서가 좋다는 평가를 받는 건 교과서가 정말 좋아서가 아니라, 스킬이 덜 실려있기 때문이다라는 게 웃고 넘길 얘기만은 아니다.





\section{국어나 영어의 스킬적 접근이 문제인 경우}
\href{https://www.kockoc.com/Apoc/735647}{2016.04.19}

\vspace{5mm}

고교수학의 문제풀이적 접근은 폴리야로 집대성된다.
\vspace{5mm}

https://en.wikipedia.org/wiki/George_P$\%$C3$\%$B3lya
\vspace{5mm}

우정호 교수님이 번역한 폴리야의 책을 읽으면 어떻게 수학문제를 풀어야하는지 고전적으로 알 수 있다.
다시 말해서 이 경우는 문제풀이의 접근이 이미 일반론적으로 정리되었다는 것이다.
그 다음 문제풀이 접근법은 일본인들의 책을 구해다 읽으면 된다(번역된 것들이 있다)
\vspace{5mm}

주의해야할 사실은 문제풀이 접근법은 문제풀이 스킬과는 다르단 것이다. 과장해말하면 접근법과 스킬은 상극이다.
스킬 위주의 공부는 무조건 스킬을 써야한다는 강박관념에 빠진다. 그래서 스킬이 안 먹히는 문제풀이가 나오면 멘붕해버린다.
문제풀이 접근법은 초딩산수를 쓰더라도 그 어려운 문제를 논리적으로 분해해나가는 과정인 것이다.
\vspace{5mm}

국어와 영어에 대해서는 스킬이 중요하다는 것이 작년까지의 생각이었는데 이걸 수정해야겠다는 생각이 들었다.
2000년대 후반까지는 국어나 영어의 독해나 문풀 스킬이라는 게 꽤 유용했다.
그런데 지금 수험생들이 그 스킬을 모르느냐 하면 그건 아니다. 사실 이것도 '과다'하다.
그럼 이건 스킬이 문제가 아니라는 이야기다.
\vspace{5mm}

왜 그런가 생각해보니 매년 줄어드는 게 있다. 그건 바로 '독서량'이다.
인터넷 속에서 살다보니 책을 읽지 않는다. 책을 읽지 않으니 다양한 텍스트를 접하지 못 한다.
알고 있는 텍스트가 없으니 창의력의 재료가 부족해지고 분석의 연습대상조차 없다.
이 상태에서 스킬을 알아 보았자 소용이 없다. 왜냐면 처음보는 지문이 나오면 스킬을 쓰기 전에 뭔 소리인지 몰라서 포기해버린다.
\vspace{5mm}

스킬이 중요하다고 가르치는 사람들은 자기 세대가 독서량이 많았다는 걸 간과하고 있다.
무엇보다 가르치는 데 있어서 텍스트들을 하나하나 떠먹여주고 소화시켜주는 것은 상당히 품이 많이 드는 일이다.
독서량이 없어도 스킬만 가지고 풀 수 있다라고 하면 몸값이 비싸질 수 밖에 없다(어차피 입시결과는 책임지지 않아도 되기 때문이다)
\vspace{5mm}

"텍스트 소화량의 결핍"이 관건인데 문제는 이걸 현재 수험생들은 인식하지 못 한다.
애꾸눈의 나라에서는 애꾸눈이 정상이다. 절대적 독서량이 부족해도 다 똑같으니 그게 정상이라고 착각하는 것이다.
당연히 이런 체제에서는 어렸을 때부터 책을 많이 읽어온 애들이나, 하다 못해 기출지문 양치기를 한 애들이 점수가 나온다.
그러나 이런 애들은 소수이다. 무엇보다 지금 많은 독서를 해야한다니... 에서 다른 좋은 스킬이 없을까 고민하게 된다.
\vspace{5mm}

올해는 모르겠지만 내년 입시 준비하는 친구들은 정말 매주 책 한권씩은 읽는다는 강행군을 해야할지도 모른다.
수험 방향은 자기가 속한 경쟁집단의 약점을 보완하는 것이다. 현재 수험생들은 정말 책을 \textbf{지지리도 읽지} 않는다.
\vspace{5mm}

과장이 아니고 난 책을 읽지 않는 사람은 인간 취급을 안 하는 것은 아니라고 쳐도 그다지 존중하지는 않는다.
책을 읽지 않고 인터넷에 올라온 글(이 글도 마찬가지)만 보거나 사이트 게시판 가서 거기 글 보고 휘둘리는 게 병신이지 어디 사람새기인가?
어떤 문제가 있으면 다수의 쥐떼근성에 휘둘리지 않고, 자기가 추합한 정보나 느낀 바를 독서로 다져진 지성으로 스스로 가공해
누가 까다로운 질문을 해도 분명히 대답할 건 대답하고 모르는 것은 모른다고 해야 인간이지, 그렇지 못 하면 노예새기나 진배 없다.
\vspace{5mm}

좀 괜찬다 하는 사람들과 대화해보면서 질문해보면 오 이 사람은 책을 읽어왔군... 하면서 어떤 책을 읽었느냐 파악한다.
만약 그 책이 뻔하디뻔한 탑셀러라면 안심해도 좋다.
그런데 뭔가 말하는 내용이 예외적인 데다가 읽는 책도 대중적이지 않으면 눈을 비비고 다시 쳐다봐야한다.
책을 읽는 사람은 알 것이다. 사람이 가장 아름다울 때가 책을 읽을 때이고, 그 때 우리의 눈길도 그 사람보다 그 책을 향한다는 걸.
\vspace{5mm}

더 오버해서 쓰면 요즘 세대들이 헬조선하는 것도 웃긴 이야기가 그들이 까는 기성세대만큼 고생하는 것도 아니라고 보지만(586 제외)
무엇보다도 책을 안 읽기 때문이다. 만약 이 세대가 일주일에 책을 3권씩 읽고 부지런히 학습한다면 나이에 상관없이 내가 굽신거렸을 것이다.
그러나 맛폰질은 하면서 책을 안 읽으므로 '까도' 별로 후환은 없어보인다.
\vspace{5mm}








\section{어른들이 공부만 하라는 거}
\href{https://www.kockoc.com/Apoc/758269}{2016.05.03}

\vspace{5mm}

적어도 어른들 얘기 중에서 단 하나만 건지면 \textbf{"쓸데없는 짓 말고 공부나 하라"}
내 입장에서 나도 꼰대들의 메시지 중 참말과 거짓말은 구분하는데 저 말은 정말 진짜다.
무능한 젊은이가 정의감에 차서 움직인다고 쳐도 현실적으로는 별 쓸모 없는 경우가 대부분이며,
그런 정의감조차도 실제로는 오랜 사색에서 나온 진정한 철학이라기보단, 선동당하거나 혹은 성욕을 감춘 격정에 불과한 경우가 많아서이다.
\vspace{5mm}

우리가 공부를 하는 이유는 노골적으로 말하지만 \textbf{자신을 비싸게 팔기 위해서}이다.
이제는 대학을 졸업해 취업이라도 하면 정말 다행이라고 한다. 물론 앞으로는 취업의 개념조차 사라질 것이다.
이제는 일자리의 시대가 아니다 일거리의 시대다. 일거리를 스스로 찾아서 물고 와야하는 시대인 것이다.
일거리들을 물고오려면 본인이 비싼 몸이어야 한다.
\vspace{5mm}

그러면 사회 부조리가 벌어져도 침묵하고 공부만 하란 말입니까... 유감스러운데 이게 진리다.
본인이 능력이 없는 한 나서보았자 고기방패 빼고 뭔 가치가 있나.
일개 잡몹은 사회에서는 신경조차 쓰지 않는다. 네임드 히어로여야 그나마 신경써주는 척이라도 하지.
\vspace{5mm}

그러나 적어도 내가 관찰한 바로는 다들 사회 부조리에 항거하는 척 하지만 실제로는 호구가 되는 루트를 밟는 쪽이 많다.
특히 20대의 젊음이라는 건 그런 능력의 감가상각을 분식처리하는 점이 있다. 자기가 건강하고 젊음이 넘칠 때는 뭐든지 할 수 있는 양 생각한다.
10년도 지나지 않아서 곧 꺼질 텐데, 심지어 어떻게 사느냐에 따라선 5년도 못 넘길 수도 있다.
역산법적 사고 $-$ 자기가 서른살, 마흔살 ... 그리고 뒈지기 직전이라면 어떤 루트를 밟았을까 하고 가정해보는 방법으로 가면 답은 보인다.
나이먹을 수록 늘어나는 건 주름살과 후회일 뿐이다라는 말도 진짜다(반대로 머리털은 줄어든다)
\vspace{5mm}

자기가 대단하다라는 생각도 버려야하고 공부도 사실 별 게 아니라고 얘기해야 한다.
다만 자본주의 사회에서의 개인이 가질 수 있는 자본은 결국 능력으로 환원된다는 것을 알아야 한다.
사회에서 바라보는 우리의 외모는 학벌과 성적표이다.
사회에서 바라보는 우리의 육체는 실무적인 능력이다.
사회에서 바라보는 우리의 마음은 교양과 전문성이다.
\vspace{5mm}

가장 큰 착각은 공부하지 않아도 나답게 살 수 있다....  공부 없이도 내가 존재한다... 라는 것.
조금만 생각해보면 이거야말로 망상이다. 공부 없이는 우리는 털없는 원숭이에 고깃덩어리에 불과하다.
배우지 않는다면 소말과 다를 바 없는 가축이나 노숙자나 거지 창녀들과 별 차이도 없다.
그 자아라고 하는 것조차도 탯줄 떼고 교육을 받기 전에 부모 등이 주입한 패턴이다.
참자아라는 건 본인이 경험하고 배우면서 학습한 그 자체이다.
\vspace{5mm}






\section{할 수 있기 때문에 인간이 아니다.}
\href{https://www.kockoc.com/Apoc/758394}{2016.05.03}

\vspace{5mm}

어른들이 하지 말라는 건 일단 안 하는 게 좋다. 그리고 왜 하면 안 되는 건가 스스로 생각하고 관찰하고 결론 내리고 생각하면 된다.
가장 간단한 건 자기의 아들 딸이 있어도 그걸 하게 허용할 것인가 자문자답하면 된다.
예컨대 포르노를 보고 히히덕거리는 사람이 그럼 내 딸이 야동을 찍는 것도 허락할 것인가... 생각하면 그냥 답이 나온다.
혹자는 이걸 가지고 가족을 언급하는 건 비겁하다고 항변하지만, 바꿔 말해서 자기 가족도 시키지 못 할 것이면 그게 문제가 아닌가 얘기하면 된다.
\vspace{5mm}

문제는 하지 말아야하는 건 일단 \textbf{저질러 버린 다음}에는 답이 없다는 것이다. 그래서 하지 말라고 강제하는 것이다.
자기 개인이 살인하고 간음하고 하는 것은 문제가 없다고 생각한다. 자기가 책임질 수 있다고 \textbf{멋대로 착각하기 때문}이다.
자기 자식이라면 그렇게 바라보지 않을 것이다.
\vspace{5mm}

아마 지금 20대들은 '우리는 $\sim$ 할 수 있다'라고 주입받았을 것이다. 그리고 그게 인간답다고 생각하겠지만 그건 틀린 얘기다.
왜 예수, 부처, 공자가 지금도 3대 성인인가. 사실 이들이 인간이었다면 지적 수준은 딱 지금의 고딩이었을지도 모른다.
그래도 종교적인 면을 떠나서 이 분들을 생각하지 않을 수 없는 건, 우리가 '인간'답다고 하는 것들이 다 이 분들의 가르침에서 왔기 때문이다.
그 이후의 인간적으로 산다는 건 이 분들 말씀에 주석을 다는 수준이라고 해도 지나친 얘기가 아니다.
\vspace{5mm}

그런데 이 분들의 가르침은 결국 "\textbf{하지 말라}"는 것이다.
할 수 있기 때문에 인간인 게 아니라, 하지 않기 때문에 인간인 것이다.
살인하지 마라, 간음하지 마라, 도둑질하지 마라.... 얼핏 보기에 사소해(?) 보이지만 이런 것들을 지키니까 인간 사회가 유지되고 발전해온 것이다.
저 분들이 위대한 건 "하지 말라"는 걸 가르쳤기 때문이다.
사실 그것만으로도 신(神) 대접을 받아도 좋다. 왜냐면 우리가 아는 역사가 제대로 쓰여진 건 그 가르침이 전파된 이후여서이다.
\vspace{5mm}

혹은 이렇게 반론할 것이다. 인류사는 자유를 쟁취함으로서 발전해 온 것이 아니냐고.
그런데그 원없는 자유는 문명 이전 야만 이전이 더 압도적이지 않았겠느냐. 그럼 그 시대가 이상향이냐.
"하지 말아야하는 것"을 지키면서 할 수 있는 것 하고싶은 것을 늘려왔으니까 자유로워진 것이다.
자유는 '하지 말라' 위에서나 성립될 수 있다.
\vspace{5mm}

그런데 그들이 장사하려면 저런 하지 말아야하는 걸 깨뜨려야 한다.
그리하여 똑똑한 어른들은 남의 자식에게 "한계는 없어, 너희들은 뭐든지 할 수 있어. 그러니 너희들은 뭐든지 파고 살 수 있어"라고 한다.
그래서 청소년들의 성도 상품화하고 노동력도 저렴하게 구입한다. 한편으로 그들을 협박해 온갖 상품을 팔아댄다.
이런 기본적인 것조차도 모르는 청소년들과 20대들이 환상 속의 기득권 탓을 한다라고 보이는 건 전혀 지나친 얘기가 아니다.
그런 사람들이 자기 자식들은 어떻게 취급하겠나.
\vspace{5mm}

가난만으로 모든 것이 보호받고 정당화되지 않는다.
가난하다는 사람들이 신나게 술을 마시고 돈을 마음대로 쓰는데 부자탓을 하는 건 이상하지 않나.
가난하다면서 하지 말라는 것을 즐기고 자기 학습을 게을리 하며 저축을 하지 않으면서 국가 탓을 하는 건 웃긴 것이다.
물론 이 글을 보고 자기는 빡세게 일하고 즐기지도 못 한다고 울분을 터뜨릴 분도 있을 것이다.
그런데 이런 사람들은 머지않아 곧 탈출한다. 그리고 자기를 이미 그 방종의 무리들과 차별화시킨다.
오히려 내가 보는 그들은 자기가 정말 가난하고 고통을 받고 힘들다라고 착각하지만 실제로는 즐길 건 다 즐기는 사람들이었다.
\vspace{5mm}





\section{왜 실패하는가}
\href{https://www.kockoc.com/Apoc/762686}{2016.05.06}

\vspace{5mm}

사람들은 말이지, 눈 앞의 푼돈 얼마를 위해서라면 웬만한 일은 다 견딜 수가 있다네.
부자들은 그 특성을 이용해, 평생을 시중받으며 안락하게 살지...
왕은 혼자서 왕이 되는 게 아니야. 왕이 혼자서 그 자리를 유지할 수 있다고 생각하나?
돈 따위는 필요없다는 천한 것들이 결속해서 반항을 하면 왕도 결국 사라지는 법일세.
하지만 가난한 자들이 왕이 되고자 돈을 바라면, 역으로 지금 있는 왕의 존재를 보다 견고하게 반석 위에 올려주지.
모두 그런 메마른 패러독스에서 빠져나오질 못해. \textbf{돈을 바라는 이상, 왕을 쓰러뜨릴 수 없네. 계속 매일 수 밖에 없지}.
왕도 폭동을 막기 위해, 다들 고만고만 윤택한 기분으로 있을 수 있도록 주의하고 있다네. 실제로는 얼마나 뜯어먹히고 있거나 말거나 말일세.
\vspace{5mm}

$-$ 도박묵시록 카이지의 진주인공 효우도 카즈타카 회장님의 말씀 $-$
\vspace{5mm}

관찰과 경험이 쌓이다보면 왜 망할 수 밖에 없나 하는 패턴들이 발견되는데
그 중 하나가 \textbf{우유부단}.
\vspace{5mm}

왜 우유부단이 문제냐면 이건 당사자가 스턴먹은 상황과 똑같아서 그렇습니다.
A할까 B할까 하면서 '시간은 계속 흘러가지'만 사실 아무 것도 선택 하지 못 하고 준비조차 하지 못 하고 그래서 기회를 날려먹죠.
속으로는 둘 다 가질 수 있을 거야라는 헛된 망상을 품습니다. 그리고 현실적으로 둘 다 얻지 못 하지요. 아니, 결국 전부 잃어버립니다.
\vspace{5mm}

"아냐, 운이 좋아서 둘 다 얻을 수도 있고 둘 다 할 수 있는 방법이 있을 거야"
\vspace{5mm}

물론 그 방법이 존재하는 경우가 대부분입니다. 그러나 그 방법도 '제때'에 공급되지 못 하면 아무 소용이 없는 것입니다.
당사자가 학과 공부도 하면서 수능까지 대비하는 방법이 없지는 않겠죠. 그런데 문제는 바로 그 때 당사자가 그걸 모른다는 것입니다.
그 방법이 존재한다고 한들 자기가 그걸 모르고 써먹지 못 한다면 아무 소용이 없어요.
\vspace{5mm}

손실을 인정할 때는 빨리 인정하고 정리해야합니다. 왜냐면 그래야 새로운 게임을 할 수 있기 때문이죠.
하지만 자존심 문제도 있고 무엇보다 자기가 낭비한다고 생각해서인지 그 손실을 인정하지 못 하고 죽은 자식 불알 만지는 사람들이 많습니다.
이것도 역시 장기적으로 보자면 '왕'과 '노예'가 갈라지는 분기점이겠죠.
\vspace{5mm}

자기가 애당초 의도하는 것에 도움이 되지 않는 건 황금빛이 나더라도 과감히 무시해버려야합니다.
예를 들어 시험보러가는데 길가에 1억 지폐 뭉치가 떨어져있더라... 하더라도 이걸 무시하고 가야한다는 것입니다.
물론 1억을 주으면 시험보는 것보다 더 현명한 선택일 수 있겠죠. 하지만 그로써 시험도 포기해야하고 앞으로의 선택이 문제가 됩니다.
자기의 목적을 망각하고 눈 앞의 이익만 좆는 거야말로 망하는 지름길이기 때문이죠.
손자병법에서도 이런 말이 나왔죠. 적에게 작은 이익을 줘서 유인하라.
그 말은 다시 말해 푼돈에 낚이는 사람은 뻔하다는 것입니다.
\vspace{5mm}

목표를 실행하는 건 전쟁과 같죠. 30:20로 싸우면 우리가 10이 남고 끝나는 게 아닙니다. 실제로는 9:4 비율 차이가 나서 우리가 20이 남는다고 하죠.
RTS 전략시뮬게임을 할 때에도 확인되지만 이기는 확실한 방법은 적보다 압도적으로 많은 병력으로 적시에 적을 치는 것입니다.
병력을 쪼개는 건 원칙적으로 미친 짓이지요.
\vspace{5mm}

마찬가지로 자기 일을 할 때에 안 그래도 시간과 체력이 제한되었는데 더 많은 걸 한다는 건 없는 병력을 더 쪼개는 것과 똑같은 짓이죠.
물론 계획을 세울 때는 자기가 그만큼 할 수 있다고 터무니없이 자신합니다만 그래서 성공한 사례는 제가 아는 한 없습니다.
한번에 여러가지 일을 해내는 사람은 이미 능력자이거나 아니면 타인의 도움과 협동 하에서 일을 효율적으로 처리하는 경우입니다.
\vspace{5mm}

좋지 않은 대학에 다니는데 수능치고 싶다... 그러면 정답은 간단합니다. 수능에 올인하는 것이죠.
다만 부모님 눈치가 있다라고 하면 학교 다니는 척 하면서 수능에 올인하겠죠.
그런데 여기서 꼭 '기왕 다니는 것 학교 졸업장도 따야지'라고 마음먹는 순간 패배는 확정되는 겁니다.
자기의 병력들이 수능이라는 거대한 군대와 싸우고 있는데 그 병력 중 일부를 뺀다.... 전쟁으로 치면 미친 짓이죠 사실
\vspace{5mm}







\section{인터넷의 수험정보}
\href{https://www.kockoc.com/Apoc/764299}{2016.05.07}

\vspace{5mm}

사실 불필요한 게 다수.
아무개 선생을 들어야 한다는 것은 정보가 아니라 공해죠.
쓸데없는 정보가 많으면 거기에 휘둘립니다. 그래서 피해 본 수험생들이 꽤 많아요.
어떤 인강을 들어야 한다거나 또 어떤 교재를 풀어야한다거나 하는 식의 정보를 가장한 광고글에 낚인 경우가 많죠.
\vspace{5mm}

그럴 바에는 나 자신의 "성격", "사고 스타일", "오답 유형", "취약 문제" 등을 기록하고 정리하는 편이 낫습니다.
아니 사실 수험판에서 굴러먹는 친구들을 보면 학원 홍보자 해도 될 정도로 참 이상한 분야까지 다들 알고 있으면서
정작 자기의 문제가 뭔지 전혀 모르고 있는 경우가 많습니다.
정보는 자기가 필요한 것만 얻으면 됩니다. 그리고 핵심적인 것만 있으면 되는 것입니다.
그것도 본인이 스스로 오프라인에서 탐문하고 곁눈질하고 캐내는 것이 좋습니다. 정말 중요한 건 인터넷에 올라오지 않죠.
\vspace{5mm}

쓸데없는 정보가 많으면 프로세스가 상당히 복잡해집니다. 프로세스가 복잡해지면 오류도 늘어나고 나중에 통제도 못 하지요.
개인이든 기업이든 구조조정과 혁신의 논리는 "불필요한 것을 없애고 핵심적인 것만 살리는" 것입니다.
핵심 프로세스만 잘 조합하고 싶으면 필요한 핵심 정보만 있어야 합니다.
그리고 그 핵심 정보는 상식과 통념과 '갈등'을 일으키는 것이어야합니다. 우리가 정보를 수집하는 건 '변화'를 알기 위해서이죠.
상식과 통념에 순응하는 정보는 변화를 가르쳐주지 않습니다. 그런 건 버려도 됩니다.
\vspace{5mm}

가령 지구과학을 선택해야 한다... 라는 건 작년까지는 매우 괜찮은 정보였습니다. 지금은 \textbf{과거}의 정보입니다만요.
지금 눈여겨보아야하는 건 지구과학 선택자가 많아졌다는 것이고, 그리고 이것만 믿다가 우리가 어떤 통수를 먹을까 하는 것입니다.
개정수학이 그 이전 수학보다 내용이 빠졌다라는 것 역시 과거의 정보입니다.
반면 개정수학 과정에서는 수포자가 나오기 힘들어서 문제가 더 어려워지는 경향이 있다는 게 핵심정보겠죠.
하지만 이것들도 시간이 지나면 쓸모가 없어집니다. 현실은 늘 바뀌죠..
\vspace{5mm}

그러니 수험생은 그냥 '하라는 공부만 하는 게' 정답입니다.
정보가 부족하기보다는 정보에 휘둘려서 \textbf{공부 방향을 못 잡아} 망하는 경우가 있습니다.
틀린 시험문제도 대부분 당사자의 교재나 학습커리로 커버되는 것이 다수입니다. 문제는 본인이 그걸 숙달하지 못 했다는 것이지만요.
\vspace{5mm}









\section{수학은 결코 쉬워진 게 아님.}
\href{https://www.kockoc.com/Apoc/767358}{2016.05.10}

\vspace{5mm}

양극화 사회에서는 평균적인 접근이 무의미함.
A는 1억을 벌고 B는 한푼도 못 번다고 하면 평균소득이 5천만원이 되는데 이게 정확한 자료라고 하지는 않음.
수학난이도가 쉬워졌다... 라는 건 걸러들을 필요가 있음.
그건 해당 응시생들이 어느 정도까지 공부했느냐하는 것까지 감안해야함.
\vspace{5mm}

20년 전에 저렇게 공부했으면 정말 신문기사에 나고 아주 천재라고 그랬을지도 모름.
그런데 이제는 중학교 때 정석을 다 마치고 간다라는 건 그렇게 놀라울 것도 아님.
선행학습은 이제 더 이상 문제가 아님, '할 놈은 거의 다 하고 있으니까'
여기서 유의할 건 상위권을 세분화시키면 그 내부에서도 격차는 매우 크다는 것이고
이건 IMF 이후에 태어난 세대부터 뭔가 더 당연하다고 여겨지는 바가 없지 않음.
\vspace{5mm}

과거 사람들보고 요즘 수학과 과학을 풀라고 해도 자신있게 풀 수 있을지는 의문임.
게다가 난이도보다도 더 힘들어진 건, 응시생들 수준이 높아지는 경향이 있다는 것임.
\vspace{5mm}

그렇다면 이미 수능 성적은 중학교 때부터 결정된다라고 해도 지나친 말이 아니게 되어버린다는 것.
왜냐면 그 때 다 끝내고 온 녀석들은 계속 공부할 테고 그럼 경험치나 레벨이 기하급수적으로 늘어나 격차가 벌어짐.
선행 안 한 친구들이 제 아무리 인강 듣고 실모 풀면서 간다고 해도 따라잡기 어려워지고
이 원인을 정확히 모르는 사람은 "머리" 이야기만할 것임.
\vspace{5mm}

물론 대치동에 가지 않더라도 방법은 있음. 학부모나 학생 본인이 중학교 때 저렇게 공부하면 되는 것이긴 함.
그러나 다수가 그래야 한다는 \textbf{생각을 하지 못 함}. 그냥 학교가 시키는대로 가야만 한다고 믿고 있기 때문임.
\vspace{5mm}

수1, 수2를 중학교 입학 전에 끝낸다는 건 남들보다 3$\sim$4년 앞선다는 이야기
그건 4$\sim$5수할 것을 미리 앞당겨 현역으로 끝내는 것과 똑같음. 즉, 이건 머리문제가 아니라는 것임.
누가 더 빨리, 많이 공부하느냐가 결국 좌우한다는 이야기임.
\vspace{5mm}











\section{수학강의에 휘둘리는 사람들}
\href{https://www.kockoc.com/Apoc/775832}{2016.05.16}

\vspace{5mm}

연구차 인강 많이 들어보았지만
개인적으로 정말 도움이 되었다 느낀 건 EBS 강의였음(...)
왜냐면 생각하는 법을 가르쳐주었고 그건 정말 나도 잘 써먹고 있음.
\vspace{5mm}

그리고 인터넷에서 강의 세세히 말하는 글은 걸러들어야죠. 공부하는 사람들이 그런 글 쓰겠나
그렇게 강의 소믈리에가 가능한 실력자면 좋은 대학 가서 공부하고 있어야지.
하지만 어떤가, 일지분석을 하건 콕콕 합격자 분석을 하건 '지독하게 공부한 사람'이 잘 나가지 강의도 사실 무차별한데
\vspace{5mm}

교재도 그냥 시중교재와 기출 신나게 풀고 교과서 연구하면 되는 것 아님?
지금 기출이라도 다 푼 사람 몇이나 있을까요.
작년 11월에 마플 잡고 그냥 달리라고 했는데 이거 다 푼 사람 있기나 하겠습니까만.
\vspace{5mm}

교과외적 내용 담은 게 소용이 있겠습니까. 교과외적 내용이라고 하면 그럴 바에는 대학수학까지 다 공부한 사람이 더 유리하겠죠.
중요한 건 교과외적 내용이 아니라 낯선 문제도 논리적으로 풀이하거나
특이점적 발상을 떠올리는 '생각하는 법'을 만들고 그런 습관을 들이는 것일터인데
정말 수학을 잘 하는 친구들은 교과외적인 것도 걍 신경도 안 써요. 수험에 필요한 것들은 어차피 저절로 유도되는 것들이고
교과외적 내용 암기해보았자 수능에서 교과서 내 내용으로도 참신하게 꼬아서 내면 그거 못 풉니다.
\vspace{5mm}

주어진 조건 도구가 20이라고 했을 때 100이라는 목적치를 달성하기 위한 80 $-$ 이건 스스로 생각하고 상상하고 전략짜는 걸로 채워야죠.
어떤 수학문제건 교과서적 기본원리를 문제의 답으로 연결시키기 위해서는 본인이 thinking을 해야하죠.
그런데 어차피 수학문제도 db화 가능한 점이 있다는 점에서 그걸 thinking이 아니라 pattern memorizing으로 해결하는 꼼수로 갑니다.
이게 내신까지는 어느 정도 먹혀요. 그런데 수능에서는 잘 안 먹힙니다, 왜냐면 수능은 꼭 새로운 걸 내니까.
\vspace{5mm}

매년마다 휘둘리는 호구들은 늘 생깁니다.
\vspace{5mm}

뭐 그건 중요한 건 아니고 올해 고2들이 참 역대급 실력자가 많다는 증거만 속속 확보되어서 걍 무섭다능.
아이엠에프 이후 세대들은 정말 계획적으로 임신, 출산, 교육해서 그런가. 부모의 지적 자산과 경험까지도 걍 상속.
시대적 환경이 꽤 무섭다는 걸 느끼죠.
\vspace{5mm}









\section{[임시 공지] 6평을 치르실 분들은}
\href{https://www.kockoc.com/Apoc/779984}{2016.05.18}

\vspace{5mm}

한시적으로 챗에서 6평 이야기를 나눌 수 있도록 조치가 있을 것 같습니다.
일지 작성하시는 분 분만 아니라 올해 6평 치르실 분들은 임시 공개챗방에서 전략을 이야기했으면 좋겠습니다.
\vspace{5mm}

개정교육과정 첫 6평이기 때문에 진득히 연구할 필요는 있어보이네요.





\section{수학교재 중간리뷰}
\href{https://www.kockoc.com/Apoc/780165}{2016.05.18}

\vspace{5mm}

즉흥적으로 쓰는 거라서리
\vspace{5mm}

아무튼 지금 와서도 수학 답이 없다 하는 부류를 위한 간략
\vspace{5mm}

\item 1. 수력충전 ★★★★☆ : 확실히 수포자 구제용 맞고 필요한 계산 드릴 다 들어감. 서울우유급
\item 2. 올림포스 ★★★★★ : EBS가 내놓은 진정한 야심작. 수수해보이는데 문제 하나하나가 서민수학의 걸작, 단원 김홍도 그림.
\item 3. 마플 ★★★★☆ : 시중 야매교재 강의 다 필요없고 이걸로 일단 정리. 그리고 부족하면 마플 교과서 추가.
\vspace{5mm}

지금 와서 쎈 보겠다는 분 시간없을 수도 있으니 쎈은 보충용으로 돌리고
아 올해 시험 포기해야하나 하는 분들은 위 3개만 돌리세요 그럼
\vspace{5mm}










\section{6월 지나면 힘듭니다.}
\href{https://www.kockoc.com/Apoc/784341}{2016.05.20}

\vspace{5mm}

또 이 이야기를 한다는 건 잔인하긴 한데 경험해보시면 아실 거예요.
그래도 이 이야기를 하는 이유는 시간낭비시키기 그래서 그런데
기온이나 계절상 제대로 공부할 수 있는 시기가 11월부터 5월까지입니다. 6월부터는 정말 양을 절반으로 줄여야하고 체력관리가야하거든요
\vspace{5mm}

그런데 6평과 9평 쓰나미로 정신적 충격 먹고 거기다가 온갖 잡서들이 유혹해서 자기 공부 제대로 못 합니다.
기본기가 되어있는 사람들이 그나마 잡서라도 소화시키지, 나머지들이야 휘둘리거든요.
이거 재밌는 게 당사자들은 제가 이런 말을 하면 반발합니다. 그런데 타자 입장 되어보면 바로 납득갈 겁니다.
\vspace{5mm}

6월 되어서 확신 안 선다 하는 분들은 깔끔하게 올해는 그냥 운빨 기대하면서 가고 내년 대비하는 게 낫습니다만...
경고드리면 올해 고2부터 이상하게 미친 듯이 잘 하는 경향이 있습니다(...) 개정교과 과정의 위력인지 학부모들 빨이어서 그런지 몰라도 그러합니다.
6월 되어서 힘들다 하는 분은 그냥 수학과 국어나 죽어라 파는 걸 권하고 싶습니다. 최상위권 실력이 아니면 언제 시험을 쳐도 답이 없으니까.
\vspace{5mm}

11월$\sim$5월에 공부를 제대로 안 한 사람이 과연 6$\sim$10월에 공부를 제대로 할 건지는 개인적으로는 심히 의문입니다.
\vspace{5mm}

이 글이야 작년에도 했던 소리이고 사실 개인적으로 트루라고 보기 때문에 다시 쓰는 건 괴로운 일이나
왜 이런 경고 안 했냐라고 욕먹기 딱 좋을 것 같아 다시 적습니다(...)
기출 돌리면 되지 않느냐... N수생도 풀지 않은 기출을 고2들이 다 풀었단 사례를 심심치않게 접하고 있습니다요(물론 상위권이겠지만)
차이가 커도 너무 크다고 생각해요.
\vspace{5mm}

사실 콕콕에서만도 몇몇 학생들이 쎈수학도 안 풀었다는 걸 보면서 한숨을 쉰 적이 있습니다만
진짜 최상위권들은 그런 건 거의 다 진작에 끝내고 공부 안 한 척 합니다. 인터넷에는 거짓 정보가 꽤 많이 올라오죠.
5월까지 그래도 공부하신 분들은 6월부터는 공부량 줄이고(더워지니까 당연합니다) 최대한 환경, 체력관리 신경쓰면서
모의고사를 포함한 문풀에서 오답정리, 분석, 그리고 불량해결을 철저히 하시길 바랍니다요.
\vspace{5mm}

6월부터 여학생들을 픽픽 쓰러지고 남학생들은 9월 정도 되면 다수가 멘탈붕괴되고
이과 하다가 올해는 안 되겠다 하면서 문과로 도망가는 일이야 한두건이 아니고
이제 애꿎은 교육청과 평가원 욕하고 무슨 모의가 좋다느니 아무개가 갓이라느니 허무맹랑한 '루저들의 사교파티'가 벌어질 겁니다.
\vspace{5mm}







\section{[수학교재] 풍산자 필수유형}
\href{https://www.kockoc.com/Apoc/784630}{2016.05.20}

\vspace{5mm}

선지와 문항만 다듬으면 쎈과 맞먹는데 그렇지 못 하다는 게 아까움
상 문제에 해당하는 건 신사고 라인이나 RPM과는 차별화되는 창의적인 문제들이 있음.
\vspace{5mm}

풀어보면서 참 잘 만들었는데 이거 출판사 뒷마무리가 부족하다라고 아쉬움.
\vspace{5mm}

올해 연구해보면서 꼽은 유망주.
\vspace{5mm}

\item 1. 올림포스 평가문제집
\item 2. 풍산자 필수유형
\item 3. 일품
\vspace{5mm}

개정수학.
내용이 쉬워진 거지 문제가 쉬워진 것이 아니죠. 내용이 어려우면 오히려 문제는 쉬워질 수 있습니다(쓸 수 있는 잡기들이 많아지니까)
하지만 내용이 쉬워지면 논리가 명쾌하니까 더 꼬아낼 수 있죠.
\vspace{5mm}







\section{모 수학교육에 관한 책을 읽어보았는데}
\href{https://www.kockoc.com/Apoc/788825}{2016.05.23}

\vspace{5mm}

어느 학원인지 어떤 책인지 짐작은 다 가실 테고
\vspace{5mm}

\item 1. 기본교재는 정석과 블랙라벨 ; 쎈조차 부족하다
\item 2. 과학고, 자사고의 수업과정에서 쓰는 고급수학 등도 공부할 필요가 있다.
\item 3. 선행은 실속있게 해야한다
\item 4. 과학고생들이 보는 필독서 시리즈 같은 것을 읽어볼 필요가 있다.
\vspace{5mm}

공부방법 면에서는 개념 복기를 해야한다 빼고는 새로운 게 없습니다.
주목할 것은 1번과 2번, 그리고 4번 같다고 보는데.
\vspace{5mm}

콕콕만해도 라벨은 커녕 쎈조차 풀지 않은 학생들도 널려있었고(그러면서 실모 얘기하는 것부터가 뭔가)
고급수학을 꼭 할 필요는 없지만 어느 정도 기본과정을 마친 분은 저걸 공부해둘 필요는 없지 않고(특히 수학적 모델링)
그리고 수학적 사고력을 키우는 데 과고생이 보는 필독서 시리즈가 다 좋은 건 아니지만 일부 책들은 확실히 인강 그 이상인 게 있습니다.
\vspace{5mm}



\section{더위 한방에 무너지는 공부}
\href{https://www.kockoc.com/Apoc/789055}{2016.05.23}

\vspace{5mm}

그러니까 사실상 5월까지라니까요.
지금부터는 의도적으로 학습량 절반 줄이고 체력보전에 신경쓰세요. 후회하지 마시고
에어컨으로 버틸 수 있다 해도 체력 떨어지는 건 못 막습니다.
이제부터 1순위는 무조건 \textbf{무더위 버티기, 그리고 6평을 치면서 '절망'을 일부러 맛보면서 인내하자}입니다.
6평 치고나서 그 결과 가지고 자살한다 뭐한다 그딴 드립치는 사람은 이미 그릇부터가 간장 종지만도 못 된다는 이야기입니다.
그냥 6평에서는 '절망감'의 예방주사를 맞는다고 생각하고 치세요.
\vspace{5mm}

남들은 열심히 하는데 그럼 어떡하느냐 그럴 건데 더위는 모두가 공평하게 겪습니다.
더위에도 불구하고 열심히 하는 사람들이야 원래 그런 놈들이니까 어쩔 수 없지만 이들은 극소수고
다른 사람들도 무너지긴 마찬가지이니까 '덜' 무너지는 게 이기는 겁니다.
괜히 만용부리다간 학습량이 저절로 줄어드니, 그냥 스스로 줄이시고 일부러 휴식시간 늘리세요
결과적으로 똑같지 않냐고 하지만 '컨트롤을 하느냐 못 하느냐' 차이는 상당히 큽니다.
학습량을 줄일 수 있어야 늘릴 수 있죠.
\vspace{5mm}

그리고 고2들은 사실상 겨울방학까지 다 끝내야한다는 걸 인지했을 겁니당.
\vspace{5mm}

