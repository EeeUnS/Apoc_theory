


\section{[상담 001] 모범적인 사례 + 추가답변}
\href{https://www.kockoc.com/index.php?mid=Apoc&search_target=title_content&search_keyword=상담&page=5&division=-810609&last_division=0&document_srl=418912}{2015.10.16}

\vspace{5mm}


특정성을 주지 않은 상태로 싣습니다.
면대면 대화는 긴장감이 없어서리.
다만 게시물인 이상 평가는 길게 해드릴 수 있음.
\vspace{5mm}

4월달에 말씀하신 국어 오천문제 영어 삼천문제 수학 삼만문제(문과라서 멋대로 비례축소해서 만오천 문제 풀었습니다..)사탐 오천문제 (합처서 만문제) 다풀었습니다만 그래도 미친듯한 자신감대신 '아..잘나올까..'라는 생각으로 두렵습니다. 이게 보통의 반응인가요.?
적은 만족할 만큼 나오고는 있습니다만.. 수능 미만 잡아니겠습니까..
\vspace{5mm}

아, 그리고 되게 옛날에 시중교제가 엄청나게 많다고 하셨습니다만 같은 내용들을 이름만 바꿔서 파는거 제외하니 저희 집 근처 xx문고 기준으로 국어는 시중교재 10권 남짓. 사탐도 10권 남짓 이던데 제가 잘못샌걸까요..? 그래도 xx문고면 초 대형 메이커 서점이고 정말로 사탐은 거기에 있는 모든 브랜드를 다 풀었습니다만 실제로 제가 보는 과목을 제외하면 10권이 전부인것 같습니다.ㅇㅇ문고 인터넷으로 검색해도 더 나오지는 않구요. (물론 문제집 형식만 풀었습니다.)
\vspace{5mm}

아, 제 키가 1xx인데 지금 제 인중.? 정도 문제가 쌓였습니다.. 근데.. 그래도 무섭습니다..
\vspace{5mm}

너무 많은 질문 드려서 죄송합니다.
집앞에 있는 대형서점에 있는 문제집들을 어제부로 전부다 풀었습니다. 사탐은 정말로 모든문제를 다풀었고, (16000정도 됩니다.) 수학은 rpm과 정석만든 회사에서 만든 책 제외하고 다풀었습니다. (수특.수완.정석.메가엔제.일품.블랙라벨. 마플.풍산자. 수능다큐 천제 이정도면 풀면 문제수가 약 15000정도 나옵니다. ) 국어도 그냥 많이 풀었고. 영어도 많이 풀었습니다. 이제 시중에서 파는 문제중에서는 안푼 문제가 없습니다.
그래서 앞으로 실모를 풀어볼 생각입니다만 시중에서 팔고있는 실모를 봐야할지, 아니면 사설 학원에서 낸 모의고사들 (범위는 전범위)인 놈들을 풀어봐야할지 모르겠습니다.  어떻게 해야 할지 모르겠습니다.. 굳이 차이가 있을까요..
\vspace{5mm}

일지쓰고 상원 들어온 분은 아닌데 나름 충고를 듣고 그대로 실천한 케이스입니다.
이런 케이스면 올해 시험이 안 좋으면 운이 정말 나빠서이겠죠. 하지만 이 정도면 내년에 시험을 친다 하더라도 1년 걍 늦게 간다 케이스가 되겠죠.
다만 오답정리는 했을지 모르겠습니다만 그걸 만회하고도 남는 양입니다.
\vspace{5mm}

그 뒤 더 이상 풀 게 없기 때문에 이런 분은 실모를 적극 권장했습니다(저 정도면 펀더먼털은 튼튼할 테이니 말이죠)
다만 실모 점수는 신경쓰지 않는 게 좋다고 지금 말씀드립니다. 왜냐면 현재 나오는 실모들은 출제적중율이 보장된 건 단 하나도 없습니다.
그냥 새로운 문제 풀어본다 그 정도이기 때문이죠.
\vspace{5mm}

이 정도 하면 노력을 했다라고 할 수 있을 것입니다.
그리고 본인 말이 진실이라고 한다면, 불안하면서도 나름 자부심이 있죠. 남들이 잘 안 하는 양치기를 끝냈기 때문입니다.
양치기가 전부는 아닙니다. 하지만 양치기가 되어있는 사람은 시험이 어떻게 나올지 전략적인 불안감을 느끼지
내 인생 망했다 그딴 드립은 안 칩니다요.
\vspace{5mm}

저기 해당하는 분은 여기 댓글 말고 쪽지로 답해주시면 되겠습니다만.
\vspace{5mm}
\begin{itemize}
    \item 첫째, 실모 점수 신경쓰지 마시라는 것. 실모는 어떻게든 팔아먹기 위해서 당치도 않게 어렵게 내려는 게 있습니다.
    그리고 오답은 원래 행운인 겁니다. 틀린 문제가 있으면 그걸 왜 '틀렸는가' 분명히 보시길 바랍니다.
    \item 둘째, 더 이상 풀 문제도 없겠지만 풀었던 문제집에서 틀렸거나 별표친 걸 다시 풀어보고 해설을 꼭 읽어보시길 바랍니다.
    맞았던 문제조차 다시 보면 이런 내용이 있었다니... 라고 놀랄 경우가 많습니다. 실력이 높아진 상태에서 보면 안 보이던 게 보이는 것이죠.
    \item 셋째, 사설인강이나 EBS 인강 중에서 '어려운 강사'의 짧은 커리큘럼 하나를 소화시켜보시길 바랍니다.
    들을 때 1.4~1.5배속으로 '문풀' 위주로 빨리 가는 게 좋습니다. 이 정도 내공이면 남들보다 10배 이상 인강에서 얻어갈 게 있죠.
\end{itemize}
\vspace{5mm}

인강은 대략 3~4일 내에 마무리할 수 있는 걸로 들으시길 바랍니다.
들으라고 하는 이유는 이 시점 되면 감성주의에 젖는 경우도 있지만 문풀이 지겨워지고 혼자 읽는 게 재미없을 수도 있어서입니다.
대략 강의평들 검색해서 괜찮다고 하는 것 과욕부리지말고 2 과목 정도만 들으십시오.
그리고 가성비가 나쁘더라도 소위 인기강사가 야심차게 냈다는 실모 같은 것(턱없이 비싼 것 있습니다)도 풀길 바랍니다.
\vspace{5mm}

정리하면
\begin{itemize}
    \item [$-$] 실모 그냥 가볍게 풀고 오답체크. 왜 오답이 나왔는지 사고과정을 반성한다
    \item [$-$] 풀었던 문제집에서 다시 봐야할 것 또 풀어볼 것. 50문제당 2~3문제 걸릴 거임. 3회독 정도 하면 좋음
    \item [$-$] 교재나 문제집이 지겨울 때 바로 남들이 듣는다는 강사 인강 파이널 문풀 수준으로 들어주길(지금이면 얻는 게 많을 거임)
\end{itemize}
\vspace{5mm}

이렇게 해보시고 피드백은 쪽지로 보내주시길 바랍니다.
\vspace{5mm}

그리고 경고인데, 다른 분들이 이 충고를 따라하지 마시길 바랍니다. 적어도 윗 수험생과 같은 코스를 걸었을 때에만 효과있는 방법이기 때문입니다.
가령 공부도 안 되었는데 저도 파이널 들으면 되어요... 라고 하는 사람들은 발암분자죠.
\vspace{5mm}

이어서 추가답변
\vspace{5mm}

답답할때마다 사탐 풀었기에 새로운거 없나 검색하러 왔다가 글 읽었습니다. 감사드립니다.  마음의 위안이 된것같습니다.
\vspace{5mm}

말씀하신것중 2가지 (실모 부담 ㄴㄴ. 파이널 듣기)에서 ebs 파이널은 다 들었습니다. 그외에는 솔직히 돈이 없어서 못듣고 있구요...
오답은 사실 전부다 꼼꼼히는 못했고, 문제풀때 주로 c (난이도별로 a.b.c로 체크했습니다.)라고 체크한건만 다시 풀었습니다.
\vspace{5mm}

아, 그리고 문제 실제로는 제 키의 1.5배정도 푼것 같습니다. 다만 처음부터 키의 1.5배정도 풀었습니다 하면 '노인증 구라즐.' 먹을것같아서 좀 축소했습니다.  지금 독서실이라 방 돌아가서 사진은 못찍고, d-100일때 사진 찍었던것이라도 인증겸 첨부합니다.
\vspace{5mm}

\href{http://dcimg2.dcinside.com/viewimage.php?id=39afd135ed80&no=29bcc427b38377a16fb3dab004c86b6f6b6cb2befefb20e3528f8026dfb46508612ef1cea9c04e95ed7932bc99f920f35041a8c1e51834bb}{링크}
\vspace{5mm}

저렇게 공부한 건 자랑할만하기에 일단 이것도 공개합니다.
저 정도라면 그 다음 추가로 권해드릴 것은 이거임.
\vspace{5mm}

"논점 실전 메뉴" 만들기
\vspace{5mm}

틀린 문제나 4점짜리 골치아팠던 것 풀었을 때 그 풀이과정은 결국 '핵심키워드'를 얼마나 잘 떠올려 논점을 연결짓느냐인 겁니다.
지표와 가수 문제의 경우 "지표들을 한변에 몰아넣어 양변을 정수조건에 묶는다',
미분 그래프의 경우 '3차 함수의 개형 3가지'
국어의 경우 틀리기 쉬운 문법지엽
이런 것들이 단골로 쓰이는데 이걸 따로 수첩에 간략히 메모하시고 즉석에서 누가 물어봐도 답할 정도로 반복하시길 바랍니다.
\vspace{5mm}

이제 이 엑기스만 모아놓고 반복해서 숙달하면 됩니다
\vspace{5mm}

이미 양치기를 했기 때문에 '실패하고 좌절하는 걸 벼슬인 줄 아는 평범한 수험생과 나는 다르다'라는 인식수준까지 도달했으니
그 다음은 님이 공부한 걸 요약정리해서 출제자가 어떻게 엿을 먹이든 대처할 수 있도록 준비해놓으시길 바랍니다.
다시 말해서 시험문제 풀 때 '그게 뭐더라 가물가물'은 겪지 마시라는 이야기입니다.
\vspace{5mm}

+
\vspace{5mm}

문풀이 된 상태에서 인강을 듣는 게 효과가 좋습니다. 대다수는 인강만 듣고 문풀을 안 해서 나중에 아이구 내 인생 그러고 있죠.
강사들은 자기들이 아는 걸 학생들도 안다고 생각하고 강의합니다. 강사의 의도는 그래서 종종 빗나들가죠.
하지만 학생이 기본서 회독수와 문풀량이 보장되어있다면 그 의도는 적중합니다.
\vspace{5mm}

이 점에서 공부도 안 한 주제에 xxx 강사가 좋은 것 같아요 하는 사람들은 발암 쇼핑홀릭이죠.
문풀도 안 되어있다면 강사 가릴 처지도 아닙니다. 그 상태에서는 명강의 들어도 실제효과는 낮죠.
\vspace{5mm}

++
컴 절제하고 원하는 목표 이루신 뒤 돌아와서 쪽지로 인생얘기 나눌 수 있길 바라겠습니다 :)
\vspace{5mm}




\section{[상담 002] 군대간 케이스}
\href{https://www.kockoc.com/Apoc/418933}{2015.10.16}

\vspace{5mm}

    
    작년에 시원하게 재수 말아먹고 삼수 가냐마냐 할 때 군대 추천해주셔서 현재 군대 사지방입니다....
    x군지원했고 xx병입니다. 하루 전체 일과 모든일이 2시간 안에 끝납니다.
    \vspace{5mm}

    일은 칼같이 9시부터 17까지 합니다. 한달에 1주일정도 18시까지 행정반 전화대기합니다. 시간이 생각보다 많습니다.
    저는 문과생이였습니다. 하지만 이과로 돌려서 \textbf{의치한수} 가고싶습니다.
    17수능 생각하고있습니다. 워드마스터 1회독 거의 다했습니다.
    \vspace{5mm}

    늦은시간이라 정리가 안되어서 죄송합니다만 급해서 그려러니 해주십쇼
    생각보다 군대가 편하고 재밌어서 노래방,BX,PC방,당구장 등... 방해요소가 많아도 참 많죠
    그 결과 입대하고 자대배치 받자마자 바로 공부를 시작하려던 마음은 선임들과 딩가딩가 놀다가 사라지고
    \vspace{5mm}

    머리가 바보가 됨을 느끼게 되어
    다시 펜을 잡고자했습니다. 아재가 제일 먼저 떠오르더군요
    \vspace{5mm}

    질문드리겠습니다.
    \vspace{5mm}

    \textbf{현재 모든과목 전체 노베이스라고 생각하시고 조언을 구하고싶습니다. 과탐도 정하지 않았습니다}
    \textbf{과탐도 추천해주실 수 있으십니까? 그 과목의 강좌나 그 과목의 선생 같이 구체적으로 가능하면 부탁드리고싶습니다.}
    \vspace{5mm}

    현재 모든과목이 노베이스기때문에 EBS 이하영 선생의 [기본개념] 왕초보를 위한 '이런 수학 처음이지?' 수강하고있습니다.
    어제부터 듣기시작했고 7강째 들었습니다.
    \vspace{5mm}

    정말 기초부터 탄탄히 잘해보고싶습니다. 이제 복지시설관 쪽은 처다도 보지않을 생각입니다.
    \vspace{5mm}

    저같은 놈이 읽을만한 칼럼이나 공부법 쓰신 것 있으면 링크 첨부해주시면 뼈에 새길정도로 읽고 실천하겠습니다.
    \vspace{5mm}

    \textbf{솔직히 17학년도 수능개정이 어떻게 진행되는지도 모르겠습니다.}
    \textbf{그에대해서도 설명해주시면 정말 감사하겠습니다.}
    \textbf{17학년도 이과 수능 도전할겁니다.}
    \vspace{5mm}


군대가신 걸 신의 한수라고 여길 겁니다. 나이처먹은 사람이 충고할 때는 (적어도 그게 상업적 의도가 없다면) 듣는 게 좋죠
그리고 병역은 빨리 끝낼 수록 좋습니다.
\vspace{5mm}

우선 그 충고를 들어주신 건 현명하다고 칭찬.
다만 상담내용은 너무 욕심이 많습니다. 물론 욕심은 많아야하지만 현실적으로는 줄여야하죠.
\vspace{5mm}

2017 수능 응시는 좋습니다만 거기서 합격을 기대하진 마십시오.
권장해드릴 것은 이겁니다.
이과수학 '2등급' 나오기, 과탐 하나는 '1등급', 다른 하나는 '2등급'
이 정도만 하더라도 군인에게는 매우 힘들지만 현실적인 목표입니다.
\vspace{5mm}

군대에서 독학공부는 어려울 겁니다. 이 경우는 그냥 욕심 너무 부리지 말고 EBS 인강만 따라가시면 됩니다.
현재 듣는 이하영 것 들으면서 '올림포스'만 따라가십시오. 당연히 문제편을 권하겠습니다.
다른 교재는 쳐다보지 마시길 바랍니다.  올림포스 한 교재만 보고 그리고 틀린 문제를 강의 따라듣고 아예 암기해버리시길 바랍니다.
\vspace{5mm}

그 다음 내년에는 2가지 과정이 있죠. 수능개념강의, 수능기출강의 수능특강 완성 강의가 있는데
개인적으로 권하는 건 수능개념강의와 수능기출강의가 나오면 그걸 들으시라는 것.
수특이나 수완은 수특과 수완을 풀 때에만 발췌로 강의를 듣길 바랍니다(EBS가 그게 잘 되어있습니다)
\vspace{5mm}

오직 이것만 유념하십시오. 왜냐면 현실적으로 저걸 하기도 힘들 것이기 때문입니다. 그리고 저것만 하더라도 2등급 이상은 반드시 나옵니다.
\vspace{5mm}

과탐 과목 선택은 물리1과 지학1을 권장해드리겠습니다.
이건 내년 수능개념강의로 들으셔도 좋고, 아니면 탐스런만 가도 됩니다(사실 탐스런으로도 넘치지요)
\vspace{5mm}

내년 시험을 치른다기보다도, 강의 한번 제대로 들어보자. 결과는 신경쓰지말자라는 마인드로 가세요.
절대 다른 건 쳐다보지 마십시오. 성과없는 과욕은 패가망신의 지름길입니다.
중간에 이거 성적이 오르는 것 맞나... 생각도 들겠습니다만 힘들수록 사실 그게 바른 길입니다.
\vspace{5mm}

+
\vspace{5mm}

올림포스는 뭔가 덜 정제되었지만 문제 참신성과 난이도는 나무랄 게 없습니다. 간혹 해설이 무리수이긴 하지만요.
\vspace{5mm}






\section{[상담 003] 강의 듣는 법}
\href{https://www.kockoc.com/Apoc/420984}{2015.10.17}



    \vspace{5mm}
    강의를 제대로 본다는 것이 정확히 어떻게 하는 겁니까?
    예로 들어주신 xx 강의를 듣겠다고 정했으면
    우선 문제풀이부터 쭉한다음에 채점을 한뒤에 1강을 틀어서 보는 겁니까? 아니면
    올림포스 강의 먼저듣고 문제풀이는 나중에 진행하는 겁니까?
    어떻게보면 아주 기본적인데 궁금해서 그렇습니다.
    꼭 좀 답변 부탁드립니다.
    \vspace{5mm}

본인이 고수라면 문제를 다 풀고 필요한 부분만 강의 듣습니다.
본인이 중수라면 기초, 중급 정도의 문제만 예습하고 고급 문제는 읽고 강의 듣습니다.
본인이 하수라면 그냥 강의대로 따라가는 게 '베이스'는 다지는 방법입니다.
\vspace{5mm}

이 수험생의 경우 기초를 쌓는 상태면 강의대로만 따라가는 게 낫죠.



\section{[상담 004] 막판정리}
\href{https://www.kockoc.com/Apoc/422874}{2015.10.17}

    \vspace{5mm}

    9월 29일에 제대를 한 상황이고 현재 아파트 독서실에서 공부중입니다.
    6월평가원에서 수학 30번 1개 영어 3점 하나
    9월평가원에서 영어 3점하나 2점하나(도표$-\_-$) 틀렸습니다.
    \vspace{5mm}

    제 원칙은 월화수목금 11 11 11 11 11 토요일 7 일요일 0 (시간)으로 잡고
    올해 3월부터 계속 이렇게 공부했습니다.
    \vspace{5mm}

    제 주제에 맞지 않게 훌륭한 동기를 만났습니다.
    이 친구가 xxx xx학과를 4.2로 졸업한 친구라. 많은 도움을 받았습니다.
    \vspace{5mm}

    토요일에 한번 격주로 수학 모르는 문제 쌓아서 질문만 받는 학원에 가서 질문 해소하고 질문 해소하는 식으로 공부했습니다.
    국어는 매주 일요일마다 xxxx다닐 때 은사님께서 배려를 해 주셔서 수업을 듣고 있습니다.
    \vspace{5mm}

    국어 역시 수업을 듣는다는 것 보다 제가 일주일 동안 공부한 것 중에서 모르는 부분을 모아서 질문하고
    제가 별명이 4차원이라 $-\_-$ 남들과 다르게 생각하지 않는 모난 사고의 부분을 사포로 밀어버리기 위함입니다.
    \vspace{5mm}

    사탐은 6월에 세계사 한국사 쳤는데 다 맞았습니다.
    그런데 한국사 하면서 제2외국어 안하기가 억울한데, 그렇다고 잉글리시 수준을 보자니
    한국사 + 제2외국어하면 언수외가 무너져서 올해 잘 하는데 막판에 실점이 클것 같았습니다.
    \vspace{5mm}

    그래서 7월 말에 xxxxx xxx 선생님을 만나서(제 은사님입니다)
    동아시아사로 바꾸는 것으로 작전을 잘 짜고, 책을 받아서 인강을 들었더니
    세계사랑 시너지를 발휘하면서 9월에 다 맞았습니다.
    \vspace{5mm}
        
    상담의 요지.
    \vspace{5mm}

    \begin{enumerate}
        \item 수학학습 부분입니다.
        \vspace{5mm}

        저는 문과입니다.ㅇ이번 9월에 처음으로 30번을 해결하였습니다.
        풀면서 음... 틀리겠지 하고 생각하고 풀었는데 $-\_-$ 응?
        올해 3월 부터 푼 문제집은
        \vspace{5mm}

        수능적해석 미적분 고득점 쟁취
        이창무 선생님 개념의 정상 교재 수1 두권
        수특 수1 미통
        매달 치는  대성 종로 중앙 비상 모의(질문 받는 학원에서 받아 풀었습니다.)
        수능완성
        \vspace{5mm}

        이창무 선생님 문제해결 전략
        히든카이스 모의고사
        일격필살 이정도 입니다.
        \vspace{5mm}

        아.... 군대 가기 전에
        05$\sim$12 수1 교유청 기출 사설 다 풀고 오답노트 만들어서 돌렸었고
        11 12 미적 교육청 사설 다 풀고 같이 정리했었습니다.
        물론 그 수능 14 수능에서 수학 4등급 맞고 12월에 $-\_-$ 얼탱무 하면서 군대갔지만. 말입니다.
        \vspace{5mm}

        요즘 공부는
        \textbf{올해 어느 순간 사고가 많이 바뀌었습니다.}
        \vspace{5mm}

        문제를 어떻게 풀이 접근 방법에 천착했던 제가
        \vspace{5mm}

        \textbf{어 이 문제는 교과서의 어떤 개념에서 파생되었지에 집착하고 있습니다.}
        \vspace{5mm}

        \textbf{부족한 개념은 수학교과서를 찾아보고 리마인드 하는 형식으로 진행되고 있습니다.}
        \vspace{5mm}

        기출과 수능특강 문해전교재 다시 풀고 있습니다.
        \vspace{5mm}

        그런데 세권의 교재를 반복해도 시간이 빠듯한데
        \vspace{5mm}

        새롭게 어려운 책을 풀어야하나 라는 조바심이 듭니다.
        수학 인강 하나 듣는데, 강필선생님 실수 방지를 들었는데,
        매 순간 해설마다 제가 무릎을 탁탁 쳐서(교과서 개념만으로 접근하는 부분과 실수 검산 부분입니다)
        수능 인덱스 지표가수 지수로그 함수 빨리 들을 생각입니다.
        \vspace{5mm}

        사실 05 12 까지 교육청 사설 돌리면서도 지표가수 지수로그함수가 가장 불편한데
        위 인강 맛보기를 보니 제가 문제 풀면서 느낀 불편함들을 건드리는것 같았습니다.
        \vspace{5mm}

        \item 공부시간의 밸련스
        \vspace{5mm}

        하루 공부시간은 대략 12시간
        2 3 4 3 으로 언 수 외 탐 인데
        제가 11수능을 볼때 9평에서 1 2 1 1 이 나왔다가
        영어 잡으려고 애쓰다가 수능에서 1 2 1 1 이 나와서 $-\_-$
        \vspace{5mm}

        이 경험으로
        \vspace{5mm}

        밸런스에 예민해진 상태인데 조언 부탁드립니다.
        \vspace{5mm}
    \end{enumerate}

    상담을 부탁하는 것보다는 이렇게 해서 극복했다란 이야기 같은데
    수학은 뒤늦게야 '양적 축적이 질적 전환'으로 바뀌어서 수리적 사고가 정착되었다고 볼 수 있습니다.
    \vspace{5mm}

    일단 웬만한 걸 익명처리하지만 수학 커리는 익명처리 안 한 이유는
    개인적으로도 저 커리는 괜찮다고 보고 있으며, 특히 수학인강에서 '강필 커리'는 수리적 사고 배양에 좋다고 생각해서입니다.
    이 점에서 님이 걸어온 길은 꽤 괜찮다고 볼 수 있습니다.
    \vspace{5mm}

    다만 너무 뒤늦게 깨달았다라는 점이 있으며, 아울러 커리가 지나치게 평균화되어있습니다.
    수리적 사고를 더 단단하게 하려면 소위 인강교재나 인기강사 것만 볼 게 아니라
    다소 평균에서 벗어난 더러운 문제나 외진 것들에도 자신의 수리적 사고를 발휘할 수 있나 보아야하는데
    문제집 풀이 지나치게 '요즘 수험생들의 평균'에 모여있습니다. 이런 경우 이 평균을 배신하는 출제가 나오면 당할 염려가 크지요.
    더군다나 인강은 너무 명쾌하게 설명하는 바 - 즉 아스팔트 도로 깔아놓고 고속버스로 드라이브 시키기 때문에
    그걸 들을 때는 뭔가 명쾌하고 실력이 늘어나는 것 같으나, 강의 다 듣고 시중교재를 풀면 어 이게 안 먹히네 좌절할 수도 있습니다.
    \vspace{5mm}

    남은 기간이 촉박하긴 하온데 강의는 거기서 늘리진 마시고, "기출" 잡다한 것들을 랜덤하게 풀어보시길 바랍니다.
    \vspace{5mm}

    그리고 지금 밸런스는 균형을 잡으려 하기보다는
    자신있는 과목은 '실점'이 나올 부분만 추려서 그걸 공부하시길.
    에컨대 국어나 영어를 어느 정도 한다고 하면 '문법'과 '어휘', 그리고 몇몇 어려운 지문만 간략히 훑는 식으로 시간단축하고
    나머지 시간은 수학과 탐구에 할애하는 게 효율적이지요.
    특히 수학은 오답 5개 나오기 프로젝트라고 해서 지금 공부해서 다 맞는 그런 게아니라
    여태껏 공부했어도 틀리는 그런 문제를 하루에 5개씩 겪으면 남은 기간 100문제를 추릴 수 있을 테고
    그 100문제에서 본 시험에서 님을 위협했을지도 모르는 바이러스의 면역력을 키울 수 있을 거라고 보며
    탐구는 공부한만큼 도움이 되지요.
    \vspace{5mm}

    다만 영어 듣기를 '몽롱한 상태'에서도 1.4배속으로 들어도 맞출 수 있도록 훈련해두시길 바랍니다.
    \vspace{5mm}


\section{[상담 005] 탐구대비}
\href{https://www.kockoc.com/Apoc/427357}{2015.10.19}

    \vspace{5mm}

    저는 현재 학교 휴학을 한 반수생입니다. 6월 모의고사는 94 96 100 46 47 (물1 생1) 이였고 9월에는 100 97 100 46 39(물1 생1) 입니다.
    10월은 집에서 시험을 봤는데 국어1개 생물1개 틀렸네요
    3월부터 인강듣는 시간 제외하고 순공부시간이 7시간 이상씩나오도록 노력했었고
    8월부터는 인강을 하루에 한두개나 안들어서 공부를 대충 10시간 정도 하는 것 같네요. 2/3/2/3 정도의 비율로 합니다.
    \vspace{5mm}

    사실 언수외는 쉬운기조로 나오고 있어서 걱정이 덜합니다.
    언어는 기출문제를 다시 한번 보면서 생각을 정리하고(3개년AB+EBS), 수학은 기출을 여러 번 봤으니 EBS에 좀더 비중을 두고 정리하면서 실모 틀린부분을 다시 보려고 합니다.영어는 그냥 EBS한번 더 보면서 한주희 파이널 문제만 풀 예정입니다.
    \vspace{5mm}

    걱정이 되는 부분은 탐구인데요. 정말 탐구를 어떻게 해야될지 모르겠습니다.
    EBS랑 기출이랑 지금껏 들어왔던 인강교재 문제들에 \textbf{사설2년치 틀린거만 복습+완자 하면 충분할거 같기도 한데 새로운 문제에 대한막연한 불안함이 가시질 않네요}
    \textbf{시험때 조금만 당황하면 멘탈이 자꾸 승천해버려서ㅠㅠ이번 9월에도 생물 7번 문제가 잘 안풀리니까 쭉 말아먹었습니다}.
    당황하는 걸 방지하려면 새롭고 어려운 문제를 좀 더 풀어봐야하나요?
    지금 남아있는 안풀어본 문제는 백호 파이널모의6회분, 와부 물리6회분, 최수준 모의3회분 있는데 좀 더 필요할까요?
    초창기때 아폭님 글을 봤으면 그냥 시중에있는 문제들 다 풀어봤을텐데 지금 그럴시간은 없고..
    \vspace{5mm}

    질문을 정리해보면
    \begin{enumerate}
        \item 언수외 계획에서 수정할 부분
        \item 탐구 마지막 정리방법
        \item 시험장에서의 멘탈+체력문제
    \end{enumerate}
    정도가 되겠네요. 조언 부탁드립니다
    \vspace{5mm}
이런 수험생의 경우는 '구김살 없는 완벽주의' 미학에 빠진 경우가 많죠.
즉, 자기가 돌리는 교재에서 자기가 오답이 안 나고 회독수가 순조롭게 높아지면 된다는 건데.
\vspace{5mm}

예방주사는 아프게 맞아야 하고, 평소에 군사훈련도 빡세게 해야 외적의 침입을 막을 수 있습니다.
6평, 9평에서 점수 잘 나왔다고 하더라도 그건 수미잡입니다. 수능은 언제든 통수를 칠 수 있고, 올해도 높다고 보고 있습니다.
국어와 수학과 영어는 무조건 오답이 나올 수 있는 문제를 푸는 식으로 마무리하세요.
그게 찝찝하다할지 모르지만 시험에서 오답날 것을 미리 잡는 거라고 보면 됩니다.
\vspace{5mm}

탐구의 경우는 지금 체계 생각하지 말고, 닥치는대로 랜덤하게 '오답' 나오도록 푸시고
오답 나온 걸 정리하는 데 시간을 할애하시길 바랍니다.
생1 문제 나갔다고 하면 한문항에 20분 이상 정리하면서 그런 문제가 나오면 안 틀릴 수 있는 프로세스를 정리,
기본서 보면서 본인이 깨달은 바, 조심할 바, 출제자는 어떤 식으로 낼까 하는 것들을 메모해보시길 바랍니다.
탐구가 마냥 어렵게 나온다라고 하면 아무 것도 안 보입니다, 그러나 내가 출제자라면 어떻게 엿먹일까 생각하면 답이 보이지요.
특히 내가 출제자라면 나라는 수험생을 어떻게 좌절시키는 출제를 할까라고 하면, 님의 무의식이 대답해줄 겁니다.
\vspace{5mm}

전체적으로 양이 부족하지 않나 싶기도 한데 지금은 그걸 어쩔 수 없고, 오답이 나오도록 문제풀고 다시 그걸 정리하는 게 도움이 됩니다
\vspace{5mm}

시험장 멘탈이라 함은 이건 재밌는게
지금은 불안한데 막판 시험날이 되면 마음이 편안해집니다. 그래서 문제가 되죠. 긴장이 풀리니까 실수할 가능성이 훨씬 높아지니까요.
지인 한명에게 '내가 한문제 틀리면 만원 준다'라는 식으로 내기 걸고 공부하십시오.
그리고 당일날은 아침은 적게 먹고 점심도 그냥 커피나 쵸콜렛으로 갈음하시길 바랍니다.
\vspace{5mm}

+ 과목별 공부법
\vspace{5mm}

국어 - 문법 특화시켜놓고, 어려운 지문을 많이 읽어두길 바랍니다. 어차피 국어는 문법 아니면 독해일 건데요
화작문이 어렵게 나오면 이건 논리, 수학문제가 됩니다. 국어는 작년 B형 수퍼문 정도까지 나올 수 있겠죠
\vspace{5mm}

수학 - 저라면 실모 문제 집착은 안 합니다. 왜냐면 특정패턴에 익숙해진 것 자체가 새로운 문제의 풀이를 방해하는 경우가 있어서입니다.
마음을 비우고 교과서 개념 다시 점검한 다음, 랜덤하게 4점급 문제 5분 안에 푸는 훈련을 하겠습니다. 틀려도 관계없어요, 그 절박함이 중요하니가요
\vspace{5mm}

영어 - 통수치지 않을까 하고 있습니다. 왜냐면 다들 쉽다고 인식하고 있는 것 자체가 임란 전 조선의 국방인식과 비슷한 느낌이 들어서
전체적으로 쉽게 내면서도 변별력 주는 방법이 없지는 않습니다.
\vspace{5mm}

탐구 - 탐구는 기본서의 지엽 정리 철저히 하고(EBS 수특수완해설까지 다 외워야합니다) 반면 킬러문제는 킬러문제접근법을 본인이 쓸 수 있게 정리
탐구가 어려운 건 암기와 응용 둘 다 가야하기 때문입니다. 즉, '용어'와 '식계산' 이 양자를 해야하기 때문에 애매해지는 겁니다.
용어 정리는 기본서 다시 여러번 보면서 철저히 외우십쇼, 지금이 분자 단위라면 나노단위까지 갈 정도.
식계산 문제는 따로 그 알고리즘 정리해두시길 바랍니다.
\vspace{5mm}

++ 돈내기는 장난이 아니라 진짜 하십시오. 그래야 뇌가 그걸 인식하고 움직여줍니다.
이런 이야기하면 또 저 꼰대가 xx한다할지 모르지만, 시험에 실패하는 수험생들은 사실 별로 절박하지 않습니다.
수험생활이 대단히 편하거든요. 무의식적으로 그 생활을 지속하고 싶다라고 느끼면, 의식은 합격인데 무의식은 불합격을 지향합니다.
그런데 이런 무의식도 '돈'에는 엄청 민감하죠.
가족 말고 절친 잡아서 한문제당 1만원 그게 싫다하면 5천원 정도로 해서 그렇게 점수가 나오면 돈을 주겠다라고 계약서 쓰고 해보시길.
이건 다른 분들도 해보시길 바랍니다. 이걸 하고 안 하고의 차이는 엄청납니다.
친구만 횡재하는게 아니냐하는데 이렇게까지 안하면 게임 좋아하고 야한 거나 밝히는 님들의 뇌를 설득시킬 수 없어요
\vspace{5mm}

+++ 사실 학원 다니거나 독서실 가면 공부 잘 되는 이유는 '지불한 돈'을 아깝게 생각해서인 것도 있죠.
나중에 글 한번 쓰겠지만 '시간'에 지불하는 돈은 아까워하지 않아도 됩니다. 왜냐면 '시간'도 비용이기 때문에
그 시간을 알차게 보낸다는 건 금전적 비용을 지불해 시간적 비용을 아껴 최적의 성과를 거두는 것이니까요.
하지만 반면 교재에 지불하는 돈은 위험할 수도 있습니다. 교재는 언제든지 볼 수 있다고 생각하니까 긴장이 풀리는 것이죠.
\vspace{5mm}






\section{[상담 006] 문과 마무리}
\href{https://www.kockoc.com/Apoc/437211}{2015.10.23}

    \vspace{5mm}

    9월은 학원에서봣고 93 100 100 50 48 (동아시아/한국사) 베트남어12점
    :틀렷던문항: 국어 문학 수필 보기문제 . 문학 현대시 첫문제 .비문학 18번?(답1번이였던거)
    .한국사 13번
    10월은 집에서 응시 98 88 100 44 47 (동아시아/한국사)
    :틀렷던 문항 : 국어 어휘(마지막지문) . 수학 21.29.30 .한국사 -19번 동아시아사 -13.19번
    10월 끝나고 멘붕와서 공부가 더 잘됫던거 같네요.
    \vspace{5mm}

    여태까지 저의 공부상태는
    1주일에 될수있는한 풀로달리자고 공부를 못할정도로 힘들면
    하루 집에서 쉬고 일찍들어가서 자는 마인드로 공부햇고
    결과적으로 1주일에 하루 12시간 공부 (인강포함) 주말 7시간 하루 x 정도 공부시간 투여한거같습니다.
    \vspace{5mm}

    [국어]. 기출 90프로 :근거찾아가면서. 문학같은경우 용어 모를경우 인터넷.사전찾아가면서 다시읽기[EX)영탄적.우화 등]
    비문학 같은경우 추론을 필요로 할경우 지문내에 있는 근거로 찾아가면서 풀기. 화작은 지문에 있는내용
    \vspace{5mm}

    평가원은 오답시비에 휘말리게 하지않아야하기떄문에 지문에 명확한(?) 근거가 있다는 마인드로 기출을 분석했습니다.
    하루에 1회씩
    비문학은 LEET 2회독
    문학은 옛날 기출 시/소설 2/3이상 풀엇고
    문법은 개념서 다시봣고 EBS 연계교재 2번 이상 풀었습니다.
    화작은 EBS연계교재 다풀었습니다.
    \vspace{5mm}

    [수학]
    실모 하루에 1회씩 풀고 + 이투스 고난도 문제집 다풀었습니다.
    \vspace{5mm}

    [영어]
    EBS 연계교재 다보고 관련인강 파이널만 뺴고 다들은 상태입니다.
    문법 자이스토리 기출로 다풀고 틀렷던 부분 다시정리햇습니다.
    영어 실전모의고사 1주에 1회정도 풀었습니다
    어휘 -자이스토리 반정도했습니다.
    고교영어듣기.수능완성 - 3/4이상
    \vspace{5mm}

    [한국사]
    수능완성. 수능특강 단권화 햇고 (선지 자료 A4용지에 잘라붙이고 외우고 ..ㅠ)
    강민성 근현대사.국사 N회독 10번이상은 돌린거같습니다.
    강민성 문제풀이
    기출 16-06 평가원
    동사X독 국사 10회독
    사설 모의고사는 1/3정도 했습니다.
    \vspace{5mm}

    [동아시아사]
    개념서 최경석 수능개념교재 20회독-진짜 토나올떄까지 봣습니다.
    수능특강.N제.파이널.완성 -한국사와마찬가지로.
    사설.평가원모의고사 모음집 끝
    [베트남어]
    개념인강만 다들었습니다.
    수특 1/2?
    \vspace{5mm}

    대략적으로 공부 이정도 한거같고(더있을수도있지만)
    9.10월 모의고사 보고느낀게
    \vspace{5mm}

    9월: 국어 너무 날림읽기햇다가 털렷다.. (솔직히 9평 쉬웟다고생각햇는데..)
    꼼꼼히 읽자.
    수학 계산실수만 주의 .
    영어 작년의 기억이있어서 더 꼼꼼히.근거찾아가며
    한국사-핵지엽대비
    동아시아사- (솔직히 1컷50인줄알아서..) 이대로만
    \vspace{5mm}

    근데 10월 모의보고
    국어 -어휘 ㅂㄷㅂㄷ,(근데 이거 풀떄 어려울줄알앗는데 쉬웟습니다.)
    수학- 왠만해선 92 이하로 내려간적없는데(실모포함) 21.29.30 털린거보고 나중에 답보니까
    아.. 내가 새로운 문제나왓을떄 추론력이 부족한건가 싶고 실모만 푸는거 바꿔야 된다는 생각을햇습니다.
    그리고 21은 쫄아서 손못댓고 30번은 하나 잘못새서틀렷고 29는 아에 접근방법이 틀린풀이..
    영어-수미잡 (근데 저도 왠지 영어가 수능떄 작년보단 어려울꺼같아서 걱정...)
    한국사- 현대사 파트 다시..
    동아시아사- 이게 제일큰 걱정이엿는데
    평가원은 개껌같이 풀엇는데 (재수.현역떄도 항상50) 이번에 44떳는데 실력으로 틀렷습니다.
    근데 보니까 한번도 나오지않앗던.. 부분에서.. 개념서에도 이상한데 구석에 박혀있는데 내고..
    근데 수능떄도 그렇게 낼수있으니까 동아시아 다시 열심히 공부하게 되는계기가 됫습니다.
    \vspace{5mm}

    현황은 대략 이렇고요..
    앞으로 수능떄 까지 공부계획은
    [국어] 1일 기출(최근 b형) 1일 사관학교 1일 사설.대종모의
    화작-천제의약속
    문법-천제의약속.인강교재.ebs 다시정리 복습.(지문에 근거한것.개념만으로 풀도록 연습.. 되도록)
    문학- ebs정리.고전시가 n회독
    비문학- peet/ leet 틀렷던 지문
    \vspace{5mm}

    [수학]
    이거 쓰기전까지는 1일1실모 하고
    2012-2016 평가원 문제 4점짜리 하루에 1회씩 .실모 틀렷던거 정리해놧는데 그거 하루에 4문제씩 풀라햇는데(대부분 21.30)
    칼럼보고 공부방법을 바꿔야되나 싶네요
    아재말대로 제가 패턴화가 되었다보니..
    근데 수학은 불로나올꺼같고
    \vspace{5mm}

    그래서 수능완성 유형.실전편 다시풀거나 일타삼피 풀려고햇습니다.
    \vspace{5mm}

    [영어]
    ebs 지문 다시풀고
    변형문제 풀기-여태까지 안풀고 모아놧습니다.
    (그동안은 단어.구문 을 안되는부분 외우자는 마인드로 공부해서)
    문법- 평가원기출 모음 다시풀고 정리
    실모 -3일에 1회 정도
    듣기 -고교영어.수완
    간접쓰기-자이스토리 모음집
    기출- 2016-2009 평가원 기출 간접쓰기.주제.빈칸.어법.어휘.장문 만 근거찾아가면서 풀기 (실전처럼 x)
    \vspace{5mm}

    [한국사]
    여태까지봣던거
    a4용지 단권화하고 싸그리 암기
    기출 15-06 다시 풀기
    강민성 파이널.모의고사
    \vspace{5mm}

    근데 이게 하면할수록 어디선가 계속 까먹는부분이나와서 매우빡치는과목입니다..아..
    \vspace{5mm}

    [동아시아사]
    개념서 다시 치사할정도로 다시보고
    교육청.평가원 최근 3개년 보기
    \vspace{5mm}

    [베트남어]
    수능완성.작년수능완성
    기출 문제 풀고 (20분정도걸립니다.)
    단어 다외우고 지문암기
    -어느정도 효과가있는거같네요 요즘풀면 30점은기본으로..
    \vspace{5mm}

    이게 전체적 현황이고
    질문좀 드리겟습니다.
    \vspace{5mm}
    \begin{enumerate}
        \item xx대에서 공부하고있는데, (집이 xx라 통학 집-공부장소 까지 50분정도걸립니다.)
        위치 변경하는게 좋을까요 ? 아침에 분당선타는게 극혐이긴하고 과동기한테 걸린적도 있지만
        공부장소.밥먹기 는 괜찮은거같네요..공부집중도 잘되는거 같고
        \vspace{5mm}
        
        \item 언수외탐 공부 계획에서 수정할 부분있으면 조언 부탁드립니다.
        \vspace{5mm}
        
        \item 공부시간은 8시 30부터 -10시 30까진데
        중간에 30분 졸고 밥먹는시간 총합 1시간 정도고
        공부시간을 더늘리는게 좋을까요?
    \end{enumerate}

    \vspace{5mm}

제 감으로는 목표를 이루는 방향은 갔지만 이건 약간 늦게 시작해서 시험보기 직전 '아, 더 일찍 시작할 걸'이라고 느낄 겁니다.
그런데 '양적인 모자람'을 느낀다면 그건 방향은 맞는 거죠. 다만 시간이 부족해서이지
\vspace{5mm}

우선 질문에 답하면
\vspace{5mm}

\begin{enumerate}

    \item 대안이 없으면 바꾸지 마시길. 전 통학은 강조합니다,
    집에 있으면 공부가 안 되지요. 그리고 가까워도 안 됩니다. 멀리서 공부해야 긴장되기 때문에 공부된다고 작년 말에도 언급했습니다.
    \vspace{5mm}
    
    \item 
    \begin{itemize}
        \item 국어 - 화작문 고난도 대비, 그리고 비문학 과학 지문 읽어두실 것
        \item 수학 - 일부러 틀리는 걸 생각하고 4점짜리 문항 랜덤하게 잡아서 20분 내에 푸실 것, 시간 내 못 풀면 오답처리하고 해설 읽고 복기
        \item 영어 - 시간압박 실전테스트
        \item 사탐 - 한국사는 오답을 기본서에 체크해보실 것. 동아시아사도 마찬가지
    \end{itemize}
    \vspace{5mm}
    
    \item 
    공부시간은 현 상태로 유지하셈.
    그런데 이제는 문제수를 늘리지 말고, 과거에 틀렸던 문제 다 체크해서 기본서에 표시하셈
    표시하다보면 뭔가 느껴지는 게 있습니다. 그걸 알아야 함.
    \vspace{5mm}
\end{enumerate}

그리고 문풀은 이제 양치기가 아니라, 어려운 4점 문제 보고 좌절하는 경험을 계속 숙련시켜서
정말 모르는 문제가 나와도 떨리지 않고 전략짜서 풀 수 있는 자세를 맞추셈.
지금 누구도 뭔 문제가 나오는지 모릅니다. 단 그 수능문제는 교과범위 내에서 다 풀립니다.
그럼 교과지식을 이용해서 어떤 문제라도 풀 수 있도록 준비해둬야지
\vspace{5mm}

국어는 문법, 그리고 수학은 4점 킬러 계속 틀리고 오답정리하기, 사탐은 기본서 전략
이거 중시하면서 가시길.
\vspace{5mm}

아, 그리고 다른 분들께 말씀드리는데 전 쪽지'만' 상담은 상원 분들 빼고는 안 합니다.
쪽지로 문의할 분은 \textbf{자기 이야기가 게시판에 공개되는 건 무조건 감수하시길} 바랍니다.
물론 인적사항이나 그런 것들은 제가 지울 것이고, 혹은 부주의해서 처리 못 하는 건 수정하겠지만요.
\vspace{5mm}






\section{[상담 007] 모범사례 분의 추가 질문}
\href{https://www.kockoc.com/Apoc/437223}{2015.10.23}

        
    \vspace{5mm}

    상담사례1의 학생(?)입니다.
    \vspace{5mm}

    다름이 아니라, 이제 웬만한 파이널까지도 다풀고 (사탐은 시중에 있는것 다 풀었습니다만, 국영은 푼게 없고 수학만 좀 풀고 있습니다. 국영은 만점마무리 3개만 풀 생각입니다.)  점수도 그럭저럭 괜찮습니다만, 풀면 풀수록 두려움이 더 생기네요.
    어짜피 목표는 서울대 상경. 점수로 말하면 만점인데, 요즘 들어서 종종 새로운 문제집 없나 xxx 이런곳을 들어가다보면 사람들의 이야기를 보고 쫄게 됩니다.
    \vspace{5mm}

    뭐, 강사의 질이라던가 그런것말이죠.
    \textbf{사실 저는 학원을 다니지도 않았고, 그냥 서점에가서 새로운 책들이 있나 없나 체크해서 있으면 전부다 풀고 그냥 아침 6시 30분에 나가서 11시에 자러 돌아가는 그런 생활의 반복이었습니다}. \textbf{수학은 애초에 멍청했기에 그냥 암기로 전부다 커버쳤고. 영어도 마찬가지. 문법 읽고 분석하고 단어 외우고 숙어. 국어는 그냥 독서는 지문요약 전체요약 근거 찾기.} 사탐은 그냥 풀면서 모르는 개념 나오면 단권화. 딱 이것들만 했습니다.
    \vspace{5mm}

    근데 요즘 들어서 과연 이정도면 충분한가 라는 생각이 듭니다. 문제집도 2열로 한게 제 명치정도 오고, (원래는 일열로 쭈욱 쌓았습니다만, 책마다 싸이즈가 달라서 어느순간 붕괴되더라구요. 그냥 이열이 좋은것 같습니다.) 인터넷강의는 선생님별 ebs 개념강의 전부다 들었고 나름..할수있는건 다했습니다..
    \vspace{5mm}

    하지만 뭔가 부족함이 느껴지는거죠. 뭔가 '비책'같은걸 배워야하지 않은가..이런생각입니다. 사실 국영수는 문과기준으로 따지면 비책같은게 없다고 생각합니다만 (뭐 스킬이 중요하다 이러는데 전 솔직히 문과레벨에서 스킬이 필요하나 모르겠습니다.) , 한국사는 실모  종종 저도차도 못푸는 문제가 몇개씩 나오니 난 충분한가.. 이런 생각이 드네요. 태표적으로 최태성 강의에서 근초고왕이 가야를 먹었다는 이야기가 없는데 (실제로는 법흥왕 금관가야 진흥왕 대가야 입니다.) 막상 실모에는 나오고.. 항상 한두개씩 틀리는걸 보면 참 기분이 묘합니다..
    \vspace{5mm}

    학원같은걸 다니지 않았지만.. 그래도 이정도면 정말로 충분한걸까요.. 시험이 다가올수록 두렵습니다.
    \vspace{5mm}

    + 인터넷에 널리 퍼진 이야기가 '실모짱짱맨'이라는건데 전 나름 많은 실모를 풀면서 (사탐은 한국에 있는거 다 풀었습니다.) 도대체 왜 그런 실모짱짱이라는 분위기가 있나 이해가 안갑니다. 솔직히 전 많은 실모들을 보면서 '아, 이거 괜찮네' 라고 느낀건 한국지리 이투스 선생꺼 말고 없습니다. xxx이라는 사람이 엄청 유명한 한국사강사던데, 이사람 실모도 말할 필요도 없었구요.
    \vspace{5mm}

    그외에도 xxx라는 실모는 미친놈들이 abcd 음영을 제대로 안줘서 문제도 못볼레벨이고. 스캔을 뭘로 했는지 미세한 글씨는 보이지도 않을정도입니다.
    \vspace{5mm}

    전 왜그렇게 사람들이 '실모짱짱. 실모만 풀면 모든게 해결된다' 이러는지 이해가 안갑니다.  그외에 유명 수학 모의고사도 '도대체 왜 이게 평가원급 퀄리티라고 사람들이 말하는거지?' 라고 생각드는놈이 있습니다.
    \vspace{5mm}





++ 그 상담사례1 글을 보니 아래에 국어 오천 수학 삼만 영어 삼천 탐구 오천의 기준 궁금해하는 사람있던데 그거 보니 저도 궁금하네요. 기준이 어떻게 되는건가요.?
\vspace{5mm}

올해 들어와서 학습론이라는 희대의 뻘글을 썼고 근거없는 비난도 먹었습니다만.
중요한 건 하라는데로 '양치기'한 케이스, 그것도 작년 겨울부터 부랴부랴 시작하든가
아니면 사정이 있어서 여름에 했지만 구력이 있어서 그걸 달성한 케이스는 나름 성과를 보여준다는 것입니다.
\vspace{5mm}

이 분의 경우는 본인이 경험하면서 느꼈겠죠. 그래서 실수를 안 한다면 실전에서 좋은 결과가 나올 것입니다.
문제풀이는 의식적으로 하는 게 아닙니다. 자동차로 말하면 의식은 그냥 '브레이크'에 불과할 뿐이죠.
문풀은 무의식의 힘으로 하는 것입니다. 다만 그 무의식이 너무 막 나가 뻘짓하지 않나 의식이 '논리'로 제동을 거는 것입니다요.
\vspace{5mm}

소설 사조영웅문에 나오는 곽정은 순진하고 멍청하지만 인생의 승리자죠.
애당초 무공에 소질도 없지만 나중에 천하제일고수 반열에 든 비결은 간단함
\vspace{5mm}

- 좋은 스승 만나서 시키는대로 다 했음
- 이상한 잡기나 사파 무공 같은 건 쳐다보지 않음, 정파내공을 다짐.
- 구음진경을 걍 \textbf{외워버렸음.}
\vspace{5mm}

이 분도 이런 케이스입니다. 그리고 더 정확히 말하면 요새 말하는 흙수저가 취할 수 있는 최선의 길이 이것 밖에 없습니다.
금수저 쪽이야 좋은 가애 선생을 만나거나 인강을 열심히 수강해서 스킬로 나옵니다. 당연히 효율성이 좋아서 단기적으로는 이길 수 없죠.
하지만 이 분과 같은 방식으로 규칙적 생활로 문제집을 모두 풀어버리고 회독수를 늘리는 건 '절대실력'입니다.
그리고 지금 느끼는 불안감이라는 것도 '최상위권의 불안감'입니다. 공부 안 한 것과는 달라요.
\vspace{5mm}

지금 고민하는 것도 사실 사소합니다만.
한국사의 경우는 두가지 권합니다.
\vspace{5mm}

하나는 namu.wiki 사이트에 가서 인물 중심으로 검색해보시면서 주욱 읽어보라는 것.
실모의 경우야 물론 교과서에 없는 거나 구석진 걸 내서 그러는데 방법은 없습니다 잘 볼 수 밖에요.
만약 이것도 모자르면 공무원 국사 \textbf{기출 문제집}을 사서 한번 돌려보시길 바랍니다. 오답이 많이 나오겠지만 불안감이 덜해질 겁니다.
동아시아사의 경우도 위키 참조를 권해드림
\vspace{5mm}

문풀량 기준은 뭐 그다지. 말 그대로 갯수에다가 회독수입니다.
100문제를 풀어서 20문제를 틀렸다 칩시다. 그럼 100문제 달성
다시 틀린 20문제를 풉니다. 그럼 20문제 달성해서 120문제 : 이런 식으로 계산하면 되겠지요.
\vspace{5mm}

님의 경우는 이제는 아무 방법이나 취해도 관계없습니다. 말씀이 진실 그대로라면 내공이 빵빵하죠.
정말 스킬 같은 걸 알고싶다고 하면 돈 조금 내서라도 유명강사 파이날 들으면 되겠지만, 사실 EBS와도 큰 차이는 없을 것입니다.
애당초 스킬이라는 것은 '애들이 공부하기 싫으니까 공부량 부족한 걸 메꾸라는 꼼수'적 성격이 강한 것입니다요.
\vspace{5mm}






\section{[상담 008] 수학 해설 비교}
\href{https://www.kockoc.com/Apoc/437233}{2015.10.23}



    \vspace{5mm}

    다소 평균에서 벗어난 더러운 문제나 외진 것들에도 자신의 수리적 사고를 발휘할 수 있나 보아야하는데
    \vspace{5mm}

    "기출" 잡다한 것들을 랜덤하게 풀어보시길 바랍니다.
    \vspace{5mm}

    이렇게 말씀하셔서
    \vspace{5mm}

    미적분은 가형 빡센 기출문제 + 실력정석 연습문제(홀수번 풀고 짝수번)
    \vspace{5mm}

    수1은  신승범 선생님 고난도 약점공략 뒤에 붙어있는 고득점 쟁취(인강 교재지만 더러운 문제 외진 것 같아서)만
    \vspace{5mm}

    문항번호가 나머지가 3인 문제 -> 나머지 2인 문제 -> 나머지 1인문제 -> 나누어 떨어지는 문제
    \vspace{5mm}

    이렇게 하고 있습니다.
    \vspace{5mm}

    제가 해설지 보는 법이 미흡한데,
    \vspace{5mm}

    제가 맞은 문제도 어 이렇게 생각하네, 이렇게 수식정리하면 깔끔하네
    \vspace{5mm}

    이 정도로 끝나는 것 같아 해설지 보는 요령을 부탁드립니다.
    \vspace{5mm}

    수학풀이란 결국 '조건', '공식', '식변환' 혹은 '논리적 추론'
    그 요소들의 결합입니다.
    \vspace{5mm}

일단 풀이는 다양할 수도 있습니다.
A는 식으로, B는 그래프로, C는 그냥 개념 논리추론으로 풀 수가 있죠.
\vspace{5mm}

만약 자기가 푼 게 해설과 같은 방향이면, 자기 풀이와 해설을 비교해서 내 풀이가 부족한가, 아니면 쓸데없는 부가내용이 있나보면 됩니다.
부족하다면 그건 빨간줄이나 형광펜으로 내 풀이에 '가필'해야합니다. 그게 내 풀이의 한계이기 때문이죠.
쓸모없는 내용도 마찬가지입니다. 쓸모없다는 건 그 문제의 추론 과정과 전혀 관계가 없단 것인데 이런 걸 떠올렸다는 건 '패턴화'의 부작용이겠죠.
\vspace{5mm}

자기가 푼 게 해설과 다른 방향이면 둘 다 수용하면 됩니다..
문제 아래 여백에다가 자기가 푼 방식을 요약해 적은 뒤 II라고 표시해놓고, 해설을 I이라고 해놓은 다음
해설 대로 풀어보면 되는 것입니다.
\vspace{5mm}

이렇게 하다보면 수학 보는 방법이 늘어납니다.
해설은 절대적이진 않죠. 어떤 집필자는 그래프를 너무 좋아하고, 다른 집필자는 식풀이만 고집하고 그런 경향이 있어서리.
그런데 수학을 잘 한다라는 건, '박쥐질'을 얼마나 잘 하느냐 그거입니다. 특정풀이에 매몰되지 않고 다양한 풀이를 쓸 수 있어야하는 것입죠.
\vspace{5mm}



\section{[상담 009] 패자부활전}
\href{https://www.kockoc.com/Apoc/468945}{2015.11.08}


    \vspace{5mm}

    고3 이과 학생입니다. 저는 입시 실패자 입니다. 초등학교 6학년말부터 중3 말까지 놀아서 공부를 한자도 안했고, 책도 안펼쳤습니다.. 수학은 제 기억으로는 중1 집합부터 그만둔것 같습니다. 현재 성적은 고1$\sim$고3 내신은 모든 과목 6$\sim$7등급대 8,9등급도 꽤 있음. 9월 모의고사 기준 국어 A형 6등급 수학B형 9등급 영어 6등급 물리1 지구과학1 6등급 7등급 입니다.각설 하고, 대학교를 가지않고, 육군을 12월에 지원할 계획입니다.. 3월까지 지원해서 붙으면 가고, 안붙으면 3월부터 재수를 할생각입니다.. 그동안은 알바로 돈을 벌생각입니다.. 몇가지 질문좀 하겠습니다.
    \vspace{5mm}
    \begin{enumerate}
        \item 군대를 간다고 가정할시, 짬이 차고 GOP 근무말고 FEBA 부대로 왔을때부터 일과 끝나고 조금이라도 공부를 할생각입니다.
        국어: 독서, 신문 구독  영어:초등학교 단어, 중학교 단어, 고등학교 단어, 수능 단어 그리고 플러스 알파로 토익 텝스 단어
        수학: 초등학교 수학부터    이렇게 공부를 할생각입니다. 시간이 안되면 다재끼고 수학만 할생각인데요.. 2019 수능 2020 수능 준비에 맞게 공부를 할려면 어떤 교육과정으로 공부를 해야하나요? 그리고 초등학교 수학은 개념원리에서 만든 쌩큐, 중학교 수학은 개념원리 중학수학 그리고 고등수학은 수학의 바이블 개념서로 공부할것입니다..
        \vspace{5mm}
        
        \item 계획이 너무 무리한가요? 초등학교 수학부터 할필요가 없을까요?
        \vspace{5mm}
        
        \item 마지막으로... 군대갔다와서 해도 늦지 않을까요..? 재수 삼수 이렇게 하다가 망하신분들 많다고 하셔서.. 군해결부터 할생각인데 현명한 선택일까요..? 제가 노력에 비해 눈이 높은것 같아서.. 지방대 전문대는 가기 싫어서 군필재수를 선택한것입니다.
        긴글 봐주셔서 진심으로 감사합니다.
        
    \end{enumerate}
\vspace{5mm}

사람은 눈이 높아야 합니다. 눈이라도 높지 않으면 정말 노비처럼 살아가는 것입니다.
부가 대물림되는 이유는 부자 자식들과 거지 자식들의 눈높이가 다르기 때문입니다.
본인이 올라가려는 의욕이 없다면 평생동안 그렇게 살게 됩니다.
\vspace{5mm}

하지만 올라가는 건 고통과 스트레스를 수반하는 일이고 자살의욕도 수십번 들게 되어있습니다.
\vspace{5mm}

다만 이 경우가 문제는 그겁니다. 사람은 자기가 안 해본 것을 너무 쉽게 생각해봅니다.
공부를 열심히 해서 입시를 앞둔 친구는 다 때려치우고 노가다 하고 싶다 그러고
공부를 안 해서 입시는 아직 엄두도 못 댄 친구는 공부하면 되지라고 생각하는 공통점은?
\vspace{5mm}

\textbf{자기가 안 해본 것을 너무 쉽게 생각한다는 것이죠}
\vspace{5mm}

일단 이 분이 군대 가시는 건 잘하는 겁니다. 어차피 군대 이전의 공부는 80$\%$ 이상 리셋되어버리니가요.
아무 것도 안 한 상태에서 군대 가는 거니까 잃을 건 없는 것입니다(이건 까는 이야기가 아니라 현실적인 이야기를 드리는 것이죠)
다만 평소에도 공부 안 했는데 군대 가면 공부할 수 있을리는 없죠. 그러니까 욕심을 너무 갖진 않으면 좋겠습니다.
\vspace{5mm}

군대에 가셔서 시간이 나시면 국어공부를 따로 하지 마시고 독서와 영어공부를 시작하십시오.
- 서양고전소설(흔히 추천되는 셰익스피어 희극이라던가 모파상 소설이라던가), 동양고전(삼국지라던가 대망이라던가)
- 영어 - 중학교 독해교재를 본다거나 토익공부를 시작한다거나
\vspace{5mm}

엥 뭔 소리야. 왜 이렇게 초라해할지도 모르는데요
\vspace{5mm}

공부 안 해본 사람들의 문제가 계획을 지나치게 거창하게 잡는다는 것입니다.
그리고 초심자일수록 과목들을 분산시켜서 다 할 수 있다라고 평균적으로 잡습니다만. 장담하는데 저것 못 합니다.
고수들은 과목을 집중해서 한 과목을 제대로 끝낸다거나, 혹은 필요한 과정만 집중적으로 제대로 이수합니다.
그렇게 함으로써 공부한 것을 자기의 '양분'으로 만들기 때문에 더 능력이 좋아져 다른 공부도 쉽게 할 수 있는 것입니다.
\vspace{5mm}

본격 재수를 하려면 군 제대 후에 하시는 걸 권합니다.
수학은 그 때까지는 차라리 안 쳐다보는 게 좋습니다. 내년부터 수학이 어찌될 지는 아무도 모릅니다.
과정은 분명 하향되었다고 하겟지만, 어차피 입시는 경쟁이라서, 그 경쟁률에 따라서 난이도가 어찌될지는 불명이어서입니다.
무엇보다 기본 교양이 없으면 수학도 어차피 제대로 공부 못 합니다.
바로 군대 다녀오시면 21$\sim$22살 정도일 건데 이때부터 정신차리고 2년간 공부해서 24살에 간다고 하면 그동안 노신 걸 다 만회할 수는 있겠죠.
이렇게 하기 위해서는 지금 감당도 안 되는 입시 코스 밟는 게 아니라, 밑바탕부터 제대로 쌓아야합니다.
\vspace{5mm}

정말 굳이 수학을 하고 싶으면, 이거 비아냥대는 게 아닙니다. '기탄수학' 시리즈 정말 초급부터 다 풀어보시길 바랍니다.
누가 보면 피식할지 모르지만 이거 남 눈치보고 그럴 게 아닙니다. 전 '현실적인 대안'을 제시하는 것입니다요.
말씀하신 사정대로라면 지금 남들 코스를 따라갈 이유가 없습니다.
\vspace{5mm}

문의는 계속 하실 수 있습니다.
\vspace{5mm}

정리
\vspace{5mm}

\begin{enumerate}
    
    \item  3월까지 책 100권을 읽을 것 : 만화책 말고 정말 활자로 쓰여진 것들. 근처 도서관에 빌리셈
    \item  영어공부 EBS인강으로 시작하실 것. 수준높은 것 듣지말고 그냥 기초적이고 쉬운 것부터 들어보셈
    \item  수학은 계산부터. 남들이 뭐라고 하든 말든 중1부터 멈춰있다면 기탄수학부터. 그런데 이거 시험모드로 100점 맞는 걸로 가셈
\end{enumerate}
\vspace{5mm}



\section{[상담 009] EBS 교재, 기출, 교과서만으로 100점}
\href{https://www.kockoc.com/Apoc/495366}{2015.11.17}


\vspace{5mm}

    16수능을 치룬 현역생입니다. 올해 초에 어찌어찌해서 알게 된 콕콕에서 칼럼글을 보고 깨달은 바가 많아 말씀하신 내용들을 지키려고 노력했죠. 다른 애들은 온갖 인강 강사의 책을 펴놓고 인강을 보면서 공부할 때 아폭님이 말씀하신 대로 \textbf{EBS 교재, 기출과 교과서에만 최대한 집중하려고} 했죠. 그렇게 해서 수학은 살면서 100점 처음 맞아보네요 ㅎㅎ. 과탐도 모의고사 칠때마다 3등급 이상을 못받았었는데 지금 컷을 보니 다 2등급이네요. 제가 기대했던 목표치를 달성할 수 있던건 아폭 님의 칼럼 덕분이었습니다. 1년 동안 칼럼 보면서 느낀 바가 많았습니다. (물론 직접 댓글이나 게시물을 달면서까지 콕질을 안했지만요.) 그리고 터무니없는 6,9월 성적보다 더 괜찮은 성적을 받게 되서 다시 한 번 감사의 말씀 드립니다.
    \vspace{5mm}
진짜인지 가짜인지 모르겠습니다만(양해하시길, 수험사이트에는 분탕, 작전, 마케팅 하도 별의별 병신들이 널려서리)
진짜라면 축하드립니다.
\vspace{5mm}

단, 그건 님이 노력한 게 99$\%$이니 \textbf{스스로를 대견히 생각하시길 바랍니다.}
그건 제 칼럼 덕분이 아니라, 올바른 방법론 믿고 자기를 믿고 달려간 과학적인 결과입니다.
제가 원하는 건 다 하나입니다. 그렇게 노력하셨다면 더 올라가시고, 부디 후학들에게 님이 생각하는 올바른 길을 제시해주심요.
다른 분들은 제 칼럼 읽는다 뭐한다 그런데 집착하지 말고  저 분의 사연에서 "\textbf{EBS 교재, 기출과 교과서에만 최대한 집중하려고}" $\rightarrow$ 이걸 눈여겨보시길 바랍니다.





\section{[상담 010] 예체능의 경우}
\href{https://www.kockoc.com/Apoc/496692}{2015.11.17}


    \vspace{5mm}

    어렸을때부터 미적감각에 센스가 있었고 나름 디자인적센스가 있다는 소리도 들어서 저는 자연스레 미대에 입학을 했습니다.
    그런데 막상 미대에 들어와보니 \textbf{제가 생각한 미래와는 많이 다른 현실을 알게되었습니다.}
    사막에서 바늘찾기 급의 대기업TO, 내가 생각한 디자인이 아닌 클라이언트의 요구에 맞춰갈수밖에 없는 디자인계의현실, 나보다 뛰어난 동기들...
    이런걸 느끼면서 전공에 대해 회의감을 느끼고 정도 떨어질때로 떨어질때쯤 우연히 울지마톤즈라는 다큐멘터리를 보게되었고 이태석 신부님의 헌신적인 모습을 보고 저도 의대에 가서 사람들에게 의료봉사를 하고 싶다는 새로운 꿈을 가지게 되었습니다. 혹자는 그럼 수능을 다시봐서 의대에 도전하면 되는거아니냐 라고 말할 수 있지만 지금 제 상황때문에 여러가지 걱정이 앞섭니다.
    만약 내년에 수능을 다시 보게 된다면 4수를 하게 되는것인데 제 나이도 그렇고 군대도 그렇고 하지만 무엇보다도 제가 바뀔수있을까하는 걱정이 듭니다.사실, 현역때 그렇게 의지가 좋은 놈은 아니었습니다. 그렇기에 현역수능에서 실패를 했고 재수를 하는 동안에도 언수외 성적은 그닥 오르지 않았고 사탐성적과 실기성적으로 지금의 대학에 붙었습니다.
    \vspace{5mm}

    \begin{enumerate}
        \item  제가 군대를 갔다오는게 나을지 조언좀 부탁드립니다.
        -근데 군대를 간다해도 언제갈 수 있을지 미지수입니다.ㅠ
        
        \item 수능을 다시보게 된다면 재종을 가는게 나을지 아니면 독학재수학원을 가는게 나을지..
        \vspace{5mm}
        
        \item 과탐선택에 도움을 주셨으면합니다.
        -현역 문과시절 사회문화, 세계지리 같은 뭔가 센스를 요구하는 과목보다는 그냥 우직하게 암기를 하는 법과정치, 역사과목등을 선호했습니다. 과탐을 선택하고자 하니 어떤 내용이 있는지도 모르겠고 각 과목이 어떤 감각(?)을 요구하는지 모르겠습니다.
    \end{enumerate}

    \vspace{5mm}

    +) 아! 제 성적에 대해서 얘기하자면 (이런 성적이 무슨 의대냐! 라고 말하실 수 있겠지만은ㅠ) 방금 풀어본 요번 수능 국어 4등급 영어 4등급이 나왔습니다. 수학은 제가 미대 준비생이었지만 학교수업을 충실하게 들어서 기본적인 내용은 알고 있습니다.(걍 교과서로 수1,미통기 2번정도 돌릴 정도..)
    \vspace{5mm}

    늦은 시간에 글을 써서 죄송하지만 이 글을 보신다면 답장 부탁드립니다.
    \vspace{5mm}

우선 말씀드릴 건 콕콕의 레벨지수에 따라 당연히 '차등'을 둬서 대접할 수 밖에 없습니다.
LV 0인 사람이 자기 이익만 채우고 가는 경우를 막기 위해서입니다. 그래서 일부러 상담도 늦게 답변합니다만
끈기가 있으면 뒤늦게라도 보지 않나 싶겠지요.
\vspace{5mm}

아마 이런 질문하겠죠.
"야, 너 만인을 위한다 하지 않았어"
"그런게 어딨어/ 난 엘리트주의자야. 가짜 엘리트가 엘리트라고 깝치는 건 싫다"
\vspace{5mm}

제가 그럴 의무는 없죠. 저는 공부하려는 사람이나 노력하는 사람이나 인정해주지
그렇지 않은 데 그러는 척 하는 사람은 무시해버립니다.
그리고 상담내용은 상원이 아닌 한 무조건 게시판 공개하는 게 방침입니다. 익명처리는 분명히 하겠지만요.
\vspace{5mm}

일단 이 분. 모든 인생판단을 '즉흥적'으로 하고 있으십니다.
다큐멘터리만 보고 의료 쪽으로 간다.... 그거 흔한 의대로 전공을 바꾸는 이유인데 이래서 잘된 사람도 못 본 것 같습니다.
아니, 사실 봉사하고 싶어 의대 간다는 사람 중에서 그 초심 지키는 사람 어딨나요?
저희 때도 의대간다는 사람 보면 인터뷰 내용 지킨 사람 없고 다 부자되서 떵떵거리며 누리기만 하더구만.
\vspace{5mm}

\begin{enumerate}
    \item 빨리 갔다오는 게 좋습니다.
    지금은 공부가 안 되는 시간만 생길 것이니 그냥 군대로 보내는 게 낫죠
    \vspace{5mm}
    
    \item 재종 가세요.
    독학재수가서 공부할 능력이 있으면 진작 갔죠.
    그러나 재종 간다고 해서 무조건 합격하는 건 아닙니다.
    하지만 공부해 본 적 경험이 없으면 한번 가보는 것 - 2$\sim$3달 다녀보는 건 권합니다.
    거기 시스템이 좋아서라기보다는, 공부하는 사람들을 보고 배울 수 있고 자기 시스템이 개판인 걸 확인해볼 수 있어서입니다.
    \vspace{5mm}
    
    \item 이 질문을 왜 하는지 모르겠습니다만(지금 이건 고민할 단계가 아닙니다만)
    보나마나 1년 공부하면 혹시 될까... 할 건데 말씀하신 케이스로는 확률이 매우 낮습니다.
    아마 재종가서 부리나케 해서 4$\sim$5월에 끝내지 않으면 불가능해요.
    굳이 선택하라면 지금으로서는 물지 선택하는 게 그나마 무난할 겁니다만.
    \vspace{5mm}
    
    \item 2등급 나와도 말씀하시는 진로에는 턱없이 부족합니다. 안정적인 1등급 나오면서
    남들에게 그 과목을 설명할 수 있어야 1년 내에 가능합니다.
    \textbf{일단 공부 안 한 분들일수록 너무 시험을 만만히 보는 경향이 있습니다만}
    까놓고 말하면 수험 가지고 장사하는 학생저자건 업자건, 그리고 저조차도 수능보라고 하면 장담하기 힘들걸요.
    시험을 잘 치르는 사람들은 평균적으로 1.5년 정도 정말 무섭게 달려와서 공부관성만으로도 사람 100명을 죽일 수 있는 자들입니다.
\end{enumerate}
\vspace{5mm}

왜 시험에 실패하냐? 간단하죠. 멀리 있는 산일수록 작아보이거든요.
산에 오르지도 않았으니 숨가쁠 리도 없고, 작아보이니까 너무 만만해보이죠.
\vspace{5mm}

그래서 다 쉬워보여요. 아, 내가 공부만 하면 의대에 가겠지.
산중턱에 오르는 순간 다 때려치우고 포기하고 싶다로 가죠.
\vspace{5mm}

현실적 조언을 드리면 지금 당장 재종을 가고 미친 듯이 하지 않으면 답이 없습니다.
무슨 꿀조언, 꿀교재. 그딴 게 지금 어딨어요.
꿀조언이라고 해보았자 다 뻔한 소리이고, 그리고 꿀교재?
시험 전에 그렇게 팔아대던 교재들 올해 시험 과목 난이도 적중도 못 시키고
발가락 닮은 것 가지고 적중했다고 야단법석 떠들면서 '또 호구가 되어주세용'이러고 있는 판이죠?
\vspace{5mm}

현실적 조언은 군대부터 다녀오는 겁니다. 왜냐면 군대가 걸리는 이상 공부도 집중할 수 없기도 하고
우선 내년 입시, 올해 출제 경향으로 보아서 매우 불확실해졌습니다.
한국사 추가되었고 수학 출제 경향도 알 수 없습니다.
\vspace{5mm}

그래도 빨리 하고 싶다면 돈 생각하지말고 그냥 재종 따라가는 게 답입니다.
돈이 없다고 하면 ebs 따라가면 될 것이지만 자제력과 공부 관성의 문제가 걸리지요.
\vspace{5mm}

요약하면
우선 군대부터. 이거 다른 상담도 한 적이 있는데 최소한 군대 빨리 가서 후회한 경우는 단 한 건도 없습니다.
작년에도 이런 상담해서 강경하게 군대부터 가라고 했는데 군대 빨리 가라 한 걸 고맙다는 쪽지들은 받았네요(...)
다녀온 다음 군자금 모집해서 재종학원 가든가
아니면 EBS 인강 위주로 죽어라 공부한 다음 사설 인강 중 좋은 것만 골라 잡고 마무리 잘하는 게 해답입니다.
\vspace{5mm}






\section{[상담 011] 하밀카르님의 경우}
\href{https://www.kockoc.com/Apoc/496696}{2015.11.17}


        
    \vspace{5mm}

    안녕하세요 ㅎㅎ 하밀카르입니다
    저를 아마 간간히 보셨을 겁니다
    사실 지금까지는 제가 매우 게으른 탓에 공부를 제대로 안했다 봐도 무방할 거 같습니다
    왜 자퇴를 그렇게 반대하는지 몸소 체험했네요
    이리저리 핑계는 대지 않겠습니다 제가 못난탓이니..
    그래도 아직 시간이 남아있는 탓에 포기할 수는 없으니 apoc님께 간절히 도움 요청합니다
    \vspace{5mm}

    국어는 박광일의 기출분석을 다 듣고 1년내내 기출뺑뺑이를 돌리려 합니다
    인강은 부족한 부분..  생기면 듣고(안들을 듯 하네요) 대종쌤 보편세만 들을겁니다
    \vspace{5mm}

    영어는 상변쌤 구문커리 타고, ebs 돌릴 생각입니다 어휘끝도 꾸준히 돌리고요
    팝송을 달고 살아서 그런지 신기하게 독해는 술술되는데 단어가 잡네요
    \vspace{5mm}

    생1은 큰 걱정은 안됩니다 교과서+기출 위주로 가고
    한종철 프패가 있긴 한데 개념 돌리고 유전 파트만 심화강의 들어보려 합니다
    \vspace{5mm}

    지1도 내신때 했어성.. 생1처럼 하면 될 듯합니다
    \vspace{5mm}

    문제는 수학입니다
    제가 초중고 모두 수학 문제집 한권을 제대로 풀어본 적 없는 게으름뱅이 빠가사리일 뿐만 아니라
    자퇴하기전 마지막 수학시험에서 30점 나온 똥멍청이입니다
    신기하게 모의고사는 1등급이었는데, 그건 다 등급컷이 기형적으로 낮아서 그런겁니다
    제 수학 머리는 딱 1을 알려주면 1을 아는 수준입니다
    계산 더럽게 못하고 공간감각 꽝입니다
    에구구..
    아 이렇게 쓰니까 희망이 없어보이네요
    \vspace{5mm}

    제가 끈기하나는 죽입니다
    상받고 싶어서 밤새서 파이500자리까지 외워봤고(대체 왜 했는지 이해불가..이상한 잠재력이 있어요)
    1학년때 친구가 놀려서 문제집 6권 풀고(수2 앞부분) 교과서랑 프린트 10회독 가까이 했는데
    86점 나왔습니다
    물론 그 후로 문과로 옮긴다 다시 이과로 옮긴다 학교를 어쩌구 저쩌구.. 휘청휘청.. 이꼴이 났지만..
    별 수 있나요 이제라도 하는수밖에요
    \vspace{5mm}

    재수는 정말 눈꼽만치도 생각하지 않고 있습니다 목표는 의대입니다
    검정고시 공부는 쉬는시간에 틈틈히 하면 될 것 같습니다
    개인 사정으로 다음주까지는 시골에 있을텐데, 그 이후로는 apoc님 말씀대로 공부를 시작해보려 합니다
    수학개념은 생각의 질서로 빨리 돌려버리고
    교과서로 다시 틀 잡고 아니 그냥 교과서는 외워버리려구요 머리에 남지를 않아서..
    국어, 영어가 '상대적으로' 튼튼합니다 국어는 작년a 100 작년b 96 이었고 영어는 ebs안보고 97이었습니다
    둘다 하루에 1시간 30분 이상 투자하지 않으려구요
    \vspace{5mm}

    강필선생님의 ox학습법으로 쎈을 하루에 최소한 100문제는 풀 수 있을 겁니다
    월화수목 양치기
    금 친구 스터디(전 x지우고 어휘끝)
    토일 양치기+개념총복습
    \vspace{5mm}

    이런 식으로 쎈 - rpm - 일품
    이걸 아마 3월달까지 끝내고 그 후에 기출을 풀려고 합니다
    \vspace{5mm}

    쓰고보니 답정너네요... 이런식으로 공부하면 적어도 시간낭비는 아니겠죠??
    수리논술도 생각하고 있는데 저같은 금붕어가 할 수 있을지 모르겠습니다
    긴글 읽어주셔서 감사합니다 칼럼 정말 잘보고 있습니다
    감사합니다:D
    \vspace{5mm}

(상담사례 게시글로 올리시던데, 저도 올리셔도 상관없습니다^^)
\vspace{5mm}

마지막 줄을 꽤 후회하실 겁니다. 왜냐면 전 상담할 때에는 끝까지 후벼파는 스타일이니까요.
\vspace{5mm}

우선 이 케이스는 주변으로부터 '산만하다'라는 평가를 상당히 받았을 것이며
사람들과의 관계는 그리 원만하지 않았을 겁니다가 아니라 매우 확정적입니다.
이건 개인비난은 아닙니다. 저도 비슷한 상태를 겪어보았으니까요.
\vspace{5mm}

우선 이 경우는 공황장애 비슷하게 갈 수도 있고, 집중력이 강하긴 한데 그걸 통제하지 못 합니다.
또한 공부하든 뭘 하든 가슴이 두근거리면서 불안감을 많이 느꼈을 것이며 그 때문에 오프에서 주변인들과 소통이 잘 되지 않았을 겁니다.
본인은 매우 선량하고 착한 학생이지만 오래 앉아서 공부하기 힘들었을 것이며 사람들과의 소통에도 다소 문제를 느꼈겠지요.
\vspace{5mm}

거기다가 이런 케이스는 강박증세도 있어서 완벽주의 집착이 매우 강합니다. 물론 이건 전혀 도움이 되지 않죠.
심하면 긴장을 자주 하기 때문에 신체의 특정부위가 이유없이 아플지도 모릅니다.
\vspace{5mm}

자, 그럼 봅시다. 하밀카르님은 지금도 그럴지 모르지만(일주일 전에 받은 쪽지군요)
이건 전체적으로 지나친 강박과 억압에 쌓인 것 같습니다.
\vspace{5mm}

$\sim$ 해야한다... 이것부터 버리십시오. 자신을 조이는 것, 그거 아무 소용이 없습니다.
지금 님은 스스로가 이자를 늘린 빚에 시달리고 있습니다.
자퇴, 검정고시한 건 모든 걸 리셋하고 다시 시작하자, 과거 빚을 청산하고 모라토리움 선언하고 새출발하면 된다 그랬던 것입니다.
하지만 지나고보니 그것도 안 된다는 것을 확인하셨을 겁니다. 좌절감은 상당히 쌓여있을 테고 지금도 속으로는 분노가 뭉쳤을 겁니다.
\vspace{5mm}

그럼 왜 그런 일이 벌어지느냐.
그건 능력의 문제라기보다는 생각하는 방식과 감정의 문제입니다.
님은 지금 반드시 $\sim$ 에 부합해야 한다라는 강박이 매우 심한 상태입니다.
이 상태로는 공부가 될 수 없습니다.
제 말을 믿기 싫겠지만, 이런 사례는 제가 한두건 본 게 아닙니다.
제가 조언을 했음에도 끝까지 듣지않고 자기 고집대로 가다가 성격 파탄나고 아주 이상해져서
인터넷에서 분탕질치는 그런 사례도 있었기 때문에 제가 강경하게 경고드리는 겁니다.
\vspace{5mm}

우선은 의대에 가야한다라거나 반드시 $\sim$ 해야한다라는 것을 다 지우세요.
그리고 지금 본인이 지고있는 짐부터 다 벗어야합니다. 목표를 최소한으로 해야해요.
대학입시 생각하지말고, 우선 검정고시'만' 공부하십시오. 대신 검정고시를 수석급으로 공부하시길 바랍니다.
님이 확실히 수석이 나온다고 확신이 들고 그게 검증될 떄까지 검정고시 공부를 해서 우선 합격하시길 바랍니다.
검정고시만 하기에는 너무 시간이 남아돈다면, 국어 영어 수학 한과목만 잡으세요.
\vspace{5mm}

그리고 제가 정말 경고드리는데, 방금 말한 검정고시 집중, 그리고 정말 시간이 남아돌아 어쩔 수 없다라고 하면
국영수 한 과목만 잡아서 문제집 한권'씩'만 끝내길 바랍니다.
만약 님이 여기서 하나라도 욕심부리면, 제가 솔직하게 말하죠. 님 인생 끝장납니다.
저주가 아니라 지금 그렇게 밖에 되어갈 수 밖에 없는 구조입니다.
10대 이후로 욕심은 냈지만, 님이 노력한만큼 하나라도 \textbf{제대로 성사시키거나 성취한 경험이 거의 없을 것입니다}.
그래서 님 뇌도 주인을 배신하고 싶어하죠. 만약 주인이 더 욕심내다가 아무 것도 못 이루면?
빡친 뇌는 그 주인을 나쁜 방향의 중독이나 타락, 폐인화, 그리고 심지어 사신을 초청할 수 도 있습니다.
\vspace{5mm}

이건 과장이 아닙니다.
그러니까 일단 검고만 집중하고, 그게 시간이 남으면 국영수 한 과목의 문제집을 한권씩만 끝장내며
절대 강의는 듣지 마십시오. 이 얘기보면 비웃을지 모르지만, 최근에 게시판 보면 제 얘기가 맞았다라는 걸 확인하는 글들이 있을 겁니다.
제 얘기 안 듣는다고 해도 제가 손해볼 건 없습니다.
하지만 또 하나 이야기드리면, 전 미운 오리 새끼 고기를 백조고기로 팔아먹는 것도 잘 하지만
남들이 망하는 것도 의외로 잘 맞춥니다(... 꼭 이런 것만 ...) 그런데 지금 님 커리대로 가면 저건 100$\%$ 망합니다.
\vspace{5mm}

계획이라는 걸 차라리 안 세우는 걸 권하겠어요. 그 계획을 세우는 님이 지금 현재 평정심이 있다고 보이지는 않아서입니다.
현재는 검고 성공시키고, 문제집을 한권씩 다 공략하는 그런 식으로 '정화'를 해야합니다.
\vspace{5mm}

만약 이대로 해서 구조조정을 하고 그걸 인증한다면 그 다음에는 제가 지시를 드릴 수 있고
콕콕 내에서도 님이 도움을 청할 수 있는 사람들은 많이 있습니다.
\vspace{5mm}

그리고 여담이지만 또 화가 나는 게
저거 보니까 수험사이트의 상술이나 선동에 영향받은 티가 사실 너무 자명하네요.
그렇게 해서 한 사람 인생 말아먹는 과정이죠.
\vspace{5mm}






\section{[상담 012] 9수생의 경우}
\href{https://www.kockoc.com/Apoc/496700}{2015.11.17}


        
    \vspace{5mm}

    안녕하세요 직접 쪽지를 드리는 건 처음입니다
    2008년도 수능부터 2016년도 수능까지 한해도 빠지지 않고 수능을 응시해 온
    장수생이라고 쓰고 인간 쓰레기라고 읽는 종자에요...
    \vspace{5mm}

    제 인생역정에는 별 관심이 없으실테니 짧게 요약하자면
    실력도 없는 주제에 자존심만큼은 하늘 끝까지 치솟을 정도로 높아서 서울대 아니면 안 가겠다고 깝치다가
    수능 응시횟수만 차곡차곡 쌓아서 결국은 지금처럼 되어 버렸습니다
    \vspace{5mm}

    그나마 올해는 서울대에 꼭 가야겠다는 강박을 버렸지만
    나이는 나이대로 처먹고 꼴랑 한학기 다니고 장기휴학중인 대학은 다시 다니기도 뭣하고
    할 줄 아는 게 수능공부밖에 없어서 한의대를 목표로 공부를 다시 했습니다
    군대는 4급으로 나왔기 때문에 다소 늦게 가도 부담이 덜했거든요...
    \vspace{5mm}

    아폭님 방법론을 처음부터 알았으면 일찍 공부를 시작하는 건데
    작년에 원서라도 잘 써보겠다고 수험사이트에서 세월을 보내다 11$\sim$2월을 날리고 3월부터 공부를 했지요
    그나마 이건 그동안에 비하면 빠른 편이었고 평소에는 6월부터 시작하거나 늦으면 8월에 시작하는 경우도 있었습니다
    보통 그 시간에는 책값과 용돈을 벌기 위해 아르바이트를 하고 수험사이트 눈팅이나 하면서 허송세월을 보냈어요
    \textbf{'내가 N년을 공부했는데 8월부터 시작하면 충분하겠지'라는 잘못된 마인드 때문이었죠}
    \vspace{5mm}

    수능 경험이 많다보니 머리를 수능에 맞게 최적화시키려면 일찍 일어나야 한다는 것을 깨달아서
    \textbf{몇년 전부터 6시에 일어나는 걸 생활화했었는데 올해는 꼬박꼬박 5시 반에 일어나서 공부를 했구요}
    \textbf{3$\sim$6월에는 일주일에 순공부 평균 6시간쯤(한번 앉으면 7$\sim$8시간을 하긴 했는데 힘들면 일주일에 이틀씩 놀고 그랬음)}
    \textbf{6월부터는 좀 체계가 잡혀서 일주일에 하루 쉬고 순공부 8시간 이상을 유지했구요}
    \textbf{9월 모의 이후부터 수능 직전까지는 9시간 이상을 유지했습니다}
    저는 요즘 학생이 아니라서(....) 컴퓨터를 했으면 했지 스마트폰 갖고노는 것을 별로 좋아하지 않기 때문에
    공부하러 나갈 때는 아예 집에 두고 다녔습니다 어차피 폰으로 소통할 친구도 없구요
    \vspace{5mm}

    이런 식으로 공부를 해서 올해는 그래도 '예전보단' 공부를 많이 했으니 더 잘 나오겠지 해놓고 시험을 쳐 보니 오늘 결과가.....
    익명이니까 뻥튀기 전혀 없이 솔직히 적자면 국어 98 \textbf{수학 80} 영어 88점이 나왔습니다
    위에서 쓴 것처럼 공부를 저딴 식으로 했어도 일정 점수는 유지됐었는데 지금은 나락으로 떨어졌습니다
    1교시부터 눈앞이 하얗게 되면서 심장이 쿵쾅거리면서 글이 눈에 안 들어오더라구요
    제가 글 읽는 속도가 빠른 편이라 모의고사 볼때는 지문 다 읽어도 40$\sim$50분이면 1바퀴 충분히 돌리는데
    이번엔 문제 먼저 읽고 지문을 어거지로 발췌해서 읽는 스킬로 간신히 시간 맞춰 풀었습니다
    \vspace{5mm}

    2교시도 분명 지금 풀면 웬만한 문제는 다 풀 수 있겠는데 시험 보는 중에는 워낙 떠는 바람에
    다른 쉬운 문제에서도 버벅거리면서 시간을 까먹는 바람에 아예 21번 30번은 손도 못 댔습니다
    졸릴까봐 점심을 아예 걸렀는데도 간밤에 잠을 3시간밖에 못 잔 탓인지 약간 잠이 와서 3교시도 저 꼴이구요
    (갑자기 뛴 난이도도 관련이 있구요)
    \vspace{5mm}

    4교시 되니 긴장도 다 풀리고 머리도 맑아져서 그럭저럭 수월하게 5교시까지 정리하고 나왔습니다
    밥맛 떨어지는 실패담은 별로 듣고 싶지 않으실 테니 여기서 줄이겠습니다...
    \vspace{5mm}

    제가 드리고 싶은 질문은 이것들입니다
    \vspace{5mm}

    \begin{enumerate}
        \item 제가 지금같은 밥만 축내는 가축이 아니고
        '사람답게' 살기 위해서 공부를 계속 해야한다고 생각하시나요?
        아니면 육체 노동을 하거나 기술을 배우는 쪽이 낫다고 보시나요?
        \vspace{5mm}
        
        \item 공부를 게속 한다면 수능을 또 봐서 한의대에 가는 게 낫다고 생각하시나요,
        아니면 7$\sim$9급 공무원 시험을 준비하는 게 낫다고 생각하시나요?
        \vspace{5mm}
        
        \item 처음엔 안 그랬는데 연차를 일년씩 쌓으면서 위에 서술한 시험 도중 겪는 불안장애?
        시험공포증 비슷한 것이 생겨서 5년 전부터 지금까지 시험 칠때마다 항상 저 모양으로 벌벌 떨면서 문제를 풉니다
        덜 중요한 시험(토익이나 한국사능력시험)을 쳐보면 전혀 저러지 않구요...
        시험 며칠 전부터 잠을 못 자고 간신히 잠들어도 2$\sim$3시간만에 깨 버리는 이상한 증세도 생겼구요
        (근데 이 증상은 시험 한두달 전부터 마스터베이션을 끊어서 그런것일수도 있어요 일본인이 쓴 오나금 수기에서 나온 초사이어인 모드랑 굉장히 흡사하더라구요)
        \vspace{5mm}
    \end{enumerate}

    이걸 극복할 수 있는 방법을 혹시 아시는지요? 정신과에 가 봤자 그냥 약이나 주고 스트레스 받지 마세요 하고 끝이더군요
    청심원을 마시고 가거나 정신과에서 준 신경안정제를 먹고 가면 덜하긴 한데 이러면 오후부터 머리가 몽롱해져요
    \vspace{5mm}

    덤으로 공부하다가 이를 악물어서 볼 안쪽이랑 혀가 톱니모양처럼 되는 질병도 얻었습니다
    \url{http://health.chosun.com/site/data/html_dir/2010/06/21/2010062102073.html} 대략 이런거....
    이것들 다 재수하기 전엔 없었는데 N수 회차 쌓아가면서 생긴 증상들이라 혹시 비슷한 경우를 아시나 해서요
    \vspace{5mm}

    쓰레기 루저들이 징징거리는거 별로 안 좋아하시는 건 알지만 정말 누구에게라도 조언을 듣고 싶었습니다
    제가 원래 외골수여서 학창시절에도 친구가 별로 없었지만 수능 횟수 쌓아가면서 이제 연락하는 사람이 아예 없어서요
    부모님이 거는 기대도 저에게는 무겁고 또 다시 실망시켜 드리는 것은 마음아프고
    친척들(특히 외가쪽)은 제가 실패해서 낙오자가 되어버린 걸 내심 고소해하는 눈치입니다
    \vspace{5mm}

    어쩌면 이 생활을 유지하는 게 내심 '편하다'고 생각해서 제가 아직도 이러고 있는게 아닌가 하는 생각도 듭니다
    하지만 이제는 달라지고 싶습니다. 달라지지 않으면 안 되구요
    예전처럼 완벽하게 엘리트 코스를 척척 밟아 성공해서 세상에 군림하겠다$-$는 망상은 버렸지만
    어쨌든 저같은 낙오자도 노력하면 보통 사람처럼은 살 수 있지 않을까 하는 희망만큼은 버리지 않았습니다
    \vspace{5mm}

    쓰던 중에 한번 날아가서 다시 씁니다.....
    정말 남에게 드러내기 부끄러운 인생이지만 조심스럽게 말씀드려 봅니다
    귀찮으시더라도 몇 마디 꼭 부탁드리겠습니다
    \vspace{5mm}

    하지만 이것만큼은 꼭 조언을 받고 싶습니다만 시험 보면서 겪는 불안에 대해서는 극복할 수 있는 방법이 없습니까?
    시험을 9번을 보다보니 한번은 페이스가 무너져서 시험 며칠 전부터 스타크래프트나 하다가 시험보러 갔던 적도 있습니다만
    우습게도 그때는 오히려 긴장이 덜 되더라구요 그 당시 점수도 지금보다는 좋았었죠
    \vspace{5mm}

    이건 그냥 공부를 많이 하는 걸로는 답이 안 나오는 부분이라서 궁금합니다
    공부를 많이 하면 패턴에 익숙해져서 기계적으로 풀 수 있는 문제는 떨면서도 어떻게든 푸는데
    조금이라도 '생각'을 해야 되는 문제들은 그냥 말립니다
    \vspace{5mm}

    수능을 접고 공무원 시험을 응시하려고 해도 저 증상이 또 도지면 답이 없습니다 ㅠㅠ
    \vspace{5mm}

    저에게 개인적으로 답변해주시는 것이 내키지 않으신다면 관련 글이라도 써 주셨으면 대단히 감사하게 생각하겠습니다
    이건 다른 수험생들도 (특히 N수생들) 겪을 수 있거나 겪고 있는 문제일테니 저 말고 다른 학생들에게도 많은 도움이 될 거에요
    \vspace{5mm}

브레인 프리즈 상태입니다.
신림동의 장수생들에게 자주 보이는 것인데
공부를 3년 이상 하면 점차 드러나죠.
\vspace{5mm}

왜 그러느냐. 이건 오래 공부하는 건 '감금상태'나 마찬가지입니다.
9년동안 공부만 했다는 건 어떤 범죄를 저지르지 않았는데도 징역 9년을 살았다는 것과 근사한 이야기입니다.
그것도 독방에서 말이지요.
\vspace{5mm}

그래서 이렇게 공부하면 정신적인 불안감이 오는 건 당연합니다.
그렇기 때문에 절대 혼자 공부해서는 아니되고, 공부하는 사람들이 있는 곳에 들어가서 소속감을 느끼고 싱크로나이징(?)을 해야합니다.
그런 리듬을 타야 정신적으로 상처입거나 맛가지 않지, 안 그러고 아 나는 성공할꺼야라고 하고 혼자 공부한다?
맛갑니다요.
\vspace{5mm}
\begin{enumerate}
    \item 육체노동을 하고 기술을 배워본 적이 있으십니까.
    그 질문의 답이 중요한게 아니라 질문자 분이 "해보았느냐"가 관건입니다.
    \vspace{5mm}
    
    분명한 사실은 공부 안 하고 걍 기술 배운다.... 치고 손이 안 부드러운 사람들이 없다는 겁니다.
    자기들이 알바도 뛰어본 적도 없고 힘든 일을 해본 적이 없으니까 다 만만해보이는 것이죠.
    노가다로 뛰거나 기술로 돈버는 사람들이 정작 '공부'는 강조해대고 자녀들을 좋은 대학에 보내려고 하는 걸 생각해보면
    \vspace{5mm}
    
    이거 할까 저거 할까는 함부로 할 말은 아니며
    정말로 그런 의지가 있다면 벌써 기술 배우고 육체노동하면서 주경야독을 했겠죠.
    육체노동하고 기술배우면서도 공부하는 분들 있습니다.
    \vspace{5mm}
    
    \item 그건 님이 결정하셔야 할 문제가 아닌가요?
    공무원 시험이 좋다? 한의대가 좋다? 이거 디씨에서 흔히 나오는 얘기 아닙니까.
    결국 어느 쪽도 제대로 노력하지 않고 공부하다가 안 되면 내가 이 길이 아닌가봐라고 하면서 자신의 게으름을 정당화하는 사람들이요.
    정말 공부하는 사람이면 둘 다 합격합니다. 공무원 시험에 합격할 사람이면 수능도 잘 보고, 그 역도 마찬가지이죠.
    반대로 어느 쪽도 하지 못 하는 사람은 수년 지나도 $\sim$ 로 갈까라는 진로놀이로 자신의 정신적 고통을 회피하려 하죠.
    어느 쪽이든 본인이 원하는 쪽으로 가십시오. 제가 보기엔 이건 정답이 없습니다.
    \vspace{5mm}
    
    \item 3-1 반드시 외출하셔야 합니다. 그리고 이거 약은 '운동'입니다.
    헬스 같은 그런 것 말고, 동네 아줌마들이 집단으로 하는 에어로빅, 요가, 필라테스, 댄스. 이런 데 참여하십시오.
    그걸 3개월 정도 하면 오랫동안 혼자 은둔생활하며 공부하며 생긴 병은 낫습니다.
    정신과 간다 그런 거 별 소용없습니다. 혼자서 감금생활하면서 얻어진 병입니다. 이건 사람들과 부대끼고 수다떨고 땀흘려야 낫습니다.
    \vspace{5mm}
    
    이렇게 말하면 "저 시간에 공부하면 되지 않느냐"라고 또 합리적인 척 하는 간사한 생각이 들죠.
    그래서 망하는 겁니다.
    그 상태에서는 우선 병부터 치료하세요.
    저렴하게 가서 같이 몸흔들고 땀흘릴 수 있는 곳 당장 찾아 내일부터라도 하시길 바랍니다.
    일주일 정도 땀을 흘리면 왜 진작 그걸 안 했을까라는 후회감이 들겁니다.
    \vspace{5mm}
    
    이건 제가 직접 경험해보았기 때문에 말씀드릴 수 있는 것입니다.
    제 인생에서 잘했다고 하는 게 동네 아줌마들 가는 요가교실에서 1년동안 몸단련한 겁니다(...)
    \vspace{5mm}
    
    \item 3-2 친척들이 그러면 아 그런가보다 생각하면 되고, 스스로를 꾸짖고 앞으로 올라가면 된다라고 생각하면 됩니다.
    시험 전 겪는 불안요? 그거야 못 떨치죠. 님은 도망만 쳤으니까.
    그럼 어떡하냐고? 불안해지셔야죠. 원래 시험은 불안한 겁니다.
    그게 무슨 말장난이냐고요?
    님이 겪는 그건 불안이 아니라, 패닉 상태입니다.
    불안하다라는 건 본인이 위험에 대비해서 긴장한 걸로 자신이 컨트롤할 수 있어요. 그 불안함을 받아들이기 때문입니다.
    하지만 님 상태는 조금만 불안해도 넋이 나가서 그 불안함을 받아들이려하지 않기 때문에 자신을 컨트롤하지 못하는 것입니다.
    원래 공부는 불안하다는 생각을 하시면 됩니다.
    \vspace{5mm}
\end{enumerate}

그럼 생각해야하는 문제는?
그거야 당연히 그런 문제만 자주 풀고 백지에 써대면서 훈련하면 되죠.
그런데 본인의 문제는 잘 아네요. '스타크래프트' - 게임중독 상태.
님의 뇌는 공부할 때마다 '불안함'을 강조해서, 그 주인이 게임을 하는 상태로 유도하고 있군요.
즉 공부에서 쾌감을 못 느끼고, 게임이나 술이나 여자로 쾌감을 느끼게 되는 상태면 답이 없습니다.
\vspace{5mm}

이 조언을 본다면 결국 게임도 무조건 금지하십시오.
과거는 일단 잊으시고 우선 당장 '집단'운동하는 데 가입해서 땀흘리세요.
아울러 공무원을 칠건지 수능쳐서 한의대를 갈건지는 본인이 선택하시되
돈이 있다면 학원에 가서 집단생활 하시고, 그게 힘드시다면 도서관이라도 가서 도서관의 공부분위기를 주도하십시오.
\vspace{5mm}

그러나 가장 중요한 건 시험당일날 상태를 보아도 그렇지만,
님은 결국 장기간 공부한다고 했다가 게임 등에 중독된 상태입니다.
게임, 술, 담배 등 기타 노력하지 않아도 뇌에 쾌감을 주는 건 무조건 차단하시길 바랍니다.
그런 걸 차단하지 않고서 불안감을 없애는 방법이 없느냐라고 물어보시면 안 되지요.
\vspace{5mm}






\section{[상담 013] 감기}
\href{https://www.kockoc.com/Apoc/496702}{2015.11.17}

\vspace{5mm}




    안녕하세요. 21살 수능망한 삼반수생입니다.... 문과구요. 수능 3 2 2 3 2 맞았습니다..네..망햇어요..... 그런데 너무 억울한게..감기에 걸린 상태에서 시험을 치뤗 거든요... 전날에 링거 맞고 하루종일 누워 있어도 낫질 않더군요.. . 감기 걸린 이유요?.. 저 2학기 6학점 반수고..(공부시간 정말 충분 했어요...하나는 애들이 대출해주고..) 룸메를 과 동기랑 햇어요 편하게. 근데 수능 3일전 10시반에 잠들고 갑자기 3시쯤 깻는데 제가 다 마르지 않은 이불을 덮고 있는겁니다.(룸메는 선의의 행동이엇어요...제가 이불차고 자는 버릇이 잇어서, 금방 빨래하고 건조대로 말린 룸메 이불을 그친구가 다마른주 알고 덮어준건데.....)전 뭐지....하고 상황 파악안되고 다시 잠들엇죠...(이때까진 왜 감기걸렷는지 몰랏어요.저 항상 도서관 갈때 무조건 마스크쓰고 다녓거든요. 옷 따시입고) 다소 무겁게 여느때와 다름없이 5시반에 일어나서 시뮬레이션하는데 평소보다 너무 힘든겁니다..전 이게 제 의지 문제인줄 알고 더 빡세게했죠....그런데 다음날 일어나니.....열이 진짜 상상초월 할정도로 올라가고...하........진짜 자살하고 싶엇어요....생각 해보니 아 그거 때문이구나...아...이렇게 내 노력이 끝이 나는 구나....... 수능 전날밤..이마에 물수건 몇시간 올려놓고 겨우겨우 잠들고.. 5시반에 깨어나 수능을 보러갔습니다. 반 포 기 한 상태로요. 가면서 제 1년이 떠오르는데.ㅈ살 ㅈ살 ㅈ살 외치며 갔어요....... 국어 비문학만 개털리고 나머지는 다맞아 3....(이 상태에선 이해가 불가 햇어요....비문학외는 다 가벼웟던 지문들이라..) 수학 계산 조온나게 말아먹고 20번 27번. 30번은 건드리지도 못하고 88 fail......... 영어 가장 자신 잇는과목....저 살면서 듣기 처음 틀려봣습니다 그것도 2개나요. 어?거기다 독해도 2개 더 틀렷네요. 빈칸3번째문제.문장삽입문제. ㅋㅋㅋㅋㅋㅋㅋ90점ㅋㅋㅋㅋㅋㅋ 생윤사 올림픽 정신으로 봤습니다. 더 이상 힘이 안나더군요. 9평 생윤사 99$\%$ 100$\%$ 수능 3등급 2등급 각각 -3문제 -2문제.. ㅋㅋㅋㅋㅋㅋㅋㅋㅋㅋㅋㅋㅋㅋㅋㅋㅋㅋ이렇게 제 노력은 감기 한방에 날라갔습니다! 차라리 재수때엿으면 좋아요 ! 삼반수 생각이라도 하지 근데 지금은!?!?!?!? 사반수요? 헤헤 어떡허죠?군대 언제가죠? 저 어떡해야 할까요? 제 지금 대학이요? 충북대라는 학교입니다 15수능 411맞고 그냥 장학금 받고 닥 반수 하러갓는데  ㅋㅋㅋㅋㅋㅋㅋㅋㅋㅋㅋㅋㅋㅋㅋㅋㅋ저 어떡하죠? 저점수로 옮기는 의미가 잇나요?  저 다시 '또'수능 봐야 하나요? 쟤 인생 왜이러죠? 충북대 간판으로 사회나오면 저 뭐하죠? .........위 몇줄은 넋두리구요..... 1.군대를 바로 가야할까요?.... 2.한번.....ㄷ...더?.....그럴만큼 가치가 잇나요 대학이?(아 물론 가치 있어서 했죠....근데 더이상...) 3.그냥..여기서 학점 관리하고 세무사준비할까요? 아......... 저 앞으로 어떻게 살아야할까요.......죽고싶네요


감기 조심하라는 글을 쓴 적이 있는데 운이 매우 나쁘셨더 것 같습니다.
그러나 그걸로 죽을 건 아니고, 그 정도로 죽는다고 해도 지옥경쟁률도 1:100인 시대입니다.
\vspace{5mm}

다만 일단 저건 감기 때문만은 아닙니다. 국어는 사실 비문학이 중요했으니 감기 때문에만 다 망했다고 보기 힘들고
수학은 20, 27번 나갔다면 이게 문제며
영어는 자신있다 하지만 그거 9평까지의 쉬운 영어 기준이 아니었나 생각해보아야하고
생윤사는 제대로 난이도 높여 통수 치지 않았습니까?
\vspace{5mm}

실패분석해보시면 이거 감기 이유는 20$\%$ 정도고 나머지 80$\%$는 평소 공부대로입니다.
내가 6평 9평 잘나왔는데... 이거 의미없어요. 6평 9평 잘 보다가 본 시험 말아먹은 케이스도 많고, 그 정반대도 많습니다.
심지어 제가 기대한 모 학생은 먹지말라던 고까페인 음료 열심히 홀짝거려서 제 기대를 벗어났(...)습니다만
정작 채점해보니 공부한 성과는 나더라는 것이죠.
\vspace{5mm}

\begin{enumerate}
    \item 군대는 그건 님 선택입니다. 그런데 빨리 가는 것도 나쁘지 않습니다. 언제든 가야하니까요
    \vspace{5mm}
    
    \item 내년에 해서 승산이 있다고 계산되면 가도 되는데. 이건 좀 아리까리하네요.
    내년 문과 수능 과목과 범위 보시면 되겟죠.
    일단 수학은 오히려 늘어납니다(집합과 명제 때문에) 한국사는 필수며
    사탐 과목도 바꿔야하겠고 국어는 통합입니다.
    만약 지금 님이 12월부터 하루 6시간 공부 달릴 수 잇다면 모를까 그게 아니면 군대를 권하겠습니다.
    \vspace{5mm}
    
    \item 세무사 공부는 쉬운 줄 아시죠?
    \vspace{5mm}
\end{enumerate}

요약하면 2가지인데
12월부터 하루 최소 6시간 찍고 공부할 수 있다면 1년 더 도전해도 됩니다.
그러나 그게 아니면 그냥 입대 신청하고 군대 갔다오고 그 전에 여행이나 하고싶은 것을 하는 걸 권합니다.
\vspace{5mm}





\section{[상담 014] 다시 가닥잡기 시작하는 장수생}
\href{https://www.kockoc.com/Apoc/501589}{2015.11.19}

\vspace{5mm}


    장수생입니다. 나이먹고 대학수험공부하는사람치고 이런저런 사연  없겠습니까만 그런 관심도없는 남의 사연 많이 들었을거라 짐작하니 생략하겠습니다.   저는 금욕적이고 외로운걸 잘 관조하는편입니다. 공부에 습관 붙이는게 힘들었지만 한번 잡고나니 7$\sim$9시에 독서실 입실해서 11시에 집에오네요. 물론 중간에 지쳐서 좀 쉬는날도 있지만 공부를 아예 놓진않고 그냥 일찍 집에 가는정도고 2,3개월 지속했으니 한순간 바람은 아닌것같습니다.  일전번에도 수학때문에 문의드렸습니다만 그때는 사칙연산빼곤 다항식도뭔지 몰랐는데 지금은 미적2 공부하고 확통과 기벡만 하면 되네요. 이제 앞으로 어떻게 공부할지가 요새 좀 막막하네요. 날로먹은 고1수학 기름칠좀 해야되는지 미적1도 진도만 나간것같고 생각보다 진도는 잘 못나가고있고....     그리고  생2지2는 어떻게 생각하시는지 궁금하네요. 두 과목 모두 화학1보다는 나은것같은데, 비현실적인 욕심인가요?

\begin{enumerate}
    \item 생2도 피하시길 바랍니다. 지2는 만점권을 가정한다면 할만하나 생2는 정말 최고수가 아니면 정말 답이 없습니다.
    지2하라고 했는데 생2 할 수 있다고 한 몇몇 수험생이 왜 그 때 안 말렸냐고 저에게 --
    \vspace{5mm}
    
    \item 수학은 천천히 가시길. 절대 빨리 가지 마세요
    기본교재 쎈 - RPM 돌리고 일품, 일등급, 라벨까지 다 돌려야 비로소 체화가 됩니다.
    마플 기출이 나오면 그것 한권만 돌리고 (일부 마음에 안 드는 해설은 네이버나 EBS 참조)
    그리고 8월 정도에 수리논술 문제 하나씩 쉬운 것부터 풀어보시면 됩니다.
    이것대로만 해도 점수는 잘 나옵니다. 할 수 있느냐가 문제일 뿐.
    \vspace{5mm}
    
    \item 앞으로 3월까지 대략 2$\sim$3번의 슬럼프가 옵니다. 아니, 반드시 거쳐야합니다.
    오랜만에 복귀해서 수학점수 말하면서 '말뿐인 조언'만 한 저에게 감사인사표하는 인증글 보시고 자신감 가지시길요.
    \vspace{5mm}
\end{enumerate}






\section{[상담 015] 고학생}
\href{https://www.kockoc.com/Apoc/501620}{2015.11.19}


\vspace{5mm}

수능 끝나자마자 운 좋게 바로 아르바이트 자리를 얻었습니다.
전에 말씀드렸던  xxx 선생님 사무실에서 인터넷 강의 교재 만들고 있습니다.
\vspace{5mm}

요즘 주로 하는 일은 공무원 xx 문제 해설지 만들고 있습니다..
일단 한달 한다고 말씀드렸는데, 교재 작업이 끝나면, 이것은 제 동생을 넣어달라고 부탁하였습니다.
다음 할 일은 xx 학원에 나가서
애들 시험지 주고 저는 제 할일하고 선생님 끝나면 다른 학원으로 같이 움직이는 일을 할 것 같습니다.
\vspace{5mm}

방금 아버지와 이야기를 했는데,
공부 해도 좋고 공부란 평생 하는 것 이니까. 어떤 공부를 해도 좋다고...
그런데 당신에게 더 이상 의지하지 말라고 하셨습니다.
이때 순간 아... 위에 써 놓은 아르바이트를 계속 해야하나 하는 생각이듭니다.
\vspace{5mm}

하게되면 제 황금의 3개월은 아마 날아갈 것 같습니다.
왜냐면 어... 겨울방학때 수업이 꽤 많고 방학 후에 한가 한것 같아요.
뭐 저랑 제 동생이 가르는 것도 방법이긴 한데...
일단 제가 선금으로 돈을 50만원 받았는데,
\vspace{5mm}

ㅈ저는 이 돈으로 질문하는 학원에 다니려고 했는데,
9수생 친구 이야기를 보니 차라리 이 돈으로 요가를 다니는 것이 낫겠다 라는 생각을 하고 있는 중 입니다.
제 계획은
5시반 기상노량진 xxx가서 xxx 선생님 강의 듣고
돌아오면 대충 10$\sim$11시 사이 아파트 독서실에서 공부하고 점심먹고 저녁먹고 공부하고 9시 30분 부터 요가
집에 와서 11시 반까지 공부
주말에는 질문 받는 학원 격주로 이렇게 하려고 했습니다.
질문 받는 학원 선생님은 일단 수 1 수 2 쎈수학 한번 빠르게 보고 오라고 하셨어요.
\vspace{5mm}

12월 중순까지 사무실에서 살까 최대한 집에 늦게 들어갈까
생각중입니다...
집에는 알바한다고 이야기 안했습니다.
제 동생 알바급여를 주택 대출로 갚고 해서...
참...그렇습니다.
번거롭게해드려 죄송합니다.....
\vspace{5mm}

아 참 그리고 릿딧밋에 관한 것인데, 제가 부대에서 릿딧밋 이야기를
그 서울대 로스쿨 다니는 친구랑 이야기했을 때는 그 친구는 닭잡는데 소잡는 칼 쓰는 격이니
차라리 영어공부를 더 하는 것이 좋을 것 같다 라고 하면서 이야기를 해서 관 뒀습니다.
\vspace{5mm}

그런데 제가 이번에 음... 비문학 마지막 지문 풀고 있을 때 20분 남아서 $-\_-$
저는 어이없었지만
제 실력이라 보입니다.
ㅇ올해 릿딧밋을 건드려야 할지 잘 모르겠습니다
\vspace{5mm}

자, 이 경우는 이렇게 생각하시죠
(기본교재 X 회독수 + 문풀량 X 문제난이도)고난이도 문제를 혼자서 설명하고 풀 수 있는 비율X10
대충 정리한 거지만 일단 수험은 저걸로 좌우됩니다.
\vspace{5mm}

쪽지만 보아서 상황을 잘 알 수는 없는데
노력을 많이 하지만 지나치게 학원에 의존하고 있다는 생각을 지우기 어렵습니다.
\vspace{5mm}

하나하나 접근해보지요.
\vspace{5mm}

\begin{enumerate}
    \item 강의는 EBS로도 충분하고 넘칩니다, 심지어 EBS조차도 번거로운 사람도 있습니다.
    콕콕에서 이번 시험에서 비주류 - 제가 권한 방법으로 올라간 사람들이 있습니다. 그 사람들 글을 읽어보면 알겠지만
    인강은 정말 아주 조금, 적당히 들고 문풀량을 적정하게 유지하고 최소 공부시간 유지하는 경우가 잘 나왔습니다.
    또 허락없이 자기 언급하냐고 하는 모양은 핫식스뽕에 취한다... 라고 해서 말아먹을 위기였으나 공부한 내공 어디 안가고 나올만큼은 나오더군요.
    \vspace{5mm}
    
    제가 권하는 건 그겁니다. 만약 님이 작년에 독학만 했다면 올해 학원에 의존할 수도 있죠.
    그러나 작년에 학원에 많이 의존했는데 성과가 미미하다면 그건 학원강의를 안 들어서가 아니라
    혼자 읽고 풀고 정리하는 게 부족해서입니다.
    \vspace{5mm}
    
    학원을 줄이면 그만큼 비용도 감소됩니다.
    정말 급하면 EBS만 들어도 된다고 말씀드립니다. 한마디로 EBS를 완강해도 모자라면 학원을 가라, 그 이야기입니다.
    \vspace{5mm}
    
    \item 학원을 줄이는 대신 스탑워치를 지참하시고 그걸 기록해보시죠.
    하루 공부시간은 순공부 7시간 제안하겠습닏. 적다고 느껴지겟죠? 2주간 해보시죠. 할 수 있나.
    현재부터는 하루 순공부 6시간으로 8월까지만 꾸준히 해도 올라갑니다.
    한가지만 일러드리면 이 순공부 6시간 비웃는 친구들, 나중에 확인해보니 그것도 안 하더군요.
    \vspace{5mm}
    
    학원을 줄이고 걸어서 30분 내외인 도서관에 다니세요. 식사는 도시락 등으로 때우시고
    모든 돈은 참고서와 문제집, 그리고 프린트에만. 그래서 하루 6$\sim$7시간동안 스탑워치로 시간재가면서 공부하는 게 학원보단 낫습니다.
    그럼 모르는 건? EBS를 전격 이용해도 좋고 수험사이트 질문게시판을 이용하십시오.
    그럼 너무 늦... 혼자서 책도 찾아보고 풀이도 궁리해보면서 해결해보려고 하세요. 여태까지 그런 걸 안 하셔서 문제였던 겁니다.
    \vspace{5mm}
    
    \item 다른 애들은 어떤 사교육서비스를 받나... 그거 아무 소용도 없으니까 눈감고 귀닫으세요.
    궁금하시면 올해 그토록 떠들어댔던 문제집, 실모가 실제 수능과 얼마나 일치했나 확인해보세요.
    물론 그런 것도 있을 수 있겠죠. 그러나 사실상 쓸모 없다는 게 제 판단입니다.
    다른 데는 모르겠으나 일단 콕콕만 보더라도 성적은 '공부한 만큼' 나왔습니다.
    공부한만큼 안 나왔다면 그건 특정 강사나 교재를 안 보아서가 아닙니다. 사고의 결함이 있거나 결핍요소가 있기 때문이죠.
    \vspace{5mm}
    
    \item 남들 뭐하나 보지말고 자기확신을 가지고 학습량을 꾸준히 적분해나가세요.
    공부의 비결이라는 게 별 게 없습니다.
    작년에는 저 혼자 일일히 답변해야했으나, 지금은 제가 상담해 준 사람들이 상담해줄 수 있는 위치에 올라왔습니다.
    그 분들이 저보다 낫죠(왜냐 직접 시험을 쳤으니까) 이 분들에게도 적극적으로 물어보시길 바랍니다.
    \vspace{5mm}
\end{enumerate}





\section{[상담 016] 디메님}
\href{https://www.kockoc.com/Apoc/504241}{2015.11.21}

\vspace{5mm}

이과 생1지1 응시했습니다. 작년 수능 3 4 3 2 3 올해 수능 3 2 2 1 1 (가채점) 집 근처 대학을 다니다가 4월 즈음에 이건 아니다 싶은 생각이 들기 시작해서 1학기 마치자마자 휴학을 하고, 7월 중순이 넘어서 수능을 한 번 더 치러야겠다는 생각이 들었습니다. 어떤 강의와 교재를 어떻게 학습해야 하는지를 커뮤니티 사이트 대부분을 찾아보다가 콕콕으로 왔고, 이것저것 들어보고, 공부해보면서 50일 분량의 커리큘럼을 만들어서 9평이 끝나고 바로 시작했습니다. 50일동안 초시계로 21538분, 하루에 7시간 조금 넘는 시간을 투자했고, 공부했던 것들은 국어 마닳 1권 3회독, 2권 2회독 플러스 알파닷 2회독 수학 지학사 교과서 익힘책 3회독 개념 백지복습 N회독 5개년 연도별 기출 5회독 수능특강 2회독 + 틀린 것 해설포함 통암기 수능완성 2회독 + 틀린 것 해설포함 통암기 영어 E적중 300제 지문만 3회독 고교영어듣기 실전파트 1회독 수능완성 실전편 듣기 1회독 파이널 그럼에도 불구하고 완강 생1지1 수능특강 2회독 + 막힌부분 강의 수능완성 2회독 + 막힌부분 강의 자이스토리 1회독 수능기출플러스 1회독 수능공부를 시작한 목적은, 사실 딱히 하고자 하는 일은 없습니다. 다만 주 4$\sim$5일정도 정시출퇴근 하면서 단란한 가정 꾸리는 것정도.. 그래서 떠오른게 수능으로 치대나 한의대에 들어가는 것, 또는 7급, 9급 공무원 시험공부를 하는 것. 예상 외로 노력한 것에 비해서 감히 받기 힘든 점수 받았고, 여러 사이트 모의지원 해보면서 지방치대, 한의대를 기대해볼 수 있는 점수인 것 같습니다. 그런데 아직 마음속에 남아있는 생각은, 제가 여러 사이트를 돌아보면서 생각했던 공부의 방향이 맞는 것 같고,  이대로 50일이 아니라 300일을 투자한다면 더 나은 결과를 받을 수 있을거라는 것.. 틀리면 안되는 문제들을 틀려서 점수를 꾀나 많이 깎아먹은 것(국어 화작 / 수학 21, 29, 30 맞추고 그 앞 주관식 실수 / 영어 듣기 등). 그래서 한 번 더 해보면 어떨까 하는 욕심이 듭니다.. 학교를 다니다 몸이 안좋아서 1년정도 치료를 받는 등의 이유로 수능은 두번째지만 나이로 치면 스물둘 미필4수생입니다. 어떻게 하는게 좋을까요?
요약하면 수험방법은 알았는데 학습량이 터무니없고 몇몇과목은 선정교재도 부족한 경우다.
이건 사실 너무 노골적이라서 굳이 충고할 필요가 있냐 그러는데 일단 국어부터보자.
\vspace{5mm}

교재비평이 되니까 조심스러워지지만, 국어를 \textbf{저 '교재만' 보고 점수가 잘 나올 리는 당연히 없다.}
수험가에서 웃긴 얘기가 이 교재 한권이면 그 과목이 완성된다는 식으로 자랑하는 교재들 있는데 단언코 말한다. '거짓말'이다.
선택한 교재가 나쁘지는 않다. 그러나 저 교재만으로 국어 실력이 오를리는 없고 그래서 성적이 납득이 간다고 보는 케이스다.
\vspace{5mm}

콕콕 내에서 국어가 잘 나온 케이스. 거의 다 미친 듯이 양치기한 경우다.
그것도 그렇지만 난 국어에서 어떤 패턴교재를 찾는 건 미친 짓거리라고 생각하는데 여기서 썰 풀어보자.
\vspace{5mm}

수학은 패턴화를 했다가 탈패턴화로 가야한다. 왜냐면 출제자는 기존의 정해진 패턴을 변형하거나 신패턴을 내기 때문이다.
하지만 국어는 반대다. 국어는 스스로 공부하면서 패턴화를 해야한다.
아니, 국어가 왜 패턴화를 해야합니가. 간단하다, 국어가 패턴화가 안 되면 정답이 절대 하나가 아니기 때문이다.
그럼 패턴화된 교재를... 헛소리다. 패턴화라는 건 본인이 국어문제를 많이 풀면서 지문을 읽고 그걸 '공식화'하는 훈련을 통해서만 가능하다.
즉 '패턴화를 하는 능력'을 키우기 위해서 국어공부를 하라는 것이다.
\vspace{5mm}

예컨대 문학, 비문학 지문이 주어지면, 그 지문들을 읽고 소재, 화제, 주제를 파악하고 여러 각도로 해석해보고 하는 것은
무질서해보이는 지문을 질서있는 소재, 화제, 주제, 글의 논리적 구조와 같은 패턴으로 바꾸는 과정인 것이다.
그리고 그 지문에 딸린 문제들은 절대 그 패턴을 못 벗어난다.
그렇기 때문에 어떤 국어교재든 만족스러운 건 없다. 왜냐면 그 패턴이라는 건 수험생 본인이 만들어내야하는 거니까.
\vspace{5mm}

본인은 화작을 틀렸다고 한다. 이건 원인과 결과가 너무 뚜렷해서 딱히 조언줄 수는 없겠지만
만약 내년에 다시 한다고 하면 저 교재만으로는 절대 안 된다.
\vspace{5mm}

수학의 경우는 교재접근과 방법은 좋았다고 본다. 다만 '교재량'은 3배로 늘렸어야한다고 보고 있다.
물론 시험에 나오는 내용은 저걸 안 벗어난다. 그러나 문제는 저것만 공부해서는 수리적 사고가 숙달될 리는 없다.
21, 29, 30을 맞고 다른 걸 틀렸다면 이건 숙련의 문제다(매우 안타까운 케이스가 아닐 수 없다)
\vspace{5mm}

영어는 뭐. 대부분 저렇게 공부해서 통수맞았기 때문에 따로 설명은 안 한다.
\vspace{5mm}

본인은 매우 깔끔한 성격이고 논리력이 좋다고 보고 있다. 하지만 깔끔한 사람들은 복병에 약하다.
그래서 공부방향이 맞다라는 확신은 격려해줄 수 있지만 '위험하다고' 지적드린다.
시험공부를 한다는 건 늘 예기치 못 한 것에 대비할 수 있도록 공부하는 것이다.
수험에 실패하는 사람들의 공통점은 \textbf{'자존심, '자기확신'}이다.
특히 디메님 같은 분은 올바른 방법론을 갖고 있더라도 자기확신에 빠지기 쉬워서 불의타에 당하기 좋다는 걸 지적드림.
\vspace{5mm}

국어를 제외한 교재방향은 내가 권하던 바라서.
다만 분량 면에서는 수학과 영어는 많이 부족했다고 보고 있고
아울러 이렇게 핵심만 집중하는 깔끔한 성격은 불확실한 출제에 당하기 좋다.
\vspace{5mm}




\section{[상담 017] 목표는 타율적인데 방법이 자율적인 케이스}
\href{https://www.kockoc.com/Apoc/505860}{2015.11.23}

\vspace{5mm}

문과졸업생이지만 이과로 시험보고싶다던 군복무 중인 놈입니다.
아재와 매일 주고 받은지 1달여가 흘렀습니다.
\vspace{5mm}

부모님과 선생님들과 상담도 진행했습니다. (휴가나가서)
결론은 문과로 준비해서 대학을 가는 편이 낫겠다는 쪽으로 기울었습니다.
\vspace{5mm}

내신이 2점 중반정도 되니, 문과였으니 문과로 수능을 준비해서 교대를 노려보는게 좋겠다고 말입니다.
\vspace{5mm}

제대하면 나이가 벌써 23입니다.
아재 생각 어떠하신가요? 줏대 없어보여도 불안해서 어떤 선택을 해야할지 모르겠습니다.
\vspace{5mm}

여태 제가 주도로 해온 선택은 모두 나쁜 선택만 냈거든요
\vspace{5mm}

작년 재수할 때도 자신있다고 6월 평가원 이후 학원 나온것,독학재수 시작해서 주변 조언 모두 무시한것
아무리 망쳤더라도 서울 낮은 곳 원서라도 쓰지않은것 등... 모두 나쁜 결과가 되어 돌아왔습니다.
\vspace{5mm}

솔직히 아직도 하고싶은 일을 찾지못했습니다. 이과로 돌려서 준비하는 것도
의치한수를 가고싶은것도 안정적인 자격증의 힘,대학진학시 미래와 직접연결되어있는것,다른직종에비해
QOL이 나은것 등...
\vspace{5mm}

이거아니면 죽는다는 마인드가 없어서 그런지 상담을 거듭할수록 마음먹은 쪽과 다른쪽으로 결과가 나올때
제선택이 또 잘못될까봐 번뇌에 휩싸입니다.
\vspace{5mm}

문과로 가서 교대를 준비할지
이과로 가서 의치한수를 노려볼지
\vspace{5mm}

뭐 어떻게 방향을 잡고 살아야할지 모르겠습니다...
\vspace{5mm}

목표가 타율적인데 방법이 자율적인 경우임.
성공하는 사람들의 특징은
목표는 자율적인데 방법은 타율적이죠.
\vspace{5mm}

\textbf{다시 말해 자기가 하고싶은 방향으로 감, 대신 방법과 과정은 절대 고집 안 부림.}
가령 내가 음악가가 되고싶다라고 터무니없는 목표 잡는다고 하더라도(요즘은 음악가 되는 건 터무니없는 건 맞죠)
\textbf{그 길을 향해서는 선배나 성공한 사람, 그리고 학교나 학원의 조언 충실히 듣고 남이 하라는대로 해본다는 것이죠.}
\vspace{5mm}

그런데 님은 왜 실패했느냐면
방법에서는 자기 고집을 피우는데, 목표는 남의 말을 따르고 있다는 겁니다.
\vspace{5mm}

목표는 님이 하고싶은데로 하세요. 뭐가 좋느냐 그렇게 살 거라면
인생 전체에 있어서 다 남이 시키는대로 할 겁니까?
남이 시키는대로만 하면서 자기 혼자 끙끙대는 게 바로 '노비'입니다.
\vspace{5mm}

겁먹어서 안정적인 길 찾으시는데.... 안정적인 코스가 지금 어딨습니까? 그런 거 있어보앗자 경쟁자 몰려서 바로 아작나는데.
\vspace{5mm}

23살이면 늦은 나이 아닙니다. 뭘 하느냐가 중요한 게 아니라 어떤 상황이 와도 문제해결하고 벌어먹을 준비가 되어있는냐가 중요한 겁니다.
님이 하고싶은 길 정하고, 그걸 준비하는 필요한 시간, 노력, 금전 자원 계산해보고
그 다음 효율성을 꾀하기 위해서 성공한 사례, 실패한 사례 참조해보고 그 다음 자기를 감시해주거나 조언해 줄 파트너 부탁해서 선정하세요.
\vspace{5mm}

이 길 가면 망하는 것 아냐... 한마디합니다. 그런 걸로 고민할 거면 걍 죽는 게 낫습니다.
안 망하는 길이 어딨습니까.
중요한 건 망하는 길로 가더라도 '안 망하도록 하는' 것입니다.
물론 노력해도 망하기 좋은 길 같은 게 있죠(가령 과탐에서 2과목 선택을 섣불리한다거나)
그러나 크게 망하는 길일수록 잘 보면 크게 흥하는 루트도 잘 보면 있습니다.
\vspace{5mm}

목표를 정하고, 그 다음 거기서 성공하려면 어떤 루트를 밟아야하나 하나하나 각론적인 걸 따져본 다음
중요한 것부터 해나가시길 바랍니다.
\vspace{5mm}




\section{[상담 018] 공대계열진학}
\href{https://www.kockoc.com/Apoc/524419}{2015.12.02}

\vspace{5mm}

혹시 아폭님이 쓰신 글에 누를 끼치지 않으려고 인증까지 해봅니다. 나머지 과목은 못본걸로(...) ㅋㅋ;
그래도 반신반의 했던 성대 공학계열에 수시로 붙었으니 후회는 없습니다. 아폭님 정말 마지막으로 인사드립니다. 감사했습니다^^
\vspace{5mm}

잘 치셨네요.
공대 계열로 진학해서 수학하신다면
1순위 : 영어
2순위 : C++, 엑셀, 통계프로그램 (기계 쪽이라면 CAD/CAM)
3순위 : 공학수학
\vspace{5mm}

입학 전까지 영어 파시고 토니 콕타크가 되시길 바랍니다.
성적표가 전반적으로 그 주인이 공부를 열심히 했다는 게 부정할 수 없군요.
\vspace{5mm}






\section{[상담 019] 이과수학 4등급에서 1등급으로 오른 케이스}
\href{https://www.kockoc.com/Apoc/524423}{2015.12.02}

\vspace{5mm}

일단 감사 인사부터 드리겠습니다. 수학b형 6월에 4등급이었다가 8월쯤에 칼럼(정리론)에서 복습하라고 하고 \textbf{3월부터 언제 까지는 미리 양치기 했어야 했다고 말씀하셔서 압축으로 8-10월 초순 까지 기출 일타삼피 일격필살 등 계속 양치기 하고 11월까지 복습을 계속 했습니다.(9월은 2등급, 수능은 1등급이 나왔네요 ㅎㅎ..)(기출은 2주만에 2014-2016.9까지 3회독, 일타삼피 기벡 2회독 나머지 5회독, 일격 틀린문제들 2회독) 이렇게 했네요..ㅎㅎ}
\vspace{5mm}

일단 저는 고3으로 수능을 쳤고 6월은 54433, 9월은 42413,  51435(익명이라..밝히겠습니다.)을 받고 재수 고민을 하게 되었습니다.
수학, 과탐은 그나마 자신감을 가지고 있어서 칼럼을 보고 응용을 하여 어떻게 적용해야지 등등을 느낌이 와서 하게 되었는데 \textbf{국어나 영어, 과탐 등은 책을 30회독 이상}씩 해보고 머리에 반복반복 하면서 열심히 했지만 성적이 오르지가 않네요.(공부 비중은 국어>영어>과탐>수학 정도로 열심히 했습니다.)
\vspace{5mm}

위에 공부 방법의 문제에 대한 상담도 받아보려고 썻지만
사실 진짜로 상담 받고자 하는 것은 지금부터 입니다.
ㅈ 저희반에 공부 시간은 별로 인데 성적이 계속 잘나오는 친구가 있었습니다.
그래서 걔를 계속 따라하며 저도 성적이 오르게 되더라구요. 그런데 성적이 오르니 안도감이 들면서 공부 시간은 늘어가는데 계속 성적은 원위치로 돌아오더라구요.(원래 -> 오름 ->살짝 더오름and유지 -> 원상태)
이렇게 계속 반복되니 문제는 알겠는데 문제에 대한 해결책을 모르겠습니다.
이것에 대해 상담받고자 상담을 쪽지로 보내게 되었습니다. ㅎㅎ
\vspace{5mm}

일단 이건 꽤 보기드문 케이스인지라.
우선 용기있게 제 방법론을 실천해주셔서 수학성적을 올리신 것에는 축하의 말씀을 드립니다.
타 과목을 공부 안 했다고 볼 수도 있지만 30회독했다면 이건 다른 문제겠죠.
수학이 저렇게 올랐다면 이건, 공부를 못 하는 게 아니라 그동안 안 하셨다는 이야기이고 그나마 수학은 갈피를 잡았단 이야기입니다.
\vspace{5mm}

이 경우는 왜 그게 가능한가 분석해보니까
수학은 그나마 수능에 대비할 수 있는 기초가 있었지만
국어, 영어는 초중고1까지 공부가 부진하거나 기초가 없었지 않나 조심스럽게 제기해 볼 수 있을 것 같습니다.
그리고 국영탐은 공부를 비교적 늦게 시작하신 걸로 보입니다(탐구는 일단 어떤 과목인지 알 수가 없고요)
\vspace{5mm}

익명처리로 진행되는 상담이니
국영과 탐구과목별로 어떤 교재들과 인강을 들었나 제시해주시길 바랍니다.
그리고 어린 시절부터 공부를 어떻게 하셨나 얘기해주는 게 더 상세한 상담이 되겠죠



\section{[상담 019-1] 이과수학 4등급에서 1등급으로 오른 케이스 2}
\href{https://www.kockoc.com/Apoc/538099}{2015.12.10}

\vspace{5mm}

저가 수학은 초등학교부터 엄마가 기탄수학 등을 꾸준히 풀리며 직접 관리해주시고 나머지 과목들은 그냥 중학교 내신 때 조금(?)하는 정도 중학교 내신도 22$\%$였습니다.(아! 윤선생영어교실(?) 초등학교때 조금 했었습니다.)
그러고 책은 진$\sim$$\sim$짜로 안 읽는 편이에요.(ㅜㅜ..) 친구들에 비해 영어나 국어 읽는 속도도 느리고 친구들에게 무언가를 설명해도 친구들이 "국어 못할 꺼 뻔히 보인다, 화작문부터 배워라. , \textbf{니 설명은 못알아 듣겠다 똑바로 설명좀 해라."}등등의 소리를 자주 듣습니다.(제대로 이해시키게 설명하려고 해도 잘 안되네요.)
\vspace{5mm}

공부를 제대로 시작한 건 고1 9월부터 친구 한 명이 공부를 하기 시작하면서 같이 시작하게 되었습니다.
이때부터 국어나 영어는 시작했고 과탐은 고2 겨울방학부터 제대로 시작하게 되었네요. 과탐은 물1 지1을 선택 했었었구요.
국어는 6월때 5등급을 맞고 xxx 선생님의 xxx강의를 듣고 xx를 2회독 그리고 더 어려운 책(지문이 실려있는 강의책)을 미리 다 이해하며 풀고 하며 2등급 후반까지 올라갔다가 아! 국어에 대해 이제 조금 알겠다!! 싶었는데 저번 상담에 제시했었던 것과 비슷하게 (원래 -> 오름 -> 유지 -> 원상태) 단계로 돌아왔구요.
\vspace{5mm}

영어는 xx선생님 강의를 들으며 xx라는 책을 30회독 이상 했고 했지만 아무리 해도 90점 이상(3-4등급)을 올라가기 힘들었습니다.
\vspace{5mm}

과탐 중 물리는 xx선생님의 xx을 20회독 정도 했고 지구과학은 xx 선생님의 xx을  40회독 정도 했네요.
\vspace{5mm}

친구들이 반복쟁이라고 할 정도로 반복을 열심히 했습니다.
\vspace{5mm}

이러니 친구들이랑 물리 실모 풀면 1-2등급 왔다갔다 거렸고 지구과학도 고정 1등급까지 가게 되었습니다.
하지만 국어 영어 만큼은 죽어도 2등급 이상은 안나오더라구요.
\vspace{5mm}

그런데 수능날 국어에서 멘탈이 너무 나가게 되면서 omr에 마킹도 한 두문제 못하게 되고 까지 하면서 멘탈 나간 채로 쳤으나 수학만은 1등급이 나와주더군요.
수학은 40분 만에 중복조합문제, 21번문제, 29번 문제, 30번 문제 빼고는 다 풀고 검토 3번하고 나머지 문제 고민하고 순서대로 풀었습니다.
\vspace{5mm}

아! 그리고 엄마가 저가 수능 망하고 뭐가 문제인지 고민하면서 엄마가 어린 시절부터 공부할때 꼼꼼함이 없더라. 라고 하셨습니다.
그리고 수험 생활때 자존심이 문제인지 엄마가 조언하는 말씀은 아니고 내가 공부하는 방법이 옳다(또는 국어나 영어가 안나오니 과외 받아보는 것이 어떻겠니? 라고 엄마가 말씀하셔도 저 혼자 잘할 수 있어요! 라고 등) 하고 밀고 나가다가 실패를 경험했다고 생각합니다.
\vspace{5mm}

자존심은 이번 수능 실패를 계기로 자존심을 버리기로 결심하게 되었고 이제 주변 소리를 듣기 싫어하기 보다는 어떻게 나에게 적용을 해볼까 라는 마음가짐으로 노력하고 있습니다.
\vspace{5mm}

상담 받아주셔서 감사합니다.
\vspace{5mm}

어린 시절부터 '주의가 산만하다'란 얘기를 들었을 것이며
말을 더듬고 그리고 뭔가 리스크를 거는 일을 할 때는 수줍어하거나 소극적으로 나서는 성격.
\vspace{5mm}

그런데 이런 일이 벌어지는 건 간단합니다.
생각과 행동을 '빨리' 하려고 해서 그럼. 즉, 성급한 거죠.
그런데 \textbf{성급하기만 하지 그걸 어떤 '순서'로 해야할지 모르니까 엉키는 겁니다.}
이건 어린 시절 부모님이나 다른 선생의 잘못이기도 함. 어느 시점부터 그 '엉켜버린' 것을 정상으로 여기고 성장한 것입니다.
\vspace{5mm}

지금부터 해야하는 건 모든 작업에 자연수 번호를 매기고, 그 순서대로 천천히 하는 습관을 들이는 겁니다.
국어든 영어든 읽을 때 끊어읽으면서 ①, ②, ... 을 매겨보면서 어떤 순서를 밟아야하나 스스로 알고리즘을 짜야합니다.
지금은 문풀보다도 10여년 넘게 엉켜있던 자신의 사고 프로세스를 교정하는 작업을 해야합니다. 이걸 안 하면 절대 못 올라가요.
문제를 풀 때에도 무조건 빨리 풀자라는 게 너무 체화되어있음.
\vspace{5mm}

수학이 왜 잘 나왔는지 아십니까? 그나마 수학은 늦게 시작했기 때문에 저런 '엉켜버림'으로부터 자유로웠기 때문입니다.
다른 건 백지가 다 낙서질에다가 라면국물 묻고 엉망인데 수학은 공부를 안 하셨기 때문에 잘못된 습관으로부터 자유로웠던 것입니다.
웃지 못 할 일입니다만 이게 본인이 왜 다년간 정체되었는가를 보여주는 것입니다.
일단 분명 부모님의 잘못입니다. 자기 아들이 어디서 엉켜버렷는가 그걸 모르고 그냥 넌 꼼꼼하지 못 해라고 하는데
저런 식의 지적이라면 누구라도 합니다. 중요한 건 대안이 아닌가요?
\vspace{5mm}

수학은 그나마 순서대로 하는 습관이 들어 있습니다. 그러나 국어나 영어는 그게 안 되네요
무작정 인강 듣지 말고, 겸손하게 고1 과정 쉬운 교재들부터 천천히 푸시길 바랍니다. 그래야 님의 잘못된 것들이 정화가 됩니다.
물론 수학과 탐구도 스피드를 버리고 정확한 순서대로 개념을 정리하고 설명해보는 훈련을 해야합니다.
이래서 모든 게 잡히면 그 다음부터는 무시무시한 속도로 회복되면서 실력이 상승할 겁니다.
\vspace{5mm}

우선 국어는 서점에서 신사고 교재 같은 건 아주 잘 나왔으니 그게 수능에 도움이 될까말까 고민하지 말고 그냥 천천히 읽고 하라는데로 하시고
최종적으로 밋딧릿 지문, 문제까지 풀도록 계획짜고 가셔야합니다.
영어의 경우는 강의보다도 독학용 문법서에다가 구문서 다시 차분히 보시길요.
특히 영어는 전치사 위주의 구문을 살펴보는 수험외적인 책들도 보시길 바랍니다.
\vspace{5mm}






\section{[상담 020] 느리게 가는 법?}


\href{https://www.kockoc.com/Apoc/580117}{2016.01.08}


우선 점수로 사람을 판단하기는 힘들겠으나, 그래도 객관적으로 저를 보실 수 있는 방법이 점수 일 것 같아 점수부터 공개하겠습니다.
모평은 의미없으니 각설하고 그냥 평타는 쳤다고 봐주시면 감사하겠습니다/
16 수능 : 84/84/80/46/45 (한국사 동아시아사)
\vspace{5mm}
 

저는 남의 말을 잘 안듣는 타입입니다.
다른건 모르겠지만 진짜 공부에 있어서는 주변 사람들 다 같잖고 제가 짱인 맛으로 살아왔습니다.
내가 푸는 교재가 짱이고, 내가 듣는 인강이 짱이었습니다. 또한 내가 공부하는 방법이 그냥 최고였습니다.
그런 생각을 갖고 수험생활을 했습니다.
\vspace{5mm}
 

 

어릴때부터 어디서나 리더였고 어디서나 센터에 있었습니다.
제 말을 안듣는 사람은 없었고, 어디서나 꼭 지도자의 입장에 있었습니다.
그래서 그런지 점점 성격은 자기중심적 꼴통이 되어갔고, 
표면적으로는 이타적인 모습을 보여 남을 챙기고 남을 위하며 사람을 끌어들였으나
내면적으로는 단지 너희들은 내 편일뿐, 나와는 다른 계급이다 라는 이상한 마인드가 박혀 있습니다.
\vspace{5mm}
 

그런데 수능 성적을 보면 제가 짱이 아니더군요ㅕ
분명 저는 잘못되었습니다. 하지만 그걸 인정을 안하고 있습니다.
아폭님께 질문 드리기 직전에도 이걸 물어보는 의미가 있을지 계속 고민했는데
분명 저는 어딘가 잘못되었으니 그걸 꼭 찾아내고 싶었습니다.

\vspace{5mm}

우선 가장 시급한 마인드부터 대화하고 싶습니다.
\vspace{5mm}
 

이제는 저 자신이 남들보다 우월하지 않다는 걸 스스로 받아들여야 합니다.
\vspace{5mm}
 

아니 사실 남들보다 하등합니다. 그걸 인정하는게 아직 어렵군요
세상에서 나라는 존재가 얼마나 작고 하등한 존재인지를 깨닫고 싶습니다.
이건 따로 세미나라던가 강연 or 탐방이나 여행으로 가능한지, 아니면 따로 독서로 해결 가능한지 
애시당초 방법이 있는가에 대해 여쭙고 싶습니다.
이 나르시시즘을 깨부수지 않는 이상 저는 그냥 혼자만의 승자로 남아 더이상의 진보가 없을 것 같습니다.
\vspace{5mm}
 

두번째로, 우선 수능을 위한 공부법에 대해 여쭙고 싶습니다.
어떤 과목은 뭐를 풀어라는 궁금한 부분이긴 하지만 현재 제 입장에서는 무의미한 말이 될 것 같습니다.
제 스스로 제 문제점을 진단해 본 결과 크게 3가지로 수능을 망했습니다.

 
\begin{enumerate}
    \item 양치기 부족
    \item 인강맹신
    \item 대충대충하는 태도
\end{enumerate}

 

1,2는 따로 말이 필요하지 않을 것 같습니다. 그냥 제 불찰이라 따로 언급할 거리가 없을 것 같아요..
문제는 3입니다.
국어가 만년 1등급이 나오면서 탄생한, 제 인생 최대 실수입니다.
\vspace{5mm}
 

국어를 풀때, 비 문학지문에서 1234단락이 있다고 할때
저는 1단락 읽다가 중간에 2로 넘어갔다가 3갔다가 4가고 다시 2갔다가 기억 안나서 다시 3 가고 이렇게
그냥 대충대충 막 읽습니다.
근데 문제는 풀리더라구요.
이게 습관이 되니 아 국어 ㅈ밥이네 ㅋㅋ 마인드가 박혔고,
이게 수학에도, 영어에도 전이가 됐습니다.
\vspace{5mm}
 

영어도 지금 풀때 잘읽다가 3문장 정도 건너뛰고 읽고 틀리고, 수학도 문제 끝까지 안읽다가 된통 당합니다.
\vspace{5mm}
 

 

전에도 쓴 글 내용인데 
나는 먹는다 치킨 근데 그건 튀김옷 $\sim\sim\sim$(눈이 아래로 내려감) $\sim\sim\sim\sim$ 치킨은 맛있다!
중간내용 안읽어서 모름; 이게 뭔지;;
논리는 완성되어 있는데 해석을 못해서, 아니 안해서!!! 망하는 케이스가 되고 있습니다.
\vspace{5mm}
 

아무튼 이 습관이 굳어져서 이제 꼼꼼함과는 거리가 멀어졌습니다 
어디서 잘못되었을까, 결론은 국어였습니다.
\vspace{5mm}
 

국어에서 잘못되었으니 결국 국어로 고쳐야 합니다.라고 결론을 내렸습니다.
\vspace{5mm}
 

이제 어떻게 국어를 고칠까 고민을 시작하고 있습니다.
일반적인 책을 읽을떄도 읽다가 눈이 아래로 자동으로 내려갑니다 천천히 모든 문장의 글을 읽어본지가 얼마나 됐는지도 모르겠군요
이제는 그러면 안되겠죠,
우선 하나 하나 한문장을 꼭꼭 눌러가며 하나하나 읽기 시작했습니다.
이거 잘 안되더군요 습관이라는게 .. ㅡㅡ 다 읽는다고 노력해도 어딘가는 건너뜁니다
계속 이 방법을 고수하면서 뿌리부터 악습을 뽑아낼 수 있는지 궁금합니다
\vspace{5mm}
 

그리고,  저대로 공부해보면 정말로 엄청난 시간이 걸립니다
정상적인 남들 1시간 동안 하는 양이 2 라면 저는 0.5 걸릴정도로 오래걸립니다.
이건 대충 산 삶에 대한 속죄로 받아들이고 천천히 느리지만 열심히 교정해 나가면 되는 부분일지.
이 부분에 대해서도 질문을 드리고 싶습니다.

%%%%%%%%%%%%%%%%%%%%%%

\begin{enumerate}
    \item  오프라인 학원 가는 걸 권하겠습니당. 어차피 수험까지 하려면야.
    그리고 리더라고 하지만 실제로는 그 나이까지는 생산적인 일을 해본 적은 없죠.
    정말 멘탈 부수고 싶다고 하면 한 일주일잡아서 상하차 알바 가서 추노해보시거나(...)
    봄날에 서울대 같은 데 가서 학생들이 공부하는 것 보면서 몸에 뭔가 뜨거운 게 올라오나 보시길.
    \vspace{5mm}
    
    
    \item  엄청난 시간이 걸리더라도 저대로 가야합니다. 느리게 가더라도 '바른' 길로 가야지, 급하다고 꼼수 부려보았자 더 시간만 낭비합니다요.
    천천히 읽으면서 머릿 속에 그걸 요약하고 그래서 하나하나 꼼곰히 논리적 검증하면서 왜 이게 답이고 저건 답이 아닌가
    '논술'적으로 접근하는 습관 들이셔야합니다. 아직까지는 시간이 있다 여겨지니 인생 전체를 위한 교정이라 생각하고 가도 좋습니다.
    국어공부를 하시려면 해당 지문의 문제를 가지고 본인이 '논술할 수준'까지 갖추면 됩니다(이건 타 과목도 마찬가지입니다)
    대충 적당한 등급 맞고 싶으면 서둘러야하지만, 정말 만점 수준까지 가고 싶다라고 하면 시간이 걸리더라도
    모든 문제를 논술해보이겠다라고 가는 게 맞습니다.
\end{enumerate}







\section{[상담 021] 자위권}
\href{https://www.kockoc.com/Apoc/583606}{2016.01.10}

\vspace{5mm}

고3때는 거의 하루도빠짐없이 한거같습니다.. 그때는 공부도 초반에만 조금 하다 안해서
게임같은것도 거의 매일했고 결국 재수하게됐는데요
이런건 제쳐두고 1월1일 딱 된 후부터 공부를 시작했는데 문제가 생겼습니다
일단 계획은 6:10기상 1시취침인데, 12시 50분쯤에 공부 다 마치고(다음날로 미룰때도 있습니다만)
(중략)
하..이런거진짜 어떻게해야될지 정말 고민돼서 쪽지보냅니다.. 간단한 답이라도 해주셨으면 ㅠㅠ
\vspace{5mm}

우선 생각해본건 3가지정돈데
1.금딸 - 이건 됄지 모르겠네요.. 일주일에 한번정도는 못한거 정리하고 남은시간에 좀 쉬는데
온라인게임, 디씨같은건 다 끊었는데 미연시 이거는 도저히 못끊겠더라구요
그래서 쉴때는 보통 미연시를하는데 이게 대부분 19금이라(..)..
\vspace{5mm}

2.일주일에 한번같이 기간마다
생각해본 가장 무난한 방법인데, 과연 이대로 됄지..
\vspace{5mm}

3.공부시간을 야얘 조금 빼서 취침시간 맞추기
으.. 이거말고도 방법이 또 있을까요..?
\vspace{5mm}

조회수가 가장 높아질지도 모르고 온갖 논쟁이 오갈지 모르겠습니다만.
\vspace{5mm}

자위에 관해서는 "하면 키가 크지 않는다", "머리가 나빠진다"라는 속설이 있사온데
키가 크지 않는다라는 건 중학교 때 저런 데 선행학습(...)하던 녀석이 키가 컸기 때문에 반례.
반면 머리가 '나빠진다'라는 건 절반 정도는 맞다고 생각하고 있습니다.
\vspace{5mm}

만화 은과 금에 나온 장면인데 -
그림 속의 자연인(?)은 원래 대학 교수였던 사람인데 도박에 패해 지하 감옥에 갇혀 있습니다. 전직 대학교수이지만 감금당하자 마자 시간도 붕괴되고 자의식마저 붕괴된 걸로 나오죠. 유일한 낙이라면 감옥에 설치된 TV를 통해 포르노가 나오면 그걸 보면서 \textbf{자위를 하는 것}입니다.간단히 말해서 자위는 '마약'의 역할을 하죠. 그 순기능은 동시에 역기능이기도 합니다.성욕을 해소해야하는데 못 한다 어쩐다 그런 자질구레한 걸 떠나서, 그걸 하는 이유는 그냥 뇌에서 쾌감호르몬이 나오는 기계적 과정 때문입니다.위험한 이야기지만 그럼 마약을 한다면 대체할 수 있나, 예. 대체될 수 있을 것입니다.그래서 해당질문은 "규칙적으로 음주, 담배, 마약을 해도 됩니까"라는 질문으로 바꿔도 극단적이지만은 않습니다.문제는 이 중독은 '역치'가 있어서 규칙적으로 하는 것으로 해소되지 않는단 것이죠.똑같은 수준으로 관리한다쳐도 그걸로 뭔가 무뎌지기 때문에 더 강렬하고 자극적인 것을 찾게 되어있습니다.그래서 그게 과연 규칙적인 자위로 끝날 건가.... 는 건 다소 의문이기도 하지만.사실 가장 심각한 것은 '공부를 힘들게 하면서 러너스 하이 효과로 쾌감을 누려야하는데'그걸 자위로 풀면서 러너스 하이 효과를 못 누린다는 것입니다.공부하는 원동력은 힘든 과정을 거치더라도 성과를 조금씩 맛보며 거기서 쾌감을 얻고 가는 건데그걸 자위로 미리 해소해놓는다면 "고행-극복-쾌감"의 선순환이 달성되기 어렵죠.힘들게 공부할 것 없이 자위만 하면 쾌감을 얻을 수 있다... 면 뇌에서 공부하고 싶어지겠습니까.\textbf{뇌에서는 더 많은 쾌감을 얻기 위해 공부하지 말고 자위를 하라는 쪽으로 움직일 터인데 말입니다.}모아놓은 미연시나 자료가 아깝다면 외장하드에 다 박아넣고 꺼낼려면 귀찮은 곳에 넣으시길 바랍니다.자료 애착(?)이 있어서 지울 수는 없을 터인데 그렇다면 한번 열람하려면 번거롭고 귀찮은 곳에 저장해놓으면서자기 행동을 제약한다...가 가장 낫습니다. 귀찮아서 그런 걸 안 보게 된다.... 는 걸 아시겠고컴접속으로 그런 자료를 접한다면 그냥 환경을 바꿔서 컴을 쓰지 않는 곳으로 가길 바랍니다.자기가 중독되었다는 걸 인정하는 마약환자는 없습니다.+ 리플에 답하면 - 공부에 성과가 보이면서 보람을 느끼기 시작하면 자위를 할 필요가 더욱 사라집니다.카지노에서 잃기만 하다가 대박연전승을 거둔 사람이 자위를 하러 갈까요. 다시 도박하러 가지하라는 공부는 안 하면서 pc방에서 롤 10시간 하는 사람이라면 자위가 뭔지 물어볼지도 모릅니다.자위의 문제는 그게 불건전해서가 아닙니다. 첫째, 노력을 안 해도 성과를 준다는 것둘째, 공부하기 싫어하는 뇌가 자위를 좋아한다는 것입니다.수험에서 실패하는 게 아니죠. \textbf{스스로 무너져내리는 것}이지.자기가 괴롭게 공부하면서 수북히 풀어댄 문제집과 더러워진 개념서를 보고어느 순간에라도 공부한 것을 바로 암송할 수 있는, 그리고 공부하다가 날카로워진 눈빛에 만족해야하는데자위를 한다... 규칙적으로 자위해서 성공했단 케이스는 못 들어보았습니다.++남자의 경우 자위를 하면 체력이 떨어진다.... 는 것도 문제겠지만사실 가장 커다란 건 바로 '현자타임'(...)이라고 불리는 시간입니다. 현자라고 하면 지력이 올라갈 것 같지만 실제로는 그게 아니지요. 실제로는 가장 멍청해지는 상태입니다.우선 인간은 유전자의 명령을 받고 있는 동물인 것을 전제해봅시다.공부하는 이유는 물론 공부가 좋아서이겠지만 대부분은 '힘'과 '돈'을 갖고 싶어서일 것입니다.그리고 그 힘과 돈은 당연히 매력적인 배우자를 갖거나 정복하는 것도 전제하는 것이지요.차분히 공부하는 남학생도 내심으로는 좋은 대학에 가서 상위 1$\%$ 이상의 여자와 결혼하고 싶다... 는 무의식으로 공부한다는 것입니다.이걸 시사적으로 보여주는 게 영화 내부자들의 그 접대장면(...)이 아닌가 합니다. 이 영화가 화제가 되었던 것도 그 악인들에 대한 분노라기보다는 그런 장면을 겉으로는 욕하면서도 무의식적으로는 '선망'해서이죠.그걸 보고나서 열심히 공부하는 사람들이 생겼을지도 모른다는 생각이 들더군요.그런데 자위를 한다고 하면 저 욕심이 사라지겠죠.공부 안 해도 좋아, 자위만 하더라도 충분히 만족할 수 있어... 라고 타협하기 시작하면 자기를 불태워서라도 공부하고 싶은 의지가 사라져버립니다.그리고 뇌에서야 주인 인생이 어찌되든 말든 오늘의 쾌감을 달성했다 하면 더 이상 학습하려 하지 않겠죠.





\section{[상담 022] 글쎄올시다.}
\href{https://www.kockoc.com/Apoc/597010}{2016.01.19}

\vspace{5mm}

저는 재수하던 시절 중간부터 강박증세, 공황장애 비슷한걸 겪어서 공부를 제대로 하지 못했었습니다.
\vspace{5mm}

공부가 하기 싫어서가 아니라 , 한순간도 집중을 놓치면 안되고 시간을 허트루 쓰지 않아야만 하고, 반드시 SKY에 가야만 한다는 강박증세가 있었는데 이게 나중에 너무 심해지니 병으로 도져서.. 가슴이 계속 답답해 숨도 잘 못쉬겠는 너무 힘든 날을 보냈었습니다.
\vspace{5mm}

그 때 생긴 마음의 병이 계속 남아 결국 공부도 원하는 만큼 못하고 수능도 망쳤었고 점수에 맞춰 학교를 갔습니다.
\vspace{5mm}

학교를 다니면서 열심히 생활을 하니 그 당시 강박때문에 힘들었던 증상들이 서서히 잊혀지기는 했지만 간간히 힘든 순간들이 찾아오고.. 완전히 치료한게 아니라 그냥 덮어둔채로 살아왔었습니다.
\vspace{5mm}

9월까진 나름대로 잘 끌어올렸다고 생각하여
\vspace{5mm}

국,수,영,생1,지1 등급이 3 2 3 1 2 가 나왔었고 원점수로는 93 96 94 47 40 이었습니다.
\vspace{5mm}

하지만 수능은 결국 자기가 나올 수 있는 가장 다운그레이드한 성적으로 나오더군요.
\vspace{5mm}

결국 국,수,영,생1,지1 등급이 4 2 4 2 3 이 나왔습니다. 과목별 원인과 해결책을 분석하였고 이번에 다시 하면 자신이 있습니다.
\vspace{5mm}

제 고민을 이야기해보겠습니다.
\vspace{5mm}

크게 두가지입니다.
\vspace{5mm}
\begin{enumerate}
    \item 
    \vspace{5mm}
    
    저는 사업을 하고 싶은 사람입니다. 그리고 이과입니다.
    \vspace{5mm}
    
    이렇게 말씀드리고 댓글로 한 번 길게 상담을 받은적이 있었는데..
    \vspace{5mm}
    
    "이과라면 굳이 학벌 높이기 위해 다시 도전해볼 이유가 없다. 의치한 같은 전문직을 가는게 아닌이상."
    \vspace{5mm}
    
    이라고 말씀해주셨습니다.
    \vspace{5mm}
    
    많이 고민을 해보았는데 사실 답이 잘 안나옵니다.
    \vspace{5mm}
    
    저는 이과지만 문과적 소양이 더 강한 학생입니다.
    \vspace{5mm}
    
    저번에 아폭님께서 산업공학과에 대해 간략하게 글을 쓰신걸 봤었는데.. 저는 학교는 어쩔수 없이 맞춰서 간거였지만 학과는 원해서 온 곳이었습니다.
    \vspace{5mm}
    
    그리고 학과 특성상 공대의 다른 학과들처럼 이과적소양이 특별하게 요구되는 학과는 아니라고 생각합니다.
    \vspace{5mm}
    
    산업공학과는 제가 이과이면서 CEO가 되고 싶다는 생각에 가게 되었던 곳이고 나름대로 매력을 느끼는 학과입니다.
    \vspace{5mm}
    
    그래서 사실 저는 어떤 분야로 나아가 기술적인 혁명을 일으켜서 사업을 하고 싶단 생각은 아직 없습니다.
    \vspace{5mm}
    
    이런 상황에서라면 아폭님이 다른 문과학생들에게 학벌을 높일 필요가 있다라고 얘기해준 것과 비슷한 이유로
    \vspace{5mm}
    
    저 역시도 학벌을 높이는게 많이 중요한 일 아닐까요?
    \vspace{5mm}
    
    사실 지금 학교를 졸업하면.. 남은기간 열심히 학점 따고 이것저것 스펙 쌓으면 대기업 들어가는게 그리 힘든 상황은 아닙니다. 근데 그렇게 되면.. 그냥 딱 그렇게 멈추게 될 것 같습니다.
    \vspace{5mm}
    
    대기업에서 열심히 돈 벌고 모으는것에 인생을 전부 쓰게 될 것 같고, 이렇게 사는게 평범한 삶이고 사실 이런 평범함을 이루기가 굉장히 힘들고 대단한 삶이라는걸 압니다..(보통 중간에 짤리겠죠..)
    \vspace{5mm}
    
    이런 삶을 폄하하는 것이 아니라 전 뜻이 그 쪽에 있지 않다보니 이 길을 걷고 싶지가 않습니다.
    \vspace{5mm}
    
    말은 멋있게 하지만 아폭님이 칼럼에서 말씀해주셨듯이 아직 만족하고 싶지 않고 더 욕심을 내고 싶은것 뿐입니다.
    \vspace{5mm}
    
    물론 이 길을 걷는다 하여도 100$\%$ 그런 삶을 살리라고 생각하진 않습니다.
    \vspace{5mm}
    
    또 그 때의 상황에 맞춰 제가 변화하고 도전해볼수도 있겠지요..
    \vspace{5mm}
    
    그래서 결국은 확률싸움이라고 생각합니다.
    \vspace{5mm}
    
    다시 한번 도전을 해서 간판을 SKY로 높이는게 내 삶에 있어서 더 이득인가,
    \vspace{5mm}
    
    그냥 여기서 다른 노력을 더 해보는게 이득인가 라는 생각을 하는 것인데
    \vspace{5mm}
    
    아무리 생각해도 제 좁은 식견으론 명확한 답이 나오지가 않습니다...
    \vspace{5mm}
    
    (애초에 '정답'은 없겠지만 아폭님처럼 식견이 넓으신 분께 조언을 구하고 싶습니다.)
    \vspace{5mm}
    
    제가 이 나이에 다시 SKY를 가는게 인생에 있어서 그렇게 비생산적인 일일까요..?
    \vspace{5mm}
    
    (SKY만 가면 무조건 성공이다 라고 생각해서 하려는건 절대 아닙니다.
    \vspace{5mm}
    
    그저 새로운 시작일 뿐이죠....
    \vspace{5mm}
    
    그 출발선을 높이고 주변 환경을 좀 더 좋은것으로 만들고.. 아직 잘 모르겠는 사회에 존재할지도 모르는 유리천장 같은것을 넘어서고 싶어서 SKY에 대한 도전을 생각해보게 됐습니다. 그동안 이루지 못했던 도전을 성취해내서 생기는 더 높아지는 자존감도 있을테구요..)
    \vspace{5mm}
    
    결국 제가 끌리는대로 선택을 하면 되는거지만..
    \vspace{5mm}
    
    \item 
    \vspace{5mm}
    
    다시 하게 되면 이번엔 서울대를 목표로 하고 싶습니다.
    \vspace{5mm}
    
    그런데 투과목은 지2밖에 할 자신이 없고.. 가장 완성되어있던 과목인 지1을 버려야 한단게 많이 아쉽습니다.
    \vspace{5mm}
    
    이거까진 그럭저럭 넘길수 있다고 하더라도.. 워낙 투과목의 백분위가 헬이다보니
    \vspace{5mm}
    
    지금 제 성적대에서 서울대를 목표로 하는게 욕심인지 아니면 가능한 도전일지 궁금합니다.
    \vspace{5mm}
    
    (투과목에 있어서 아폭님이 많이 부정적인 시각을 갖고 계신것 같아서요.. 저 역시도 불안한게 사실이구요)
    \vspace{5mm}
    
    그래서 이에 대한 해결책으로 6월까진 생1,지1,지2를 다 공부해보자라는 생각도 해보았는데 (그러다가 지2가 이번에도 폭발할 분위기면 지1으로 돌리는쪽으로..) 제 통찰력으론 어느정도 선에서 지2를 그만둬야 할지 감이 잘 안잡힐것 같습니다.
    \vspace{5mm}
    
    이에 대한 아폭님의 조언도 들어보고 싶습니다.
\end{enumerate}
\vspace{5mm}

먼저 말씀드릴 것은 수신제가치국평천하(修身齊家治國平天下)입니다.
님이 뭘 하고 싶다 그게 중요한 게 아니라 지금 뭘 하고 있으며 어떻게 공부하고 있느냐가 중요한데 그게 안 나와있습니다.
말씀하신 등급 가지고는 알 수가 없어요. 왜냐면 비슷한 데 있는 사람은 널렸기 때문입니다.
\vspace{5mm}

현재로선 님이 SKY를 가든 대기업을 가든 그 뒤는 알 수 없는 문제입니다.
거꾸로 대기업에 가서 임원이 될 수 있었는데 SKY 간다고 수능쳤다가 말아먹을 수도 있고, SKY에 가서도 안 풀릴지도 모르기 때문입니다.
사실 그 전에 대기업에 가서라도 자기가 어떤 일을 하게 될 것인가, 그럼 자기가 사업을 한다면 어떤 사업에서 뭔 일을 할 건지 안 나와있는데
이렇게 막연하게 얘기해놓고 비교하는 건 상당히 안이하다고 보고 있습니다.
만약 올해내로 수능쳐서 SKY에 합격한다면 지금보다 '나을' 건 자명합니다만 문제는 그게 가능하냐의 여부이고
실제로 본인이 얼마나 공부하느냐의 문제입니다.
\vspace{5mm}

하지만 분명한 건 글쓰는 것으로는 사업에 부적합합니다.
행동은 자기가 하되 판단을 남에게 맡겨서 그냥 열심히 사는 사람이면 그냥 공무원이나 대기업 사원으로 사는 게 편합니다.
이런 사람들은 자기들이 돈을 적게 번다고 탓하면 안 됩니다. 왜냐면 리스크를 감수하지 않으니까요.
반면 자영엽자나 사업하는 사람들은 돈을 많이 벌어도 됩니다. 왜냐면 빚도 많이 지니까요.
이런 사람들은 행동 뿐만 아니라 자기 목숨이 날라갈 수 있는, 하다 못해 자기 아들을 머슴, 딸을 술집여자로 만들어버릴지도모르는
위험한 선택도 본인들이 합니다. 그렇기 때문에 그만큼 대접받을 가치가 있는 것입니다.
\vspace{5mm}

그런데 본인이 평범한 삶을 살기 싫다고 하는 데 질문은 매우 평범합니다.
사업을 하고 싶다 어쩐다하면 SKY는 필요조건은 아닙니다. 본인이 좋은 아이템이 있고 사람 부릴 줄 알며 여기저기 다리도 잘 놓고
빚이 수억이어도 허허 웃을 줄 아는게 중요한 것이지, 나 공부 잘 해서 SKY 졸업장 땄다 이런 건 상관없는 것입니다.
정말 그런 포부라면 님이 저에게 사업안을 얘기해줘야할 건데 그거 물어보면 답은 없을 것 같은데요
구체적인 게 없다면 그건 좋게 말하면 몽상이고 나쁘게 말하면 망상입니다.
\vspace{5mm}

다만 공부를 잘해서 간판따고 싶다라면 목표라면 주저할 것 없이 1년동안 빡세게 하면 되는데
문제는 요전번에도 똑같은 질문을 하신 것 같고, 그렇다면 지금 나 공부해서 미쳐 죽을 것 같다라는 하소연을 했어야하는데 같은 질문이란 거죠.
이렇게 텀 주면서 같은 질문하는 사람은 반년 뒤에도 또 같은 질문하더군요.
\vspace{5mm}

과탐 과목선택은 공개적으로 말할 건 아닌데 이걸 수학 이상으로 공부해야한다고 하던 건 2년 전부터 얘기했던 바라서요.
그런데 지금 공부하려면 등급보다도 님 하기 쉬운 것 하세요. 만약 지금 공부 그냥 하는 수준이면 서울대는 커녕 한양대 에리카도 힘듭니다.
무슨 소리냐 할지 모르는데 날로 n수생은 적체되어 쌓이고 있고 수험 노하우는 미친 듯이 높아지고 있습니다.
저 뿐만 아니라 소위 수험생들 돈 갈취하는 업자들이 다시 수능쳐도 좋은 대학 갈 점수 나올가.... 그렇지 않을 걸요.
다년간 죽어라 공부하면서 겉으로 공부 안 하는 척 하는 애들이 입시에서 잘 나갑니다.
\vspace{5mm}

지금 질문한 상황보면 이런 걸 고민할 게 아니라 빨리 문풀하면서 공부에 미쳐있어야하지 않나요.
\vspace{5mm}

그리고 콕콕 내애서 입으로만 공부하겠다... 저는 이런 거 정말 싫어합니다.
실제로 매년마다 합격하는 사람들은 공부하겠다라고 말한 적도 별로 없이 북괴가 땅굴파듯 남몰래 공부한 사람들이 대부분이죠.
입으로만 공부하겠다 하는 사람들은 이상한 교재들이나 구입해서 부모나 자기 등골이나 빨아먹고 인생낭비할 가능성도 있고.
\vspace{5mm}

이제 1월인데 추상적인 이야기는 그만두고 공부하는 이야기나 합시다.
\vspace{5mm}




\section{[상담 023] 내년 바라봐야하는 케이스}
\href{https://www.kockoc.com/Apoc/699642}{2016.03.28}

\vspace{5mm}

저는 열등감이 심하고 환경탓을 하는 재수생입니다.
\vspace{5mm}

고3땐 공부를 해보겠다고 인터넷에 여러좋다는 인강을 다 신청해 듣기만 했습니다. 학교에서 그 누구보다 인강강사를 꾀뚥고 다녓지만 정작 하위권이였습니다. 이과였지만 수학을 a형으로 돌리고 수능을 봣습니다. 국수영물지 44564로 기억합니다. 그후 공부를 아애 안하던 친구들과 같은 대학을 간다는 것이 자존심이 너무상해서 아무런 생각없이 재수를 한다고하고 부모님과 대판 싸웠습니다. 부모님은 어쩔수없이 허락을 해주셨고 저는 공부를 시작했습니다. 하지만 생각했던것과 달리 공부는 전혀 안됬습니다.. 그래서 내가 진정으로 원하는건 뭘까 하는생각에 아무생각없이 버스를 타고 종점까지 가서 걸어오거나 하면서 시간을 보냈습니다.
\vspace{5mm}

그후 공부를 해야겠다는 생각이 팍 들었습니다. (2월쯤) 몇일간은 공부법같은걸 짜면서 시간을 보냈고 마지막으로 고3때도 들었던 모 강사님의 쓴소리 영상을 듣고 공부를 하려고 영상을 들었는데 눈물을 흘렸습니다. 저가 쓴소리 영상을 듣고 울거라곤 생각치도 못했는데 말이죠.. 그래서 작년보다 열심히 하고 (10시간 정도..) 있었습니다. 하지만 최근 공부를 하면서 느꼈습니다. 작년과 똑같은 방법이 계속되고 있다는것을요.. 전과목을 인강을 듣고있었습니다. 60$\%$이상은 강의를 듣느라 허비했었습니다.. 수학은 주재도모르고 또 개념도 없는채 상위권 강사의 강의를 들었습니다.. 갑자기 여태까지한 공부가 아무 쓸모없는것같고 강의를 들은 부분을 다시보니 기억이 안나는 부분이 있어서 또 실패한건가 라는 생각을 하게되서 또 방황하고 있었습니다.
\vspace{5mm}

하지말라는 모든것을 열심히 실천중이였고 실패로가는 지름길을 그 누구보다 빨리 달리고 있었습니다. 주저리 주저리 말이 많았는데 이제와서 또 공부법을 물어보는게 기분나쁘다고 하신글을 읽고 핑계같은.. 말로 사정?을 말씀드리고 싶었습니다.
\vspace{5mm}

현재 공부한것은 국어 : 개념강의 끝 , 마닳 진행중 영어 : 문법진행중 수학 : 미적1 2 통계까지 기본강의 미2 알텍같은 강의를 듣고 있고 복습도 철저히 하고 있다고 생각하는데 그냥 풀이를 외운정도에 미적 12 확률 강의는 거의 날림으로 들어서 남는게 없어서 노베이스라고 생각됩니다.. 과탐은 개념강의 한바퀴돌리고 교과서랑 함께 복습중입니다. 길게 주저리 주저리 죄송하고 기분나쁘셨다면 재차 죄송합니다.
\vspace{5mm}

상담내용은 없앨 건 없앴습니다. 그리고 말씀드리지만 비굴해보일 수 있는 표현은 진심이 아닌 것에나 가깝고
업자나 좋아할 이야기입니다.
\vspace{5mm}

우선 님의 경우는 특수한 케이스가 아닙니다. 그냥 공부를 안 하는 친구들의 흔한 전형입니다.
공부 하겠다... 라고는 하면서 좋은 인강은 찾는데 시중교재도 제대로 풀지 않은 것이죠.
그럼 해결책은 간단합니다.
"시간"을 많이 확보하고, 자기가 불성실하고 게으름 피운다는 전제 하에서 하루에 할 수 있는 "최소 공부량" 정하고,
그대로 밀고 나가면 됩니당.
\vspace{5mm}

강의가 이해가 안 가는 건 '강의 내용을 이해할 생각도 없으면서 그냥 끝까지 들으면 공부가 될 것이다"라는 잘못된 신념,
거기다가 이미 두뇌가 그렇게 최적화되어버린 겁니다.
본 사연에서는 안 적었겠지만, 강의듣는 시간 말고 나머지 '놀았던 것'을 보면 분명 많이 놀았을 걸요.
까고 말하면 본인이 공부가 안 되었다고 하지만, 실제로는 뇌에서 공부를 핑계로 노는 시간을 확보해버린 겁니다.
\vspace{5mm}

지금 기준에서는 올해는 어렵고 내년이나 바라볼 수 있습니다.
현실적으로 제시해줄 것은 '가장 쉬운 문제집'만 골라서 1단원씩 풀어나가라는 겁니다.
쉬운 문제집이라고 하면 풍산자 개념, 연개수문, 쎈 A, B형, RPM, 그리고 기출의 2, 3점 문제가 있습니다.
어려운 것 손대지말고 저런 것들만 풀어나가면 되겠습니다만...
\vspace{5mm}

저야 눈치볼 것 없어서 걍 말하면, 과연 할 것인가... 의심이 드네요.
여기서 정답은 욕심 버리고 자기가 밑바닥인 걸 인정하면서 특정 범위 정한 뒤 그 쉬운 문제를 계속 반복해 풀어서 바닥을 다지고
모르는 것만 인강을 발췌해서 듣고 오답정리해나가면서 기회를 틈타 슬며시 어려운 문제에 도전하는 식으로 가는 건데
사실 다수가 이걸 못 합니다. 자기들 실력은 개차반인데 꿈만 높으니까 밑바닥부터 공부해야하는 걸 인정 못 하죠.
더 쉽고 빠른 길이 있을 것이다라는 생각 하에 또 이상한 인강이나 교재만 두리번 거리면서 부모들 돈이나 쓰는 패턴이 많습니다.
\vspace{5mm}

자세한 공부방법은 아래 회독수에서 얘기했지만 처음부터 천천히 가면서 회독수 올리는 룰을 지켜라.
그것입니다.
\vspace{5mm}

그 이상은 제가 말씀드릴 수 밖에 없어요.
\vspace{5mm}






\section{[상담 024] 슬로우비디오가 필요한 경우}
\href{https://www.kockoc.com/Apoc/699646}{2016.03.28}

\vspace{5mm}

시험 당일 : 국어시간 멘붕.
평소와 다른행동의 시전.
\vspace{5mm}

평소와같이 차분하게 화작문 다풀고(한 몇분 늦더라도) 비문학 쓱 훑어만보고 차분하게 문학을 풀고 비문학을 차 분 하 게 풀어야 하는데 매번 순서대로 풀던 옛날의 관습을 잊지 못하고 화작문 다풀고 (안그래도 20분 좀 넘게 한 25분정도 걸린상황에서) 비문학 쓱 볼 때 어려워보이는 지문들과 싫어하는주제들 (돌림힘, 법 지문)을 보고선 아 이거 \textbf{미리 한두지문이라도 풀어버리고 가야겟다 하는 생각으로 욕심부리다가 생각보다 돌림힘지문등등이 잘 풀리지 않자 멘붕온상태로 시간만 날리고 문학으로 넘어감.}
\vspace{5mm}

문학 갔다와서 비문학풀 시간을 보니 9시 30분이라 30분밖에 남지않은 평소보다 훨씬 쪼들거리는상황에 당황, 거기다가 읽는지문들은 머리에 하나도 들어오지 않는상황. 머리가 하얘지고 시간은 흘러만 갔음. 읽긴 읽는데 내가 지금 뭘읽고있는건지를 모르겠는 패닉상황의 연출. 평소처럼 지문을 통찰하며 읽고 지문을 이해한상황에서 문제를 푸는 것이 아닌 허겁지겁 지문 다읽었는데 하나도 기억안나는 상황에서 문제읽고 그부분 찾아서 풀고 또 다음문제 그렇게 풀고 하는 모든문제의 일대일대응화 현상이 발생.
한지문당 평소 4$\sim$5분이면 모든문제를 해결 가능했던 본인이 한지문에 10분씩 걸려서 풀어내고 그나마도 지문마다 존재하는 3점짜리 보기문제들은 멘붕상태에서 구체적 답의 근거도 확실치 못하게 대강짐작으로 풀어내는 현상 발생.
\vspace{5mm}

결국 10분남긴상황에서 두지문이상을 다 찍고 날려먹어버림. 찍은 문제중 한문제 맞고 국어 85점 3등급 끝자락 리타이어
\vspace{5mm}

결정적으로 국어 한과목 때문에 이번수능 의대입시도 실패함.
\vspace{5mm}

수학: 29번까진 만족스러웠는데 30번을 30분$\sim$40분정도를 시간을 쏟아부엇음에도 불구하고 결국 왠지 답일거같은 상황 2개를 놓고 이게 어느 지점부터 쭉 상수함수일까 아니면 다른상황일까(이때 가정한 이상황은 기억이 안나네요..)고민하다가 후자로 찍고 후자의 답을 냈는데 채점후 대충 반응을 보니 앞의 상수함수였던 상황 같더라구요.. 결국 30번 하나 틀려서 96점
\vspace{5mm}

1등급 턱걸이. 수학은 각종 고수들이 말씀하시는 `탈패턴화`의 부족이 실패요인이 아닌가
\vspace{5mm}

싶습니다. 아직은 30번을 맞출정도의 수준이 되지 못한다는..
\vspace{5mm}

영어: 국어시간때의 멘붕과 포기심정이 영향을 미친 시간.
\vspace{5mm}

정말 될데로 되라는 심정으로 시험을 치렀고 결론적으로 시간이 모자라진 않앗는데
\vspace{5mm}

정말 실력대로 틀렸다. 빈칸 한지문? 순서한지문? 넣기 한지문? 등 3점짜리로만 3문제를
\vspace{5mm}

고루 틀려주심. 영어는 평소 어느정도의 방심,자만?(시험이 절대 영어는 어렵게 나올수 없다)으로 인해 기본적 실력학습의 부족과 이비에스 내용정리 암기위주의 학습의 치부가 그대로 강타당한 것으로 분석됨. 너무 대충읽고 대충풀었으며 본인의 문제풀이 답 결정의 근거조차 명확히 정립하지 못한상황에서 `어차피 올 수능영어도 쉬울거임 ㅋ 이비에서 만세`라는 나댐정신으로 시험 응시한 꼴이 되었음.
\vspace{5mm}

지1-> 공부안한 부분(화학식)문제를 하나 틀려서(2번) 꼼꼼한 공부부족으로 분석됨 47점
\vspace{5mm}

생2->결론부터 말하면 3문제 찍고 두문제를 답갯수법칙으로 (9평 20번처럼)맞춰서
\vspace{5mm}

운좋게 47점 이지만 2등급 (1등급컷 48점) 시간도 모자랐고 수능끝난 직후에도, 채점하고 찍은거 맞은거 확인한 직후에도 내린 판단이 이과목은 해서는 안되는과목이구나 하는 생각이었음.
\vspace{5mm}

강대 아이들과도 문제도 만들고 유형별 희안하고 참신한 풀이방법 공유 및 문제빨리 푸는 스킬과 함께 대치동에서 돌아다니는 xxxx모의고사등을 단체로 복사해서 사용하는등 수학과 함께 공부를 가장 열심히 했던 과목이었음에도 불구하고 수능날 3문제나 찍어야 했고 운좋게 두 개를 맞추고 하나만 틀린 상황에서도 2등급이 떠주시는걸 보고 노답이라는걸 느낌.
\vspace{5mm}

다시 한다면 생2는 버려야겠다. 다니던 대학 전공과 관련있는 물1이나 생2보단 집단수준이나 표점 등급컷 백분위 모든면에서 나은 생1을 해야겠다는 생각이 듬.
\vspace{5mm}

결론은 실패요인은 국어시간의 멘탈관리부족, 평소 연습때와 다른짓거리를 한 것? 이 가장 실패에 큰 영향을 미쳤다고 생각합니다...
\vspace{5mm}

가볍게 기출문제만 대충 쓱(날마다)풀어보고 시험쳤던 15수능 국어에서도 98점을 받았었는데
\vspace{5mm}

그보다 훨씬 열심히 일년내내 날마다 30분이라도 공부해가면서 기출,ebs정리까지 다하고 문법 암기까지 열심히 했던
\vspace{5mm}

16수능 국어가 85점으로 국어 한과목때문에 전과목이 리타이어당한 상황입니다.
\vspace{5mm}

잘 정리하셨습니다.
그런데 이건 국어 때문만이 아니라 너무 학원가에 최적화되어있어서 그래요.
\vspace{5mm}

전 문풀을 시킬 때 일부러 천천히 더 풀라고 합니다. 오답이 나오는 이유는 '빨리 풀어서' 그래요.
학원가에서는 빨리 풀라고 가르치는 게 대세인데 이건 조금만 생각해보면 상당히 위험한 공략입니다.
수능문제는 내신과 달리 야매 써서 빨리 풀면 이득이 되는 문제를 안 냅니다. 예술적으로 꼬아내버리죠.
평소보다 천천히 풀어야만 함정에 안 걸리고 출제자의 의도를 읽을 수 있는 문제를 냅니다.
\vspace{5mm}

그렇기 때문에 평소에 빨리 푸는 친구들이 위험하다는 것이지요. \textbf{슬로우비디오 돌리는 능력이 없으면} 본 시험장 가서 망하거든요
슬로우비디오를 돌리는 능력이 있어야 디테일하게 가는 것입니다.
님은 국어 때문이라고 하지만 실제로는 디테일한 영역에서 나가리나신 겁니다.
그리고 님이 다닌 학원공부가 그것까지 보장해주진 못 한 겁니다.
\vspace{5mm}

평소에 공부하실 때 스피드를 강조한 결과입니다. 올해 공부하실 때에는
\begin{enumerate}
    \item 일부러 천천히 풀어서 정답률 높일 것
    \item 수능과 관계없지만 디테일을 보장해줄 수 있는 교재 볼 것 : 영어 - 실력, 수학 - 성문종합영어나 고난도 지문.
    \item 난해한 것들을 '도해'로 정리하는 습관 들일 것.

\end{enumerate}
\vspace{5mm}

노력이 부족한 건 아닌데 '디테일'에 필요한 노력은 되어있지 않죠.
이런 스타일들은 과거의 전설적인 본고사나 요즘의 논술, 수리논술에서는 쥐약이 됩니다.
답을 맞추는 데에만 주력하지, 왜 그런 답이 나오는가를 상세히 설명하는 공부가 덜 되어있기 때문이죠.
\vspace{5mm}

작년에 수능을 위한 공부는 했는데 그게 디테일이 부족한 사연자 분의 단점을 채워주진 못 했습니다.
올해에는 디테일을 채워줄 수 있는 교재들을 집중적으로 보시면서 '슬로우 비디오'를 찍는 연습을 하셨으면 합니다.
\vspace{5mm}

참고로 콕콕에서 이 조언이 필요한 다른 분이 계시죠
\vspace{5mm}






\section{[상담 025] HOT6의 경우 : 실적내지 못 하는 건 버린다}
\href{https://www.kockoc.com/Apoc/768967}{2016.05.11}

\vspace{5mm}

고민 글 : \url{http://kockoc.com/Nogari/768952}
\vspace{5mm}
\begin{enumerate}
    \item 손실은 바로 인정해버리는 게 편한 길이다.
    \item 기한 내에 실적을 낼 수 없으면 차라리 노는 게 낫다
    \item 과거는 잊으라.
\end{enumerate}
\vspace{5mm}

당사자는 여태까지 뚜렷한 실적을 내본 적이 없어요.
그래서 계획을 세우는데 실패하니까 \textbf{만회}하려고 더 거창한 계획을 세우는데 계속 말아먹습니다.
결국 자포자기에 빠져서 현실을 긍정하지만 현시창이라서 더 우울해집니다.
그래서 무리하게 올해 수능으로 가려고 하지만 당연히 실적이 나오기가 힘들죠.
\vspace{5mm}

빨리 군대에 가라는 건 다름이 아닙니다. 제가 보기에는 2년을 잡아야하는데 벌써 그 2년에서 3개월 정도를 날리셨기 때문입니다.
이런 걸 지적하면 당사자들(그리고 제가 그 입장이 되면 저 역시)은 \textbf{변명}을 합니다.
그런데 \textbf{변명≒실패} 라고 정리하면 끝납니다. 성공한 사람이 변명 같은 걸 합니까.
성공한 사람들은 남들이 이 정도면 되겠지 하는 것도 부족하다라고 보면서 '나는 더 배고프다'라고 노오력을 하죠.
\vspace{5mm}

여태껏 보낸 시간은 교훈으로 여기고 미련을 안 가지시면 됩니다. 손실을 바로 인정하면 마음은 편해집니다.
그리고 수능이 가능성이 없다라는 걸 바로 인정하고 수학만 공부하면서 군대에 가기 전에 여행도 다니고 여러가지 경험을 해보는 것이 낫습니다.
그래야만 자아성찰이 가능해집니다. 실패하는 패턴을 바로 잡을 수가 있어요.
꼬인 상태에서는 계속 공부해보았자 성과가 안 나옵니다요.
여담이지만 콕콕에서도 그런 친구들 꽤 많습니다만 자존심 문제건으로 제가 개입을 안 합니다(그럴 이유도 명분도 자격도 없으니까)
그러나 그렇다고 꼬인 상태라는 것이 부정되는 것은 아니지요.
\vspace{5mm}

20대 때 5년 실패한 걸 자산으로 죽을 때까지 실패하지 않는 것과
20대 때 잘 나가닥 30대 때 왕창 말아먹고 재기 불가능한 것 중 어느 것을 택하겠느냐 물어보면 그리 우울해지지도 않을 겁니다.
실패, 고생해보지 않은 사람들은 절대 크게 될 수가 없습니다.
우리사회의 잘못된 인식이 금수저가 계속 잘 나갈 거라고 보는 인식인데 전혀... 스펙이나 능력이 빠방하면 뭐하겠어요.
실패하고 고생해본 경험이 없는 사람은 결정적인 착오로 그릇된 판단을 하기 좋습니다.
\vspace{5mm}

실패를 선행학습한 셈 치고 재출발하시길 바랍니다. 그리고 지금의 그 쓰라린 패배감이야말로 배신하지 않는 소중한 벗이니 잊지 마시길.
가장 위험한 건 '자아도취'입니다.
\vspace{5mm}



\section{[상담 026] 막연하디 막연한 목표}
\href{https://www.kockoc.com/Apoc/787223}{2016.05.22}

\vspace{5mm}

평소엔 건강에 대해 별 생각을 안했었는데 수술받고 나니 건강의 소중함이 많이 느껴집니다.
\vspace{5mm}

다름이 아니라 이제 수험준비를 해야하는데 수술전에는 대략 평균 2등급대가 나왔었는데 근 2년간 공부시기를 많이 놓쳐 많이 하락했을것 같습니다. 목표는 의대입니다. 제가 아팠던 동안 환자에게 의사라는 이름이 주는 어떤 위안? 이런것들을 느끼며 저도 그러한 사람이 되고 싶다고 느끼게 되었습니다.
\vspace{5mm}

군문제는 면제도 가능하다 했는데 그냥 공익이라도 가게 되었습니다. 내년에 가게될 예정인데 수험기간은 공익복무 기간 2년을 포함해 대략 2$\sim$3년을 생각하고 있습니다. 중요한건 이 시간동안 목표를 이룰수 있는지라고 생각합니다, 남들보다 뒤떨어져 시작하는 주제에 말입니다..
\vspace{5mm}

수학은 나름 좋아하고 아폭님의 커리와 칼럼을 보고 따라서 진행하고 있는 중이고요, 영어는 유학준비로 나름 탄탄하다고 생각합니다.
과탐도 ebs강의를 들으며 시중교재들을 풀고있는중이고 수학과 마찬가지로 많은 문제들을 접하며 공부할 계획입니다. 하지만 국어는 참 애매합니다.
\vspace{5mm}

이 경우 문제는 막연합니다. 즉, 본인이 '공부 못 하는 것은 아니다'라고 얘기하고 있지 실제로 자기 문제점이 뭔지 파악이 안 되는 경우입니다.
수학은 어디까지 진행되었는가 알 수 없고, 영어는 이미 절평인 것에 대한 어떤 대비가 되지 않았으며
과탐은 시중교재 어떤 걸 푸는가 막연히 나오지 않으며 역시 '계획'이라고 봅니다.
국어가 막연한 건 문법도 그렇겠지만 신속히 읽고 정확히 논점 파악해서 객관식 문제를 푸는 것이 되지 않아서겠죠.
\vspace{5mm}

수험은 자기 꿈을 내세운다고 되는 게 아니라, 응시하는 사람들을 얼마나 짓밟고 올라서느냐 그걸로 결정됩니다.
그리고 시험 때 나를 배신하는 '나'를 억제할 수 있도록 실력을 키워야하는 것입니다.
이 질문의 경우 문제는 자기가 열심히 한다고만 말하고 있지, 구체적으로 어떤 문제집들을 언제까지 얼마나 끝낼 것이냐
그리고 모의를 쳤을 때 자꾸만 실수하거나 풀지 못 하는 킬러문제 어디가 빈약해서 무엇을 대비할지 나오지 않았단 겁니다.
\vspace{5mm}

이러면 구체적인 조언은 어려워집니다. 개정교육과정이라고 해서 내용들이 쉬워졌다고 하지만, 역설적으로 내용이 쉬우니까 문제가 어려워집니다.
그런데 막연히 의대 목표로 삼았다는 분이 그냥 '따라가면' 된다고만 생각합니다. 핵심은 남들보다 월등히 잘해 격차를 벌여서
타인이 보고 '저 새끼는 괴물이야' 소리나올 정도로 공부하는 것인데 적어도 그 자세는 여기서 나타나지는 않네요.
\vspace{5mm}

콕콕에서 일지를 쓰라고 한 게 이것 때문입니다. 본인들은 공부한다고 하는데 일지 진행만 보아도 그건 다수가 거짓말인게 드러나니까요.
말로는 $\sim$ 하겠다라고 하지만 실제로 일정을 제대로 지키는 경우는 많지 않습니다.
거꾸로 말해서 자기가 세운 계획을 3/4만 제대로 끝내더라도 성적은 잘 나옵니다
이렇게 막연히 하면 제가 다 조언드릴 수 없습니다. 가장 중요한 게 본인 스타일과 단점의 파악일 터인데 님은 '수험'을 소비로만 생각하시는 것 같습니다.
실제로 빡세게 공부했다고 보기도 힘들고, 2등급이라고 하지만 그건 정말 공부하는 자세와 거리가 있습니다.
\vspace{5mm}

올해 시험은 올림픽으로 치시더라도 지금부터 일정 잡아서라도 시중교재들을 한권씩 끝내고 그걸 일지로든 어떤 식으로든 기록하셔서 보고하시길요.
\vspace{5mm}
