



\section{실패와 재도전 01}
\href{https://www.kockoc.com/Apoc/481466}{2015.11.13}

\vspace{5mm}

어제 어디서 오답이 났는가 반드시 분석해보시길 바랍니다.
\vspace{5mm}

당연히 답은 정해져있죠.
\vspace{5mm}
\begin{enumerate}
    \item \textbf{아예 건드리지도 못 했다}
    \item \textbf{문제를 잘못 읽거나, 풀이과정에서 실수했다.}
\end{enumerate}
\vspace{5mm}

그런데 더 무서운 사실은
\vspace{5mm}

\textbf{건드리지 못 했던 문제는 알고보니 평소에도 못 풀었던 그런 문제들의 집합에 속해있고}
\textbf{문제를 잘못 읽거나 풀이과정 실수한 것도 역시 평소에 하던 그런 것이었다.}
\vspace{5mm}

일 가능성이 매우 높단 것입니다. 자, 이걸 파악하셨다면
그 다음에는 이걸 극복하고 재도전할지, 아니면 다른 길로 갈지는 본인이 선택하기 나름입니다.
(그 외에는 시험날 컨디션 관리를 개차반으로 했다거나, 아니면 트라우마가 떠오른다거나 기타 등이 있지만요)
\vspace{5mm}

이런 것들은 평가원이 의도한 바는 아닙니다. 평가원이 수험자 개개인의 성향까지 알 수는 없으니까요.
수험자 개인이 냉정하게 파악하면서 스스로 넘어서야 하는 라인입니다.
그리고 이 라인은 유감스럽지만 수능에만 나타나는 게 아닙니다. \textbf{수능이 무섭다고 다른 데 도망가도 따라오는 귀신}입니다.
\vspace{5mm}

어제의 오답을 정리한 다음 그 틀린 문제, 아리까리한 문제와 관련된 것을 집중적으로 다시 공부하기 시작하기만 해도 내년에는 넘어설 수 있죠.
그런데 보통은 여기서 "아 수능 그만둘까"하면서 \textbf{실은 놀고 자빠지다가 내년 3월까지 날려드십니다.}
\vspace{5mm}

무엇보다 점수가 잘 나오건 안 나오건 그건 정확한 이유라는 게 있습니다.
이번에 영어점수만 해도 자명합니다. 전 영어가 어렵게 나올 거라는 데 있지도 않은 머리털을 걸어댄 사람입니다(...)
근거 : 다들 너무 만만하게 생각하더란 것이죠. 그런데 평가원은 통수의 달인 아닙니까.
그리고 다들 공부한다는 게 EBS 연계만 보더라던 것(...) 이렇게 공부하는데 평가원이 어렵게 내면 끝장나겠군이라는 생각이 들덥니다.
실제로 점수가 안 나온 사람들 물어보니까 공부한 건.... 그렇게 점수 받아도 싸다라는 것이었습니다.
\vspace{5mm}

다른 과목들도 그렇습니다. 생각보다 잘 나온 사람들은 제가 아는 한 그래도 열심히 공부한 케이스입니다(제가 미친 듯이 잔소리하고 욕까지 한 경우가 많죠)
반면 생각보다 안 나온 케이스들 - 즉 제가 관리하는 범주에 들어간 경우(즉 상원)는 -
결국 충고를 안 지키거나 본인이 공부량이 부족하거나 아니면 위에서 얘기한 자기의 약점을 못 넘어선 경우입니다.
책임이 본인에게 있다 그런 걸 강조하는 게 아닙니다. 고득점이든 저득점이든 그건 분명히 인과관계가 있단 것이지요.
따라서 "나는 수능에 안 맞나봐, 이만 포기하고 현실에 '수능'할래"라고 하는 것은 호환과 마마에 필적하는 구시대 관습적인 미신이란 이야기입니다.
\vspace{5mm}

그리고 여기서 찌릿하자면 작년 11월부터 제가 빨리 공부하라고 했는데도
\textbf{아아주 여유있게 2월 3월에야 시동걸다가} 6~7월달에야 비로소 공부다운 공부를 시작해서 올라갈려는 찰나에 컷당한,
즉 공부를 늦게 시작해서 아쉬운 결과 거둔 케이스도 있습니다. 아니 사실 이게 많을 거예요.
어제 챗방에서 어그로 끄신 어느 할아버지 가수 좋아 하던 소녀는 자기가 놀았다고 하지만 이건 기만질이죠.
이 분 현역이어도 점수 잘 나온 이유는 제 입장에서 보면 간단합니다.
\begin{itemize}
    \item 첫째, 이미 올해 초에 제가 권하던 문제집은 거의 다 풀어서 남들보다 반년은 앞서있었다.
    \item 둘째, 본인이 자기 약점을 잘 극복한 케이스였더라는 것
    \item 셋째, 남들은 막판에 후달릴 때 본인은 보위 빠질을 하면서 심신안정이 가능했다라는 것.
\end{itemize}
\vspace{5mm}

뒤늦게야 공부시작한 분은 아이고 n수 어떡해하지말고 내년 수능 대비 빨리 하세요. '겪어'보셨을 테니가 이제는 제가 설득할 필요도 없겠죠?
공부 남 위해서 하는 게 아니죠. \textbf{자기가 좋으라 하는 거지.}
\vspace{5mm}






\section{실패와 재도전 02 - 노오력 이야기}
\href{https://www.kockoc.com/Apoc/481647}{2015.11.13}

\vspace{5mm}

노오력노오력 이야기하는 사람들이 많습니다만
어제 상담하고 탐문하면서 다시 느꼈지만, 그냥 노오력한만큼 나옵니다.
\vspace{5mm}

\begin{itemize}
    \item 첫째, 노오력은 연속함수여야 한다. 그런데 사람은 1분간 열심히 한 걸 1년동안 열심히 한 걸로 착각한다.
    \item 둘째, 정말 노오력하는 사람들은 자기 노오력이 부족하다고 여기고, 노오력을 하지 않은 사람들이 자기가 노오력을 많이 한다고 착각한다.
    \item 셋째, 노오력을 포기하는 사람이 끌려갈 건 약장수 아니면 점쟁이들이다.
\end{itemize}
\vspace{5mm}

자기는 노오력했는데 점수가 안 나왔다 징징댈 시간이 있으면
정말로 많이 공부했는데 왜 안 나왔을까 하는 \textbf{'치명적인 이유'를} 찾겠죠.
왜냐면 그 이유만 찾으면 신발 속 압정을 꺼내듯 문제가 해결되기 때문입니다.
어제 시험이 망했더라도 그 이유를 찾으면 다시 공부하면 되는 것입니다.
\vspace{5mm}

시험 실패로 인한 손해나 좌절. 이거 도망간다고 극복이 안 됩니다.
시험은 시험으로 이길 수 밖에 없어요. 도망간 곳에 낙원 따위도 없을 뿐더라, 추노꾼들은 끝까지 따라옵니다.
1번 실패해도 도망가면 그 1번이 죽을 때까지 따라오지만
9번 실패해도 1번 성공하면 그 9번 실패가 자산이 되는 게 시험입니다.
우리가 알고있는 일상의 합리적인 계산과는 차원이 달라요.
\vspace{5mm}

아, 물론 노오력은 스칼라가 아닌 벡터입니다. '방향'이 매우 중요하죠.
가령 잘못된 방향 - 예컨대 수학에서 탈패턴화를 하지 못 했다라거나, 영어가 쉽게 나올 거라고 근거없이 짐작해 소홀히 했다거나
10월에 막판 정리를 잘 해야하는데 자꾸만 실모나 실모 이벤트에 끌려다녔다거나 하는 것 - 대로 공부하면 당연히 좋은 결과가 나오기 힘들지요.
\vspace{5mm}

성공하는 사람은 세가지 특징이 있죠.
\vspace{5mm}
\begin{enumerate}
    \item 남이 뭐라든 노오력한다. 자기가 좋아서 노오력한다.
    \item 남의 말을 듣는 척 하고 듣지 않는다. 자기가 꼼꼼히 판단하고 검증한 방침에 따라 '침묵하는 소수'로 행동한다.
    \item 실천은 자기 뇌와 심장에 맡긴다. 단지 평소에 '준비'만 철저히 한다.
\end{enumerate}
\vspace{5mm}

어제 액상탄마님이 제 입장에서는 매우 감격스럽지만 경악스러운 글을 올리셨죠.
\textbf{한국에 있는 문제집은 그냥 다 풀었다.}
아주 명쾌한 해답입니다. 수험가지고 장사질하면서 그걸로 떼돈벌고 자랑질하는 천민들의 그럴싸한 방법론 따위는 불태우는 것이죠.
이 분 말대로라면 이래서 사탐은 만점이 나옵니다.
\textbf{그럼 수학은 왜 그러느냐.}
\textbf{그건 간단하죠 - 수학은 패턴화에서 반드시 탈패턴화를 해야하는데 이 분은 그 단계까지는 못 갔죠.}
그리고 제가 아는 한 이 분, 이렇게 공부 안했으면 수학도 그 점수도 안 나옵니다.
본인은 매우 분하겠지만 제 입장에서 보자면 "오우, 정말 드라마틱하게 올라갔네"라고 박수라도 쳐주고 싶죠.
\vspace{5mm}

현재는 한탄할 때가 아니죠.
시험 못 보았으니 포기한다거나 아니면 노오력해서 안 되나봐.
이거 우리들에게 잠재된 \textbf{노비 유전자의 발현입니다}.
남들 눈치나 보고 어떻게 하면 실패를 합리화할까 정당화할까 핑계를 댈까하는.
승분는 항상 지는 경우도 없고 항상 이기는 경우도 없습니다. 확실히 져야 또 확실히 이기죠.
"나는 수능이 안 맞나봐." - 이건 패배를 인정하기 싫고 주인이 매질할까봐 무서운 종놈의 마인드입니다.
\textbf{"아, 이번에는 졌네. 제기랄.... 으아악.... 영어 열심히 공부할 걸.... 내년에 두고보자 다 짓밟아주마"}라는 게 고구려 무사의 마인드죠.
\vspace{5mm}

그럼 남은 370일은 충분한 기간인가.
아니죠. 저 기간도 모자라죠. 하지만 좀 쉬셔야할 테니 다음 주에는 빨리 시작하셔야겠죠.
\vspace{5mm}






\section{실패와 재도전 03 - 출제경향과 어리석은 다수 이야기}
\href{https://www.kockoc.com/Apoc/481696}{2015.11.13}

\vspace{5mm}

다수를 따라간다는 건 정답이기도 하고 오답이기도 하죠.
그런데 다수를 따라간다는 건 엄밀히 말하면 '위험회피'를 하기 위해서입니다.
다수로 뭉치면 명백히 위험을 줄일 수 있는 것들이 많죠.
\vspace{5mm}

- 전자제품, 식료품 : 이 경우 다수가 소비하면서 검증해준 케이스
\vspace{5mm}

하지만 다수를 따라가면 안 되는 케이스가 있죠. 그건 바로 '소수'만이 승리하는 경우, 즉 \textbf{투자 아니면 시험}입니다.
매년 수능 보시면 알지만 수험생들은 자기들이 '대비 안 한' 과목에서 털리는 경향이 있습니다.
그럼 왜 대비를 안 하느냐. 그 과목은 쉽게 출제될 거라고 믿어서입니다.
그럼 왜 그렇게 믿느냐, 여태까지 \textbf{그렇게 출제되었고, 다른 친구들도 그렇게 말해서랍니다.}
\vspace{5mm}

자, 제3자로 타자화시켜보니까 얼마나 어리석은지 보이십니까.
이 글 읽는 상당수가 저 케이스죠.
\vspace{5mm}

수험사이트에서 뭐라고 지껄이든 - 이 글조차도 - 자기 소신껏 다 어렵게 나올 거라고 생각하고 공부하는 게 정답이었지,
사실 생각해보면 얼마든지 뒤집힐 수 있는 게 출제 경향인데 이건 어렵게 나오고 저건 쉽게 나올 것이다를 믿는게 바보죠.
어제 시험은 \textbf{전부 어렵게} 나왔습니다.
그 전까지 수험사이트 들락나락거리던 (아마 공부도 제대로 안 했을 걸요) 사람들이
뭔 깡으로 물수능 비난하고 100 100 100 을 외쳤는지 생각하면 어이가 없지 않나요?
\vspace{5mm}

인터넷의 단점이 이거죠. 어리석은 데 선동당하거나 세뇌당하기도 좋다는 것입니다.
어떤 특정 정보 A가 있으면 그걸 자기만의 정보라고 착각하기 쉽죠. 컴퓨터는 개인용이니까.
하지만 그걸 수백명 이상이 보았다면 더 이상 필요한 정보가 아닙니다. 정보는 '소수'만 누릴 때 가치가 있죠.
\vspace{5mm}

자 본격적으로 비판해볼까요. 실모가 좋다고 얘기들합니다. 그런데 이것도 '소수만' 누릴 때나 의미가 있죠.
많이 팔리는 실모라면 저라면 구입하지 않을 겁니다. 개인적으로는 실모의 문제 퀄이라는 건 평가원의 그것과 거리가 있거니와
특히 해설은 시중교재만도 못 한 경우도 많았지만, 무엇보다 "다수"가 본다면 그건 큰 의미가 없기 때문이죠
그걸 풀 시간이 있으면 제 단점을 극복했을 것이고 수학의 경우라면 차라리 어떤 발상들이 가능한가 그거 노트정리나 했을 겁니다.
\vspace{5mm}

어제도 흔한 고득점 인증을 보면 사일런트 마이너리티더군요.
즉, 침묵하던 소수.
비결에 대한 대답을 보니 '목동'이었습니다, 예, \textbf{양치기}였단 것이죠.
그럼 수험에 대해 떠들던 수험생들은 어땠나
안 보이던 사람들도 많더군요.
\vspace{5mm}

새로울 게 전혀 없는 현상입니다.
어떤 의미에서는 수험사이트가 없는 게 본인 인생에 도움이 되는 케이스도 있지 않을까하는 생각도 듭니다.
\vspace{5mm}






\section{실패와 재도전 04 - 나쁜 공부 야기}
\href{https://www.kockoc.com/Apoc/482111}{2015.11.13}

\vspace{5mm}

여기서부터는 논란이 되는 이야기이고 아마 이 글 이후로 제가 안 보일 수도 있는 민감한 이야기입니다.
하지만 어떤 정밀한 법칙을 추구하자면 성역은 없어야겠죠.
\vspace{5mm}

허혁재님의 오수썰을 겨냥하죠(그 후 아폭을 본 사람은 아무도 없었다)
\vspace{5mm}

이 글을 다들 어떻게 감동적으로 읽었는지 모르겠습니다만 - 아무튼 허혁재님은 막판에 역전승을 거두었으니가 저걸로는 해피엔딩이죠 -
그 글에 쓰이는대로 하라고 한다면 저는 반대입니다.
사실 그 글은 보충과 해명이 필요한 게, 매번마다 왜 본인이 실패했는가에 대한 분석과 반성이 있었어야죠.
그렇지 않으면 아 5수해도 되는구나라고만 안일하게 생각하기 좋습니다.
\vspace{5mm}

제가 왜 여기 창을 겨누냐면 이유는 간단합니다.
사람은 좋은 것만 공부하는 게 아니라 \textbf{나쁜 것}도 배웁니다.
청년기에 술담배 피우고 불건전한 19금에 빠져들고 하는 것도 '공부'입니다. '나쁜 공부'이죠.
문제는 우리 뇌는 나쁜 공부를 좋아한다는 것입니다. 뇌는 이율배반적인 기관입니다.
다른 장기와 달리 '자살'을 명령하는 곳입니다. 다른 장기들은 주인의 생명을 이어가려고 하는데 뇌는 그렇지 않죠.
\vspace{5mm}

오수썰의 문제는 거기서 잘못된 공부습관을 배울 수 있기 때문입니다.
주인장 자신도 의식 못 하는 좋지 않은 스타일이  영웅적으로 게시되어있다면 다른 수험생들이 그걸 보고 배울 가능성이 높죠.
만약 이걸 본인이 잘 거른다면 괜찮겠지만요.
\vspace{5mm}

그런데 수험가들의 상품을 보면 제가 아는 한 "SKY 현역합격"은 별로 없더군요.
꼭 그런 사람이 잘 가르친다라고 할 수는 없겠습니다만, 제 입장에서는 그렇습니다. "수험 실패 바이러스"가 그렇게 전파되는 것이 아닐까.
지금 잘못된 관습 중 하나가 10월에 모의고사를 막 풀어대거나 새로운 문제를 도전하는 것이지요
아마 다른 공무원이나 고시 시험이면 난리날 이야기입니다.
10월은 매일 시험시각에 맞춰 뇌를 컨트롤하는 것, 그리고 그동안 공부한 것을 다 정리하는 시간이지
무슨 실모를 풀어댄다든가 파이널 문제를 도전한다든가 그럴 시기가 아니죠.
그런 걸 하려면 8~9월에 했어야합니다. \textbf{그걸 풀고 정리하고 자기 것으로 만들 시간이라는 건 생각보다 오래 걸리기 때문이죠}.
수험이란 뇌를 만들어나가는 과정입니다. 문제를 풀고 충격먹은 뒤 정리하고 하는 데 족히 1개월 이상은 되어야 자기 것이 됩니다.
\vspace{5mm}

거기서 제가 느낀 건 그것이죠. 저 사람들, 정말 \textbf{수험이 뭔지 모르는구나}.
공무원 수험가 같은 데 보시죠. 거기 강사들이 강의를 언제 마무리하고 그리고 언제까지 진모하는지, 그리고 막판에 뭘 시키는지
시험 5~6개월 전에 기본강의 다 끝내고, 2~3개월 전에 진모강의 진행하고, 그리고 1달 전까지 OX 정리하는 식으로 스케줄을 짜놓습니다.
거기 강사들은 물론 오랜 장수생도 없지 않지만 대부분이 합격생 출신(변호사 포함)이죠. 수험의 대가들입니다.
\vspace{5mm}

그런데 웃긴 게 정반대로 수능 수험시장은 그런 수험의 고수들이 와서 힘든 판에 말이지요.
\vspace{5mm}

무슨 새로운 문제 푼다거나 기발한 발상.... 어제 시험에 그런 게 있었는지 물어보고 싶네요.
그런 건 하나도 없습니다. 다들 인수분해해보면 쓰인 것들은 단순한 것들입니다.
\textbf{물론 일부 과목고수들이 묘기를 보여주죠}. 한데 수험새 본인들이 시험장에써 쓰지도 못 할 묘기라면 그게 소용이 있나요?
시험장에서는 무조건 제 시간에 풀고 맞는 게 장땡입니다. 노가다 풀이를 하더라도 점수 따는 게 목표죠.
\vspace{5mm}

노오력만 가지고는 안 된다고 한다면 이걸 지적하겠습니다.
input이 많아도 효율성이 꽝일 수도 있죠. 똑같은 물도 젖소가 마시면 맛있는 우유가 되지만 뱀이 마시면 맹독이 됩니다.
잘못된 시스템에 올라타면 아무리 노력하더라도 성과를 낳긴 커녕 꼬라박는 수가 있어요.
\vspace{5mm}

합격하는 사람들은 대개 침묵하는 소수들입니다. 제 수험은 과거 수험이므로 세대 차이가 없지는 않을 것이기 때문에
그런 사람들의 합격기를 분석해보거나, 콕콕에서 성공한 사람들 이야기를 다 종합해보면 결론은 간단하더군요.
\vspace{5mm}

\textbf{문제집을 남들보다 많이 풀 것, 남들보다 빨리 많이 공부할 것, 집중할 것만 집중할 것, 제대로 정리할 것}
\vspace{5mm}

재미없죠? 너무 기본적이죠? 그런데 기본을 강조하는 이유는 그 기본을 지키는 사람이 별로 없기 때문입니다.
\vspace{5mm}

그런데 저건 다른 사람들도 재미없죠. 이른바 '업자'들 입장에서 그렇죠.
학생들이 다 저렇게 공부를 해대면 장사가 되지 않습니다.
만약 시중교재만 열심히 풀어대고 딴 것 신경쓰지 않고 자기 공부만 해서 성적이 오르면 일확천금의 꿈은 날라가는 것이죠.
그래서 그들은 자기들만의 뭔가 검증되지 않은 썰을 풀어대죠(그 사람들 스펙이 좋은지는 뭐 본인들에게 물어보시죠)
상당수가 그런데 현혹당하지요. 그리고 나중에는 인지부조화를 거쳐서 자기가 대학에 가려는 건지, 수험상품 마니아인지 혼동해버립니다.
\vspace{5mm}

그런 상품들은 어디까지나 자기 공부가 어느 정도 된 상태에서나 선택적으로 보는 것입니다.
하지만 현실은 기본 공부도 안 된 친구들이 막판에 공부에 질려서 그런 상품에 돈바치면서 굽신굽신거리고들 있더군요.
\vspace{5mm}

1년동안 한 이야기고 어제 시험 결과로 어느 정도 드러난 것도 있으니 굳이 설명할 필요도 없겠죠
성적이 좋았다는 경우는 역시 침묵하는 소수의 양치기파들입니다.
무엇보다 영어가 쉽다는 수험가 이야기나 수험사이트 썰만 듣고 그걸 믿은 사람은 손해를 보았죠.
\vspace{5mm}

내년에는 부디 그러지 마시기들 바라겠습.... 니다만 그러지 않을 정도면 진작 공부 잘 하거나 합격했겟지요.
아무튼 저는 부조리한 것을 지적 안 하는 건 비겁하다 여기므로 여기서 통렬히 지적드립니다.
\vspace{5mm}






\section{실패와 재도전 05 - 상식 파괴}
\href{https://www.kockoc.com/Apoc/484851}{2015.11.14}

\vspace{5mm}

\begin{enumerate}
    \item 1. 수학 5000문제를 풀어야 실력이 오르는 게 아니라, \textbf{5000문제를 틀리고 오답정리해야} 실력이 늡니다.
    실상 강의라는 것도 강사 하는 말대로 받아쓰기이니까 편해서 그렇죠. 그건 스스로 생각하는 기회를 막아버립니다.
    틀려보고 왜 틀렸나 분석하고 그걸 넘어서야 실력이 늘어납니다. 인간이 원래 그렇게 진화된 동물입니다.
    \vspace{5mm}

    \item 배경지식이 필요없는 건 수학입니다.
    \textbf{국어, 영어만큼 배경지식이 도움되는 과목도 없습니다}. 배경지식이 필요없다라고 하는 건 뭐라고 해야할지.
    물리 빠삭한 사람은 국어 A형 더 쉽게 풀었겠죠. 물론 배경지식이 없으면 추론해야겠지만 이 역시 다른 배경지식의 도움을 받아야합니다.
    +1수 하시는 분들은 1년 아깝다 또 이런 오만방정 떠실 건데 그럼 1년동안 책 200권을 읽어보세요. 책 읽을 시간이 이 때 빼고 있나요?
    \vspace{5mm}

    \item 좋은 교재 찾다가 좋은 n수 합니다.
    안 좋은 교재를 피해가는 건 맞습니다. 안 좋은 교재 피하는 방법? \textbf{비닐랩 씌운 것만 피해도} 80$\%$는 다 피합니다.
    그리고 수험사이트에서 자주 회자되는 수상쩍은 교재는 그냥 안 보는 게 좋습니다. 그런 게 적중했단 이야기 듣지도 보지도 못 했음.
    서점에서 딱 내용 공개해놓은, 현직 교사들이 쓴 교재들이나 우선 다 푸세요. 그리고 기출 문제집 나오면 기출 2종 이상 반복해 돌리시길.
    어차피 교재 추천해달라는 초보들, 그거 해줘도 소화 못 시킵니다. 또 인터넷 사이트 어슬렁거리고 있지요
    \vspace{5mm}

    \item n수하는데 돈 나간다 그딴 이야기 하지 말고 EBS 들으세요. EBS 강의 퀄이 사설을 능가하는 사례도 있는 판국에 왜 비싼 돈 쓰나요.
    강의평 검색해보면 2013년 강의인데 2015년에도 '고맙습니다'라고 후기 올라오는 강의 같은 게 좋은 강의입니다.
    EBS 강의도 듣지 않고 나 돈 없어 어떻게 공부해 하는 사람들은 그냥 공부하기 싫어서 돈없다고 핑계대는 것일텐데요
    \vspace{5mm}
    
    \item 어렵게 공부하세요. 이번 수능에서 열심히 공부한 사람들도 있겠죠. 하지만 그 사람들은 2015 수능 기준으로 공부했다는 게 문제입니다.
    이번에 영어 쉽게 나올 거라 한 사람들 책임지나요? (저야 영어 쉽게 나오면 머리칼이 자라겠지에 한 표 던진 사람입니다)
    수학만 잘 하면 된다, 국어나 영어는 EBS 연계 보면 된다고 한 사람들이나 업자나 그 업자 패거리들이 뭐 하나 책임진 것 보았습니까.
    사실 이거 고민할 것도 없음. 그냥 전과목이 어렵게 나온다라고 생각하고 미친 듯이 공부하면 되죠
    \vspace{5mm}
    
    \item 이제 시험은 360일 밖에 안 남았습니다.
    내년 시험 준비를 내년 3월 오기 전까지 마무리한다고 해야 그나마 성공할 겁니다.
    열심히 공부하면 서울대 가겟지... 가 아니라
    그냥 지방대가 300일 걸리면 평범한 사람 기준으로 서울대는 1500일 걸린다 - 이 기간을 어떻게 단축하느냐가 관건인 겁니다.
    즉 원래 4수해서 서울대 간다가 아니라 평범히 공부해서는 4년이 넘게 걸렸던 거다, 이걸 어떻게 줄이는 것이냐로 바꿔 생각하시길 바랍니다.
    \vspace{5mm}
\end{enumerate}
    
    
    
    
    
