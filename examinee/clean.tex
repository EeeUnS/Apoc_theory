%https://tex.stackexchange.com/questions/25903/how-to-put-a-long-piece-of-text-in-a-box
% \noindent\fbox{%
%     \parbox{\textwidth}{%
%     퀴즈\\
%     2개의 문제집이 있다.
%     A는 3년간 이의제기도 없고 오탈자가 발견된 적도 없다.
%     그리고 그걸 푸는 사람들은 좋은 점수를 받았다고 한다.
%     B는 꼭 특정단원에서 무리수를 두는 경향이 있다.
%     그리고 특정단원만 갈 때마다 가필이나 수정을 해야한다는 악평이 있다.
%     단 B도 푼 사람들이 합격하는 경향이 있다.
%     이 중 한권을 골라야한다면 어느 것을 고르겠는가.   
%     }%
% }

\section{[정리론 001] 시험 때까지 하지 말아야할 것은 무엇인가.}
\href{https://www.kockoc.com/Apoc/215719}{2015.07.26}

\vspace{5mm}

학습론이 연초에 공부 시작하는 방법과 방향을 통틀어 말한 것이라면
정리론은 이제 본 시험에 대비하는 자세에 대해서 정리 겸 해서 올릴 것이고 시험 이후에 지울 것입니다.
\vspace{5mm}

그럼 왜 정리론이냐고 하냐면 지금부터는 이제 '학습'이 아니라 정리를 준비해야하기 때문입니다.
멘탈정리와 정신적 자세에 대해서도 얘기하겠지만, 아마 현실적인 문제풀이 및 정리에 대해서 갈 것입니다.
\vspace{5mm}

자, 이제 100일이 남았습니다. 사실상 학습론에서 얘기했던대로 양치기를 꾸준히 했고 EBS 스펙 따라잡았은 분들이라면
'많이들 지쳐있을 터이고', "내가 이래서 성적이 올라갈 것이긴 하느냐"하는 회의가 들겠지만 무엇보다 지쳐있을 것입니다.
그리고 제가 드렸던 말씀이 기억나실 겁니다. 여름이 되면 공부하기 매우 힘들어지고, 특히 체력 때문에 맛이 가버린다,
겨울방학 때부터 봄까지 제대로 학습할 수 있는 시간이다.
\vspace{5mm}

아무튼 지나간 과거는 과거이고 지금부터 님들이 해야하는 건 학습이 아니라 정리입니다.
학습론에서 말하는 학습이 "점수를 높이기" 위한 것이라면, 이제부터 쓰는 정리론에서 말하는 정리란 "점수를 깎아먹지 않기 위한" 것입니다.
\vspace{5mm}

그럼 우선적으로 해야할 것을 적자면,  "먼저 하지 말아야할 것"을 알아둬야할 것입니다.
\vspace{5mm}

\begin{itemize}
    \item \textbf{첫째, 남들의 말에 휘둘리지 마라.}
    \vspace{5mm}

    역시나 7월 정도 되어서 공포심을 매개로 마케팅을 하는 세력들도 등장합니다.
    사실 이런 상품 따라간다고 점수가 올라가는 건 아닙니다. 만약 그런 걸로 점수가 올라갔다면 모두가 그런 상품을 소비하겠지요.
    또한 각종 사이트에서 수험생들의 마음을 흔드는 온갖 글들이 올라옵니다. 그런데 냉정히 생각해보시길. 그 중 하나라도 실제 도움되는 게 있는지.
    딱 하나 도움 된다면, 공부하기 싫고 포기하고 싶은 자기 자신을 '변명'하고 '정당화'하는 핑계거리 찾는 데는 매우 도움이 될 것입니다.
    \vspace{5mm}

    적어도 수험사이트 같은 데를 방문한다면 그냥 베스트로 올라오는 낄낄거리기 좋은 막장글이나 논쟁글 보면 되는 것이지
    그 외 수험정보 글은 실제로는 마케팅을 배후에 깔고 있다는 데 유의하시길 바랍니다.
    실제로 님들이 지금 멘탈 추스리고 시간을 투입하면 그나마 수능점수를 지킬 수라도 있지만 시험보는 날까지 그런 노이즈에 좌우되면
    올해도 정말로 물 건너가겠고, 사람에 따라선 자기 인생에 대한 강력한 회의에 사로잡힐지도 모릅니다.
    \vspace{5mm}

    \item \textbf{둘째, 책을 늘리지 마라.}
    \vspace{5mm}

    이건 양치기를 어느 정도 했다는 걸 전제하기도 하지만, 사실 양치기가 안 된 사람일지라도 이제는 책이든 실모든 사로잡히지 말아야합니다.
    대놓고 말합니다만 수능에 나오는 문제들 대부분은 시중교재에 나온 것 그대로입니다. 킬러는? 새로 판촉되는 책들이든 실모든 소용없습니다.
    오히려 자기가 보았던 책들을 다시 한번 읽으면서 회독수를 높이고, 무엇보다 \textbf{내가 어떤 단원, 어떤 문제에 약한가,}
    \textbf{주로 하는 실수가 무엇인가를 점검해보는 게 좋으며, 특히 "나노 단위로 공부해야하는 것"에 더욱 집중하는 것이 낫습니다.}
    \vspace{5mm}

    간혹 수험사이트에 보면 참고서를 한번만 보고 집어던지면서 건방지게 문제집을 평가하는 사람들이 있습니다만,
    그건 좋은 학습태도가 아닐 뿐더러, 사실 그런 사람들이 정말 좋은 실적을 거두는지도 의문인 경우가 많습니다.
    참고서는 5번은 보아야 제대로 보았다고 할  수 있을 뿐더러, 10회독 이상 가면 처음에는 몰랐던 행간의 지식이 드러나게 되고
    그 이후 회독수가 높아질수록 거기서 배운 지식이 유기체처럼 움직이게 된다는 것을 느끼게 됩니다(이건 경험해보아야 알 수 있습니다)
    \vspace{5mm}

    예컨대 쎈을 한번 돌려도 1등급이 안 나온다... 라는 이야기의 문제는 그것이죠. '1번만' 보았기 때문입니다.
    어느 책이든 1번만 보면 그건 제대로 공부한 것이라고 할 수 없습니다. 여러번 돌리고 자기가 취약한 부분 빠짐없이 다시 점검하고
    책 내용 구석구석을 훑어야 도움이 되는 것입니다.
    \vspace{5mm}

    여태껏 보았던 교재들을 다시 한번 겸손하게 돌리고, 틀렸던 문제나 의문이 갔던 것 또 풀어보고 자기만의 약점노트에 정리하며
    - 약점노트를 만들기 싫으면 페이지를 따로 표시해서 다시 찾아볼 수 있게 색인을 만드는 것을 권합니다 -
    그렇게 해서 교재 한권의 회독수를 10번 이상 만드는 데 주력하시길 바랍니다.
    돌리면 돌릴수록 한 겨울 302동에서 굴린 눈덩이가 샤 대문을 박살내듯 님들의 지식은 미친 듯이 불어나게 됩니다.
    \vspace{5mm}

    \item \textbf{셋쨰, 비판하는 사람을 멀리하라}
    \vspace{5mm}

    정말로 진정한 의미에서 걱정하고 배려해주는 사람은 비판은 안 합니다. 그냥 뭐가 부족한가 지적만 하겠지요.
    그런데 여전히 정신 못 차리고 이 시점에서 근거없이 까내리거나 비난하는 케이스가 있습니다만,
    그런 얘기는 사실 "너 망하라"고 하는 이야기이니 싹 무시해도 좋지만, 일단 그런 사람은 멀리하는 게 좋습니다. 재앙바이러스 감염자거든요.
    \vspace{5mm}

    실제로 수험 결과는 어떻게 될지 아무도 모릅니다.
    저런 사람들은 자기 친구가 좋은 결과가 나오면 찡그리고, 나쁜 결과가 나오면 속으로 고소해하면서 겉으로 위로하는 척 합니다.
    하지만 더 비극적인 사실은 저런 사람들 자체가 결과가 좋은 경우가 별로 없는데다가, 그러다가는 심지어 '자살'까지 하는 케이스도 있단 것이죠.
    저런 자살예정자들의 한마디에 휘둘려서 입시날까지 망하는 웃지 못할 경우가 있습니다.
    그런 말을 들으면 귀를 씻길 바랍니다.
    \vspace{5mm}

    \item  \textbf{넷째, 공부환경에 있어서 돈이나 다른 자잘한 걸 생각하지 마라.}
    \vspace{5mm}

    1시간 이상 쾌적하게 집중할 수 있다면 그 1시간에 5000원을 지불하는 건 아깝지 않습니다. 인생이 바뀌는 문제이니까요.
    적어도 공부를 어느 정도 한 사람이면 지금부터 시험 보기까지 1시간은 5월 이전의 하루에 맞먹는 가치가 있습니다.
    구체적으로 하루에 10만원이면 1000만원 꼴일 것인데, 이건 나중에 좋은 대학에 가서 갚는다라고 마음먹고 그 정도 투자를 하셔도 됩니다.
    맛있는 걸 먹고 싶다거나 좋은 필기구를 쓰고 싶다, 고급독서실에 다니고 싶다 - 돈 생각하지말고 하길 바랍니다.
    이래야만 '포기하고 싶어하는 자기 내면의 또 다른 악마'가 설치지 못하게 할 수 있습니다.
    그 악마는 어떤 핑계거리를 찾아내서 무조건 포기하게끔 만들려고 하고, 막판에는 자기 주인이 '죽어서 지옥'에 가는 걸 자기 사명으로 삼고 있죠.
    \vspace{5mm}

    \item  \textbf{다섯째, 시간을 날렸다 하더라도 그건 '필연적인 휴식'으로 생각할 것. 절대 낙담하지마라.}
    \vspace{5mm}

    가령 하루이틀 뻗었다면 그건 그동안 쌓인 피로를 풀기 위해서 나가야 할 '세금'이었던 것입니다.
    특히 한여름이면 일주일에 하루이틀 정도 그러는 건 당연할 수도 있습니다.
    그러니 그런데 괘념치 말고 지금부터 형식 구애 안 받고 '그동안 풀었던 교재들'을 정리하는 즐거움을 삼길 바랍니다.
    시험 당일날 문제가 되는 건 문제를 못 풀어서가 아니라, 가령 ㄱㄴㄷ 선지가 있는데 한 단어나 심지어 한 글자 차이로 헷갈리는 것.
    그 1인치의 차이 때문이라는 걸 상상해보시면서 정리하시면 됩니다.
    \vspace{5mm}

\end{itemize}
우선 이 다섯가지만 지키더라도 많은 손실을 예방할 수 있을 것입니다.
\vspace{5mm}






\section{[정리론 002] 문제를 풀고 정리한다의 의미.}
\href{https://www.kockoc.com/Apoc/217718}{2015.07.26}

\vspace{5mm}
\begin{enumerate}
    \item 단순히 답만 맞춘다
    \item 해설을 읽고 이해한다.
    \item 각 선지에 대한 해설을 모두 기억하고 남에게 설명할 수 있다.
    \item 자기가 보았던 기본서나 강의까지 덧붙여 그 해설들을 평하고 보강할 수 있다.
    \item 새로운 해설을 만들 수 있다.
\end{enumerate}
\vspace{5mm}

보통 공부했다고 한다는 게 1, 2번에서만 그치는 게 문제이겠습니다.
\vspace{5mm}

기본적으로 공부했다고 하는 것은 3번부터 출발해야 하며
지금부터 시작해야하는 정리론에 따르면 5번 경지까지 이를 수 있어야합니다.
새로운 해설을 만들 수 있는 사람은, 새로운 문제를 보고 즉석에서 해설을 만들 수 있으니, 그만큼 정답률이 높아지겠지요.
\vspace{5mm}

또한 기출 정리에 있어서도 마찬가지인데 - 기출에 대해서도 자기가 직접 해설을 쓸 수 있도록 그 기출문제를 중심으로 개념정리를 해야합니다.
그것도 자기의 학습 스타일과 주로 하는 실수패턴으로 말이지요.
가령 그림에 강한 친구라면 기존 해설을 '그림'으로 고쳐서 생각, 정리할 수 있으면 좋겠고
반면 논리가 강한 친구라면 그림 해설조차도 논리적 명제로 환원시켜 정리할 수 있어야겠지요.
\vspace{5mm}

눈치빠른 친구들은 파악할 겁니다. 예, 저렇게 하면 지금 절대시화되는 강의라는 건 정말 보조적인 도구가 되는 것입니다.
모든 강의는 강사 편의적으로 재해석된 것에 불과하기 때문이죠.
정리하라는 건 다름이 아니라 자기에게 맞는 무기를 갈고 닦고 탄환을 넉넉히 준비해나가는 것입니다.
학습량을 늘리라고 하는 것은 그만큼 많이 풀어보아야 많은 오답이 나오고 많은 오답이 나와야 자기 문제를 더 많이 파악하기 때문입니다.
수능에 나올 수 있는 문제들을 위해서 양치기하라는 게 아니라, 더 많은 양치기를 해야 "나 자신의 공부가 어디서 문제인지" 드러난다는 것입니다.
그 문제점을 이제 파악하기 시작해야합니다. 100문제 중에서 오답이 30개라면, 1번 정리해서 15개로, 2번 정리해서 5개로, 3번 정리해서 2개로.
이런 식으로 줄여서 자기가 틀리거나 모르는 문제에 대해서 등급을 매기고, 왜 틀렸는지 철저히 반성문을 쓰고 파악합니다.
그리고 거기서 추출된 순도 100$\%$의 자기 약점을 커버해줄 수 있는 부분만 다른 교재를 발췌해보거나 EBS 인강을 발췌해서 듣습니다.
하다 못해 콕콕에서 그런 걸로 질문하고 토론해보는 것도 좋은 시도입니다.
\vspace{5mm}

웃긴 것이 겨울방학 때 수험사이트에서 놀다가 여름에 와서 뒤늦게 학습량 늘리겠다는 것인데요. 또 이러면 이러는데로 망합니다.
저는 분명 겨울방학 때부터 봄까지 양치기하라고 했지, 늦여름까지 질질 끌라고 한 적은 없습니다.
지금 학습량을 막 늘리는 건 자살행위입니다. 이제는 정리를 제대로 해야할 시기이고, 사실 지금부터 출발해도 정리가 늦습니다.
굳이 학습량을 늘리고 싶으면 저런 정리를 꼼꼼히 한 후 자기 약점을 파악한 뒤, 그 약점에 해당하는 부분만 드릴링하시면 되는 것입니다.
\vspace{5mm}

썰렁한 농담입니다만 양치기를 했으면 이제 양의 털을 깎아야하지요.
\vspace{5mm}






\section{[정리론 003] 단권화에 대하여}
\href{https://www.kockoc.com/Apoc/218740}{2015.07.27}

\vspace{5mm}

현실적인 단권화
\vspace{5mm}

\sout{"한권으로 몰아넣는다"}

"자기가 익숙하고 가필하기 쉬운 책을 중심으로 핵심정보 인덱스만 적어나간다"
\vspace{5mm}

단권화를
시중 기본서로 할 것인지
교과서로 할 것인지
해설 풍부한 기출문제집으로 할 것인지
아니면 모의고사집으로 할 것인지는 자기 마음입니다만.
보통 중하위권이면 EBS 중심으로 가시게 되시겠지요.
\vspace{5mm}

보통 단권화라함은 온갖 잡다한 것을 다 알고있어야하는 고시나 공무원 시험 준비생들이 '과거'에 쓰던 작업입니다.
각 교재마다 서로 부족하거나 잘못된 기술이 있으니 그걸 바로 잡아서 한권으로 몰아넣는 것이죠.
- 이거야말로 진정 한권으로 완성한다라는 의미이겠습니다, 마케팅을 위해서 제목만 그럴싸하게 붙인 게 아니라 -
하지만 요즘은 그럴 필요가 없어졌죠. 곧 사라지는 고시든 공무원 시험이든 나오는 기본서가 이미 단권화된 것들이 많으니까요.
\vspace{5mm}

그리고 이런 질문도 나올 겁니다. 수능은 사고력 테스트인데 단권화가 필요한가.
\vspace{5mm}

단권화할 대상을 바꿔 말하면 되겠지요.
그건 겨울방학부터 지금까지 여러분들이 공부하면서 확인하고 정리한 스스로의 '약점'들입니다.
잘 이해하지 못 한 개념이라거나 틀린 문제 등을 빨간펜 처리하면서
개념 - 기본문제 - 기출 - 변형문제 - 고급문제 - 논술급 문제
의 연계관계까지 잘 고려해서 자기가 약하거나 잘 모르는 것, 혹은 출제유력하다고 생각하는 것을 정리해서
손에 익은 교재에 가필하거나 다른 문제집 문항이나 기출문항을 표시하고 알고리즘을 구체적으로 적는다면 그것이 훌륭한 단권화입니다.
책 이것저것 다 필기할 필요 없이, "자기가 정말 모르거나 약한 부분"만 추려서 정리하면 되는 것이지요.
\vspace{5mm}

그 대상은 자기에게 손이 익은 것으로 해야합니다.
수능 시험장에 가져갈 것으로 정한 다음 거기다 적어두면 되죠.
물론 가장 좋은 것은 '머리'로 기억하는 것입니다만 그것이 어렵기 때문에 자주 보는 교재 중심으로 해야합니다.
\vspace{5mm}

그렇게 단권화를 시키다보면 1차원적으로 쓰여져 있는 개념이나 기본문제를 스스로 변형하거나 재창조할 수도 있게 됩니다.
내가 출제자라면 어디서 까다롭게 낼까를 생각해보게 될 것이고, 그런 출제자의 마인드를 흉내낼 수 있게 되면 이것이 점수 향상으로 이어지죠.
\vspace{5mm}

정리할 때는 막연히 보기보다는 서브노트를 만들던가 단권화를 하든가 해서
"출제자"의 입장에서 나를 어떻게 요리할까라고 3인칭 관찰자 시점에서 바라보고, 그에 맞게 어떻게 대처할까 게임하는 기분으로 전략을 짜야합니다.
님들이 학원이나 인강에서 시시콜콜 적어주는 필기는 사실 별 소용이 없습니다(대부분은 시중 두꺼운 기본서에 있거나 스스로 발견할 수 있습니다)
그러나 \textbf{나 자신의 약점이나 보완해야 할 사항}만큼은 반드시 기록하고 정리해야 합니다(그 점에서 콕콕 일지가 도움이 되었으면 좋겠지만요)
\vspace{5mm}

중요한 것을 추리는 건 8:2 파레토의 법칙대로 가시면 되겠습니다.
지금부터 정리과정을 밟으면서 그동안 공부한 것 중에 자기에게 크리티컬한 20$\%$를 찾아 그걸 반복할 것이며
그 20$\%$에서 또 20$\%$를 추려내는 식의 등비수열적인 과정을 거쳐야합니다.
\vspace{5mm}

\begin{itemize}
    \item 국어 
    \begin{itemize}
        \item 문법서 : 아무거나 한권 제대로 정리하시길 바랍니다.
        \item 문학, 비문학독해 : 스스로 글을 읽고 사고하는 법을 정리해보시는 편이 좋습니다.
    \end{itemize}
    
    
    \item 수학 - 시중교재 어떤 것이든 좋으나, 자기가 빈번하게 까먹거나 약한 공식, 개념, 스킬, 그리고 특정문제만 따로 정리해두는 것도 좋습니다.
    \vspace{5mm}
    
    \item  영어 - 영문법 중 정말 취약한 20$\%$만 추려서 보실 것, 그리고 역시 EBS에서 중요한 것만 추려서 보실 것.
    \vspace{5mm}
    
    \item 탐구 - 시중 기본서, EBS, 그리고 사설을 다 돌린 뒤 중요한 것만 자기가 손에 익은 책에 가필해두는 게 좋으시겠죠
    \vspace{5mm}
\end{itemize}

정리과정은 양치기보다도 힘든 겁니다. 양치기가 점수의 외연을 넓히는 과정이라면 - 즉 마그마가 분출해 용암으로 흐른다면
정리는 그 용암이 굳어져서 산이나 고지대 형성하는 과정이기 때문입니다.
사실 양치기를 그동안 안 하시고 왜 지금까지 하시느냐라고 여쭙는 건 좀 잔인한 짓이긴 한데 잊지는 마시길.
양치기한만큼 정리해야할 것도 그만큼 늘어납니다.
만약 양치기를 많이 했고 정리도 그만큼 했다면 고득점을 기대할 수 있겠죠.
하지만 양치기를 못 했다고 하더라도 끝까지 양치기 끌고가면 '공부는 했는데 점수가 안 오르는' 게 당연해집니다. 정리를 못 했으니까요.
즉, 정리없는 양치기는 소용없단 것입니다. 목표한 것의 1/3 정도 밖에 못 했다면, 그 1/3을 먼저 '정리'하는 게 8월부터 취해야할 과정입니다.
\vspace{5mm}

적어도 스케줄에 있어서는 뒷북 안 치고 이렇게 하라고 말씀드리는 것이니 이 충고대로 해서 손해보는 일은 적을 겁니다.
아울러 9평의 경우도 몇등급 나왔느냐보다도, 철저히 오답정리를 해서 단권화를 꼭 하시길 바랍니다.
\vspace{5mm}






\section{[정리론 004] 수능접수시즌과 추석}
\href{https://www.kockoc.com/Apoc/218893}{2015.07.27}

\vspace{5mm}

12월달부터 꾸준히 하라는대로 공부한 사람들이면
드라마틱한 변곡점을 겪기 마련인데 그 시기가 바로 7월 중순부터 8월 중순까지입니다.
수험 생활을 잘 꾸려오던 여학생들이 체력 때문에 무너지고, 반면 곰팅이 같은 남학생들이 반복학습으로 치고올라가는 시기이기도 하죠.
\vspace{5mm}

11월달까지 본의아닌 불연속점을 최소 2개 이상은 거치게 되어있으니
하나는 수능접수, 다른 하나는 추석입니다.
전자의 경우는 아무개는 어디 추천받았다더라, 지균 전형 TO 먹었다다러, 누구는 그냥 재수한다더라하는 노이즈 속에서
공부한 것은 없는데 벌써 접수야, 수능 응시일이 한달만 늦춰졌으면 좋겠다라고들 생각합니다.
당연히 제가 권하는 건 간단합니다. 자기가 가는 응시전형만 바라보고 남들이 뭐라하건 그건 일체무시하십시오.
\vspace{5mm}

그런데 유별나게 신경써야하는 건 바로 추석이온데
고향에 내려가야한다 부모님 도와야한다 그러겠지만, 이건 지금부터라도 부모님께 확약받아내서 '공부하는 시간'으로 돌리도록 하시길 바랍니다.
시골에 내려가야한다거나 내려가지 않더라도 집안일이 있다고 하더라도 분명 "열외받을" 수 있도록 지금부터 강조하고 확인받는 게 좋습니다.
당연히 그 때 어디가서 공부할 것인가 하는 것도 미리 잡아두고 뭘 공부할 것인가 - 문제집을 정리할 건가 파이날 강의를 들을 건가
이런 것을 미리 스케줄 잡아놓아야 추석 때문에 말아먹는 비극을 피할 수 있습니다.
(만약 입시학원이라면 이건 어느 정도 해결되지 않을까 싶은데, 그게 해결되지 않는 경우라면 이 역시 미리 준비하십시오)
\vspace{5mm}

헛된 희망 중 하나가 그동안 공부 못 한 걸 추석 때 다 해소할 수 있다하는 건데
현실은 하루만 놀자 하다가 결국 끝까지 탱자탱자 놀아먹고 더 좌절해버리는 비극이지요.
저런 준비를 안 한다면, 차라리 추석 전까지 밤새면서 미친 듯이 공부하고 추석 때 그냥 먹고 자고 하는 게 나을 것입니다.
\vspace{5mm}

즉, 정말 지금부터 추석을 어떻게 쓸 것인지 미리 스케줄 세우고 공부하기 위한 준비를 철저히 해서 이용하든가,
아니면 추석 전까지 체력 갉아먹으면서 하루 3$\sim$4시간 자고 카페인 음료 마시면서 공부하고 추석 연휴동안 힐링하든가.
\vspace{5mm}

저 두가지 중 하나에 해당해야합니다.
다시 강조하지만 아무런 준비 없이 "아, 추석에 공부해야지"라고 해보았자 절대 안 합니다.
그거 할 사람이었으면 이미 지금 해야할 공부 다 마쳤죠.
\vspace{5mm}






\section{[정리론 005] 회독수}
\href{https://www.kockoc.com/Apoc/221864}{2015.07.29}

\vspace{5mm}

똑같은 시간을 할당한다고 할 때 가장 효과적인 독서는?
\vspace{5mm}

    
\begin{enumerate}[label=\alph*]
    \item -  밑줄 그으면서 책을 정독한다
    \item - 메모해나가면서 읽는다
    \item - 빨리 10번 읽는다    
\end{enumerate}


어떤 것이 효과적인 공부법인가?

\begin{enumerate}[label=\alph*]
    \item -  노트필기를 한다.
    \item - 인강을 듣는다
    \item - 백지에 써본다.
\end{enumerate}

정답은 캠퍼스 커플.
\vspace{5mm}

수험공부는 백지를 보고 무에서 유를 창조하지는 못 하더라도 이끌어내는 능력에 좌우된다.
이걸 요약하면 "백지인출력"이라고 하겠는데
사실 깊은 이해이든 어떤 신박한 암기든, 저 백지인출력이 없으면 아무 소용이 없다.
문제를 잘 푸는 사람은 냄새나는 시험지를 받아보면 한문항을 볼 때마다 자기가 공부한 것들을 '짜임새있게 인출해야' 한다.
\vspace{5mm}

그리고 여기에 도움이 되는 독서법은 무엇인가.
개인적으로도 그렇고 다른 사람들의 사례나 직접 임상실험(?)시켜보면서 느낀 것이지만
밑줄도 형광펜도 심지어 노트필기도 "그냥 많이 반복해서 읽는 것"을 못 따라간다는 것.
당연히 반복해 읽는다는 것은 그에 따른 문제풀이나 백지에 서술형 요약문 써보는 작업이 병행되어야 효과가 좋은데
아무튼 밑줄은 100줄 당 한줄, 형광펜은 한쪽에 한 단어 정도면 족하고
정말 중요한 내용이면 책갈피에 페이지 표시하고 키워드 적으면 되고
그 외는 그냥 부지런히 읽고, 소리내어 읽어서 남들이 3번 읽으면 난 30번 읽는다는 걸 실천에 옮기는 게 '반박할 수 없는 공부법'이 아닐까.
\vspace{5mm}

그런데 재밌는 건 하라는 양치기를 안 했던 사람들이 하필 이제 와서 양치기를 시작하며(...)
더군다나 실모라든지 요약교재 같은 것을 마구 추가하기 시작하는데. 다른 것 떠나서 이거 타이밍이 정말 안 맞는 것이다.
오히려 문제풀이를 처음에 많이 해보고, 그렇게 약점파악을 한 후, 그 다음에 개념서를 더 자세히 반복해 보면서 숙달한다가 좋은 게 아닐까.
지금부터 해야하는 것은 자기가 풀었던 문제집 다시 훑고 그 회독수를 늘려나가면서 그래도 모자라면 교재를 추가하는 것이지
지금이라도 양치기 하면 안 되어요하다간 정말 날라가버릴 수가 있다(왜 하라고 할 때는 안 하고 뒤늦게 하는지)
\vspace{5mm}

지금까지 그래도 꾸준히 해서 기본서 한권이 제대로 숙달되어 있다면, 그 기본서를 이틀이나 사흘에 한번씩 돌리면서
계속 가필하고 참조할 것 색인번호 적는 식으로 단권화하면서 자기가 틀리거나 어려운 문제를 스스로 해설을 만들어보는 게 지금부터 해야할 일이다.
뒤늦게 양치기하는 애들은 정말 중요한 것과 덜 중요한 것을 구분 못 하다가 나중에는 양에 못 이겨서 넉다운당한다.
이와 달리 여러분들이 하셔야하는 것은 이제 옥석경중을 가리고 보았던 교재 다시 반복해보면서 숙달을 하는 것이고
어느 과목이든 매우 어렵다고 느껴지던 기출문제의 풀이를 기본교재 어디서 '추론해낼 수 있을까' 스스로 머리를 굴려 사고해보는 것이다.
이런 사고를 안 하면 킬러를 푸는 풀이는 자칭 고수들이 독점하고 있으니 그런 것을 비싼 돈 주고 강의를 들을까 교재를 살까 낚이게 되는 것이다.
\vspace{5mm}

중하위권 기준으로 보자면 EBS 수특 수완은 10회독 이상할 것. 9번이면 어떻고 8번이면 어떻느냐, 그건 해보지 않으니까 하는 얘기이다.
특히 수학의 경우는 기본 개념, 공식, 관련 정의는 정말 여러번 백지에 따라써서 마치 정확한 영어대사처럼 구사될 수 있도록 하는 게 좋다.
영어나 탐구는 뭐. EBS 교재는 매일 하루에 돌려서 시험 전날에는 달달 외우고 있다... 는 건 기본이 되어야하지 않을까.
\vspace{5mm}






\section{[정리론 006] 친구}
\href{https://www.kockoc.com/Apoc/222305}{2015.07.30}

\vspace{5mm}

유명한 만화 "쥐" 첫장면.
\vspace{5mm}

수험에 있어서 서로 격려해주는 친구는
\textbf{"나보다 못 하거나" 아니면 "나보다 너무 월등해 내가 얻어먹을 게 많다"} 이외에는 사실 없다고 봐도 좋다.
왜 이리 인생을 비관적으로 사세요라는 지적이 먼저 들리니까 얘기하지만
그런 지적한 녀석들이 100$\%$ 다 나중에 비관주의로 전향하더라는 개인적 경험, 그리고 이건 절대 과거완료가 아니라 현재진행형이다.
여담이지만 친구관계가 오래가기 위한 비결은 3가지
\begin{description}
    \item [첫째], 금전적인 것은 아예 언급을 피한다
    \item [둘째], 가능하면 '덜' 만난다.
    \item [셋째], 서로 일하는 분야가 '꼬인' 관계다.    
\end{description}
\vspace{5mm}

앞으로 수험생들은 수능날까지 적지 않은 noise에 시달릴 것이다.
하나는 수험생들을 위하는 척 하지만 실제로는 cash에나 관심이 많은 장삿꾼들이 발산하는 것이고
다른 하나는 수험에 투자한 부모님들이 안절부절 못하는 초보투기꾼 간섭을 하신다거나, 주변 친구들이 격려를 가장한 악담을 퍼붓는 것이다.
\vspace{5mm}

실제로 자기 친구가 너무 스펙이 좋아서 내가 뭘 빨아먹을 수 있겠구나 하는 계산이 아니라면
자기 친구가 좋은 대학에 간다라는 건, 설사 내가 성공했다고 할지라도 \textbf{기분나쁜 일인 것은 틀림없다}.
그게 왜 그럴까 묻는다는 건 동물의 본성을 이성적으로 풀이하겠다는 것만큼이나 어리석은 일이다.
그 이야기는 결국 관점을 바꿔 말하면 어느 정도 거리 유지가 되지 않는 한, 친구에게 수험 이야기를 주고받는 건 어리석은 짓이란 얘기다.
나이를 먹으면 철이 드는데 왜 상호격려를 못 하죠 얘기할 필요가 없다. 괜히 사촌이 땅 사면 대장항문외과에 가는 것이 아니다.
\vspace{5mm}

지금부터는 사실 수험사이트에 가서 시험 경향이 어떤지 자기가 몇등급인지 그런 것 따질 필요가 없다.
시험 경향을 수험사이트에서 적중시킨 경우는 별로 없고(게다가 적중시킨 경우라고 하더라도 바넘효과스러운 애매모호한 이야기)
자기가 몇등급이냐 그런 것 따질 시간에 더 많은 공부를 해서 등급을 높이는 게 낫다.
그럼 3등급이면 만족하고 공부를 중단하고 \textbf{1}등급이면 펑펑 놀아도 된다 그런 게 아니지 않은가.
\vspace{5mm}

하지만 더욱 중시해야하는 건 '친구'라고 여겼던 애들이 기대하지 않았던 모습을 보여주는 것,
특히 그들의 언어가 비수로써 자신의 수험을 좌지우지하는 웃지 못 할 상황에 대비하는 것인데
그들이 격려를 가장한 악담을 하는 경우라면 그냥 "너나 잘 하세요"라는 식으로 직접적으로 대꾸해주는 것이 좋다.
사실 이 경우라면 험한 말을 해도 좋다고 보는 게, 그런 식으로 간사하게 공격하는 녀석은 정면돌파로 대처해주는 게 가장 낫기 때문이다.
그리고 거기서 진심이 드러난 이상, 이 녀석은 영 안 되겠어라는 경우는 "절교"까지 고려하는 것도 나쁘지 않다.
친구가 없어서 꼬인 삶보다는 친구를 가장한 좋지 않은 인연에 꼬여서 망한 삶들이 더 많기 때문이다.
\vspace{5mm}

아무튼 정리에 들어간 수험생들은 외부로부터의 신호라는 것을 수능날까지 더욱 더 차단할 필요가 있다.
그 중에서 도움되는 정보, 그딴 건 아무 것도 없다.
만약 특정과목이 $\sim$ 하게 나온다라는 정보가 공개된다? 그럼 그 정보는 이미 공개되었으므로 쓸모가 없는 것이다.
혹시 매우 좋은 교재가 있지 않을까요?
저자 능력도 의심스러운데 마케팅만 해댄 비싼 책들이 적중한 적은 역시 단 한번도 없다.
수능에서 실점하는 문제들은 모두 시중교재를 수회독씩 하고 정리하다보면 도출해낼 수 있는 것들이다.
\vspace{5mm}

수험에 유의미한 signal은 봄부터 같이 해와서 내 손때가 탄 교재에만 있다는 걸 명심하면 된다.
\vspace{5mm}






\section{[정리론 007] 광신}
\href{https://www.kockoc.com/Apoc/222317}{2015.07.30}

\vspace{5mm}

수험과 투자의 공통점은 많지만 그 중 하나는 "피리부는 사나이를 쫓아다니는 풍토".
\vspace{5mm}

수험과 투자는 공통적으로 정보비대칭성이 잠재돼있다.
그래서 과학적 접근을 해야하는데, 현실적으로는 뭔가 수상쩍은 사람의 말과 글을 맹신하는 경향이 있다
- 이 글도 비슷한 범주에 들지 않을까 싶으니 읽는 사람들은 알아서 잘 걸러 읽으시길 바란다 -
\vspace{5mm}

인강 강사나 교재 저자는 절대 교주가 될 수 있는 입장이 아니다.
모든 컨텐츠 생산자들은 사실 혹독한 비판을 받고 그걸 넘어섬으로써 고객을 만족시켜 먹고사는 것이고 이게 당연하다.
하지만 시장은 비이성적인 광분을 보여주고 있고, 특히 공포심이 좌우하기 좋은 수험계일수록 이게 심하다.
특정 강사나 특정 저자를 맹신하고 심지어 팬덤을 보이는 것은 그렇게 함으로써 \textbf{"공포심"을 가릴 수 있을 것이기 때문}이다.
마치 당신은 100년 안에 뒈집니다라는 예언을 듣고 그 점술가나 무당에게 수천, 수억원을 퍼다주듯이 말이다.
\vspace{5mm}

그런데 이게 일시적이라면 모르겠는데 절대 그렇지 않다,
인지부조화 현상이 있기 때문이다.
분명 상대가 아무 것도 아닌데다가 내놓는 것도 형편없고 거기다가 자기가 속은 사실까지 확인되어도
그 대상에 대한 맹신이 압도적인 나머지, 그 맹신을 정당화하기 위해 '상식'이야말로 잘못된 것이고 누군가의 음모라고 모는 것이다.
이런 사람들은 절대 소수가 아니다.
상식적인 사람들이 유리수라면, 저런 맹신도 혹은 그런 예비맹신도들은 초월수이다.
- 중간 퀴즈 : 유리수와 초월수 중 어느게 많을까?
\vspace{5mm}

여기서부터는 민감한 이야기인데 논란의 여지가 있겠지만, 논란의 여지가 없는 게 아니면 글이 재미가 없으므로 쓸 필요가 없다는 변명.
공부경력이 없는 고3과 짬밥있는 n수 중 어느 쪽이 희망적이냐하고 하겠냐면 난 당연히 '영계'를 고른다
그건 젊어서도 힘이 좋아서도 아니다. 그냥 고3들은 최소한 "맹신"에 타락했을 확률이 낮다.
그런데 n수 이상부터는 본인들은 부정하겠지만, 적어도 내 입장에서 보자면 "낙방"으로 인한 트라우마 때문에
계획과 행동을 냉정하게 하는 것은 이미 물건너갔고, 오히려 어떤 객관성이나 확률을 가장한 \textbf{일종의 맹신 증세를 보이는 경우가 많다}.
\vspace{5mm}

구체적인 예를 들어보자.
맹신 증세를 보이지 않는 냉정한 사람이라면 우선 기출분석을 냉정하게 일주일만에 끝낼 것이고
그 문제들이 어디서 나왔는가 교과서 문제집 등을 확인해보고 자기 약점이 무엇인가 따진 뒤 그에 해당하는 것만 공부하지
아무개 강사가 좋다더라 어떤 야매책 꼭 봐야한다더라는 '노이즈' 같은 건 그냥 차단해버릴 것이다.
노이즈를 차단하고 자기가 정한 커리를 뚝심있게 밀고나간 뒤 그 다음 모의고사를 치고 나서 약점파악한 후 그 때 그 때 건전한 대책을 만들 것이다.
사실 저거야말로 '가장' 쉬운 것이고, 그리고 공부시간을 많이 확보하며 집중해 나갈 수 있는 방법이다.
\vspace{5mm}

그러나 현실은 어떤가. 다수의 n수생들이 보이는 경향은 "특정상품"에 대한 맹신을 보여준다.
- 그게 여기 주인장님이 내신 교재라고 할지라도 예외없는 건 당연하다 -
아무개 강의를 들어야 한다거나 모 교재만 보면 된다거나 하면서 1년의 절반을 오락가락 노이즈에 좌우되어 허송세월한다.
그러니 처음에는 반짝하지만 성과가 나올 리가 없다. 그리고 자기 공부 기준은 이미 날려먹어서 제조업 사업 망한 그리스나 똑같다.
그래서 여름이 되어서야 이제 양치기를 해볼까 하는데 또 되는 게 없다, 그리고 가을이 되면 또 다시 '한권'에 올인해보자라면서
수상쩍은 교재나 인강에'만' 의존한다 - 그리고 자기의 문제점이 뭔지 바로잡지 못 한다.
그런데 더 무서운 건 저 맹신은 3$\sim$4년 지속되는 경우도 있고, 심지어 심하면 10년 이상도 간다는 얘기다.
\vspace{5mm}

구소련이나 나치독일이나 북한에서 포로들을 제정신이 아닌 상태로 만드는 방법이 이거라든가

\begin{itemize}
    \item  24시간 빛이 안 드는 곳에 가둬버린다.
    \item  3일 내내 계속 같은 질문을 반복한다.
    \item  구덩이를 파게 한다. 또 메꾸게 한다. 다시 파게한다. 또 메꾸게 한다.
\end{itemize}
\vspace{5mm}

고작 수능시험에 2$\sim$3번 실패하고 끝나는 게 아니다. 사실상 그로써 게슈탈트 붕괴까지 이르는 경험을 2$\sim$3년간 하는 것이다.
거기다가 날려버린 세월을 한번에 보상받아야겠다는 초조함이 오히려 정상저인 공부를 방해하는 것이다.
이러다보니 그들이 xxx 최고예요라고 하는 신도로 전락하는 건 일이 아니다.
아니, 이것이 바로 사이비 종교에서 평범한 사람들을 신도로 만들어 돈, 노동력, 심지어 몸과 자녀들까지 갈취해먹는 방법이기도 하다.
\vspace{5mm}

재밌는 건 그 맹신도들이 이런 글을 보면 절대로 자기들이 그렇다라는 걸 '인정'하지 않는다는 것이다.
오히려 강한 거부감이나 공격을 드러낸다. 왜냐면 진실을 인정하는 건 "나를 부정"하는 것이기 때문이다.
사실상 이건 '세뇌된' 케이스라고 할 수 있는 바인데, 이건 정말 어지간해서는 풀리기 어렵다.
그렇다고 내가 당사자에게 "너 세뇌되었어"라고 해보았자 칼맞지 않겠나.
\vspace{5mm}

이와 관련해 또 재밌는 경험을 적으면 그렇다.
상대가 나에게 어떡하냐고 물어본다
나는 $\sim$ 하라고 가르쳐준다(그건 너무 상식적이고 사실 너무 당연해서이다)
상대는 $\sim$ 보면 안 되느냐, $\sim$ 하면 안 되느냐라고 반문한다.
나는 다시 반론하면서 $\sim$ 는 $\sim$ 해서 문제고 $\sim$는 $\sim$ 해서 그렇다라고 지적한다.
\vspace{5mm}

그런데 결론은? 10명이 물어보면 \textbf{결국 8명은 자기 하고싶은대로 하더라는 것}.
그럼 애당초 왜 질문을 던진단 말인가.
아니, 그런 질문을 한 이유가 무엇일까 생각해보면 그것이다.
자기 방법이 '옳다'라고 \textbf{지지를 얻고 싶었던 것}이다.
그건 그 사람이 자기 방법이 타당한가 아닌가 그런 건 관심없이, 자기 방법을 '믿음'의 영역에 두고 있단 얘기다.
\vspace{5mm}

그렇기 때문에 그런 사람들은 자기 수험 방법이 비판, 비난당하면 절대 가만히 있지 못 한다.
가령 야매교재를 보는 사람이 건전하다면 그 야매교재의 문제점이 지적되면 그 점을 감안해 보완하려고 하는 게 맞다.
하지면 현실은 그런 야매교재 왜 까나요 하면서 키배를 벌이려고 한다.
가장 중요한 건 본인 수험의 결과이고 따라서 문제가 되는 걸 바로잡는 것인데, 그게 아니라 왜 까나요라고 감정적으로 나오는 것이다.
(재밌는 건 이런 사람들도 타 교재를 까거나 그런 흐름에 동참한 경우가 많다는 것이다)
\vspace{5mm}

위와 비슷한 이야기는 학습론으로 작년 말 정도에 했던 걸로 기억한다.
그 이후로 쭉 관찰하고 보았지만 사실 크게 달라진 건 없는 것으로 안다.
잔인한 지적을 하자면 그 사람들에게 중요한 건 자기 인생이 어떻게 풀리느냐가 아니라, 현재 이 상태의 '자존심'이다.
내 인생이 어떻게 될 지는 신경쓰고 싶지않아 부모님이 다 해줄거야. 하지만 \textbf{내 자존심에 상처입히지마!}
\vspace{5mm}






\section{[정리론 008] 투자자 마인드 1}
\href{https://www.kockoc.com/Apoc/223860}{2015.07.31}

\vspace{5mm}

%https://tex.stackexchange.com/questions/25903/how-to-put-a-long-piece-of-text-in-a-box

\noindent\fbox{%
    \parbox{\textwidth}{%
    퀴즈\\
    2개의 문제집이 있다.
    A는 3년간 이의제기도 없고 오탈자가 발견된 적도 없다.
    그리고 그걸 푸는 사람들은 좋은 점수를 받았다고 한다.
    B는 꼭 특정단원에서 무리수를 두는 경향이 있다.
    그리고 특정단원만 갈 때마다 가필이나 수정을 해야한다는 악평이 있다.
    단 B도 푼 사람들이 합격하는 경향이 있다.
    이 중 한권을 골라야한다면 어느 것을 고르겠는가.   
    }%
}


만약 당신이 평범한 수험생이라면 A를 당연히 고르는 것이다. B를 고르는 건 미친 짓이기 때문이다.
A 문제집은 특히 '우등생'을 좋아하는 수험생 입장에서 보자면 우등생 문제집이기 때문이다.
\vspace{5mm}

그러나 당신이 대국적 관점을 보는 투자자라면 다르다. 나도 그렇지만 B를 선택할 것이다.
그 이유는 B가 특정단원에서 무리수를 두는 경향이 있으며, 가필과 수정을 해야한다는 \textbf{리스크를 '알고' 있기 때문}이다.
그리고 알고있는 리스크인 동시에 해결할 수도 있다.
\vspace{5mm}

반면 A는 어떤가. 지금까지는 아무 '이상'이 없었다. 그러나 그런 상태가 계속 보장된다는 보장은 어디에도 없다.
이런 경우 특정시험에서 A를 본 친구들이 아작이 나버리는 경우 어떻게 할 것인가.
한번도 문제제기가 된 적이 없기 때문에 어디가 부족하거나 잘못되었는지, 또 어떻게 대처해야할지 모르는 것이다.
그래서 만약 앞으로 치러질 시험이 A의 약점을 겨냥한 것이면 A만 믿고 가는 사람들은 크리티컬을 제대로 먹게 되어있다.
\vspace{5mm}

고3들은 그래도 학교나 부모를 통해 리스크 관리를 받는다. 기대수익(=기대점수)은 낮을지라도 변동은 비교적 경미한 편이다.
그러나 n수 이상들의 문제는 아래에서 지적한 대로 일종의 종교를 믿는 경향도 있지만
자기들이 응당 해야하는 리스크 관리를 '안 한다'라는 문제가 있다.
리스크 관리를 안 하면서 점수만 높이려고 한다.
\vspace{5mm}

투자자의 목표는 많이 버는 게 아니라 얼마나 리스크를 줄이느냐이다.
잘못된 판단을 하면 그 투자액 전체가 제로, 심지어 마이너스로 돌아설 수도 있다.
1년동안 모든 것을 투자하는 수험생도 이와 다를 바가 없다.
공부를 많이 하는 것보다도, \textbf{어떻게 하면 공부에 방해가 되는 요소를 줄일까 고민해야하고}
\textbf{자기가 실제 시험에서 감점당하는 부분이 어디일까 하는 것들을 추려서 이걸 공략해야하는 것이다}.
\vspace{5mm}

실제로 열심히만 공부해서 점수가 안 나오는 이유가 여기에 있다.
공부는 노동이지만, 시험은 도박이자 투기장에 가깝다.
노동을 열심히 한다고 그것이 무조건 고수익을 보장해 주는가?
갑이라는 애는 평균 10개 정도 틀리는 녀석이다. +-1개이다.
을이래는 애는 평균 5개 정도만 틀린다. 그런데 +-20개 - 즉 만점권 아니면 25개까지 틀리는 녀석이다.
\vspace{5mm}

계속 실패하는 n수생들은 '을'의 입장이 많다.
\vspace{5mm}

정리론의 핵심은 기대점수를 높이는 것이 아니다. \textbf{리스크를 '줄이는' 것}이 목표라는 점을 명심하자.
물론 이 글을 읽는 사람이 고2 이하라면 리스크는 신경 쓰지 않고 기대점수를 높이는 방향으로 양치기를 해야한다.
그러나 올해 시험을 앞두고 있다면 리스크를 줄이는 것만으로도 시간이 모자라다는 경고를 해야겠지만.
\vspace{5mm}

각설하고 강의든 교재든 현재 선택한다면 위험을 최대한 줄이는 방향으로 가야할 것이고
그렇다면 찬양되는 것보다는, \textbf{'해결될 수 있는 범위 내'의 문제점}이 지적된 것을 고르는 것이 사실은 현명한 것이다.
만약 투자자가 아니라 투기꾼, 즉 리스크를 신경쓰지 않고 질러서 고수익, 즉 고점수를 원한다면
검증되지 않은 정체불명의 상품 - 그것도 생각없는 사람들의 찬양성 댓글이 달린 것들로 가도 좋다.
하지만 망하면 그건 본인 책임임을 분명히 해둬야할 것이다.
\vspace{5mm}






\section{[정리론 009] 투자자 마인드 2}
\href{https://www.kockoc.com/Apoc/226264}{2015.08.01}

\vspace{5mm}

콕콕 수험생 A와 B가 똑같이 오답갯수 -5로 목표를 잡고 수험에 몰두햇다.
\vspace{5mm}

그리고 실제로 모의시험을 친 결과
A는 1개를 틀렸고 B는 4개를 틀렸다.
\vspace{5mm}

저번 글을 읽은 사람은 "또 B를 예찬하려고 그러지요"라고 그럴 것이다.
그리고 똘똘한 사람은 "얼마나 목표에 접근했느냐 하는 오차를 가지고 얘기하려 하지요"라고 선수를 칠 것이다.
\vspace{5mm}

자, 그럼 부연 설명을 하자. 만약 이게 본 시험이라면 승자는 무조건 A다.
어찌되었든 시험은 고득점을 받으면 되는 것이고 원하는 대학만 들어가면 장땡이기 때문이다.
하지만 잊지 말아야할 것은 점수라는 건 본인의 "실력"(평균)에다가 "운세"(편차)가 더 해진 결과란 사실이다.
\vspace{5mm}

그렇다면 모의평가로만 보자면 당연히 B 쪽이 더 안정적이다.
이 경우는 본인의 약점을 분명히 짚을 수 있기 때문에 본 시험에 필요한 '리스크 관리'가 가능하기 때문이다.
반면 이 글 보고 찔리는 사람들도 많겠지만 3평부터 뽕맞고 그 다음 헤롱헤롱한 사람이 어디 한두명이겠는가.
\vspace{5mm}

"만점을 받겠다"라는 포부와 "시험에서 4개 정도 틀리는 경향이 있다"라는 진술은 병치가능하다.
목표는 높을 수록 좋다, 하지만 현실은 냉정하게 지적해야 한다.
자기 현실에 냉정해질수록 본 시험에서 '운세' - 즉 변동이 긍정적으로 작용해서 더 많은 순이익을 얻을 수 있다.
실제 시험은 자기 실력대로만 나오는 것이 아니기 때문이다.
누구든지 시험에서 행운과 불운은 둘 다 경험한다. 운이 좋아서 5개를 더 많이 맞는다 치자, 다만 운이 나빠서 6개를 틀려서 -1개가 되는 것이다.
행운을 늘릴 수는 없다. 그건 평소에 대머리 아재에게 예의바르게 굴었는가 하는 선행이 좌우할지는 모르지만 이건 믿거나말거나.
\textbf{그러나 본인이 실수하지 않으려고 침착하게 집중하고 자기가 취약하거나 잘 낚인다고 하는 분야 준비를 한다면 실점을 막을 수는 있다.}
\vspace{5mm}

많은 수험생들이 킬러에'만' 긴장을 한다. 그리고 킬러에서 득점하고 자기가 취약한 비킬러에서 실점하는 경우가 많다.
학습론에서 강조하는 것이 가능하면 일찍, 많이 공부하고 양치기해서 스노우볼 효과를 노려 평균을 높이라는 것이면
정리론에서 강조하는 건 시험 당일의 '변동'을 가능하면 플러스로 유지하기 위해서 리스크 관리를 하라는 것이다.
지금부터는 쭉 정리해나가면서 실제 시험에서 자기가 털리거나 실수하는 '단원', '패턴' 등을 정리해보아야 한다.
그리고 잘못된 습관이나 사고패턴을 고치고 그 단원에서 긴장하고 주의하는 습관을 들임으로써 실점을 철저히 막아야하는 것이다.
\vspace{5mm}

앞에서 든 예시대로지만 본인이 고수가 되는 건 '만점 받을 거야'라고 부르짖는 게 아니라
자기가 문제를 푸면 정답율이 얼마일지 몇개를 틀릴지, 그리고 다 푸는 데 얼마나 소요하는지 개략적으로 \textbf{예측하고 말할 수 있는 것이다.}
\vspace{5mm}






\section{[정리론 010] 인강듣기 수준이 위험한 경우.}
\href{https://www.kockoc.com/Apoc/226313}{2015.08.01}

\vspace{5mm}

\href{https://www.youtube.com/watch?v=dD2ggMnjgbg}{인트로}


만약 '인강'을 듣지 않으면 공부가 아예 안 된다 수준이면.
이건 학습이 문제가 아니라, \textbf{"세뇌"된 것을 걱정해야하는 것}이지요.
\vspace{5mm}

그것이 알고싶다 같은 데 보면 교주나 무속인이 호화사치를 부리거나 섹스스캔들이 있어도
거기에 낚인 호구피해자들이 절대 벗어나지 못 하는 이유는 단순히 '속아넘어간' 것 이상이 아닙니다.
"세뇌되어서" 그런 것이죠.
\vspace{5mm}

세상을 보고 읽는 틀을 넘어 생각하는 방법부터 문제푸는 것까지 '강사'에게 의존하는 것을 지속하다보니
강사가 원하던 원하지 않던(원하는 경우도 없지 않다 보이는데) 그 강사는 학생에게는 예수님이나 부처님처럼 비치는 것입니다.
의식적으로는 그렇지 않다쳐도 무의식적으로는 그렇게 반복 자극을 받으면서 그 강사 목소리만 들어도 마음이 편안해지는 상태에 돌입하는 거죠.
\vspace{5mm}

그래서 실제로는 학습에 도움이 되느냐. 그건 실적을 봐야하겠지만 꽤 회의적인 면도 많습니다.
점수가 오른다고 쳐도 그게 '세뇌된 것'에만 이끌려나온 것이라면 그 점수가 오래 갈 리도 의문이고 본인 멘탈은 안전할 것인가.
똑같은 강의라고 쳐도, 위 cowboy bebop brain scratch 에피소드에 나오듯이 그게 TV 화면의 형태를 띠면 무시무시한 효과를 발휘합니다.
\vspace{5mm}

그래서 그나마 괜찮은 게 EBS라고 보는 겁니다. 왜냐고요? 강사 카리스마가 그나마 '덜' 발휘되니까요.
카리스마가 덜 발휘되고 교육 내용 전달에 집중하니까 재미는 없지만 이것이 결과적으로는 더욱 안전한 것이죠.
거기다가 '부분 수강'이 가능하기 때문에, 한 강사의 커리 끝까지 따라가서 거기에 물드는 것은 아무래도 덜 하다는 점도 있죠.
\vspace{5mm}

만약 이 글을 읽는 분들이 인강은 단지 보조도구일 뿐이야, 스스로 개념서 읽고 문제풀이 알아서 만들어 푼다 하면 별 걱정없겠습니다만
문제는 생각 외로 '인강'에 지나치게 의존하는 것을 넘어 금단현상까지 겪는 사람들입니다.
이 정도면 본인은 모르겠지만 스스로 사고하는 능력이 증발되어버리고 나중에는 인강을 안 들으면 아예 공부가 안 되고 마음도 휘청거립니다.
\vspace{5mm}

뜬금없지만 중세가 날라가고 근대가 시작되는 전환의 변화를 보면
\textbf{구어가 문어로 바뀌고 말씀이 서적으로 교체될 때}라는 데 주목할 필요가 있습니다.
인쇄술 발달로 책이 보급되었다는 건, 각 개인이 혼자 책을 차분히 읽고 독자적으로 생각하는 경우가 늘어났다는 것이죠.
그런데 요즘은 기술이 너무 발달하다보니까 다시 '말씀이 책'을 앞섭니다. 다시 신중세시대로 되돌아가는 것이 아닌가 하는 우려가 생길 정도지요.
\vspace{5mm}






\section{[정리론 011] 공부를 안 하는 거지 못 하는 건 아닌 듯요.}
\href{https://www.kockoc.com/Apoc/229108}{2015.08.03}

\vspace{5mm}

"공부를 정말 많이 했는데 성적이 안 오른다"
라는 건 아무리 보아도 '거짓말'이라는 확신이 듭니다.
\vspace{5mm}

온라인에서는 일지, 그리고 오프에서는 공부량 비교를 해보면서 느낀 건데
해당 과목을 못 하는 건 엄밀히 학습량 추적해보고 어떤 참고서 몇회독했느냐 따져보면
거의 예외없이 일관되는 경향이 있습니다.
\vspace{5mm}

\begin{itemize}
    
    \item  최소한 어느 과목이든 그 해당 내용에 대해서 기본과 패턴 5회독 이상은 분명히 되어있어야만 \textbf{비로소 판에 낄 수 있다}.
    \item  그런데 보통 어떤 책을 보았다라고 하는 경우, \textbf{딱 한번만 보거나 인강만 듣고 가필하고 끝나는 경우가 대부분이다}.
\end{itemize}
\vspace{5mm}

이런 패턴은 지겹게 확인되거니와
무엇보다 공부량에 대해서 각자 기준이 너무 다른데
평범한 학생이 교재 2권에 한 3번 돌리는 걸 기준으로 하면, 잘 하는 애들은 교재 8권에 4번은 돌린다... 라는 재미없는 사실이 너무 잘 드러납니다.
그게 꼭 교재 8권이라고 할 수는 없겠지만 대신 회독수가 높거나, 다른 강의를 많이 들어서 그런 효과를 낸다거나.
\vspace{5mm}

5등급 정도의 하위권이 자신감을 가지려면 최소 3회독을 해야합니다. 이래야 뭔가 '겨우 알 수 있다' 정도이고
5회독 이상은 가야지 그럭저럭 2$\sim$3등급 수준으로 풀 수 있고, 그 이후 해야하는 공부량은 더욱 기하급수적으로 늘어납니다.
어린 시절부터 사교육을 받거나 공부훈육을 잘 받은 경우는 틀이 잘 잡혀있어 효율적이지만
뒤늦게 공부를 시작하는 경우는 정말 20회독을 해야 99$\%$는 보장되지 않느냐... 라는 것은 아마 부정하기 힘든 것 같습니다.
\vspace{5mm}

냉정히 지적하면 올해 시험에서 원하지 않는 결과를 받아서 좌절하는 케이스도 많을 것인데
이 경우 다들 본인들이 프라이드 문제가 걸리고 아픈 상처 왜 후비느냐할지 모르지만
미래를 향해서는 분석 들어가야겠지만, 사실 대부분은 결국 원인이 분명히 나온다고 보입니다.
이 경우 만약 내년 시험을 준비하는 경우라면 시험 끝난 뒤 일주일만에 \textbf{다시 미친 듯이 시작해야할 거라고 생각해요}.
11월달부터 시작하느냐 마느냐하는 것의 차이가 사실 과장해 말하면 20배 차이는 난다고봅니다(...)
\vspace{5mm}

작년말부터 올초부터 쭉 정리해보지만, 사실 안 되는 분들은 교재 문제도 학원 문제도 아닙니다.
\begin{itemize}
    \item  첫쨰로 충고를 안 드는 경향이 있어요. 뭐 제 충고가 신뢰성이 없다면 모르겠습니다만 - 그런데 안 들어서 지금 후회하는 사람 계시죠?
    \item 둘째로 본인들이 조금이라도 고통이 오면 바로 포기하고 밖으로 도는 게 있습니다. 이거 공통적으로 확인된 케이스입니다.
    \item 셋째로 지나치게 공부에 있어서 타율적입니다. 적어도 제가 보는 명문대 합격자들은 '자율적'인 경향이 강한데 말이지요.
\end{itemize}
\vspace{5mm}

사실 제가 관여할 건 아니겠습니다만, 만약 내년에도 콕콕이 수험정보사이트로 명맥유지하고 있다면
허혁재님이나 기타 분들이 어떻게 스터디 조직 잘 꾸리시든지 감시망 조직 만들든지 해서
최소 30일간은 정말 특훈시키는 그런 것을 좀 가시는 게 낫지 않나 제안드리고 싶을 정도입니다.
지금 실패할 거라고 보이거나 그럴 가능성이 높은 분들은 사실 저런 스파르타식 특훈을 거쳐서 본인의 벽을 넘어섰으면 좋았을텐데 하는 생각.
이거 일지만으로는 한계가 있더군요.
일지를 읽어보거나 상담해보면서 느낀 건 정말 분신이라도 보내서(...) 사정없이 이주일이라도 후려잡고 싶다 생각도 들더라능.
\vspace{5mm}

아, 그리고 잊기 전에 적으면
아마도 학습일지나 콕콕일지도 결과는 '꾸준히 한 분이 잘 나올 거다'라는 상식적이지만 이게 꼭 이변으로 들리는 쪽에 걸겠습니다.
지금부터 11월까지 꾸준히 하는 분들이라면 - 이제 정리모드 잘 들어가서 단점 잡은 분이라면 좋은 성과 나올 거예요.
\vspace{5mm}






\section{[정리론 012] 채팅방의 콕콕충}
\href{https://www.kockoc.com/Apoc/229516}{2015.08.04}

\vspace{5mm}

게시글을 가장한 광고글도 올라오지 않는 K 모 사이트가 있었다.
어느 날 챗방에 콕콕충들이 번식하기 시작햇다
그 우두머리 H씨의 스캔들이 터지면서 H씨를 눈엣가시처럼 여기던 안티들이 이제 발 뻗고 누워잘 수 있다 등장한 것이다.
콕콕충들은 분열해서 1분만에 그 수가 2배씩 늘어났다.
10분이 지나자 최대 인원의 절반까지 번식했다.
이 채팅창 인원이 꽉 차서 방이 터질 때까지는 몇분이 걸릴까?
\vspace{5mm}

(가짜)정답 : 20분.
\vspace{5mm}

\newpage

페이크다 병신들아 진짜 정답은 11분이지.
\vspace{5mm}

이것도 못 맞추냐.
\vspace{5mm}

햄스터로 위를 긁어본 다음에 아랫 글을 읽어보자.
\vspace{5mm}

지금 공부하는 사람들에게는 통수일 수 있으나 작년 말부터 소감을 보면 그렇다.
성적이 안 나온다 하더라도 본인의 원래 위치보다 나아졌다 보는 케이스는 11월달부터 어떻게든 공부한 케이스고
그거 안 하고 2, 3월부터 공부한 사람들인 내가 보기에는 실력 향상이 그리 크지 않다.
나는 충고만 했을 뿐인데 왜 5월달에 공부가 끝나냐고 이의제기한 사람들이 있을 것인데, 이제 굳이 설명할 필요가 있을까.
\vspace{5mm}

내가 원하는 건 장사가 아니라 '최적의 시스템' 찾기. 물론 이 최적의 시스템은 장사와는 거리가 멀다.
하지만 최적의 시스템을 찾는다면 이건 수험말고 다른 데에서도 응용할 수 있기 때문에 이 검증은 매우 중요하다.
왜 5월 이후의 공부는 극단적으로는 무의미하다고 보는 걸까.
\vspace{5mm}

그건 복리효과 때문이다.
\vspace{5mm}

학습능력을 X라고 치고, 시간을 하루단위로 T 라고 하자. 그럼 실력은 $X^T$의 함수대로 따라가게 되어있다.
72 법칙에 따르면 (공비)($\%$) X 시간단윗수 = 72 : 원금이 2배가 되는 데 걸리는 시간을 근사적으로 계산하는 함수다.
만약 원금이 100만원이고 이자가 연 10$\%$ 붙는다고 하면 200만원이 되기까지는 7년이 걸리는 것이다.
이건 공부도 마찬가지이다.
하루에 12시간 공부(가 가능한지도 의문이지만)해서 30일하는 경우 $12^30$과
하루에 6시간 공부해서 90일 하는 경우인 $6^90 $중 어느 쪽이 큰지는 암산으로 충분히 가능할 것이다(이거 못 하면 수1 공부 안 한 거지 뭘)
\vspace{5mm}

이제 100일차이다.
성적이 어떻든 간에 꾸준히 공부해 온 사람들은 이 100일동안 엄청난 복리효과를 누릴 수 있다.
학습양을 줄이고 정리에 치중하더라도, 이제 막 14시간 미만잡거리는 학생들보다 더 많은 학습효과를 누리는 것이다.
그러나 어떤 사정 하에서든 여태까지 꾸준히 공부하지 않은 학생들이 100일동안 누릴 수 있는 건, 저 꾸준파에 한참 못 미친다.
합격에 필요한 공부량의 절반을 달성한 사람이면 50일 안에 목표에 도달하겠지만
이제야 출발한 사람이면 100일 안에 채울 수 있을 가능성은 냉정히 말하면 '낮다'.
\vspace{5mm}

그래서 작년 말에 무조건 일찍 시작하라고 한 것이다. 뭘 하든 일찍 시작하고 처음부터 양치기를 하는 것이 이 점에서 좋은 것이다.
그런데 수험 커뮤니티를 돌아보거나 게시물을 보면 이제야 양치기를 하는 경우가 있는데, 이건 복리효과를 모르는 것이 아닌가.
아니 수1을 공부하면서 지수함수나 등비수열의 합을 자기 학습에 응용하지 못 한다는 이야기이다.
\vspace{5mm}

날카로운 사람은 이런 질문을 던질 것이다. 스트레스는 어떡하나요.
\vspace{5mm}

이것도 참 빈익빈부익부다.
\vspace{5mm}

연초부터 졸라 빨리 시작한 사람들은 성적 향상을 맛보면서 스트레스가 감소된다. 공부로써 스트레스를 푸는 것이다.
그런데 늦게 시작한 사람들은 100일에 들어서 겨우 성적이 오르거나, 오르지 않으면 스트레스가 가중된다(...)
여기서도 또 격차가 벌어져버린다.
\vspace{5mm}

작년 말에 글 쓰면서 이러면 안 될 터인데 느끼는 것.
"내가 쓰는 상식적인 충고를 본인이 꼭 '경험'해보아야만 납득한다면 매우 아쉬운 일일텐데"
\vspace{5mm}

우선은 복리효과를 노리기 위해서라도 수능까지는 달리시길 바란다. 절대 결과는 생각하지 마시고.
그러나 플랜 B 정도는 생각해보시길 바란다. 만약 현재 하는 공부를 내년까지 이을 수 있다면 그럼 엄청난 복리를 누릴 수 있기 때문이다.
아니 지금 와서 무슨 희망고문하느냐라고 하지만 이 글쓰는 사람은 말 바꾼 적 없습니다.
분명 5월까지가 좌우한다고 얘기했으므로
\vspace{5mm}

+ 개인이 자신의 학습 복리 시스템을 만드는 건 입시와 상관없이 매우 중요한 일,
그 학습 습관과 환경을 포괄한 시스템을 확립하는 게 대학 입학 이후에도 유용하게 작용하기 때문이다.
\vspace{5mm}

+ 가끔 시중교재 추천하는 것 가지고 시비(?)거는 사람들 보고 느끼는데 내 입장에서는 이 사람들은 공부 제대로 한 사람들이 아니다.
회독수 높이고 복리효과 누리다보면 \textbf{교재가 문제가 아니라}는 걸 깨달을 것이고
어떤 교재든 그거 10회독, 20회독 이상하다보면 처음에는 보이지 않던 행간의 의미가 보이기 시작한다.
야매교재는 정말 이런 게 없다, 교과서나 시중교재는 가능하지만.
\vspace{5mm}

+ 어차피 입시는 붙는 게 장땡이지만, 요행으로 합격해서 잘 먹히는 건 님들 부모님 세대들까지고(학점 신경 안 써도 기업에서 취업해달라 사정)
대학 4년이 입시보다 더 징하다죠.
\vspace{5mm}






\section{[정리론 013] 미래를 위한 선택인데 어째서 현재가 기준인가?}
\href{https://www.kockoc.com/Apoc/236333}{2015.08.08}

\vspace{5mm}

"그 아파트 많이 올랐구나. 당장 사야지"
"이 주식 어제도 상한가고 오늘도 상한가네. 질러볼까"
\vspace{5mm}

당연히 저렇게만 생각하면 어리석은 선택이다.
투자는 미래를 보고 하는 것이지 현재를 보고하는 것이 아니다.
그 대상이 부동산이든 주식이든 클라우드펀딩이든 하다 못해 사람이어도 마찬가지이다.
\vspace{5mm}

의치한에 가야한다면 미래에 의치한이 어떨까 그것 정도는 당연히 시나리오는 쓰고 가야한다.
하다 못해 철학과를 간다고 하더라도 자기가 철학과에서 배운 것으로 $\sim$ 할 수 있다라는 확신이 있으면 비웃음당할 게 아니다.
오히려 비웃음당해야하는 건 \textbf{"남들이 좋다고 하니까"} 선택하는 것이다.
요즘은 9급 공무원이 좋다고 한다고 하니까 또 이것만 따라가는 풍조도 있지만, 이 역시 남들의 카더라, 그리고 "현재 상태"만 본 것이다.
\vspace{5mm}

투자는 한번 선택하고 나면 '손해'를 보지 않고서는 그걸 다시 무를 수가 없다.
그래서 \textbf{선택을 신중히 해야한다.}
대신 선택을 제대로 하고 나면 그 다음부터는 아무 노력을 기울이지 않고 딴짓을 하더라도 return이 돌아온다.
내가 투자한 시간이나 돈이 마치 대신 일해주면서 수익을 뽑아주는 것이다.
가령 서울대에 가면 얻는다는 서울대 학벌도 서울대 졸업한 선배나 동기후배들이 대신 일해주는 걸 조금씩 받아먹는 것과 같다.
\vspace{5mm}

물론 수험생에게 20년 뒤 미래까지 내다보고 판단하라는 것은 무리일 수 있다.
그러나 이게 입시공부보다 어렵냐하면 그것은 아니다.
거시적인 움직임 정도는 어느 정도 예측되기 때문이다.
가령 의치한이 무조건 좋은가. 뉴스 검색을 해보거나 수익이 어떻다는 찌라시 글이라도 찾아 읽거나
특히 앞으로의 인구구조 변화를 찾아보면 되는 일이다.
\vspace{5mm}

대한민국 직업사를 살피면 꿀빠는 직업군은 정작 입시 때에는 인기가 적었다.
의느님들 잘 나가신다... 하시는데 그 40대 이상 때는 의치한 열풍이라는 게 없었다.
IMF 이후 살인적인 경쟁률 시절에 들어간 사람들이 그 선배만큼이나 꿀빨 수 있을 것인지에 대해선 당연 회의적.
공대가 좋다 소리가 다시 나오지만 현재 공대 나와 취업시장에서 상대적으로 꿀빠는 세대는 상대적인 입결 하락으로 득본 사람들이다.
사법시험은 어떨까. 인원수 줄여 헬게이트 된 시험부터 합격한 수재들이 받는 대우는 ?
\vspace{5mm}

즉, 입결이 실제로 그 분야가 잘 나간다라는 걸 보장해주지 못 한다는 이야기이다.
\vspace{5mm}

정말 잘 나가느냐를 좌우하는 것은 "수급' - 수요와 공급 뿐이다.
그 점에서 선택을 한다면
\begin{itemize}
    \item  미래에 업그레이드 되거나 다시 수요가 발생하리라고 보는 분야
    \item  다른 사람들이 외면하거나 신경쓰지 않아 저평가된 분야
\end{itemize}
이걸 고르는 게 탁월한 선택일 것이다.
\vspace{5mm}

난 열심히 공부했고 국영수 잘하니까 대학 졸업하면 나라가 일자리 보장해줘야해...
라는 게 흔한 착각이겠지만 현실은 그딴 것은 없다는 데 유의하시길 바랍니다.
\vspace{5mm}

100$\%$ 확실한 건 현재 꿀빤다고 하는 것 반드시 또 내려갑니다.
대한민국은 그런 적 없다 하는데 그거야 당연하죠. 그동안은 정말 \textbf{고성장}이었으니까.
이제는 사람들이 힘들어도 말도 못 하고 심지어 정권 탓이다라고 시위도 안 하죠. 이런 게 "저성장"인 겁니다.
과거에야 싸워서 이기면 그래도 더 나눠먹을 건 있겠구나 했는데 이제는 그런 것 조차 없습니다.
\vspace{5mm}






\section{[정리론 014] 문제집에 오답을 남기기 싫어하는 이상한 착각}
\href{https://www.kockoc.com/Apoc/237400}{2015.08.08}

\vspace{5mm}

흔히 공부 잘 했다는 학생들의 습성.
\vspace{5mm}
\begin{enumerate}
    \item  중학교 때는 잘 나갔어요.
    \item  전교에서 놀았던 적도 있어요.
    \item  지금은 시궁창이예요.
\end{enumerate}
\vspace{5mm}

과거의 좋았던 시절에 집착하는 건 망조의 실마리다.
중학교 때 잘 한 케이스가 사실 가장 위험하다.
중학교 공부와 고등학교 공부는 다른데도 이걸 생각하지 못 하고
잘나갔던(?) 중학교 시절의 중세시절 공부법만 고민하다보니
기관총이 등장하는 고등학교 교과목에 맞는 학습으로 진화하지 못하는 것이다.
\vspace{5mm}

한가지만 예를 들자면 - 아니 이 글의 핵심인데
중학교 공부는 '오답'이 정말 해악으로 간주된다.
그래서 문제집을 빨리 풀고 무조건 다 맞고 그 다음 부모나 선생님에게 칭찬받는다.
라는 패턴을 고수하려 한다.
\vspace{5mm}

중학교 공부의 주인공이 정답이라면, \textbf{고등학교 공부의 주인공은 오답이다.}
중학교 마인드에 쩐 사람들은 100개에서 99개는 맞아야 공부가 되는 거라고 착각한다.
99개를 맞을 실력이 될 때까지 인강 듣고 기본서만 읽으면서 기다려야 한다고 착각. 그렇게 귀중한 시간을 낭비한다.
하지만 고등학교 학습은 오히려 '오답'이 나야한다.
\textbf{오답을 탐구해야만 근본적인 실력이 늘기 때문이다}.
\vspace{5mm}

중학교 수험은 대평원에서 떼거지 저글링 러쉬만 보내면 되는 유치한 빠른무한 스타크 유즈맵 게임과 같다.
그에 비해 고등학교 수험은 '롤'에 가깝다(그렇다고 롤 한다고 성적은 오르지 않으니 착각하지 말자)
어떤 과목이든 출제자가 뭘 겨냥하는지 정확히 읽고, 그에 부응한 풀이과정대로 정밀한 답을 도출해야 한다.
\vspace{5mm}

이걸 모르다보니까 문제집을 풀면 점수가 안 나오네 평소에 모의는 잘 나오는데 실모는 낮네 그러고 있지만.
중요한 건 본 시험의 점수이지, 자기가 공부하는 참고서나 모의 점수가 아니다.
물론 그 점수가 낮은 것은 위험하다. 그러나 점수가 안 나온다면 그 근본원인을 찾기 위해 오답정리를 해야지
전혀 소득이라고는 없는 타인과의 비교나 신세한탄이나 하는 케이스가 많으니 문제다.
\vspace{5mm}

오답을 내기 싫어하는 결벽증의 극단적 사례는 레알 수능 직전까지 유명강사 인강만 듣는 케이스다.
강의 다 듣고 해야만 문제집을 풀면 오답이 없을 거야라는 전혀 근거없는 확신만 가지고 밀어붙인다.
그런데 정작 문풀에는 도움이 되지 않지, 실제 시험에서는 발려버리지.
그런데 더 심각한 건 이런 자기 공부법이 전혀 문제가 없다고 생각한다(사실 누가 자기 오류를 인정하고 싶겠는가)
그래서 또 "\textbf{n수험적 귀납법}" 인생을 살면서 n=k+1 과정에서 또 다른 인강을 찾고 있거나 어디 전설의 교재가 없나 그런다.
시간 낭비는 당연한 것이고 더욱 심각한 건 스스로 공부하는 방법마저 상실해버린다라는 것이다.
\vspace{5mm}

10개에서 9개를 틀리건 100개에서 99개를 틀리건. 그게 본 시험이 아니라면 신경쓸 바는 아니다.
단지 오답정리를 제대로 했느냐, 그리고 그 틀린 문제와 비슷한 문제를 풀어봄으로써 자기 오류를 정정하느냐가 문제다.
수험생들이 xx 책을 보았다라는 것은 절대 믿으면 안 되는 게, 그거 1번만 보고 심지어 오답정리도 안 하고 보았다 하는 게 대부분이어서이다.
저 xxxx 보았는데요... 그게 뭔 소용이 있나. 오답 정리 분명히 하고 맞은 것도 재확인하고 하는 식으로 회독수를 오진수 두자리는 만들어야지.
그 정도까지 하다보면 교재가 문제가 아니라는 것을 확인해야 할텐데.
오답 정리를 안 하면서 인강을 들으면 뭔가 그럴싸하게 느껴지긴 한다. 그런데 이것도 치명적인 단점.
강사가 정리해주는 건 - 특히 뛰어난 강사일수록 자기 수강생의 평균치에 맞는 처방을 해준다.
그럼 그 평균치 수험생과 나의 편차는 해결될 수 있을 건가?
\vspace{5mm}

몇점이 나오니 몇개가 틀리니 하지 말고 오답정리나 꾸준히 하시기들 바랍니다.
\vspace{5mm}






\section{[정리론 015] 실모를 어떻게 활용할 것인가.}
\href{https://www.kockoc.com/Apoc/242648}{2015.08.11}

\vspace{5mm}

아마 제작자들이야 실모가 신나게 팔려서 - 말로는 수험생을 위한다고 하면서 -
\textbf{본심은 돈벼락맞아 죽고 싶다}가 될 것이다.
(정말 돈생각을 안 했다면 그냥 무료공개했겠지)
\vspace{5mm}

자본주의 사회에서 돈생각하는 게 나쁜 것은 아니다. 문제는 품질이 가격이 비례하냐는 것.
이렇게 이성적으로 접근하면 사실 실모를 '선택적으로 본다면' 모를까, 반드시 봐야한다는 어떤 이유도 \textbf{없다}.
수년간 그런 실모를 통해서 극적인 점수향상을 했다던가, 기출문제 적중을 했다던가하는 것도 역시 검증된 바 없다.
(혹자 비슷한 문항을 이야기한다하면, 그런 수준이면 시중교재는 100$\%$ 적중이 된다)
\vspace{5mm}

그럼에도 불구하고 실모를 굳이 보아야한다면 다음과 같은 걸 유념하자
\vspace{5mm}
\begin{enumerate}
    \item 브랜드는 그냥 무시해버리는 게 좋다.
    \vspace{5mm}
    
    소위 브랜드라고 하는 것은 정말 그것이 검증된 경우,
    그리고 그 업자가 그 브랜드의 명예를 지키기 위해 개인 이익까지 포기한 경우를 말한다.
    하지만 내가 아는 한 참고서에 그런 경우는 별로 없다.
    하다 못해 브랜드라고 하려면 10년 넘게 시장에 버티고 있어야하는 게 아닌가.
    여기도 일격필살과 무관치 않은 곳이라고 하지만, 일격필살이 좋다 안 좋다 그런 건 그냥 흘려들어도 된다는 이야기다.
    \vspace{5mm}
    
    \item  가격 / 문제숫자로 일단 계산해보자.
    \vspace{5mm}
    
    이번에 일격을 검증(?)해보기로 마음먹은 게 그나마 이 점이 높다고 보여서이다.
    21600원하니까 으악 졸라 비싸네 그러는데 12회분이라고 하면 21600원/360문항=60원/문항이므로 오히려 싼 것이다.
    다만 그 12회분이 그냥 대충 질소문제인가 하는 건 더 확인해봐야하겠지만.
    생각해보면 다른 실모들도 그런 점이 검증되었는가 하면 그냥 '숫자상'으로 본다면 이 접근은 그래도 합리성을 갖추고 있다.
    단 이렇게 보자면 그렇게 욕처먹던 쎈, RPM, EBS가 얼마나 혜자스러운지 느낄 수 있을 것이다.
    \vspace{5mm}
    
    \item  막판 정리용이 아니라 막판 정리를 위한 촉매용이라고 보자.
    \vspace{5mm}
    
    가장 바보같은 사람들이 이 점수가 90이 안 나왔니 만점이 안 나왔니하는 건데.
    고교수학은 원래 오답을 통해 실력을 키우는 것이지, 중학교 때처럼 100개 풀고 100개 다 맞는 그런 것이 아님을 다시 한번 확인하자.
    실모 문제는 어찌되었든 평균적이지 않기 때문에 시중 출제경향과 편차가 있을 수 밖에 없다.
    그것이 실전에서 도움이 될지 아니면 방해가 될지는 그걸 구입한 사람의 도박이겠지만
    어찌되었든 중요한 건 문제를 맞추는 게 아니라, 틀릴 수 있으면 최대한 틀린 뒤 오답정리 제대로 하고
    자기가 어디가 부족한지 재차 확인한 후, 교과서-시중기출-시중교재로 열심히 보강을 하는 것이다.
    막판정리라고 하는 건 결국 자기가 부족한 것을 베꾸는 것이지 한권으로 완성한다.. 가 아님을 유의하자.
    \vspace{5mm}
    
    \item  실모를 마구 사두는 건 어리석은 짓이다.
    \vspace{5mm}
    
    하라는 공부는 안 하고 풀지도 않을 참고서를 사두는 케이스, 찔리는 사람들 당연히 많을 것이다.
    실모는 하나 정도, 그게 모자라면 2개 정도면 족할 것이고 그걸 풀 시간이 있으면
    교과서 기출 그리고 실력정석 같은 것이나 수리논술 문제를 풀어보는 게 나을 것이다.
    그런데 현실은 실모 사두고 결국 못 풀거나, 다 풀었는데 정작 효과가 없는 케이스가 사실 많다.
    실모는 어디까지나 "새로운 관점에서 자기 실력을 평가한다" 그 수준 정도일 뿐이라는 점을 유의하자.
    다년간 교정, 교열되고 검증을 거쳐 현직 전문가급 분들이 만든 시중교재를 무시하고
    그저 남들이 푼다니까 무조건 실모사서 푼다는 것은 아무리보아도 바보같지 않은가?
    \vspace{5mm}
\end{enumerate}

다시 강조하지만 이 시기에 양 무조건 늘리는 건 이제 바보같은 짓이다(늘릴 때는 안 늘리고 무슨 청개구리도 아니고)
이 시기에는 잘 정리된 오답문제가 대충 푼 100문제보다 낫다.
특히 학생들이 모르는 건
대충 풀어서 맞은 문제가 시험 때 통수치지만
틀려도 오답정리 잘 되어서 달달 외우다시피 하는 문제는 반드시 보답한다는 것.
그럼에도 불구하고 이런 과정을 게을리하면서 xx만 들으면 돼, xx만 풀면 돼 라고 하는 사람들이 많다.
\vspace{5mm}

+ 뭘로 레이드갈까 하다가 일격을 구매했다.
그런데 망할 인터넷 서점이 배송을 뒤늦게(...). 그 4만원이면 금발로리안경녀가 나오는 쪽국화집을 사볼 수 있었을텐데.
12회분이면 이걸로 충분할 것 같다.
혹자는 너 돈받아먹었냐 광고해주느냐 할지 모르겠는데 그 딴 것은 없고 풀면서 문제되거나 해설이 불친절한 건 매의 눈으로 지적할 궁리
소비자의 권리 행사하는데 너 고소미 혹은 2D 안경녀 그림 찢는 만행이 없기만을 바랄 뿐.
\vspace{5mm}

    
    
    
    





\section{[정리론 016] 정리시점에서 듣는 인강이라면}
\href{https://www.kockoc.com/Apoc/248225}{2015.08.13}

\vspace{5mm}

공부를 초기에 시작했다면 '다수가 듣는 메이저 인강'을 권하겠죠.
왜냐면 메이저는 '평균'이기 때문에 안심하고 탑승할 수 있고
평균을 따라간다고 하는 생각 때문에 자기가 혹시 잘못된 길로 가고 있지 않나 하는 불안감이 사라집니다.
\vspace{5mm}

그러나 정리시점이 되면 달라지죠.
\vspace{5mm}

\textbf{이 때에는 '평균'이 실패이기 때문입니다.}
그렇기 때문에 일타강사를 듣는다가 현명한 전략이냐.... 하면 아니라고 생각합니다.
강의 자체는 만족스러울지도 있죠. '평균치'로서는 말입니다.
그리고 이게 매년마다 왜 그런 메이저 강사를 따라간 사람들이 별로 이득이 없느냐를 설명해주는 원리입니다.
\vspace{5mm}

막판에 인강을 듣는다면
\begin{itemize}
    \item \textbf{첫째, 저평가되어있을 것.}
    \item \textbf{둘째, 광고나 자본의 도움이 없는데도 성장하고 있는  케이스임.}
    \item \textbf{셋째, 듣는 사람이 적을 것.}
\end{itemize}
\vspace{5mm}

이걸 고르는 게 낫죠.
저평가된 것 : 설명이 필요?
광고나 자본의 도움이 없는데 성장한다 → 이거야말로 성장할 수 밖에 없는 이유가 있다는 얘기지요.
그리고 듣는 사람이 적을 것 : 그래야 평균에 묻어가는 함정을 피할 수 있습니다.
\vspace{5mm}

따라서 현명한 수험생이 인강주의 전략을 폈다면
수험초기에는 메이저 인강을 빨리 들어 끝냈을 것이고
시험 100일차에 와서는 아마 메이저 인강을 탈피해서 마이너하지만 저평가에 성장하는 케이스를 고르겠죠.
이 경우 일타를 고른다는 건 결국 남들과 차별성을 드러내지 못 하기 때문입니다.
\vspace{5mm}

"나는 평균도 안 되니까요"한다면 물론 이런 전략은 피해야겠지만. 사실 이 시점에서 평균도 안 되면 결과야(...)
\vspace{5mm}

그러나 수험생도 군중심리를 이기긴 힘들고
더군다나 결과는 알 바 없지만 일단 불안하지 않은 심리, 그러니까 남들도 다 듣는 것 가자라고 가기 때문에
이 정리론이 먹힐 일은 별로 없을 것입니다.
\vspace{5mm}






\section{[정리론 017] 수험을 대단한 걸로 생각하면 망한다.}
\href{https://www.kockoc.com/Apoc/250805}{2015.08.13}

\vspace{5mm}

관찰해보면 실제로 답답한 케이스 중 하나가 무슨 수험을 '예술'로 아는 사람들이 있다는 것이다.
\vspace{5mm}

"난 반드시 xxx 인강 풀커리 소화해야해"
"필기노트 예쁘게 만들고 말겠어"
"열심히 해서 실모 내서 돈 많이 벌거야"
\vspace{5mm}

다 들어보면 그럴싸한데 그거. 시험장에서 문제 하나라도 더 틀리면 \textbf{끝이다}.
반면 인강을 대충 듣던, 필기를 지저분하게 하든, 돈생각없는 수험생이라도 한문제라도 맞추면 \textbf{성공한} 것이다.
\vspace{5mm}

사교육을 넘어서 수험산업, 아니 수험파생상품들까지 넘쳐나는 미쳐돌아가는 판이다보니까 온갖 사이비 에술 과대망상까지 판친다.
특정 강사 인강을 듣던 교재를 풀건 그건 "본 시험에서 점수를 못 올리면" 아무 소용이 없는 것이다.
\vspace{5mm}

수험이 대단한 것이라고 보나?
수능시험은 일단 인격 검증은 거의 못 한다. 성격 더럽거나 치졸한 변태에다가 싸이코패스도 대학에 보내는 게 수능이다.
아니, 수능시험이 지적능력을 제대로 평가한다고 보나? 그런 점도 없지 않지만 잘 보면 구멍이 송송 뚫려있다.
\vspace{5mm}

극단적으로 말해서 제대로 된 공부를 하지 않더라도 \textbf{점수만 잘 맞으면 되는 게 수험}이다.
착실하게 공부하고 진리를 깨달아야죠... 라는 절규가 나오면 그 친구는 참 멍청하다고 한소리하겠다.
야, 그럼 교과서 보고 양치기 하라는 건 왜 그러는데? 그거야 다행히도(?) 이 정도는 해야 점수가 나오니까 그런 것이지.
수험의 본질은 극단적으로 말해 범죄행위에 해당하지만 않으면 점수만 잘 나오면 되는 것인 거다.
꿈에서 조상님이 가르쳐준 번호만 체크해서 만점이 나오든, 본인이 샤프 굴리기를 잘 해서 만점이 나오든 그건 어찌되었든 성공한 것이다.
\vspace{5mm}

이 이야기를 하는 이유는 간단하다.
목적보다도 수단을 더 중요시하는 웃지 못 할 사례가 너무나도 많기 때문이다.
가령 인강을 듣다가 시간이 없으면 과감히 끊어버리고 문풀로 가야하는 데도 끝까지 들으면 나오겠지하는 케이스.
그냥 필요한 문제만 풀고 나머지 문제는 안 풀어도 좋은데도 한권 샀으면 다 뽕뽑아야지 하며 시간낭비하는 케이스.
오답 정리만 하면 점수가 오를 수 있는데도 다른 친구들이 보는 교재나 강의가 끌려서 거기에 낚이는 케이스.
\vspace{5mm}

독학재수가 실패하는 결정적 요인 중 하나도 이것이다.
차라리 학원에 다니는 경우는 다른 친구들의 평균적인 데 묻어가기 때문에 자신의 세계에 틀어박히지는 않는다.
그러나 독학의 경우 처음부터 완벽히 짰다고 믿는 자신만의 과대망상에 빠져서 나중에는 자기 방법이 맞다고 고집해버리며 폐인이 된다.
그냥 수험은 점수만 잘 맞으면 되는 일종의 도박이자 요식행위라는 것을 모른 채
수험의 의미를 과대평가하고 자신의 자존심을 채우려하다가 스스로 사이비 종교 교주 겸 호구의 자웅동체 무간도에 휘말리는 것이랄까.
\vspace{5mm}

본질 하면 무슨 아름답고 숭고하며 어쩌구... 하는 경우가 있는데 그건 모두 개소리다.
예컨대 음식의 본질은 자연과 태양이 빚은 요리사의 혼... 은 오글거리는 헛소리다. 건강에 좋고 맛있으면 장땡인 것이다.
만약 돈을 번다고 한다면 어떤 일을 하느냐 관게없이 - 똥을 치우든 남에게 욕을 먹더라도 그냥 '많이 벌면' 되는 것이다.
남에게 인정받고 싶다고 하면 무슨 존스홉킨스니 판검사니... 이전에 적당히 돈벌고 욕먹지 않은 일 하면서 선행하고 베풀면 되는 것이다.
\vspace{5mm}

생각해보면 그냥 간단하고 상식적인 것들인데 의미과잉 때문에 이성을 잃고 헤메는 경우가 많다.
수험에서 성공하는 것도 "내가 더 점수를 잘 받으려면 어떡하지"라는 것을 실전적으로 고민하면 된다.
실패하는 요인들을 하나하나 나열해보고, 그것들을 더 요약, 압축해보면서 실험하다보면 자기 문제점을 발견하는 건 그리 어렵지 않다.
그 문제점들을 조금씩 개선해나가면 된다.
\vspace{5mm}

그런데 문제점을 개선한다는 것에 대해서도 참 많은 사람들이 결정적으로 착각하는 게 있다.
만화나 영화의 악영향이랄까. 그런데에서는 문제점이 발견되면 '빨리 해결되는 것'처럼 보인다. 그거야 그 부분이 생략되어서 그렇지.
'개선'이라함은 조금씩 이뤄지는 것이다. 조금씩 이뤄진다는 건 이게 결국 '오래' 걸린다는 이야기이다.
오래 걸리기 때문에 더 불안해지고 또한 고통이 따르는 건 당연하다. 그런데 다들 이걸 모르고 왜 빨리 안 고쳐지지 초조해한다.
단기간에 개선되는 건 아무 것도 없다. 3일 걸릴 거라고 생각하면 사실은 30일은 각오해야 한다.
그 30일동안 인내하면서 고쳐나가야 성과를 보는 것이고, 그렇지 못 하면 계속 같은 바보짓을 하게 된다.
반성(反省)이란 말은 바꿔 말해서 반복(反復)을 생략(省略)하는 것이라는 말이 있다.
입으로만 반성 어쩌구 하는 게 아니라, 자기가 하는 잘못된 습관이나 실수를 다시는 반복하지 않도록 생략하는 게 반성이란 얘기다.
\vspace{5mm}

10분 정도만 생각하면 자기 수험생활에 걷힌 거품이 보일 것이다.
그걸 과감히 걷어내야하는데 쉽지 않다. 데이트에 나간 여자들이 화장을 벗겨내고 민낯을 보이는 것과 비슷하다.
즉, 수험의 거품은 자아도취를 유발하는 치명적인 아름다움(?)이나 비장미(?) 같은 게 있다.
그런 만큼 눈감고 그런 걸 걷어내야 한다.
\vspace{5mm}







\section{[정리론 019] 자기불신이 중요하다.}
\href{https://www.kockoc.com/Apoc/259767}{2015.08.18}

\vspace{5mm}

공교육과 사교육의 구분만큼 사실 무의미한 것도 없다.
사교육계 쪽에서 공교육을 살리자라는 매우 위선적인 이야기를 하지만 피식 웃을 수 밖에.
그들이 원하는 공교육은 지금과 같이 물반고기반으로 먹을 것을 많이 던져주는 막장 공교육이다.
\vspace{5mm}

교육을 구분하려면 자율 교육과 타율 교육으로 나눠야한다.
쉽게 말해서 독학이냐 타학이냐로 부르면 될 것이다.
그런데 독학 vs 타학이라고 하니까 마치 이게 대등한 것처럼 착각하는데
실제로 독학이 성공하는 비율은 1$\%$도 되기 어렵다.
\vspace{5mm}

후까시를 잡으면서 이런 질문부터 던지자
"이 글을 읽는 사람은 정말 자유의지를 갖추고 있느냐"
뭔 더운 날에 뜨거운 캔커피 마시며 하악대는 개소리냐고 하면 쉽게 말해보자
\textbf{"너, 혼자 공부하라고 하면 정말 딴 짓 안 하고 할 수 있니?"}
\vspace{5mm}

실제로 독학할 수 있는 사람은 정말 없다. 까놓고 말하면 내 입장에서는 - 그 정도로 애새끼들이 배가 부르다.
왜 갑자기 막말하느냐라고 하는데
연초에 열심히 공부해서 세상에 복수할 거야 만점 받을 거야 하는 애들이
여기 콕콕 회원들도 예외가 아니지만 한달도 안 되어서 초심이고 나발이고 다 버린다.
\vspace{5mm}

그러나 이 글을 쓰는 아재새끼도 배부른 놈인 것은 확실하다. \textbf{나 역시 지키지 못 한 약속이나 실천하지 못 한 게 훨씬 많다}.
\vspace{5mm}

그럼 이 얘기를 하는 이유는 "현실을 인정하자"는 것이다.
독학할 수 없으면 뭐하러 독학을 고집하나, 타학으로 가는 게 현실적이자.
아니, 독학과 타학을 적당히 섞을 수도 있어서 자기에게 좋은 배합을 찾을 수 있는 게 아닌가.
\vspace{5mm}

기억하시겠지만 이런 의미에서 내 경우는 오프라인 학원을 권한다. 강의 질이고 그 이전에 '독학'의 해악은 막아주기 때문이다.
그리고 집에서 공부가 안 되면 독서실이나 도서관에 가면 좋은 것이 그것이다. 공부 분위기를 어느 정도 강제해주기 때문이다.
공부를 할 수 없으면 \textbf{공부하는 상태로 끌려가면 되고}, 공부하기 싫으면 \textbf{'지식노동'을 하면 된다}.
성적이 높은 애들이 난 공부 안 하는데... 라는 건 이 점에서 진실일 수도 있다.
그들은 공부를 하는 게 아니라 부모님 등에게 끌려가 지식노동형을 수행 중이기 때문이다.
\vspace{5mm}

공부를 자기 의지대로 할 수 있는 사람이 정말 몇이나 있을까.
이게 소위 '수험론' 같은 게 안 먹히는 이유일 것이다. 수험론들은 개인 의지를 강조한다.
그러나 적어도 내가 보는 한, \textbf{수험생의 의지는 정부의 약속만큼이나 믿을 수 없다.}
반면 성적이 잘 나오는 친구들에게 의지? 그런 건 없다. 다만 그들은 지식노동을 하는 고통을 못 느낀지 오래다.
\vspace{5mm}

그럼 수험의 시작은 \textbf{"자기불신"}이다.
\textbf{즉 "나란 놈년부터 믿을 수 없다"라고 하면서 나를 훈육하고 강제노동시키는 것 밖에 없다.}
우선 이것부터 지켜진 다음에야 수험사이트 허세질이라도 하는 것 아니겠나.
\vspace{5mm}

+
\vspace{5mm}

그리고 역시 우려한대로 이 시기에 새로운 교재를 마구 사들이는 풍조가 보이는데
\textbf{이거 99$\%$ 망한다.}
지금 봐아햐는 교재들은 연초부터 돌린 그것이어야한다. 사실 그거라도 제대로 보고나서 실모를 본다고 하는 건가.
정말 양심적인 업자들이면 '스케줄'에 맞게 소화할 수 있게 일찍 책을 내줘야한다.
그런데 지금 책을 '모의고사' 형식으로 낸다고 하지만 이게 정말 효용이 있을까.
\vspace{5mm}

여기서 눈여겨볼 것은 저런 쇼핑을 하는 것 자체가 "내가 공부한다", "나의 의지를 믿어", "난 자유롭다"라는 착각이다.
즉, 수험생 스스로의 자유의지에 대한 착각은 성적은 올려주지 못 하지만 \textbf{쇼핑의 핑계로는 충분하다는 것}.
그래놓고 나서 또 수능이 끝나면 자기는 열심히 공부했는데 징징징 뭐 이러겠지.
\vspace{5mm}

적어도 내가 보기에는 수험생은 그냥 중세시대로 돌아가는 게 맞다.
실력을 갖춘 다음에야 르네상스로 가는 거지, 그게 없으면 아무 생각 말고 지식노동을 하는 게 맞다.
지식노동을 강제당해서 망한 케이스는 사실 없다.
혹자 수험사이트에 자기가 열심히 했는데 망했다... 그거 본인 이야기이지 검증된 것 있나?
재미난 것은 명문대에 합격한 애들은 열심히 공부했단 말은 정말 별로 안 한다.
반면 실패한 애들이 열심히 공부했다라고 정말 \textbf{강조}해댄다.
\vspace{5mm}

앞으로 먹힐 사업은 폭력 시비가 덜한 스파르타 밖에 없지 않나
아니 사실 교육의 본질은 이런 게 아닌가. 그냥 내비두면 모글리 꼴 나는 애들을 두들겨패서 \textbf{문명을 세뇌시켜 인간 만들기}.
정작 매와 교과서로 인간이 된 사람들이 "아이들을 자유롭게 풀어주세요, 창의적인 게 중요해요"라는 것만큼 우스운 게 없지 않나.
\vspace{5mm}






\section{[정리론 020] 시험모드로 들어가기}
\href{https://www.kockoc.com/Apoc/261675}{2015.08.19}

\vspace{5mm}

"노동도 하지 않는 자본가들이 돈을 버는 것은 착취"이다라는 건 흔한 초보들의 궤변 아닌 궤변이다.
사실 공짜는 없지 않는가.
돈을 굴리는 사람들도 대가를 치른다. 그건 "투자한 돈이 허공에 증발할지도 모른다라는 \textbf{불안함}"이다.
실제로 성공한 사람들은 그만큼 많은 것을 잃는다. 다만 조기에 잃어서 게임을 포기한 루저들과 다른 건,
전재산을 날렸더라도 그런 실패로부터 얻은 지혜와 경험으로 다시 일어났다는 것이다.
\vspace{5mm}

수험이 공부라는 건 헛소리이다. 공부가 필요한 건 맞다. 그러나 수험은 절대로 공부가 아니다.
다만 공부하면 다소 유리해지는 투기인 것이다.
\vspace{5mm}

정리모드로 들어간 사람들이 해야하는 작업은
첫째로는 그동안 양치기했던 교재들의 오답정리가 되겠지만
둘째로는 바로 실전형으로 들어가 더욱 많이 깨져보면서 그 불안과 고통을 즐겨야하는 것이다.
\vspace{5mm}

알보칠을 발라보신 적이 있는가.
고통이 어느 정도인지는 직접 발라보지 않으면 모른다.
재밌는 건 바르다보면 어느새 두근두근해진다는 것이다.
5분 정도 괴로워하다보면 구내염이 느껴지지 않으면서 치유되어간다는 '새로운 쾌감'을 발견하기 때문이다.
\vspace{5mm}

불안과 고통을 즐기는 건 다른 게 아니다.
\textbf{직접 실전형으로 제한시간을 두고 시험모드로 돌려본다(실모는 비추)}
\textbf{오답을 가능한 한 내본다.}
\textbf{그리고 오답을 정리해 본다.}
\textbf{그 다음 같은 시험지나 비슷한 문제로 또 시험모드 들어가 이전의 모의시험보다 '개선된' 걸 확인한다.}
\vspace{5mm}

지금부터 해야하는 것은 이런 것이다.
30문제를 치른다고 하면 10문제는 오답이 나오도록 일부러 시험시간을 줄이면서 빡세게 친다.
(하지만 실모는 비추. 왜냐? 그거 업자들이 정말 제대로 적중시키는 문제 내는 건 드물고 변태같은 문제를 내는 경우가 많아서이다)
일부러 빡세게 치면 조금이라도 실력에 균열이 가있는 부분은 여지없이 걸린다.
그럼 그 부분만 정말 집중 케어한다.
\vspace{5mm}

그 다음 비슷한 시험을 쳐서 그 문제를 안 틀리는 걸 확인하고 '쾌감'을 느끼는 것이다.
\vspace{5mm}

실제 시험에서의 두려움과 공포를 이겨내는 방법은 단 하나이다. 그 두려움과 공포를 쾌감의 원천으로 삼는 수 밖에.
쾌감을 얻으려면 절대 피해서는 안 된다. 사전에 겪고 그걸 넘어서 동기부여의 단계로까지 끌어올려야 한다.
\vspace{5mm}

똘똘이 스머프라면 이런 이야기를 할 것이다. 공부하면 안 틀리는데 뭐하러 일부러 틀리고 앉아있느냐.
말도 안 되는 헛소리이다. 명품 인강 하나 듣는 것보다는 제대로 틀려보고 그 다음 오답 정리해서 극복하는 게 훨씬 낫다.
본인이 그런 두려움과 공포로 쫄아보고 고통과 스트레스를 겪으면서 그걸 쾌감의 원천으로 삼는 프로세스를 직접 경험해야 한다.
\vspace{5mm}








\section{[정리론 021] 실모 고민하실 분}
\href{https://www.kockoc.com/Apoc/276055}{2015.08.27}

\vspace{5mm}

어차피 실모 본다고 되는 것도 아님.
\vspace{5mm}

\begin{itemize}
    
    \item 국어 - 마땅한 게 없음, 기출 돌렸으면 EBS 보시고, EBS 다 보았을 경우
    \begin{itemize}
        \item  자신이 중상위권이다 → LEET나 PSAT 한번 도전해보시고
        \item  자신이 중하위권이다 → 오답정리나 하셈
    \end{itemize}
    
    \item 수학 - 쓸데없는 실모가 너무 많아서.
    

    개인적인 추천 : 일격필살 A, B형(올해판), 신승범 모의고사
    이것 빼고 나머지는 볼 시간이 있을지 모르겠음. 솔직히 일격으로도 힘들지 않을까싶고
    저 정도 풀어본 다음에 오답정리하고 EBS 수특 수완과 4점 기출 오답정리하거나 쎈, 라벨, 정석 어려운 문제'만' 골라보기
    이것만 해도 시간 모자랄걸?
    
    \item 영어 - 사실 이거 실모가 필요하나 의문
    개인적으로 변형류는 비추합니다. 연계교재라면 그 시간에 EBS 지문 달달 외우는 게 낫고, 비연계 대비할 거라면 다른 것 보셈.
    연계류라면 그냥 그 시간에 지문 해설 20번 이상 돌려 읽기를 권하겠고
    비연계 - 빈칸추론 대비할 거라면 정말 어려운 것 하나 사서 푸시는 게 나음.
    추천할만한 교재가 없는 건 아닌 데 다른 분들을 위해서 기밀보안. 서점 가서 자알 찾아보셈(저평가된 것 하나 있음)
\end{itemize}


역시나 하고 또 실모 대홍수인데 그거 솔직히 수험생들 처지 고려해주는지 의문이고 - 좌절시켜서 실모신앙 강화시킨다면 모를까 -
지금 가야하는 건 이제 마무리에다 정리이지 혹시나 하고 욕심내는 게 아니죠.
찬바람 부니까 분발, 어쩌구할 건데 그런 거 망상입니다. 추석 때 또 한번 퍼지게 되어있다가 10월에 정말 최종자포자기 모드 맞이합니다.
\vspace{5mm}

시험 때 어디서 점수 나갔을까 가정해보면 답이 나오는데
그런 거 실모 안 풀어서가 아니라, 정말 자기가 점검하지 못 했던 실수패턴이나 제대로 알지 못 했던 것,
혹은 킬러문제를 침착하게 푸는 실력이 덜 되어 있어서 그런 것일 게 뻔하니까 이런 거나 잘 단속들 하시는 게 나음.
\vspace{5mm}

분명 실력은 있는데 수험판에서 이상하게 엿먹는 애들이 있는데 그거 이상할 게 없습니다.
고난이도는 열심히 추구하죠, 그런데 고난이도 뽕 맞아서 자기의 불량을 못 보아서 시험 때 털리는 거예요.
성능이 아무리 좋아보았자 결정적일 때 뻑나고 고장나면 답없는 것임.
\vspace{5mm}

공부를 잘 한다라는 건 세가지임.
\begin{itemize}
    \item 첫째, 기본이 정말 잘 되어있어서 실수를 잘 하지 않는다
    \item 둘째, 상위권 마인드가 잡혀있다.
    \item 셋째, 특정 과목의 어려운 문제를 해결할 수 있는 사고 훈련이 되어있다.
\end{itemize}
\vspace{5mm}

저건 평소에는 하나라도 어긋나더라도 드러나지 않죠. 그런데 시험 때는 저게 확 드러나버리지요.
\vspace{5mm}

그런 차원에서 본다면 그럼 실모들 보는 게 도움이 되나.
도움이 될 수는 있는데 지금과 같은 과열풍조는 제가 보기엔 '거품'입니다.
정말 도움이 되려면 실모에서 나온 문제가 그 해 기출에서 나왔다... 여야하는데 그럴 가능성은 매우 낮죠.
그러나 기본적인 데 털려서 오답 날 가능성은 매우 높습니다.
\vspace{5mm}

사람들이 재밌는 건 확률이 높은 건 신경을 쓰지 않고, 확률이 낮은 것만 신경쓰더군요.
교통사고 나 죽을 확률과 희귀병 걸려 죽을 확률 중 전자가 높지요.
그런데 사람들은 차조심은 안 하면서 무슨 바이러스 퍼진다라고 하면 나 그걸로 죽는 거 아냐라고 3일간 호들갑을 떨죠.
수능도 마찬가지임,
본인이 오답이 난다면 그건 정말 어려운 문제여서가 아니라, 난이도가 낮지만 함정이 있어서 실수하기 쉬운 문제일 건데
대부분 어려운 문제만 대비하려고 하고, 실점 가능성이 높은 문제는 대비하지 않죠.
특히 후자는 수험고수일수록 정말 자주 겪는 게 보이덥니다.
\vspace{5mm}

현명히들 처신하시기를. 올해도 이변이 없는 한(이변이 있다고 하도 집단이 응시하니 어차피 달라질 건 없음) 저 지적대로일 겁니다.
\vspace{5mm}






\section{[정리론 022] 오답이 난다면 어디서 날까.}
\href{https://www.kockoc.com/Apoc/277121}{2015.08.28}

\vspace{5mm}

만약 당신이 출제자라면 어디서 애색기들이 틀리게 만들까
그런데 다음 조건을 지켜야 함
\vspace{5mm}

\begin{itemize}
    \item \textbf{a) 애들 자살하는 소리 나오지 말게 어렵게만 내지마라}
    \item \textbf{b) 컷에 따라 아름다운 분포가 나오도록 조절하라}
    \item \textbf{c) 강남 대치동과 사설학원가에 유리하게 내지마라}
    \item \textbf{d) 계산 너무 지저분하게 하지마라}
\end{itemize}


\vspace{5mm}

자, 이 정도는 너무 당연한 이야기인데요
여기서 바로 보이지가 않음?
\vspace{5mm}

저건 지금 잘 팔리는 사교육 상품이나 소위 실전모의고사와는 k배 닮음변환(k≠0)이라는 것
오메, 닮음변환이면 똑같은 방향이잖아요.
개뿔, 닮음변환은 k<0인 경우도 있잖아.
\vspace{5mm}

저거 조금만 생각해본다면 얼마나 많은 사람들이 망상에 빠져있는지 알지요.
\vspace{5mm}

수험에는 attack 모드와 defense 모드가 있죠.
attack 모드는 남들이 못 푸는 어려운 것을 싹싹 풀어내겠다는 것
defense 모드는 사소한 것도 틀리지 않는 것, 즉 출제자의 공격을 방어해내겠다는 것.이죠.
\vspace{5mm}

그런데 재밌는 건 유명한 수험상품 어떤 것도 defense 모드는 절대 생각해내지 않습니다.
한데 올해보다 광기가 서렸던 작년 시험에서 수험고수들이 망한 게 별 게 아니죠. defense에서 죄다 털렸어요.
반면 최상위권을 감히 생각 못 했지만 기본에 충실한 친구들은 난이도 혜택을 톡톡히 보았습니다.
실점이 거의 없으니까 일어나고보니 만점자 수준 나온 거죠.
\vspace{5mm}

그럼 올해도 뭐가 유리할까. 최소한 'defense'가 attack보다 유리하다는 건 부인할 수 없죠.
그러나 수험생들의 문제는 '집단의 레밍즈 광기'에 결국 휩쓸린다는 것입니다.
자기 약점을 잡을 시기에 쓸데없는 교재들을 막 늘리고 있죠.
그런데 그 교재들을 만드는 사람들이 정말 수험생들 생각해줄 리야 없지 않습니까.
참신한 문제를 내겠다.. 그거 좋은데요, 그것만으로는 도움이 된다고 확신할 수 없습니다.
attack 모드에는 도움이 되겠죠, 하지만 그게 수험생들의 약점을 잡는데 도움이 될까요?(이게 제가 실모비관주의적인 이유입니다)
\vspace{5mm}

사실 이럴 때에는 개나소나 푼다(?)하는 실모 보는 것보단
자기 약점 철저히 정리해서 그 약점 보완해줄 교재만 선별해 보는 게 승률이 높습니다.
자기 방어선 철저히 하고 약점공략을 해야 통수맞는 일 없지,
아, 남들 푸는 x모 풀거야하다가 빈집털이 신나게 당하지요.
\vspace{5mm}

가끔 실모 잘 팔린다라고 거드름피우는 경우 그냥 한마디만 물어보면 됩니다.
\textbf{"그래서 작년에 적중시켰나요"}
유감스럽지만 그런 예는 단 한번도 없는 것 같습니다.
출제문제가 비슷하다고 내는 예시 논리대로라면 EBS, 쏀 등은 적중율이 99$\%$이게요.
\vspace{5mm}

이건 증권방송과 똑같은 겁니다. 증권은 뉴스에 호재라고 하면 오히려 떨어지고, 뉴스에서 다루지 않을 때 올라가죠.
교재 만드는 사람이나 가르치는 사람이 뭔가 잘 알고 있을 거다... 그런 것은 착각입니다.
그러나 한가지 꼼수로 예측율을 높이는 방법은 있죠.
\vspace{5mm}

"수험은 다수가 실패하는 게임이다"
"다수와는 다른 방향으로 가면 망하지 않는다"
\vspace{5mm}

이 논리대로 가서 다수가 안 하는 방향으로 예측질을 하면, \textbf{적중률이 높아지는 건 정말 당연한 겁}니다.
수험과 같은 일종의 도박은 다수가 원하는 방향으로 이뤄질 수가 없으니까요.
100명 중에 5명이 성공하는데 80명이 하는 방향대로 가보세요. 그게 성공할 수가 있나 - 그게 맞으면 80명이 다 대박나야하는데요?
수험 중에는 나머지 20명의 소수전략이 비웃음을 삽니다. 그런데 수험이 끝나고 20명이 그럭저럭 성공하면
그 20명의 전략이 정론이 되고, 심지어 그 20명을 비웃던 사람들이 슬그머니 자기가 그런 주장을 했다고 말바꾸기 시작하죠.
\vspace{5mm}

요즘은 업자들이 자기 교재를 홍보하면서 기출과 교과서를 봐야한다고 얘기하죠.
그런데 이건 더 간단하지 않나요? 그 교재 안 보고 그냥 \textbf{기출과 교과서만 보면 되지}
업자들이 자기 이미지를 관리하고 자기가 바른 말을 했다고 이미지 관리를 하기 위해 \textbf{물타기하는 거죠.}
그럴 바에는 기출과 교과서를 볼 필요도 없도록 자기 교재나 업그레이드 하면 될 건데, 그럴 자신은 사실 없는 것입니다.
(게다가 그 사람들의 책을 보면 교과서와 정반대입니다)
\vspace{5mm}

그리고 제가 수험생이라면 - 아마 이건 꽤 어그로 끄는 이야기지만 그래도 한마디해야겠어요 -
\textbf{재수 없이 현역으로 합격한 사람 말을 재수 이상보다 더 신뢰할 것입니다.}
이 대목에서 정말 많은 사람들이 빡치겠지만, 현실은 현실대로 인정합시다.
한번에 합격한 사람과 그렇지 않은 사람은 사실 엄청난 차이가 있어요. 그게 투명해서 보이지 않을 뿐이지
그건 일종의 '마인드'와 '습관'과 같은 것입니다.
한가지 예만 들면 제가 관찰하거나 경험해본 '현역합격자'들은 정말 소박합니다. 절대 화려하지가 않아요,
한데 특징은 "쓸데없는 것을 안 한다", "낭비 같은 것도 안 한다", "수험에만 집중한다"입니다. defense는 당연히 잘 되어있죠.
그에 비해서 n>=2인 사람들은 정말 낭비가 많습니다, 그리고 서울에만 가도 되는데 더 가서 평양까지 가려합니다(...)
게다가 수험 후반이 되면 불안감을 이기기 위한 과대망상이 심해지는 경향이 있어요.
어느 시험이건 분석해보면 정말 단기에 실패없이 합격한 사람들이 쓰는 수기나 내는 책.
정말 '간결'합니다. 그리고 '중요한 것'만 담고 있더군요.
그에 비해서 실패를 자주 하는 사람들은 본인들은 인정하지 않지만 정말 '장황'합니다.
중요한 것 외에 중요하지 않은 것도 담고 있고, 그래서 그걸 보는 수험생들도 똑같은 실패를 할 가능성이 있습니다.
\vspace{5mm}

아마 이 마지막 대목에서는 제가 생존율이 낮아지지 않나 하겠습니다만
진실은 진실대로 적어야죠. 그리고 당사자들이 배포가 크다면 저걸 인정하면서 개선하기 위해 노력하겠지요.
하지만 여기서 끝나면 메시지가 재미가 없죠
지금 잘 팔린다는 x모니 뭐니하는 것들의 대부분 문제는 저기서 고스란히 확 드러난다는 것입니다.
\vspace{5mm}

이런 걸로 머리 아프기 싫으면 그러니까 교재 이제 쓸데없는 것 늘리지 마시고
약점공략만 하시길요.
\vspace{5mm}

+
\vspace{5mm}

수험생이 얼마나 우매한지 확인할 수 있는 현상. 그건 작년에 기억나는데
국어가 어렵게 나온다, 탐구에 정말 집중해야한다고 했을 때 공격먹었죠. 하지만 결과는 아시다시피
아, 물론 수학이 더 어렵게 나오지 않을까라는 예측은 완전히 빗나갔죠.
\vspace{5mm}

그런데 저 예측도 별 게 아님. 딱 돌아가는 걸 보면 수험생들이 국어와 탐구를 우습게 보았거든요.
그럼 평가원도 비슷하게 생각하지 않겠습니까.
결국 다수가 생각하는 것과 다른 방향으로 생각하면 예측이 맞아떨어지기 쉬운 거죠.
다만 설마 수학까지 건드리겠어라는 것은 저도 생각 못 했던 것이고.
\vspace{5mm}

끝까지 이성의 끈을 놓지 말고 자기 약점만 공략해나가는 게 답입니다. 전반 양치기 잘 되었다면 지금 그거 되겠고
뭐 자기 고집대로 인강만 듣겠다하면서 문풀량 달성 못 한 친구들이야 지금 정신 헥 나가있겠고
\vspace{5mm}

+
\vspace{5mm}

일격을 추천한 이유는 실모로서라기보다는 생각할 껀덕지를 주는 문제들이 '가격'치고는 많이 있어서입니다.
만약 6회였다하면 당연히 추천할 이유는 없습니다. 그 정도 가격까진 먹을 필요 없다고 생각.
가격을 문제수로 나눠서 대략 50$\sim$70원 정도면 양호한 겁니다. 그것보다 더 비싸면 정말 좋은 문제고 해설이 괜찮냐하는 건데
냉정히 말하면 아직까지도 실모는 기대 이하인 경우입니다.
수험생들이야 "너희 대학도 못 가면서 뭘 평가해"라고 하니까 말을 못 하겠지만, 저는 그럴 입장이 아니니까요.
그럼에도 불구하고 일격은 더 쓴소리할 겁니다. 욕먹는 그런 실모 집단에 계속 낄 이유가 없으니 말입니다.
썹모는 그래도 가성비 좋다고 봅니다. 달라붙은 검토위원 명단들에 비해 해설이 간단하단 느낌이 들지만 그래도 이건 돈값을 한다고 봅니다.
\vspace{5mm}

게다가 위 두가지는 제가 권하던 전통적인 시중교재 위주의 학습에서 빠지기 쉬운 걸 제대로 보충해주고 있다고 체감해서리
그러나 반드시 봐야한다... 할 필요는 없다고 봅니다. 다만 남들이 하니까 나도 실모 봐야겠어하면 우선적으로 검토해보라는 거지
아마 다른 실모를 본다고 해도 저것 이상으로 만족하는 일은 없을 거라고 보네요.
게다가 어떤 실모든 그건 attack 모드에만 가는 경우가 많아요. 저자들 스펙과 문제 성향을 분석해보면서 '역시 그랬군'하고 생각하기도 합니다.
\vspace{5mm}

+
\vspace{5mm}

지식보다 중요한 건 습관과 마인드입니다. 통칭해 성격이라고 봐도 좋을 것 같은데
올해 시험 끝나고 다시 시작하는 분들은 업자들 말 듣지 말고
콕콕 사이트든 어디든 정말 '합격자'들 말 잘 취합해 보시고, 특히 현역합격자들 충고 잘 추려보시길 바랍니다.
어떤 교재를 보았느냐도 중요하겟지만, 그 전에 그 친구들이 어떻게 생활했는가 그거 추려서 공통점 추출해내고 따라가십시오.
그게 일지를 보라는 이야기이기도 하니까 확인해보면 되겠지만 사실 별 건 없습니다.
\vspace{5mm}

콕콕 하면 똑같이 5수한다... 주인장 디스이기도 할 건데 진짜 주인장 빡치실 수도 있겠지만,
근거없는 이야기만은 아닐 수도 있습니다. 왜냐면 선도자의 마인드나 습관에서 안 좋은 것도 '걸러지지' 않고 전파될 수 있으니까요.
(앞으로 글 안 올라오면 저 살해당한 줄 아셈, 범인은 선량한 시민이라고 적어놓겠음)
뭐 중요한 건 그걸 어떻게 거르느냐고 주인장 분도 열린 분이니 오히려 그걸 자산으로 삼겠지만
적어도 제가 관찰한 바로는 수험사이트의 선도자 상당수가 자기의 실패까지 전염시키는 걸 막는 구조는 절대 아니덥니다.
참 쓸데없는 걸 너무 많이 강조하고, 그걸로 '장사'한다는 느낌을 지우기가 어렵더군요.
그래서 저자들 스펙을 확인해보면 역시... 라는 말이 안 나올 수가 없습니다.
\vspace{5mm}

한가지 예만 들면 수험고수들은 수학을 고급화된 '자기만의 방법' - 즉 한가지 방법으로 가려는 경향이 있습니다.
자기가 인상깊게 배운 방법이나 마인드를 강조하는 건 좋지요. 그건 드라마틱하고 감동적이니까요.
하지만 평가원은 병아리 감별사 집단입니다. 수평아리들을 분쇄기에 넣는 걸 서슴치 않는 잔인한 집단이죠.
문제가 아름답게 풀리든 말든 그런 것은 신경쓰지 않습니다. 그렇다고 문제를 잘 내느냐. 사실 그런 것도 아닙니다.
(여기서 또 아이러니한 건 업자들이나 수험고수들은 평가원에 대해 맹목적 추종과 비판 양쪽에서 와리가리한다는 것이지요)
평가원이 잘 하는 건 단 하나입니다. \textbf{잘 걸러낸다, 그것이죠.}
작년과 올해 모평 추세를 보면 정말 사설부터 인강까지 저격질만은 제대로 합니다. 문제가 좋은지 아닌지 떠나서요.
2014 출제와 2015 출제 중 저는 당연 후자를 높이 칩니다. 그렇게 난이도 낮추면서도 참 잘 걸러내고 제대로 한방 먹였다 생각이 들어서이죠.
\vspace{5mm}

업자들이 하는 말은 사실 거짓말이지요. \textbf{"평가원이 우리 업자들을 노리고 있다"}라는 진실이면 끝나지 않나요?
성공한 출제란 업자들을 엿먹이는 출제고, 실패한 출제는 업자들과 내통한 출제겠죠
난이도가 문제라고요? 그게 아닙니다. 난이도가 불지옥반도라도 출제경향이 사교육이 원하는 것이면 그게 물수능이 되는 것입니다.
난이도라는 건 출제하는 문제와 그걸 푸는 수험생들의 수준이 조응해서 나오는 경향이지 단지 쉽게 낸다 어렵게 낸다 그 수준이 아니죠.
\vspace{5mm}




\section{[정리론 023] 계획을 짜면 왜 망하는지 설명해드리겠음.}
\href{https://www.kockoc.com/Apoc/290346}{2015.09.04}

\vspace{5mm}

계획에 대해선 수많은 책들이 있지만 사실 거의 다 허무맹랑한 이야기임.
그리고 계획을 쓰는 경우 지키는 경우는 사실 드물다(나도 역시 그렇고)
이 역시 나름대로 연구해보고 내린 결론이 있는데 다 쓰면 밑천이 떨어지므로 암시를 드리겠음.
\vspace{5mm}

식량이 주어짐, 일주일동안 그걸로 버티라고 함
\textbf{100kg 나가는 살아있는 돼지 한마리 vs 삼겹살 두근, 족발 10인분, 햄 5인분.}
당신의 선택은?
\vspace{5mm}

이거 십중팔구 다 '전자' 선택함.
왜냐고. 다들 이렇게 대답하죠. "100kg짜리 돼지가 양이 많잖아요"
그리고 이게 댁들이 계획을 세우면 실패하는 이유죠.
\vspace{5mm}

저기서 계량해야 할 건 고기근이 아니라, "내가 당장 간편하게 먹을 수 있는 식량"임
살아있는 돼지는 도축시키는 것부터 시작해 요리할 수 있는 상태로 만드는 것까지가 매우 '귀찮은' 작업입니다,
그리고 자칫하단 놓쳐버릴 것이고
심하면 내가 거꾸로 잡혀먹힐지도 모릅니다..
\vspace{5mm}

계획실패의 주된 원인은 살아있는 돼지를 선택해서입니다.
선택하려면 당장 먹을 수 있는 손바닥만큼의 햄을 선택해야지요.
앞의 것을 먹을 수 있는 확률은 매우 낮거니와, 중간에 귀찮은 일을 많이 해야하지만,
뒤의 것은 양이 적더라도 간편하게 먹을 수 있어서 나의 양분으로 삼을 수 있습니다.
\vspace{5mm}

계획을 짤 때에는
첫째, 반드시 \textbf{'바로 실천가능한 단위'}까지 세분화시켜야하고
둘째, 절대 욕심을 부리지 말아야하며
셋째, 그 계획을 실천하지 않으면 안 되는 환경을 만들어야 합니다.
\vspace{5mm}

가령 "쎈수학 한권 다 풀기"라는 것과 "쎈 수학 매일 1문제씩 풀기"라는 계획
양으로 친다면 후자는 1년에 365문제 밖에 되지 않으미 엉터리 계획이라고 하겠지요.
\vspace{5mm}

하지만 현 단계에서 성공한 계획은 후자입니다. 에게게, 1문제 밖에 안 돼? 라는 이야기는 '바로 실천가능하다'는 얘기거든요.
하지만 쎈수학 한권을 빨리 풀어라... 이거 몇명이나 실천하겠습니까. 바로 중도포기해버리죠.
그리고 1문제 풀다가 이게 넘 적다하면 대략 4$\sim$50문제로 늘어날 것입니다. 그래서 매일 실천가능하면 그게 적정량을 찾는 것입니다.
\vspace{5mm}

실제로 플래너 과시나 자랑을 할 때에 이 점을 읽어야죠
플래너를 정말 잘 쓰는 사람들은 "누가봐도 즉시 실천가능한 프로세스"로 적습니다.
막연하게 $\sim$하기라고 적지 않아요. 그런 플래너를 쓰는 사람이라면 어차피 플래너도 포기해버리게 됩니다.
\vspace{5mm}

살아있는 돼지가 도망간다고 했죠?
나 올해 서울대 갈거야라는 목표가 그런 겁니다. 처음에야 저 돼지 근수가 많이 나오겠구나 좋아합니다만,
정작 도살을 시작하려할 때부터 고민되기 시작합니다. 그리고 내가 저 돼지를 잡아야하나 말아야하나부터 고민하죠.
목표를 크게 잡은 사람들이 개똥철학에 빠져서 중도방황하다가 막판에 자빠지는 패턴이 괜한 게 아닙니다.
반면 "하루에 30문제만 꾸준히 푼다"라고 하는 사람은 연초부터 했다면 벌써 6000문제는 풀어서 벌써 성장했겠지요.
\vspace{5mm}

돼지에게 먹혀버리는 것은 바로 이 시점의 멘붕일 것입니다.
차라리 목표를 크게 잡지 않았다면 마음이라도 편했을 건데, 목표를 너무 크게 잡고 시간낭비만 하다가 지금 완전히 다운된 상태죠.
이런 친구들은 돼지에게 잡아먹힌 다음에 환생해도 그 다음 소에게도 잡혀먹힐 것입니다.
\vspace{5mm}

현재 xx 과목이 되지 않아요 어떻게 할까요.
수험사이트 돌아다니면서 인강이든 교재든 추천받아보았자 그거 별 의미없습니다.
당장 필요한 건 그냥 평범한 교재 찾아서 10문제라도 풀어보고 틀리면서 깨지고 오답정리하는 '실천'입니다.
그런 경험을 계속 하면서 문제점의 평균을 찾아낸 뒤 해결하라고 수학에서 '통계'를 배운 게 아닐까요?
\vspace{5mm}

물론 새로울 건 없지만 여기서도 수험상담하거나 일지쓰거나 중도이탈하는 케이스들 보면서 정리한 건 그겁니다.
어떤 식으로든 변화가 있는 쪽은 '실천'하는 쪽입니다. 본인은 처음에는 성과가 없다고 하지만 그냥 생각없이 실천하다보면
자기도 모르는 사이에 실력이 높아져있지요. 대신 그만큼 '눈높이'도 높아졌기 때문에 자기 성적이 불만이 되죠.
이건 꽤 고무적인 일입니다. 이런 친구들온 올해 안 되더라도 내년, 더 운이 안 좋다고 해도 내후년이면 목표는 이룹니다.
\vspace{5mm}

하지만 정말 불치병인 친구들은 한문제라도 안 풀면서 난 xx대에 갈꺼야, 의치한에 수석합격이야라고 꿈만 꾸는 사람들이죠.
곱등이도 못 잡는 주제에 살아있는 돼지 키워서 잡아먹을 꿈에 부풀어 있습니다.
결과는 그 돼지가 eaten하는 게 아니라 eating 하는 것이겠습니다만요.
\vspace{5mm}

