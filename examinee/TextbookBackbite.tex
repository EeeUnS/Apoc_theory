
\section{교재 뒷담화 : C D E F}
\href{https://www.kockoc.com/Apoc/465559}{2015.11.05}

\vspace{5mm}

추정하지 말라고 했는데 이번에도 또 엉터리로 추정하는 사례가 없기만을 바란다.
이 글을 쓰는 이유는 특정 대상을 언급하기보다도, 책을 이딴 식으로 쓰면 안 된다는 '추상화된 문제'를 지적하는 것임.
간혹 이 글 보면서 자기를 씹는 건가하면서 도둑이 제 발 저린 케이스 있을지도 모르지만 그럴 시간에 공부나 하거나 책이나 고치길.
\vspace{5mm}

저번 알파벳은 A, B로 썼음. 그리고 내가 보기엔 그 저자들은 남자가 아니라고 분명 밝혔음.
(적어도 악플(?)다는 사람들은 내가 보기에는 잘 모르는 것 같아)
\vspace{5mm}

그럼 이번에는 C, D, E, F를 논하겠음
\vspace{5mm}

\textbf{C : 서문만 거창한 문제집}
\vspace{5mm}

이 문제집을 쓰는 사람은 집필을 위해서 일까지 그만두었다고 서문에 밝힘. 나야 컬렉터니까 구입
그런데 저자는 집필 기간동안 놀았나... 라는 생각이 든다.
문제 100$\%$ 모두 그냥 기출문제다. 그럼 기출문제를 그냥 무단으로 베낀 것이 아닌가.
해설은 그럭저럭 가독성이 있다, 썰을 잘 읽는 기분이긴 한데, 이 사람 강사라지? 그런데 책은 많이 읽은 것 맞나?
\vspace{5mm}

이 케이스는 강사가 함부로 책 써서 돈벌고 싶어서 출판 나선 케이스다.
그런데 강의와 문제를 만드는 건 별개의 문제지. 문제를 만든다라는 건 보통 어려운 일이 아니다.
이것도 나중에 썰 풀겠지만, 문제를 제대로 만드려면 정말 교수급은 되어야 한다. 그렇지 않은 나머지 자작문제는 한계가 있다.
그런데 이 저자는 문제 만들 능력은 없는 것이다. 그래서 기출만 몽땅 베껴넣었다.
그럼 해설은 어떤가?
\vspace{5mm}

뭐. 실모들보단 낫다. 이게 뭔 이야기인지 알겠지? 그런데 그 해설이 정말 돈주고 살만한 것인지는 의문.
비슷한 값에 10배는 많은 문제를 풀 수 있는 마플에 비하면 그다지 두드러지지도 않는다.
그런데 한권 가격이 참 더럽게 비싸다, 저번에 말했던 B 가성비만 최악이 아니었어.
그나마 B는 짜깁기 내용으로 새로운 걸 접할 수라도 있는데(수능에 도움이 되는지는 의문)
이건 뭐 기출 해설 소프트하게 해놓은 주제에 폭리를 받아먹고 있나.
\vspace{5mm}

내년에도 설마 나올리는 없겠지. 대형서점 런칭은 어느 정도 했던 모양인데 판매량은 글쎄.
\vspace{5mm}

\textbf{D : 고시 방법론을 응용한 문제집.}
\vspace{5mm}

서브노트가 좋아보여서 구입한 문제집이다. 이것도 지금 생각해보니 가성비 \textbf{최악}.
이 문제집은 저자들 학벌 스펙이 그럴싸하다. 그리고 문제푸는 방법론 $-$ 즉 형식적 측면도 새로운 건 있다.
그런데. 그게 전부다(...) 그리고 그 모든 엑기스가 서브노트 한권에 다 있다(...)
다시 말해서 서브노트만 챙기면 나머지는 볼 필요도 없다.
\vspace{5mm}

이 저자들도 C와 비슷한 케이스다. 그런데 C는 저자 한명이기라도 했지, 이 문제집은 저자가 여러명이고 다들 스펙이 좋은데도
모두 기출해설만 했다라는 게 문제다. 그리고 지금 가격 확인해보니까 이것도 가성비 F4 에 들 정도네
저자들이 문제 해설 썰만 그럴싸하게 풀지, 자기들도 \textbf{문제를 만들 능력은 '없다'.}
그런데도 서문에서 만들라고 고생했다라고 하는 건 뭘까.
\vspace{5mm}

문제푸는 방법론도 사실 냉정히 보면 그다지 새롭지는 않다. 이거 내가 알기론 고시 2차 답안작성 방식 그냥 베껴온 경우다.
수학문제 푸는 방식은 이채형이나 강필 인강을 들어도 좋지만 일본 책들 찾아봐도 잘 나와있다.
아니, 조금 부담되더라도 수리논술 양식을 참조해도 된다.
\vspace{5mm}

이거 뭐 책 내용도 없는 거 비싼 종이에다가 칼라풀 인쇄만 했는데 저자들이 뭔 생각이었나 궁금하다.
아무리 보아도 학벌빨만 더럽게 믿고 애들 현혹해 학원 수입이 많아지니까 욕심이 생겨 출판계로 나아가보자라고 한 모양이다.
책 함부로 쓰는 게 아니야, 그리고 전에 학생저자들 개념서 함부로 보지 말라고 했지만 '강사' 책도 이건 마찬가지라고 생각한다.
\vspace{5mm}

\textbf{E : 강사가 낸 낚시용 문제집}
\vspace{5mm}

참고로 이 강사는 꽤 스펙과 실력이 좋다. 그런데 책을 내려면 다 낼 것이지 일부만 내놓고 낼름 도망가버렸다(...)
이건 씹는다기보다는 그 강사의 안목이 참 근시안적이라고 탓하고 싶다.
우선 이 교재는 지수, 로그, 수열에서 나름 접근 스킬이 나쁘지만 않다, 특히 군수열적 풀이에서는 접근이 꽤 명쾌했다고 본다.
그래서 이 교재 후편을 기다렸는데 아무 소식이 없다, 그렇다고 이 강사가 그럼 일타로 잘 나가나. 그런 것도 없다.
걍 어차피 수명이 있는 것 실전개념서 출판이나 다 해서 출판계 점유율이나 높이지 뭐 했나.... 라는 생각이 든다. 지금도 일하고 있으려나?
\vspace{5mm}

\textbf{F : 명문대 합격을 핑계로 낸 질소과자}
\vspace{5mm}

이름부터가 노골적. 이거 학생저자가 낸 책이고 일단 스펙이 후덜덜해서 좋아보이지만 속지 말자.
수학 전분야를 다 터치하는 것 같은데 알고보면 기출문제 그럴싸하게 만만한 것만 추려내서 해설 적당히 함.
수험론적인 걸 강조는 하는데 그거 어차피 다 아는 내용, 그나마 챙길 게 '수험생들이 하는 xx'를 추려낸 것.
나머지는 볼 게 없다. 저자 스펙 내세워서 장사질한 사례고 내가 이래서 학생 저자들을 일단 불신하고 보는 것이다.
목차 구성은 나쁘지 않아서 이 녀석 머리는 좋구나 했는데, 이거 결국 설렁설렁 대충 내용 채워넣었다라는 의심을 지우기 어렵다.
\vspace{5mm}

여기서 E 빼고 나머지는 \textbf{도대체 왜 출판했나 하는 생각이.}
물론 부분적으로는 챙길 게 없진 않다.
C는 말투(...), D는 서브노트, 그리고 F는 목록. 시간 되면 이것만 다 추리고 나머지는 다 버릴 생각이다.
원래는 다 버리려고 했는데 지금 보니 저것들'만' 쓸모가 있다.
그런데 나야 뭐 급박하지 않으니까 저것 빼먹기만 할 수 있지, 적지않은 거금을 투자해 저것들을 구입한 수험생들 인생은 어찌된 것일까.
E는 괜찮다면서 왜 깠냐라고 하는데 후속편 수년간 기다려봐라. 짜증 안 나나.
(는 건 뻥이고 후속편 나중에 적당히 알아보았는데 생각보다 별 게 없더라는 함정)
\vspace{5mm}

가성비는 뭐. 지금까지 보면 C, D도 만만치 않게 최악이란 느낌이 들면서
그 돈을 모아서 미소녀 화보를 직구했으면 눈보신은 할 수 있지 않았을까라는 위험한 생각이 들기 시작하고 있다.
일단 C의 경우는 해설 말투, 그런 것 필요없고 그냥 EBS 기출강의 들으면 되는 것이고
D의 경우는 수학문제풀이와 관련된 책에 훌륭히 나와있거니와 고시 2차 답안 작성 같은 것 따로 구해보면 된다.
그리고 F의 xx목록은 본인이 직접 일기를 쓰면 해결할 수 있는 것이고 사실 풀이만 차분히 쓰면 다 해결된다.
그리고 E도 사실, 저 정도 팁은 풍산자에 다 있다(풍산자는 저자 분이 불편한 현실에서도 책 잘만 쓰시더라)
\vspace{5mm}

수학교재 구입하려는 분은 이 글의 추상적인 단점들 보고 구매할 때 신중히 고려하길 바란다.
C는 그냥 뭐랄까, 경기 어려운 시절에 살아남을 수 있을까. 이 경우는 내가 보기엔 걍 썰만 풀 줄 알고 자기 공부는 안 하는 도태 케이스인데
D는 ‥ 강의까지 내가 찾아서 다 확인해보았는데 나도 조심해야겠단 생각이 들었다. 이 역시 별로 발전은 없다, 그냥 학벌빨로 버티고 있어.
F는 이제 책이 안 나오지 않을까 싶은데 어떤 의미에서 잘 된 일이다. 그런데 이건 양심이 아니라 그냥 저자가 바빠서 도망간 것 같다.
일단 C는 그거 풀다간 같이 한심해져서 서민이 될 판이고
D는 형식이 그럴싸한데 형식만 만들고 나머지 개념과 기출은 걍 표절한 수준, 그래서 내가 블랙리스트에 넣었다.
F는 뭐. 그런데 다른 의미로는 리스트를 만들어서 조심해야 할 판이다. 이런 놈은 사기를 잘 칠 것 같다.
그럼 E는? 소심하지. 실력과 내용이 좋은데 통크게 장사 못 하니까 결국 뜨지 못한 듯.
\vspace{5mm}

그렇다능.
\vspace{5mm}






\section{교재 뒷담화 : G H I J K}
\href{https://www.kockoc.com/Apoc/468220}{2015.11.07}

\vspace{5mm}

가르치는 사람은 배움을 게을리하지 않으면 안 되는데
이건 두가지 이유임.
\vspace{5mm}
\begin{itemize}
    \item 첫째, 아는 게 있어야 가르친다.
    \item 둘째, 배우지 않으면 본인이 \textbf{교주가 되어버린다}.
\end{itemize}
\vspace{5mm}

그런데 첫째까지는 아는데 둘째를 경시하고 그렇게 괴물이 되어가는 사람들을 본다.
가르치면서 선생님 선생님 받들여지고 하면 처음에는 기분이 좋은데, 그게 1년동안 지속되면 안하무인.
그래서인가 강사들이든 저자들이든 인간성 측면에서는.... 언급을 하고싶지 않은 경우라고 정리할 수 있다.
\vspace{5mm}

교재 G : \textbf{지금은 스산해져버린 온갖 스킬 꼼수의 향연}
\vspace{5mm}

이 교재는 구하기 어렵다. 그리고 개인적으로는 버릴 생각도 없다.
왜냐하면 스킬 꼼수와 사파적인 개념에 대해서 이만큼이나 잘 모아놓은 기록도 없기 때문이다.
읽고 정리해보면서 오, 한 때 인기를 끄셨다는 분이 어떻게 독창적으로 해석했는지 그리고 이게 왜 인기를 끌었는지 그것도 알 수 있을 것 같다.
사실 이 저자에게는 나 개인적으로는 고마움을 표하고 싶다. 단지 이 분이 '인기'에 취해서 이미지 관리나 하다가 발전없이 도태된 것이 아쉬울 뿐이지만.
\vspace{5mm}

우선 저 스킬은 재밌게 가르치고 싶은 사람에게는 매우 요긴하다.
그리고 수험적으로 쓸만한 건 있다. 가령 문과 21번 킬러에서 다항함수 조건 주고 f(k)=? 하는 문제에 쓰일 스킬 같은 것이 다 나와있다.
하지만 그 뿐. 그 외에는 수능에 도움될 건 없다. 이게 상당히 유감스럽다.
끝까지 읽고 정리해보면서 느낀 건 한편으로 매우 감사하다는 것, 그러나 다른 한편으로는 사파로 가면 진화를 못 하는구나하는 경고였다.
(그리고 이것이 내가 강의에 너무 중독되지 말란 이유이기도 하다)
\vspace{5mm}

지금 수학교재 함부로 쓴다고 하는 사람들이 날고 기어보았자 저 분의 책을 따라가긴 힘들 것이다.
이 책은 노량진 홀로서기에서 구매한 책이다(pdf 파일 따로 복사하려고 커피 마시면서 노가리까다가 재밌어보이기에 구매)
가성비는 좋다고 할 수 있다. 매우 저렴했으니까 $-$ 수십만원 들어야 겨우 몇줄 챙기는 스킬들이 2만원 내에로 수십개는 나와있음.
어떻게 보면 사파 수학으로서 실력자인데 지금은 어디서 뭐하는지 알 수 없다(...) 그냥 은퇴하신 건가.
교주놀이 끝을 보는 기분?
잠깐, 이거 뒷담화 맞긴해?
\vspace{5mm}

교재 H : \textbf{대학교수들이 쓴 책으로 기대를 모았지만}
\vspace{5mm}

수학교육과 전공이 아니면 그다지 기대하지 말자, 걍 경문사 책이나 보거나 일본책을 보는 게 맞다는 것만 얻었다.
이 책은 가격이 꽤 세다(담긴 내용에 비하면). 일단 저자들은 스펙이 상당한 교수님들.
왕년에 나 수학 잘 했어 라는 심정으로 쓰신 책인 것 같은데 아뿔사.
기출 실어놓은 게 1990년대 초기 수능 기출. 그거 해설로 매진한 건 좋은데 책 내용 상당수가 걍 오일러 정리(...)
복소수 개념에 대한 회전변환 같은 거야 괜찮게 적긴 했는데 괴수님들, 이거 요즘 수능과 무관하다고욧.
스킬이 안 실린 건 아닌데 그 스킬들이 사실 별 소용이 없는 것들(...) 어떻게 된 게 저기 G만도 못 하냐(사회적 스펙은 더 좋구만)
이건 뭐 똥뱃살 더럭더럭 찐 왕년 권투선수가 이종격투기 게임에 나와 상대도 경로우대를 안 할 수 없는 그런 상황.
\vspace{5mm}

교수라고 무조거 믿으면 안 된다라는 선례를 남겨주었다.
수험용이 아닌 그냥 심심풀이 책이면 괜찮지만, 수험용으로 보았다면 그 학생은 +1 확정.
\vspace{5mm}

교재 I : \textbf{패턴정리 잘 되어있다라고 해서 기대했는데 영$\sim$}
\vspace{5mm}

이 사람은 학벌도 괜찮고 아예 이 분야에서 가르친다라고 해서 구매. 이거 일다 문과용인데 $-$
뭐랄까 상당히 실망스러웠다. 저자는 방송에 등장한 걸 보니 선량한 사람이고 게다가 이 분야만 적극적으로 판 경우이긴 한데
아무래도 강의만 하다보니까 책은 그냥 실망스럽게 만들 수 밖에 없지 않나 그런 느낌이 들었다.
문과 수학인데 문제에 대해서 '패턴적인 접근'만 하고 있다, 결국 탈패턴까지는 못 가고 있다.
다시 말해 $\sim$ 하게 풀어야한다라고만 기술하고 있지, 왜 그런 풀이가 나오나, 그리고 수학적 정의에서 어떻게 그런 자명함이 나오느냐 없다.
게다가 문과수학인데 편미분은 왜 동원했냐 도대체(...)
수학 외 다른 자전적인 얘기까지 쓰고 그건 감동스럽긴 한데. 결국 뭐야, 발전이 없잖아(...)
\vspace{5mm}

그래도 인간적으로 이 사람은 깔 필요는 없는 것 같다.
이 분야 세계를 보면 정말 '신사적'인 사람고 그렇지 않은 사람들이 있다.
그냥 아무 것도 아닌데 대충 올라가서 수험생들에게 위세떠는 천민들이 있고,
자기들이 원래 공부를 잘 했기 때문에 중하위권을 배려하는 신사들이 있다. 적어도 이 저자는 신사이다.
다른 건 몰라도 힘든 사람들을 위해 상담해주고 선량한 마음으로 소박하게 일하면서 교육 본연에 충실한 건 인정할 수 밖에 없다.
그리고 이 사람, 무엇보다 교재 추천 목록이 나랑 거의 흡사하다. 비슷한 교육코스다보니까 그런 게 아닌가.
\vspace{5mm}

G는 어떻게 되었는지 모르겠지만 책의 스킬은 감사. 이건 내가 따로 연구해보아서 콕콕에서 공개할 수도 있음.
다만 G로 공부하면 수학을 제대로 공부했다... 라고 하긴 힘들다는 점에서는 문제인 책이다. 실력자들이 보면 읽을 수 있지만.
H는 대학교 수학은 고교 입시수학과 다르다는 반례로서 충분치 않나 싶다.
그리고 사실 본고사 세대가 '암기수학'을 했다는 반증임.
\vspace{5mm}

I는 맨 앞에 쓴 그 교주가 되지 않길 바란다.
자전적 기록이나 고발서를 보면 이 사람 꽤 괜찮다 느낌도 있지만 안 타락한다라는 보장도 없지.
이 사람도 2d에 취향이 있다면 내가 적극옹호하지 않을까 싶은데 잠깐 뭔소리하는거지 내가?
\vspace{5mm}

... 그런데 여기서 끝나면 섭하지
\vspace{5mm}

\textbf{J : 교주놀이의 대가}
\vspace{5mm}

이 책도 지금 버려야할 것 같아서 보고 있다.
이 경우도 신랄하게 까야할 것 같다.
\vspace{5mm}

\begin{itemize}
    \item 첫째, 저자가 내가 보기엔 공부를 안 한다
    \item 둘째, 교재가 제목이든 내용이든 자가당착이다(교과서를 강조하긴 하는데 해설은 why가 없지 이상한 패턴 위주냐. 이런 책들 꽤 있네?)
    \item 셋째, 실으라는 문제는 기출 대충 짜깁기로 실어놓고 북한에서 수령님 찬양하는 듯한 글들은 곳곳에 박혀있냐?
\end{itemize}
\vspace{5mm}

이 역시 가성비는 시궁창. 아, 이걸 내가 왜 샀지 $-$$-$
저자 스펙은 글쎄, 내 입장에서는 뭐 이래놓고 자랑질을 하고 있냐 그런 생각이 들고있다.
그나마 위의 G는 저자가 스킬 정리라도 했고, H는 교수답게 걍 수험과 무관한 신선놀음 볼만했고, I는 착하기라도 하지.
그나마 스킬이라는 것도 사실 스킬이라고 보기에도 민망한 수준이다. 걍 그렇게 풀어라 하고 적혀있다 $-$$-$
그렇다고 수학 전범위를 포괄하나, 이것도 전혀 아니올시다이다. 설명 필요한 대목은 다른 책을 참조하라고 버젓이 적혀있다.
그냥 기출 중심으로만 해설 그럴싸하게 해놓고 걍 마무리. 그런데 설명이든 해설이든 참 말투가 '교주'스럽다.
\vspace{5mm}

무엇보다 용서할 수 없는 게, 개념 엉터리로 인용한 부분이 한두군데가 아니다.
이게 왜 그런가보니까 교과서나 기본문헌을 제대로 인용한 게 아니다. 아무래도 어디 강의 베껴서 대충 적은 티이다.
강의 중에서 교과서 개념 부정확하게 인용하는 것만 골라서 그게 전부인 줄 알고 딱 적은 모양인데(...). 이렇게 잘못된 지식은 전파된다.
수학사랑에서 나온 박교식씨의 수학용어사전 같은 것도 안 보았나. 용어도 부정확하게 쓰고 있고 '야매용어'를 야매라고 언급도 안 하고 적었다.
\vspace{5mm}

두가지가 미스테리하다. 일단 이 책이 왜 잘 팔렸지? 그리고, 이걸 내가 왜 샀지(...) 이것도 구매하면 10만원은 넘어가는데 $-$$-$
어떻게 보면 나도 사기당한 셈인 것 아닌가 해서 지금 가슴이 갑자기 답답해져오기 시작한다.
열람본을 보고 문장의 거시기함을 직감했을 때 그 때 구매하지 말았어야하는데.
\vspace{5mm}

아마 다른 책들을 안 보고 이 책을 보았다면(뭐 그럴 가능성은 낮긴 하지만) 난 아무래도 중대한 인도적 범죄를 저질렀을지도 모른다.
수학책인지 아니 주체사상 경전인지 헷갈릴 정도의 책이다 $-$$-$
저자 소개부터 머릿말부터 중간중간 자뻑이 아무래도 '수령님 쓰시는 축지법$\sim$'이라는 노래가 딱 BGM으로 적절하거든 $-$$-$
시장을 믿을 수 밖에.
\vspace{5mm}

K : \textbf{아재 xx 서요?}
\vspace{5mm}

이 책은 마이너하다. 대충 고중숙 스타일의 교양서를 꿈꾼 듯 하다(고중숙씨가 쓴 책은 이과 학생이면 꼭 읽을만하다)
그런데 이 아재께서는 xx 서는지 안 서는지 모르겠지만 일단 스펙만 보면 최고대학 대빵이신데....
다른 건 모르겠고 미분과 극한에서 뭔가 썰을 장황하게 푸신다. 뭐 이건 재밌게 나도 공부를 했는데 $-$
중간부터 갑자기 xxxxx은 틀렸다라고 하면서 신림동에서 가끔 플랫카드 걸리는 유사과학 세미나 비슷한 이야기를 하시는 게 문제(...)
이거 아재 치고는 참 귀여우시다(...)라는 생각도 들면서
이공계가 생각보다 사이비 종교에 낚이거나 괴상한 자기 확신에 빠지는 경우가 많다는 걸 떠올리고 있다.
\vspace{5mm}

그래도 중간에 미적분에 관한 그럴싸한 썰은 읽을만해서 보는데. 아무래도 조심을 해야겠다.
유사종교 경전도 자주 읽다보면 정들어서 빠지거든(...)
그래도 이 아재는 J보단 낫다. J는 척 보아도 사기꾼인데 이 아재는 사회스펙상 돈욕심낼 처지는 아니고 정말 순수한 의지가 있는 것 같다.
\vspace{5mm}

지금 J와 K를 동시에 버릴까, 아니면 J만 불살라버릴까 고민 중.
J를 그냥 유포시키는 건 대한민국 학생들 IQ를 떨어뜨려 국력의 저하를 꾀하는 중대한 범죄로 보인다 말야.
K는 그래도 논리라도 분명하지, 종교적인 메시지가 강해서 문제지만(...).
\vspace{5mm}

+
그래도 다시 한번.... 이라고 훑었지만   J는 챰 구제불능이다. 얘는 대놓고 돈만 벌려고 책썼네.
그것도 강사 여러명이 협조해서 써도 힘든 판인데 뭔 깡으로 대충 쓰면서 주체사상을 피력한 건지 이해가 안 간다.
이 사람도 참 멘탈이. 어떤 면에서는 대단한 것 같다.
스펙 보니 이해가 가긴 하다. 확실히 I와 대조적이다.
\vspace{5mm}

++
내 이야기는 아니고(...)
과고 출신들은 참 선량하다는 게 문제.
이게 웃긴 게 과고 출신들은 수학, 과학 같은 걸로 사람들 계몽해야한다 그런 실속없는 선민의식(?)이 있어서 손해보고 있고
비과고출신들은 장사질해먹으려 하면서 실속 챙기고 있다.
\vspace{5mm}












\section{교재 뒷담화 : L M N}
\href{https://www.kockoc.com/Apoc/470245}{2015.11.08}

\vspace{5mm}

우물 밖을 나가보지 못 한 개구리는 우물이 우주인 줄 안다.
동전 밖에 만져 보지 못 한 아이는 아빠가 500원 내와, 1000원 줄께라고 하면 500원을 꽉 쥐고 잽싸게 도망간다.
\vspace{5mm}

한 때 동양적인 것에 대해서 비결에 대한 열풍이 불었던 적이 있다.
동양에는 뭔가 숨겨진 것이 있을꺼야... 그렇게 도사들은 밥벌이를 하였던 것으로 안다.
그러나 그 중 어느 하나도 제대로 검증된 건 없었다.
\vspace{5mm}

태권도가 고구려 무술이라고 믿는 바보는 없을 것이며
역시검도가 삼국시대로 거슬러 올라간다고 믿는 바보는 없을 것... 이었으면 좋겟지만 그렇지 않다.
속는 사람은 끝까지 속게 되어있다. 알지 못 하기 때문에 그럴싸한 말에 속아넘어가는 것이다.
\vspace{5mm}

비급이 사라진 자리 : 대용량 알고리즘을 해결할 수 있는 메모리.
\vspace{5mm}

\textbf{L : 배운 사람 눈에는 시큰둥, 안 배운 사람에게는 비급}
\vspace{5mm}

신비주의 마케팅으로 나온 케이스다.
(입시수학을 공부 안 한 사람)에게는 그럴싸해보이는 책이다. 왜냐면 패턴화, 유형화시켰는데 분량이 적어보여서다.
반면 공부를 제대로 한 사람에게는 턱없이 부족한 책이고 이게 어째서 비급으로 취급되나 의아스럽다.
하지만 우리는 대놓고 말을 할 수가 없다. 한국에서 태권도와 검도가 실제로는 일본 무술을 받아들인 것이라고 하면 맞아죽기 밖에 더 있겠나.
어찌보면 싸구려 마케팅과 과장광고에 속아넘어가는 수험생들이 죄다, 이건 누굴 탓할 이유가 없는 것이다.
비급이라고 하는 것의 내용을 대조해보았는데 풍산자와 비슷? 쎈과 일품 조합에는 그냥 쨉도 되지도 않았다
\vspace{5mm}

일부러 어렵게 꼬아낸 문제를 낸다 $\rightarrow$ 그걸로 애들을 공포감에 빠뜨린다 $\rightarrow$ 이 교재를 보면 해결할 수 있어.
그럼 그 교재로 일본 본고사 문제 같은 것을 풀 수 있을까. 전혀 아니올시다.
역대 수능 문제가 그럼 저런 식으로 어렵게 꼬아냈나. 올해까지도 어떻게 낼지 봐야 알겠지만, 꼬아내는 것과 새롭게 내는 건 다르다.
\vspace{5mm}

그 말많던 2014 기하 문제. 이거 제대로 예측하거나 풀이방법 제시한 강의나 교재가 시험 전에 있었던가.
작년 2015년 30번 문제만 하더라도 고1 수학 $-$ 짝홀 발상에다가 2x2 매트릭스 사고법이었는데 이거 작년에 얘기했었던가.
난 무턱대고 까지는 않는다. 정말 제대로 적중시키고 좋은 방법론을 제기한다면 칭찬할 것이다.
하지만 2년째 그런 사례, 내가 아는 한 단 한번도 없었다. 사실 있다고 한다면 그 강사나 교재는 광고없이도 대박났을 것이다.
\vspace{5mm}

\textbf{M : 책은 매우 좋았으나}
\vspace{5mm}

뒷담화 대상은 아니다. 만약 이 사람이 지금도 활발히 활동했으면 아마 콕콕 이상은 아니었을까 싶다.
책을 쓰는 법도 알고 있고, 무엇보다 돈보다도 수학을 좋아하는 인물이라고 확신할 수 있는 근거들이 많았다.
과거 책을 지금도 걍 출판하면 되는데도, 공포마케팅으로 가도 되는데도, 이 사람은 그리 하지 않았다는 게 신뢰의 근거다.
\vspace{5mm}

일단 A형 수학에 있어서 가장 직관적인 접근법을 제시해주는 책이라고 하겠다. 다만 꼼수는 꼼수다,
엄밀한 정의에 기초하기보다는, '감각'적인 풀이로만 간다는 한계가 분명히 있다. 이 점만 명심한다면 매우 좋은 보충서가 될 것이다.
다만 이 책을 연구해보다가 다시 쎈, 풍산자를 보고 다른 교양서를 공부하고 나서 느낀 건, 이 책의 스킬이 그리 약빨이 좋은 것만은 아니란 것이다.
학생 저자의 한계점이 지적된 경우다.
\vspace{5mm}

하지만 과거 책이 그렇다는 것이지 이 사람의 실력만큼은 대단히 좋다. 간헐적으로 이 사람이 올리는 기출 풀이 같은 것을 보면
요즘 수험가에서 떠도는 떠들썩한 것은 다 알고 있고, 관능적(?)인 언어로 풀어내는 것부터가 다음 책을 기약할만하다고 본다.
사실 이 사람의 신간을 개인적으로는 매우 고대하고 있다.
\vspace{5mm}

이 사람은 돈버는 걸 포기하는 대신 제대로 수학을 배우기 위하여 학업코스로 갔다. 정말로 현명한 선택이다.
그저 인기에 휘말려서 대충 책 내면서 결국 돈벌이의 마수에 빠져서 젊은 시절을 날리는 사람과 비교해보면 그렇다.
나중에 콕콕에서 이 사람의 책을 리뷰할 수 있길 바란다.
\vspace{5mm}

\textbf{N : 명문이공계를 졸업했다고 해도 그 당시 수학과 지금의 수학은 다른데}
\vspace{5mm}

이 책은 단행본이다. 제목은 참 거창하다. 중딩 버전까지 나왔다.
만화 xxx트를 원용한 건 뭔가 손발이 오글거리는데 수학과 관련된 어설픈 썸, 연애 스토리 도입은 뭐란 말인가.
수포자가 어떻게 해서 수학을 공부하게 되는가 하는 식의 접근은 나쁘지는 않은데, 이미 이 분야 선구자는 일본의 \textbf{'수학걸'}이 있다.
게다가 책값이나 두께에 비하면 담긴 내용은 보잘 것 없다.
\vspace{5mm}

도대체 이 저자들은 뭔 생각으로 그런 책을 썼을까...
전형적으로 자기들이 이과의 명문대를 갔으니 자기들이 수학을 잘 한다라는 포부 때문일 거다.
하지만 본인이 수학을 잘 하는 것과, 정말 입시에 도움이 되는 수학 썰을 푸는 건 정말 다른 문제다.
실제로 학원가나 과외 쪽에서 가장 잘 못 가르치는 사람들이 수학과라는 이야기가 괜히 나오는 게 아니다.
가르친다는 건 소통이지, 일방적으로 푸는 게 아니다.
그런데 이 책이 그런 경우다. 저자들은 자기들이 똑똑하니까 자기들이 말하는 게 전부인 줄 알고 있는 것이다.
\vspace{5mm}

얄궃지만 같은 제목의 다른 책은 매우 좋다. 다만 그 다른 책의 저자는 스펙만 보면 정말 지잡대란 소리를 들을 정도다
그리고 그 사람의 정치성향도 너무 노골적.
하지만 수학적 사고에 있어서는 그 다른 책의 저자에서 훨씬 더 많은 걸 배울 수 있었다.
\vspace{5mm}

우물 안 두꺼비가 되지 않기 위해서는 끈임없이 배울 수 밖에 없다.
왜 사람들이 학교를 졸업하면 머리가 돌이 될까.... 라는 것에 대해서 생각해보고 상상해보고 직접 겪으며 탐구.
교훈은 그거더라
학교에서는 매일이 사실 새롭다. 교육도 교육이지만 학년도 올라가고 시험도 치르고 진도를 강제로 나간다.
변화를 강제당하니 뇌가 자극을 받고 그러므로 머리를 쓸 수 밖에 없다.
그러나 학교를 졸업하면 대체로 반복적인 일상과 업무에 갇힌다.
새로운 자극이 없으니 그런 변화없는 삶에 순응해버리면서 머리는 점차 둔해진다.
\vspace{5mm}

그나저나 이 단계까지 가니까 결국 모른다... 라고 다들 그러시는데 그래야 당연하지만
생각보다 교재풀이라는 게 좁다.
아마 학생들은 누군가 마케팅한 교재가 정말 다인 줄 알고 그렇게 믿고 공부하겠지.
\vspace{5mm}

진정한 비급이라면 다음 주 목요일에 나올 문제가 들어있어야하지 않나.
\vspace{5mm}





\section{교재 뒷담화 : O P Q}
\href{https://www.kockoc.com/Apoc/472974}{2015.11.10}

\vspace{5mm}

우리나라에서 수학 대중서들도 역시 마케팅의 힘으로 팔린다.
\vspace{5mm}

김용운 교수님의 "재미있는 수학여행"이 표준일 것 $-$ 이 시리즈는 좋다 : 단 김용운 교수님이 과거에 썼던게 더 좋다.
왜 이런 건 복간을 안 하는 걸까.
\vspace{5mm}

그런데 그 이후로 나오는 교양서들은 저 재미있는 수학여행을 못 벗어나고 있고
사실 하나마나한 이야기를 하는 경우가 많다.
\vspace{5mm}

그런데도 왜 잘 팔리는 걸까.
\vspace{5mm}

학부모들의 무지이다.
\vspace{5mm}

자기 자식들이 수학을 잘 하길 바라는 어머니들은 수험을 잘 모른다
그래서 수학에 도움되는 책이라고 알려지면 무조건 구입을 하는 것이다.
이 시장이 생각 외로 크기 때문에, "수학을 잘 할 수 있다"라는 표제만 달고 그럴싸하게 내용만 채우면 팔린다.
\vspace{5mm}

\textbf{O : 범죄뉴스 인용을 해도 관계없습니다.}
\vspace{5mm}

최근에 관악 성추행으로 검색해보시면 된다.
빙산의 일각.
걍 썰 나온 김에 조심스레 적어볼까?
이제야 정의가 구현되는구나... 가 아니라 '\textbf{에게}해'를 조망하는 기분이다.
대다수 의식있는 학생들이 왜 대학원 진학을 기피하는가, 그리고 전공보다도 교수님들의 인성과 성격을 따지나 그 이유란 게 있다.
꼭 말은 안 하겠지만 장년교수 $-$ 20대 여대생의  \textbf{I.얼레리꼴레리.YOU}는 쉬쉬해서 그렇지 원래부터 있었다.
\vspace{5mm}

더 듣고 싶겠지만 여기서 끊고(댓글로도 묻지 마슈. 그리고 난 기사만 찾아보라고 했수다)
일단 책만 보자면 내용이 역시 에게해 2탄이다. 그리고 저 사건이 터진 후 헌책방에 10권 이상 늘어선 그랜드 캐년까지 목격했다.
출판보다는 본인의 이미지 메이킹을 위한 책일수록 책 표지에 저자 사진을 크게 싣는다(그것도 미소짓은 걸로)
뭐 그건 모르겠고 내용만 좋으면 되잖아 하는데 입시수학에 전혀 도움이 되지도 않고,
그렇다고 다른 것이 쓸모가 있느냐하면 그건 아니다.
\vspace{5mm}

썰 나온 김에 적으면 대학교부터 아름다워질 것이다... 그런 건 없다.
고교까지 사교육은 그냥 돈만 챙기지 그래도 웬만한 경우는 지킬 건 다 지킨다. 퇴출이 그만큼 빠른 시장이어서이다.
것보다도 한낱 과외교사조차도 공부해야하기 때문에 딴 걸 신경쓸 여유는 별로 없다(막장사례가 없지 않겠지만)
그러나 대학 진학 이후는 캠퍼스의 가면을 쓴 사회 현실이다. 이게 좋은 게 아니다.
대학생은 보호받지 못 한다. '개인 책임'으로 산다. 하지만 대학이기 때문에 은연 중의 갑을 관계라는 건 존재한다.
\vspace{5mm}

상상도 못 하는 범죄가 일어나더라도 쉬쉬하는 경우는 많다. 그건 명문대일수록 심하다.
그 사건이 노출되면 피해자도 많은 걸 잃는다. 성범죄를 저지르는 남자들은 그것을 믿고 덫을 판다.
\vspace{5mm}

이전에 적었지만 '가르치는 입장'에 있는 사람이 교주놀이 하다가 더 막 가는 거, 이거 정말로 심하다.
사회에서 까이는 게 크리스천이던가. 한데 개인적 경험으로는 크리스천들이 차라리 이 점에서는 나았다.
물론 아닌 경우도 있긴 하지만, 신앙심으로 살기 때문에 굽힐 줄 알고, 굽힐 줄 아니까 그래도 최소한의 도덕이란 게 있는 거다.
반면 신 좆까 하면서 자연과학과 합리주의로 모든 게 설명된다.... 이런 사람들이 자기도 모르는 사이에 교주가 되고 막 간다.
\vspace{5mm}

P : \textbf{다작}
\vspace{5mm}

뒷담화 대상은 아닐지도 수학교양서 어디든지 올라와 계시더라.
그런데 내용은 그냥. 한번 읽어볼 수준인데 신기한 게 정말 어디든 이름이 올라와 있다.
좋게 말하면 열정적인데.... 중요한 건 이 분의 대표작이라는 게 있냐 보면 사실 기대한 수준만큼은 아니라고 보았는데.
\vspace{5mm}

지금 생각해보면 이렇게까지 다작하는 사람도 없고 정열적이라는 점에서는 높이 쳐줘야하지 않을까.
그게 실제 현재 입시수학에는 별로 도움이 되지 않는다 하지만, 수학의 대중화에서는 확실히 기여하는 게 있기 때문이다.
다만 아쉬운 건 그 본인만의 개성이라는 게 부족하다가 되겠음.
\vspace{5mm}

\textbf{Q : 사실 쓸모가 없는 직관수학}
\vspace{5mm}

한 때 휩쓸었던 책이다. 그리고 추천사나 스펙 등을 보면 으리으리하다.
그 직업인 분이 수학에 관한 책을 쓴 경우는 거의 없을 걸? 그런데 이 분, 원래 가르치는 일을 했었다.
P와 달리 이 분은 개성이 있다. 철저히 직관수학을 강조하고 있어서이다.
최근에 냈었던 그 책보다 훨씬 10년도 넘은 옛날에 '직관수학'에 관한 책을 낸 적이 있고 당시는 혁명적인지라 인상깊게 읽었다.
그래서 직관수학이라는 걸 나도 추종한 적이 있는데.
\vspace{5mm}

직관수학의 단점 : 그래서 뭐 어쩌라는 거야. 머리만 좋으면 된다고? 멍하니 보면 된다고?
\vspace{5mm}

만약 직관수학으로 가능하다면 뭐하러 사람들이 식을 만들고 그래프를 그리고 그랬겠냐.
그런 걸로 가능하다면 수학이란 학문이 필요없었을지도 모른다.
수학적인 사고가 되지 않기 때문에 이걸 보완하려고 만든 학문이 수학이다, 속칭 말해서 '머리 나쁜 사람들을' 위한 것이 수학이 아닌가.
\vspace{5mm}

밥 로스의 참 쉽죠$\sim$ 급.
물론 나쁜 책은 절대 아니다. 하지만 이걸 고교생들이 보았다하면 써먹기는 어려울 것이다.
이 책은 저자 분이 더 상세히 분설하고 해설하면서 한 여러권의 시리즈를 만들어서 난이도를 낮추고 정식교재화하면 대박났을 것이다.
한 때 잘 팔렸으니까 안 본 사람은 없겠지만, 끝까지 읽은 애들이 몇이나 있을 것이며, 읽었다고 해도 그걸로 도움이 되었을까.
\vspace{5mm}

사실 위 중 어느 것도 고교생에게는 도움이 되진 않는다고 본다.
굳이 수학적 사고에 대해서 가장 쉽게 접하고 싶다면 아래 책들을 읽길 바란다.
\vspace{5mm}

현직 교사가 고교생의 눈높이를 잘 헤아린 책이다.
입시에는 당장 상관관계가 없어보일 수도 있다. 그러나 고교수학의 접근법에 대한 '수학 철학'으로서 이만한 입문서는 없고
사실 그 이상도 필요하지 않다(그 이상 필요하다면 문제접근방법이겠지만 이건 차후에 논할 듯)
어째서 $\sim$ 한 풀이가 나오느냐, 그리고 $\sim$한 접근이 어떤 의미를 지니느냐 하기 위한 철학적인 교양을 쉽게 설명해놓았다고 할 수 있다.
\vspace{5mm}

무엇보다 중요한 건 쉽다. 그리고 요점은 웬만큼 다 담아놓았다.
\vspace{5mm}

그렇다고 저기서 스킬이나 무슨 수험 꼼수 기대하진 말길.
하지만 제대로 읽고나면 '수학적으로 사고한다'의 올바른 길을 걷는다는 건 보장한다.
\vspace{5mm}








\section{교재 뒷담화 : R}
\href{https://www.kockoc.com/Apoc/500147}{2015.11.19}

\vspace{5mm}

진짜를 보지 못 한 사람들은 가짜에 환호하고
그 가짜에 환호하는 우매한 사람들을 보면서 비웃는 게 참즐거움이 아닐까 싶다.
\vspace{5mm}

일단 R의 저자는 $-$ 현재 e$-$book으로 밖에 못 구하지만 $-$
그리스 기하학과 현대 수학에 관한 꽤 탁월한 수필(?)을 냈다.
고교수학의 차원에서 뭔가 서양수학의 정신(?)이라는 걸 알고싶다면 꽤 읽어볼만한 책이다.
무엇보다 국내저자가 저런 책을 쓸 수 있었다는 점에서 놀랍다.
\vspace{5mm}

그런데 저건 수험서 아니잖아요. 그런데 왜 교양타령해요? 아니 뭔 책인지는 아나
그런데 R은 매우 탁월한 책이다. 이 책은 일본책을 누를 수 있는 마스터피스임.
그런데 저기 목록에는 안 올라갔더라는 것. 정말 저자나 일부 구매자 아니면 알 수 없는 꿈 속의 책이 되어버렸다.
물론 나야 맛있게 잘 먹겠습니다라고 얌냠거리고 있는데 누구든 소개해줄 생각은 없다.
궁금한 사람은 저 저자 분과 개인적 연락을 해서 구해보시든지(그렇다고 개인적 연락을 해서 구했다는 건 아니니까)
아니 e$-$book은 있을지 모르지만 뭐 알아서 찾아보시도록
\vspace{5mm}

일본 책과 다른 것 $-$ 일본 책은 마치 독일인들처럼 정교한 방법론을 제시하고 사실적이다. 그런데 도통한 맛은 없다.
반면 R은 세세한 지엽적인 건 빠져있는데, 문풀 방법을 고민해본 사람이라면 결국 도달하게 되는 도통한 경지를 적어놓았다.
앞의 두 책을 쓴 것도 본인이 열심히 공부해서여서일건데, 이 정도면 현직교사는 가볍게 넘어서는 내공이다.
게다가 우리나라 수학에서 어떻게 잘못된 방법론을 가르치는지도 세세히 적어놓은 건 현직 교사도 못 하는 거지.
\vspace{5mm}

아마 저자가 출판사 컨택을 잘 하면서 이 책에다가 현재 기출 킬러문제만 잘 융합시켰어도
그냥 컨셉만 잘 잡은 병신같은 책들을 누르고 잘 팔리지 않았을까도 싶은데
개인적으로는 이런 건 공유되어보았자 나도 재밌는 게 없기 때문에 그냥 내 머리칼 갯수만큼이나 비밀로 처리.
뭐 찾을 사람은 찾겠지, 정말 적당히만 힌트 던져도 다 찾더라.
\vspace{5mm}




